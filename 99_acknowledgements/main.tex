\chapter*{Acknowledgements} \phantomsection
\markboth{\leftheadstyle{Acknowledgements}}{}
\addcontentsline{toc}{chapter}{Acknowledgements}

First and foremost I would like to thank
my supervisor Prof.~Andreas Dreuw,
my co-supervisor Prof.~Guido Kanschat
as well as my advisor Dr.~James Avery.
They provided me with a constant source of support
and helpful comments throughout.
Andreas, thanks for giving me the chance to work on such
a highly interdisciplinary project
and for allowing me the scientific freedom to fully explore it in all aspects.
Guido, thanks for the enlightening discussions about numerics
as well as the occasional criticism to make me invest a second thought.
James, thanks for introducing me to Coulomb-Sturmians,
for the great time I had whenever I visited you in Copenhagen
and of course for showing me Heisenberg's bath tub.

Along the same line I wish to commemorate Dr.~Michael~Wormit,
with whom I had the pleasure to work on the topic of finite elements,
albeit for too short.
It brings me a consoling thought that some of his
ideas managed to yield fruit in this thesis.

% -----

An aspect of quantum chemistry,
which has always fascinated me,
is the interdisciplinarity of the field.
It is absolutely astonishing
to find that the current status could only be achieved
by a combination developments guided by chemical as well as physical intuition,
coupled with a careful study of the beautiful and complex
underlying mathematical structures
as well as the many dirty tricks of high-performance computing.
In this respect I am very grateful
to all the impulses and ideas from
every scientific field and research topic I was exposed to.
I wish to express my thanks for all the discussions with
(in random order)
Dr.~Tobias~Setzer,
Henrik~Larsson,
Dr.~Klaus~Birkelund,
Dr.~Mads~Kristensen,
Maximilian~Scheurer,
Prof.~Reinhold~Schneider,
Manuel~Hodecker,
Adrian Dempwolff,
Dr.~Denis~Davydov,
Jie~Han,
Lucas~Fabian~Hackl,
Jan~Janßen,
Dr.~Tim~Stauch,
Jan~Christoph~Peters,
Dr.~Katharina~Kormann,
Prof.~Eric~Cancès,
Fabian~Klein,
Dr.~Jenny~Wagner%
and everyone I forgot to mention in this list.

% -----

In the same respect I would like to acknowledge the curious and thoughtful
people I met at the Chaos Communication Congress
and various other hacker events I attended.
The many night of talking
about science, society, ethics and programming often brought a
different perspective to my own work.
A special thanks goes to 
(again in random order)
supaake,
cherti,
hauro,
kungi,
rami,
janx2%
and all the other guys from the NoName e.V.
Similarly I acknowledge my fellow jazzers
Susanne, Rudolph, Jürgen, Uli and Walter.
Thanks for taking my mind off science for a while.

% -----

I gratefully acknowledge
the University of Kopenhagen,
KTH Royal Institute of Technology,
Mathematical Research Institute of Oberwolfach
and above all Heidelberg University for hosting me.
Further I thank the Heidelberg Graduate School
for the Mathematical and Computational Methods for the Sciences
for providing me with financial support
and for the opportunity to take action myself
in the form of organising conferences or teaching to fellow students.

I thank the past and present members of the Dreuw group,
for the relaxed atmosphere
with constant running gags about
stuffed animals and awful singers.
Thanks for being more than just colleagues,
for all the fun
at our retreats in Kleinwalsertal, our trips to Wurstmarkt and
the beers and wines after work.
In the same way I wish to thank
the group of Prof. Dr. Brian Vinter at the Niels Bohr Institute
for the open and welcoming atmosphere.
There is no question I immediately felt part
and truely enjoyed the time I spent with you guys.
I am already looking forward to coming back for another visit!

I gratefully acknowledge the computational time I was provided
at the computing facilities of the Vinter group,
as well as all the excellent support from Jonas Bardino
with respect to getting my code to run.
I thank Manfred Trunk
for having an open door for all requests related
to our cluster in Heidelberg
and for providing almost instant answers for all emails I sent.
Thanks to Ellen Vogel for taking care of the bureaucracy.

% -----

For proof-reading earlier drafts of the thesis I thank
(in random order)
Andreas~Fink,
Henrik~Larsson,
Dr.~James~Avery,
Dr.~Jenny~Wagner,
Reena~Sen,
Marvin~Hoffmann,
Manuel~Hodecker,
Adrian~Dempwolff,
Carine~Dengler,
Maximilian~Scheurer and
Fabian~Faulstich,%
who provided helpful comments and detailed feedback.
Furthermore I wish to express my thanks to the many people developing
the software I used during the preparation of the presented work,
namely
\texttt{bash},
\texttt{clang},
\texttt{git},
\texttt{gcc},
\texttt{i3},
\LaTeX,
\numpy,
\python,
\TeX,
\texttt{vim}
and all the other tiny utilities I use without thinking about them.
Especially I want to acknowledge all the people
involved with compiling the best 
GNU/LinuX distribution in the world, namely Debian.

% -----

I would like to express my thanks to Henrik Larsson
for a lasting friendship,
which has lead to many weekends of fruitful
scientific debate,
followed often by new insights into my own subject.
I like that we do not always agree
and that the only way to convince you is scientific evidence.
Thanks for teaching another Deutsche Schülerakedemie course with me this summer.
In the same way I say thanks to my old friends
from the days in Grünstadt,
Jan~Janßen,
Jan~Christoph~Peters,
Alexandra~Schulte,
and
Peter Schwalb,
which have accompanied me for many years by now
and have always found the time for a spontaneous visit
in one direction or the other.
Thanks to all the new friends I made in Cambridge,
in Heidelberg and the rest of Germany.
I know we have met far too rarely in the recent months,
but rest assured I am really
grateful for every minute we have spent so far
and I look forward for the time to come.

% ----

E große Dank sei ach
mei ganz Familie vun de Palz.
Ganz bsonders will ich danke Sven, Oliver,
Michael, Markus un Jun,
als ach mei Unkel Klaus un Hans
un Tante Gerda, Edith un Änne.
Mit eich ebbes zu unnernemme oder efach blos zu babble
des war oft e Quell vun Kraft fer die mehr anstrengende Dage.
Vielmols merci sag ich Sonja und Jim
fer die offene Ohre und die herzliche Güte.
Danewe dank ich meine Eltre Ingrid un Ortwin,
bei dene efache Worte net genuch sei kenne
um mei Dank im entferndeschde zu fasse.

% ----

Finally and above all I thank my beloved fianc\'ee Carine
for every minute we have been, are and will be together.
Thanks for the continuous support and being exactly the way you are.
