\chapter{Elements of quantum mechanics}
\chapquote{
	Whenever we proceed from the known into the unknown we
	may hope to understand,
	but we may have to learn at the same time a new
	meaning of the word ``understanding''.
}{Werner Heisenberg~(1901--1976)}

\section{Laplace equation and spherical harmonics}
\nomenclature{$Y_l^m(\theta, \varphi)$}{Spherical harmonic function with angular
momentum quantum number $l$ and azimuthal quantum number $m$}

\section{Poisson equation}
\todo[inline,caption={}]{
	\begin{itemize}
		\item Briefly introduce Laplace equation
		\item Briefly introduce spherical harmonics (i.e. laplace eigenfunctions)
		\item Introduce poission equation in the context of electrodynamics
		\item Show how this leads to a linear problem in the end
	\end{itemize}
}

\section{Hydrogenic atoms}
\label{sec:HAtom}
\nomenclature{$L^{b}_a$}{Associated Legendre polynomial with orders $ä$ and $b$.}

\todoil{Show that generally it makes sense that larger values of $n$ give rise to larger
	energy due to the kinetic energy operator. Maybe just generally take $\mbra{\Psi} \Delta \mket{\Psi}$ or argue it.}

In an analogous manor to section \vref{sec:HAtom}
one can argue for molecules that larger principle quantum numbers
$n$ give rise to an associated Lagrange polynomial of higher degree,
thus more sign changes in the basis function.
This in turn implies a larger curvature and thus a higher kinetic energy.
The potential energy of a basis function, however,
is not affected very much,
since the radial extent is pretty much unchanged by $n$
(exponential decay does not depend on $n$).


\subsection{The Aufbau principle for the periodic table}
\label{sec:PSEAufbau}
Mention the Madelung rule


\section{$N$-body Schrödinger equation}
% Argue that we can do the mathematical background in real function
% arithmetic only and still use the results for complex input due to linearity.

$ \Psi \in H^2(\R^{3d}, \R)$
\nomenclature{$\Psi$}{State of an $N$-body quantum system, typically the \emph{exact} solution to the $N$-electron electronic TISE}
\defineabbr{TISE}{TISE\xspace}{Time-independent Schrödinger equation,
usually only the electronic part}
\nomenclature{$\Phi$}{Slater determinant or $N$-particle basis functions}
\nomenclature{$\varphi_i$}{One-particle function, typically solutions to the Hartree-Fock equation, i.e. a Hartree-Fock orbital}
\nomenclature{$\Op{H}$}{$N$-electron Schrödinger Hamiltonian}
\nomenclature{$\Delta$}{Laplace operator}
\nomenclature{$\Op{V}$}{Combined potential energy operator (nuclear attraction and Coulombic electron-electron repulsion)}

\nomenclature{$\Ibas$}{Index set of the one-particle basis functions.
	Typically a set of multi-indices of quantum numbers.}
\nomenclature{$\Nbas$}{Cardinality of $\Ibas$, i.e. the number of one-particle basis functions.}
\nomenclature{$\Iorb$}{Index set of computed SCF orbitals, typically $\{0, \ldots \Norb\}$}
\nomenclature{$\Norb$}{The number of computed SCF orbitals.
Note, that $\Norb \le \Nbas$.}
\nomenclature{$\Nelec, \NelecA, \NelecB$}{Number of electrons, number of $\alpha$ electrons, number of $\beta$ electrons. Note $\NelecA + \NelecB = \Nelec$ and $\NelecA \le \NelecB$ by convention.}
\nomenclature{$\chi_\mu$}{$\mu$-th one-particle basis function of the one-particle basis $\{\chi_\mu\}_{\mu \in \Ibas}$}

\nomenclature{$\vec{x}$}{Vector in $\R^{3 \Nelec}$, which typically specifies the positions of all electrons of the system.}
\nomenclature{$\vec{r}$}{Vector in $\R^3$, which specifies the position of a particle in space}



\todo[inline,caption={}]{
	\begin{itemize}
		\item Define it
		\item TDSE and TISE
		\item Properties
		\item Show how difficult
	\end{itemize}
}

\section{BO separation}
\todo[inline,caption={}]{
	\begin{itemize}
		\item Eckard conditions
		\item Rotational, vibrational, translational, electronic separate
		\item PES and electronic schrödinger
	\end{itemize}

	% follow conical intersection talk of ivona
}

%\section{Second quantisation and Fock spaces}
%\todo[inline,caption={}]{
%	\begin{itemize}
%		\item Structure of the spaces
%		\item Properties
%	\end{itemize}
%}
