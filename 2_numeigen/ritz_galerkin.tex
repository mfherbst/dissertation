\section{Projection methods for eigenproblems}
\label{sec:Projection}

Let $\Op{A}$ be a self-adjoint, bounded below operator on a separable Hilbert space
$\hilbert$ with domain $D(\Op{A})$.
We already saw in remark \vref{rem:OpRietz}
that $\Op{A}$ uniquely defines a sesquilinear form
\[ a(u, v) = \braket{u}{\Op{A} v}_\hilbert \]
for $(u, v) \in D(\Op{A}) \times D(\Op{A})$.

\subsection{Form domains of operators}
Even though $\Op{A}$ might only be self-adjoint on the domain $D(\Op{A})$,
the form $a(\slot, \slot)$ can often be defined sensibly
on a larger domain $Q(\Op{A})$, called the \newterm{form domain} of $\Op{A}$.
Its construction will be sketched in this section.
For more details see \cite[p. 77]{Teschl2014} or \cite[p. 276]{Reed1980}.

\noindent
Since $\Op{A}$ is semi-bounded from below, one can define a scalar product
\[
	\braket{u}{v}_{\Op{A}} \equiv \braket{u}{\Op{A} v + (C + 1) v}_{\hilbert}
	= a(u, v) + (C+1) \braket{u}{v}_{\hilbert},
\]
for all $u,v \in D(\Op{A})$.
Here $C$ is the constant of semi-boundedness of definition \vref{defn:SemiBounded}.
Clearly the associated norm $\norm{\slot}_{\Op{A}}$ satisfies
\begin{equation}
	\label{eqn:FormDomainNorm}
\norm{u}_{\Op{A}}
= \braket{u}{\Op{A} u} + (C+1) \norm{u}_{\hilbert}
\stackrel{def. \ref{defn:SemiBounded}}{\geq} \norm{u}_{\hilbert}.
\end{equation}
We now take the completion of $D(\Op{A})$ under the norm $\norm{u}_{\Op{A}}$
and call it $Q(\Op{A})$.
\eqref{eqn:FormDomainNorm} assures that all sequences,
which are Cauchy in $D(\Op{A})$ wrt. $\norm{\slot}_{\hilbert}$
are Cauchy in $Q(\Op{A})$ wrt. $\norm{\slot}_{\Op{A}}$ as well.
One can show further~\cite{Teschl2014} that such sequences
have the same limit in $Q(\Op{A})$ irrespective of the norm used.

\noindent
This allows to uniquely
extend $a(\slot, \slot)$ to $Q(\Op{A}) \times Q(\Op{A})$ by setting
\[ a(u, v) := \braket{u}{v}_{\Op{A}} - (C+1) \braket{u}{v}_{\hilbert}. \]
Constructed as such $Q(\Op{A})$ is the largest Hilbert space on which the form
$a(\slot, \slot)$ is defined and continuous.
The form domain satisfies
\[ D(\Op{A}) \subseteq Q(\Op{A}) \subseteq \hilbert, \]
where the subspaces are dense in the respective larger space.

\begin{exmp}
	For all cases we discussed in the previous chapter,
	that is the Laplace operator $\Delta$ and the hydrogenic
	Hamiltonian $-\frac12 \Delta - \frac{Z}{r}$,
	the form domain is $H^1(\R^3)$.
	This can be easily verified by constructing the expression
	for the form $a(\slot, \slot)$ and applying partial integration.
\end{exmp}

\subsection{The Ritz-Galerkin projection}
\label{sec:RitzGalerkin}
The defining property of any eigenpair
$(\lambda_i, v_i) \in \sigma_P(\Op{A}) \times D(\Op{A})$
of the operator $\Op{A}$ is of course the condition
\begin{equation}
	\Op{A} v_i = \lambda_i v_i.
	\label{eqn:EigenproblemStrong}
\end{equation}
By a simple projection onto an arbitrary test function $u$,
one can show that any such eigenpair satisfies
\begin{equation}
	\forall u \in \hilbert: \quad a(u, v_i) = \lambda_i \braket{u}{v_i}_\hilbert.
	\label{eqn:EigenproblemWeak}
\end{equation}
as well, the so-called \newterm{weak formulation} of the eigenproblem.
In contrast to this, \eqref{eqn:EigenproblemStrong} is sometimes
referred to as the \newterm{strong formulation}.
A consequence of the Lax-Milgram theorem~\cite[p. 23]{Helffer2013}
% TODO OPTIONAL \vref{thm:LaxMilgram}
and the semi-boundedness of $\Op{A}$
is that a solution in the weak sense implies a solution in the strong sense
as well, making both formulations equivalent.

This suggests the \newterm{Ritz-Galerkin projection},
where one attempts to find an approximate solution
for \eqref{eqn:EigenproblemStrong}
by considering \eqref{eqn:EigenproblemWeak}
in a sequence of subspaces of $Q(\Op{A})$.

\begin{defn}[Ritz-Galerkin projection]
	\label{defn:RitzGalerkin}
	Let $\Op{A}$ be a self-adjoint, bounded below operator
	with form domain $Q(\Op{A})$ and associated sesquilinear form $a(\slot,\slot)$.
	Given a sequence $(S_n)_{n\in\N} \subset Q(\Op{A})$
	of finite-dimensional subspaces satisfying
	\begin{equation}
		\forall v \in Q(\Op{A}) \quad \inf_{v^{(n)} \in S_n} \norm{v - v^{(n)}}_{Q(\Op{A})} \xrightarrow{n\to \infty} 0,
		\label{eqn:CondSubspaces}
	\end{equation}
	obtain a sequence of approximate eigenspectra $\sigma^{(n)}(\Op{A})$
	by solving --- for each $n$ --- the variational problem
	\begin{equation}
		\label{eqn:varProblem}
		\left\{
		\begin{aligned}
			\text{Search $(\lambda_i^{(n)}, v_i^{(n)}) \in \R \times S_n$ such that} & \\
			\forall u^{(n)} \in S_n: \quad a(u^{(n)}, v_i^{(n)})
			&= \lambda_i^{(n)} \braket{u^{(n)}}{v_i^{(n)}}_\hilbert \\
			\norm{v_i^{(n)}}_\hilbert &= 1
		\end{aligned}
		\right\}.
	\end{equation}
\end{defn}
For ease of our discussion let $\Op{A}^{(n)}$ denote the self-adjoint
operator on a particular $S_n$, which is defined by the variational problem
\eqref{eqn:varProblem}, \ie which satisfies
\[ \forall (u^{(n)}, v^{(n)}) \in S_n \times S_n \quad \braket{\Op{A}^{(n)} u^{(n)}}{v^{(n)}}_\hilbert = a(u^{(n)}, v^{(n)}). \]
Since $S_n$ is finite-dimensional, $\Op{A}^{(n)}$ is compact%
\footnote{See proposition \vref{prop:FiniteDimCompact}.}
and thus it will have a discrete spectrum $\sigma(\Op{A}^{(n)})$.
By definition $\sigma^{(n)}(\Op{A}) = \sigma(\Op{A}^{(n)})$.

Our hope is now to construct such a sequence $(S_n)$ of subspaces,
that $\sigma(\Op{A}^{(n)})$ converges to $\sigma(\Op{A})$.
Unfortunately this is \emph{not} possible in general,
see \cite{Helffer2013} for details.
What can be achieved, however, is a method to obtain
sensible approximations for the lower end of the spectrum,
especially all discrete eigenvalues below the essential spectrum.

Let us first state the theoretical basis
in the form of the celebrated \newterm{min-max theorem}~\cite[p. 146]{Helffer2013}.
In our discussion here, we follow the usual convention,
where the eigenvalues in the discrete spectrum are indexed%
\footnote{This can always be done,
	since by definition the discrete spectrum is \emph{always} countable.}
in increasing order, \ie
\[ \lambda_1 \leq \lambda_2 \leq \lambda_3 \leq \cdots. \]

\begin{thm}[Courant-Fischer min-max theorem]
	\label{thm:CourantFischer}
	Let $\Op{A}$ be a self-adjoint operator on $\hilbert$,
	which is bounded below with form domain $Q(\Op{A})$
	and associated sesquilinear form $a(\slot,\slot)$.
	For each $0 < n \in \N$, we define
	\begin{equation}
		\lambda_n(\Op{A}) := \inf_{W \in \set{S}_n} \ \sup_{u \in W \backslash \{0\}}
		\frac{a(u,u)}{\norm{u}_\hilbert^2}
		\label{eqn:CourantFischer}
	\end{equation}
	where $\set{S}_n$ is the set of all $n$-dimensional subspaces of $Q(\Op{A})$.
	Then
	\begin{itemize}
		\item if $\Op{A}$ has at least $n$ eigenvalues lower
			than $\inf \sigma_\text{ess}(\Op{A})$ (counting multiplicities
			the appropriate number of times),
			then $\lambda_n(\Op{A})$ is the $n$-th eigenvalue of the discrete spectrum
			of $\Op{A}$,
		\item otherwise, $\lambda_n(\Op{A}) = \inf \sigma_\text{ess}(\Op{A})$.
	\end{itemize}
\end{thm}

\noindent
Combining this with the Ritz-Galerkin projection of definition \ref{defn:RitzGalerkin}
yields:

\begin{cor}
	\label{cor:Convergence}
	Let $\Op{A}$ be a bounded below, self-adjoint operator on $\hilbert$
	and let $(S_n)$ be a sequence of subspaces of the form domain,
	which satisfy condition \eqref{eqn:CondSubspaces}.
	If we denote with $\Op{A}^{(n)}$
	the approximations to $\Op{A}$
	according to the variational Ritz-Galerkin ansatz
	\eqref{eqn:varProblem}, then
	\[ \forall 0 < i \in \N \quad \lambda^{(n)}_i := \lambda_i(\Op{A}^{(n)}) \xrightarrow{n\to\infty} \lambda_i(\Op{A}), \]
	where the convergence is always from above.
\end{cor}

As discussed in section \vref{sec:SpectralTakeAway}
those operators,
which will be considered in this thesis,
always possess a discrete spectrum located below the essential spectrum.
Furthermore we will always be interested
in those bound states located at the lower end of the discrete spectrum
for physical reasons.
With the aforementioned results we can
sketch an approximation method for our setting.

\begin{rem}[Approximation of the bottom of the discrete spectrum]
	\label{rem:ApproxBottomDiscrete}
	Let $\Op{A}$ be a self-adjoint, bounded below operator
	and let us assume that we seek approximations
	for a few discrete eigenvalues,
	which are all located at the bottom of the spectrum $\sigma(\Op{A})$
	and well below the essential spectrum.

	Let $U \subset Q(\Op{A})$ be a dense subspace.
	We can span a sequence of subspaces $(S_{\Nbas}) \subset Q(\Op{A})$
	by selecting larger and larger sets $\{\varphi_\mu\}_{\mu \in \Ibas}$
	of $\Nbas = \abs{\Ibas}$ orthonormal basis functions $\varphi_\mu \in U$
	as the bases.
	Since $U$ is a dense subspace of the separable Hilbert space
	$\hilbert$, it is separable as well.
	Therefore we know that
	in the limit of $\Nbas \to \infty$,
	$\spacespan \{\varphi_\mu\}_{\mu \in \Ibas}$ will tend towards $U$.
	Thus $\spacespan \{\varphi_\mu\}_{\mu \in \Ibas}$ eventually allows
	to construct Cauchy sequences, which approximate each $v \in Q(\Op{A})$
	up to arbitrary accuracy.
	In other words the sequence $(S_{\Nbas})$ with $\Nbas$ increasing
	satisfies condition \eqref{eqn:CondSubspaces}.

	Because of corollary \ref{cor:Convergence}
	we can thus get arbitrarily accurate approximations
	to our eigenvalues of interest
	by solving the variational problem \eqref{eqn:varProblem}
	in subspaces spanned by larger and larger
	basis sets $\{\varphi_\mu\}_{\mu \in \Ibas} \subset U$.
	This results in more and more accurate approximations
	of the corresponding bound eigenstates as well.
\end{rem}

\begin{rem}[Discrete formulation of \eqref{eqn:varProblem}]
	\label{rem:DiscreteFormulation}
	We are again in the setting of remark \ref{rem:ApproxBottomDiscrete}.

	\noindent
	If $S_{\Nbas} = \spacespan \{\varphi_\mu\}_{\mu \in \Ibas}$, we can expand
	\[ v^{(n)}_i = \sum_{\nu\in\Ibas} C^{(n)}_{\nu i} \varphi_\nu \qquad \text{with} \quad C^{(n)}_{\nu i} \equiv \braket{\varphi_\nu}{v^{(n)}_i} \]
	and thus reformulate \eqref{eqn:varProblem} to become
	\begin{equation}
		\label{eqn:DiscretisedEigenproblemHalfWay}
		\left\{
		\begin{aligned}
			\text{Search $\lambda_i^{(n)}$ and $C^{(n)}_{\nu i}$ such that} \hspace{30pt} & \\
			\forall \varphi_\mu \in S_{\Nbas}:
			\quad \sum_{\nu\in\Ibas} C^{(n)}_{\nu i} \, a(\varphi_\mu, \varphi_\nu)
			&= \lambda_i^{(n)} \sum_{\nu\in\Ibas} C^{(n)}_{\nu i} \, \braket{\varphi_\mu}{\varphi_\nu}\\
			1 &= 
			\sum_{\nu\in\Ibas} \sum_{\mu\in\Ibas} \cc{\left(C^{(n)}_{\mu i}\right)} \braket{\varphi_\mu}{\varphi_\nu} C^{(n)}_{\nu i}
		\end{aligned}
		\right\}.
	\end{equation}
	Introducing the matrix $\mat{A}^{(n)} \in \C^{\Nbas \times \Nbas}$
	and the vectors $\vec{c}^{(n)}_i \in \C^{\Nbas}$
	with elements
	\begin{align}
		\label{eqn:DiscretisationExpression}
		A_{\mu\nu} &= a(\varphi_\mu, \varphi_\nu) & \big(c^{(n)}_i\big)_\mu &= \braket{\varphi_\mu}{v^{(n)}_i} = C^{(n)}_{\mu i}
	\end{align}
	we can write \eqref{eqn:DiscretisedEigenproblemHalfWay}
	as the matrix eigenvalue problem
	\begin{equation}
		\left\{
			\begin{aligned}
				\text{Search $(\lambda_i^{(n)}, \vec{c}^{(n)}_i) \in \R \times \C^{\Nbas}$}& \text{ such that}\\
				\mat{A} \vec{c}^{(n)}_i &= \lambda_i^{(n)} \vec{c}^{(n)}_i \\
				\norm{\vec{c}^{(n)}_i}_{\C^{\Nbas}} &= 1
		\label{eqn:DiscretisedEigenproblem}
		\end{aligned}
		\right\}.
	\end{equation}
	In this formulation the eigenpairs $(\lambda_i^{(n)}, \vec{c}^{(n)}_i)$
	can be determined by standard diagonalisation schemes
	like the ones we will discuss in section \vref{sec:DiagAlgos} below.
\end{rem}

\begin{rem}[Requirements regarding the basis function type]
	\label{rem:ChoiceBasisFunction}
In light of the numerical approach sketched in
remarks \ref{rem:ApproxBottomDiscrete} and \ref{rem:DiscreteFormulation}
let us summarise the requirements towards the basis functions $\varphi_\mu$
for solving the discretised problem \eqref{eqn:DiscretisedEigenproblem}.
\begin{itemize}
	\item The basis function type should admit
		to construct a dense subspace of $Q(\Op{A})$
		if an infinitely large basis set is chosen,
		since this is needed in order to satisfy \eqref{eqn:CondSubspaces}.
		Some basis function types even admit to span $Q(\Op{A})$ itself.
		We shall call these \newterm{complete}.
	\item It should be numerically feasible to solve \eqref{eqn:DiscretisedEigenproblem}.
		In other words \emph{both} computing $\mat{A}$
		and determining its eigenpairs
		should be viable.
	\item The convergence in corollary \ref{cor:Convergence}
		should be fast and systematic.
		In other words the basis type should
		allow to construct a suitable basis set
		in case certain requirements
		regarding accuracy, computational demands,
		description of properties, \ldots~
		should be met.
		Any prior knowledge about the physical problem
		or the properties of $\Op{A}$ can ideally be incorporated
		in such a basis set choice.
		% With this goes a priori and a posteriori error estimation hand in hand.
\end{itemize}
See chapter \vref{ch:NumSolveHF} for some basis function types,
which are used in quantum chemistry,
in the light of solving the Hartree-Fock problem.
\end{rem}

Before we discuss some basic diagonalisation
algorithms in the next section,
let us conclude our discussion about the discretisation
of eigenvalue problems
with a word of warning about the essential spectrum.

\begin{rem}
Remark \vref{rem:NumericalEssentialSpectrum}
stated that it was difficult
to obtain numerical approximations to the essential spectrum.
The min-max theorem \vref{thm:CourantFischer}
provides some theoretical justification for this.
In corollary \vref{cor:Convergence} we saw,
that all eigenvalues from a Ritz-Galerkin approximation
of $\Op{A}$ tend to $\lambda_i(\Op{A})$
as the subspace size is increased.
Unfortunately this value is equal to $\inf \sigma_\text{ess}(\Op{A})$,
the infimum of the essential spectrum,
as soon as we exhausted the discrete spectrum.
In other words the methods we developed in this section
will only help to find the bottom end of the essential spectrum,
but no further information about it at all.

Another consequence of corrollary \ref{cor:Convergence} is
that only a part of the eigenpairs obtained by
diagonalising the matrix $\mat{A}^{(n)}$ of \vref{eqn:DiscretisedEigenproblem}
can be trusted to carry any meaning regarding
the spectrum of the exact physical operator $\Op{A}$.
This is because the larger eigenvalues $\lambda_i^{(n)}$
of $\mat{A}^{(n)}$ will only provide an \emph{artificial}
discretisation of the essential spectrum:
Their values will all tend to $\inf \sigma_\text{ess}(\Op{A})$
as the basis set is increased.
Since the convergence to the bottom of the essential spectrum
as well as the discrete eigenvalues is always from above,
one sometimes has trouble judging whether
an eigenpair of $\mat{A}^{(n)}$
is a true discrete eigenpair of the operator or already
part of the essential spectrum.
In either case the bottom end of the spectrum of $\mat{A}$
will always carry physical meaning in our cases if the basis
set satisfies remark \vref{rem:ChoiceBasisFunction}.
\end{rem}

\todo[inline,caption={}]{
	Talk about resonance processes and the stabilisation method
	for modelling continuum states.
	\begin{itemize}
		\item \url{https://journals.aps.org/pra/abstract/10.1103/PhysRevA.1.1109}
		\item \url{http://pubs.acs.org/doi/abs/10.1021/j100396a027}
		\item In \cite{Riss1993} Reference of Hazi/Taylor 1970
	\end{itemize}
}
