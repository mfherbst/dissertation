\section{Projection methods for eigenproblems}
\label{sec:Projection}

Let $\Op{A}$ be a self-adjoint, bounded below operator on a separable Hilbert space
$\hilbert$ with domain $D(\Op{A})$.
We already saw in remark \vref{rem:OpRietz}
that $\Op{A}$ uniquely defines a sesquilinear form
\[ a(u, v) = \braket{u}{\Op{A} v}_\hilbert. \]
for $(u, v) \in D(\Op{A}) \times D(\Op{A})$.

\subsection{The form domain}
Even though $\Op{A}$ might only be self-adjoint on the domain $D(\Op{A})$,
the form $a(\slot, \slot)$ can often be defined sensibly
on a larger domain $Q(\Op{A})$, called the \newterm{form domain} of $\Op{A}$.
Its construction will be sketched in this section.
For more details see \cite[p. 77]{Teschl2014} or \cite[p. 276]{Reed1980}.

\noindent
Since $\Op{A}$ is semi-bounded from below, one can define a scalar product
\[
	\braket{u}{v}_{\Op{A}} \equiv \braket{u}{\Op{A} v + (C + 1) v}_{\hilbert}
	= a(u, v) + (C+1) \braket{u}{v}_{\hilbert},
\]
for all $u,v \in D(\Op{A})$.
Here $C$ is the constant of semi-boundedness of definition \vref{defn:SemiBounded}.
Clearly the associated norm $\norm{\slot}_{\Op{A}}$ satisfies
\begin{equation}
	\label{eqn:FormDomainNorm}
\norm{u}_{\Op{A}}
= \braket{u}{\Op{A} u} + (C+1) \norm{u}_{\hilbert}
\stackrel{\text{\abbr{def}{n} \ref{defn:SemiBounded}}}{\geq} \norm{u}_{\hilbert}.
\end{equation}
We now take the completion of $D(\Op{A})$ under the norm $\norm{u}_{\Op{A}}$
and call it $Q(\Op{A})$.
\eqref{eqn:FormDomainNorm} assures that all sequences,
which are Cauchy in $D(\Op{A})$ wrt. $\norm{\slot}_{\hilbert}$
are Cauchy in $Q(\Op{A})$ wrt. $\norm{\slot}_{\Op{A}}$ as well.
One can show further~\cite{Teschl2014} that such sequences
have the same limit in $Q(\Op{A})$ irrespective of the norm used.

\noindent
This allows to uniquely
extend $a(\slot, \slot)$ to $Q(\Op{A}) \times Q(\Op{A})$ by setting
\[ a(u, v) := \braket{u}{v}_{\Op{A}} - (C+1) \braket{u}{v}_{\hilbert}. \]
Naturally $Q(\Op{A})$ is the largest Hilbert space on which the form
$a(\slot, \slot)$ is defined and continuous
and we have
\[ D(\Op{A}) \subseteq Q(\Op{A}) \subseteq \hilbert, \]
where all these are dense.

\subsection{The Ritz-Galerkin projection}
\label{sec:RitzGalerkin}
The defining property of any eigenpair
$(\lambda_i, v_i) \in \sigma_P(\Op{A}) \times D(\Op{A})$
of the operator $\Op{A}$ is of course the condition
\begin{equation}
	\Op{A} v_i = \lambda_i v_i.
	\label{eqn:EigenproblemStrong}
\end{equation}
By a simple projection onto an arbitrary test function $u$,
one can show that any such eigenpair satisfies
\begin{equation}
	\forall u \in \hilbert: \quad a(u, v_i) = \lambda_i \braket{u}{v_i}_\hilbert.
	\label{eqn:EigenproblemWeak}
\end{equation}
as well, the so-called \newterm{weak formulation} of the eigenproblem.
In contrast to this, \eqref{eqn:EigenproblemStrong} is sometimes
referred to as the \newterm{strong formulation}.
A consequence of the Lax-Milgram theorem \vref{thm:LaxMilgram}
and the semi-boundedness of $\Op{A}$
is that a solution in the weak sense implies a solution in the strong sense
as well, making both formulations equivalent.

This suggests the \newterm{Ritz-Galerkin projection},
where we attempt to find an approximate solution
for \eqref{eqn:EigenproblemStrong}
by considering \eqref{eqn:EigenproblemWeak}
in a sequence of subspaces of $Q(\Op{A})$.

\begin{defn}[Ritz-Galerkin projection]
	\label{defn:RitzGalerkin}
	Let $\Op{A}$ be a self-adjoint, bounded below operator
	with form domain $Q(\Op{A})$ and associated sesquilinear form $a(\slot,\slot)$.
	Given a sequence $(S_n)_{n\in\N} \subset Q(\Op{A})$
	of finite-dimensional subspaces satisfying
	\begin{equation}
		\forall v \in Q(\Op{A}) \quad \inf_{v^{(n)} \in S_n} \norm{v - v^{(n)}}_{Q(\Op{A})} \xrightarrow{n\to \infty} 0,
		\label{eqn:CondSubspaces}
	\end{equation}
	obtain a sequence of approximate eigenspectra $\sigma^{(n)}(\Op{A})$
	by solving --- for each $n$ --- the variational problem
	\begin{equation}
		\label{eqn:varProblem}
		\left\{
		\begin{aligned}
			\text{Search $(\lambda_i^{(n)}, v_i^{(n)}) \in \R \times S_n$ such that} & \\
			\forall u^{(n)} \in S_n: \quad a(u^{(n)}, v_i^{(n)})
			&= \lambda_i^{(n)} \braket{u^{(n)}}{v_i^{(n)}}_\hilbert \\
			\norm{v_i^{(n)}}_\hilbert &= 1
		\end{aligned}
		\right\}.
	\end{equation}
\end{defn}
For ease of our discussion let $\Op{A}^{(n)}$ denote the self-adjoint
operator on a particular $S_n$, which is defined by the variational problem
\eqref{eqn:varProblem}, \ie which satisfies
\[ \forall (u^{(n)}, v^{(n)}) \in S_n \times S_n \quad \braket{\Op{A}^{(n)} u^{(n)}}{v^{(n)}}_\hilbert = a(u^{(n)}, v^{(n)}). \]
Since $S_n$ is finite-dimensional, $\Op{A}^{(n)}$ is compact%
\footnote{See proposition \vref{prop:FiniteDimCompact}.}
and thus it will have a discrete spectrum $\sigma(\Op{A}^{(n)})$.
By definition $\sigma^{(n)}(\Op{A}) = \sigma(\Op{A}^{(n)})$.

Our hope is now to construct such a sequence $(S_n)$ of subspaces,
that $\sigma(\Op{A}^{(n)})$ converges to $\sigma(\Op{A})$.
Unfortunately this is \emph{not} possible in general,
see for example \cite{Helffer2013} for details.
What this ansatz gives us, however, is a method to obtain
sensible approximations for the lower end of the spectrum,
especially all discrete eigenvalues below the essential spectrum.

For this let us first recall that the eigenvalues
in the discrete spectrum $\sigma_\text{disc}$
of a self-adjoint operator are typically indexed%
\footnote{Note, that by definition the discrete spectrum is \emph{always} countable.}
in increasing order
\[ \lambda_1 \leq \lambda_2 \leq \lambda_3 \leq \cdots. \]
by the positive natural numbers.
Now we proceed to the celebrated \newterm{min-max theorem}~\cite[p. 146]{Helffer2013}.

\begin{thm}[Courant-Fischer min-max theorem]
	\label{thm:CourantFischer}
	Let $\Op{A}$ be a self-adjoint operator on $\hilbert$,
	which is bounded below with form domain $Q(\Op{A})$
	and associated sesquilinear form $a(\slot,\slot)$.
	For each $0 < n \in \N$, we define
	\begin{equation}
		\lambda_n(\Op{A}) := \inf_{W \in \set{S}_n} \ \sup_{u \in W \backslash \{0\}}
		\frac{a(u,u)}{\norm{u}_\hilbert^2}
		\label{eqn:CourantFischer}
	\end{equation}
	where $\set{S}_n$ is the set of all $n$-dimensional subspaces of $Q(\Op{A})$.
	Then
	\begin{itemize}
		\item if $\Op{A}$ has at least $n$ eigenvalues lower
			than $\inf \sigma_\text{ess}(\Op{A})$ (counting multiplicities
			the appropriate number of times),
			then $\lambda_n(\Op{A})$ is the $n$-th eigenvalue of the discrete spectrum
			of $\Op{A}$.
		\item otherwise, $\lambda_n(\Op{A}) = \inf \sigma_\text{ess}(\Op{A})$.
	\end{itemize}
\end{thm}

\noindent
Combining this with the Ritz-Galerkin projection of definition \ref{defn:RitzGalerkin}
yields:

\begin{cor}
	\label{cor:Convergence}
	Let $\Op{A}$ be a bounded below, self-adjoint operator on $\hilbert$
	and let $\Op{A}^{(n)}$ be the approximation to $\Op{A}$
	according to the variational Ritz-Galerkin ansatz
	\eqref{eqn:varProblem}
	in a sequence of subspaces $(S_n)$,
	which satisfy condition \eqref{eqn:CondSubspaces}.
	Then
	\[ \forall 0 < i \in \N \quad \lambda^{(n)}_i := \lambda_i(\Op{A}^{(n)}) \xrightarrow{n\to\infty} \lambda_i(\Op{A}), \]
	where the convergence is always from above.
	\begin{proof}
		\todoil{Find reference}
	\end{proof}
\end{cor}


% -------------------------------



These results give us twofold.
First they provide us with a route to numerically obtain
the first few eigenvalues, which we are particular interested in.
See the discussion in section \vref{sec:SpectralTakeAway}.
Important is that the subspaces we construct satisfy \eqref{eqn:CondSubspaces},
which is always the case for dense subspaces.

Secondly they provide theoretical basis
for our claim in remark \vref{rem:NumericalEssentialSpectrum}.
We said that it was very difficult to obtain
numerical approximations to the essential spectrum.
In fact we now can state that all variational
approaches based on Ritz-Galerkin projections
will fail and only provide us with a sequence of eigenvalues
tending to the infimum of the essential spectrum
for all approximations above the infimum to the essential spectrum.
\todoil{Maybe an example for this breakdown}


% -------------------------------

Since all $\Nbas$-dimensional complex-valued Hilbert spaces $S_n \subseteq \hilbert$
are isomorphic to $\C^{\Nbas}$ by the means of choosing an appropriate
basis set $\{\chi_\mu\}_{\mu \in \Ibas}$ to express the space $S_n$.
We denote with $\chi_\mu$ the $\mu$-th basis function.
In this convention we can identify $\Op{A}^{(n)}$ with the matrix $\mat{A}$
consisting of elements
\[ A_{\mu\nu} = a(\chi_\mu, \chi_\nu) \]
and thus a plain diagonalisation of $\mat{A}$ yields $\sigma(\mat{A})$,
an approximation to the spectrum of $\Op{A}$.

\todoil{
We can relax $H^2(\R^3)$ to $H^1(\R^3)$ if we use test functions
from $H^1(\R^3)$ as well in Ritz-Galerkin scheme.
}
