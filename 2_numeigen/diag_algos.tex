\section{Diagonalisation algorithms}
\label{sec:DiagAlgos}

This section discusses a few algorithms in order to obtain
numerical approximations to a part of the eigenpairs of a matrix $\mat{A}$.
In this thesis we will only discuss combinations of operators
and discretisation methods with basis sets
$\{\varphi_\mu\}_{\mu \in \Ibas}$, which have the property that
\[ \forall \mu, \nu \in \Ibas: \quad a(\varphi_\mu, \varphi_\nu) \in \R. \]
As a result all matrices in \eqref{eqn:DiscretisedEigenproblem}
will be real and symmetric.
This section is therefore only concerned with eigenproblems of the type
\[ \mat{A} \vec{u}_i = \lambda_i \vec{u}_i \]
where $\mat{A} \in \R^{\Nbas \times \Nbas}$,
$\lambda_i \in \R$ and $\vec{u}_i \in \R^{\Nbas}$.

\todoil{Improve and extend this section once Chapter 4 and 5 are done}
\subsection{Power method}
The most simple numerical method to obtain
a single eigenvalue from a particular matrix $\mat{A}$
is the Power method.
The algorithmic scheme consists of the steps

\todoil{Algorithm}

where
\[ \rho_R(\vec{u}) := \frac{\vec{u}^T \mat{A} \vec{u}}{\vec{u}^T \vec{u}} \]
is known as the Rayleigh quotient,
the discretised version of \eqref{eqn:CourantFischer}.

Let us define
\[ \lambda_\text{max} = \argmax_{\lambda \in \sigma(\mat{A})} \abs{\lambda}. \]
Provided that $\lambda_\text{max}$ is unique,
\ie no other eigenvalue of this magnitude exists,
this algorithm converges to $\lambda_\text{max}$.
To see this, let us write
\begin{align}
	a
\end{align}



\todoil{Do later if time}

\subsection{Shift and invert}
\todoil{Do later if time}

\subsection{Arnoldi's method}
\todoil{Do later if time}

\subsection{Davidson's method}
\label{sec:Davidson}
\todoil{Do later if time}


\subsection{Generalised eigenvalue problems}
\label{sec:GeneralisedEigenvalueProblem}
Give some hints how such algorithms
could be adapted for a generalised eigenvalue problem.

\todoil{Copy stuff from status\_Sturmians}

\begin{itemize}
	\item $S^{-1/2}$
	\item Choleski
	\item Change orthogonalisation
\end{itemize}


