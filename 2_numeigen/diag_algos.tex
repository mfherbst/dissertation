\section{Diagonalisation algorithms}
\label{sec:DiagAlgos}

This section discusses a few algorithms suitable for obtaining
approximations to eigenpairs of a matrix $\mat{A}$ numerically.
Whilst the regime of quantum mechanics is a complex-valued Hilbert space,
in this thesis
we will only consider combinations of operators and discretisation basis
$\{\varphi_\mu\}_{\mu \in \Ibas}$,
which have the property that
\[ \forall \mu, \nu \in \Ibas: \quad a(\varphi_\mu, \varphi_\nu) \in \R. \]
As a result all matrices in \eqref{eqn:DiscretisedEigenproblem} will be real and symmetric.
In this section we will therefore only consider eigenproblems of the type
\[ \mat{A} \vec{u}_i = \lambda_i \vec{u}_i \]
where $\mat{A} \in \R^{\Nbas \times \Nbas}$,
$\lambda_i \in \R$ and $\vec{u}_i \in \R^{\Nbas}$.

Furthermore we will restrict our discussion here to so-called
\newterm{iterative diagonalisation methods},
where better and better approximations to the eigenpairs of interest
are found in an iterative procedure.
These methods are in contrast to so-called \newterm{direct diagonalisation methods},
where one attempts to find the elements of an orthogonal matrix $\mat{O} \in \R^{\Nbas\times\Nbas}$,
such that
\[ \rtp{\mat{O}} \mat{A} \mat{O} = \mat{L} = \diag(\lambda_1, \lambda_2, \ldots, \lambda_{\Nbas}), \]
the diagonal matrix of eigenvalues, can be computed directly.
This requires random access into the elements of $\mat{A}$,
especially since the efficient implementations of such methods
are usually destructive towards $\mat{A}$.
This means that they avoid memory allocations by building the product
$\rtp{\mat{O}} \mat{A} \mat{O}$ directly inside the memory already allocated for $\mat{A}$.
Direct methods are generally only suitable for small matrices,
where $\Nbas$ is at most on the order of $1000$.
The most widely adopted library of dense methods is LAPACK~\cite{LAPACK}.

\subsection{Power method}
The most simple numerical method to obtain
a single eigenvalue from a particular matrix $\mat{A}$
is the Power method.
The algorithmic scheme consists of the steps

% https://www.math.tu-berlin.de/fachgebiete_ag_modnumdiff/fg_modellierung_simulation_und_optimierung_in_natur_und_ingenieurswissenschaften/v_menue/mitarbeiter/prof_dr_reinhold_schneider/publikationen/


\todoil{Algorithm}

where
\[ \rho_R(\vec{u}) := \frac{\vec{u}^T \mat{A} \vec{u}}{\vec{u}^T \vec{u}} \]
is known as the Rayleigh quotient,
the discretised version of \eqref{eqn:CourantFischer}.

Let us define
\[ \lambda_\text{max} = \argmax_{\lambda \in \sigma(\mat{A})} \abs{\lambda}. \]
Provided that $\lambda_\text{max}$ is unique,
\ie no other eigenvalue of this magnitude exists,
this algorithm converges to $\lambda_\text{max}$.
To see this, let us write
\begin{align}
	a
\end{align}

Gerschgorin's circles


\todoil{Do later if time}

\subsection{Shift and invert}
\label{sec:ShiftInvert}
\todoil{Do later if time}
Select the range of interest
Wilkinson's method

\subsection{Arnoldi's method}
\label{sec:Arnoldi}
Arnoldi~\cite{Arnoldi1951}
\todoil{Do later if time}

\subsection{Davidson's method}
\label{sec:Davidson}
\todoil{Do later if time}

mention and explain preconditioning in this context

\subsection{Generalised eigenvalue problems}
\label{sec:GeneralisedEigenvalueProblem}
Give some hints how such algorithms
could be adapted for a generalised eigenvalue problem.

\todoil{Copy stuff from status\_Sturmians}

\begin{itemize}
	\item $S^{-1/2}$
	\item Choleski
	\item Change orthogonalisation
\end{itemize}


