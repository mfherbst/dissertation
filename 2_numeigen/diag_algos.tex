\section{Diagonalisation algorithms}
\todoil{Improve and extend this section once Chapter 4 and 5 are done}

This section tries to shine some light on a few numerical techniques
to obtain the exact eigenspectrum or an approximation for the eigenspectrum
of a matrix $\mat{A}$.

In this thesis we will only discuss combinations of operators
and discretisation methods which have the property
that
\[ a(\chi_\mu, \chi_\nu) \in \R \]
such that we are overall only left with symmetric matrices
$\mat{A} \in \R^{\Nbas \times \Nbas}$.
In this section we will therefore restrict our discussion
of diagonalisation algorithms to plainly real-symmetric matrices.
We can thus assume that both the matrix element
as well as the eigenvalues and eigenvectors are entirely real.

\subsection{Power method}
The most simple numerical method to obtain
a single eigenvalue from a particular matrix $\mat{A}$
is the Power method.
The algorithmic scheme consists of the steps

\todoil{Algorithm}

where
\[ \rho_R(\vec{u}) := \frac{\vec{u}^T \mat{A} \vec{u}}{\vec{u}^T \vec{u}} \]
is known as the Reighleigh quotient,
the discretised version of \eqref{eqn:CourantFischer}.

Let us define
\[ \lambda_\text{max} = \argmax_{\lambda \in \sigma(\mat{A})} \abs{\lambda}. \]
Provided that $\lambda_\text{max}$ is unique,
\ie no other eigenvalue of this magnitude exists,
this algorithm converges to $\lambda_\text{max}$.
To see this, let us write
\begin{align}
	a
\end{align}



\todoil{Do later if time}

\subsection{Shift and invert}
\todoil{Do later if time}

\subsection{Arnoldi's method}
\todoil{Do later if time}

\subsection{Davidson's method}
\todoil{Do later if time}
