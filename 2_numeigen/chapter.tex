\chapter{Numerical treatment of spectral problems}
\label{ch:numeigen}

\chapquote{
	It is a well-known experience that the only truly
	enjoyable and profitable way of studying mathematics
	is the method of ``filling in details'' by one's
	own efforts.
}{Cornelius Lanczos~(1893--1974)}

\section{Ritz-Galerkin projection}
\label{sec:RitzGalerkin}
Let us consider in this chapter a self-adjoint operator $\Op{A}$
on a Hilbert space $\hilbert$ with domain $D(\Op{A})$.
We already saw in \eqref{eqn:formA} that one is able
to associate to such an operator an appropriate quadratic form
\[ a(u, v) = \braket{u}{\Op{A} v}_\hilbert. \]
Whilst $\Op{A}$ might only be self-adjoint the domain $D(\Op{A})$,
one may define the \newterm{form domain} $Q(\Op{A})$ as the largest
natural Hilbert space on which the quadratic form $a(u, v)$
is defined and continuous by the means of functional calculus.
We have
\[ D(\Op{A}) \subseteq Q(\Op{A}) \subseteq \hilbert. \]

Clearly an eigenpair $(\lambda_i, v_i) \in \sigma_P(\Op{A}) \times D(\Op{A})$
satisfies the variational formulation of the eigenproblem
\[ \forall u \in \hilbert: a(u, v_i) = \lambda_i \braket{u}{v_i}_\hilbert. \]
This suggests the \newterm{Ritz-Galerkin projection},
which attempts to find an approximate solution by considering
the related variational problem:
\begin{center}
\begin{minipage}{0.9\textwidth}
	Search $(\lambda_i^{(n)}, v_i^{(n)}) \in \R \times S_n$ such that
\end{minipage} \\[-1.3em]
\begin{align}
	\label{eqn:varProblem1}
	\forall u^{(n)} \in S_n: \quad a(u^{(n)}, v_i^{(n)})
		&= \lambda_i^{(n)} \braket{u^{(n)}}{v_i^{(n)}}_\hilbert \\
	\label{eqn:varProblem2}
	\norm{v_i^{(n)}}_\hilbert &= 1
\end{align}
\end{center}
in a sequence of finite-dimensional subspaces $(S_n)_{n\in\N}$
of $Q(\Op{A})$ satisfying
\begin{equation}
	\forall v \in Q(\Op{A}) \quad \inf_{v^{(n)} \in S_n} \norm{v - v^{(n)}}_{Q(\Op{A})} \xrightarrow{n\to \infty} 0.
	\label{eqn:CondSubspaces}
\end{equation}
For ease of future discussion let $\Op{A}^{(n)}$ denote the self-adjoint
operator on $S_n$ defined by the variational problem, \ie
\[ \forall (u^{(n)}, v^{(n)}) \in S_n \times S_n \quad \braket{\Op{A}^{(n)} u^{(n)}}{v^{(n)}}_\hilbert = a(u^{(n)}, v^{(n)}). \]

Since $S_n$ is finite-dimensional and hence $\Op{A}^{(n)}$ compact,
which allows us to find a discrete spectrum for $\Op{A}^{(n)}$.
Our hope is now to construct such a sequence of subspaces,
that $\sigma(\Op{A}^{(n)})$ converges to $\sigma(\Op{A})$.
Unfortunately this is \emph{not} possible in general.
See~\cite{Helffer2013} for details.

We will discuss in this section some results,
which hold for some special cases, which are applicable to us.





In the following we assume the convention,
that the eigenvalues in the discrete spectrum $\sigma_\text{disc}$
are indexed such that
\[ \lambda_1 \leq \lambda_2 \leq \lambda_3 \leq \cdots \]
in case this set is countable.

\begin{thm}[Courant-Fischer min-max theorem]
	\label{thm:CourantFischer}
	Let $\Op{A}$ be a self-adjoint operator on $\hilbert$,
	which is bounded below with form domain $Q(\Op{A})$
	and associated quadratic form $a(\slot,\slot)$.
	For each $0 < i \in \N$, we define
	\begin{equation}
		\lambda_i(\Op{A}) = \inf_{W \in \set{S}_i} \ \sup_{w \in W \backslash \{0\}}
		\frac{a(w,w)}{\norm{w}_\hilbert^2}
		\label{eqn:CourantFischer}
	\end{equation}
	where $\set{S}_i$ is the set of all $i$-dimensional subspaces of $Q(\Op{A})$.
	Then
	\begin{itemize}
		\item if $\Op{A}$ has at least $i$ eigenvalues lower
			than $\min \sigma_\text{ess}(\Op{A})$ (counting multiplicities multiple times),
			then $\lambda_i(\Op{A})$ is the $i$th eigenvalue of the discrete spectrum
			of $\Op{A}$.
		\item otherwise, $\lambda_i(\Op{A}) = \min \sigma_\text{ess}(\Op{A})$.
	\end{itemize}
% see Helffer2013 p.144
\end{thm}

\begin{thm}
	Let $\Op{A}$ be a bounded below, self-adjoint operator on $\hilbert$
	and let $\Op{A}^{(n)}$ be the approximation to $\Op{A}$
	according to the variational Ritz-Galerkin ansatz
	\eqref{eqn:varProblem1}--\eqref{eqn:varProblem2}
	in a sequence of subspaces $(S_n)$,
	which satisfy condition \eqref{eqn:CondSubspaces}.
	Then
	\[ \forall 0 < i \in \N \quad \lambda^{(n)}_i := \lambda_i(\Op{A}^{(n)}) \xrightarrow{n\to\infty} \lambda_i(\Op{A}) \]
	where the convergence is always from above.
	\begin{proof}
		\todoil{Find reference}
	\end{proof}
\end{thm}

This implies that we can approximate the first few eigenvalues
of a semibounded operator,
which sit below the essential spectrum via a variational approximation.
For the hydrogenic operator of section \vref{sec:HydrogenAtom}
this implies that we are fine
as long as the subspaces we construct satisfy \eqref{eqn:CondSubspaces},
which is always the case for dense subspaces.

Since all $\Nbas$-dimensional complex-valued Hilbert spaces $S_n \subseteq \hilbert$
are isomorphic to $\C^{\Nbas}$ by the means of choosing an appropriate
basis set $\{\chi_\mu\}_{\mu \in \Ibas}$ to express the space $S_n$.
We denote with $\chi_\mu$ the $\mu$-th basis function.
In this convention we can identify $\Op{A}^{(n)}$ with the matrix $\mat{A}$
consisting of elements
\[ A_{\mu\nu} = a(\chi_\mu, \chi_\nu) \]
and thus a plain diagonalisation of $\mat{A}$ yields $\sigma(\mat{A})$,
an approximation to the spectrum of $\Op{A}$.

\todo[inline,caption={}]{
	\begin{itemize}
		\item Note examples where it breaks down
		\item Note some stuff, how to do this in practice
		\item Make the connection to matrices and linear algebra
	\end{itemize}

We can relax $H^2(\R^3)$ to $H^1(\R^3)$ if we use test functions
from $H^1(\R^3)$ as well in Ritz-Galerkin scheme.
}

\section{Diagonalisation algorithms}
This section tries to shine some light on a few numerical techniques
to obtain the exact eigenspectrum or an approximation for the eigenspectrum
of a matrix $\mat{A}$.

In this thesis we will only discuss combinations of operators
and discretisation methods which have the property
that
\[ a(\chi_\mu, \chi_\nu) \in \R \]
such that we are overall only left with symmetric matrices
$\mat{A} \in \R^{\Nbas \times \Nbas}$.
In this section we will therefore restrict our discussion
of diagonalisation algorithms to plainly real-symmetric matrices.
We can thus assume that both the matrix element
as well as the eigenvalues and eigenvectors are entirely real.

\subsection{Power method}
The most simple numerical method to obtain
a single eigenvalue from a particular matrix $\mat{A}$
is the Power method.
The algorithmic scheme consists of the steps

\todoil{Algorithm}

where
\[ \rho_R(\vec{u}) := \frac{\vec{u}^T \mat{A} \vec{u}}{\vec{u}^T \vec{u}} \]
is known as the Reighleigh quotient,
the discretised version of \eqref{eqn:CourantFischer}.

Let us define
\[ \lambda_\text{max} = \argmax_{\lambda \in \sigma(\mat{A})} \abs{\lambda}. \]
Provided that $\lambda_\text{max}$ is unique,
\ie no other eigenvalue of this magnitude exists,
this algorithm converges to $\lambda_\text{max}$.
To see this, let us write
\begin{align}
	a
\end{align}



\todoil{Do later if time}

\subsection{Shift and invert}
\todoil{Do later if time}

\subsection{Arnoldi's method}
\todoil{Do later if time}

\subsection{Davidson's method}
\todoil{Do later if time}
