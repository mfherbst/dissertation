\chapter{Numerical treatment of eigenvalue problems}

\chapquote{
	It is a well-known experience that the only truly
	enjoyable and profitable way of studying mathematics
	is the method of ``filling in details'' by one's
	own efforts.
}{Cornelius Lanczos~(1893--1974)}

\section{Ritz-Galerkin projection}
\label{sec:RitzGalerkin}
Let us consider in this chapter a self-adjoint operator $\Op{A}$
on a Hilbert space $\hilbert$ with domain $D(\Op{A})$.
We already saw in \eqref{eqn:formA} that one is able
to associate to such an operator an appropriate quadratic form
\[ a(u, v) = \braket{u}{\Op{A} v}_\hilbert. \]
Whilst $\Op{A}$ might only be self-adjoint the domain $D(\Op{A})$,
one may define the \newterm{form domain} $Q(\Op{A})$ as the largest
natural Hilbert space on which the quadratic form $a(u, v)$
is defined and continuous by the means of functional calculus.
We have
\[ D(\Op{A}) \subseteq Q(\Op{A}) \subseteq \hilbert. \]

Clearly an eigenpair $(\lambda_i, v_i) \in \sigma_P(\Op{A}) \times D(\Op{A})$
satisfies the variational formulation of the eigenproblem
\[ \forall u \in \hilbert: a(u, v_i) = \lambda_i \braket{u}{v_i}_\hilbert. \]
This suggests the \newterm{Ritz-Galerkin projection},
which attempts to find an approximate solution by considering
the related variational problem
\begin{align}
	\text{Search $(\lambda_i^{(n)}, v_i^{(n)}) \in \R \times S_n$ such that} \\
	\forall u^{(n)} \in S_n \quad a(u^{(n)}, v_i^{(n)})
		= \lambda_i^{(n)} \braket{u^{(n)}}{v_i^{(n)}}_\hilbert \\
	\norm{v_i^{(n)}}_\hilbert = 1
	\label{eqn:varProblem}
\end{align}
in a sequence of finite-dimensional subspaces $(S_n)_{n\in\N}$
of $Q(\Op{A})$ satisfying
\[ \forall v \in \Op{A} \quad \inf_{v^{(n)} \in S_n} \norm{v - v^{(n)}}_{Q(\Op{A})} \xrightarrow{n\to \infty} 0. \]
For ease of future discussion let $\Op{A}^{(n)}$ denote the self-adjoint
operator on $S_n$ defined by the variational problem, \ie
\[ \forall (u^{(n)}, v^{(n)}) \in S_n \times S_n \quad \braket{\Op{A}^{(n)} u^{(n)}}{v^{(n)}}_\hilbert = a(u^{(n)}, v^{(n)}). \]

Since $S_n$ is finite-dimensional and hence $\Op{A}^{(n)}$ compact,
which allows us to find a discrete spectrum for $\Op{A}^{(n)}$.
Our hope is now to construct such a sequence of subspaces,
that $\sigma(\Op{A}^{(n)})$ converges to $\sigma(\Op{A})$.
Unfortunately this is \emph{not} possible in general.
See~\cite{Helffer2013} for details.

We will discuss in this section some results,
which hold for some special cases, which are applicable to us.





In the following we assume the convention,
that the eigenvalues in the discrete spectrum $\sigma_\text{disc}$
are indexed as such
\[ \lambda_0 \leq \lambda_1 \leq \lambda_2 \leq \cdots. \]
\todoil{Check that the discrete spectrum is countable}

\begin{thm}[Courant-Fischer min-max theorem]
	\label{thm:CourantFischer}
	Let $\Op{A}$ be a self-adjoint operator on $\hilbert$,
	which is bounded below with form domain $Q(\Op{A})$
	and associated quadratic form $a(\slot,\slot)$.
	For each $0 < i \in \N$, we define
	\begin{equation}
		\lambda_i(\Op{A}) = \inf_{W \in \set{S}_i} \sup_{w \in W \backslash \{0\}}
		\frac{a(w,w)}{\norm{w}_\hilbert^2}
		\label{eqn:CourantFischer}
	\end{equation}
	where $\set{S}_i$ is the set of all $i$-dimensional subspaces of $Q(\Op{A})$.
	Then
	\begin{itemize}
		\item if $\Op{A}$ has at least $i$ eigenvalues lower
			than $\min \sigma_\text{ess}(\Op{A})$ (counting multiplicities multiple times),
			then $\lambda_i(\Op{A})$ is the $i$th eigenvalue of the discrete spectrum
			of $\Op{A}$.
		\item otherwise, $\lambda_i(\Op{A}) = \min \sigma_\text{ess}(\Op{A})$.
	\end{itemize}
% see Helffer2013 p.144
\end{thm}

\begin{thm}
	If we are in the setting of theorem \vref{thm:CourantFischer}
	and follow the Ritz-Galerkin ansatz, then
	\[ \forall 0 < i \in \N \quad \lambda^{(n)}_i := \lambda_i(\Op{A}^{(n)}) \xrightarrow{n\to\infty} \lambda_i(\Op{A}) \]
	where the convergence is always from above.
	% TODO I copied this from Eric \ldots check where he got this from
\end{thm}


\todo[inline,caption={}]{
	\begin{itemize}
		\item Note examples where it breaks down
		\item Note some stuff, how to do this in practice
		\item Make the connection to matrices and linear algebra
	\end{itemize}

We can relax $H^2(\R^3)$ to $H^1(\R^3)$ if we use test functions
from $H^1(\R^3)$ as well in Ritz-Galerkin scheme.
}

\section{Diagonalisation algorithms}
\todo[inline,caption={}]{
	\begin{itemize}
		\item Introduce linear algebra and sketch some diagonalisation algos
		\item Brief rationale
		\item Basic ideas
		\item Sketch Davidson and Lanczos
	\end{itemize}
}
