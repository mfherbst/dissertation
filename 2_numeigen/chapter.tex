\chapter{Numerical treatment of spectral problems}
\label{ch:numeigen}

\chapquote{
	It is a well-known experience that the only truly
	enjoyable and profitable way of studying mathematics
	is the method of ``filling in details'' by one's
	own efforts.
}{Cornelius Lanczos~(1893--1974)}

\noindent
This chapter discusses numerical techniques
and algorithms,
which can be used for obtaining a few of the discrete
eigenvalues and corresponding eigenstates
of a self-adjoint operator.
For simplifying the discussion we will restrict ourselves
to the cases where the eigenvalues of interest
are located at the lower end of the discrete spectrum
and are well-separated from the essential part of the spectrum.
Notice that this is not the case for all regimes of quantum chemistry
or even electronic structure theory.
See the discussion in section \vref{sec:SpectralTakeAway} for examples,
where this assumption is violated.

\section{Projection methods for eigenproblems}
\label{sec:Projection}

Let $\Op{A}$ be a self-adjoint, bounded below operator on a separable Hilbert space
$\hilbert$ with domain $D(\Op{A})$.
We already saw in remark \vref{rem:OpRietz}
that $\Op{A}$ uniquely defines a sesquilinear form
\[ a(u, v) = \braket{u}{\Op{A} v}_\hilbert \]
for $(u, v) \in D(\Op{A}) \times D(\Op{A})$.

\subsection{Form domains of operators}
Even though $\Op{A}$ might only be self-adjoint on the domain $D(\Op{A})$,
the form $a(\slot, \slot)$ can often be defined sensibly
on a larger domain $Q(\Op{A})$, called the \newterm{form domain} of $\Op{A}$.
Its construction will be sketched in this section.
For more details see \cite[p. 77]{Teschl2014} or \cite[p. 276]{Reed1980}.

\noindent
Since $\Op{A}$ is semi-bounded from below, one can define a scalar product
\[
	\braket{u}{v}_{\Op{A}} \equiv \braket{u}{\Op{A} v + (C + 1) v}_{\hilbert}
	= a(u, v) + (C+1) \braket{u}{v}_{\hilbert},
\]
for all $u,v \in D(\Op{A})$.
Here $C$ is the constant of semi-boundedness of definition \vref{defn:SemiBounded}.
Clearly the associated norm $\norm{\slot}_{\Op{A}}$ satisfies
\begin{equation}
	\label{eqn:FormDomainNorm}
\norm{u}_{\Op{A}}
= \braket{u}{\Op{A} u} + (C+1) \norm{u}_{\hilbert}
\stackrel{def. \ref{defn:SemiBounded}}{\geq} \norm{u}_{\hilbert}.
\end{equation}
We now take the completion of $D(\Op{A})$ under the norm $\norm{u}_{\Op{A}}$
and call it $Q(\Op{A})$.
\eqref{eqn:FormDomainNorm} assures that all sequences,
which are Cauchy in $D(\Op{A})$ wrt. $\norm{\slot}_{\hilbert}$
are Cauchy in $Q(\Op{A})$ wrt. $\norm{\slot}_{\Op{A}}$ as well.
One can show further~\cite{Teschl2014} that such sequences
have the same limit in $Q(\Op{A})$ irrespective of the norm used.

\noindent
This allows to uniquely
extend $a(\slot, \slot)$ to $Q(\Op{A}) \times Q(\Op{A})$ by setting
\[ a(u, v) := \braket{u}{v}_{\Op{A}} - (C+1) \braket{u}{v}_{\hilbert}. \]
Constructed as such $Q(\Op{A})$ is the largest Hilbert space on which the form
$a(\slot, \slot)$ is defined and continuous.
The form domain satisfies
\[ D(\Op{A}) \subseteq Q(\Op{A}) \subseteq \hilbert, \]
where the subspaces are dense in the respective larger space.

\begin{exmp}
	For all cases we discussed in the previous chapter,
	that is the Laplace operator $\Delta$ and the hydrogenic
	Hamiltonian $-\frac12 \Delta - \frac{Z}{r}$,
	the form domain is $H^1(\R^3)$.
	This can be easily verified by constructing the expression
	for the form $a(\slot, \slot)$ and applying partial integration.
\end{exmp}

\subsection{The Ritz-Galerkin projection}
\label{sec:RitzGalerkin}
The defining property of any eigenpair
$(\lambda_i, v_i) \in \sigma_P(\Op{A}) \times D(\Op{A})$
of the operator $\Op{A}$ is of course the condition
\begin{equation}
	\Op{A} v_i = \lambda_i v_i.
	\label{eqn:EigenproblemStrong}
\end{equation}
By a simple projection onto an arbitrary test function $u$,
one can show that any such eigenpair satisfies
\begin{equation}
	\forall u \in \hilbert: \quad a(u, v_i) = \lambda_i \braket{u}{v_i}_\hilbert.
	\label{eqn:EigenproblemWeak}
\end{equation}
as well, the so-called \newterm{weak formulation} of the eigenproblem.
In contrast to this, \eqref{eqn:EigenproblemStrong} is sometimes
referred to as the \newterm{strong formulation}.
A consequence of the Lax-Milgram theorem~\cite[p. 23]{Helffer2013}
% TODO OPTIONAL \vref{thm:LaxMilgram}
and the semi-boundedness of $\Op{A}$
is that a solution in the weak sense implies a solution in the strong sense
as well, making both formulations equivalent.

This suggests the \newterm{Ritz-Galerkin projection},
where one attempts to find an approximate solution
for \eqref{eqn:EigenproblemStrong}
by considering \eqref{eqn:EigenproblemWeak}
in a sequence of subspaces of $Q(\Op{A})$.

\begin{defn}[Ritz-Galerkin projection]
	\label{defn:RitzGalerkin}
	Let $\Op{A}$ be a self-adjoint, bounded below operator
	with form domain $Q(\Op{A})$ and associated sesquilinear form $a(\slot,\slot)$.
	Given a sequence $(S_n)_{n\in\N} \subset Q(\Op{A})$
	of finite-dimensional subspaces satisfying
	\begin{equation}
		\forall v \in Q(\Op{A}) \quad \inf_{v^{(n)} \in S_n} \norm{v - v^{(n)}}_{Q(\Op{A})} \xrightarrow{n\to \infty} 0,
		\label{eqn:CondSubspaces}
	\end{equation}
	obtain a sequence of approximate eigenspectra $\sigma^{(n)}(\Op{A})$
	by solving --- for each $n$ --- the variational problem
	\begin{equation}
		\label{eqn:varProblem}
		\left\{
		\begin{aligned}
			\text{Search $(\lambda_i^{(n)}, v_i^{(n)}) \in \R \times S_n$ such that} & \\
			\forall u^{(n)} \in S_n: \quad a(u^{(n)}, v_i^{(n)})
			&= \lambda_i^{(n)} \braket{u^{(n)}}{v_i^{(n)}}_\hilbert \\
			\norm{v_i^{(n)}}_\hilbert &= 1
		\end{aligned}
		\right\}.
	\end{equation}
\end{defn}
For ease of our discussion let $\Op{A}^{(n)}$ denote the self-adjoint
operator on a particular $S_n$, which is defined by the variational problem
\eqref{eqn:varProblem}, \ie which satisfies
\[ \forall (u^{(n)}, v^{(n)}) \in S_n \times S_n \quad \braket{\Op{A}^{(n)} u^{(n)}}{v^{(n)}}_\hilbert = a(u^{(n)}, v^{(n)}). \]
Since $S_n$ is finite-dimensional, $\Op{A}^{(n)}$ is compact%
\footnote{See proposition \vref{prop:FiniteDimCompact}.}
and thus it will have a discrete spectrum $\sigma(\Op{A}^{(n)})$.
By definition $\sigma^{(n)}(\Op{A}) = \sigma(\Op{A}^{(n)})$.

Our hope is now to construct such a sequence $(S_n)$ of subspaces,
that $\sigma(\Op{A}^{(n)})$ converges to $\sigma(\Op{A})$.
Unfortunately this is \emph{not} possible in general,
see \cite{Helffer2013} for details.
What can be achieved, however, is a method to obtain
sensible approximations for the lower end of the spectrum,
especially all discrete eigenvalues below the essential spectrum.

Let us first state the theoretical basis
in the form of the celebrated \newterm{min-max theorem}~\cite[p. 146]{Helffer2013}.
In our discussion here, we follow the usual convention,
where the eigenvalues in the discrete spectrum are indexed%
\footnote{This can always be done,
	since by definition the discrete spectrum is \emph{always} countable.}
in increasing order, \ie
\[ \lambda_1 \leq \lambda_2 \leq \lambda_3 \leq \cdots. \]

\begin{thm}[Courant-Fischer min-max theorem]
	\label{thm:CourantFischer}
	Let $\Op{A}$ be a self-adjoint operator on $\hilbert$,
	which is bounded below with form domain $Q(\Op{A})$
	and associated sesquilinear form $a(\slot,\slot)$.
	For each $0 < n \in \N$, we define
	\begin{equation}
		\lambda_n(\Op{A}) := \inf_{W \in \set{S}_n} \ \sup_{u \in W \backslash \{0\}}
		\frac{a(u,u)}{\norm{u}_\hilbert^2}
		\label{eqn:CourantFischer}
	\end{equation}
	where $\set{S}_n$ is the set of all $n$-dimensional subspaces of $Q(\Op{A})$.
	Then
	\begin{itemize}
		\item if $\Op{A}$ has at least $n$ eigenvalues lower
			than $\inf \sigma_\text{ess}(\Op{A})$ (counting multiplicities
			the appropriate number of times),
			then $\lambda_n(\Op{A})$ is the $n$-th eigenvalue of the discrete spectrum
			of $\Op{A}$,
		\item otherwise, $\lambda_n(\Op{A}) = \inf \sigma_\text{ess}(\Op{A})$.
	\end{itemize}
\end{thm}

\noindent
Combining this with the Ritz-Galerkin projection of definition \ref{defn:RitzGalerkin}
yields:\todo{Find reference}

\begin{cor}
	\label{cor:Convergence}
	Let $\Op{A}$ be a bounded below, self-adjoint operator on $\hilbert$
	and let $(S_n)$ be a sequence of subspaces of the form domain,
	which satisfy condition \eqref{eqn:CondSubspaces}.
	If we denote with $\Op{A}^{(n)}$
	the approximations to $\Op{A}$
	according to the variational Ritz-Galerkin ansatz
	\eqref{eqn:varProblem}, then
	\[ \forall 0 < i \in \N \quad \lambda^{(n)}_i := \lambda_i(\Op{A}^{(n)}) \xrightarrow{n\to\infty} \lambda_i(\Op{A}), \]
	where the convergence is always from above.
\end{cor}

As discussed in section \vref{sec:SpectralTakeAway}
those operators,
which will be considered in this thesis,
always possess a discrete spectrum located below the essential spectrum.
Furthermore we will always be interested
in those bound states located at the lower end of the discrete spectrum
for physical reasons.
With the aforementioned results we can
sketch an approximation method for our setting.

\begin{rem}[Approximation of the bottom of the discrete spectrum]
	\label{rem:ApproxBottomDiscrete}
	Let $\Op{A}$ be a self-adjoint, bounded below operator
	and let us assume that we seek approximations
	for a few discrete eigenvalues,
	which are all located at the bottom of the spectrum $\sigma(\Op{A})$
	and well below the essential spectrum.

	Let $U \subset Q(\Op{A})$ be a dense subspace.
	We can span a sequence of subspaces $(S_{\Nbas}) \subset Q(\Op{A})$
	by selecting larger and larger sets $\{\varphi_\mu\}_{\mu \in \Ibas}$
	of $\Nbas = \abs{\Ibas}$ orthonormal basis functions $\varphi_\mu \in U$
	as the bases.
	Since $U$ is a dense subspace of the separable Hilbert space
	$\hilbert$, it is separable as well.
	Therefore we know that
	in the limit of $\Nbas \to \infty$,
	$\spacespan \{\varphi_\mu\}_{\mu \in \Ibas}$ will tend towards $U$.
	Thus $\spacespan \{\varphi_\mu\}_{\mu \in \Ibas}$ eventually allows
	to construct Cauchy sequences, which approximate each $v \in Q(\Op{A})$
	up to arbitrary accuracy.
	In other words the sequence $(S_{\Nbas})$ with $\Nbas$ increasing
	satisfies condition \eqref{eqn:CondSubspaces}.

	Because of corollary \ref{cor:Convergence}
	we can thus get arbitrarily accurate approximations
	to our eigenvalues of interest
	by solving the variational problem \eqref{eqn:varProblem}
	in subspaces spanned by larger and larger
	basis sets $\{\varphi_\mu\}_{\mu \in \Ibas} \subset U$.
	This results in more and more accurate approximations
	of the corresponding bound eigenstates as well.
\end{rem}

\begin{rem}[Discrete formulation of \eqref{eqn:varProblem}]
	\label{rem:DiscreteFormulation}
	We are again in the setting of remark \ref{rem:ApproxBottomDiscrete}.

	\noindent
	If $S_{\Nbas} = \spacespan \{\varphi_\mu\}_{\mu \in \Ibas}$, we can expand
	\[ v^{(n)}_i = \sum_{\nu\in\Ibas} C^{(n)}_{\nu i} \varphi_\nu \qquad \text{with} \quad C^{(n)}_{\nu i} \equiv \braket{\varphi_\nu}{v^{(n)}_i} \]
	and thus reformulate \eqref{eqn:varProblem} to become
	\begin{equation}
		\label{eqn:DiscretisedEigenproblemHalfWay}
		\left\{
		\begin{aligned}
			\text{Search $\lambda_i^{(n)}$ and $C^{(n)}_{\nu i}$ such that} \hspace{30pt} & \\
			\forall \varphi_\mu \in S_{\Nbas}:
			\quad \sum_{\nu\in\Ibas} C^{(n)}_{\nu i} \, a(\varphi_\mu, \varphi_\nu)
			&= \lambda_i^{(n)} \sum_{\nu\in\Ibas} C^{(n)}_{\nu i} \, \braket{\varphi_\mu}{\varphi_\nu}\\
			1 &= 
			\sum_{\nu\in\Ibas} \sum_{\mu\in\Ibas} \cc{\left(C^{(n)}_{\mu i}\right)} \braket{\varphi_\mu}{\varphi_\nu} C^{(n)}_{\nu i}
		\end{aligned}
		\right\}.
	\end{equation}
	Introducing the matrix $\mat{A}^{(n)} \in \C^{\Nbas \times \Nbas}$
	and the vectors $\vec{c}^{(n)}_i \in \C^{\Nbas}$
	with elements
	\begin{align}
		\label{eqn:DiscretisationExpression}
		A_{\mu\nu} &= a(\varphi_\mu, \varphi_\nu) & \big(c^{(n)}_i\big)_\mu &= \braket{\varphi_\mu}{v^{(n)}_i} = C^{(n)}_{\mu i}
	\end{align}
	we can write \eqref{eqn:DiscretisedEigenproblemHalfWay}
	as the matrix eigenvalue problem
	\begin{equation}
		\left\{
			\begin{aligned}
				\text{Search $(\lambda_i^{(n)}, \vec{c}^{(n)}_i) \in \R \times \C^{\Nbas}$}& \text{ such that}\\
				\mat{A} \vec{c}^{(n)}_i &= \lambda_i^{(n)} \vec{c}^{(n)}_i \\
				\norm{\vec{c}^{(n)}_i}_{\C^{\Nbas}} &= 1
		\label{eqn:DiscretisedEigenproblem}
		\end{aligned}
		\right\}.
	\end{equation}
	In this formulation the eigenpairs $(\lambda_i^{(n)}, \vec{c}^{(n)}_i)$
	can be determined by standard diagonalisation schemes
	like the ones we will discuss in section \vref{sec:DiagAlgos} below.
\end{rem}

\begin{rem}[Requirements regarding the basis function type]
	\label{rem:ChoiceBasisFunction}
In light of the numerical approach sketched in
remarks \ref{rem:ApproxBottomDiscrete} and \ref{rem:DiscreteFormulation}
let us summarise the requirements towards the basis functions $\varphi_\mu$
for solving the discretised problem \eqref{eqn:DiscretisedEigenproblem}.
\begin{itemize}
	\item The basis function type should admit
		to construct a dense subspace of $Q(\Op{A})$
		if an infinitely large basis set is chosen,
		since this is needed in order to satisfy \eqref{eqn:CondSubspaces}.
		Some basis function types even admit to span $Q(\Op{A})$ itself.
		We shall call these \newterm{complete}.
	\item It should be numerically feasible to solve \eqref{eqn:DiscretisedEigenproblem}.
		In other words \emph{both} computing $\mat{A}$
		and determining its eigenpairs
		should be viable.
	\item The convergence in corollary \ref{cor:Convergence}
		should be fast and systematic.
		In other words the basis type should
		allow to construct a suitable basis set
		in case certain requirements
		regarding accuracy, computational demands,
		description of properties, \ldots~
		should be met.
		Any prior knowledge about the physical problem
		or the properties of $\Op{A}$ can ideally be incorporated
		in such a basis set choice.
		% With this goes a priori and a posteriori error estimation hand in hand.
\end{itemize}
See chapter \vref{ch:NumSolveHF} for some basis function types,
which are used in quantum chemistry,
in the light of solving the Hartree-Fock problem.
\end{rem}

Before we discuss some basic diagonalisation
algorithms in the next section,
let us conclude our discussion about the discretisation
of eigenvalue problems
with a word of warning about the essential spectrum.

\begin{rem}
Remark \vref{rem:NumericalEssentialSpectrum}
stated that it was difficult
to obtain numerical approximations to the essential spectrum.
The min-max theorem \vref{thm:CourantFischer}
provides some theoretical justification for this.
In corollary \vref{cor:Convergence} we saw,
that all eigenvalues from a Ritz-Galerkin approximation
of $\Op{A}$ tend to $\lambda_i(\Op{A})$
as the subspace size is increased.
Unfortunately this value is equal to $\inf \sigma_\text{ess}(\Op{A})$,
the infimum of the essential spectrum,
as soon as we exhausted the discrete spectrum.
In other words the methods we developed in this section
will only help to find the bottom end of the essential spectrum,
but no further information about it at all.

Another consequence of corrollary \ref{cor:Convergence} is
that only a part of the eigenpairs obtained by
diagonalising the matrix $\mat{A}^{(n)}$ of \vref{eqn:DiscretisedEigenproblem}
can be trusted to carry any meaning regarding
the spectrum of the exact physical operator $\Op{A}$.
This is because the larger eigenvalues $\lambda_i^{(n)}$
of $\mat{A}^{(n)}$ will only provide an \emph{artificial}
discretisation of the essential spectrum:
Their values will all tend to $\inf \sigma_\text{ess}(\Op{A})$
as the basis set is increased.
Since the convergence to the bottom of the essential spectrum
as well as the discrete eigenvalues is always from above,
one sometimes has trouble judging whether
an eigenpair of $\mat{A}^{(n)}$
is a true discrete eigenpair of the operator or already
part of the essential spectrum.
In either case the bottom end of the spectrum of $\mat{A}$
will always carry physical meaning in our cases if the basis
set satisfies remark \vref{rem:ChoiceBasisFunction}.
\end{rem}

\todo[inline,caption={}]{
	Talk about resonance processes and the stabilisation method
	for modelling continuum states.
	\begin{itemize}
		\item \url{https://journals.aps.org/pra/abstract/10.1103/PhysRevA.1.1109}
		\item \url{http://pubs.acs.org/doi/abs/10.1021/j100396a027}
	\end{itemize}
}

\section{Diagonalisation algorithms}
\label{sec:DiagAlgos}

This section discusses the key ideas of a few algorithms for obtaining
approximations to the eigenpairs of a matrix $\mat{A}$.
Whilst the regime of quantum mechanics is a complex-valued Hilbert space,
in this thesis
we will only consider combinations of operators and discretisation bases
$\{\varphi_\mu\}_{\mu \in \Ibas}$,
which have the property that
\[ \forall \mu, \nu \in \Ibas: \quad a(\varphi_\mu, \varphi_\nu) \in \R. \]
As a result all matrices in \eqref{eqn:DiscretisedEigenproblem} will be real and symmetric.
In this section we will therefore only consider eigenproblems of the type
\[ \mat{A} \vec{u}_i = \lambda_i \vec{u}_i \]
where $\mat{A} \in \R^{\Nbas \times \Nbas}$,
$\lambda_i \in \R$ and $\vec{u}_i \in \R^{\Nbas}$.

\subsection{Direct methods}
One approach to solve such eigenproblems are
so-called \newterm{direct diagonalisation methods}.
These methods directly attempt to perform a transformation
\[ \rtp{\mat{O}} \mat{A} \mat{O} = \mat{L} = \diag(\lambda_1, \lambda_2, \ldots, \lambda_{\Nbas}), \]
where $\mat{O} \in \R^{\Nbas\times\Nbas}$ is an orthogonal matrix.
Typically this is performed in steps by inspecting the elements of $\mat{A}$
and gradually building both $\mat{O}$ as well as the matrix-matrix product
$\rtp{\mat{O}} \mat{A} \mat{O}$
using techniques such as Householder reflectors~\cite{Arbenz2010}
or Givens rotation~\cite{Arbenz2010}.
In either case this requires random access into the memory of $\mat{A}$.
This is one of the reasons why
direct methods are typically only suitable for either small matrices,
where $\Nbas$ is at most on the order of $1000$,
or matrices with special structure,
like being tridiagonal or banded.
Important dense diagonalisation methods include
QR factorisation~\cite{Arbenz2010,Saad2011}
as well as Cuppen's divide and conquer algorithm~\cite{Arbenz2010,Saad2011}.
They generally yield \emph{all} eigenvalues of a matrix at once
and little or no
extra work is required to additionally obtain all eigenvectors as well.

\subsection{Iterative diagonalisation methods}
Unlike direct methods, which directly access the matrix elements,
\newterm{iterative diagonalisation methods}
only probe the matrix $\mat{A}$ indirectly,
namely by iteratively gathering more and more information
about the eigenpairs of interest.
The way this is done in practice is to repetitively form
the matrix-vector product
\[ \vec{y} = \mat{A}\vec{x} \]
of the problem matrix $\mat{A}$
with suitably constructed trial vectors $\vec{x}$.
The resulting vector $\vec{y}$ is then used
to improve upon the approximation for the eigenpairs as well as to
build the $\vec{x}$ for the next step.
This implies that random access into $\mat{A}$
is not required for such methods and thus specific storage schemes
or well-crafted algorithms going beyond a typical matrix-vector product
can be employed for forming $\vec{y}$.
The latter aspect is most important for the contraction-based
methods, which will be developed in chapters
\vref{ch:NumSolveHF} and \vref{ch:LazyMatrices} of this thesis.

Iterative methods are typically not ideal for
computing many or all eigenpairs of a matrix $\mat{A}$,
which is in contrast to direct methods.
They do perform, however, much better than direct methods
if only few eigenpairs are desired
and it is well-known \emph{where} in the spectrum they are located.
Examples for cases where iterative methods tend to work well
is if one requires some of the largest eigenvalues of $\mat{A}$
or some of those which are closest to an estimated value $\sigma$.
Some important iterative methods are sketched in the following sections.

\subsection{The power method}
\label{sec:Power}
The simplest iterative approach to obtain a single extremal eigenvalue
from a particular matrix $\mat{A}$ is the power method.
Starting from a random initial vector $\vec{v}^{(0)} \in \R^{\Nbas}$,
the algorithm only consists of applying the matrix $\mat{A}$
repetitively to the current vector, \ie
\begin{equation}
\begin{aligned}
	\vec{v}^{(1)} &= \mat{A} \vec{v}^{(0)}, \\
	\vec{v}^{(2)} &= \mat{A} \vec{v}^{(1)} = \mat{A}^2 \vec{v}^{(0)}, \\
	\vec{v}^{(3)} &= \mat{A} \vec{v}^{(2)} = \mat{A}^3 \vec{v}^{(0)}, \\
	&\vdots \\
	\vec{v}^{(j)} &= \mat{A} \vec{v}^{(j-1)} = \mat{A}^j \vec{v}^{(0)}.
\end{aligned}
	\label{eqn:PowerScheme}
\end{equation}
In each step we may compute an estimate $\theta^{(j)}$ for the eigenvalue by
the expression
\begin{equation}
	\theta^{(j)} = \rho_R\left(\vec{v}^{(j)}\right) \equiv
	\frac{\vec{v}^T \mat{A} \vec{v}}{\vec{v}^T \vec{v}},
	\label{eqn:RayleighQuotient}
\end{equation}
where $\rho_R$ is the \newterm{Rayleigh quotient},
the discretised version of \eqref{eqn:CourantFischer}.
In well-behaved cases this algorithm will find an approximation for
the largest eigenvalue in $\theta^{(i)}$
and an approximation for the corresponding eigenvector as
\[ \frac{\vec{v}^{(i)}}{\norm{\vec{v}^{(i)}}_2}. \]
To understand this,
let us write $\vec{v}^{(0)}$ as an expansion in the exact
eigenvectors \linebreak $\vec{u}_1, \vec{u}_2, \ldots, \vec{u}_{\Nbas}$:
\begin{equation}
	\vec{v}^{(0)} = \sum_{i=1}^{\Nbas} \alpha_i \vec{u}_i = \alpha_{\Nbas} \vec{u}_{\Nbas} + \sum_{i=1}^{\Nbas-1} \alpha_i \vec{u}_i
	\label{eqn:Vexpansion}
\end{equation}
Without loss of generality%
\footnote{The case $\alpha_{\Nbas} = 0$ is handled by the limited precision
floating point arithmetic. After a single application of $\mat{A}$,
this is cured and we are back to the case we consider here.}
we can normalise $\vec{v}^{(0)}$ such that $\alpha_{\Nbas} = 1$.
Keeping this in mind, the application of $\mat{A}$ to \eqref{eqn:Vexpansion}
results in
\[
	\mat{A} \vec{v}^{(0)} = \lambda_{\Nbas} \left( \vec{u}_{\Nbas} + \sum_{i=1}^{\Nbas}
	\frac{\lambda_k}{\lambda_{\Nbas}}
	\alpha_i \vec{u}_i \right).
\]
After the $j$-th step and subsequent normalisation we hence get
\[
	\frac{\vec{v}^{(j)}}{\norm{\vec{v}^{(j)}}_2} =
	\vec{u}_{\Nbas} + \bigO\left(
	\left( \frac{\lambda_{\Nbas-1}}{\lambda_{\Nbas}} \right)^j
	\right).
\]
Provided that $\abs{\lambda_{\Nbas-1}} \neq \abs{\lambda_{\Nbas}}$,
\ie that the largest eigenvalue (by magnitude) is single,
the iterate $\vec{v}^{(j)}$
therefore converges linearly against the eigenvector corresponding to this
largest eigenvalue $\lambda_{\Nbas}$.
Similarly $\theta^{(j)}$ converges against $\lambda_{\Nbas}$ in this case.

% TODO OPTIONAL
% Gerschgorin's circles

\subsection{Spectral transformations}
\label{sec:ShiftInvert}
With the power method at hand to obtain the largest eigenvalue,
the question is now,
how one could generalise this approach
for getting the lowest eigenvalue or even one directly from the middle
of the spectrum.
This is the purpose of so-called \newterm{spectral transformations}.

\begin{prop}
	\label{prop:Spectral}
	Given a symmetric matrix $\mat{A} \in \R^{\Nbas \times \Nbas}$,
	the following holds for each eigenpair
	$(\lambda_i, \vec{u}_i) \in \R \times \R^{\Nbas}$:
	\begin{enumerate}[label=(\alph*)]
		\item If $\mat{A}$ is invertible,
			$\vec{u}_i$ is an eigenvector of $\mat{A}^{-1}$ with eigenvalue
			$1 / \lambda_i$.
		\item For every $\sigma \in \R$, $\vec{u}_i$ is an eigenvector
			of the matrix $\mat{A} - \sigma \mat{I}_{\Nbas}$
			with eigenvalue $\lambda_i - \sigma$.
		\item If $\sigma \in \R$ is chosen such that
			$\mat{A} - \sigma \mat{I}_{\Nbas}$
			is invertible, then  $\vec{u}_i$ is an eigenvector
			of $\left( \mat{A} - \sigma \mat{I}_{\Nbas} \right)^{-1}$
			with eigenvalue $1 / (\lambda_i - \sigma)$.
	\end{enumerate}
	\begin{proof}
		All can be shown in a single line:
		\begin{enumerate}[label=(\alph*)]
			\item By definition $\mat{I}_{\Nbas} = \mat{A}^{-1} \mat{A}$
				and thus we have
				\[ \frac{1}{\lambda_i} \vec{u}_i
					= \frac{1}{\lambda_i} \mat{I}_{\Nbas} \vec{u}_i
					= \frac{1}{\lambda_i} \mat{A}^{-1} \mat{A} \vec{u}_i
					= \frac{1}{\lambda_i} \mat{A}^{-1} \lambda_i \vec{u}_i
					= \mat{A}^{-1} \vec{u}_i.
				\]
			\item Direct calculation shows
				\[ \left( \mat{A} - \sigma \mat{I}_{\Nbas} \right) \vec{u}_i
					= \mat{A} \vec{u}_i - \sigma \vec{u}_i
					= \lambda_i \vec{u}_i - \sigma \vec{u}_i
					= \left( \lambda_i - \sigma  \right) \vec{u}_i.
				\]
			\item Follows from (a) and (b).
		\end{enumerate}
	\end{proof}
\end{prop}

Proposition \ref{prop:Spectral} provides us with a toolbox for
changing the spectrum of a matrix in a desired way without changing its eigenvectors.
For example if we are interested in obtaining
the lowest eigenvalue of a matrix $\mat{A}$ using the power method,
we essentially only need to apply the scheme \eqref{eqn:PowerScheme}
to the inverse%
\footnote{Usually the inverse is computed iteratively as well, see discussion
	in section \ref{sec:GeneralisedEigenvalueProblem}.}
$\mat{A}^{-1}$ instead of $\mat{A}$.
Since the largest eigenvector of $\mat{A}^{-1}$ will be the smallest of $\mat{A}$,
this yields the required result.
Similarly, by proposition \ref{prop:Spectral}(c),
we can tune the power method into a particular eigenvalue of interest
by guessing an appropriate shift $\sigma$.
Such spectral transformations are not restricted to the Power method,
since the equivalent effect can be achieved for other iterative
methods by passing them an appropriate matrix.

% TODO OPTIONAL
% Wilkinson's method

\subsection{Krylov subspace methods}
\label{sec:Arnoldi}
Applying the power method
effectively amounts to generating a sequence of vectors
\begin{equation}
	\vec{v}, \, \mat{A} \vec{v}, \, \mat{A}^2 \vec{v}, \, \ldots,
	\label{eqn:PowerSequence}
\end{equation}
starting from an initial guess $\vec{v}$.
Given that the eigenvalue of largest magnitude of $\mat{A}$ is not degenerate,
the above sequence will approach the eigenvector corresponding to this
extremal eigenvalue~(see discussion in section \ref{sec:Power}).
In each iteration the power method does, however,
only keep one of the vectors in \eqref{eqn:PowerSequence}
and throws away all information encoded in the history of the iteration.
An alternative approach which avoids doing so, is
to explicitly keep all vectors in \eqref{eqn:PowerSequence}.
This leads to the construction of a Krylov subspace~\cite{Arbenz2010}
\begin{equation}
	\mathcal{K}_j = \left\{\vec{v},\, \mat{A} \vec{v},\, \mat{A}^2 \vec{v},
	\ldots, \mat{A}^j \vec{v} \right\}.
	\label{eqn:KrylovSubspace}
\end{equation}
A large number of iterative methods both for solving eigenproblems
as well as linear problems
can be boiled down to an iterative construction of such a Krylov subspace.
Once or while it is found the original problem matrix $\mat{A}$
is projected onto this subspace, yielding $\tilde{\mat{A}} \in \R^{j \times j}$.

A key step to exploit the notion of Krylov subspaces
is the construction of an orthogonal basis for $\mathcal{K}_j$.
The Arnoldi algorithm~\cite{Arnoldi1951}
was devised to achieve this in a very efficient manner.
It exploits the fact that each vector \eqref{eqn:KrylovSubspace}
is related to its predecessor by an application of the problem matrix $\mat{A}$
to produce a simple recursion scheme minimising the work needed in each step.
Alongside with the construction of the basis, the Arnoldi algorithm
\emph{at the same time} constructs $\tilde{\mat{A}}$,
the projection of $\mat{A}$ into the Krylov subspace.
Since $\tilde{\mat{A}}$ is both smaller than $\mat{A}$
and has a much simpler form%
\footnote{It is a so-called upper Hessenberg matrix,
	\ie only the upper triangle and a single subdiagonal in the lower
	triangle are non-zero.}
it can be diagonalised by a shifted QR factorisation,
a direct method.
This leads to the Arnoldi method for diagonalising
non-symmetric real matrices,
where one first uses the Arnoldi procedure to construct a sufficiently
good Krylov subspace%
\footnote{Some error estimates exist to judge this without performing
the next step of actually diagonalising the upper Hessenberg matrix.},
followed by a dense diagonalisation of the subspace matrix
to yield estimates for the eigenpairs.

A modification of the Arnoldi method for symmetric matrices $\mat{A}$
is the Lanczos method~\cite{Lanczos1950},
which implicitly exploits the fact
that the subspace matrix has to be tridiagonal%
\footnote{Since $\mat{A}$ is symmetric, so is the subspace matrix
and a symmetric upper Hessenberg matrix is tridiagonal.}
already while constructing the Krylov subspace basis.

Even though the basic idea of Arnoldi and Lanczos are comparatively easy,
the implementation is still involved
due to a range of subtleties.
For example one can show~\cite{Arbenz2010} that the unmodified
Lanczos procedure
leads to an Arnoldi basis of poor numerical quality
with potentially linearly dependent vectors
roughly speaking
exactly when achieving convergence for an eigenpair.
Similarly both Arnoldi and Lanczos tend to have difficulties
when reporting multiplicities.
So if $\mat{A}$ has a triply degenerate eigenvalue $\lambda_i$
it can happen that these algorithms only find it twice,
even though the eigenspaces for $\lambda_{i-1}$
and $\lambda_{i+1}$,
\ie of the next smallest and next largest eigenvalue,
are completely described.
For such issues a large range of remedies have been proposed
over the years~\cite{Arbenz2010,Saad2011},
stressing the importance of Arnoldi methods in numerical linear algebra.
Examples include block modifications ---
where not a single vector, but a collection of vectors is iterated
in the Arnoldi procedure ---
or concepts such as implicit restart, deflation or locking.

\subsection{The Jacobi-Davidson algorithm}
\label{sec:Davidson}
Related to the Krylov subspace methods
sketched above, the Jacobi-Davidson approach
finds approximations to the eigenpairs of \eqref{eqn:DiscretisedEigenproblem}
by constructing suitable small subspaces
and solving the projected problem with dense methods.
The algorithm used for constructing the subspace is, however,
somewhat different%
\footnote{One should mention that similarities to the Lanczos procedure
	can be found, however. See \cite{Arbenz2010} for details.}.
Let us sketch the procedure for a matrix $\mat{A} \in \R^{\Nbas \times \Nbas}$,
where an approximation to the unknown, exact eigenpair $(\lambda_i, \vec{u}_i)$
is desired.
Following \citet{Davidson1975} we define the residual
\begin{equation}
	\vec{r}^{(j} = \mat{A} \vec{v}^{(j)} - \lambda_i \vec{v}^{(j)}
	\label{eqn:DavidsonResidual}
\end{equation}
of our current approximation $\vec{v}^{(j)}$ to the eigenvector $\vec{u}_i$.
In order to correct, we employ the
\newterm{Jacobi orthogonal component correction},
\ie we want to add a vector $\vec{t}^{(j)} \perp \vec{v}^{(j)}$ to our
subspace, such that
\[
	\mat{A} \left(\vec{v}^{(j)} + \vec{t}^{(j)}\right) = \lambda_i \left(\vec{v}^{(j)} + \vec{t}^{(j)}\right).
\]
In other words, we attempt to find the vector missing from the subspace,
such that it is able to span the exact solution,
which implies that it would be able to find it the next time
we solve the projected problem in the subspace.

Since $\lambda_i$ is in general not known at the $j$-th step of the algorithm,
$\vec{t}^{(j)}$ cannot be found exactly in practice.
Instead one employs the value returned by the Rayleigh quotient
\eqref{eqn:RayleighQuotient}
instead of $\lambda_i$ to make progress.
Incorporating the condition $\vec{t}^{(j)} \perp \vec{v}^{(j)}$ leads to the
\newterm{correction equation}
\begin{equation}
	\left(\mat{I}_{\Nbas} - \vec{v}^{(j)} \vec{v}^{(j)\,\ast} \right)
	\left(\mat{A} \vec{v}^{(j)} - \theta^{(j)} \mat{I}_{\Nbas}\right)
	\left(\mat{I}_{\Nbas} - \vec{v}^{(j)} \vec{v}^{(j)\,\ast}\right)
	\vec{t}^{(j)}
	= - \vec{r}^{(j)}.
	\label{eqn:JacobiDavidsonCorrectionEquation}
\end{equation}
Since a vector $\vec{t}^{(j)}$ is required
in each iteration, it needs to be solved many times.
Fortunately, it does, however, not need to be solved exactly.
In practice, one therefore
employs preconditioning techniques~\cite{Saad2003,Saad2011,Arbenz2010,Grossmann1992}
to speed up the performance of the iterative procedures
needed to solve \eqref{eqn:JacobiDavidsonCorrectionEquation}.
An alternative is to avoid using the \emph{exact} matrix $\mat{A}$
in favour of an approximation,
which makes solving \eqref{eqn:JacobiDavidsonCorrectionEquation} easier.
A combination of both is possible as well.

In the original paper \citet{Davidson1975}
assumed $\mat{A}$ to be diagonal-dominant
and thus only used the diagonal
\[ \mat{D}_A = \diag\left( A_{11} A_{22} \ldots A_{\Nbas,\Nbas} \right) \]
instead of the full $\mat{A}$
for the correction in \eqref{eqn:JacobiDavidsonCorrectionEquation}.
This leads to the identification
\[
	\vec{t}^{(j)} = \left( \mat{D}_A - \theta^{(j)} \mat{I}_{\Nbas} \right)^{-1} \vec{r}^{(j)},
\]
which is trivially computed elementwise as
\[
	\left(t^{(j)}\right)_i = \frac{\left(r^{(j)}\right)_i}{A_{ii} - \theta^{(j)}}.
\]
This is the basis of many diagonalisation routines
employed in quantum-chemistry packages nowadays.

\subsection{Generalised eigenvalue problems}
\label{sec:GeneralisedEigenvalueProblem}
Many eigenproblems occurring in quantum chemistry are in fact
not of the form \eqref{eqn:DiscretisedEigenproblem},
but are so-called \newterm{generalised eigenproblems},
\begin{equation}
	\mat{A} \vec{u}_i = \lambda_i \mat{S} \vec{u}_i
	\label{eqn:GeneralisedEigenproblem}
\end{equation}
where the right-hand side contains a
real, positive-definite matrix $\mat{S} \in \R^{\Nbas\times\Nbas}$ as well.
These typically arise because the
basis set $\{\varphi_\mu\}_{\mu \in \Ibas}$
used for the discretisation is not orthogonal.
For the typical basis sets employed to numerically solve the Hartree-Fock
problem,
one of the central aspects of this thesis,
this is the usual case~(see section \vref{sec:BasisTypes}).

One way to deal with \eqref{eqn:GeneralisedEigenproblem} is to reduce
it to a normal eigenproblem by formally inverting $\mat{S}$
and multiplying from the right-hand side.
This leads to
\[
	\left( \mat{S}^{-1} \mat{A} \right) \vec{u}_i = \lambda_i \vec{u}_i,
\]
a normal eigenproblem with the problem matrix $\mat{S}^{-1} \mat{A}$.
In iterative methods this amounts to replacing all occurrences
of the matrix-vector product $\mat{A}\vec{x}$ by the expression
\[ \vec{y} = \mat{S}^{-1} \mat{A} \vec{x}. \]
In this expression the vector $\vec{y}$ can be computed by solving the linear system
\[ \mat{S} \vec{y} = \mat{A} \vec{x} \]
using an inner preconditioned iterative method.
Whilst this would work,
this approach is hardly ever followed in practice.
The reason is that even for a real symmetric, positive-definite $\mat{S}$
and a real symmetric $\mat{A}$,
the matrix $\mat{S}^{-1} \mat{A}$ might not be symmetric,
which would imply that less advantageous solution algorithms need to be employed.

% TODO OPTIONAL
% 
%Another way is to exploit the fact that $\mat{S}$
%is symmetric and compute the Choleski factorisation~\cite{Arbenz2010} of $\mat{S}$,
%\ie to find lower-triangular matrix $\mat{L}$ which satisfy
%\[ \mat{S} = \mat{L} \rtp{\mat{L}}. \]
%Since $\mat{S}$ is non-singular, so is $\mat{L}$ and
%we can write \eqref{eqn:GeneralisedEigenproblem} as
%\[
%	\left( \mat{L}^{-1} \mat{A} \left(\rtp{\mat{L}}\right)^{-1} \right)
%		\left( \rtp{\mat{L}} \vec{u}_i \right)
%		= \lambda_i \left( \rtp{\mat{L}} \vec{u}_i \right).
%\]
%At first this does not look like a big improvement,
%but since triangular matrices are cheap to invert~\to do{cite}
%
%such that the
%symmetric matrix $\left( \mat{L}^{-1} \mat{A} \left(\rtp{\mat{L}}\right)^{-1} \right)$
%can be diagonalised using a normal diagonalisation algorithm.

An alternative approach to avoid this
is to try to modify the iterative procedures towards supporting the generalised
eigenproblems straight away.
By properly following the derivations, one finds that
appropriate formulations of the algorithms in the setting of generalised
eigenproblems can be achieved by replacing the explicit or implicit
occurrences of the orthonormality condition
\[ \rtp{\vec{u}}_i \vec{u}_j = \delta_{ij} \]
by
\[  \rtp{\vec{u}}_i \vec{u}_j = S_{ij}. \]
In other words, only the way the orthonormalisation of the subspace vectors
is performed as well as some expressions in which the identity matrix
occurs,
like in \eqref{eqn:JacobiDavidsonCorrectionEquation},
need to be changed.

Yet another option is to orthogonalise the basis before performing
the discretisation and thus avoid the appearance of the generalised
eigenproblem all together.


% TODO OPTIONAL
%\subsection{Other algorithms and further approaches}
%See \cite{Saad2011,Arbenz2010}
%
%Some generalisations based on forming polynomials over the matrix $\mat{A}$
%exist as well.
%For example Chebychev polynomial filtering works similar to this
%\to doil{Check}


% https://www.math.tu-berlin.de/fachgebiete_ag_modnumdiff/fg_modellierung_simulation_und_optimierung_in_natur_und_ingenieurswissenschaften/v_menue/mitarbeiter/prof_dr_reinhold_schneider/publikationen/



