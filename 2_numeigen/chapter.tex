\chapter{Numerical treatment of eigenvalue problems}

\chapquote{
	It is a well-known experience that the only truly
	enjoyable and profitable way of studying mathematics
	is the method of ``filling in details'' by one's
	own efforts.
}{Cornelius Lanczos~(1893--1974)}


\section{Ritz-Galerkin projection}
% min - max principle
% 

\section{Numerical solution of partial differential equations}
We can relax $H^2(\R^3)$ to $H^1(\R^3)$ if we use test functions
from $H^1(\R^3)$ as well in Ritz-Galerkin scheme.
\todo{Ask guido}
Is this the energetic extension
\url{https://en.wikipedia.org/wiki/Energetic_space}

\todo[inline,caption={}]{
	\begin{itemize}
		\item Define the problem
		\item Galerkin projection
	\end{itemize}
}



\section{Diagonalisation algorithms}
\todo[inline]{
	\begin{itemize}
		\item Brief rationale
		\item Basic ideas
		\item Sketch Davidson and Lanczos
	\end{itemize}
}
