\chapter{molsturm: Flexible and modular quantum-chemistry package}

\section{The need for modular quantum chemistry software}
\todo[inline,caption={}]{
	\begin{itemize}
		\item pyscf \cite{Sun2017}
	\end{itemize}
}

\section{Design of molsturm}
\begin{figure}
	\centering
	\includeimage{8_molsturm/molsturm_structure}
	\caption[Structure of the \molsturm package]{%
	Structure of the \molsturm package: Shown are the five modules of the package,
	along with the set of integrals accessible from \gint and the set of post-HF method,
	which can be used from \molsturm. The greyed-out parts are not yet implemented.
	Only the modules inside the red box are part of \molsturm. The blue boxes are all external packages.}
	\label{fig:structureMolsturm}
\end{figure}



\todo[inline,caption={}]{
	\begin{itemize}
		\item molsturm paper
		\item gint: Integrals are lazy matrices
		\item gscf: Independent algorithms (refer back to chapter about numerical HF)
		\item Interfacing with other packages
	\end{itemize}
}

\section{Example scripts and use cases}
\todo[inline,caption={}]{
	\begin{itemize}
		\item molsturm paper examples
	\end{itemize}
}
