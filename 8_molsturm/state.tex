\section{Current state of \molsturm}
\label{sec:MolsturmState}

After about two years of development,
\molsturm has become a light-weight quantum-chemistry package of about 45000
lines of \cpp and \python code%
\footnote{The number includes the code from the dependencies
\gint, \gscf, \lazyten, \krims as well as
all examples and tests, but excludes comment lines and blanks.},
which tries to explicitly keep the requirements of more than
one basis function in mind.
The key ingredient to reach the necessary flexibility
is a \contraction-based self-consistent field scheme,
where the details of the Fock matrix \contraction expression
are varied according to the numerical properties of the basis function type
by our integral back end library \gint.
In contrast the code describing the \SCF algorithm
is well-separated from the details of the \contraction expression
using the high-level lazy matrix language of \lazyten.
As such it becomes independent from the type of basis function used for the discretisation.
Even if changes to the \SCF scheme or some back end
library are needed in the future our modular design
will assure that the other parts of the \molsturm package will stay unaffected.
Once the \SCF orbitals have been obtained,
the remainder of a calculation, \eg a Post-\HF method,
can be typically be formulated entirely in the \SCF orbital basis
and without any reference to the underlying basis functions.
Because of this \molsturm can be thought of as a mediator
to produce \SCF results in a very general fashion
on top of which one may stick any Post-\HF method.

In \molsturm the \SCF procedure can be started from either a core Hamiltonian guess,
a completely random guess or any other arbitrary set of initial
coefficients supplied by the user via a \numpy array.
For some cases, \eg if Coulomb-Sturmians are employed,
\molsturm offers means to interpolate a guess
from the converged state of a previous calculation.
During the \SCF \molsturm automatically switches between
the implemented \SCF schemes.
This is necessary since the plain Roothaan
repeated diagonalisation~\cite{Roothaan1951}
as well as the truncated optimal damping algorithm schemes
are cheaper, but typically do not converge as efficiently
as a coefficient-based Pulay DIIS.
Switching the algorithms allows us to balance this.
As demonstrated in the examples in section \ref{sec:MolsturmExamples},
the algorithmic details of the \SCF procedure can be fully controlled
from \python,
such that one often only needs to change a single keyword
in order to switch to a different solver algorithm
or to employ a different basis function type.

Once an \SCF computation has finished the obtained results can be archived
in either in YAML~\cite{YAML} plain text or in HDF5~\cite{HDF5Manual}
binary files.
Such an archive not only contains the full final state of the calculation
but also the precise set parameters which were used to start the \SCF.
A file therefore becomes self-explanatory and reproducible
without any special measures taken from the user.
On top of that an archived calculation can be continued
or analysed at a later point or on a different machine with ease
by just restoring the state.

On top the \SCF a range of interfaces for
performing Post-HF calculations or further analysis are available.
Right now \molsturm only implements second order Møller-Plesset perturbation theory%
~(see section \ref{sec:MP})
and some utility functions for plotting.
Some selected methods from third-party libraries can be easily
invoked on any \SCF result using third party libraries.
For example full configuration interaction is available via \pyscf~\cite{Sun2017}
and the excited states methods
{\ADC}(1), {\ADC}(2), {\ADC}(2)-x~\cite{Schirmer1982}
and {\ADC}(3)~\cite{Trofimov1999}
based on the algebraic diagrammatic construction scheme
are available via \adcman~\cite{Wormit2014}.
Calculations in \molsturm may be performed based on
contracted Gaussians~\cite{Hehre1969} --- using
the integral libraries \libint~\cite{Libint2_231,Libint2}
or \libcint~\cite{Sun2015} --- and based on
Coulomb-Sturmians~\cite{Shull1959,Avery2011} ---
using \sturmint~\cite{sturmintWeb}.
Implementing further types of basis functions
takes nothing more than providing appropriate interface classes in
our integral interface library \gint.
Thereafter such basis functions are available for the full \molsturm ecosystem
including the Post-HF methods
provided by the third-party libraries mentioned above.

In section \ref{sec:MolsturmExamples} the abilities of \molsturm
have been demonstrated by three practical examples.
We put particular emphasis on how
our \python interface integrates with existing third-party \python packages
such that additional functionality can be quickly
combined with \molsturm,
potentially to extend \molsturm in ways we as the authors
would have never thought of.
In the examples it was further shown how to aid repetitive calculations,
implement novel quantum-chemical methods
or rapidly amend functionality in a preliminary way,
where a proper implementation would be much more involved.
We hinted how systematic comparisons with established basis functions
as well as subsequent graphical analysis
is convenient to perform by the means of our
readily scriptable interface.
We hope that in this manner \molsturm
will be a useful package to rapidly try novel basis function types
and get a feeling for their range of applicability.
