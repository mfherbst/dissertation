\section{Current state and future of \molsturm}
\label{sec:state}

After about two years of development
\molsturm offers a \contraction-based
self-consistent field~SCF algorithm for solving the Hartree-Fock equations
in a basis-function independent fashion
as well as interfaces for performing Post-HF calculations or further analysis.
\molsturm's functionality can be fully controlled
from \python as mentioned in the examples in section \ref{sec:examples},
such that it often takes only changing a single keyword
in order to switch to a different solver algorithm or to a different
basis function type.

In this manor the discretisation of the Hartree-Fock problem
can be realised in all basis function types implemented in \gint~\cite{gintWeb}.
Right now this is Coulomb-Sturmians as well as contracted Gaussians,
both in multiple implementations for computing the required integrals.
For contracted Gaussians one may employ either the third-party
\libint~\cite{Libint2,Libint2_231} or \libcint~\cite{Sun2015} libraries,
whereas Coulomb-Sturmians are available via multiple back ends
in our own \sturmint~\cite{sturmintWeb} library.
In the future further basis function types
as well as backend implementations
may be added easily to \molsturm
by extending \gint with the appropriate interfaces.

The SCF procedure can be started from either a core Hamiltonian guess,
a completely random guess or any other arbitrary set of initial
coefficients supplied by the user via a \numpy array.
For some cases, \eg if Coulomb-Sturmians are employed,
\molsturm offers means to interpolate a guess from a previous result
in a different basis.
During the SCF \molsturm automatically switches between
our implemented SCF schemes.
This is necessary since the plain Roothaan
repeated diagonalisation~\cite{Roothaan1951}
as well as the truncated optimal damping algorithm~(tODA) schemes
are cheaper, but not as effective as a coefficient-based Pulay DIIS.
Switching the algorithms allows us to balance this.
A good improvement for \molsturm on this front would be a cheap,
but nevertheless effective \contraction-based SCF scheme.

Once an SCF computation has finished the obtained results can be archived
in either in YAML~\cite{Ben-Kiki2009} plain text or in HDF5~\cite{HDF5Manual}
binary files.
Such an archive not only contains the full final state of the calculation
but also the precise set parameters which were used to start the SCF.
A file therefore becomes self-explanatory and reproducible
without any special measures taken from the user.
On top of that an archived calculation can be continued
or analysed at a later point or on a different machine with ease
by just restoring the state.

For treating electron correlation,
\molsturm currently only implements
second order Møller-Plesset perturbation theory~(MP2),
but further methods can be easily called
via interfaces to third-party libraries.
In this manor full configuration interaction~(FCI) is available via \pyscf
and a range of excited states methods
based on the algebraic diagrammatic construction~(ADC) scheme
via \adcman~\cite{Wormit2014}, namely
ADC(1), ADC(2), ADC(2)-x~\cite{Schirmer1982}
and ADC(3)~\cite{Trofimov1999}.
As was demonstrated in section \ref{sec:ex:ccd} and \ref{sec:ex:geo},
the extension of \molsturm to other methods or packages
is easily accomplished by \molsturm's
\python interface if required.

A disadvantage of said interface is, however,
that all data is stored as plain and dense memory.
Both symmetry as well as sparsity properties of the involved tensors
are entirely ignored at the moment.
This yields an unacceptably large requirement in memory
as well as a very unfavourable computational scaling
for the demanding tensor contractions
typically required in Post-HF methods like CCD (see \fig \ref{fig:codeCCD}).
In the future we want to tackle this by extending
\lazyten towards lazy tensors as well.
In practice this would imply that the evaluation of
tensor contractions like in \fig \ref{fig:codeCCD} could be implicitly delayed
for as long as possible.
Only once the results are needed in the code,
the calculation would be underdone
for all delayed expressions at once.
This could allow to automatically deduce an ideal evaluation strategy
taking all sparsity and symmetry information
of the involved tensors into account.
Ideally this would happen transparently in the background
without changing the \python interface at all.

Another angle for future work would be to integrate more closely
with existing quantum-chemistry libraries or program packages.
A good example is the \texttt{libxc}~\cite{Lehtola2018} library,
which offers a range of exchange-correlation functionals.
Implementing an interface to this library inside appropriate
lazy matrix objects would allow to construct the Kohn-Sham matrix
inside \molsturm's \contraction-based SCF as well,
such that density-functional theory~(DFT) calculations
could be performed.
Similarly we would like to simplify
data exchange with other quantum chemistry packages like \pyscf or \psifour
by supporting some wide-spread file formats in quantum chemistry,
\eg \texttt{MolPro}'s FCIDUMP~\cite{Knowles1989}.
For facilitating plotting of SCF orbitals or densities
we plan to implement the \texttt{molden}~\cite{Schaftenaar2000} format
as well as routines for dumping the data
in \texttt{Gaussian}~\cite{Frisch2016} cube or VTK~\cite{Avila2010} files.

%%% Local Variables:
%%% mode: latex
%%% TeX-master: "paper"
%%% End:
