\section{Current state and future of the program}
\label{sec:state}

\todoil{Maybe refer to the figures in the design section?}

\molsturm currently allows to perform Hartree-Fock calculations
on small molecules using GTO basis functions or using Coulomb-Sturmians.
For the former type of basis functions \gint offers interfaces
to the backend libraries \libint and \libcint and for the latter
type various implementations inside \sturmint are available.
During the SCF procedure \molsturm automatically switches between different
SCF algorithms, currently tODA and a coefficient-based Pulay DIIS,
in order to achieve a good balance between convergence and cost for each iteration.
The default guess are the orbitals resulting from a diagonalisation of the core Hamiltonian,
but the user is free to supply any arbitrary guess from the \python interface.
This includes orbitals from a previous HF calculation or even a totally random guess.
By default \molsturm tries to choose an ARPACK-based eigensolver algorithm
in case the structure of the problem matrix and the other Hartree-Fock
parameters allow this.
What results form this is a fully \contraction-based SCF procedure,
where the Fock matrix is never built in memory at all.
Again the default behaviour can be influenced by the means of suitable
parameters.

After the SCF procedure the quantities like the core Hamiltonian,
the Fock matrix and the electron repulsion integrals
are available via a convenient syntax for post-processing in \python.
This has been demonstrated for example in section \ref{sec:ex:ccd}.
These result objects are also used to realise a simple  MP2 implementation
within \molsturm and to interface to third-party libraries:
Currently MP3 as well as ADC(1), ADC(2)-s, ADC(2)-x and ADC(3)
are available via the \adcman package
and Full-CI can be performed using \pyscf.

Furthermore these HF results can be archived in HDF5 format
in order to restore the full state of the finalised calculation at a later point.
Especially for larger cases where the computations take considerable time,
this allows to perform the SCF on a cluster in advance and
transfer the archived result to a local machine for analysis.
Since the archive contains the full state
including the orbital coefficients and the repulsion integrals,
the local analysis can make use of these quantities
when extracting potential energy surfaces or plotting MOs.
This can be done conveniently in an iteractive \python shell
or a Jupyter notebook for instant feedback.

\molsturm aids with the task of analysing the SCF results
by providing a few basic utility functions.
For example the function values of the SCF orbitals can be exported on an
arbitrary shaped grid to a \numpy array.
This array can than be further manipulated or
plotted in third party libraries or programs.

% What does the molsturm ecosystem consist of right now
%    - python interface to run HF calculations
%        - SCF switches between different algorithms automatically
%    - keywords to select integral backend and control parameters for calculation
%        - sturmint and libint, libcint (multiple backends)
%        - selection of the guess (random, previous, hcore)
%        - selection of the eigensolver
%    - archival of intermediate results and state in HDF5 format
%        - allows to perform big calculation on cluster
%        - restore archive later in a local session
%        - analyse results interactively in a shell or a browser
%    - interactive investigation of results:
%        - summary of SCF
%        - easy access to all relevant quantities (including fock, hcore and eri)
%        - obtain MO data or density on grid for plotting or to 
%    - interface to posthf methods
%        - adcman
%        - pyscf
%    - LA backend:
%        - Bohrium
%        - LAPACK / armadillo
% What is planned for the future
%    - DFT using libxc (gscf has the flexibility)
%    - More analysis tools (direct export of data to standardised output formats like VTK)
%    - Plot wavefunction directly to VTK
%    - openbabel to parse third-party input formats
%    - other standards to interface to further external methods
%    - SCF switching criteria
%    - lazyten
%    - More clever structures for exported quantities

Even though the \python interface of \molsturm allows to control many
aspects of the computation,
there are still a couple of directions for possible improvement.
Most importantly the underlying storage format used
for the Fock matrix or the repulsion integrals are plain and flat \numpy arrays.
In other words the inherent symmetry properties due to spin or index permutations
are entirely unused
giving rise to a much larger memory requirement than theoretically possible.
We currently pay this price in order to make use of the simple and intuitive
syntax employed by \numpy.
For supporting larger problems it is,
however, certainly necessary to replace this by a data structure,
which implicitly considers these symmetries.

One way to do this would be to extend the capabilities of \lazyten.
Right now it only allows to use lazy matrices,
i.e. structures supporting matrix-matrix and matrix-vector
multiplication.
In order to transparently make use of the symmetry and spin structure
of the repulsion integral tensor for example,
we would require lazy tensors,
which support arbitrary \contraction calls.
If we choose the interfaces of such lazy tensors to reflect
the interface of \numpy,
one could get the best of both worlds.
The bohrium project,
which is one of the available computational backends inside \lazyten already,
uses a vary similar approach in their interface,
which we aim at incorporating into our project, too.

Apart from \lazyten there are a couple of other directions,
where molsturm could be easily extended.
This includes making use of the generality of the SCF procedure
in order to support further types of basis functions.
Next steps could be to include basis functions based on
a finite-difference approximated radial function or
further Sturmian basis functions like molecular Sturmians.

Similarly solving the Kohn-Sham equations or the Hartree-Fock
is conceptionally very simple,
since one plainly substitutes the HF exchange term by an appropriate
term representing the exchange-correlation functional.
In the language of lazy matrices one therefore just needs to implement
one extra lazy matrix for representing exchange-correlation functionals
in order to be able to perform density-functional theory
calculations with \molsturm in a basis-function independent way.

Another angle for future work would be to integrate more
closely with the existing \emph{de facto} standards of quantum-chemical software.
On the one hand we plan to use \texttt{openbabel} in order to
support setting up calculations in \molsturm using a range of already existing
input formats.
Similarly we want to integrate routines to export
the orbital coefficients or the denisity to cube or VTK files for plotting
or to export the HF results in MolPro's FCIDUMP or the molden format
for interfacing with other third-party programs.
