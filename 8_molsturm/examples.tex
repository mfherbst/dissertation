\section{Example scripts and use cases}
\label{sec:examples}


\subsection{Fitting a dissociation curve}
\label{sec:ex:data}
\begin{figure}
	\lstinputlisting[firstline=9,firstnumber=9]{examples/dissociation/h2_dissociation.py}
	\caption{A script for plotting the \ce{H2} dissociation curve. The import statements are skipped.}
	\label{fig:codeDissociation}
\end{figure}

\begin{figure}
	\centering
	\includeimage{8_molsturm/h2_dissociation}
	\caption{Result of script, dissociation fit for \ce{H2} and morse potential}
	\label{fig:dissociation}
\end{figure}

The code in fig. \ref{fig:codeDissociation} shows an example python script,
which computes the MP2 energy of the hydrogen molecule at various hydrogen-hydrogen bond distances
and fits a Morse potential through the resulting data.
In the function \code!compute_curve! the code uses a standard python
\code!for! loop to generate a sequence of \ce{H2} molecules with varying bond lengths.
For each of these first a Hartree-Fock calculation is run, followed by MP2.
The result of the previous calculation is used as a guess for the next to speed things up.
The resulting MP2 ground state energy is saved in a numpy array for the later fitting
and returned together with the distance vector.
The function \code!plot_morse_fit! on the other hand
takes the computed data, fits a Morse potential through it and plots
both the original data and the fit.
The final part of the script just calls both functions and shows
the resulting plot to the user.

Note, how compared to the traditional ASCII-in - ASCII-out approach taken
by most quantum chemistry program, the high-level \python interface of \molsturm
allows to write a single readable script where everything happens.
There is no need to write shell scripts which first extract the relevant information
from one plain-text output file of a quantum chemistry program just to transfer the
data to another program for plotting.
This firstly simplifies it for people who are new to the field to get started
with doing calculations,
because they only need to familiarise themselves with a single framework
before being able to amend or modify existing procedures.
Secondly this reduces the chance that existing scripts fail
or silently produce the wrong results,
when the output or input format of one of the employed programs
changes subtly in the future.
Last but not least the present script serves automatically as documentation
of the steps which have been followed to reach a particular result
up to the point that others can automatically reproduce it by just re-executing
the script as well.

Furthermore the high-level access makes the scripts themselves very flexible.
Just like the previous computation was used in \code!compute_curve!
as a guess for the next one,
one could supply an arbitrary guess for the SCF coefficients to \molsturm as
a \texttt{numpy} array.
Furthermore computational details like the eigensolver or the integral backend,
which is used in the SCF can be changed as simple parameters.
This is shown here with the \texttt{backend} parameter,
which effects which backend is used to compute the basis functions.
Switching from \libint to \libcint, just requires the substitution of the
string in line 50.

Similarly the \texttt{state} and \texttt{mp2} structures returned by the
Hartree-Fock and MP2 calculation functions have an interface to return
many of the final results of the relative procedures.
For example the  fock matrix or individual terms of the fock matrix as well as
the repulsion integrals in the orbital basis can be obtained rather easily
for \texttt{state} as will be shown in the next example.


\subsection{Coupled-cluster doubles~(CCD)}
\label{sec:ex:ccd}
\todoil{I feel we should rearrange, such that the performance drawbacks get mentioned earlier and the advantages are at the end}

% TODO Show the equations and the python code side-by-side
%      (like in the presentation I gave at the HGS school)

Another case which shows how the interplay of the high-level interface
of \molsturm and the \numpy machinery is the coupled-cluster doubles~(CCD)
\todo{cite}
implementation presented in \ref{fig:codeCCD}.

Even though \molsturm right now has neither any coupled-cluster
method implemented at thee moment nor does it possess an interface to any third-party
coupled-cluster code,
it took us only roughly 100 lines of code and two days of work
to add this functionality.
For this we used the convenient \numpy function \texttt{einsum} as well
as the datastructures \molsturm provides for accessing the
electron repulsion integrals as well as the Fock matrix in the orbital basis.
The interfaces of both the \texttt{einsum} function as well as the
\texttt{state.eri} and \texttt{state.fock} structures from \molsturm
are designed such that the resulting code has a strong resemblance
with the actual equations one needs to implement.
This not only makes it fast to implement Post-HF methods like CCD
on top of \molsturm,
but it also gives rise to code which is very readable, too.
Futhermore the design of \molsturm, where all the basis set type
or backend-dependent things are handled in \gint implies,
that the code shown in fig. \ref{fig:codeCCD}
is automatically applicable to any backend library which is implemented in \gint.

As such this code is ideal to teach Post-HF methods
or to experiment with new basis types or algorithms.

\begin{figure}
	\lstinputlisting[firstline=6,firstnumber=6,lastline=90]{examples/ccd/ccd.py}
	\caption{CCD code}
	\label{fig:codeCCD}
\end{figure}

\begin{figure}
	\colorlet{CcRes}{dkred}
\colorlet{CcEri}{dkgreen}
\colorlet{CcTOne}{dkblue}
\colorlet{CcTTwo}{mauve}

\begin{align*}
	\textcolor{CcRes}{r_{ij}^{ab}} =
%		&= \langle ab||ij \rangle
%		%
%		+ \sum_e f_{ae} \, t_{ij}^{eb}
%		- \sum_e f_{be} \, t_{ij}^{ea}
%		- \sum_m f_{mi} \, t_{mj}^{ab}
%		+ \sum_m f_{mj} \, t_{mi}^{ab} \\
%		%
%		&+ \frac12 \sum_{mn} \langle mn||ij \rangle \, t_{mn}^{ab}
%		+ \frac12 \sum_{ef} \langle ab||ef \rangle \, t_{ij}^{ef} \\
%		%
%		&+ \sum_{me} \langle mb||ej \rangle \, t_{im}^{ae}
%		- \sum_{me} \langle mb||ei \rangle \, t_{jm}^{ae} 
%		- \sum_{me} \langle ma||ej \rangle \, t_{im}^{be}
%		+ \sum_{me} \langle ma||ei \rangle \, t_{jm}^{be} \\
		%
		&- \frac12 \sum_{mnef} \textcolor{CcEri}{\langle mn||ef \rangle} \, \textcolor{CcTOne}{t_{mn}^{af}} \, \textcolor{CcTTwo}{t_{ij}^{eb}}
		+ \frac12  \sum_{mnef} \textcolor{CcEri}{\langle mn||ef \rangle} \, \textcolor{CcTOne}{t_{mn}^{bf}} \, \textcolor{CcTTwo}{t_{ij}^{ea}}
		- \frac12  \sum_{mnef} \textcolor{CcEri}{\langle mn||ef \rangle} \, \textcolor{CcTOne}{t_{in}^{ef}} \, \textcolor{CcTTwo}{t_{mj}^{ab}} \\
		&+ \frac12 \sum_{mnef} \textcolor{CcEri}{\langle mn||ef \rangle} \, \textcolor{CcTOne}{t_{jn}^{ef}} \, \textcolor{CcTTwo}{t_{mi}^{ab}}
		+ \frac14  \sum_{mnef} \textcolor{CcEri}{\langle mn||ef \rangle} \, \textcolor{CcTOne}{t_{mn}^{ab}} \, \textcolor{CcTTwo}{t_{ij}^{ef}}
		+ \frac12  \sum_{mnef} \textcolor{CcEri}{\langle mn||ef \rangle} \, \textcolor{CcTOne}{t_{im}^{ae}} \, \textcolor{CcTTwo}{t_{jn}^{bf}} \\
		&- \frac12 \sum_{mnef} \textcolor{CcEri}{\langle mn||ef \rangle} \, \textcolor{CcTOne}{t_{jm}^{ae}} \, \textcolor{CcTTwo}{t_{in}^{bf}}
		- \frac12  \sum_{mnef} \textcolor{CcEri}{\langle mn||ef \rangle} \, \textcolor{CcTOne}{t_{im}^{be}} \, \textcolor{CcTTwo}{t_{jn}^{af}}
		+ \frac12  \sum_{mnef} \textcolor{CcEri}{\langle mn||ef \rangle} \, \textcolor{CcTOne}{t_{jm}^{be}} \, \textcolor{CcTTwo}{t_{in}^{af}}
\end{align*}

% fock = state.fock
% eri_phys = state.eri.transpose((0, 2, 1, 3))
% eri = eri_phys - eri_phys.transpose((1, 0, 2, 3))
%
% res = np.einsum("abij->iajb", eri.block("vvoo")) \
%     + np.einsum("ae,iejb->iajb", fock.block("vv"), t2) \
%     - np.einsum("be,ieja->iajb", fock.block("vv"), t2) \
%     - np.einsum("mi,majb->iajb", fock.block("oo"), t2) \
%     + np.einsum("mj,maib->iajb", fock.block("oo"), t2)
%
% res += \
%     + 0.5 * np.einsum("mnij,manb->iajb", eri.block("oooo"), t2) \
%     + 0.5 * np.einsum("abef,iejf->iajb", eri.block("vvvv"), t2) \
%     + np.einsum("mbej,iame->iajb", eri.block("ovvo"), t2) \
%     - np.einsum("mbei,jame->iajb", eri.block("ovvo"), t2) \
%     - np.einsum("maej,ibme->iajb", eri.block("ovvo"), t2) \
%     + np.einsum("maei,jbme->iajb", eri.block("ovvo"), t2)
%
% res += \
%     - 0.5  * np.einsum("mnef,manf,iejb->iajb", eri.block("oovv"), t2, t2) \
%     + 0.5  * np.einsum("mnef,mbnf,ieja->iajb", eri.block("oovv"), t2, t2) \
%     - 0.5  * np.einsum("mnef,ienf,majb->iajb", eri.block("oovv"), t2, t2) \
%     + 0.5  * np.einsum("mnef,jenf,maib->iajb", eri.block("oovv"), t2, t2) \
%     + 0.25 * np.einsum("mnef,manb,iejf->iajb", eri.block("oovv"), t2, t2) \
%     + 0.5  * np.einsum("mnef,iame,jbnf->iajb", eri.block("oovv"), t2, t2) \
%     - 0.5  * np.einsum("mnef,jame,ibnf->iajb", eri.block("oovv"), t2, t2) \
%     - 0.5  * np.einsum("mnef,ibme,janf->iajb", eri.block("oovv"), t2, t2) \
%     + 0.5  * np.einsum("mnef,jbme,ianf->iajb", eri.block("oovv"), t2, t2)

\newcommand{\eri}{\textcolor{CcEri}{eri.block("oovv")}}
\newcommand{\eriIdx}{\textcolor{CcEri}{mnef}}
\newcommand{\tampOne}{\textcolor{CcTOne}{t2}}
\newcommand{\tampTwo}{\textcolor{CcTTwo}{t2}}
\newcommand{\resIdx}{\textcolor{CcRes}{iajb}}

\begin{BVerbatim}[commandchars=\\\{\},fontsize={\smaller}]
eri_phys = state.eri.transpose((0, 2, 1, 3))
eri = eri_phys - eri_phys.transpose((1, 0, 2, 3))
{\textcolor{CcRes}{res}} = {\textbackslash}
    - 0.5  * np.einsum("{\eriIdx},{\color{CcTOne}manf},{\color{CcTTwo}iejb}->{\resIdx}", {\eri}, {\tampOne}, {\tampTwo}) {\textbackslash}
    + 0.5  * np.einsum("{\eriIdx},{\color{CcTOne}mbnf},{\color{CcTTwo}ieja}->{\resIdx}", {\eri}, {\tampOne}, {\tampTwo}) {\textbackslash}
    - 0.5  * np.einsum("{\eriIdx},{\color{CcTOne}ienf},{\color{CcTTwo}majb}->{\resIdx}", {\eri}, {\tampOne}, {\tampTwo}) {\textbackslash}
    + 0.5  * np.einsum("{\eriIdx},{\color{CcTOne}jenf},{\color{CcTTwo}maib}->{\resIdx}", {\eri}, {\tampOne}, {\tampTwo}) {\textbackslash}
    + 0.25 * np.einsum("{\eriIdx},{\color{CcTOne}manb},{\color{CcTTwo}iejf}->{\resIdx}", {\eri}, {\tampOne}, {\tampTwo}) {\textbackslash}
    + 0.5  * np.einsum("{\eriIdx},{\color{CcTOne}iame},{\color{CcTTwo}jbnf}->{\resIdx}", {\eri}, {\tampOne}, {\tampTwo}) {\textbackslash}
    - 0.5  * np.einsum("{\eriIdx},{\color{CcTOne}jame},{\color{CcTTwo}ibnf}->{\resIdx}", {\eri}, {\tampOne}, {\tampTwo}) {\textbackslash}
    - 0.5  * np.einsum("{\eriIdx},{\color{CcTOne}ibme},{\color{CcTTwo}janf}->{\resIdx}", {\eri}, {\tampOne}, {\tampTwo}) {\textbackslash}
    + 0.5  * np.einsum("{\eriIdx},{\color{CcTOne}jbme},{\color{CcTTwo}ianf}->{\resIdx}", {\eri}, {\tampOne}, {\tampTwo})
\end{BVerbatim}

	\caption{CCD comparison code and equation}
	\label{fig:comparisonCCD}
\end{figure}


The \code{ccd} function takes a HF \texttt{state} from \molsturm and transparently
performs a CCD computation on top.
For this it first invokes the generation of the repulsion integral tensor
via a call to \texttt{state.eri} and antisymmetrises it in line 66.
\todoil{This is not what happens right now, but it will be soon}
By an iterative quasi-newton minimisation scheme with an
with an approximate Jacobian~(from \code{ccd_approx_jacobian})
it then minimises the CCD residual implemented in \code{ccd_residual}.

All operations inside \code{ccd_residual} are single-threaded tensor contractions
working on blocks of the full repulsion integrals tensor or the fock matrix.
Furthermore the code makes no use of the index or spin symmetry relationships
inside the tensor expressions.
Nevertheless simple calculations of small molecules with small basis sets are possible.
For example a \ce{O2} 6-31G calculation on a recent laptop took
about an hour to converge up to 1e-4.
In the future we plan to work on the performance drawback by incorporating
knowledge of the spin and symmetry inside the tensor objects
and by the means of automatic parallelisation techniques.

\subsection{Gradient-free geometry optimisation}
\label{sec:ex:geo}
Fig. \ref{fig:codeGeoOpt} shows a \python script,
which performs the geometry optimisation of a water molecule.
Even though molsturm has no interface to obtain gradients
from backend libraries at the moment,
one can still employ the gradient-free optimisation schemes from \texttt{scipy}
to perform simple optimisations.

Similar to the other examples presented here,
this shows that by the combination of features offered by third-party
libraries in \python with \molsturm's built-in high-level interface
one is often to quickly code preliminary implementations
for features that we as the molsturm developers did not yet think of
or which we did not yet implement.

Even though the code by itself is in fact comparatively slow
its simplicity allows students to learn about the methods involved
in quantum chemistry and it allows for rapid development
of new methods as the resulting code is easily accessible.

\begin{figure}
	\lstinputlisting[firstline=7,firstnumber=7]{./examples/geo_opt.py}
	\caption{Gradient-free geometry optimisation using Powell's method}
	% References 103 and 104 in
	% https://docs.scipy.org/doc/scipy-0.14.0/reference/generated/scipy.optimize.minimize.html#r103
	\label{fig:codeGeoOpt}
\end{figure}

Note how the code employs two optimisations in a chained manor.
First the geometry is optimised in a rather small sto-3g basis set
only up to a small error threshold.
Then this is used as the initial guess for a more accurate basis set
and the geometry is converged to a larger accuracy.
On a recent laptop the present code needs about 6 minutes
to converge to the equilibrium geometry shown in fig. \ref{fig:OptimalGeometryWater}.

% For the H2O we could show our plotting capablilties with VTK
% Maybe have a video of the sto-3g optimisation.
% Show at least a density plot of the converged density at the end.

\begin{figure}
	\centering
	\missingfigure{Final \ce{H2O} geometry}
	\caption{Density plot of the final optimised \ce{H2O} geometry with a
	\ce{O-H} bond length of \unit[1.79612]{\AA} and a \ce{H-O-H} bond angle of \unit{104.958}{$^\circ$}}
	\label{fig:OptimalGeometryWater}
\end{figure}
