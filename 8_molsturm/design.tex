\section{Design of molsturm}
\begin{figure}
	\centering
	\includeimage{8_molsturm/molsturm_structure}
	\caption[Structure of the \molsturm package]{%
	Structure of the \molsturm package: Shown are the five modules of the package,
	along with the set of integrals accessible from \gint and the set of post-HF method,
	which can be used from \molsturm. The greyed-out parts are not yet implemented.
	Only the modules inside the red box are part of \molsturm. The blue boxes are all external packages.}
	\label{fig:structureMolsturm}
\end{figure}



\todo[inline,caption={}]{
	\begin{itemize}
		\item molsturm paper
		\item gint: Integrals are lazy matrices
		\item gscf: Independent algorithms (refer back to chapter about numerical HF)
		\item Interfacing with other packages
	\end{itemize}
}






Unlike most existing quantum-chemistry packages
\molsturm (see \ref{fig:structureMolsturm}) is not tailoring
a specific type of basis function.
Even though this certainly requires
a certain flexibility with respect to the algorithms
and program flow,
the very basic structure of a typical quantum chemical
calculation is pretty much independent
of the type of basis function used.

Typically one first performs a self-consistent field calculation,
either by solving the Hartree-Fock or the Kohn-Sham equations.
The obtained self-consistent solution is then used for example in
linear response calculations or Post-HF methods.
It is important to note that these steps
typically do not need any knowledge of the underlying basis function type
at all, since they are formulated in terms of SCF orbitals only.
The \molsturm program package therefore concentrates
on a basis-function independent SCF code
with interfaces to simplify using the resulting SCF solution
in third-party code for further processing.

As can be seen in fig. \ref{fig:structureMolsturm} this is realised
by a modular design,
where each of the five major modules has a very
distinct responsibility.
Most importantly only the integral module \gint
exploits the properties of the type of basis functions.
\gint for example decides the sparsity structure of the
repulsion integrals or the Fock matrix.
This structure is encoded inside the lazy matrices,
which \gint exposes to the SCF algorithms inside \gscf.

From the point of view of \gscf,
on the other hand,
the lazy matrices for all types of basis functions look alike.
In other words by the means of the lazy matrices
the algorithms in \gscf
can be written in such a high-level manor,
that they are entirely independent of the precise matrix structure.
Consequently they are independent of the choice of basis function type as well.

The lower layers \lazyten and \krims provide
the lazy matrix implementation as well as other utilities,
which are needed to pass information between the individual modules.
\lazyten furthermore contains an interface,
which allows lazy-matrix based algorithms like the ones in \gscf to
call standard iterative solver implementations.

The next sections describe the individual modules of the \molsturm
package in greater detail together with their interaction
with external packages.

%
% ----------
%
\subsection{\molsturm interface layer}

The topmost \molsturm interface layer is concerned with the setup of an
SCF calculation as well as the analysis of the results or setting
up Post-HF calculations in third-party code.

Unlike most traditional quantum-chemistry packages \molsturm does not have
a particular ``Input format'' to drive the calculation.
Much rather one uses \molsturm by writing a \python
script which calls the appropriate functions from the interface
layer to perform the SCF or Post-HF calculations.
One can hence think of this interface layer as a library
which provides all the functionality needed to write
a calculation \python script.
Compared to the traditional input file approach this has
a couple of advantages.

Firstly it makes it very easy to interactively work with \molsturm
from a shell.
This reduces the feedback loop for small calculations
or during debugging.
If one uses \molsturm from a Jupyter notebook
\todoil{cite \url{https://jupyter.org/}}.
one can even perform calculations and view plots
interactively from within a web browser.

Secondly this gives the user full flexibility
in how he or she embeds \molsturm in his or her workflow.
Since all data from \molsturm is available
in plain \numpy arrays the user has full control to do
any kind of standard processing from \python scripts
after or during a calculation.
This implies that the automatisation of repetitive
analysis work or repetitive calculations
becomes almost effortless as well.
See section \ref{sec:ex:data} for an example
\python script which automatically computes and plots
simple dissociation curves with \molsturm.

Thirdly the high-level \python interface of \molsturm 
makes it even possible to go one step further.
Since the repulsion integral tensors or the fock matrix
are available from \molsturm as very convenient data structures,
implementing new Post-HF methods,
which sit on top of \molsturm is comparatively simple
as we will demonstrate with a CCD example in section \ref{sec:ex:ccd}.
Similarly it is possible to employ the features of other \python
packages to extend what can be done with \molsturm.
In section \ref{sec:ex:geo} we show a small \python script
which performs a geometry optimisation of water,
even though \molsturm currently has no support for SCF gradients.

By breaking with the traditional paradigm of plaintext-in, plaintext-out,
which most quantum chemistry programs still employ up to now,
we furthermore gain the flexibility we need to exchange data with
other third-party packages, for example for Post-HF caluclations.
Most packages common in quantum chemistry either use data which is
dumped to disk in the form of plain text or binary files
in order to store their intermediate or final results of
a calculations.
In many cases the programs are not very flexible with regards to
how the data is stored in these files and one hence needs to reproduce
the file format very carefully for interfacing.
In cases where the documentation of these file formats is sparse
or absent,
one might even need to reverse-engeneer in order to work with these programs.
This requires a lot of flexiblity on the calling side.
Luckily \python offers a large range of functions to serialise \numpy
arrays in text or binary form.

Furthermore \numpy arrays have become the \textit{de facto} standard
for storing and manipulating matrices or tensors in \python.
All third-party quantum chemistry codes which
offer a \python interface
either use \numpy arrays themselves or
provide utility functions to convert to and
from \numpy arrays.
\todoil{check if this is true and cite them}
The same is true for typical \python packages used for plotting or
data manipulation (matplotlib, scipy, pandas).

Thirdly by the means of interface generators
like SWIG \todo{cite} \numpy arrays can be converted to plain
\cee arrays for calling \cpp, \cee or \fortran code,
such that even low-level libraries can be employed directly
from \numpy arrays.

%
% ----------
%
\subsection{The integral interface via \gint}
Similar to the \molsturm interface layer when it comes to Post-HF methods,
\gint does not implement any routine for computing the integrals.
Instead it just acts as a broker,
which presents a common interface for all available basis function types
and third-party integral backend libraries.

When a calculation starts,
\molsturm passes \gint the set of parameters,
which are relevant for the selection and setup of the integral backend,
like for example the Coulomb-Sturmian exponent or the Gaussian basis set information.
In return \gint provides \molsturm with a collection of lazy matrices,
where each lazy matrix represents a single integral for this
type of basis functions.
For example the nuclear attraction matrix, the kinetic operator matrix,
the coulomb matrix and the exchange matrix each exist as separate
lazy matrix objects.
These integral objects may then be combined
in an arbitrary manor to form the Fock matrix or Kohn-Sham matrix
expression by just forming linear combinations.
The resulting Fock matrix expression is then typically passed onto \gscf,
where the SCF problem is solved.

The generality of the lazy matrices allows the interface of \gint
to be entirely independent of the structure
which is used to represent the integral data internally.
This implies that all code inside \molsturm,
which works with the integral objects,
hence becomes independent of the way the integral data is represented
or how the evaluation of the \contraction-function is actually performed.
\gint may freely choose the scheme for evaluating a matrix-vector
product or for example the sparsity structure which is used to
represent the data of the integral matrix
without this having any impact towards other parts of \molsturm.

Using lazy matrices furthermore has an advantage for
dealing with the non-linearity of the coulomb and exchange matrices:
The lazy matrix formalism of \lazyten provides the possibility
to implement a special \update function,
which can be used to update internal state of a lazy matrix.
The coulomb and exchange lazy matrices
may use this function to retrieve a new set of coefficients from an
SCF algorithm and update
their internal state accordingly.
As we will discuss in the next section \gscf makes frequent use of this feature.

Currently \gint only offers two types of basis functions,
namely Gaussian integrals and atomic Coulomb-Sturmians,
both available in at least two different implementations.
For example Gaussian integrals for \gint are available from either one of 
\libint by Valeev et al.\cite{Libint2,Libint2_231}
or \libcint by Sun et al.\todo{cite} 
under a common interface.
The design of \gint makes it furthermore very simple to extend the types of integrals
available in \molsturm:
The function calls which compute the integral values only need to be wrapped
inside lazy matrix objects and the resulting integral collection needs to be registered
with \gint.
For example the support for \libcint was added in just two days of work.

Note that this does not even require changing a single line of code inside \gint.
One could achieve this by linking the library into \molsturm at compile time.

%
% ----------
%
\subsection{Design of SCF algorithms in \gscf}
\todoil{Fock matrix $\Leftrightarrow$ Kohn-Sham matrix $\Leftrightarrow$ Problem matrix
to make clear that \gscf is not only for HF problems }

In \gscf all self-consistent field algorithms are assumed to be
iterative processes alternating between two basic steps.
Firstly an \term{Fock-update} step where a new Fock matrix is produced from a current set of
orbital coefficients.
Secondly a \term{coefficient-update} step where the current Fock matrix guess
is used to update the current set of orbital coefficients.
Let us briefly rationalise this choice by 
looking at the two mathematically equivalent
\todo{cite}
ways a
Hartree-Fock or Kohn-Sham self-consistent field problem is typically
viewed in order to solve it.

% TODO TODO TODO TODO
% TODO These plots should live in extra tikz files!
% TODO TODO TODO TODO
\tikzset{
	flobo/.style={rectangle,draw,fill=white,minimum width=4.7cm,align=center,
					font=\smaller, text width=4.5cm},
	post/.style={->,shorten <=0.5pt,>=stealth',semithick},
}
\begin{figure}
	% Work the update function and the contraction function calls into these examples
	\centering
	\todoil{Look into this algo a little more closely}
	\begin{tikzpicture}
		\node [flobo]              (c0)
			{Initial coefficients $C^{(0)}$};
		\node [flobo,below=of c0]  (fi)
			{Build coefficient gradient (involves Fock matrix \update: $F^{(i)} = F\left[ C^{(i)} \right]$)};
		\node [flobo,below=of fi]  (ci)
			{Coefficient update};

		\draw [post] (c0.south) -- (fi.north);
		\draw [post] (fi.south) -- (ci.north);
		\draw let \p0 = (fi), \p1 = (ci) in 
			[post] (ci.west) -- (-3, \y1) -- (-3, \y0) -- (fi.west);

		% Coefficient guess -> Gradient (involves Fock update)
		% -> Coefficient update (as minimum along geodesic)
		% -> repeat to 2
	\end{tikzpicture}
	%\includegraphics{<+file+>}
	\caption{Geometric direct minimisation algorithm scheme}
	\label{fig:GDM}
\end{figure}
\begin{figure}
	\centering
	\begin{tikzpicture}
		\node [flobo]              (c0)
			{Initial coefficients $C^{(0)}$};
		\node [flobo,below=of c0]  (fi)
			{Fock matrix \update: $F^{(i)} = F\left[ C^{(i)} \right]$};
		\node [flobo,below=of fi]  (ci)
			{Diagonalise $F^{(i)}$ to get next $C^{(i+1)}$ (coefficient update)};

		\draw [post] (c0.south) -- (fi.north);
		\draw [post] (fi.south) -- (ci.north);
		\draw let \p0 = (fi), \p1 = (ci) in 
			[post] (ci.west) -- (-3, \y1) -- (-3, \y0) -- (fi.west);

		% Coefficient guess -> Fock matrix update
		% -> Diagonalisation to get new coefficients
		% -> repeat to 2
	\end{tikzpicture}
	%\includegraphics{<+file+>}
	\caption{Roothaan repeated diagonalisation (Plain SCF)}
	\label{fig:Roothaan}
\end{figure}
\begin{figure}

	\centering
	\begin{tikzpicture}
		\node [flobo] (d0) {Initial density $D^{(0)}$};
		\node [flobo, below=of d0] (fi)
			{Fock matrix \update: $F^{(i)} = F\left[ \tilde{D}^{(i)} \right]$};
		\node [flobo, below=of fi] (diag)
			{Diagonalise $F^{(i)}$ to get $C^{(i)}$.
			 Get $D^{(i+1)}$ from Aufbau principle};
		\node [flobo, below=of diag] (di)
			{Find $\tilde{D}^{(i+1)} =
				\argmin\limits_{D \in \text{line}\left(D^{(i+1)}, \tilde{D}^{(i)}\right)}
				E_\text{HF}(D)$};

		\draw [post] (d0.south) -- (fi.north);
		\draw [post] (fi.south) -- (diag.north);
		\draw [post] (diag.south) -- (di.north);
		\draw let \p0 = (fi), \p1 = (di) in
			[post] (di.west) -- (-3, \y1) -- (-3, \y0) -- (fi.west);

		% Density guess -> Fock matrix update
		% -> Diagonalisation and new density
		% -> linear combination with previous densities giving optimal energy (geodesic in Density manifold)
		% -> repeat to 2
	\end{tikzpicture}
	%\includegraphics{<+file+>}
	\caption{Optimal damping algorithm scheme}
	\label{fig:ODA}
\end{figure}

The first viewpoint thinks of it as the variational minimisation
of the ground
state SCF energy with respect to the orbital coefficients making up
the best single Slater determinant approximation to the ground state.
An example would be the Geometric direct minimisation algorithm~(see fig. \ref{fig:GDM}).
In these algorithms the Fock matrix occurs in the gradient expression
\todoil{Cite Cances paper and Head-Gordon paper}
of the energy with respect to the coefficients.
An iterative optimisation procedure hence
usually involves the computation of the Fock matrix~(Fock update)
in each iteration in order to determine the search direction
for the next step.
This direction then leads to an updated set of coefficients.

The second viewpoint considers the non-linear Eigenproblem
which results from the aforementioned optimisation problem.
This is solved by linearisation,
i.e. one assumes a guess for the coefficients
in order to build the Fock matrix~(Fock update)
followed by a diagonalisation of this very Fock matrix
in order to obtain new update coefficients~(coefficient update).
This is repeated iteratively until self-consistency is reached.
An example is Roothaan's repeated diagonalisation algorithm~(see fig. \ref{fig:Roothaan}).

In both cases the typical iterative algorithms used for
solving optimisation problems or diagonalisation
can be formulated in a way that they are fully \contraction-based,
i.e. only require matrix-vector products between the Fock matrix
and other iterated vectors.
In \gscf we therefore require only two things from any given Fock matrix:
Firstly a special \update function which updates the internal state of
the matrix from the current set of coefficients
during the Fock-update step of an algorithm.
Secondly the aforementioned \contraction function for the
required matrix-vector products
during the coefficient-update step.

This is exactly the interface offered by the lazy matrix integral objects
which are returned by the \gint integral library.
We can therefore simply use the \gint integral terms to build the appropriate expressions
of the Fock or Kohn-Sham problem matrices
and hand them over to \gscf for finding the self-consistent set of SCF coefficients.
In \gscf itself the SCF algorithms are only implemented in the
high-level language of the lazy-matrix formalism.
Since the lazy matrix interface fully abstracts the internal structure of the matrix
the SCF algorithms of \gscf
are completely agnostic to the structure of the integrals and the Fock matrix,
which may be freely decided by the integral backend of \gint
depending on what suits the particular type of basis function best.

In the discussion about \gscf so far we have focused only on
\term{coefficient-based} SCF algorithms,
i.e. algorithms where the orbital coefficients are iterated to self-consistency.
Yet another way to think about the SCF procedure is to iterate the
density matrix to self-consistency instead.
Mathematically such \term{density-matrix-based} algorithms
are just solving the SCF problem in a different parametrisation
compared to the \term{coefficient-based} algorithms
and are otherwise equivalent.
\todoil{cite papers from Eric and others about this.}
An example for a density-matrix-based algorithm is the optimal damping algorithm
by Eric Cancès and Claude leBris,
\todoil{Cite}
which is shown in fig. \ref{fig:ODA}.

Currently all algorithms which are implemented in \gscf are coefficient-based SCF algorithms.
This is not strictly speaking required by the library,
but it has the advantage that for some basis function types
the computational scaling of the \contraction-function
of the Fock matrix expression can be reduced by an order of magnitude.
An example are the Coulomb-Sturmians.
\todoil{Maybe go into this a little in the previous sections and refer up}
Furthermore the density-matrix based algorithms
we have encountered so far could be reformulated as coefficient-based schemes
or were easily approximated to work as such.
For example we have implemented a variant of the original ODA,
which is suitable for a coefficient-based setting by truncating
the history of SCF steps considered in order to compute the optimal damping.

Compared to a scheme where the Fock matrix is held fully in memory
the \contraction-based design has some disadvantages, too.
For example our implementation of Pulay's DIIS conversion accelerator
cannot always form the linear combination of Fock matrices by adding the
individual Fock terms in memory.
Instead the linear combination stays around as a lazy matrix expression
until it is applied to a trial vector.
When this happens during an iterative diagonalisation algorithm
we hence need to call the \contraction-function for each of the
Fock terms separately instead of only once.
This is a trade-off we are happy to pay since it allows to
reduce the computational scaling of the individual \contraction
calls by an order of magnitude as mentioned before.

%
% ----------
%
\subsection{\lazyten: \contraction-based linear algebra backend}

\lazyten is the linear algebra interface of \molsturm.
It not only implements the lazy matrix datastructures,
which define the common interface of \gint and \gscf,
but further contains code to make standard external
iterative or direct solver implementations available
from a lazy matrices-based setting.

As can be seen in figure \ref{fig:structureLazyten} \lazyten provides
a common interface for the \contraction-based algorithms in \gscf,
regardless of the linear algebra~(LA) backend or the solver implementation,
which is used to solve a particular problem.
Lazy matrices like the integrals from \gint as well as built-in structures
like for example a lazy matrix representing the inverse of a matrix
can be used transparently and even in combination with stored matrices
by the means of the automatic bookkeeping done by \lazyten.
Whenever a call to for example the \contraction function enforces
evaluation of a lazy matrix expression,
\lazyten is flexible with respect to the LA backend which is used for
this step as well.
Both armadillo as a LAPACK-based backend as well as Bohrium as a array-operation based backend
are currently available and by recompiling \molsturm with the appropriate configure options,
the backend can be switched.

When it comes to the choice of the solver algorithms, \lazyten is even more flexible.
All algorithms for solving linear problems or eigenproblems are available as 
convenient high-level interface from \gscf.
For example a linear problem $A x = b$ with a lazy matrix $A$, known right-hand side $b$
and unknown $x$ can either be solved by a call to a \texttt{solve} function or by
the means of calling \texttt{inverse(A)} on $A$ and then subsequently
multiplying the inverse matrix with $b$, quite literally coding it as $x = A^{-1} b$.
In both cases \lazyten will perform some introspection regarding the properties of
the lazy matrix expression behind $A$ and consider the number of solution
vectors as well as the required accuracy in order to automatically determine the algorithm to use
for solving the linear problem at hand.
By the means of keyword arguments the user can influence or override the automatic
choice made by \lazyten.

Similarly the eigensolvers can be accessed from \gscf via a common interface,
which abstracts the precise algorithm and allows \lazyten to make an automatic
choice by looking at the precise structure of the Fock matrix at hand.
If all eigenpairs are desired or the matrix is stored in memory anyway,
\lazyten will favour the dense eigensolver algorithms from LAPACK,
whereas if only few eigenpairs are desired,
Krylov-subspace based methods from ARPACK are used.
In either case the precise algorithm can also me chosen from the \python interface
of \molsturm by supplying an appropriate keyword string.

%  - L10 (but short-ish, because there will be a full paper)
%  -- Idea: Write high-level formula similar to the equations on paper, generate efficient
%     streamed/lazy tensor-code.
%     (currently only supports matrices/linear operators and vectors).
%  -- Transparent combination of dense, sparse, and matrix-free
%  -- Solver-switching on the fly
%  -- How is multi-backend support solved
%  -- Design towards automatic generation of efficient (Only simple evaluation currently supported)
%  -- Include simplified L10-figure from talk

%
% ----------
%
\subsection{The \molsturm test suite}
A very important subsidiary to good software design is a flexible testing framework.
A test suite which is simple to execute, fast and easy to expand not only allows to verify
that the current status of a piece of software is correct,
but it also allows to verify that all future changes do not break anything.
This includes of cause the potential adaption of the design in the future,
which might involve changing any of the already existing interfaces.
A strong test suite aids with this procedure,
since one can perform these changes in a sequence of many small steps,
verifying the correctness of the software on the way.

For this reason \molsturm comes with an extensive test framework
with roughly four types of tests.
Firstly there are \term{unit tests},
which test the functionality of a single function or code unit in a couple of hard-coded examples.
Further we have \term{functionality tests},
which test a larger portion of code and are meant to ensure that
the results of \molsturm agree between different versions.
Thirdly the \term{reference tests} compare
the results of \molsturm to other quantum-chemistry packages.
Especially in \lazyten we furthermore make use of yet another type of tests,
namely so-called \term{property-based tests}.
This testing technique uses the expected properties of a code unit
to randomly generate test cases and to verify the result.
On failure the generated test cases are simplified until the most simple,
failing test case is found.
This is really helpful for debugging and in order to reduce the human aspect in testing.

Next to combining various different types of tests
other key principles in the design of our test suite were that our tests should be simple
to execute and take only a couple of minutes.
This not only makes it easier to motivate others to use the tests, too,
but it also allows to fully automate the whole testing process as well.
\todoil{I feel we should explicitly mention github and Travis-CI to show them some credit for this awesome work.}
In our case any commit to our repositories starts up a couple of virtual machines,
which build \molsturm and run the test suite for a couple of common compilers and
build configurations.
A commit into the stable source code branch is only accepted if the test suite passes.

In our test suite even the work of adding a new test is somewhat automated.
For example in order to add a new reference test for a new feature
or a case where a bug was discovered,
one simply adds a small configuration file to a special directory,
which defines the testing parameters.
A \python script, which is part of the test suite,
then automatically executes the third party program and extracts the reference results in a format,
which is understood by the \molsturm test suite.
When the test suite is executed all such reference results are read and
\molsturm is executed and tested for each of them.

\todoil{Mention test coverage ???}

\todoil{We probably should not keep this next paragraphs here}
On the side of the \molsturm code we employ \term{assertive programming} to assure that
\molsturm crashes with a very detailed backlog whenever an unexpected behaviour happened.
By the means of a dual build system setup, we achieve that a Debug version
(including heavy assertive checks) and a Release version
(without heavy assertive checks) are compiled side-by-side.
This allows to use the Debug version with a lot of assertions,
potentially even in the inner loops,
to be used for trial runs and the Release version with less assertions for productive runs.
Switching between Debug and Release from \molsturm's python interface is very simple and
only requires to recompile a few files,
since in normal build conditions Release and Debug have already been built.

By the means of checkpointing one could further use the Release version to get an erroneous
part of \molsturm and the Debug version to step-by-step test what goes wrong.
The detailed traceback on failure of an assertion can then be used to analyse the problem quickly.
