\section{The need for modular quantum chemistry software}
\todo[inline,caption={}]{
	\begin{itemize}
		\item pyscf \cite{Sun2017}
	\end{itemize}
}

\subsection{The need for a flexible quantum chemistry software platform}

% TODO Ideas of Tobias Setzer:
% Typical programs are ASCII-in -> ASCII-out
% An idea going in a similar direction is for example TURBOMOLE
%   -> apparently they have very modular modules, where the state of a compuation
%      is incoded in the file on a harddrive
%   -> issue is getting out of the TURBOMOLE ecosystem
% Actual analysis, which is not part of the program package is compartively difficult.
% Especially the transfer of data from one package to another is extremely tough.
%
% Molsturm:
%     fully dynamic, real-time analysis
%     future of molsturm: overcome software package boundaries
%     Journals: JCTC, JCP  (siehe turbomole jungs)
%
% Overall: Focus more on the advantages of the package architecture


% TODO copy -> build into this section
Most existing quantum-chemistry packages
are built around a 
particular basis function type: either Gaussians,
Slaters, or numerical orbitals. This is hard-coded 
deeply in the programs, which are usually enormous
projects written in  highly optimized \cpp or \fortran.
This makes experimentation with different and more physically
appropriate basis functions exceedingly difficult:
even when all electronic integrals have been worked out,
adapting an existing quantum chemistry code to use
them is a daunting project.
% end copy



% TODO Flowify
\todoil{Check and enhance the story by looking at more references}
Starting from the exponential form of the analytical solutions of the
Schrödinger equation for hydrogen,
\citeauthor{Slater1930} started introducing exponentially decaying basis functions
in order to compute atomic properties,~\cite{Slater1930}
the nowadays well-known Slater-type orbitals~(STOs).

Applying the STOs to molecules turned out to be much more
challenging due to the demanding structure of the electron repulsion integrals~(ERIs)
for such types of basis functions.
\todoil{I cannot download that paper so im not too sure what is in it}
In \citeyear{Boys1950} \citeauthor{Boys1950} realised that employing
atom-centered Gaussian type basis functions~(GTOs) leads to integrals
which are much easier to evaluate numerically.

Since they do, however, not describe the physical reality as closely as STOs
\todoil{Check whether the pople paper goes into this}
much larger GTO basis sets are needed to get similar accuracies than STOs.

John Pople
\todoil{Paper: Hehre, W. J.; R. F. Stewart; J. A. Pople (1969). "Self-Consistent Molecular-Orbital Methods. I. Use of Gaussian Expansions of Slater-Type Atomic Orbitals". Journal of Chemical Physics. 51 (6): 2657–2664. Bibcode:1969JChPh..51.2657H. doi:10.1063/1.1672392.}
later realised that one could combine the best of both worlds if one used
basis sets of contracted GTOs.
This set off a vivid development of quantum chemical theory and software packages,
where a lot of work has gone into efficient algorithms
improving the computational
scaling as well as the overall runtime of calculations based on GTOs.
scaling as well as the overall runtime of calculations.
\todoil{Examples for program packages ??}
% end flowify

Almost all major quantum chemistry programs,
\todoil{Cite a few}
which are actively developed nowadays focus on Gaussian-type
basis functions or Gaussian-type orbitals~(GTOs) as they are usually called.
Compared to the Slater-type orbitals~(STOs)
they certainly have a couple of well-known advantages.
Most importantly the structure of the electron repulsion integrals~(ERIs)
is much less demanding and hence makes it feasible to model much larger systems.

The downside of GTOs is of course that they do not describe the physical
reality that well and as a result fail to capture all of the properties
of an electronic structure in a truely black-box fashion.
For example the computation of proper electron affinities or the description
of Rydberg-like excited states requires incorporating diffuse functions
with small Gaussian exponents into the GTO basis set.

As such it is not surprising that other types of basis sets have been attempted as well.
Next to plane-waves or projector-augmented wave methods,
which are commonly employed in the density-functional theory~(DFT)
calculations of metal surfaces,
recently a couple of projects have also investigated the use
of so-called numerical basis functions.
Examples would be the use of wavelets and the finite-element medhod.
\todoil{Cite madness and Dage Sundholm}
\todoil{The Jensen review paper has a couple of nice citations there}

% structure is vastly different
% need flexible framework to try them
% how do they compare -> need fair comparison

\begin{figure}
	\centering
	\missingfigure{Instead of having the figures here,
	refer to fig. \vref{fig:StructureFiniteElementFock}, \vref{fig:StructureGaussianFock}
	and \vref{fig:StructureSturmianFock}}
	\caption{The different structure of the Fock matrix when FEs,
		Sturmians and GTOs are used.
		All matrices are shown early in the SCF process with a Pulay error norm
		larger than 0.1.
		The matrix structures generally improve as the SCF proceeds and contain more zeros.
		%
		Sturmian and Gaussian are almost diagonal dominant,
		even almost strictly diagonal dominant,
		FE matrix is far from it (About half the rows do not satisfy the condition).
		%
		Only the alpha-alpha part of the fock matrix is shown in each case
		and we perform an RHF calculation of the Berylium atom in a small basis
		for each set of basis functions.
		%
		The Sturmian basis is given in lmn order.
		% TODO maybe use mln order ?
	}
	\label{fig:FockStructure}
\end{figure}

The structure which is required in a program package in order to perform even a simple
self-consistent field~(SCF) calculation
based on GTOs and on numerical basis functions differs quite a lot.
As can be seen in \fig~\ref{fig:FockStructure} the structure of the Fock matrix
varies rather dramatically depending on the type of basis function,
which is used.
Most importantly the matrix size and matrix sparsity properties
changes a lot between numerical approaches like
Wavelets or Finite Elements and classical atom-centered AO approaches like GTOs.
As a result none of the widely used program packages
focus on exactly one type of basis function and are highly specific for exactly this type.

Since the structure of the numerical problems to be solved differ
implementing further basis function types into existing packages
is often challenging.
Moreover unlike the well-understood Gaussians in novel basis function types
the kind of solver algorthims to use is less clear and one often needs to be able to experiment.
\todoil{
all separate packages 
see wikipedia page
\url{https://en.wikipedia.org/wiki/List_of_quantum_chemistry_and_solid-state_physics_software}
}
Furthermore we only know of very few programs which have more than experimental
support in more than one type of basis function.
\todoil{I do not want to make this statement to harsh to piss anyone off,
	but the point is that those packages, which do exist are either not widespread
	or lack features in some of the basis functions
	or are hard to obtain (i.e. closed-source, even wikipedia knows no website)
	or I have literally not heard of them anywhere}

\todoil{I feel some more details introducing SCF in general is also appropriate here}

In previous work we have investigated the use of so-called generalised Sturmians as basis functions
in electronic structure theory.
%
\todoil{I think this is already done in one of the previous chapters in the diss}
Generalised Sturmians $\Phi_\nu(\rpack)$ are the solutions to $N$-body Schrödinger-like equation
\begin{equation}
	\left( -\frac12 \sum_{j=1}^N \Delta_j + \beta_\nu V_0(\rpack) - E \right) \Phi_\nu(\rpack) = 0
	\label{eqn:GenSturm}
\end{equation}
where $V_0(\rpack)$ is
a zero-th order potential.

In case of atoms a good choice is the nuclear attraction:
The electrostatic electron-nucleus interaction
\[
	V_0(\rpack) = - \sum_{j=1}^N \frac{Z}{r_j}.
\]
\todoil{Is many nuclei also possible here?}
Note that the electron-nucleus interaction is scaled by the factor $\beta_\nu$ in \eqref{eqn:GenSturm}
such that the solutions $\Phi_\nu(\rpack)$ all become isoenergetic.
This makes Sturmians reproduce the correct long-range decay behaviour of the electron density
as well as properly represent the nuclear cusp at the core.
As such the functional form of generalised Sturmians is very much related to STOs.
Unlike STOs the two-electron integrals can, however, be reformulated in a particular way
to make computing them less demanding.
This requires, however, that the mathematical properties of the Sturmians
can be fully exploited during the computation,
which in turn requires the Fock matrix to be arranged in a specific way.
Moreover it is advantageous to not build the coulomb and exchange matrix in memory at all,
but much rather only use these matrices in the form of matrix-vector products,
since this overall saves an order of magnitude in the computational scaling.
%
\todoil{How much shall we go into detail here to make this clear?}
% Not at all, do it in the lazyten section.
%
This rather unusual demand turned out to make it very difficult to implement Sturmians
as a basis function type into existing packages alongside STOs and GTOs.

% TODO do this do this
\todoil{Go a bit more into why we only do SCF (Explain: Post-HF available, we only need MO basis, structure of AOs not important)}
% TODO do this do this
In \molsturm we have achieved to write an Hartree-Fock SCF code,
which allows us to use both GTOs and Sturmians side by side by the means
of so-called lazy matrices (see section \ref{sec:lazymat}).
We achieve true generality in the type of basis function to be used
up to the point where most of our code is entirely
independent of the basis function types.
Enwrapping \molsturm into a \python layer allowed us to easily re-use existing
software modules (\adcman and \pyscf) in order to go beyond Hartree-Fock.
Even though we currently only have support for GTOs and Sturmians,
extending our code to numerical basis functions can be achieved rather easily
as we will demonstrate in this paper.

\todoil{Decide on one of them}
The name \molsturm origates from \textbf{mol}ecular \textbf{sturm}ians
or \textbf{mo}dular and \textbf{l}azy with \textbf{sturm}ians.

\todoil{The previous section gets rather lengthy. It should probably be split up somehow}


\subsection{Lazy matrices and \contraction-based algorithms}
\label{sec:lazymat}
\todoil{Maybe illustrate the points mentioned here using the coulomb and exchange matrices as actual examples}

% TODO a good paragraph for this part:
Whenever operations are done on a matrix \lazyten does not automatically
evaluate them in all cases.
Much rather it usually just builds up a datastructure,
which keeps track of the operations which \textit{should} be done
at some point in the future.
Whenever an actual call to the \contraction-function happens,
the expression history is considered and evaluated with respect
to the other arguments supplied by the \contraction call.
% end TODO

% For the design chapter it is important to mention here that \contraction-based methods
% are a generalisation of non-contraction based methods.

\todoil{Maybe this section is too long given that we want to publish on this again}

As mentioned above for some basis function types it is advantageous
to not build the coulomb and exchange matrices in memory,
but instead only use them in the form of matrix-vector applications.
Since iterative solvers like Krylov-subspace based methods or Davidson's algorithm
only need the matrix-vector product in order to find a few of the eigenvalues
of a particular matrix,
this is not an issue with respect to the eigenproblem which needs to be solved
as part of the SCF.
Note, that this approach is furthermore not new in quantum chemistry at all.
Other examples where one typically avoids to place the matrix to diagonalise
into memory are the configuration interaction~(CI) matrix
or the algebraic-diagrammatic construction~(ADC) matrix to name a few.

We shall refer to algorithms which focus on implementing a matrix-vector product
instead of the full matrix as \contraction-based algorithms.

This concept has been invented many times
and has hence been called a number of different things
in the different communities.
Other names include \texttt{matrix}-free
\todoil{papers from Kronbichler}
or apply-based
\todoil{papers from Michi Wormit}


The latter name is supposed to indicate that such a scheme is in principle
not only restricted to the case of multiplying a matrix with a vector,
much rather general tensor contractions could be expressed implicitly
by the means of a function computing the contraction rather than
by explicitly performing the contraction form tensor elements
which reside in memory.
\todoil{Also mention the term \texttt{apply}-based?
Check what term is used in literature.
Michi Wormit's thesis uses matrix-vector product as term for ADC
}
% Yes mention them all.

Note that a disadvantage of \contraction-based algorithms is that the expressions
computing the tensor contraction can become rather complicated.
Furthermore it usually is more intuitive to think of the
numerical modelling in terms of tensors, matrices and vectors
rather than \contraction functions.
For this purpose we generalise the concept of a matrix
to so-called \term{lazy matrix} objects.
Whilst a normal or \term{stored matrix} is dense and has all its elements
residing in memory,
a lazy matrix is more general.
It may follow a particular sparse storage scheme
like compressed-row storage
or even more general all its elements may just be arbitrary expressions.
The values of lazy matrices are hence only computed upon request
or whenever a contraction with another object is required.
As such the lazy matrix may have, \ie a special \update function
may be used to modify the expression of the lazy matrix
at a later point.
Note, however, that special storage schemes for storing
sparse matrices are similarly just lazy matrices 
\todoil{
	Updating lazy matrices is a bit like reactive programming
	\url{https://en.wikipedia.org/wiki/Reactive_programming}.
	In fact one could use them to get reactive programming
	into C++ in a simple way.
	Mention this here (or in the \lazyten paper)
}

This on the one hand makes obtaining the lazy matrix elements
expensive, but on the other hands gives a nice matrix-like
interface to more complicated objects.
All evaluation between lazy matrices
(i.e. addition or matrix-matrix multiplication)
is usually delayed until for example a contraction of the resulting
expression with a vector is performed.
One typically refers to this strategy as \term{lazy evaluation}.

Since lazy matrices are a straight generalisation
of usual matrices,
all algorithms written in terms of lazy matrices
are at the same time applicable to dense matrices,
sparse matrices or special matrices like in the case of Sturmians
where a lazy evaluation scheme is needed.
So high-level code written in terms of lazy matrices
does not need to be changed if the low level matrix implementation
is changed from one basis function type to another.

\todoil{One could mention the  processor-memory performance gap \ldots
	    but I would leave it for the \lazyten paper}

% 1/2 paragraph: Introducing the problem of large tensors in QM - apply based algorithms
% 1/2 paragraph: One of the flexibilities of Molsturm is that we can switch between
%                apply-based, sparse, and dense methods without changing high-level
%                code (which simply mimic the formulæ). 
%
% define stored vs. lazy matrix
% lazy: sparse or expressions
% lazy matrix may have state
% other key features of lazy matrices


\subsection{Basis-function independence and hot-swapping}
Conceptionally an SCF algorithm is entirely independent of the type of basis function.
One could think of it very much as a numerical technique to solve
an non-linear eigenvalue problem by linearisation.
Once the SCF orbitals have been obtained,
the remainder of a calculation (\eg a Post-HF method)
can be formulated entirely in the SCF orbital basis
and without any reference to the underlying type of basis functions.
As such it should be possible to design a quantum chemistry package,
which apart from the integral backend itself is entirely independent
of the basis function types which are implemented.

Achieving this has several advantages.
First of all it naturally implies that trying out some new basis functions is very easy,
since most of the code already exists and can be reused.
Just the integral computation by itself needs to be adapted.
Secondly comparing different basis function types can occur in a fair setting,
since for all types of basis functions the algorithms are optimised to a similar level.
Thirdly it even allows for a fair comparison between different integral backends,
\eg different GTO libraries.

Last but not least the ability to treat different kinds of basis functions
in the same framework simplifies the construction of hybrid basis sets,
where possibly numerical basis functions and GTOs or STOs are combined.
Similarly one could employ the strengths of multiple backends for the same type of basis function
in the sense that one mixes and matches different backends such that one can exploit
the advantages of both implementations best.

This works since at the level of the SCF all basis function types and backends
share exactly the same common interface,
which facilitates combinations between them.

% why is basis function independence useful
% how lazy matrices help here
% Also discuss hybrid basis sets
%% Coulomb Sturmians with (hot-swappable with Gaussians)

\subsection{Rapid development of new QM algorithms}
\todoil{Mention other python effords like SecondQuatizationAlgebra, pyscf, PyQuante,
python backends ins psi4 or GPAW ??}
% Covered in the related work
% refer downwards?

In the early stages of developing a new quantum-chemical algorithms
it is often not clear how well these algorithms perform
or if they even meet the expected requirements.
Before worrying about making the algorithm fast,
one first wants to know whether it even works.
For this a light-weight framework which possesses the flexibility
to quickly combine or amend the already implemented
functionality is very important.
Furthermore well-documented and open interfaces
as well as compliance with the available standards
allow easy integration of other frameworks to extend
the functionality of a package even beyond the scope
which was intended in the first place.
\todoil{Define rapid development better or leave the term "RAD" out?}
% leave the term out.

The \python interface of molsturm for example is deliberately designed
with simplicity in mind.
The downside is that many potential symmetries when storing the data
cannot be exploited,
but on the upside we managed to integrate \molsturm
with some third-party Post-HF libraries rather quickly.
For example the Full-CI interface to \pyscf was realised in only two days,
but still is general enough to work for Sturmians GTOs and theoretically
all basis function types which are implemented in our integral backend.
\todoil{Change to also mention CC if we get there}
In another proof-of-concept project of ours we just managed to
implement ADC(2)-x on top of \molsturm in less than 10 days in \python.

The careful reader might have noticed that a well-designed interface
might not only facilitate rapid development of new algorithms
but also prevent one from the need to re-implement the wheel
and much rather exploit the code which has already has been written
for one's purposes.

% why do we want that?
% what is rapid development
%    design of prototype to explore
%    tests to fixate the requirements
%    iterative update of the prototype to get to the final program
%
% Rapid development techniques require a flexible platform
% well-documented and open interfaces
% compliance to standards and easy integration with other frameworks


% phd should not spend a year to implement something which might not work
% need flexiblity to try things
% code should be easy and close to the physical formulae
%
% We no not want to re-implement the wheel -> integrate with what exists

% ADCman, MP2 
% Full CI (two days of work), PySCF
% ADC example? (ADCman)
% Can we do CC (PySCF)?


% Programmable from Python 
% PyADC [Proof of concept, done in 10 days]
% PyADC+Bohrium [Automatic parallel on CPU, GPU. Work in progress]

\subsection{Automated visualization and data analysis}
\todoil{This paragraph is about the user perspective. Make this more clear}
A large part of the everyday work in quantum chemistry
is the analysis of data which is generated by quantum-chemical programs.
Naturally when investigating a research question it is often
hard to tell what angle is needed to explain what is going on
before doing any preliminary analysis.
In other words the process of understanding what is going on is iterative
and usually supporting calculations and further modelling
needs to be done once more about the problem is known.

\todoil{Reword. This is not the best phrasing}
Therefore a flexible quantum-chemical framework allows to
easily amend calculations which have already been done
at a later stage, making maximal use of what is already known.
Surely in many cases where a more accurate method (e.g. CCSD) is employed
on top of preliminary results (e.g. MP2),
the cost of re-computing the preliminary result is negligible.
In some cases, however,
where the further investigation just involves obtaining
extra quantities like some density plots or similar,
it is rather unnecessary to run the same calculation again,
just because some parameters to request such properties
have been forgotten in the first instance.
Many quantum chemistry packages therefore allow
some mechanism to reuse previous results.
In some cases these mechanism can be rather inconvenient, however.

Furthermore the analysis of obtained data should be easily scriptable.
Most quantum chemistry programs produce human-readable plain-text output
for this very reason.
By the means of standard unix tools like \texttt{grep}, \texttt{awk},
\texttt{bash} or \python or similar those files are then parsed and
the data post-processed.
An alternative and in our opinion more advantageous approach
would be to instead offer a scriptable interface to control
the program package by itself,
along with some utility functions to print summaries
or plot results interactively.
This way results are fully available for a user to analyse
and he himself can decide what information to look at
and what not.
By the means of utility functions
archiving results as well as standard analysis
like printing summary information or plotting some orbitals or densities
should be easy.

Note that on the one hand gives the user much more flexiblity and control
what to look at,
but on the other hand still does not prevent the traditional
mechanism of converting the results into a human-readable output file
by the means of a simple wrapper script.

Especially the ability to archive a large portion of the calculation
furthermore allows to exploit previous results when performing
further calculations as well as delaying the analysis
to a later point when \eg more knowledge about the
problem has been obtained to decide what exactly to look at.
