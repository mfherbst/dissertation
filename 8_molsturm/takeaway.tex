\section{Takeaway}

The implementation of quantum-chemical methods
employing novel types of basis functions often necessitates
unusual numerical techniques as well.
Implementing these into existing quantum chemistry packages
can be rather challenging,
since these are on the one hand highly optimised towards the methods
they already accommodate,
but on the other hand typically not flexible enough
to meet the differing requirements.

The \molsturm package we have presented here,
tries to fill this gap
by providing a light-weight package designed with a range of different
basis functions in mind.
The key ingredient to reach the necessary flexibility
is our \contraction-based self-consistent field~(SCF) scheme,
which we employ for solving the Hartree-Fock problem.
In such an \contraction-based ansatz one formulates
the numerical algorithms in a way that they only require the
contraction of the Fock matrix with an arbitrary set of vectors.
The details how this matrix-vector product is computed
can be varied flexibly according to the numerical properties
of the basis function type.
In other words we have reached a scheme where the code describing
the SCF algorithm and the code for doing the linear algebra
computation are separated
and thus where the SCF itself is basis-function independent.
Even if changes to the SCF scheme or some back end
library are to be underdone in the future our modular design
will assure that the other parts of the \molsturm package will stay
unaffected.

On top of that we have made the interfaces
of our SCF easy-to-use and readily extensible.
This allows to quickly incorporate the functionality
of third-party packages in order to extend \molsturm in ways we as the authors
would have never thought of.
Right now \molsturm may be used to perform
calculations based on contracted Gaussians~\cite{Hehre1969} --- using
the integral libraries \libint~\cite{Libint2_231,Libint2}
or \libcint~\cite{Sun2015} --- and based on
Coulomb-Sturmians~\cite{Shull1959,Avery2011} ---
using \sturmint~\cite{sturmintWeb}.
Selected Post-HF methods from \pyscf~\cite{Sun2017} as well as
excited states methods from \adcman~\cite{Wormit2014} are available on top.
Implementing further types of basis functions
takes nothing more than providing appropriate interface classes in
our integral interface library \gint.
Thereafter such basis functions are available for the full \molsturm ecosystem
including the Post-HF methods
provided by the third-party libraries mentioned above.

We have demonstrated the abilities of \molsturm
by three practical examples with particular emphasis
on the way our \python interface integrates with existing
\python packages.
We showed how to aid repetitive calculations,
implement novel quantum-chemical methods
or rapidly amend functionality in a preliminary way,
where a proper implementation would be much more involved.
We hinted how
systematic comparisons with established basis functions
as well as subsequent graphical analysis
is convenient to perform by the means of our
readily scriptable interface.
We hope that in this manner \molsturm
will be a useful package to rapidly try novel basis function types
and get a feeling for their range of applicability.

%%% Local Variables:
%%% mode: latex
%%% TeX-master: "paper"
%%% End:
