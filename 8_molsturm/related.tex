\section{Related quantum-chemical software packages}
\label{sec:MolsturmRelated}
\newcommand{\psifour}{\texttt{Psi4}\xspace}
\newcommand{\pyquante}{\texttt{pyQuante}\xspace}
\newcommand{\horton}{\texttt{HORTON}\xspace}
\newcommand{\gpaw}{\texttt{GPAW}\xspace}
\newcommand{\ASE}{\texttt{ASE}\xspace}
\newcommand{\CPtK}{\texttt{CP2K}\xspace}

Apart from \molsturm I am unaware of another quantum-chemistry package
which has achieved
a similar flexibility with respect to the type of basis functions
in its \SCF procedure.
Many packages still have related goals towards flexibility or generality of their codes
and should therefore not go completely unmentioned here.

When it comes to flexibility of a program package
a key ingredient is a versatile interface.
This allows to invoke or extend the methods already available elsewhere.
Recently the scripting language \python has become very popular
for achieving this.
Even even meta-projects like \ASE~\cite{Larsen2017},
which aim at extending existing packages by a common \python front end,
have emerged.
Other packages like \horton~\cite{Verstraelen2017}, \pyscf~\cite{Sun2017},
\pyquante~\cite{PyQuante} and \gpaw~\cite{Mortensen2005,Enkovaara2010} are written
almost exclusively in \python and only employ low-level \ccc or \cpp
code for the computational hot spots to various extents.
Starting from the opposite direction \psifour~\cite{Parrish2017} has
gradually introduced a more and more powerful \python interface on top of
their existing \cpp core over the years.

The popularity of the combination of \fortran or a \ccc-like
language in the core and \python as the high-level interface language
can be understood by considering the recent publication of \citet{Sun2017}
about the \pyscf package.
They rationalise the choice of \python as follows:
\begin{itemize}
	\item There is no need to learn a particular domain-specific
		input format.
	\item All language elements from \python are immediately
		available to \eg automatise repetitive calculations
		with loops or similar.
	\item The code is easily extensible beyond what is available
		inside \pyscf, for example to facilitate plotting
		or other kinds of analysis.
	\item Computations can be done interactively,
		which is helpful for testing or debugging.
\end{itemize}
Additionally one should mention,
that \python as a high-productivity, multi-paradigm language
often permits to achieve even complicated tasks with few lines of code,
whilst still staying surprisingly easy to read.
In the context of quantum chemistry
this has the pleasant side effect that a \python script
used for performing calculations and subsequent analysis
is typically not overly lengthy,
but still documents the exact procedure which is followed in a readable manner.
All this comes at pretty much no downside
if \python is combined with
carefully optimised low-level \ccc or \fortran
code in the computational hot spots.
\citet{Sun2017} for example claim that \pyscf is as
fast as any other existing quantum chemistry packages
written solely in \ccc or \fortran.

Another common feature between \pyscf and \psifour
is their modular design inside the package.
They vividly facilitate well-established open standards
like HDF5~\cite{HDF5Manual} or \numpy arrays~\cite{Walt2011}
for data exchange,
such that linking their codes to external projects is easily feasible.
\psifour for example managed to integrate more than 15 external packages
into their framework.
This includes three completely different back ends for computing integrals.
In the case of \pyscf it only took us about a day to link
our \molsturm to the \FCI algorithms of \pyscf
via an interface based on \numpy.
Nevertheless the numerical requirements of Gaussian-type orbitals
are currently hard-coded inside the optimised
\ccc or \cpp parts of both these projects,
such that extending them by other types of basis functions could still be involved.

With respect to supporting a large range of basis function types,
especially the quantum Monte Carlo packages CASINO~\cite{Needs2010}
and QMCPACK~\cite{Kim2012} should be mentioned.
They both allow to start a quantum Monte Carlo calculation
from discretisations of the trial wave function
in terms of {\cGTO}s, {\STO}s, plane-waves or numerical orbitals like splines.
Similarly the packages \CPtK~\cite{Hutter2014}, \ASE~\cite{Larsen2017}
and \gpaw can be employed to perform and post-process computations
using more than one type of basis function.
Both \gpaw and \CPtK even allow to perform calculations
employing hybrid basis sets with a mixture of
Gaussian-type orbitals and plane waves.
To the best of our knowledge,
the design of these packages is, however,
very specific to the particular combinations of basis function type
and method they support.
