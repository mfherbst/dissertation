\section{Related Work}
\newcommand{\psifour}{\texttt{Psi4}\xspace}
\newcommand{\pyquante}{\texttt{pyQuante}\xspace}
\newcommand{\horton}{\texttt{HORTON}\xspace}
\newcommand{\gpaw}{\texttt{GPAW}\xspace}

\label{sec:related}
\todoil{See \url{https://en.wikipedia.org/wiki/List_of_quantum_chemistry_and_solid-state_physics_software}}

\todoil{We should really mention TURBOMOLE here as well}

% Structure by topic and by defining features what we would like to have
% Order by defining feature of molsturm not by program

% Introductory sentence: What are the defining features we would like to solve
% Introduce programs
% Conclusion: All things somewhere, but not all combined in one package

The idea of using a high-level scripting language to implement
a substantial part of a quantum-chemical package has been around
for quite a while.

\psifour\cite{Parrish2017} has gradually introduced a Python layer to its existing
low-level \cpp and \cee code, which allows to access most features of the backend.
The internal data structures of \psifour to represent the matrices and tensors
of quantum-chemical calculations can be easily converted to numpy arrays and back,
thereby simplifying rapid prototyping by the means of the tools available
from scipy and numpy.
\psifour has recently adopted a very modular design based on a set of external
libraries to drive the computation of GTO integrals or DFT functionals.
In fact Psi4 has support for multiple electron repulsion integral libraries,
a fact they call ``unusual''.
\psifour makes vivid use of open standards like HDF5 to facilitate data exchange
and integrates well with a wide range of open-source libraries for Post-HF
methods or other features like solvent models.
Even though Psi4 allows to multiplex between different GTO libraries,
the changing between backends is implemented at a very low level
and the GTO matrix structures are assumed throughout \eg the SCF code.

A slightly different ansatz is chosen by projects like
\pyscf\cite{Sun2017}, \pyquante\todo{cite} and \horton\cite{Verstraelen2017}.
They all implement most of their functionality in high-level
\python and only employ low-level \cpp or \cee code to implement the computational
hot spots.
Especially \pyscf has a very large range of methods implemented, some of which
rely on heavily optimised and parallelised computational cores, which makes the
package ``as efficient as the best existing \cee or \fortran based quantum
chemistry programs''\cite{Sun2017}.

They stress the use of python as the frontend language, since
\begin{itemize}
	\item There is no need to learn a new input file language
	\item All python language features are immediately available
	\item The code is easily extensible beyond what is available
		inside \pyscf
	\item Computations can be done interactively.
\end{itemize}
\todoil{I could subscribe to all thees four.
I think we should nick these ideas for the introduction and add our 5ct on top.}

According to the description of the authors \cite{Sun2017},
one of the design principles of \pyscf is generality.
The use of GTOs and coefficients, however, is currently
hard-coded in the library unfortunately.
Nevertheless the design of the library in general is very flexible from
the user perspective.
For example the interfacing of \molsturm with the full-CI code
from \pyscf was only about a day of work.
\todoil{Maybe this is because I don't understand the code well enough yet}
The internal structure of the code, however,
is comparatively complex,
since a functional-style syntax with little object-oriented paradigms is chosen.
This makes it often very unclear where things are happening.


\pyquante and \horton on the other hand focus more on
simplicity of the code.
They do implement computational hot spots in \cee or \cpp,
but unlike \pyscf have optimised the low-level code
much less with a principle target of readability in mind.
GTOs are also very much hard-coded in both these packages.

Another package, which is mainly implemented in \python is \gpaw \todo{cite}.
\todoil{Note that these guys are around the corner. They are from DTU in Kongens Lyngby ;)}
Mainly GPAW makes use of the projector-augmented wave method
in a DFT context using plane-waves or finite-difference approximations
on real-space uniform grids as the basis function types.
They also support GTOs including amending GPAW basis sets with GTOs,
i.e. hybrid basis sets.
The computationally intensive parts of \gpaw are again implemented in \cee.
\todoil{GPAW is mainly used in the solid state community.}
\todoil{GPAW is focused on DFT}

Another package with support for hybrid basis functions is cp2k \todo{cite}.
They invoke a hybrid representation of the electronic
density in GTOs and plane-waves during their calculations.

\todoil{QMC is some sort of Post-HF method, so maybe it is not that applicable
to talk about it in our paper}
Amongst the packages which have support for a large number of different types
of basis functions are the two quantum Monte Carlo programs CASINO\cite{Needs2010}
and QMCPACK \todo{cite}.
In order to generate the trial-function for the quantum Monte Carlo
computations both these packages can invoke interfaces to a range of 3rd party
program packages
in order to get representations of the trial-function in
GTOs, STOs, Plane-Waves or numerical orbitals like splines as well.
For example CASINO\cite{Needs2010} has interfaces to
GTO programs including  Psi4, CFOUR, TURBOMOLE, GAMESS-US, MolPro and ORCA,
ADF for STOs, Quntum ESPRESSO for plane-waves
and many others.
In the case of QMCPACK \todo{cite} this has lead to a design
which is focused on facilitating data exchange.
Its input file is written in XML and output data is available via HDF5.
Analysis of QMCPACK's results is done using a set of python scripts
facilitating for example matplotlib.

Whilst some of the afforementioned packages
deal with the aspect of a flexible interface for rapid prototyping
and others deal with the combination of multiple basis functions
all of them rely on particular hard-coded backends for 
performing the required linear algebra.
Mostly this is standard packages like numpy
(from \python) or BLAS/LAPACK (from \cee, \fortran or \cpp).
Some of them are CPU parallelised
\todoil{Research which}
QMCPACK has partial support for GPU parallelisation via CUDA.
All packages use strict evaluation for evaluating the linear algebra subexpressions,
which is realised using high-level calls to the linear libraries
mentioned above.
% One should probably mention that the GPU implementation very much seperate and a diffirent
% one from the CPU parallelised

\todoil{Some words about the \lazyten competition?}

% - Related to gint+(sturmint|gaussint|numerics|...)
% -- Does anything exist at all to solve this problem? Have a search!
% -- Some have multi-functions:
% --- CASINO (QMC)
%       -> Gaussian via 3rd party (psi4, cfour, turbomole, gamess-us, molpro, orca, .. )
%       -> PW via 3rd party (quantum expresso, abinit ...)
%       -> splines via 3rd party (quantum expesso)
%       -> STOs via ADF
%       TODO do they have post-hf (only via 3rd party it seems)
%       only does few things by itself.
% --- QMCPACK (QMC) -- tailored for quantum monte-carlo
%       -> Modular
%       -> HDF5 and xml for data exchange
%       -> integrates with
%       -> open source
%       -> quantum espresso for PW
%       -> internally bsplines, GTO, STO, NMO ???
%       -> GTOs from GAMESS-US
%       -> analysis via python and matplotlib
%       -> They do not seem to have traditional post-HF
%
% -- Hybrid basis functions:
% --- cp2k has hybrid gaussians (check)
%      -> mixed representation of matrices / density in GTO and plane-wave basis
% --- bigdft afaik has some hybrid stuff (check)
%      -> DFT expanded in wavelets
%
% - Related to the molsturm project
% -- PySCF
%      -> Large set of methods
%      -> only GTO
%      -> Interface to GTO integral library libcint
%      -> GTO assumptions deep inside the code and at many places due to non-OO design.
%      -> Implementing sturmians probably hard. Needs to be done at many places
%      -> For post-HF sometimes ok, hence used for FCI.
% -- PyQuante2 github
%      -> GTOs
%      -> not much going on
%      -> gaussians very much hard-coded
%      -> density matrix-based SCF algorithm hard-coded
% -- HORTON (Python interface, idea is to be helpful and rapid prototype, only GTO, but different backends)
%      -> Similar motivation than we have
%      -> Work entirely controlled via python (like pyquante, pyscf and us)
%      -> Gaussians assumed inside library
% -- GPAW   (Projector augmented wave, pw, fd, lcao)
%      -> Mostly python
%      -> DFT
%      -> projector-augmented wave
%      -> many people work on it
% -- Psi4
%      -> entirely GTO based
%      -> external libraries for GTOs and DFT
%      -> Modular with rapid prototyping in mind
%      -> Multiple integral libraries !!
%         -> multiplexing at low level. Other kinds of basis functions not supported.
%      -> Psithon: Some weird input language designed around python,
%           which is not python, however
%         -> full interface for Psi4
%         -> internally invokes legacy code
%      -> looks like something to integrate molsturm with some day
%
% - Related to L10
% -- Cyclops tensor framework, pytorch (strict, tensors must live fully in memory, cyclops is restricted to MPI)
% -- Tensorflow, theano, etc. (all strict, not simple to program)
% -- SecondQuantizationAlgebra, FemTo, Tensor Contraction Engine: Code generators from high-level interface.
%
% - Related to gscf? Have a search.

