\chapter{$\RMSOl$ plots for Dunning basis sets}
\label{apx:GaussianUHF}

This appendix shows $\RMSOl$~(see definition \vref{defn:RMSOl})
plots for the \cGTO basis sets
cc-pV5Z and cc-pV6Z%
~\cite{Dunning1989,Woon1993,Wilson1996,VanMourik2000,Prascher2011}
similar to the ones depicted for \CS discretisations
in section \vref{sec:CSconvergenceHF}.
The values were computed \molsturm~\cite{molsturmWeb,molsturmDesign}
using \libint~\cite{Libint2_231} as a back end
for the integral values.
The most notable differences compared to figures \ref{fig:RMSOl_period2}
and \vref{fig:RMSOl_period3}
is that the $\RMSOl$ of lithium and phosphorus
decreases much slower and
that beryllium has some pronounced spikes of larger $\RMSOl$
values for $l=2$ and $l=4$.

\begin{figure}[hp!]
	\centering
	\includeimage{99_appendix/ccpvnz_rmso_period2_vs_l}
	\caption
	[Plot of $\RMSOl$ vs. $l$ for the \HF ground state of the atoms of the 2nd period]
	{
		Plot $\RMSOl$ vs. $l$ for the \HF ground state
		of the atoms of the second period.
		All calculations done with the cc-pV6Z basis set,
		except \ce{Li} and \ce{Be}
		for which at cc-pV5Z was used.
		For \ce{Be} and \ce{Ne} a \RHF procedure was used,
		for the other cases \UHF.
	}
	\label{fig:RMSO_period2_l}
\end{figure}

\begin{figure}[hp!]
	\centering
	\includeimage{99_appendix/ccpvnz_rmso_period3_vs_l}
	\caption
	[Plot of $\RMSOl$ vs. $l$ for the \HF ground state of the atoms of the 3rd period]
	{
		Plot $\RMSOl$ vs. $l$ for the \HF ground state
		of the atoms of the third period.
		All calculations done with the cc-pV6Z basis set,
		except \ce{Na} and \ce{Mg}
		for which at cc-pV5Z was used.
		For \ce{Mg} and \ce{Ar} a \RHF procedure was used,
		for the other cases \UHF.
	}
	\label{fig:RMSO_period3_l}
\end{figure}

%\begin{figure}[hp]
%	\centering
%	\includeimage{99_appendix/ccpvnz_rms_lf_N}
%	\caption{Plot root mean square coefficient vs $l$ and orbital index for N}
%	\label{fig:RMSLF_N_Gauss}
%\end{figure}
%
%\begin{figure}[hp]
%	\centering
%	\includeimage{99_appendix/ccpvnz_rms_lf_C}
%	\caption{Plot root mean square coefficient vs $l$ and orbital index for C}
%	\label{fig:RMSLF_C_Gauss}
%\end{figure}

\begin{figure}[hp!]
	\centering
	\includeimage{99_appendix/ccpvnz_rms_lf_O}
	\caption[
		Root mean square coefficient value
		per angular momentum for nitrogen
	]
	{
		Root mean square coefficient value per
		basis function angular momentum quantum number $l$
		for selected orbitals of oxygen.
		The atom is modelled
		in a cc-pV6Z basis using \UHF.
	}


	\label{fig:RMSLF_O_Gauss}
\end{figure}


