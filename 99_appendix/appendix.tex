\begin{appendix}
\chapter{Estimating the CBS limit for UHF}
\label{apx:CbsLimit}
No UHF literature values found.
Makes sense since RHF is real HF and UHF is ficiticious.
Go into that a little.
Apparently already Hylleraas\cite{Hylleraas1929} has some idea that this could
happen if the spatial components of the spin are not the same any more
says \cite{Mestechkin1981}


We follow \cite{Jensen2005} and use cc-pvNz basis set families
using the builtin procedure of \molsturm.

\chapter{Gaussian $\RMSOl$ plots}
\label{apx:GaussianUHF}

% Here the old version is used of molsturm and libint
Here we show $\RMSOl$ plots for Gaussian cc-pv5z and cc-pv6Z basis sets.

\begin{figure}
	\centering
	\includeimage{99_appendix/ccpvnz_rmso_period2_vs_l}
	\caption{Plot $\RMSOl$ vs $l$ for period 2 atoms. All calculations are cc-pv6Z,
	except lithium and beryllium, which are at cc-pv5z level.}
	\label{fig:RMSO_period2_l}
\end{figure}

\begin{figure}
	\centering
	\includeimage{99_appendix/ccpvnz_rmso_period3_vs_l}
	\caption{Plot $\RMSOl$ vs $l$ for period 3. All calculations are cc-pv6Z,
	except sodium and manganese, which are at cc-pv5z level.}
	\label{fig:RMSO_period3_l}
\end{figure}

\begin{figure}
	\centering
	\includeimage{99_appendix/ccpvnz_rms_lf_N}
	\caption{Plot root mean square coefficient vs $l$ and orbital index for N}
	\label{fig:RMSLF_N}
\end{figure}

\begin{figure}
	\centering
	\includeimage{99_appendix/ccpvnz_rms_lf_C}
	\caption{Plot root mean square coefficient vs $l$ and orbital index for C}
	\label{fig:RMSLF_C}
\end{figure}

\begin{figure}
	\centering
	\includeimage{99_appendix/ccpvnz_rms_lf_O}
	\caption{Plot root mean square coefficient vs $l$ and orbital index for O}
	\label{fig:RMSLF_O}
\end{figure}


\chapter{Computational results of the Coulomb-Sturmian correlation convergence study}
\label{apx:CSCorrelationConvergence}

% TODO Maybe also plot convergence of Be FCI with the basis sets
%      we did computations with.

\begin{table}
	\centering
	\input{Be_hf_mp2_fci_nlm_table.tex}
	\caption{Coulomb-Strumian calculations of the Beryllium atom
		at Hartree-Fock, MP2 and Full-CI level of theory.
		For all calculations $\kexp = 1.985$ was used.
		The heading of the table shows the values for
		$(\nmax, \lmax, \mmax)$ as well as the
		number of basis functions
		in the truncated CS basis.}
	\label{tab:CStruncationEnergies}
\end{table}


The data in this appendix shows the CS HF and MP2 energies
for various nlm values but the same k

%\section{Second period}
\input{CS_HF_MP2_Li_table.tex}
\input{CS_HF_MP2_Be_table.tex}
\input{CS_HF_MP2_B_table.tex}
\input{CS_HF_MP2_C_table.tex}
\input{CS_HF_MP2_N_table.tex}
\input{CS_HF_MP2_O_table.tex}
\input{CS_HF_MP2_F_table.tex}
\input{CS_HF_MP2_Ne_table.tex}

%\section{Third period}
\input{CS_HF_MP2_Na_table.tex}
\input{CS_HF_MP2_Mg_table.tex}
\input{CS_HF_MP2_Si_table.tex}
\input{CS_HF_MP2_Al_table.tex}
\input{CS_HF_MP2_P_table.tex}
\input{CS_HF_MP2_S_table.tex}
\input{CS_HF_MP2_Cl_table.tex}
\input{CS_HF_MP2_Ar_table.tex}

\end{appendix}
