\documentclass[a4paper,10pt,twoside]{MFHbook}

\newcommand{\imageformat}{@IMAGE_FORMAT@}
\newcommand{\gitcommit}{@GIT_COMMIT@}

\usepackage{MFHcolours}
\usepackage{MFHcode}               % Definitions for code
\usepackage{MFHscience}            % Default sciency packages
\usepackage[british]{babel}        % Use british-english conventions
\usepackage{xspace}

% Tikz configuration:
\usetikzlibrary{positioning,arrows,backgrounds,fit,calc}

\usepackage{nomencl}
\usepackage[nomencl]{MFHabbr}      % Default abbreviation commands
\usepackage[book]{MFHtheoremdefs}  % Default theorem definitions

\usepackage{import}                % for relative imports
\usepackage{makeidx}               % a global index
\usepackage[version=3]{mhchem}     % For \ce chemistry typesetting
\usepackage{ctable}                % For some nice-loking tables
\usepackage{todonotes}             % For in-document notes
\usepackage{fancyvrb}              % For formatted verbatim typesetting
\usepackage{xfrac}                 % For some more fancy fractions

\usepackage{xmpincl}
\includexmp{metadata}

%---------------------------------------------------

% Start making a global index
\makeindex

% Include defined commands
%% Programming languages
\newcommand{\python}   {\texttt{python}\xspace}
\newcommand{\cpp}      {\texttt{C++}\xspace}
\newcommand{\ccc}      {\texttt{C}\xspace}
\newcommand{\fortran}  {\texttt{FORTRAN}\xspace}

% 3rd party packages
\newcommand{\adcman}   {\texttt{adcman}\xspace}
\newcommand{\adcc}     {\texttt{adcc}\xspace}
\newcommand{\pyscf}    {\texttt{pyscf}\xspace}
\newcommand{\libint}   {\texttt{libint}\xspace}
\newcommand{\libcint}  {\texttt{libcint}\xspace}
\newcommand{\numpy}    {\texttt{numpy}\xspace}
\newcommand{\dealii}   {\texttt{deal.ii}\xspace}

% our packages and terms
\newcommand{\molsturm} {\texttt{molsturm}\xspace}
\newcommand{\gscf}     {\texttt{gscf}\xspace}
\newcommand{\gint}     {\texttt{gint}\xspace}
\newcommand{\sturmint} {\texttt{sturmint}\xspace}
\newcommand{\lazyten}  {\texttt{lazyten}\xspace}
\newcommand{\krims}    {\texttt{krims}\xspace}
\newcommand{\contract} {\texttt{contract}\xspace}
\newcommand{\update}   {\texttt{update}\xspace}

\newcommand{\contraction}{\textcolor{red}{\contract}}

% Arbitrary hilbert space
\newcommand{\hilbert}{\mathcal{H}}

% Index sets
% One-particle basis functions index set
\newcommand{\Ibas}{\ensuremath \mathcal{I}_\text{bas}}

% Number of one-particle basis functions
\newcommand{\Nbas}{\ensuremath N_\text{bas}}

% Number of SCF orbitals
\newcommand{\Norb}{\ensuremath N_\text{orb}}
\newcommand{\Iorb}{\ensuremath \mathcal{I}_\text{orb}}


% Number of electrons
\newcommand{\Nelec}{\ensuremath N_\text{elec}}
\newcommand{\NelecA}{\Nelec^\alpha}
\newcommand{\NelecB}{\Nelec^\beta}

% Indices of the occupied orbitals
\newcommand{\Ivirt}{\mathcal{I}_\text{virt}}
\newcommand{\Iocc}{\ensuremath \mathcal{I}_\text{occ}}
\newcommand{\IoccA}{\ensuremath \mathcal{I}_\text{occ}^\alpha}
\newcommand{\IoccB}{\ensuremath \mathcal{I}_\text{occ}^\beta}


% Fock operator
\newcommand{\OpFock}{\Op{F}}
\newcommand{\OpFockFull}{\OpFock\left[\{\varphi_i\}_{i \in \Iorb}\right]}

% Convergence of processes
\newcommand{\epsilonconv}{\varepsilon_\text{conv}}

% Misc math
\newcommand{\slot}{\,\cdot\,}
\DeclareMathOperator{\spacespan}{span}


% Transparently switch between pdf and png
\newcommand{\includeimage}[2][]{\expandafter\includegraphics[#1]{#2.\imageformat}}

% Title and subtitle of the thesis
\newcommand{\thesistitle}{Development of a modular quantum-chemistry framework
for the investigation of novel basis functions}
\newcommand{\thesissubtitle}{with an application to Couloumb-Sturmians}
\newcommand{\thesisurl}{https://michael-herbst.com/publications/2018.02_phd.pdf}
\newcommand{\thesisrepo}{https://github.com/mfherbst/dissertation}

% Style of the header of the thesis
\newcommand{\leftheadstyle}{\uppercase}

% Include a quote at the beginning of a chapter
\newcommand{\chapquote}[2]{%
	\begin{minipage}{0.9\textwidth}%
	\begin{quotation}\noindent \textit{#1}\end{quotation}%
	\vspace{0.1em}
	\begin{flushright}--- #2\end{flushright}%
	\end{minipage}%
	\vspace{1em}
}

% Index and term
\newcommand{\newtermmark}{\textbf}
\newcommand{\newterm}[1]{\index{#1}\newtermmark{#1}}
\newcommand{\term}[1]{\index{#1}#1}

% Inline todo shortcut
\newcommand{\todoil}{\todo[inline]}


% Custom hyphenation
\hyphenation{Schrö-dinger}


% Setup the lstlisting style and code typesetting-related setup
\lstset{
	style=python,
	basicstyle={\small\smaller[2]\ttfamily},
	showstringspaces=false,
}

%---------------------------------------------------

\hypersetup{pdftitle={\thesistitle}}
\hypersetup{pdfauthor={Michael F. Herbst}}
\hypersetup{pdfkeywords={Interdisziplinäres Zentrum für wissenschaftliches Rechnen;Ruprecht-Karls-Universität Heidelberg;}}

%---------------------------------------------------

\begin{document}
\begin{titlepage}
\begin{center}

	\Huge \textsc{Inaugural - Dissertation} \\[0.4cm]
	{\Large zur Erlangung der Doktorwürde der \\
	Naturwissenschaftlich-Mathematischen Gesamtfakultät \\
	der Ruprecht-Karls-Universität Heidelberg\\
	}

	\vfill

	{\Huge
	\textbf{\thesistitle}\\
	\vspace{0.5cm}
	\LARGE \thesissubtitle
	}

	\vfill

	{\large  vorgelegt von} \\
	\vspace{1.0cm}
	{\LARGE Michael F. Herbst, {\large \mbox{MA} \mbox{MSci}} }\\[-0.3em]
	{\large aus Grünstadt-Asselheim } \\
	\vspace{1.0cm}
	{\large im Februar 2018} \\

	\vspace{2cm}

	\begin{flushleft}
	\normalsize
		\textbf{Gutachter:} \qquad Prof. Dr. Andreas Dreuw \hfill und \hfill Prof. Dr. Guido Kanschat\\
		\textbf{Datum der mündlichen Prüfung:} \qquad xx.yy.2018\todo{change}
	\end{flushleft}
\end{center}
\end{titlepage}

\pagenumbering{roman}	%Use Roman until after preamble

% Make sure this sits on the *left* page
\clearpage
\ifodd\value{page}\hbox{}\newpage\fi

\thispagestyle{plain}
\
\vfill
\section*{Licensing and redistribution}
\phantomsection
\addcontentsline{toc}{chapter}{Licensing and redistribution}

This work is licensed under the Creative Commons Attribution-ShareAlike 4.0
International License.
To view a copy of this license,
visit \url{http://creativecommons.org/licenses/by-sa/4.0/}.
\begin{center}
	\includeimage[scale=0.7]{cc-by-sa}
\end{center}
An electronic version of this document is available from
\url{\thesisurl}.
If you use any part of my work,
please include a reference to this URL along with my name and email address
(\texttt{info@michael-herbst.com}).

\section*{Source code repository}
\phantomsection
\addcontentsline{toc}{chapter}{Source code repository}
This thesis was generated using \LaTeX, \python and \molsturm.
The full source code is available
on \texttt{github} at \url{\thesisrepo}.
To generate this very document use \gitcurrent.

\newpage
\section*{}
\vspace{6cm}
To God, who made us question and wonder \\
%curious and question \\

\noindent
To the world for everyone to read \\
\todo[inline]{To satisfy their curious needs, get enlightened}

\noindent
To my parents, who have always taken care \\

\noindent
To Carine for whom I will always be there
\vfill

\defineabbr{HF}{HF\xspace}{Hartree-Fock}
\defineabbr{SCF}{SCF\xspace}{Self-consistent field approach}

%
% English
%
\chapter*{Abstract}
\markboth{\leftheadstyle{Abstract}}{}
\addcontentsline{toc}{chapter}{Abstract}

For the calculation of electronic structures of molecules
state-of-the art methods predominantly employ Gaussian-shaped basis functions.
The algorithms employed inside existing code packages
are consequently often highly optimised keeping
only their numerical requirements in mind.
For the investigation of novel approaches
utilising other basis functions, this is an obstacle,
since requirements might differ.
In contrast, this thesis develops the light-weight program package \molsturm,
which is designed in order to support
Hartree-Fock~(\HF) self-consistent field~(\SCF) calculations
based on many basis function types.
Additionally it is demonstrated how \molsturm
can be easily linked to third-party code,
both on the level of the integral libraries as well as the Post-\HF stage,
thus allowing to leverage existing functionality as much as possible.

In order to arrive at this point,
the mathematical background of quantum mechanics
as well as some numerical techniques are reviewed.
Care is taken to emphasise the often overlooked peculiarities when
discretising an infinite-dimensional spectral problem
in order to obtain a finite-dimensional eigenproblem.
Common quantum-chemical methods such as Full-CI
and Hartree-Fock are discussed providing insight
into their mathematical properties.
Different formulations of Hartree-Fock are derived
and appropriate self-consistent field~(\SCF)
solution schemes formulated.

Next discretisation approaches based on four different types of basis functions
are contrasted
both with respect to the computational challenges as well as
their ability to describe the physical features of the wave function.
Next to Slater-type orbitals and contracted Gaussian-type orbitals,
the discussion considers finite elements,
which are piecewise linear polynomials on a real-space grid,
as well as Coulomb-Sturmians,
which are the analytical solution to a Schrödinger-like equation.
An algorithmic approach based on matrix-vector contraction expressions is suggested,
which is able to incorporate the numerical requirements of all cases considered.
It is shown that this ansatz not only allows to formulate
\SCF algorithms in a basis-function independent way,
but furthermore even amounts to improve the theoretically achievable
computational scaling for finite-element-based discretisations
as well as Coulomb-Sturmian-based discretisations.
The adequacy of standard \SCF algorithms with respect to a contraction-based
setting is reviewed
and for the example of the optimal damping algorithm
an appropriate approximate scheme is constructed.

With respect to recent trends in the development of modern computer hardware
the potential of contraction-based approaches is evaluated
and one drawback,
namely the arising unintuitive contraction expressions,
is identified.
To overcome this the data structure of a lazy matrix is introduced,
which is a generalisation to the usual matrix concept,
suitable for encapsulating contraction expressions.
From the user perspective such objects still look like matrices,
such that programming contraction-based algorithms
becomes similarly convenient and intuitive.
An implementation of lazy matrices in the \lazyten linear algebra library
is presented followed by examples
demonstrating the applicability in the context of the Hartree-Fock problem.

Building on top of the aforementioned concepts the design of \molsturm is outlined.
It is shown how a combination of lazy matrices and a contraction-based \SCF scheme
separates the code describing the \SCF procedure
from the code dealing with the basis function type.
It is discussed how this allows to add a new basis function type
to \molsturm by only making code changes in a single integral interface library.
On top of that we demonstrate by the means of examples
how the readily scriptable interface of \molsturm
can be employed to implement and assess novel quantum-chemical methods
or to combine the features of \molsturm with existing third-party packages.

The final part of the thesis discusses an application of \molsturm
towards the investigation of the convergence properties
of Coulomb-Sturmian-based quantum-chemical calculations.
First results for the convergence
of the Hartree-Fock ground-state energies at \HF level
are reported for atoms of the second and the third period
of the periodic table.
Particular emphasis is put on a discussion about the required
maximal angular momentum in order to achieve convergence
of the discretisation of the angular part of the wave function.
Some modifications required for a treatment at correlated level are suggested,
followed by a discussion of the effect of the Coulomb-Sturmian exponent.
A routine for obtaining an optimal exponent is presented
and some optimal exponents for the atoms of the
second and the third period of the periodic table at \HF level are given.
Furthermore the first result of a Coulomb-Sturmian-based excited
states calculation based on the algebraic-diagrammatic construction
scheme for the polarisation propagator is presented.


%
% German
%
\chapter*{Zusammenfassung}
\addcontentsline{toc}{chapter}{Zusammenfassung}
\markboth{\leftheadstyle{Zusammenfassung}}{}
\todo[inline]{TODO}

\begin{adjustwidth*}{-1em}{}
\cleardoublepage
\phantomsection
\addcontentsline{toc}{chapter}{Contents}
%\addcontentsline{toc}{section}{Table of contents}
\renewcommand{\contentsname}{Table of contents}
\tableofcontents

\cleardoublepage
\phantomsection
\addcontentsline{toc}{section}{List of tables}
\listoftables

\cleardoublepage
\phantomsection
\addcontentsline{toc}{section}{List of figures}
\listoffigures
\end{adjustwidth*}

% start nomenclature
\makenomenclature

\phantomsection
\renewcommand{\nomname}{Symbols and conventions}
\markboth{\leftheadstyle{Symbols and conventions}}{}
\addcontentsline{toc}{chapter}{Symbols and conventions}
\printnomenclature[6em]

\todoil{Consider making the abbreviations in a different font,
like Henrik does it in smallcaps.}



\newpage
\pagenumbering{arabic}
\subimport{0_intro/}     {chapter}

\subimport{1_qm/}        {chapter}
\subimport{2_numeigen/}   {chapter}
\subimport{3_qchem/}     {chapter}

\subimport{4_solving_hf/}{chapter}
\subimport{6_contraction_lazy_matrices/}{chapter}
\subimport{8_molsturm/}  {chapter}

\subimport{10_results/}  {chapter}
\subimport{13_conclusion/}{chapter}
\subimport{15_future/}   {chapter}


\subimport{99_appendix/} {appendix}
\newpage
\phantomsection
\bibliographystyle{unsrtnat}
\addcontentsline{toc}{chapter}{Bibliography}
\bibliography{references}

\chapter*{Publications and code contributions}
\addcontentsline{toc}{chapter}{Publications and code contributions}

\section{Own code projects}
\begin{itemize}
	\item \molsturm
	\item \lazyten
	\item \texttt{adcc}: Python frontend for \texttt{adcman} module\todo{cite}
\end{itemize}

\section{Third-party}
Major contribution to Bohrium, namely
\begin{itemize}
	\item Rewrite of \cpp frontend and high-level \cpp interface
\end{itemize}

Minor contributions to a number of other projects:
\begin{itemize}
	\item deal.ii: Bugfixes and typos in FullMatrix
	\item Spack: Added \lazyten to spack repository
	\item clang-tidy: Allowed export of run-clang-tidy results
	\item libint: Simplified build and bug fixes
	\item libcint: Typos and minor improvements
\end{itemize}

\section{Block-courses and lectures}
\begin{itemize}
	\item Bash course
	\item Awk course
\end{itemize}

\section{Journal articles and papers}
\begin{itemize}
	\item None so far
	\item Point at what is coming
\end{itemize}

\chapter*{Acknowledgements}
\addcontentsline{toc}{chapter}{Acknowledgements}
\todo[inline]{Write something nice}



I greatefully acknowledge the computational time at the octuplet cluster at
the eScience group of Køpenhavns Universitet, which was used to carry out
most of the calculations in chapter 8

Support by HGS Mathcomp

Powered by chaos and the ease by which one gets it touch with novel ideas.

\cleardoublepage
\thispagestyle{plain}
\phantomsection
\addcontentsline{toc}{chapter}{Eidesstattliche Versicherung}
\markboth{\leftheadstyle{Eidesstattliche Versicherung}}{}
\section*{Eidesstattliche Versicherung gemäß \S 8 der Promotionsordnung der Naturwissenschaftlich-Mathematischen Gesamtfakultät der Universität Heidelberg}

{ \large
\vspace{2em}

\begin{enumerate}[itemsep=1em]
\item Bei der eingereichten Dissertation zu dem Thema
	\begin{center}
	\textbf{``\thesistitle''}
	\end{center}
	handelt es sich um meine eigenständig erbrachte Leistung.

\item Ich habe nur die angegebenen Quellen und Hilfsmittel benutzt
	und mich keiner unzulässigen Hilfe Dritter bedient.
	Insbesondere habe ich wörtlich oder sinngemäß aus anderen Werken
	übernommene Inhalte als solche kenntlich gemacht.

\item Die Arbeit oder Teile davon habe ich bislang nicht an einer
	Hochschule des In- oder Auslands als Bestandteil einer Prüfungs-
	oder Qualifikationsleistung vorgelegt.

\item Die Richtigkeit der vorstehenden Erklärungen bestätige ich.

\item Die Bedeutung der eidesstattlichen Versicherung und die strafrechtlichen
	Folgen einer unrichtigen oder unvollständigen eidesstattlichen Versicherung
	sind mir bekannt.
\end{enumerate}
\vspace{1.5em}
Ich versichere an Eides statt, dass ich nach bestem Wissen die reine
Wahrheit erklärt und nichts verschwiegen habe.
}
\vspace\fill

\begin{center}
\begin{tabular}{ccccc}
	\hspace{5cm} & \hspace{1cm} & \hspace{5cm} \\
	\cline{1-1}
	\cline{3-3}
	\\
	Ort / Datum& & Unterschrift
\end{tabular}
\end{center}

\cleardoublepage
\phantomsection
\addcontentsline{toc}{chapter}{Index}
%\markboth{\leftheadstyle{Index}}{}
\printindex


\end{document}
