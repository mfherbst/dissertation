\subsection{Contracted Gaussian-type orbitals}
\label{sec:cGTO}

In \citeyear{Boys1950} \citeauthor{Boys1950} suggested to replace
the exponential factor $\exp(- \zeta r)$ in the radial part \eqref{eqn:STOradial}
by a Gaussian factor $\exp(-\alpha r^2)$ instead,
resulting in the so-called Gaussian-type orbitals~({\GTO}s).
Such \GTO basis functions still follow the ansatz radial part times real-valued spherical harmonic
\eqref{eqn:ACproduct}
\begin{equation}
	\varphi^\text{GTO}_\mu(\vec{r}) = R^\text{GTO}_\mu\!\left(\norm{\vec{r} - \vec{R}_\mu}\right)
	\, Y_{l_\mu}^{m_\mu}(\theta, \phi),
	\label{eqn:ACproduct}
\end{equation}
but their radial part is given as
\begin{equation}
	R^\text{GTO}_\mu(r) = N_\mu r^{l_\mu} \exp(-\alpha_\mu r^2)
	\label{eqn:GTOradial}
\end{equation}
with Gaussian exponent $\alpha_\mu$ and normalisation constant
\[
	N_\mu =  \sqrt{\frac{2^{l+2}}{(2l+1)!!}} \, \sqrt[4]{\frac{(2\alpha)^{2l+3}}{\pi}}
\]
This rather \textit{ad hoc} replacement allows to perform the
evaluation of the integrals involved in building the Fock matrix $\mat{F}$
much more efficiently.
Because of the \newterm{Gaussian product theorem}%
~\cite{Boys1950,Szabo1996,Besalu2011},
the product of two Gaussians may be expressed \emph{exactly} as
\[
	R^\text{GTO}_\mu\!\left( \norm{\vec{r} - \vec{R}_\mu}\right)
	\,
	R^\text{GTO}_\nu\!\left( \norm{\vec{r} - \vec{R}_\mu}\right)
	= R^\text{GTO}_\kappa\!\left( \norm{\vec{r} - \vec{R}_\kappa}\right)
\]
where $l_\kappa$, $\alpha_\kappa$ and $\vec{R}_\kappa$ are chosen appropriately.
This allows to evaluate certain integrals
like the \ERI integrals \eqref{eqn:ERI} involving only
$s$-like Gaussian basis functions (where $l_\mu=0$) analytically.
The other  \ERI integrals with higher angular momentum can then be computed by
employing a set of recursion formulas~\cite{Gill1994}.
Similar things hold for the one-electron integrals involved in building
$\mat{T}$ and $\mat{V}_0$,
making the construction of $\mat{F}$ overall much more
feasible for larger basis sets compared to large sets of {\STO}s.

Unfortunately employing such \GTO basis functions implies
that certain physical aspects like the exponential decay
are no longer directly built into the basis set.
More on that later.
Since $\varphi^\text{GTO}_\mu(\vec{r}) \in C^\infty(\R^3, \R)$,
which is a dense subset of $H^1(\R^3, \R)$
this is not \textit{per se} a problem:
The denseness assures that we can still represent every function
from $H^1(\R^3, \R)$ up to arbitrary accuracy
if enough {\GTO}s are used.
Furthermore this ensures that both the \HF problem as well as
the \FCI problem can still be solved
in a Ritz-Galerkin ansatz up to arbitrary accuracy if {\GTO}s are used.
But compared to the {\STO}s the {\GTO}s by themselves span a much smaller subset
of $H^1(\R^3, \R)$,
such that convergence is in general slower and more {\GTO}s are needed
to reach the same accuracy as a calculation employing {\STO}s.

As a remedy \citet{Hehre1969} introduced so-called
\newterm{contracted Gaussian-type orbitals}~(\cGTO),
where the radial part of a basis function $\varphi_\mu$
is expressed as a fixed linear combination of $N_\text{contr}$
\newterm{primitive Gaussians}%
\footnote{Here we follow the usual convention to include the normalisation constant
	inside the contraction coefficients}
\[ R^\text{cGTO}_\mu(r) = r^{l_\mu} \sum_i^{N_\text{contr}} c_{\mu,i} \exp(-\alpha_{\mu,i} r^2). \]
The idea is to get the best out of both worlds:
The easily solvable integrals in terms of primitive {\GTO}s
and the more accurate description of the wave function by using
special sets of \newterm{contraction coefficient}s
$c_{\mu,i}$ and exponents $\alpha_{\mu,i}$,
which are made to yield
a sensible basis set of {\cGTO}s $\{\varphi_\mu\}_{\mu\in\Ibas}$.
Essentially by the means of this trick
one splits the parameter space of the variational problem
\eqref{eqn:HFOptCoeff} or remark \ref{rem:FCI}
into two parts,
where one --- the contraction coefficients ---
is predetermined to fit a large range of problems
and one --- the expansion coefficients in the basis ---
is solved during the actual calculation with respect to a precise
chemical system.

Out of the pragmatic desire to perform molecular calculations
on systems larger than what was feasible with \STO basis sets at that time,
\citet{Hehre1969} initially focused on contracting primitive Gaussians
in a way that they most closely represented a minimal set of \STO functions.
This resulted in the famous STO-$n$G family of basis sets.
Later it was realised that more accurate basis sets could be constructed
by trying to minimise the energy,
which is resulted from an actual \HF or MP2 calculation.
Other strategies include a rigorous construction
considering the convergence of obtained approximations
with respect to the amount of recovered correlation energy
in various types of methods.
Naturally this has lead to a number of different approaches
and thus to a number of different basis set families over the years.

Most of these share common concepts,
of which a few should be briefly sketched here.
For a more in-depth discussion of the construction of Gaussian basis sets
the reader is referred to the excellent reviews by \citet{Hill2013}
and \citet{Jensen2013}.
All basis sets, which are considered state-of-the-art nowadays,
are so-called \newterm{split-valence basis set}s,
which is meant to indicate that multiple contracted Gaussians are available
for describing the valence shell of an atom.
How many are used is typically referred to by the $\zeta$-level,
\eg a double-$\zeta$ basis set contains two \emph{contracted} Gaussians
for each valence orbital,
a triple-$\zeta$ basis set three and so on.
For this letters like \texttt{V}~(for split-valence)
and \texttt{D}, \texttt{T}, \texttt{Q}, \texttt{5}, \ldots
for double, triple, quadruple, quintuple level are usually found.
Notice, that each contracted Gaussian inside these basis sets
is typically in turn made up from multiple primitives.
For a particular basis set family the error decreases going to higher
zeta levels.
For some families like Dunning's correlation-consistent
basis sets~\cite{Dunning1989,Jensen2005}
empirical formulas for estimating the error at a particular basis set
level exist, see appendix \vref{apx:CbsLimit} for details.
To the best of my knowledge none of these results are
backed up by rigorous analysis.
One should, however, mention in this context,
that a recent work by \citet{Bachmayr2014}
has managed to derive error estimates
for Gaussian basis sets made purely out of primitives,
in the relevant $H^1(\R^3,\R)$-norm
such that I would expect certain error bounds
to hold for {\cGTO} basis sets as well.

%
% -------------------------------------------------
%

\begin{figure}
	\centering
	\includeimage{4_solving_hf/relative_error_cgto}
	\caption{Relative error of a selection of
		contracted Gaussian basis sets for
		the solution of a \HF/\FCI calculation
		of the ground state of hydrogen
		with respect to the exact solution.
	}
	\label{fig:RelativeErrorCgto}
\end{figure}
We now consider the accuracy of a \HF or \FCI%
\footnote{Notice: For hydrogen \HF is equivalent to solving the full Schrödinger equation.}
calculation of the hydrogen atom in a range of standard \cGTO basis sets.
\todoil{Complete citations}\noindent
This includes the minimal basis set STO-3g~\cite{Hehre1969},
the double-$\zeta$ basis sets cc-pVDZ~\cite{Dunning1989}
and pc-1~\cite{   },
the quadruple-$\zeta$ basis set pc-3~\cite{   }
as well as the sextuple-$\zeta$ basis set cc-pV6Z~\cite{   }.
The relative error against the exact analytic solution
of the hydrogen ground state can be found in figure \vref{fig:RelativeErrorCgto},
wheres as the local energies can be found in figures \vref{fig:LocalEnergyCgto}
and \vref{fig:LocalEnergyCgtoZoom}.

Relative error:
- cusp obvious
- long range tail failure (different decay)
- both recovered with larger basis sets

Local energy:
- shows wiggles
- region of trust similar
- all basis sets decay to -infinity at zero
- the better the basis the more fluctuations at the origin
- real issue for qmc
- issue for methods/properties which need either the core-like region
  or the long-range region
- partial solution: specially constructed basis sets
- but: no longer black-box
- some issues (like the missing cusp) cannot be corrected fully




\begin{figure}
	\centering
	\includeimage{4_solving_hf/local_energy_cgto}
	\caption{Local energy of contracted Gaussian basis sets for hydrogen.}
	\label{fig:LocalEnergyCgto}
\end{figure}

\begin{figure}
	\centering
	\includeimage{4_solving_hf/local_energy_cgto_zoom}
	\caption{Zoom-in for local energy of contracted Gaussian basis sets for hydrogen.}
	\label{fig:LocalEnergyCgtoZoom}
\end{figure}



Other common modifications include functions of increased angular momentum $l_\mu$,
to mimic the effect of polarisation of the electron cloud,
and diffuse functions,
\ie {\cGTO}s with comparatively small exponents $\alpha_{\mu,i}$
to allow the representation of extended features in the electron density.
The former kind of basis sets are called \newterm{polarised}
and the latter ones \newterm{diffuse} or \newterm{augmented}.



An important point to take away is, however,
that the modelling of certain properties of the electronic structure
like Rydberg-like excited states or the computation of proper electron affinities
requires specific basis sets~\cite{Hill2013,Jensen2013}.
Otherwise such features can be missed.
In this sense {\cGTO}s are not black-box and require some sense of chemical intuition.
Conversely if one already has a clear idea about the electronic structure
of a chemical system a basis set,
which forms a good compromise between cost and accuracy,
can usually be found amongst the off-the-shelf basis set families.


%
% -------------------------------------------------
%



Discuss numerical properties:
integral evaluation comparatively easy
fock matrix structure -> maybe shift that to fe section
dense methods are fine
keeping matrix in memory is fine

\begin{landscape}
\begin{figure}
	\centering
	\includeimage{4_solving_hf/fock_gaussian}
	\caption{Structure of the Fock matrix for a cGTO-based SCF
		for the beryllium atom
		using a pc-2 basis set.
		The three figures show the SCF with a Pulay error
		Frobenius norm of $0.18$, $0.0063$ and $4.1 \cdot 10^{-7}$
		from left to right.}
		% 12 out of 15 rows are diagonal-dominant in each case.
		% Matrix is symmetric
	\label{fig:StructureGaussianFock}
\end{figure}
\end{landscape}

% https://www.math.tu-berlin.de/fachgebiete_ag_modnumdiff/fg_modellierung_simulation_und_optimierung_in_natur_und_ingenieurswissenschaften/v_menue/mitarbeiter/prof_dr_reinhold_schneider/publikationen/


%
% -------------------------------------------------
%



Available in a large range of quantum chemistry programs
\cite{Ahlrichs1989,ORCA,Hutter2014,Sun2017,Verstraelen2017,PyQuante}
\todoil{Cite some more}
