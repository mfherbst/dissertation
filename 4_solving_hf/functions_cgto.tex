\subsection{Contracted Gaussian-type orbitals}
\label{sec:cGTO}

In \citeyear{Boys1950} \citeauthor{Boys1950} suggested to replace
the exponential factor $\exp(- \zeta r)$ in the radial part \eqref{eqn:STOradial}
by a Gaussian factor $\exp(-\alpha r^2)$,
resulting in the so-called Gaussian-type orbitals~({\GTO}s).
Such \GTO basis functions still follow the ansatz radial part times real-valued spherical harmonic
\eqref{eqn:ACproduct}
\begin{equation}
	\varphi^\text{GTO}_\mu(\vec{r}) = R^\text{GTO}_\mu\!\left(\norm{\vec{r} - \vec{R}_\mu}\right)
	\, Y_{l_\mu}^{m_\mu}(\theta, \phi),
	\label{eqn:ACproduct}
\end{equation}
but their radial part is now given as
\begin{equation}
	R^\text{GTO}_\mu(r) = N_\mu r^{l_\mu} \exp(-\alpha_\mu r^2)
	\label{eqn:GTOradial}
\end{equation}
with Gaussian exponent $\alpha_\mu$ and normalisation constant
\[
	N_\mu =  \sqrt{\frac{2^{l+2}}{(2l+1)!!}} \, \sqrt[4]{\frac{(2\alpha)^{2l+3}}{\pi}}
\]
This replacement allows to perform the
evaluation of the integrals involved in building the Fock matrix $\mat{F}$
much more efficiently.
Because of the \newterm{Gaussian product theorem}%
~\cite{Boys1950,Szabo1996,Besalu2011},
the product of two Gaussians may be expressed \emph{exactly} as
\[
	R^\text{GTO}_\mu\!\left( \norm{\vec{r} - \vec{R}_\mu}\right)
	\,
	R^\text{GTO}_\nu\!\left( \norm{\vec{r} - \vec{R}_\mu}\right)
	= R^\text{GTO}_\kappa\!\left( \norm{\vec{r} - \vec{R}_\kappa}\right)
\]
where $l_\kappa$, $\alpha_\kappa$ and $\vec{R}_\kappa$ are chosen appropriately.
With this result the evaluation of certain \ERI integrals \eqref{eqn:ERI}
can be done analytically~\cite{Boys1950}.
An example would be those involving four basis functions with $l_\mu = 0$.
All other \ERI integrals, potentially involving higher angular momentum $l_\mu$,
can be computed from the initial ones employing
a set of recursion formulas~\cite{Gill1994}.
Similar strategies can be found for the one-electron integrals
in order to build $\mat{T}$ and $\mat{V}_0$.
Overall the construction of $\mat{F}$ therefore becomes much more
feasible for larger basis sets of Gaussians compared to large sets of {\STO}s.

Unfortunately certain physical aspects like the exponential decay
or the cusp are no longer directly built into the basis set
if such \GTO basis functions are used.
Since $\varphi^\text{GTO}_\mu(\vec{r}) \in C^\infty(\R^3, \R)$,
which is a dense subset of $H^1(\R^3, \R)$
this is not \textit{per se} a problem:
The denseness ensures that we can still represent every function
from $H^1(\R^3, \R)$ up to arbitrary accuracy if we use enough {\GTO}s.
In other words Ritz-Galerkin ansatz still allows us to
solve problems like \HF or \FCI up to arbitrary accuracy.
Since the {\GTO}s, however,
span only a smaller subset of $H^1(\R^3, \R)$ than the {\STO}s,
we will need more \GTO basis functions to achieve the same accuracy.

As a remedy \citet{Hehre1969} introduced so-called
\newterm{contracted Gaussian-type orbitals}~({\cGTO}s),
where the radial part of a basis function $\varphi_\mu$
is expressed as a fixed linear combination of $N_\text{contr}$
\newterm{primitive Gaussians}%
\footnote{Here we follow the usual convention to include the normalisation constant
	inside the contraction coefficients.}
\[ R^\text{cGTO}_\mu(r) = r^{l_\mu} \sum_i^{N_\text{contr}} c_{\mu,i} \exp(-\alpha_{\mu,i} r^2). \]
The idea is to get the best out of both worlds:
The easily solvable integrals in terms of primitive {\GTO}s
and an accurate description of the wave function by using
special sets of \newterm{contraction coefficient}s
$c_{\mu,i}$ and exponents $\alpha_{\mu,i}$
predetermined to give a good set {\cGTO}s $\{\varphi_\mu\}_{\mu\in\Ibas}$.
By the means of this trick
one is able to effectively split the parameter space of the variational problem
\eqref{eqn:HFOptCoeff} into two parts.
One --- the contraction coefficients ---
is fitted once and for all in order to fit a large range of problems
and another --- the coefficient matrix \eqref{eqn:HFCoeffMatrix} ---
is the search space over which one minimises during the actual calculation.

Out of the pragmatic desire to perform molecular calculations
on systems larger than what was feasible with \STO basis sets at that time,
\citet{Hehre1969} initially focused on contracting primitive Gaussians
in a way that they most closely resembled a particular \STO function.
This resulted in the famous STO-$n$G family of basis sets.
Later it was realised that more accurate basis sets could be constructed
by trying to minimise the energy,
which is resulted from an actual \HF or a MP2 calculation.
Other strategies included a rigorous construction of the basis set
in order to obtain convergence in the amount of recovered correlation energy,
or to be consistent in certain computed properties.
These deviating approaches have lead to a number of
different basis set families over the years,
most of which share common concepts, however.
Our discussion here should remain rather brief.
Interested readers are referred to the excellent reviews by \citet{Hill2013}
and \citet{Jensen2013}.

\defineabbr{CBS}{CBS\xspace}{complete basis set:
The limiting case of a basis set containing an
infinitely large number of functions of a particular type.}
%
All basis sets, which are considered state-of-the-art nowadays,
are so-called \newterm{split-valence basis set}s,
which is meant to indicate that multiple contracted Gaussians are available
for describing the valence shell of an atom.
How many are used is typically referred to by the $\zeta$-level,
\eg a double-$\zeta$ basis set contains two \emph{contracted} Gaussians
for each valence orbital,
a triple-$\zeta$ basis set three and so on.
For this characters like \texttt{D}, \texttt{T}, \texttt{Q}, \texttt{5}, \ldots
--- for double, triple, quadruple, quintuple level ---
may be found in the name of the basis set.
Notice, that each contracted Gaussian inside such basis sets
is typically in turn made up from multiple primitives.
For a particular basis set family the error generally decreases
going to higher zeta levels.
For some families like Dunning's correlation-consistent basis sets~\cite{Dunning1989}
empirical formulas for estimating the error at a particular zeta
level exist~\cite{Jensen2005}.
These results have been used for many years
to estimate properties at the so-called \newterm{complete basis set}~(\CBS) limit,
\ie the theoretical value obtained if an infinitely large basis set of {\cGTO}s
would be employed for the calculation.
See appendix \vref{apx:CbsLimit} for more details.
A recent work by \citet{Bachmayr2014}
provides some mathematical support for such formulas
by rigorously deriving error estimates in the relevant $H^1(\R^3,\R)$-norm.
One should note, however, that these results strictly speaking
only apply to a basis of uncontracted even-tempered {\GTO}s.
The authors point out, however, that a generalisation towards
{\cGTO}s should be possible.

\begin{figure}
	\centering
	\includeimage{4_solving_hf/relative_error_cgto}
	\caption[Relative error in the hydrogen \HF ground state
		for a range of \cGTO basis sets]
	{Relative error in the hydrogen \HF ground state
		employing various \cGTO basis sets~%
		\cite{Hehre1969,Dunning1989,Jensen2001,Wilson1996}.
		The error is plotted against
		the relative distance of electron and proton.
	}
	\label{fig:RelativeErrorCgto}
\end{figure}
A large range of \cGTO basis sets are available nowadays,
which offer a spectrum of compromises between accuracy and computational cost.
Nevertheless some systematic issues related to the non-physical shape
of the primitive {\GTO}s can not be fully accounted for,
even in the largest basis sets.
To illustrate this, consider figure \ref{fig:RelativeErrorCgto}.
In this plot the
relative error of the hydrogen \HF ground state $\Phi_0$
with respect to the exact
electronic ground state $\Psi_{1s}$ \eqref{eqn:FunctionGShydrogen}%
\footnote{For hydrogen \HF is equivalent to solving the full Schrödinger equation.}
\[ \frac{\Psi_0(\vec{r}) - \Psi_{1s}(\vec{r})}{\Psi_{1s}(\vec{r})} \]
is shown at various electron-proton distances.
We show this quantity for a range of standard \cGTO basis sets,
namely the minimal basis set STO-3G~\cite{Hehre1969},
the double-$\zeta$ basis sets cc-pVDZ~\cite{Dunning1989}
and pc-1~\cite{Jensen2001},
the quadruple-$\zeta$ basis set pc-3~\cite{Jensen2001}
as well as the sextuple-$\zeta$ basis set cc-pV6Z~\cite{Wilson1996}.
In each case the error is smallest at intermediate
electron-proton distances, but increases both at the origin
as well as larger distances.
The former feature can be explained due to the faster fall-off
of the Gaussians, $\exp(-\alpha r^2)$,
compared to the exact solution,
which goes as $\exp(-\zeta r)$.
The latter feature originates from the failure of Gaussians
to represent the electron-nuclear cusp.
Larger basis sets like pc-3 or cc-pV6Z amount to recover the
correct decay behaviour as well as the cusp somewhat,
such that the error stays below $0.02 \equiv 2\%$
in the complete inner part of the plot up to distances
of about $7.5$ Bohr.
Eventually all relative errors tend towards $-\infty$ as $r \to \infty$,
however.
This includes the case of pc-3,
where the relative error has a local maximum around $r = 10$
and then follows a downhill slope similar to the pc-1 error around $r = 5$.
Overall the plots agree with the rule of thumb
that results become more accurate at higher zeta levels:
Both the relative errors get smaller as well as the region
where the wave function is well-represented becomes larger
as we proceed from STO-3G to double-$\zeta$ and higher $\zeta$ levels.

\begin{figure}[p]
	\centering
	\includeimage{4_solving_hf/local_energy_cgto}
	\caption[Local energy of the hydrogen \HF ground state]{
		Local energy $E_L(\vec{r})$ of the hydrogen atom \HF ground state
		obtained using the indicated contracted Gaussian basis sets%
		~\cite{Hehre1969,Dunning1989,Jensen2001,Wilson1996}.
		$E_L(\vec{r})$ is plotted against the relative distance
		of electron and nucleus.
	}
	\label{fig:LocalEnergyCgto}
\end{figure}

\begin{figure}[p]
	\centering
	\includeimage{4_solving_hf/local_energy_cgto_zoom}
	\caption[Local energy of the hydrogen \HF ground state (magnified)]{
		Magnified version of figure \vref{fig:LocalEnergyCgto}
		around the origin.
	}
	\label{fig:LocalEnergyCgtoZoom}
\end{figure}

In figures \ref{fig:LocalEnergyCgto} and \ref{fig:LocalEnergyCgtoZoom}
the local energies \eqref{eqn:LocalEnergy}
of the aforementioned basis sets are depicted
--- again as a function of relative distance.
These plots not only diverge to $-\infty$ as $r \to \infty$,
but at the origin as well, see particularly figure \ref{fig:LocalEnergyCgtoZoom}.
At intermediate electron-proton distances
the local energies of all basis sets
fluctuate around the exact ground state energy of $0.5$,
where the amplitude of the fluctuations are lowest for cc-pV6Z and pc-3.
Recall, that the local energy is related to the relative residual error
and that ideally it should be a constant.
At intermediate distances, where the fluctuations are small,
the \HF ground state thus agrees well with the exact ground state.
Unsurprisingly the parts of figure \ref{fig:LocalEnergyCgto},
where $E_L(\vec{r})$ is almost constant,
agree roughly the parts of figure \ref{fig:RelativeErrorCgto},
where the relative error is small.
Similarly the wrongful decay behaviour of the \cGTO solutions
is observed in both the plot of the relative error as well as
the local energy plot.
The most notable discrepancy of both error metrics
is close to the nucleus, see \ref{fig:LocalEnergyCgtoZoom}.
Whilst the relative error gets smaller and smaller for the larger
pc-3 and cc-pV6Z basis sets close to the core as well,
these show rather vivid fluctuations in $E_L(\vec{r})$ as $r \to 0$.
Eventually they diverge to $-\infty$ exactly like the result
employing any other basis set.
In other words, whilst these basis sets amount to produce
a very good description of the ground state from distances around
$0.5$ Bohrs up to $7.5$ Bohrs,
they fail to do so close to the core in a rather misbehaving manner.
Since the relative error is small,
the issue is not that the function value
of the exact ground state is missed.
Much rather the culprit is the gradient of the approximated
\HF ground state.

This can be explained following \cite{Ma2005}.
The potential term in the local energy \eqref{eqn:LocalEnergy} diverges
as $-Z_A/r$ close to the nucleus $A$,
such that the kinetic energy term inside \eqref{eqn:LocalEnergy}
needs to provide an equal and opposite
divergence in order for the resulting local energy to be constant.
Since the gradient of every {\cGTO} basis functions is zero at the origin,
so is the gradient of the final \HF ground state,
thus the local energy goes to $-\infty$.
Furthermore the gradient of each individual primitive Gaussian
goes to zero at a different rate
depending on its exponent $\alpha_{\mu,i}$.
Overall this leads to an overcompensation
of the diverging potential in the kinetic term at some points
and an undercompensation at others,
giving rise to the oscillatory behaviour.
This oscillatory feature close to the nucleus is well-known
in the quantum Monte-Carlo community~\cite{Foulkes2001,Ma2005},
since it can lead to problems when sampling the local energy,
especially in diffusion Monte-Carlo.

In \HF, \DFT and Post-\HF methods
the failure of the {\cGTO}s to represent the nuclear cusp
or the long-range behaviour is typically only an issue,
if either parts of the wave functions are especially important
for a particular property.
Examples would be the determination of Rydberg-like excited states,
the computation electron affinities,
the computation of X-ray absorption spectroscopy
or the computation of nuclear-magnetic resonance properties.
In such cases specific basis sets
are required~\cite{Hill2013,Jensen2013},
which include further \cGTO basis functions
to either sample the core region or the long-range tail more accurately.
If such basis sets are not employed
it may happen that the desired features are completely missed or described very inaccurately.
In this sense {\cGTO}s are not fully black-box and
picking a reasonable basis set for a particular problem
usually requires some idea of the electronic structure already.
On the other hand if such special basis sets are employed,
one may encounter numerical issues
due to rather large overlap matrix elements $\braket{\varphi_\mu}{\varphi_\nu}_1$
and almost linearly dependent basis functions $\varphi_\mu$.
This problem arises
since firstly
the full set of possible \cGTO basis functions is over-complete
and secondly
since the special basis sets for sampling the core and tail region
contain {\cGTO}s of similar exponents,
which makes over-completeness manifest more strongly.

%\begin{landscape}
\begin{sidewaysfigure}
	\centering
	\includeimage{4_solving_hf/fock_gaussian}
	\caption[Structure of the Fock matrix for a \cGTO-based \SCF]
		{Structure of the Fock matrix for a \cGTO-based \SCF
		of the beryllium atom
		in a pc-2~\cite{Jensen2007} basis set.
		The three figures show the matrix
		at different convergence stages during the \SCF.
		From left to right the Pulay error
		Frobenius norm is $0.18$, $0.0063$ and $4.1 \cdot 10^{-7}$.
		The coloring depends on the absolute logarithm
		of the respective Fock matrix entry.
	}
	\label{fig:StructureGaussianFock}
\end{sidewaysfigure}
%\end{landscape}
We already mentioned that the Gaussian product theorem
allows to evaluate the integrals
required for building the Fock matrix $\mat{F}$
rather efficiently.
Furthermore the resulting Fock matrix is comparatively small:
Even for systems with hundreds of atoms one typically only needs
in the order of thousands of basis functions.
In other words both building the Fock matrix
as well as diagonalising it can be performed using dense methods%
\todoil{put citations to those methods in footnote 9}\noindent
\footnote{%
	A standard procedure would be to reduce the matrix to
	tridiagonal form using Householder reflections
	and then use Cuppen's divide and conquer
	or an algorithm based on relatively robust representations.
}.
Ignoring the basis sets suffering from over-completeness for a second
the numerical structure of a \cGTO-based Fock matrix
is rather advantageous in most cases.
Figure \vref{fig:StructureGaussianFock}, for example,
shows some Fock matrices from an \SCF calculation
of a beryllium atom
in the pc-2~\cite{Jensen2007} basis set.
The matrices are taken as snap shots during the \SCF procedure.
From left to right the Pulay error Frobenius norm decreases
from $0.18$ to $0.0063$ and finally $4.1\E{-7}$.
As the error gets smaller the matrix becomes more and more diagonal
as the off-diagonal elements in the occupied-virtual block of the
Fock matrix all have to vanish%
\footnote{This is another way equivalent to \eqref{eqn:PulayError}
to express \SCF convergence.}.
Already the leftmost matrix is almost diagonal-dominant with 12
out of 15 rows $\mu$
satisfying the condition for \newterm{diagonal-dominance}
\[
	\sum_{\nu=1}^{\Nbas} F_{\mu\nu} < 2 F_{\mu\mu}.
\]
For larger systems, the structure generally gets
worse due to interactions between the atoms,
but if a proper description of the core region or the tail is not required
$\mat{F}$ stays numerically manageable and almost diagonal-dominant.
This allows to additionally employ
iterative eigensolver methods like Davidson's method~(see section \vref{sec:Davidson})
to efficiently obtain eigenpairs of the Fock matrix if not all are required.

Since for most cases in chemistry the core-like region
and the long-range tail are not extremely important
both obtaining and diagonalising Fock matrices
from a \cGTO discretisation is straightforward.
Even though {\cGTO}s are physically not the most sensible basis function type
this has historically made \cGTO-based methods
the most predominant approach to describe
a chemical systems within decent accuracy
such that these methods are now implemented in countless quantum-chemistry packages.
In light of this it is remarkable,
that only in \citeyear{Bachmayr2014}
error bounds were rigorously derived by \citet{Bachmayr2014} for some special
kinds of Gaussian basis sets
and these results are not employed on a daily basis.

I personally consider it a problem
that selecting the \cGTO basis set for performing a calculation is often done
based on habit rather than proper scientific evaluation
in light of the chemical problem at hand.
I do not dare to estimate how many published works
draw conclusions based on calculations,
where a more appropriately chosen basis
could have given a totally different picture.
