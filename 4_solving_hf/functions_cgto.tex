\subsection{Contracted Gaussian-type orbitals}
\label{sec:cGTO}







Applying the STOs to molecules turned out to be much more
challenging due to the demanding structure of the electron repulsion integrals~(ERIs)
for such types of basis functions.
\todoil{I cannot download that paper so im not too sure what is in it}
In \citeyear{Boys1950} \citeauthor{Boys1950} realised that employing
atom-centered Gaussian type basis functions~(GTOs) leads to integrals
which are much easier to evaluate numerically.





\todo[inline,caption={}]{
	\begin{itemize}
		\item Brief rationale
		\item Problems
		\item E.g. Local energy => what happens if $\psi = 0$
	\end{itemize}
}
% Convention:
%    Use \chi_\mu for primitives


\begin{figure}
	\centering
	\includeimage{4_solving_hf/local_energy_cgto}
	\caption{Local energy of contracted Gaussian basis sets for hydrogen.}
	\label{fig:LocalEnergyCgto}
\end{figure}

\begin{figure}
	\centering
	\includeimage{4_solving_hf/local_energy_cgto_zoom}
	\caption{Zoom-in for local energy of contracted Gaussian basis sets for hydrogen.}
	\label{fig:LocalEnergyCgtoZoom}
\end{figure}

\begin{figure}
	\centering
	\includeimage{4_solving_hf/relative_error_cgto}
	\caption{Relative error of a selection of contracted Gaussian basis sets
		for hydrogen.}
	\label{fig:LocalEnergyCgto}
\end{figure}


\begin{landscape}
\begin{figure}
	\centering
	\includeimage{4_solving_hf/fock_gaussian}
	\caption{Structure of the Fock matrix for a cGTO-based SCF
		for the beryllium atom
		using a pc-2 basis set.
		The three figures show the SCF with a Pulay error
		Frobenius norm of $0.18$, $0.0063$ and $4.1 \cdot 10^{-7}$
		from left to right.}
		% 12 out of 15 rows are diagonal-dominant in each case.
		% Matrix is symmetric
	\label{fig:StructureGaussianFock}
\end{figure}
\end{landscape}
