\section{Standard \SCF algorithms}
\label{sec:SCFAlgorithms}

Pick up on the point raised about the \contraction-based
methods in the previous section
and explain why for them typically
coefficent-based \SCF schemes are best.



% TODO Maybe talk about basis functions first, because then it is more
% clear why we focus so much on coefficient-based stuff


Notice, that many \SCF algorithms only consider the necessary condition \eqref{eqn:PulayError},
but do not check the sufficient conditions as well.

% general remarks -> follow cances, bris book

This is just a representative selection of algorithms,
which is by no means complete or exhaustive.

\subsection{Roothaan repeated diagonalisation}
\label{sec:RoothaanRepeatedDiag}

\subsection{Level-shifting modification}

\subsection{Optimal damping algorithm}
\label{sec:ODA}
$\tilde{\mat{F}}^{(n)}$ gradient of $\mathcal{E}_D(\mat{D})$
wrt. $\mat{D}$~\cite{Lions1988,Cances2000}.
Can define~\cite{Lions1988,Cances2000}
\[
	\mat{C}^{(n+1)} = \arginf_{\mat{C} \in \mathcal{C}}
\left\{ \tr \tilde{\mat{F}}^{(n)} \mat{C} \tp{\mat{C}} \right\} \]

\subsection{Truncated optimal damping algorithm}

\subsection{Direct inversion of the iterative subspace}

Standard commutator DIIS by Pulay

Other ways~\cite{Shepard2007}
LSIIS (least-squares commutator DIIS)~\cite{Li2016}

\subsection{Geometric direct minimisation}

\subsection{Second-order \SCF algorithms}

\subsection{Combinations of algorithms}
