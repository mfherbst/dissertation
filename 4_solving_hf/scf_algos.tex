\section{Self-consistent field algorithms}
\label{sec:SCFAlgorithms}

In this section we want to discuss a few standard self-consistent field algorithms
in the light of the various types of basis functions we discussed in the previous section.
Even though it is my hope that the selection of algorithms discussed here
is representative,
the vast number of methods,
which has been developed over the years,
makes it impossible to be exhaustive.

Most \SCF algorithms are designed only with a \cGTO-based
discretisation of the \HF and Kohn-Sham \DFT problem in mind.
The deviating numerical properties of the finite-element method
or a \CS-based discretisation
therefore often call for minor modifications of the schemes.
For example both finite elements as well as Coulomb-Sturmians
favour \contraction-based methods
due to the better scaling of equations like \eqref{eqn:ExchangeApply}
and \eqref{eqn:ApplicationKcs}
compared to building the full matrix.
Therefore the Fock matrix might not be built in memory any more,
which implies that a linear combination of Fock matrices
cannot be computed in memory either.
This does not imply that \SCF schemes
which form linear combinations of Fock matrices
are completely ruled out,
but they might become less favourable compared to other schemes.

On the other hand, in \FE-based approaches all quantities
which scale quadratically in $\Nbas$ cannot be stored in memory.
This applies not only to the iterated Fock matrix $\mat{F}^{(n)}$,
but to the density matrix $\mat{D}^{(n)} \in \R^{\Nbas\times \Nbas}$ as well.
Even though clever low-rank approximation methods
like hierarchical matrices~\cite{Hackbusch1999,Hackbusch2002,Grasedyck2003,Hackbusch2015}
or tensor decomposition methods~\cite{Kolda2009,Oseledets2011,%
Schollwoeck2011,Khoromskaia2015}
could reduce the memory footprint of the density matrix,
this work will try to indicate ways
by which building the density matrix in an \SCF can be avoided.
Naturally this implies
a focus on coefficient-based \SCF schemes as well,
where the number of iterated parameters
--- the coefficient matrix $\mat{C} \in \R^{\Nbas\times\Norb}$ ---
scales only linearly in $\Nbas$.
Furthermore coefficient-based \SCF schemes have the advantage
that iterating the density matrix
destroys the possibility to follow the optimal contraction scheme
for the application of $\mat{K}$ in \CS-based methods.
See equation \eqref{eqn:ApplicationKcs} for details.

It was already pointed out in section \vref{sec:OverviewSCF}
that focusing on coefficient-based schemes
is hardly a restriction in terms of the number of possible approaches,
since coefficient-based and density-matrix-based schemes can be interconverted,
at least approximately.
For the case of the optimal damping algorithm~(ODA)~\cite{Cances2000a}
a modification will be suggested
in section \ref{sec:tODA}
to bring this method to the coefficient-based setting.

Most of the \SCF algorithms we will consider here
only converge the \HF equations \eqref{eqn:HFiterated}
until the Pulay error \eqref{eqn:PulayError} vanishes
following our general description in remark \vref{rem:SCFcoeff}.
Regarding the \HF optimisation problem \eqref{eqn:HFOptCoeff}
this is only the necessary condition for a stationary point on the Stiefel
manifold $\mathcal{C}$.
Only some \SCF algorithms,
termed \newterm{second-order self-consistent field method}\textbf{s},
take at least approximate measures
to ensure that the stationary point they find is a minimum.
They are briefly considered in section \ref{sec:SOSCF}.

\subsection{Roothaan repeated diagonalisation}
\label{sec:RoothaanRepeatedDiag}

Roothaan's repeated diagonalisation~\cite{Roothaan1951}
approach to the \HF problem \eqref{eqn:HFiterated} is by far the simplest.
In the formalism of remark \vref{rem:SCFcoeff}
this algorithm can be described by building the next Fock matrix
$\tilde{\mat{F}}^{(n)}$ only by considering
the current occupied coefficients $\mat{C}^{(n)}$,
\ie $\tilde{\mat{F}}^{(n)} = \matFnfull$.
The two-step iteration procedure of figure \ref{fig:RoothaanODA}a
on page \pageref{fig:RoothaanODA}results.

Even though Roothaan's algorithm already works for a few simple cases,
it is far from being reliable.
For example one can show~\cite{Cances2000,Cances2000b} that it
either converges to a stationary point of the
discretised \HF problem \eqref{eqn:HFOptCoeff}
or alternatively it oscillates between two states,
where none of them is a stationary point of \eqref{eqn:HFOptCoeff}.
In practice it depends both on the system
as well as the basis set which of these cases occurs.
Furthermore there is no guarantee that the resulting stationary point found by
Roothaan's algorithm is the \HF ground state.
All these cases can already be observed for \HF calculations
on atoms of the first three periods of the periodic table~\cite{Cances2000}.

\subsection{Level-shifting modification}
If one uses essentially the same \SCF scheme as figure \ref{fig:RoothaanODA}a
but instead diagonalises the matrix
\[ \tilde{\mat{F}}^{(n)} = \matFnfull
	- b \mat{S} \, \mat{C}^{(n)}\!\tp{\left(\mat{C}^{(n)}\right)} \, \mat{S} \]
where $b > 0$,
already a much better convergence is achieved.
This modification is called \newterm{level shifting}%
~\cite{Saunders1973,Guest1974},
where $b$ is the \textbf{level-shifting parameter},
typically chosen in the range between $0.1$ and $0.5$.
Effectively this modification increases the energy gap between occupied and virtual
orbital energies.
To see this, let us consider the converged case, where
\[ \mat{F} \mat{C}_F = \mat{S} \mat{C}_F \mat{E} \]
exactly and let us partition the full coefficient matrix%
\footnote{We assume \RHF here and furthermore only consider the $\alpha$ block.
	For \UHF the analysis is exactly the same with
	the relevant equations just replicated in $\alpha$ and $\beta$ block.}
\[
	\mat{C}_F = \mm{\mat{C} & \mat{C}_\text{virt}}
\]
into occupied and virtual parts.
Now let $\tilde{\mat{F}} = \mat{F} - b \mat{S} \, \mat{C}\tp{\mat{C}} \, \mat{S}$
such that
\begin{align*}
	\tilde{\mat{F}} \mat{C}_F
	&= \left( \mat{F} - b \mat{S} \, \mat{C}\tp{\mat{C}} \, \mat{S} \right) \mat{C}_F \\
	&= \mat{S} \mat{C}_F \mat{E} - b \mat{S} \mm{
		\mat{C}\tp{\mat{C}} \mat{S} \mat{C} &
		\mat{C}\tp{\mat{C}} \mat{S} \mat{C}_\text{virt}
	} \\
	&= \mat{S} \mat{C}_F \mat{E} - b \mat{S} \mm{ \mat{C} & 0 } \\
	&= \mat{S} \mat{C}_F \mat{E} + \mat{S} \mat{C}_F \mm{ -b  & 0 } \\
	&= \mat{S} \mat{C}_F \tilde{\mat{E}}
\end{align*}
where
\[ \tilde{\mat{E}} = \diag\left(\varepsilon_1 - b, \varepsilon_2 - b, \ldots,
	\varepsilon_{\Nelec} - b, \varepsilon_{\Nelec+1}, \ldots, \varepsilon_{\Norb}\right).
\]
In other words the virtual orbitals are unaffected
whereas the occupied orbitals are shifted downwards in energy by an amount $b$.

The effect of this is that coupling between both orbital spaces is reduced,
which tends to lead to faster convergence
especially if the gap between $\varepsilon_{\Nelec}$ and $\varepsilon_{\Nelec+1}$ is small.
This empirical observation is backed up by a more sophisticated
mathematical analysis by \citet{Cances2000b}.
Their result shows that for sufficiently large $b$,
the level-shifted Roothaan procedure is guaranteed to converge to a stationary
point of the \HF problem \eqref{eqn:HFOptCoeff}.
They also provide an expression for the lower bound of $b$.
In this manner convergence to a stationary point
can be forced even for cases where the original \HF equations \eqref{eqn:HFMO}
have no solution (like the negative ions with $N > 2Z + M$).
In such a case the result is no physical ground state, however.

One can show~\cite{Saunders1973}
that the level-shifting modification is mathematically equivalent
to another modification of Roothaan's repeated diagonalisation,
called \newterm{damping}.
In this procedure one chooses a \newterm{damping factor} $0 < \alpha < 1$
and sets
\begin{equation}
	\tilde{\mat{F}}^{(n)} = (1-\alpha) \tilde{\mat{F}}^{(n-1)} + \alpha \matFnfull,
	\label{eqn:FockDamping}
\end{equation}
such that the new Fock matrix to diagonalise contains still a share
of the old Fock matrix.

\begin{figure}
	\centering
	\includeimage{4_solving_hf/roothaan_oda}
	% TODO OPTIONAL
	% Add step which indicates where error is computed and convergence flagged
	\todoil{Check equation number}
	\caption[Roothaan repeated diagonalisation and optimal damping algorithm]
	{Schematic of Roothaan repeated diagonalisation and optimal damping algorithm.
	The step which updates the Fock matrix is highlighted in red
	and the step which updates the coefficients is highlighted in blue.
	}
	\label{fig:RoothaanODA}
\end{figure}
\defineabbr{ODA}{ODA\xspace}{Optimal damping algorithm}
\subsection{Optimal damping algorithm}
\label{sec:ODA}
The optimal damping algorithm~(\ODA) was proposed by \citet{Cances2000a}
based on their analysis of the Roothaan algorithm
including the level-shifting modification.

In unmodified form~\cite{Cances2000,Cances2000a} it is a density-matrix-based \SCF algorithm.
Starting from an initial density $\tilde{\mat{D}}^{(0)} = \mat{D}^{(0)}$,
the procedure is roughly~(compare figure \ref{fig:RoothaanODA}a)
for $n=1, 2, 3, \ldots$
\begin{itemize}
	\item Build the Fock matrix
	\begin{equation}
	\tilde{\mat{F}}^{(n-1)} = \mat{F}\left[ \tilde{\mat{D}}^{(n-1)} \right]
		\label{eqn:ODAupFock}
	\end{equation}
	and diagonalise it to obtain the new coefficient $\mat{C}_F^{(n)}$.
	Form the new density $\mat{D}^{(n)}$ according to the Aufbau principle from these
	as
	\[ \mat{D}^{(n)} = \mat{C}^{(n)} \tp{\left( \mat{C}^{(n)} \right)}. \]
	%
	\item Evaluate the Pulay error $\mat{e}^{(n)}$ \eqref{eqn:PulayError}
		from $\mat{F}\left[ \tilde{\mat{D}}^{(n)} \right]$
		and $\mat{D}^{(n)}$.
		End the process if $\norm{\mat{e}^{(n)}}_\text{frob} \leq \epsilonconv$.
	%
	\item Solve the line search problem
		\begin{equation}
			\tilde{\mat{D}}^{(n+1)}
			= \arginf_{\tilde{\mat{D}} \in
			\text{Seg}\left[\tilde{\mat{D}}^{(n)}, \mat{D}^{(n+1)}\right]}
			\mathcal{E}_D^\text{HF}[\tilde{\mat{D}}]
			\label{eqn:ODAupDens}
		\end{equation}
		where
		\[
			\text{Seg}\left[\mat{D}_1, \mat{D}_2\right]
				= \left\{ (1-\lambda) \mat{D}_1 + \lambda \mat{D}_2 \, \Big|\,
				\lambda \in [0,1] \right\}
		\]
		is a line segment of density matrices
		and the energy functional $\mathcal{E}_D^\text{HF}$
		is defined as in \eqref{eqn:HFEnergyFunctionalDens}.
		Repeat the process thereafter.
\end{itemize}
One can show~\cite{Cances2000} that the \ODA
\emph{always} converges to a local minimum of \eqref{eqn:HFOptDens}.

The remaining question to complete the picture of the \ODA
from a computational point of view
is to find a way to obtain the minimal density $\tilde{\mat{D}}^{(n+1)}$.
First notice that in general the density matrix segment
\[\text{Seg}\left[\mat{D}_1, \mat{D}_2\right] \not\subset \mathcal{P}\]
even if $\mat{D}_1, \mat{D}_2 \in \mathcal{P}$.
Much rather this line segment is fully contained only in a superset
$\tilde{\mathcal{P}} \supset \mathcal{P}$,
where we relax the constraint $\mat{D}^2 = \mat{D}$ to%
\footnote{Let $\mat{A}, \mat{B} \in \R^{n\times n}$,
then $\mat{A} \leq \mat{B}
\Leftrightarrow \forall \vec{x} \in \R^n \
\tp{\vec{x}}\mat{A} \vec{x} \leq \tp{\vec{x}}\mat{B}\vec{x}$}
$\mat{D}^2 \leq \mat{D}$.
See \cite{Cances2000} for details.
For ease of notation let us define
%
\newcommand{\Gfct}[1]{  \mat{G}\left[ #1 \right]   }
\newcommand{\Pet}[1]{\textcolor{dkred}{\tilde{\mat{D}}^{(#1)}}}
\newcommand{\Pe}[1]{\textcolor{dkblue}{\mat{D}^{(#1)}}}
\newcommand{\Fot}[1]{\textcolor{dkred}{\tilde{\mat{F}}^{(#1)}}}
\newcommand{\Fo}[1]{\textcolor{dkblue}{\mat{F}^{(#1)}}}
\newcommand{\la}[1]{\lambda^{(#1)}}
\newcommand{\ila}[1]{\left( 1 - \la{#1} \right)}
\newcommand{\EHFD}[1]{\mathcal{E}_D^\text{HF}\left[#1\right]}
%
\newcommand{\Pn}{\Pet{n}}
\newcommand{\Pnn}{\Pe{n+1}}
\begin{align*}
	E_1 \left[\mat{D}\right] &\equiv \tr\left( \mat{T} \mat{D} + \mat{V}_0 \mat{D} \right), \\
	\mat{G}\left[ \mat{D} \right] &\equiv
		\mat{F}\left[ \mat{D} \right] + \mat{K}\left[ \mat{D} \right] \\
\intertext{and}
	E_2 \left[\mat{D}\right] &\equiv
		\frac12 \tr\left( \mat{D} \, \mat{G}\left[ \mat{D} \right] \right).
\end{align*}
For all matrices $\mat{D}_1, \mat{D}_2 \in \tilde{\mathcal{P}}$
we can show the properties~\cite{Cances2000a}
\begin{align}
	\label{eqn:TermGsymmetry}
	\tr \left( \mat{D}_1 \mat{G}\left[ \mat{D}_2 \right] \right)
		&= \tr \left( \mat{D}_2 \mat{G}\left[ \mat{D}_1 \right] \right) \\
	\label{eqn:GmatTrace}
	\tr\left( \mat{F}\left[ \mat{D}_1 \right] \mat{D}_2 \right)
	&= E_1[\mat{D}_2] + \tr\left( \mat{D}_1 \Gfct{\mat{D}_2} \right)
\end{align}
These imply for $E_2$ and arbitrary $\alpha, \beta \in \R$
\begin{equation}
	\begin{aligned}
	E_2[\alpha \mat{D}_1 + \beta\mat{D}_2]
	&= \frac 1 2 \tr\left( \alpha^2  \mat{D}_1 \Gfct{\mat{D_1}} \right)
		+ \frac 1 2 \tr\left( \alpha\beta \mat{D_1} \Gfct{\mat{D}_2} \right) \\
		\nonumber
		&\hspace{30pt}
		+ \frac 1 2 \tr\left( \alpha\beta \mat{D}_2 \Gfct{\mat{D_1}} \right)
		+ \frac 1 2 \tr\left( \beta^2 \mat{D}_2 \Gfct{\mat{D}_2} \right) \\
	 &\stackrel{\eqref{eqn:GmatTrace}}{=}
	 \alpha^2 E_2[\mat{D_1}] + \beta^2 E_2[\mat{D}_2]
	 + \alpha\beta \tr\left( \mat{D_1} \Gfct{\mat{D}_2} \right),
	\end{aligned}
	\label{eqn:HFE2}
\end{equation}
whereas $E_1$ is linear
\begin{equation}
	E_1[\alpha \mat{D}_1 +\beta \mat{D}_2] = \alpha E_1[\mat{D}_1] + \beta E_1[\mat{D}_2].
	\label{eqn:HFE1}
\end{equation}
These results allow to expand
the \HF energy for a member $\Pet{n+1}$
of the density matrix segment
$\text{Seg}\left[\Pet{n}, \Pe{n+1}\right]$
as
\begin{equation}
\begin{aligned}
	\EHFD{\Pet{n+1}} &= \EHFD{\ila{n+1} \Pet{n} + \la{n+1} \Pe{n+1}} \\
	&= \EHFD{\Pet{n} + \la{n+1} \left( \Pe{n+1}  - \Pet{n} \right)} \\
	&= E_1\left[ \Pet{n} + \la{n+1} \left( \Pe{n+1}  - \Pet{n} \right) \right] \\
		&\hspace{50pt}+E_2\left[\Pet{n} + \la{n+1} \left( \Pe{n+1}  - \Pet{n} \right) \right] \\
	&= E_1[\Pet{n}] + \la{n+1} E_1\left[\Pe{n+1}  - \Pet{n}\right] + E_2[\Pet{n}]\\
		&\hspace{50pt} + \la{n+1} \tr \left( \Pet{n} \Gfct{\Pe{n+1}-\Pet{n}} \right) \\
		&\hspace{50pt}
		+ \left( \la{n+1} \right)^2 E_2[\Pe{n}-\Pet{n}] \\
	&= \EHFD{\Pet{n}}
	+ \la{n+1} \underbrace{\tr\left( \Pet{n} \mat{F} \left[ \Pe{n+1}-\Pet{n} \right] \right)}_{=s} \\
	&\hspace{50pt}
	+ \left( \la{n+1} \right)^2 \underbrace{E_2[\Pe{n}-\Pet{n}]}_{=c} \\
	&= \EHFD{ \Pet{n} } + \la{n+1} s + \left( \la{n+1} \right)^2 c
\end{aligned}
\label{eqn:ODAquadratic}
\end{equation}
The coefficients $s$ and $c$ can alternatively be written as
\begin{equation}
\begin{aligned}
	s &= \tr\left( \Fot{n} \left( \Pe{n+1} - \Pet{n} \right) \right) \\
	&= \tr \left( \Fot{n} \Pe{n+1} \right) - E_\text{HF}[\Pet{n}] - E_2 [\Pet{n}] \\
	&= \tr \left( \Fot{n} \Pe{n+1} \right) - E_1[\Pet{n}] - 2 E_2 [\Pet{n}]  \\
\end{aligned}
\label{eqn:ODAs}
\end{equation}
and\footnote{Note that the original paper \cite{Cances2000a} uses a deviating formalism %
which causes an extra factor of $2$ to appear in their expression for $c$.}
\begin{equation}
\begin{aligned}
	c &= E_2\left[ \Pe{n+1} - \Pet{n} \right] \\
	&\stackrel{\eqref{eqn:HFE2}}{=}
		E_2 [\Pe{n+1}] - \tr\left( \Gfct{\Pet{n}} \Pe{n+1} \right) + E_2 [\Pet{n}] \\
	&= E_2 [\Pe{n+1}] - \tr \left( \Fot{n} \Pe{n+1} \right) + E_1[\Pe{n+1}] + E_2 [\Pet{n}]
\end{aligned}
\label{eqn:ODAc}
\end{equation}
Now the stationary point along the density matrix segment
can be determined by differentiating \eqref{eqn:ODAquadratic} resulting in
\begin{align*}
	\frac{\partial \EHFD{\Pet{n+1}}}{\partial \la{n+1}}
	&= s + 2 \la{n+1} c
	&&\text{and}&
	\frac{\partial^2 \EHFD{\Pet{n+1}}}{\partial \left(\la{n+1}\right)^2} &= 2c
\end{align*}
Due to $E_2[\mat{D}] \geq 0$~\cite{Cances2000a} for all $\mat{D} \in \tilde{\mathcal{P}}$
one easily deduces $c \geq 0$,
such that the stationary point of the above expression is always a minimum.
Since $\la{n+1} \in [0,1]$ the minimiser is
\begin{equation}
	\label{eqn:ODALambdaOpt}
	\la{n+1}_\text{min} = \left\{
	\begin{array}{cl}
		1 & \text{if $2c \leq -s$} \\
		- \frac{s}{2c} & \text{else}
	\end{array}
	\right.,
\end{equation}
where the cases $c=0$ and $s=0$ have been ignored, since they only occur at convergence.
This closes the missing link and allows to implement a \ODA
in as a density-matrix-based \SCF.

Let $\alpha, \beta \in \R$ and $\mat{D}_1, \mat{D}_2 \in \tilde{\mathcal{P}}$.
Since $\matFfullD = \mat{T} + \mat{V}_0 + \matJfullD + \matKfullD$
and the two-electron terms are linear in the density matrix, we have
\begin{equation}
	\mat{F}\!\left[\alpha \mat{D}_1 + \beta\mat{D}_2\right]
	= \alpha \mat{F}\!\left[\mat{D}_1\right] + \beta \mat{F}\!\left[\mat{D}_2\right]
	\label{eqn:FockLinearCombination}
\end{equation}
iff $\alpha + \beta = 1$. Defining
\begin{align*}
	\Fot{n} &\equiv  \mat{F}\left[ \Pet{n} \right] &
	\Fo{n} &\equiv \mat{F}\left[ \Pe{n} \right]
\end{align*}
this allows to rewrite \eqref{eqn:ODAupFock} as
\begin{equation}
	\label{eqn:ODAdamping}
	\Fot{n}
	= \mat{F}\left[ \ila{n} \Pet{n-1} + \la{n} \Pe{n} \right]
	= \ila{n} \Fot{n-1} + \la{n} \Fo{n},
\end{equation}
where the ``$\text{min}$'' subscripts were dropped.
Comparing with equation \eqref{eqn:FockDamping}
one can identify with $\la{n}$ the damping factor $\alpha$.
Since $\la{n}$ is optimal in the sense of minimising the energy along the line segment
spanned by $\Pe{n}$ and $\Pet{n-1}$,
the \ODA can be described by repetitively finding the optimal damping
parameter from \SCF step to \SCF step.
Notice that its construction guarantees that the \SCF energy will always decrease.
It is hence guaranteed to converge to a local minimum of the \HF problem
\eqref{eqn:HFOptDens}~\cite{Cances2000,Cances2000a}.
The \ODA is only a particularly simple example from a whole family
of density-matrix-based \SCF algorithms called relaxed constraints algorithms,
which are discussed in detail in \cite{Cances2000a}.

\noindent
Using \eqref{eqn:ODAdamping} one can show by induction that
\begin{align}
	\label{eqn:ODAFock}
	\Fot{n} &= \sum_{j = 0}^n \Fo{j} \la{j} \prod_{i=j+1}^n \ila{i}, \\
	\label{eqn:ODADens}
	\Pet{n} &= \sum_{j = 0}^n \Pe{j} \la{j} \prod_{i=j+1}^n \ila{i},
\end{align}
where we set $\la{0} \equiv 1$. Since
\begin{align*}
	\Fo{j} &= \matFnfull \\
	\Pe{j} &= \mat{C}^{(j)} \tp{\left(\mat{C}^{(j)}\right)} \\
\end{align*}
these results in theory allow to express the complete \ODA
in terms of the coefficients
such that expressions like \eqref{eqn:ExchangeApply}
or \eqref{eqn:ApplicationKcs} could be used for a \FE-based
or a \CS-based discretisation respectively.

In practice this is usually not a fruitful approach for two reasons.
Firstly it requires to store a growing list of coefficients,
namely one for each \SCF step.
Especially for a \FE approach this becomes increasingly costly in terms of memory.
Secondly for a \contraction-based ansatz
we especially want to avoid storing the Fock matrices $\Fo{j}$
in favour of \contraction expressions like \eqref{eqn:ExchangeApply}
and \eqref{eqn:ApplicationKcs}.
In other words each application of $\Fot{n}$ to a vector $\vec{x}$
would need to be performed by first computing $\Fo{j} \vec{x}$ for each $j$
and then adding the results.
This procedure is roughly $n$ times as expensive as a single apply.
Even though the \contraction expressions formally scale better,
the increasing number of times they need to be invoked
should make this ansatz rather expensive.

Overall the \ODA is very suitable for {\cGTO} and \CS-based discretisations,
since for these density-matrix-based \SCF schemes are fine.
However, this algorithm is not suitable for solving the \HF problem
with a \FE-based discretisation without further modifications.

\defineabbr{tODA}{tODA\xspace}{Truncated optimal damping algorithm}
\subsection{Truncated optimal damping algorithm}
\label{sec:tODA}

Let us again consider \eqref{eqn:ODAFock}.
Due to $\la{i} \in [0,1]$ the Fock matrix prefactor
\begin{equation}
	\label{eqn:ODAFockCoeff}
	\lambda^{(j)} \prod_{i=j+1}^{n} \left(1- \lambda^{(i)}\right) \in [0,1]
\end{equation}
is a product of factors,
which are all between $0$ and $1$.
Therefore this prefactor may become
rather small for small values of $j$ as $n$ increases.
In other words in the later \SCF steps the $\Fo{j}$ terms
which were produced at the beginning of the \SCF procedure
may be accompanied by a small prefactor and hence can at some point
be neglected in \eqref{eqn:ODAFock}.
This is the justification for the truncated optimal damping algorithm~(\tODA),
which approximates the \ODA by artificially restricting the number
of terms in \eqref{eqn:ODAFock} to the $m$ most recently
obtained Fock matrices.
If we define
\[
	j_0(n) \equiv n-m+1
\]
this allows to write the approximated sums as
\begin{align}
	\label{eqn:ODAFockApprox}
	\Fot{n} &= \frac{1}{\la{j_0(n)}} \sum_{j = j_0(n)}^n \Fo{j} \la{j} \prod_{i=j+1}^n \ila{i}, \\
\intertext{and analogously for the density matrices}
	\label{eqn:ODADensApprox}
	\Pet{n} &= \frac{1}{\la{j_0(n)}} \sum_{j = j_0(n)}^n \Pe{j} \la{j} \prod_{i=j+1}^n \ila{i}.
\end{align}
The factor $1/\la{j_0(n)}$ is required to make sure that
the Fock matrix prefactors sum to $1$,
\ie to make sure that the condition for
the linear combination of Fock matrices
\eqref{eqn:FockLinearCombination} is fulfilled.

The simplest case of this class of approximations is $m=1$.
This implies $j_0(n) = n$ such that \eqref{eqn:ODAFockApprox}
and \eqref{eqn:ODADensApprox} simplify to read
\begin{align*}
	\Fot{n} &= \Fo{n} & \Pet{n} &= \Pe{n}
\end{align*}
In other words this 2-step \tODA is equivalent to an \textit{adhoc} modification
of the exact ODA where we replace
$\tilde{\mat{D}}^{(n)}$ by $\mat{D}^{(n)}$,
the density of the previous \SCF step.
Taking this into account the expressions \eqref{eqn:ODAs} and \eqref{eqn:ODAc}
may be written as
\begin{align}
	\label{eqn:tODAs}
	\nonumber
	s &= \tr \left( \Fot{n} \Pe{n+1} \right) - E_1[\Pet{n}] - 2 E_2 [\Pet{n}] \\
	  &= \tr \left( \tp{\left(\mat{C}^{(n+1)}\right)} \Fo{n} \mat{C}^{(n+1)} \right)
	  - E_1\!\left[\mat{C}^{(n)} \tp{\left(\mat{C}^{(n)}\right)}\right]
	  - 2 E_2\! \left[\mat{C}^{(n)} \tp{\left(\mat{C}^{(n)}\right)}\right] \\
\intertext{and}
	\label{eqn:tODAc}
	\nonumber
c &= E_2[\Pe{n+1}] - \tr \left( \Fot{n} \Pe{n+1} \right) + E_1[\Pe{n+1}]
		+ E_2 [\Pet{n}] \\
	\nonumber
	&= E_2\!\left[\mat{C}^{(n+1)} \tp{\left(\mat{C}^{(n+1)}\right)}\right]
		- \tr \left( \tp{\left(\mat{C}^{(n+1)}\right)} \Fo{n} \mat{C}^{(n+1)} \right) \\
		&\hspace{50pt}
		+ E_1\!\left[\mat{C}^{(n+1)} \tp{\left(\mat{C}^{(n+1)}\right)}\right]
		+ E_2\!\left[\mat{C}^{(n)} \tp{\left(\mat{C}^{(n)}\right)}\right]
\end{align}
In contrast to the exact \ODA
this yields a coefficient-based \SCF algorithm.
Starting from an initial set of coefficients $\mat{C}^{(0)}$
with corresponding initial Fock matrix $\Fot{0} =
\mat{F}\!\!\left[\mat{C}^{(0)}\!\tp{\left(\mat{C}^{(0)}\right)}\right]$
we proceed for $n=1,2,3,\ldots$ as follows.
\begin{itemize}
	\item Diagonalise $\Fot{n-1}$ in order to obtain coefficients
		$\mat{C}_F^{(n)}$.
	\item According to the Aufbau principle select $\mat{C}^{(n)}$
		and build $\Fo{n} = \matFnfull$.
	\item Evaluate the Pulay error $\mat{e}^{(n)}$ \eqref{eqn:PulayError}
		and end the process if $\norm{\mat{e}^{(n)}}_\text{frob} \leq \epsilonconv$.
	\item Compute $s$, $c$ and $\lambda^{(n)}$
		according to \eqref{eqn:tODAs}, \eqref{eqn:tODAc}
		and \eqref{eqn:ODALambdaOpt}.
	\item Set
		\[ \Fot{n} = \ila{n} \Fo{n-1} + \la{n} \Fo{n} \]
		and repeat.
\end{itemize}
In this process one only needs the history of two Fock matrices $\Fo{n-1}$ and
$\Fo{n}$, such that $\Fot{n}$ can be applied when needed.
This in turn implies that only the coefficient matrices $\mat{C}^{(n-1)}$
and $\mat{C}^{(n)}$ are required, such that $\Fo{n}$ and $\Fo{n-1}$
can be applied whenever needed.

Compared to the Roothaan algorithm~(see section \ref{sec:RoothaanRepeatedDiag})
the \tODA only roughly doubles
the cost of each diagonalisation, since two Fock matrices need to be applied.
Additionally one needs to evaluate the trace
\[
	\tr \left( \tp{\left(\mat{C}^{(n+1)}\right)} \Fo{n} \mat{C}^{(n+1)} \right)
\]
and compute the energies
$E_1[\mat{D}^{(n)}]$ and $E_2[\mat{D}^{(n)}]$ in order to obtain $c$ and $s$
for each iteration.
The former step costs about as much as a single matrix-vector product
and the latter is usually done during the \SCF anyways
to display the progress to the user,
thus representing no extra cost.

Even though about twice as expensive as the Roothaan algorithm
if a \contraction-based \SCF is performed,
the advantage of the \tODA is
that it automatically finds the damping coefficient $\la{n}$,
which reduces the energy at each iteration as much as possible.
This amounts to break the oscillatory behaviour
of the standard Roothaan repeated diagonalisation scheme
in a slightly improved manner than the default damping
or level-shifting modifications.

One should mention, however, that the \tODA
does not inherit all of the nice mathematical properties from the \ODA.
For example it is no longer guaranteed that the \tODA
converges to a minimum of the \HF problem \eqref{eqn:HFOptCoeff}%
~\cite{CancesODAprivate}.
Especially close to convergence it may for example happen,
that $\lambda^{(n)} \not\in [0,1]$ since both $c$ and $s$
become rather small, thus $2c \leq s$ ill-defined.
One can get around this by explicitly setting $\lambda^{(n)} = 1$ in the cases,
where $\abs{c}$ and $\abs{s}$ become small.
The \tODA is thus best used in the initial \SCF steps in order to
effectively prevent the Roothaan oscillations from happening.

\subsection{Direct inversion in the iterative subspace}
\label{sec:DIIS}
\defineabbr{DIIS}{DIIS\xspace}{Direct inversion in the iterative subspace}

In his celebrated \citeyear{Pulay1982} paper \citeauthor{Pulay1982}
not only introduced the aforementioned Pulay error \eqref{eqn:PulayError},
but also improved upon his previously introduced
\SCF convergence acceleration scheme~\cite{Pulay1980}.
This effort resulted in the procedure now widely known
by the term \newterm{direct inversion in the iterative subspace}~(\DIIS).
In his variant of the DIIS procedure the next Fock matrix was found
as a linear combination
\begin{equation}
	\tilde{\mat{F}}^{(n)} = \sum_{i=0}^{m-1} c_i \mat{F}^{(n-i)}
	\label{eqn:DIISfock}
\end{equation}
of Fock matrices from the $m$ most recent \SCF steps, \ie
\[ \mat{F}^{(j)}
	= \mat{F}\!\!\left[\mat{C}^{(j)}\!\tp{\left(\mat{C}^{(j)}\right)}\right].
\]
The coefficients $\{c_i\}_{i=1,\ldots,m}$ are to be determined such that
the norm of the corresponding linear combination of Pulay errors
\begin{equation}
	f_\text{DIIS}(c_0, c_1, \ldots, c_{m-1})
	= \norm{\sum_{i=0}^{m-1} c_i \mat{e}^{(n-i)}}_\text{frob}^2 
	= \sum_{i=0}^{m-1} \sum_{j=0}^{m-1} c_i c_j \tr\left( \mat{e}^{(n-i)} \mat{e}^{(n-j)}  \right)
\end{equation}
is minimal.
Defining a real-symmetric matrix $\mat{B} \in \R^{m\times m}$ with elements
\begin{equation}
	B_{ij} = \tr\left( \mat{e}^{(n-i)} \mat{e}^{(n-j)} \right)
	\label{eqn:BMatElem}
\end{equation}
we can alternatively write
\begin{equation}
	f_\text{DIIS}(\vec{c}) = \tp{\vec{c}} \mat{B} \vec{c}.
	\label{eqn:DIISobjective}
\end{equation}
In agreement with what was discussed in equation \eqref{eqn:FockLinearCombination},
we need to additionally impose the constraint
\[
	\sum_{i=0}^{m-1} c_i = 1,
\]
such that the resulting Fock matrix $\tilde{\mat{F}}^{(n)}$ is physically sensible.
In other words Pulay's \DIIS scheme can be expressed
as the quadratic programming problem
\[
	\vec{c} = \argmin\left\{
		\tp{\vec{c}} \mat{B} \vec{c}
		\, \middle| \,
		\sum_{i=0}^{m-1} c_i = 1
	\right\},
\]
which has corresponding Euler-Lagrange equations
\begin{equation}
	\mm{
		\mat{B} & \vec{1} \\
		\tp{\vec{1}} & 0
	}
	\mm{\vec{c} \\ \lambda}
	= \mm{\vec{0} \\ 1},
	\label{eqn:DIISlinsys}
\end{equation}
where $\vec{1}, \vec{0} \in \R^m$ are column vectors of $m$ ones and $m$ zeros
and $\lambda$ is the Lagrange multiplier corresponding to the constraint
$\sum_{i=0}^{m-1} c_i = 1$.
Typically one takes $m$ between $2$ and $10$,
such that the linear system \eqref{eqn:DIISlinsys} can be solved by an iterative
method rather fast.

There are a few issues,
which typically occur close to self-consistency,
where the error matrices $\{\mat{e}^{(n-i)}\}_{i=0,\ldots,m-1}$
will be almost identical.
This causes multiple rows in $\mat{B}$ to be extremely similar
and thus gives rise to an ill-conditioned linear system \eqref{eqn:DIISlinsys}.
There are a couple of remedies typically used in practice~\cite{Kudin2002}.
For example one may drop the Fock matrices $\mat{F}^{(n-i)}$
where the coupling to the most recent Fock matrix $\mat{F}^{(n)}$,
\ie the matrix element $B_{0i}$ is smaller than a certain threshold.
Alternatively one may artificially bias the lowest-energy solution,
by multiplying all other diagonal entries $B_{ii}$ by a penalty
factor slightly larger than $1$.
Last but not least one may always drop the oldest Fock matrix
$\mat{F}^{(n-m+1)}$ if an ill-conditioned linear system is detected.
Alternatively one can remove linear dependencies by a singular-value
decomposition of $\mat{B}$.
Another point worth noting is that the \DIIS procedure is an \emph{extrapolation}
technique.
In other words there is no guarantee,
that the Fock matrix $\tilde{\mat{F}}^{(n)}$ is of any physical
or mathematical significance.
It could lead into totally the wrong direction causing the \SCF
procedure to eventually diverge.

Since there is no reason to build the density matrix
in Pulay's \DIIS procedure
it is suitable for both density-matrix-based as well as coefficient-based \SCF settings.
Moreover given that the $m$ coefficients $\{ \mat{C}^{(n-i)} \}_{i=0,\ldots,m-1}$
are stored,
the Fock matrix terms of \eqref{eqn:DIISfock}
can be applied using expressions like \eqref{eqn:ExchangeApply}
or \eqref{eqn:ApplicationKcs} without any problems,
making the \DIIS suitable for a \contraction-based \SCF as well.

Let us summarise the procedure in a \contraction-based \SCF.
Start from an initial set of occupied coefficients $\mat{C}^{(0)} \in \mathcal{C}$.
Set $B_{00} = 1$ and run for $n=1, 2, 3, \ldots$
\begin{itemize}
	\item Use the overlaps $\mat B$ to setup and solve \eqref{eqn:DIISlinsys}
		for the new set of \DIIS coefficients $\vec{c}$.
	\item Build the Fock matrix $\tilde{\mat{F}}^{(n)}$
		according to \eqref{eqn:DIISfock}.
		In this process skip coefficients $c_i$ below a certain threshold
		in order to save some matrix-vector products when
		$\tilde{\mat{F}}$ is contracted with trial vectors during
		the diagonalisation.
		If one entry $c_i$ is very large, say $>10$,
		then only keep this matrix in the expression.
		In all cases be sure to renormalise,
		such that all coefficients still sum to $1$.
	\item Diagonalise $\tilde{\mat{F}}$
		to obtain $\mat{C}_F^{(n)}$.
		Select $\mat{C}^{(n)}$ by the Aufbau principle.
	%
	\item Evaluate the Pulay error $\mat{e}^{(n)}$ \eqref{eqn:PulayError}
		from $\mat{C}^{(n)}$ and
		$\mat{F}\left[\mat{C}^{(n)} \tp{\left(\mat{C}^{(n)}\right)} \right]$.
		End the process if $\norm{\mat{e}^{(n)}}_\text{frob} \leq \epsilonconv$.
	%
	\item Calculate the new error overlaps
		$B_{0i} = \braket{\mat{e}^{(n)}}{\mat{e}^{(n-i)}}$.
		Note that only one row of $\mat{B}$ has to be calculated,
		since the others can be kept from the previous \SCF iteration.
		Drop coefficients, which are beyond the $m$ Fock matrices to keep
		and repeat the process thereafter.
\end{itemize}
The Pulay \DIIS scheme outlined here is rather general
and has been frequently applied to problems
other than solving the \HF or Kohn-Sham equations~\cite{Hamilton1986}.
For example it can be applied to accelerate
the convergence of fixed-point problems
as they occur for example in coupled-cluster theory,
see section \vref{sec:CC}.
In the optimisation community the \DIIS technique is known as
\newterm{Anderson acceleration}\cite{Anderson1965} in this context.

%$\tilde{\mat{F}}^{(n)}$ gradient of $\mathcal{E}_D(\mat{D})$
%wrt. $\mat{D}$~\cite{Lions1988,Cances2000}.
%Can define~\cite{Lions1988,Cances2000}
%\[
%	\mat{C}^{(n+1)} = \arginf_{\mat{C} \in \mathcal{C}}
%\left\{ \tr \tilde{\mat{F}}^{(n)} \mat{C} \tp{\mat{C}} \right\} \]
Last but not least one should mention that a few
improvements to the \DIIS have been suggested recently.
This includes the \textbf{energy DIIS}~\cite{Kudin2002},
which effectively interpolates between densities resulting
from the \ODA accelerating \ODA convergence
whilst showing mathematically highly desirable properties.
\citet{Shepard2007} have suggested ways to improve the \DIIS
by reformulating the original problem into a least squares
or a linear least squares problem.
The augmented Roothaan-Hall \DIIS~\cite{Hoest2008} is another take
to yield a linear-scaling method with improved convergence.
The
\textbf{least-squares commutator in the Iterative Subspace}~\cite{Li2016} approach
can also be seen as a variant of the \DIIS
trying to correct some of its issues.

\subsection{Second-order self-consistent-field algorithms}
\label{sec:SOSCF}

To conclude our review of selected \SCF schemes
this section will briefly touch upon so-called \newterm{second-order \SCF algorithms}.
Generally these methods try to go beyond incorporating
gradient information of the \HF minimisation problem \eqref{eqn:HFOptCoeff}.
Next to solving the \HF equations \eqref{eqn:HFDiscreteEquations}
these methods thus incorporate the  Hessian
of the energy functional \eqref{eqn:HFEnergyFunctionalCoeff}
with respect to the parameters $\mat{C}$ as well.
The aim is both to achieve faster convergence
and to ensure that convergence is not to any stationary state on $\mathcal{C}$,
but instead to a true \SCF minimum.
Since forming the Hessian of \eqref{eqn:HFOptCoeff} can become rather expensive,
many approaches use approximate Hessians instead.
Typically it is observed
that convergence only becomes quadratic
once the \SCF procedure is already reasonably
close to the minimiser of \eqref{eqn:HFOptCoeff}.

One of the first approaches, which fall in this category,
is the \textbf{quadratically-convergent \SCF}~(QCSCF)~\cite{Bacskay1981}.
This approach minimises the \SCF energy by finding the minimum of a
configuration interaction singles-doubles expansion
based on the current set of \SCF orbitals.
Similar to most methods discussed in this section,
QCSCF employs normalised basis functions $\{\varphi_\mu\}_{\mu\in\Ibas}$.
These have the advantage, that the occupied \SCF coefficients $\mat{C}^{(n)}$
can be alternatively parametrised~\cite{Helgaker2013} as
\[ \mat{C}^{(n)} = \mat{C}^{(0)} \mat{U}^{(n)} = \mat{C}^{(0)} \exp(-\mat{K}^{(n)}), \]
where $\mat{U}^{(n)} \in \R^{\Norb \times \Norb}$ is a unitary rotation matrix
and $\mat{K}^{(n)}$ is an anti-Hermitian matrix.
Further details about the orbital rotation ansatz for \SCF methods
can be found in \cite{Helgaker2013}.

\begin{figure}
	\centering
	\includeimage{4_solving_hf/gdm}
	\caption[Geometric direct minimisation algorithm]%
	{Schematic of the geometric direct minimisation algorithm.
	The step which updates the Fock matrix is highlighted in red
	and the step which updates the coefficients is highlighted in blue.
	}
	\label{fig:GeometricDirectMinimisation}
\end{figure}
Related to the idea of an orbital rotation \SCF
is the \textbf{geometric direct minimisation} method~\cite{Voorhis2002},
see figure \vref{fig:GeometricDirectMinimisation}.
This approach tries to directly
optimise the coefficients $\mat{C}_{(n)}$ in the sense of \eqref{eqn:HFOptCoeff}
on the Stiefel manifold $\mathcal{C}$.
The algorithm determines the energy gradient $\mat{g}^{(n)}$
of the current coefficient set $\mat{C}^{(n)}$.
Then it follows the geodesic defined by $\mat{g}^{(n)}$
to find a unitary rotation matrix $\mat{U}$
which defines the new set $\mat{C}^{(n+1)} \mat{U}$.
The step size is determined by a Newton-Raphson-like step,
where the Hessian is constructed
approximately using the Broyden-Fletcher-Goldfarb-Shanno update scheme.

Last but not least one should mention recent linear scaling \SCF approaches,
for example the one by \citet{Salek2007}
as well as the augmented Roothaan-Hall method~\cite{Hoest2008}.
Both of these can be seen as approximate second-order \SCF methods,
which try to directly minimise the coefficients as well.

\subsection{Combining self-consistent field algorithms}
In the previous sections we mentioned quite a few approaches
to solve the \HF minimisation problem using a self-consistent field ansatz.
Needless to say that different algorithms tend to perform best in different cases.
For this reason in practice often a mixture of methods
is employed in order to guarantee fast and reliable convergence.
This section represents my own judgement of the situation
and give some suggestions based on my own experience.
Hardly any of this is resulted from any kind of proper
scientific evaluation%
\footnote{Unfortunately I am not aware of a work,
which properly compares the large range of \SCF algorithms
with another.
Most papers only compare to the Pulay \DIIS.}
and should therefore not be taken as a final answer,
much rather as a guideline.

In the beginning of the procedure \ODA or \tODA work great,
since they essentially direct the coefficients reliably into the right direction,
breaking the oscillatory behaviour of the plain Roothaan algorithm.
The energy \DIIS can be seen as an accelerated improvement of those methods,
which is recommendable for \cGTO-based discretisations
as the initial \SCF algorithm in my point of view.

For the intermediate steps a Pulay \DIIS shows typically a faster
convergence than the energy \DIIS~\cite{Kudin2002}.
This can be rationalised by considering the conditions
on the coefficients for the linear combination of Fock matrices.
In the Pulay \DIIS these conditions are much laxer
compared to the energy \DIIS%
\footnote{The Pulay \DIIS \emph{extrapolates},
whereas the energy \DIIS \emph{interpolates}.},
making it easier to explore the \SCF manifold
and search for directions
which lead to nearby stationary points.

Close to convergence the \DIIS becomes numerically more unstable,
but conversely
second-order \SCF schemes like QCSCF now show the fastest and most reliable
convergence to the \SCF minimum.
These should be considered in the final \SCF steps in order to obtain
a highly-accurate \SCF minimum after the \DIIS.
