\section{Self-consistent field algorithms}
\label{sec:SCFAlgorithms}

In this section we want to discuss a few standard self-consistent field algorithms
in the light of the various types of basis functions we discussed in the previous section.
Even though it is my hope that the selection of algorithms discussed here
is representative,
the vast number of methods,
which has been developed over the years,
makes it impossible to be exhaustive.

Most \SCF algorithms are designed only with a \cGTO-based
discretisation of the \HF and Kohn-Sham \DFT problem in mind.
The deviating numerical properties of the finite-element method
or a \CS-based discretisation,
therefore often call for minor modifications of the schemes.
For example both finite elements as well as Coulomb-Sturmians
favour \contract-based methods
due to the better scaling of equations like \eqref{eqn:ExchangeApply}
and \eqref{eqn:ApplicationKcs}
compared to building the full matrix.
Therefore the Fock matrix might not be built in memory any more,
which implies that a linear combination of Fock matrices
cannot be computed in memory either.
This does not imply that \SCF schemes,
which form linear combinations of Fock matrices
are completely ruled out,
but they might become less favourable compared to other schemes.
See section \ref{sec:DIIS} below for details.

On the other hand in \FE-based approaches all quantities,
which scale quadratically in $\Nbas$ cannot be stored in memory.
This applies not only to the iterated Fock matrix $\mat{F}^{(n)}$,
but to the density matrix $\mat{D}^{(n)} \in \R^{\Nbas\times \Nbas}$ as well.
Even though clever low-rank approximation methods
like hierarchical matrices~\cite{Hackbusch1999,Hackbusch2002,Grasedyck2003,Hackbusch2015}
or tensor decomposition methods
\todo{cite}
could reduce the memory footprint of the density matrix,
this work will try to indicate ways
by which building the density matrix in an \SCF can be avoided.
Naturally this implies
a focus on coefficient-based \SCF schemes as well,
where the number of iterated parameters
--- the coefficient matrix $\mat{C} \in \R^{\Nbas\times\Norb}$ ---
scales only linearly in $\Nbas$.
Furthermore coefficient-based \SCF schemes have the advantage,
that iterating the density matrix
destroys the possibility to perform the
linear-scaling application of $\mat{K}$ for \CS-based methods,
see equation \eqref{eqn:ApplicationKcs} for details.

\todoil{Pagebreak here?}
\pagebreak
It was already pointed out in section \vref{sec:OverviewSCF}
that focusing on coefficient-based schemes
is hardly a restriction in terms of the number of possible approaches,
since coefficient-based and density-based schemes can be interconverted,
at least approximately.
We will suggest modifications of the
optimal damping algorithm~(ODA)~\cite{Cances2000a} in section \ref{sec:tODA}
and Pulay's direct inversion of the iterative subspace~\cite{Pulay1982}
in section \ref{sec:DIIS},
which bring these methods to the coefficient-based setting.

Most of the \SCF algorithms we will consider here
only converge the \HF equations \eqref{eqn:HFiterated}
until the Pulay error \eqref{eqn:PulayError} vanishes
following our general description in remark \vref{rem:SCFcoeff}.
Regarding the \HF optimisation problem \eqref{eqn:HFOptCoeff}
this is only the necessary condition for a stationary point on the Stiefel
manifold $\mathcal{C}$.
Only some \SCF algorithms
termed \newterm{second-order self-consistent field method}s
guarantee (usually only appropriately)
that the stationary point they find is truly a minimum.
They are briefly considered in section \ref{sec:SOSCF}.

\subsection{Roothaan repeated diagonalisation}
\label{sec:RoothaanRepeatedDiag}

Roothaan's repeated diagonalisation~\cite{Roothaan1951}
approach to the \HF problem \eqref{eqn:HFiterated} is by far the simplest.
In the formalism of remark \vref{rem:SCFcoeff}
this algorithm can be described by building the next Fock matrix
$\tilde{\mat{F}}^{(n)}$ only by considering
the current occupied coefficients $\mat{C}^{(n)}$,
\ie $\tilde{\mat{F}}^{(n)} = \matFnfull$.
The two-step iteration procedure of figure \vref{fig:RoothanRepeatedDiag} results.
\begin{figure}
	\centering
	\missingfigure{Roothaan algorithm diagram}
	\caption{Roothaan algorithm diagram}
	\label{fig:RoothanRepeatedDiag}
\end{figure}

Even though Roothaan's algorithm already works for a few simple cases,
it is far from being reliable.
For example one can show~\cite{Cances2000,Cances2000b} that it
either converges to a stationary point of the
discretised \HF problem \eqref{eqn:HFOptCoeff}
or alternatively it numerically oscillates between two states,
where none of them are a stationary point of \eqref{eqn:HFOptCoeff}.
In practice which of these cases occurs is dependent on the system
as well as the basis set.
Furthermore there is no guarantee that the resulting stationary point
the Roothaan's algorithm finds is the \HF ground state.
All these cases are already found for \HF calculations
on the first three periods of the periodic table~\cite{Cances2000}.

\subsection{Level-shifting modification}
If one uses essentially the same \SCF scheme as \ref{fig:RoothanRepeatedDiag}
but instead diagonalises the matrix
\[ \tilde{\mat{F}}^{(n)} = \matFnfull
	- b \mat{S} \, \mat{C}^{(n)}\!\tp{\left(\mat{C}^{(n)}\right)} \, \mat{S} \]
where $b > 0$,
already a much better convergence is achieved.
This modification is called \newterm{level shifting}%
~\cite{Saunders1973,Guest1974,Mehrotra1977},
where $b$ is the \textbf{level-shifting parameter},
typically chosen in the range between $0.1$ and $0.5$.
Effectively this modification increases the energy gap between occupied and virtual
orbital energies.
To see this, let us consider the converged case, where
\[ \mat{F} \mat{C}_F = \mat{S} \mat{C}_F \mat{E} \]
exactly and let us partition the full coefficient matrix%
\footnote{We assume \RHF here and furthermore only consider the $\alpha$ block.
	For \UHF the analysis is exactly the same with
	the relevant equations just replicated in $\alpha$ and $\beta$ block.}
\[
	\mat{C}_F = \mm{\mat{C} & \mat{C}_\text{virt}}
\]
into occupied and virtual parts.
Now let $\tilde{\mat{F}} = \mat{F} - b \mat{S} \, \mat{C}\tp{\mat{C}} \, \mat{S}$
such that
\begin{align*}
	\tilde{\mat{F}} \mat{C}_F
	&= \left( \mat{F} - b \mat{S} \, \mat{C}\tp{\mat{C}} \, \mat{S} \right) \mat{C}_F \\
	&= \mat{S} \mat{C}_F \mat{E} - b \mat{S} \mm{
		\mat{C}\tp{\mat{C}} \mat{S} \mat{C} &
		\mat{C}\tp{\mat{C}} \mat{S} \mat{C}_\text{virt}
	} \\
	&= \mat{S} \mat{C}_F \mat{E} - b \mat{S} \mm{ \mat{C} & 0 } \\
	&= \mat{S} \mat{C}_F \mat{E} + \mat{S} \mat{C}_F \mm{ -b  & 0 } \\
	&= \mat{S} \mat{C}_F \tilde{\mat{E}}
\end{align*}
where
\[ \tilde{\mat{E}} = \diag\left(\varepsilon_1 - b, \varepsilon_2 - b, \ldots,
	\varepsilon_{\Nelec} - b, \varepsilon_{\Nelec+1}, \ldots, \varepsilon_{\Norb}\right).
\]
In other words the virtual orbitals are uneffected
whereas the occupied orbitals are shifted downwards in energy by an amount $b$.

The effect of this is that coupling between both orbital spaces is reduced,
which tends to lead to faster convergence
especially if the gap between $\varepsilon_{\Nelec}$ and $\varepsilon_{\Nelec+1}$ is small.
This empirical observation is backed up by more sophisticated
mathematical analysis by \citet{Cances2000b}.
Their result shows that for large enough $b$,
the level-shifted Roothaan procedure is guaranteed to converge to a stationary
point of the \HF problem \eqref{eqn:HFOptCoeff}.
The also provide an expression for the lower bound of $b$.
In this manner convergence to a stationary point
can be forced even for cases where the original \HF equations \eqref{eqn:HFMO}
have no solution (like the negative ions with $N > 2Z + M$).
In such a case the result is no physical ground state, however.

\subsection{Optimal damping algorithm}
\label{sec:ODA}
ODA~\cite{Cances2000,Cances2000a}

$\tilde{\mat{F}}^{(n)}$ gradient of $\mathcal{E}_D(\mat{D})$
wrt. $\mat{D}$~\cite{Lions1988,Cances2000}.
Can define~\cite{Lions1988,Cances2000}
\[
	\mat{C}^{(n+1)} = \arginf_{\mat{C} \in \mathcal{C}}
\left\{ \tr \tilde{\mat{F}}^{(n)} \mat{C} \tp{\mat{C}} \right\} \]

\subsection{Truncated optimal damping algorithm}
\label{sec:tODA}
ODA not great for coefficent-based setting
Approximation, which destroys many nice mathematical properties
Still: Automatically determine sensible damping factor (rather than a fixed one)

\subsection{Direct inversion of the iterative subspace}
\label{sec:DIIS}

Cances~\cite{Cances2000b} has something about this too

Standard commutator DIIS by Pulay~\cite{Pulay1982,Hamilton1986}

Other ways~\cite{Shepard2007}
LSIIS (least-squares commutator DIIS)~\cite{Li2016}

\subsection{Geometric direct minimisation}
Only brief

\subsection{Second-order \SCF algorithms}
\label{sec:SOSCF}
Only brief
mention that approximate methods exist, too

% https://www.sciencedirect.com/science/article/pii/0301010481851567

Quadratically-convergent scf \cite{Ochsenfeld1997}
Augmented Roothan-Hall \cite{Hoest2008}

\todoil{Check out: Linear scaling SCF as minimisation \cite{Salek2007} --- this is second order
and based on the density, preconditioned CG-like minimisation}


\subsection{Combinations of algorithms}
Ideally want to switch every now and then

