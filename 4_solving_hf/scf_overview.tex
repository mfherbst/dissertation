\section{Overview of the \SCF procedure}
\label{sec:OverviewSCF}
In remark \ref{rem:PropertiesDiscretised}
of the previous chapter
we suggested a simple procedure for iteratively solving the \HF equations.
The idea was to start from an initial guess $\mat{C}^{(0)}$
from the Stiefel manifold $\mathcal{C}$ as defined in \eqref{eqn:HFStiefel}
and repetitively construct occupied coefficient
matrices $\mat{C}^{(1)}, \mat{C}^{(2)}, \ldots, \mat{C}^{(n-1)} \in \mathcal{C}$
by solving the discretised \HF equations \eqref{eqn:HFiterated}
and considering the Aufbau principle.
Since the minimiser%
\footnote{in the discrete setting}
of the discretised \HF problem \eqref{eqn:HFOptCoeff}
is unique, there is no need to diagonalise exactly $\matFnfull$ in each iteration.
Instead we could well diagonalise an arbitrary matrix $\matFnt$
for obtaining the new coefficients $\mat{C}^{(n)}$.
It is important, however, to ensure that the final coefficients,
say $\mat{C}_0$,
satisfy the necessary conditions for being a minimiser of $\mathcal{E}_C$,
namely that $\mat{C}_0 \in \mathcal{C}$ and that the Pulay error \eqref{eqn:PulayError} vanishes.
At least its norm should stay within a finite value.
Notice, that for the computation of the Pulay error
in each case the unmodified matrix $\matFnfull$ needs to be employed
in order for the resulting value to be meaningful.
As discussed in remark \vref{rem:PropertiesDiscretised}
even if both these conditions are satisfied,
this is no guarantee, however,
that $\mat{C}^{(n)}$ is a minimiser for \eqref{eqn:HFOptCoeff},
since both are only \emph{necessary} but no sufficient conditions.
Ignoring this fact for a moment, this leads to the following general approach.

\begin{rem}[\SCF procedure]
	\label{rem:SCFcoeff}
	Pick a \newterm{convergence threshold} $\epsilonconv \in \R$,
	a \textbf{basis set} $\{\varphi_\mu\}_{\mu\in\Ibas} \subset H^1(\R^3, \R)$
	and an \newterm{initial guess} $\mat{C}^{(0)} \in \mathcal{C}$
	of occupied coefficients.
	From this build an initial Fock matrix
	$\tilde{\mat{F}}^{(0)} = \mat{F}\!\!\left[\mat{C}^{(0)} \tp{\left(\mat{C}^{(0)}\right)}\right].$

	\noindent
	For $n=1,2,3,\ldots$
	\begin{itemize}
		\item Diagonalise
			\[ \tilde{\mat{F}}^{(n-1)} \mat{C}_F^{(n)} = \mat{S} \mat{C}_F^{(n)} \mat{E}^{(n)} \]
			under the condition
			\[ \tp{\left(\mat{C}_F^{(n)}\right)} \mat{S}  \mat{C}_F^{(n)} = \mat{I}_{\Norb} \]
			where
			\[
				\mat{E}^{(n)}
				= \text{diag}\left(\varepsilon_1^{(n)},
				\varepsilon_2^{(n)}, \ldots,
				\varepsilon_{\Norb}^{(n)}\right)
			\]
			is the diagonal matrix of orbital eigenvalues.
		\item Construct the occupied matrix $\mat{C}^{(n)}$
			from the full matrix $\mat{C}_F^{(n)}$ by the Aufbau principle.
		\item Build the Fock matrix $\matFnfull$.
		\item Compute $\mat{e}^{(n)}$ according to \eqref{eqn:PulayError}
			\[
			\mat{e}^{(n)}
			= \matFnfull \mat{C}^{(n)} \tp{\left(\mat{C}^{(n)}\right)} \, \mat{S}
			- \mat{S} \, \mat{C}^{(n)} \tp{\left(\mat{C}^{(n)}\right)} \matFnfull
			\]
			Check the necessary condition:
			If $\norm{ \mat{e}^{(n)} }_\text{frob} \leq  \epsilonconv$
			the procedure is considered converged%
			\footnote{In finite dimensions all norms are equivalent,
				so the choice of the Frobenius norm is arbitrary here.}
			with final coefficients $\mat{C}_0 \equiv \mat{C}^{(n)}$.
		\item Build a Fock matrix $\tilde{\mat{F}}^{(n)}$ somehow
			using $\mat{C}^{(n)}$ and all insight into the problem gathered so far.
	\end{itemize}
	The final \HF energy is given by $\mathcal{E}_C(\mat{C}_0)$
	according to \eqref{eqn:HFEnergyFunctionalCoeff} and the final
	\SCF orbitals $\Theta^0$ by \eqref{eqn:HFDiscretisationAnsatz}.
\end{rem}
This scheme still leaves a couple of important questions unanswered,
which we will address in the following sections, namely:
\begin{itemize}
	\item What is a suitable method for choosing the initial guess $\mat{C}^{(0)}$?
	\item What type of basis function is suitable?
	\item What algorithms are sensible for building the next
		Fock matrix $\tilde{\mat{F}}^{(n)}$?
\end{itemize}
Furthermore remark \ref{rem:SCFcoeff}
considers the \HF problem to be parametrised
in terms of the occupied coefficients $\mat{C}^{(n)}$
and solves it by producing a sequence of coefficients
$\mat{C}^{(1)}, \mat{C}^{(2)}, \ldots, \mat{C}^{(n)} \in \mathcal{C}$
until convergence.
By the arguments discussed in section \ref{sec:DiscreteHF}
one can alternatively parametrise the \HF problem in terms of
density matrices $\mat{D}^{(n)}$.
In this light some \SCF algorithms are better understood
if one thinks about them as schemes producing
a sequence of density matrices
$\mat{D}^{(0)}, \mat{D}^{(1)}, \ldots, \mat{D}^{(n)} \in \mathcal{P}$
instead.
See the optimal damping algorithm in section \vref{sec:ODA} for an example.
To distinguish both approaches,
the first kinds of algorithms iterating $\mat{C}^{(n)}$
will be called \newterm{coefficient-based \SCF} schemes
whilst the second kind of algorithms iterating $\mat{D}^{(n)}$
we will call \newterm{density-based \SCF} algorithms.

\noindent
The identification
\[ \mat{D}^{(n)} = \mat{C}^{(n)} \tp{\left(\mat{C}^{(n)}\right)}, \]
which we already presented in \eqref{eqn:DensityReplacement} in section \ref{sec:DiscreteHF},
allows to build the density matrix from the coefficients by a matrix-matrix product
and in the reverse direction we can find matching coefficients for each density
matrix by a factorisation,
\eg a diagonalisation or a singular-value decomposition%
\footnote{Thanks to Eric Cancès for pointing this out to me.}.
This allows --- at least theoretically --- to convert
every density-based algorithms
into a coefficient-based scheme like remark \ref{rem:SCFcoeff}
and vice versa.
In practice the factorisation from density matrices to coefficients
could become rather costly and might not be always applicable.
