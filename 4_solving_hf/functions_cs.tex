\subsection{Coulomb-Sturmian-type orbitals}
\label{sec:BasisCS}

\todoil{TODO}

The last type of basis function
we want to consider here is again
an atom-centered basis function,
which is closely related to {\STO}s.


% https://www.math.tu-berlin.de/fachgebiete_ag_modnumdiff/fg_modellierung_simulation_und_optimierung_in_natur_und_ingenieurswissenschaften/v_menue/mitarbeiter/prof_dr_reinhold_schneider/publikationen/



\begin{figure}
	\centering
	\includeimage{4_solving_hf/relative_error_cs}
	\caption{Relative error for Coulomb-Sturmian basis sets for hydrogen.}
	\label{fig:LocalEnergyCS}
\end{figure}

\begin{figure}[p]
	\centering
	\includeimage{4_solving_hf/local_energy_cs}
	\caption{Local energy of a Coulomb-Sturmian basis sets for hydrogen.}
	\label{fig:LocalEnergyCS}
\end{figure}

\begin{figure}[p]
	\centering
	\includeimage{4_solving_hf/local_energy_cs_zoom}
	\caption{Zoom-in for local energy of Coulomb-Sturmian basis sets for hydrogen.}
	\label{fig:LocalEnergyCSZoom}
\end{figure}



% A few historic words and references
\cite{Rotenberg1970}
\cite{Gruzdev1990}
\cite{Avery2006} % Blue book
\cite{Avery2011} % Purple book
\cite{Morales2016}
\cite{Avery2017} % Fast evaluation of STOs

The simplest representative of the latter type of functions
are the Coulomb-Sturmians~(CS)~\cite{Shull1959,Rotenberg1962,Rotenberg1970}.
These can be formally obtained as the isoenergetic solutions
$\varphi^\text{CS}_{nlm}$ to a hydrogenic Schrödinger-like equation
\begin{equation}
	\left( \Delta + \frac{k n}{r} - E \right) \varphi^\text{CS}_{nlm}(\vec{r}) = 0.
	\label{eqn:CS}
\end{equation}
Their functional form is
\begin{equation}
	\varphi^\text{CS}_{nlm}(\vec{r}) = P_{nl}(r) \exp(- k r) Y_{lm}\left(\vec{r} / r \right),
	\label{eqn:CSProduct}
\end{equation}
\ie a product of a particular polynomial $P_{nl}$,
an exponential and a spherical harmonic.
The structure of these functions is very similar compared to STOs
with the important difference that the parameter $k$ is identical for
all functions of the basis set.
On the one hand this allows to reformulate the computation
of the electron-repulsion integrals in a rather efficient way%
~\cite{Avery2011PhD,Avery2011}.
On the other hand this similarity permits such a basis to simultaneously
represent the long-range tail and the nuclear cusp physically
if the $k$ is chosen appropriately,
which makes them rather promising.
While Coulomb-Sturmians are only applicable for simulating atoms,
a range of more generalised Sturmian-type basis functions exist%
~\cite{Avery2011},
where efficient algorithms for evaluating the repulsion integrals
have been published recently~\cite{Avery2006,Avery2015,Morales2016,Avery2017}.
\todo[inline]{James: Add citations for the ionising Sturmians.}


\begin{equation}
	bla
	\label{eqn:CSEquation}
\end{equation}

% Sturmians form a complete basis for H^2(R^3) and for H^1(R^1) (only the radial part)

% complete but not over-complete

In previous work we have investigated the use of so-called generalised Sturmians as basis functions
in electronic structure theory.
%
\newcommand{\rpack}{\vec{r}_1, \ldots, \vec{r}_N}
Generalised Sturmians $\Phi_\nu(\rpack)$ are the solutions to $N$-body Schrödinger-like equation
\begin{equation}
	\left( -\frac12 \sum_{j=1}^N \Delta_j + \beta_\nu V_0(\rpack) - E \right) \Phi_\nu(\rpack) = 0
	\label{eqn:GenSturm}
\end{equation}
where $V_0(\rpack)$ is
a zero-th order potential.

In case of atoms a good choice is the nuclear attraction:
The electrostatic electron-nucleus interaction
\[
	V_0(\rpack) = - \sum_{j=1}^N \frac{Z}{r_j}.
\]
\todo[inline]{Is many nuclei also possible here?}
Note that the electron-nucleus interaction is scaled by the factor $\beta_\nu$ in \eqref{eqn:GenSturm}
such that the solutions $\Phi_\nu(\rpack)$ all become isoenergetic.
This makes Sturmians reproduce the correct long-range decay behaviour of the electron density
as well as properly represent the nuclear cusp at the core.
As such the functional form of generalised Sturmians is very much related to STOs.
Unlike STOs the two-electron integrals can, however, be reformulated in a particular way
to make computing them less demanding.
This requires, however, that the mathematical properties of the Sturmians
can be fully exploited during the computation,
which in turn requires the Fock matrix to be arranged in a specific way.
Moreover it is advantageous to not build the coulomb and exchange matrix in memory at all,
but much rather only use these matrices in the form of matrix-vector products,
since this overall saves an order of magnitude in the computational scaling.


Coulomb-Sturmian basis function is composed of
\[ \varphi_\mu(\vec{r}) = \]

\todo[inline,caption={}]{
	\begin{itemize}
		\item What are Coulomb-Sturmians and where do they come from
		\item Show what needs to be done
		\item Show problems $\Rightarrow$ Lazy matrices
	\end{itemize}
}


\begin{equation}
	todo
	\label{eqn:ApplicationKcs}
\end{equation}
Explain how this only works if the coefficients are around as separate quantities
and not contracted as a density matrix
another example of contraction-based approach


% TODO Things to show:
%    - The larger n, the larger the energy under the action of the fock operator
%    - HF Kinetic energy scales with k: Lower if k lower
%    - HF Nuclear attraction energy scales with k: Higher if k lower
%    - HF electron-electron interaction scales with k: Lower if k lower
% Refer to these results in the chapter about Sturmian calculations

\begin{sidewaysfigure}
	\centering
	\includeimage{4_solving_hf/fock_sturmian}
	\caption{Structure of the Fock matrix for a Coulomb-Sturmian based SCF
		for the beryllium atom starting from using a $(5,1,1)$
		Coulomb-Sturmian basis in $mln$ order
		and a Sturmian exponent of $\kexp = 1.99$.
		The three figures show left to right the Fock matrix
		at an SCF step with a Pulay error Frobenius norm of
		$0.13$, $0.0079$, $6.7 \cdot 10^{-8}$.
		The colouring depends on the absolute value
		of the respective Fock matrix entry
		with white indicating entries below $10^{-8}$.
		}
		% 14 out of 17 are diagonal-dominant in each case.
		% Matrix is symmetric
	\label{fig:StructureSturmianFock}
\end{sidewaysfigure}




