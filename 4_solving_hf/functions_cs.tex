\subsection{Coulomb-Sturmian-type orbitals}
\label{sec:BasisCS}
Coulomb Sturmians~(CS) are another type of atom-centred basis functions,
which so far have seen little attention in electronic structure theory.
Similar to Slater-type orbitals
they were introduced by \citet{Shull1959} in \citeyear{Shull1959}
as a generalisation to the solutions of the Schrödinger equation for hydrogen-like atoms.
CS functions cannot be used for molecules, only for atoms,
but related functions exist,
which are more generally applicable
justifying our detailed treatment in this work.
The main motivation for \citet{Shull1959} to look into alternative
exponential functions was that they wanted to construct one-electron basis functions,
which could be used to compute the spectra of many-electron atoms.
From previous approaches based only on hydrogen-like orbital functions it was known,
that a proper representation of the cusps of the wave function
required the inclusion of the continuum,
which was exceedingly difficult in practice~\cite{Avery2006}.
Therefore they artificially modified the Schrödinger equation \eqref{eqn:HydrogenComplete}
for hydrogenic atoms and used the resulting analytical solutions
to model the spectrum of Helium up to a decent level of accuracy
without explicit inclusion of the continuum.
Effectively their trick was to multiply the Coulomb term
in \eqref{eqn:HydrogenComplete} by a prefactor
\begin{equation}
	\beta_n = \frac{\kexp n}{Z}
	\label{eqn:CSbeta}
\end{equation}
with $k \in \R$ arbitrary to yield
\begin{equation}
	\left( \Delta - \beta_n \frac{Z}{r} - E \right) \varphi^\text{CS}_\mu(\vec{r}) = 0.
	\label{eqn:CS}
\end{equation}
This equation has a countably infinite number of solutions
$\varphi^\text{CS}_\mu \in H^1(\R^3, \C)$,
which are the so-called \newterm{Coulomb Sturmians}.
They are \newterm{isoenergetic},
\ie all have the identical energy eigenvalue
\begin{equation}
	E = -\frac{\kexp^2}{2},
	\label{eqn:CSenergy}
\end{equation}
such that the underlying self-adjoint operator
\[ \Op{H}^\text{CS} = - \frac12 \Delta - \frac{n\kexp}{r} \]
has the very simple point spectrum
\[ \sigma_P(\Op{H}^\text{CS}) = \left\{ -\frac{\kexp^2}{2} \right\}, \]
but an empty discrete spectrum,
thus $\sigma_P(\Op{H}^\text{CS}) \subset \sigma_\text{ess}(\Op{H}^\text{CS})$.

Since \eqref{eqn:CS} and the hydrogen-like Schrödinger equation
\eqref{eqn:HydrogenComplete} are very similar,
we can follow the solution approach discussed in section \vref{sec:HydrogenAtom}
for \eqref{eqn:CS} as well.
Inserting a product ansatz of radial part and spherical harmonic
\begin{equation}
	\varphi^\text{CS}_\mu(\vec{r}) \equiv
	\varphi^\text{CS}_{nlm}(\vec{r}) = R^{\text{CS}}_{nl}(r) Y_l^m(\theta, \phi)
	\label{eqn:CSproduct}
\end{equation}
into \eqref{eqn:CS} allows to obtain the Coulomb Sturmian radial equation
\begin{equation}
	\left( - \frac{1}{2r^2} \laplaceRadial + \frac{l (l+1)}{2 r^2}
	- \frac{n\kexp}{r} - E \right) R_{nl}(r) = 0.
	\label{eqn:CSRadial}
\end{equation}
Its solutions have the form
\begin{equation}
	R^\text{CS}_{nl}(r) = N_{nl} (2\kexp r)^l e^{-\kexp r}
	\;_1F_1\left(l+1-n \middle| 2l+2 \middle|2\kexp r\right)
	\label{eqn:CSradialSolution}
\end{equation}
with normalisation constant
\[ N_{nl} = \frac{2 \kexp^{3/2}}{(2l+1)!} \sqrt{ \frac{(l+n)!}{n (n-l-1)!}}. \]
Unsurprisingly this solution is closely related to the radial part of the
hydrogen-like orbitals \eqref{eqn:HydrogenRadialSolution} as well.
In fact the Coulomb Sturmians can be constructed
from the functional form of the equivalent hydrogen-like orbitals
just by replacing the factors $Z/r$ by $\kexp$.
In analogy one therefore commonly uses the spectroscopic terminology
$1s$, $2s$, $2p$, $\ldots$
on Coulomb Sturmians as well to describe the respective triples
of quantum numbers $(n, l, m)$.
Given that both {\STO}s as well as \CS functions are exponential type orbitals
of the form radial part times spherical harmonic,
their radial parts are \eqref{eqn:CSradialSolution} and \eqref{eqn:STOradial}
are naturally very much related.
The important difference between both types of orbitals is
that{\STO} basis sets may use a different Slater exponent $\zeta_\mu$ for each
\STO basis function,
whereas all \CS functions share the same exponent $\kexp$
as a commonly modified parameter.
As we will discuss in more detail towards the end of this section,
this allows to rewrite the \ERI tensor in a computationally
much more feasible manner compared to the {\STO}s.

In their original work \citet{Shull1959} did not yet use the term ``Coulomb Sturmians''
to refer to the functions $\varphi^\text{CS}_\mu$.
It was only introduced a few years later
by Rotenberg~\cite{Rotenberg1962,Rotenberg1970},
who managed to find a link between the \CS radial equation \eqref{eqn:CSRadial}
and the special class of Sturm-Liouville differential equations.
These are second order differential equation of the form
\begin{equation}
	\left(
	\frac{\D}{\D r} \left( p(r) \frac{\D}{\D r} \right) + q(r) + \lambda_n w(r) \right) u_n(r) = 0,
	\label{eqn:SturmLiouville}
\end{equation}
where $p(r) \in C^1(\Omega, \R)$ and $q(r), w(r) \in C^0(\Omega, \R)$ are all positive
functions and $\Omega = (a,b) \subset \R$ is an open interval.
Provided that on $a$ and $b$ suitable boundary conditions
\begin{align*}
	u_i(a) \cos \alpha - p(a) u_i'(a) \sin \alpha &= 0 & 0 &< \alpha < \pi \\
	u_i(b) \cos \beta - p(b) u_i'(b) \sin \beta &= 0 & 0 &< \beta < \pi
\end{align*}
are chosen,
the eigenvalues $\lambda_i$ are real and non-degenerate
\[ \lambda_1 < \lambda_2 < \lambda_3 < \cdots < \lambda_n < \cdots \to \infty \]
and the eigenfunctions $u_i$ can be normalised
to satisfy the weighted orthonormality condition
\begin{equation}
	\int_a^b u_i^\ast(r) w(r) u_j(r) \D r = \delta_{ij}.
	\label{eqn:SturmLiouvilleWeightedOrtho}
\end{equation}
Following Rotenberg~\cite{Rotenberg1962,Rotenberg1970} we can use the ansatz
\[ R_{nl}(r) = \frac{u_{nl}(r)}{r} \]
as well as \eqref{eqn:CSenergy} to rewrite the Coulomb-Sturmian radial equation
\eqref{eqn:CSRadial} as
\[ \left(\frac{\partial^2}{\partial r^2} - \frac{l (l+1)}{r^2} - \frac{k^2}{2} + \frac{kn}{r} \right)
u_{nl} = 0, \]
which is of Sturm-Liouville form if one sets
\begin{align*}
	p(r) &= 1 & q(r) &= \frac{\kexp^2}{2} + \frac{l (l+1)}{r^2} & \lambda_n w(r) = \frac{n\kexp}{r}
\end{align*}
A consequence of this is that Coulomb Sturmians satisfy
the \newterm{potential-weighted orthonormality} condition~\cite{Avery2006}
\begin{equation}
	\int_{\R^3} \cc{\left(\varphi^\text{CS}_{nlm}(\vec{r})\right)}
		\frac{n}{r\kexp}  \varphi^\text{CS}_{n'l'm'}(\vec{r})
		\D \vec{r} = \delta_{n n'} \delta_{l l'} \delta_{m m'}.
	\label{eqn:CoulombSturmianWeightedOrtho}
\end{equation}
Most importantly, however, one is able to show
that the countably infinite set of all Coulomb Sturmians $\{\varphi^\text{CS}_\mu\}_{\mu}$
is a complete basis for $H^1(\R^3, \R)$~\cite[Theorem 2.3.4]{Avery2008}.
In the original context of Shull and Löwdin this implies that Coulomb Sturmians
are not only able to represent the bound states
of any atomic Schrödinger operator $\Op{H}_{\Nelec}$,
but the continuum as well.
In the more context of the discretised \HF problem~(see section \ref{sec:DiscreteHF})
or the \FCI problem~(see remark \ref{rem:FCI}),
this makes \CS basis functions rather promising since
they are mathematically guaranteed to yield the exact solution if one includes
more and more \CS functions in the basis.

\begin{figure}[t]
	\centering
	\includeimage{4_solving_hf/relative_error_cs}
	\caption[
		Relative error in the hydrogen \HF ground state
		for selected \CS basis sets.
	]
	{Relative error in the hydrogen \HF ground state
		for selected \CS basis sets.
		The error is plotted against
		the relative distance of electron and proton.
		The optimal value for $\kexp$ for hydrogen is $1.0$.
	}
	\label{fig:RelativeErrorCS}
\end{figure}

\noindent
Since \CS functions contain the term
\[ \exp(-\kexp r) = \exp\left( - \sqrt{-2E}r \right), \]
which both gives rise to a cusp at $r=0$ as well as
an energy-dependent exponential decay at $r\to\infty$,
they reflect the physical properties summarised in remark \vref{rem:PhysicalProperties}
already at the level of basis functions.
In general both the slope at the cusp as well as the decay at infinity
of the individual basis functions will \emph{not} be exactly the physically
required values, however.
Due to the completeness property of the \CS basis
this is not an issue, since a large enough basis will recover the error
for each $\kexp$.%
%
\begin{figure}[p]
	\centering
	\includeimage{4_solving_hf/local_energy_cs}
	\caption[Local energy of the hydrogen \HF ground state for {\CS} bases]{
		Local energy $E_L(r)$ of the hydrogen atom \HF ground state
		of selected Coulomb-Sturmian basis sets.
		$E_L(r)$ is plotted against the relative distance
		of electron and nucleus.
		The optimal value for $\kexp$ for hydrogen is $1.0$.
	}
	\label{fig:LocalEnergyCS}
\end{figure}
%
\begin{figure}[p]
	\centering
	\includeimage{4_solving_hf/local_energy_cs_zoom}
	\caption[Local energy of the hydrogen \HF ground state for {\cGTO}s (magnified)]{
		Magnified version of figure \vref{fig:LocalEnergyCS}
		around the origin.
		The orange curve theoretically goes to $-\infty$ as well,
		but the slope is so large that this is not visible
		at the resolution level of the plot.
	}
	\label{fig:LocalEnergyCSZoom}
\end{figure}
%
To illustrate this consider figure \vref{fig:RelativeErrorCS},
which shows the relative error in the hydrogen ground state
versus the relative electron-nucleus distance
for a few selected Coulomb-Sturmian basis sets.
The labels of the plots both indicate the $\kexp$ value as well as the
triple  $(\nmax, \lmax, \mmax)$, which is short for indicating
the set
\[ \left\{ \varphi^\text{CS}_{nlm} \middle| n \leq \nmax, l \leq \lmax, m \leq \mmax \right\} \]
of \CS functions.
The best value for $\kexp$ for hydrogen is $1.0$,
which would in fact reproduce the exact solution in the $\varphi^\text{CS}_{1s}$ function.
Figure \vref{fig:RelativeErrorCS} shows in agreement with our previous discussion
that both the size of the basis as well as the value for $\kexp$ has an influence
on the relative error.
Since the slope at which the \CS functions decay
at infinity depends on $\kexp$ ---
with larger values leading to faster decay ---
it is not surprising to find that
a too large value for $\kexp$ leads to a negative relative error at $\infty$,
whilst a too small value for $\kexp$ leads to a positive error.
Similarly larger deviations of $\kexp$ from $1.0$
in general cause the relative error to become larger in magnitude
throughout the curve.
Compare the blue and the orange curve with $\kexp = 1.4$ and $\kexp = 1.2$
for example.
Notice that the error values do not scale linearly with $\kexp$
and furthermore
that overestimating $\kexp$ is less problematic than underestimating $\kexp$.
Even though $\kexp = 0.8$ and $\kexp = 1.2$ each miss the exact exponent
by $0.2$ the orange curve is less steep as $r\to\infty$ and has a lower value
at the cusp than the green one.
This is a feature, which can be found in other systems as
well~(see chapter \vref{ch:CSQChem}).
Compared to the effect which $\kexp$ has on the error,
the effect of increasing the basis is much more pronounced.
Even though the green and the red curve both use a $\kexp$,
which is off by $0.2$,
the red curve stays below a relative error of $0.05$
over the full depicted range of distances,
whilst the green one starts to become rather inaccurate from distances
of $7.5$ Bohr and larger.
Very similar conclusions can be drawn
from figure \vref{fig:LocalEnergyCS},
which shows the local energy versus relative distance.
Comparing this plot to the local energy obtained for the \cGTO
discretisations in figure \vref{fig:LocalEnergyCgto}
one notices how the \cGTO local energy has much more wiggles
and overall deviations from the exact value of $0.5$.
Even though the \CS discretisations depicted
in figure \ref{fig:LocalEnergyCS} are not everywhere perfect eigenfunctions
of the hydrogen atom,
the still largely encode most of the physics.
At the level of a $(5,1,1)$ basis even with a too small value $\kexp = 0.8$,
the eigenfunction is acceptable over the full depicted range
except the nucleus.
Figure \ref{fig:LocalEnergyCSZoom} shows a detailed plot
of the local energies of a $(3,1,1)$, a $(5,1,1)$
and a $(7,1,1)$ discretisation for $\kexp = 0.8$
around the nucleus.
The $(5,1,1)$ discretisation already is very close to the $0.5$ everywhere,
but still decays visibly to $-\infty$ at the origin.
The $(7,1,1$ discretisation already corrects for this
up to the point, where the resolution of the plot is no longer
enough to fully show the divergence to $-\infty$.
For more examples discussing the convergence behaviour of
\CS discretisations consider chapter \vref{ch:CSQChem}.


\todoil{TODO evaluation of Fock matrix terms}


For evaluating the Fock matrix $\matFnfull$,
let us first consider the terms

Let us now consider the evaluation of the 


integrals of the \HF procedure

%---
The individual terms of the Fock matrix and the overlap matrix
are given by expressions \eqref{eqn:Tbas} to \eqref{eqn:Sbas}

For evaluating the Fock matrix $\matFnfull$,
let us first consider the terms
$\mat{T}$ and $\mat{V}_0$ as well as the overlap matrix $\mat{S}$.
This amounts to evaluating integrals
\begin{align*}
O_{\mu\nu} &=  \int_\Omega \varphi_\mu(\vec{r}) \, \Op{O} \, \varphi_\nu(\vec{r}) \D\vec{r}
&&\text{where} \quad \Op{O} = \Op{T}, \Op{V}_0 \text{ or } \id_{H^1(\Omega,\R)}.
\intertext{
With reference to the grid $\mathcal{M}_h$ we can write this as a
sum of cell contributions $O^c_{\mu\nu}$
}
O_{\mu\nu} &= \sum_{c\in\mathcal{M}_h} O^c_{\mu\nu}
&&\text{where} \quad O^c_{\mu\nu} = \int_c \varphi_\mu(\vec{r})\, \Op{O}\, \varphi_\nu(\vec{r}) \D\vec{r}.
\end{align*}
%---




% evaluation of the integrals
% one-electron integrals super easy
% two-electron integrals methods, contraction approach

\begin{equation}
	todo
	\label{eqn:ApplicationKcs}
\end{equation}
Explain how this only works if the coefficients are around as separate quantities
and not contracted as a density matrix
another example of contraction-based approach


% structure of the fock matrix diagonal dominance
% Fock matirx similar to gaussians

\begin{figure}
	\centering
	\includeimage[width=\textwidth]{4_solving_hf/fock_sturmian}
	\caption{Structure of the Fock matrix for a Coulomb-Sturmian based SCF
		for the beryllium atom starting from using a $(5,1,1)$
		Coulomb-Sturmian basis in $mln$ order
		and a Sturmian exponent of $\kexp = 1.99$.
		The three figures show left to right the Fock matrix
		at an SCF step with a Pulay error Frobenius norm of
		$0.13$, $0.0079$, $6.7 \cdot 10^{-8}$.
		The colouring depends on the absolute value
		of the respective Fock matrix entry
		with white indicating entries below $10^{-8}$.
		}
		% 14 out of 17 are diagonal-dominant in each case.
		% Matrix is symmetric
	\label{fig:StructureSturmianFock}
\end{figure}




\todoil{A few lines summarising why they are useful alternatives}
% Include solution approaches: Direct and iterative, both possible


This requires, however, that the mathematical properties of the Sturmians
can be fully exploited during the computation,
which in turn requires the Fock matrix to be arranged in a specific way.
Moreover it is advantageous to not build the coulomb and exchange matrix in memory at all,
but much rather only use these matrices in the form of matrix-vector products,
since this overall saves an order of magnitude in the computational scaling.

\todoil{From here overview of other types of Sturmian functions}

While Coulomb Sturmians are only applicable for simulating atoms,
a range of more generalised Sturmian-type basis functions exist
~\cite{Avery2011}
which share some of the properties~\cite{Avery2011},
such that Coulomb Sturmians can be seen as a simple test case.
In the most general sense Sturmian-type orbitals are the solution
to the $\Nelec$-body Schrödinger-like equation
\begin{equation}
	\left( -\frac12 \Delta + \beta_{\mu} V_0(\vec{x}) - E \right) \Phi^\text{St}_\mu(\vec{x}) = 0
	\label{eqn:SturmianGeneral}
\end{equation}
where
\[ V_0(\vec{x}) = V_0(\vec{r}_1, \vec{r}_2, \ldots, \vec{r}_{\Nelec}) \]
is a good zeroth order approximation to the actual potential
\[ V(\vec{x}) = \sum_{A=1}^M \sum_{i=1}^{\Nelec} \frac{Z_A}{\norm{\vec{R}_A - \vec{r}_i}}
+ \sum_{i=1}^{\Nelec} \sum_{j=i+1}^{\Nelec} \frac{1}{\norm{\vec{r}_i - \vec{r}_j}_2} \]
such that \eqref{eqn:SturmianGeneral} can be analytically solved
and $\beta_{\mu}$ is chosen to make the solutions isoenergetic.
For many-electron atoms Goscinski for example suggested
$V_0(\vec{x}) = \sum_i \frac{Z}{r_i}$.
Many-centre Sturmians are possible, too~\cite{Avery2003,Avery2013}.
In this broad sense Sturmian-type orbitals are an active
field of research%
~\cite{%
	Gruzdev1990,Avery2006,Hoggan2009,%
	Randazzo2010,Mitnik2011,%
	Avery2011,Avery2011PhD,%
	Avery2013,Avery2015,Randazzo2015,Granados2016,Abdouraman2016,%
	Morales2016,Avery2017,Avery2018%
}
with many efforts to yield fast evaluation schemes
for the \ERI tensor~\cite{Avery2013,Avery2017,Avery2018}
for generalised Sturmians as well.

Methods to use generalised Sturmians or Coulomb Sturmians
for \STO integral evaluation~\cite{Morales2016,Avery2017}.

Methods to combine numerical methods and Sturmians
to yield Sturmians including modelling of continuum%
~\cite{Randazzo2010,Mitnik2011,Randazzo2015,Granados2016,Abdouraman2016}
