\subsection{Coulomb-Sturmian-type orbitals}
\label{sec:BasisCS}
Coulomb-Sturmians~(CS) are another type of atom-centred basis functions,
which so far have seen little attention in electronic structure theory.
Similar to Slater-type orbitals
they were introduced~\cite{Shull1959}
as a generalisation to the solutions of the Schrödinger equation for hydrogen-like atoms.
\CS functions cannot be used for molecules, only for atoms,
but closely related functions exist,
which are more generally applicable.
The main motivation for \citet{Shull1959} to look into alternative
exponential functions was that they wanted to construct one-electron basis functions,
which could be used to compute the spectra of many-electron atoms.
From previous approaches it was known that a
proper representation of the cusps of the wave function
required the inclusion of the continuum
if hydrogen-like orbital functions were used.
This was, however, exceedingly difficult in practice~\cite{Avery2006}.
To avoid this dilemma, \citet{Shull1959} artificially modified
the Schrödinger equation \eqref{eqn:HydrogenComplete}
for hydrogen-like atoms,
such that it was on the one hand still analytically solvable,
but on the other hand the spectrum of Helium could be modelled
up to a rather good level of accuracy,
even without explicit inclusion of the continuum.
Effectively their trick was to multiply the Coulomb term
in \eqref{eqn:HydrogenComplete} by a prefactor
\begin{equation}
	\beta_n = \frac{\kexp n}{Z}
	\label{eqn:CSbeta}
\end{equation}
with $\kexp \in \R$ arbitrary to yield
\begin{equation}
	\left( - \frac12 \Delta - \beta_n \frac{Z}{r} - E \right) \varphi^\text{CS}_\mu(\vec{r}) = 0.
	\label{eqn:CS}
\end{equation}
This equation has a countably infinite number of solutions
$\varphi^\text{CS}_\mu \in H^1(\R^3, \C)$,
which are the so-called \newterm{Coulomb-Sturmians}.
They are \newterm{isoenergetic},
\ie all have the identical energy eigenvalue
\begin{equation}
	E = -\frac{\kexp^2}{2},
	\label{eqn:CSenergy}
\end{equation}
such that the underlying self-adjoint operator
\[ \Op{H}^\text{CS} = - \frac12 \Delta - \frac{n\kexp}{r} \]
has the very simple point spectrum
\[ \sigma_P(\Op{H}^\text{CS}) = \left\{ -\frac{\kexp^2}{2} \right\}, \]
but an empty discrete spectrum,
thus $\sigma_P(\Op{H}^\text{CS}) \subset \sigma_\text{ess}(\Op{H}^\text{CS})$,
which such that continuum effects may be captured in a truncated basis.

Since \eqref{eqn:CS} and the hydrogen-like Schrödinger equation
\eqref{eqn:HydrogenComplete} are very similar,
we can apply the solution approach discussed in section \vref{sec:HydrogenAtom}
to equation \eqref{eqn:CS} as well.
Inserting a product ansatz of radial part and spherical harmonic
\begin{equation}
	\varphi^\text{CS}_\mu(\vec{r}) \equiv
	\varphi^\text{CS}_{nlm}(\vec{r}) = R^{\text{CS}}_{nl}(r) Y_l^m(\uvec{r})
	\equiv
	R^{\text{CS}}_{nl}(r) Y_l^m(\theta, \phi)
	\label{eqn:CSproduct}
\end{equation}
into \eqref{eqn:CS} we obtain the Coulomb-Sturmian radial equation
\begin{equation}
	\left( - \frac{1}{2r^2} \laplaceRadial + \frac{l (l+1)}{2 r^2}
	- \frac{n\kexp}{r} - E \right) R_{nl}(r) = 0.
	\label{eqn:CSRadial}
\end{equation}
Its solutions have the form
\begin{equation}
	R^\text{CS}_{nl}(r) = N_{nl} (2\kexp r)^l e^{-\kexp r}
	\;_1F_1\!\left(l+1-n \middle| 2l+2 \middle|2\kexp r\right)
	\label{eqn:CSradialSolution}
\end{equation}
with normalisation constant
\[ N_{nl} = \frac{2 \kexp^{3/2}}{(2l+1)!} \sqrt{ \frac{(l+n)!}{n (n-l-1)!}} \]
and $\;_1F_1\!\left(a|b|z\right)$ being the confluent hypergeometric function
as defined in \eqref{eqn:ConfluentHypergeometric}.
Unsurprisingly this functional form is closely related to the radial part of the
hydrogen-like orbitals \eqref{eqn:HydrogenRadialSolution}.
In fact the Coulomb-Sturmians can be constructed
from the equivalent hydrogen-like orbitals
just by replacing the factors $Z/r$ by $\kexp$.
In analogy one therefore commonly uses the spectroscopic terminology
$1s$, $2s$, $2p$, $\ldots$
to describe the respective triples
of quantum numbers $(n, l, m)$ for Coulomb-Sturmians as well.
Given that both {\STO}s as well as \CS functions are exponential type orbitals
of the form radial part times spherical harmonic,
their radial parts \eqref{eqn:CSradialSolution} and \eqref{eqn:STOradial}
are related as well%
\footnote{In fact, some recent work~\cite{Avery2017} exploits this to evaluate
\STO \ERI integrals with Coulomb-Sturmians.}.
The important difference between both types of orbitals is
that {\STO} basis sets may use a different Slater exponent $\zeta_\mu$ for each
\STO basis function,
whereas all \CS functions share the same exponent $\kexp$
as a commonly modified parameter.
Even though this difference is subtle,
it is the key ingredient to derive the efficient evaluation schemes
of the \CS-\ERI tensor discussed further down this section.

In their original work, \citet{Shull1959} did not yet use the term ``Coulomb-Sturmians''
to refer to the functions $\varphi^\text{CS}_\mu$.
This name was only introduced a few years later
by Rotenberg~\cite{Rotenberg1962,Rotenberg1970},
who managed to find a link between the \CS radial equation \eqref{eqn:CSRadial}
and the special class of Sturm-Liouville differential equations.
Sturm-Liouville equations are second order differential equation of the form
\begin{equation}
	\left(
	\frac{\D}{\D r} \left( p(r) \frac{\D}{\D r} \right) + q(r) + \lambda_n w(r) \right) u_n(r) = 0,
	\label{eqn:SturmLiouville}
\end{equation}
where $p(r) \in C^1(\Omega, \R)$ and $q(r), w(r) \in C^0(\Omega, \R)$ are all positive
functions and $\Omega = (a,b) \subset \R$ is an open interval.
Provided that on $a$ and $b$ suitable boundary conditions
\begin{align*}
	u_i(a) \cos \alpha - p(a) u_i'(a) \sin \alpha &= 0 & 0 &< \alpha < \pi \\
	u_i(b) \cos \beta - p(b) u_i'(b) \sin \beta &= 0 & 0 &< \beta < \pi
\end{align*}
are chosen,
the eigenvalues $\lambda_i$ are real and non-degenerate
\[ \lambda_1 < \lambda_2 < \lambda_3 < \cdots < \lambda_n < \cdots \to \infty \]
and the eigenfunctions $u_i$ can be normalised
to satisfy the weighted orthonormality condition
\begin{equation}
	\int_a^b u_i^\ast(r) w(r) u_j(r) \D r = \delta_{ij}.
	\label{eqn:SturmLiouvilleWeightedOrtho}
\end{equation}
Following Rotenberg~\cite{Rotenberg1962,Rotenberg1970}, one can use the ansatz
\[ R_{nl}(r) = \frac{u_{nl}(r)}{r} \]
as well as \eqref{eqn:CSenergy} to rewrite the Coulomb-Sturmian radial equation
\eqref{eqn:CSRadial} as
\[ \left(\frac{\partial^2}{\partial r^2} - \frac{l (l+1)}{r^2} - \frac{k^2}{2} + \frac{kn}{r} \right)
u_{nl} = 0, \]
which is of Sturm-Liouville form with
\begin{align*}
	p(r) &= 1 & q(r) &= \frac{\kexp^2}{2} + \frac{l (l+1)}{r^2} & \lambda_n w(r) = \frac{n\kexp}{r}
\end{align*}
One consequence of this is that Coulomb-Sturmians satisfy
the \newterm{potential-weighted orthonormality} condition~\cite{Avery2006}
\begin{equation}
	\int_{\R^3} \cc{\left(\varphi^{\text{CS}}_{nlm}(\vec{r})\right)}\,
		\frac{n}{r\kexp} \, \varphi^\text{CS}_{n'l'm'}(\vec{r})
		\D \vec{r} = \delta_{n n'} \delta_{l l'} \delta_{m m'}.
	\label{eqn:CoulombSturmianWeightedOrtho}
\end{equation}
Most importantly, however, it is possible to show
that the countably infinite set of all Coulomb-Sturmians $\{\varphi^\text{CS}_\mu\}_{\mu}$
is a complete basis for $H^1(\R^3, \R)$~\cite[Theorem 2.3.4]{Avery2008}.
In the original context of Shull and Löwdin this implies that Coulomb-Sturmians
are not only able to represent the bound states
of any atomic Schrödinger operator $\Op{H}_{\Nelec}$,
but the continuum as well.
When it comes to the discretised \HF problem~(see section \ref{sec:DiscreteHF})
or the \FCI problem~(see remark \ref{rem:FCI}),
this makes \CS basis functions rather promising,
since the completeness property
provides a mathematical guarantee that the exact solution can be approximated
arbitrarily closely
if more and more \CS functions are included.

\begin{figure}[t]
	\centering
	\includeimage{4_solving_hf/relative_error_cs}
	\caption[
		Relative error in the hydrogen \HF ground state
		for selected \CS basis sets.
	]
	{Relative error in the hydrogen \HF ground state
		for selected \CS basis sets.
		The error is plotted against
		the relative distance of electron and proton.
		The optimal value for $\kexp$ for hydrogen is $1.0$,
		which is exact.
	}
	\label{fig:RelativeErrorCS}
\end{figure}

\noindent
Since \CS functions contain the term
\[ \exp(-\kexp r) = \exp\left( - \sqrt{-2E}r \right), \]
which both gives rise to a cusp at $r=0$ as well as
an energy-dependent exponential decay at $r\to\infty$,
they reflect the physical properties summarised in remark \vref{rem:PhysicalProperties}
already at the level of basis functions.
In general both the slope at the cusp as well as the decay at infinity
of the individual basis functions will \emph{not} be exactly the physically
required values, however.
Due to the completeness of the \CS basis
this is not an issue, since a large enough basis will recover the error
for each $\kexp$.%
%
\begin{figure}[p]
	\centering
	\includeimage{4_solving_hf/local_energy_cs}
	\caption[Local energy of the hydrogen \HF ground state for {\CS} bases]{
		Local energy $E_L(r)$ of the hydrogen atom \HF ground state
		of selected Coulomb-Sturmian basis sets.
		$E_L(r)$ is plotted against the relative distance
		of electron and nucleus.
		The optimal value for $\kexp$ for hydrogen is $1.0$,
		which is exact.
	}
	\label{fig:LocalEnergyCS}
\end{figure}
%
\begin{figure}[p]
	\centering
	\includeimage{4_solving_hf/local_energy_cs_zoom}
	\caption[Local energy of the hydrogen \HF ground state for {\cGTO}s (magnified)]{
		Magnified version of figure \vref{fig:LocalEnergyCS}
		around the origin.
		The orange curve theoretically goes to $-\infty$ as well,
		but the slope is so large that this is not visible
		at the resolution level of the plot.
	}
	\label{fig:LocalEnergyCSZoom}
\end{figure}
%
To illustrate this for an example consider figure \vref{fig:RelativeErrorCS},
which shows the relative error in the hydrogen ground state
versus the relative electron-nucleus distance
for a few selected Coulomb-Sturmian basis sets.
The labels of the plots both indicate the $\kexp$ value as well as the
triple  $(\nmax, \lmax, \mmax)$, which is a short hand for indicating
the finite basis set
\[ \left\{ \varphi^\text{CS}_{nlm} \, \middle| \, n \leq \nmax, l \leq \lmax, \abs{m} \leq \mmax \right\} \]
of \CS functions.
The best $\kexp$ value for hydrogen is $1.0$,
which would in fact give the exact ground state in the $\varphi^\text{CS}_{1s}$ function.
Figure \vref{fig:RelativeErrorCS} shows, in agreement with our previous discussion,
that both the size of the basis as well as the value for $\kexp$ has an influence
on the relative error.
Since the slope at which the \CS functions decay
at infinity depends on $\kexp$ ---
with larger values leading to faster decay ---
it is not surprising to find that
a too large value for $\kexp$ leads to a negative relative error at $r = \pm \infty$,
whilst a too small value for $\kexp$ leads to a positive error.
Similarly, larger deviations of $\kexp$ from $1.0$
cause the relative error to become larger in magnitude throughout the curve:
Compare the blue and the orange curve with $\kexp = 1.4$ and $\kexp = 1.2$,
for example.
The relative error does, however, not scale linearly with $\kexp$.
Yet furthermore it is not even symmetric with respect to the direction
into which $\kexp$ deviates from the optimal value.
In this case the orange curve is less steep as $r\to\infty$ and has a lower value
at the cusp than the green one,
even though both miss the best exponent by $0.2$.
In all systems I investigated so far,
I made the similar observation that the error is more pronounced if the optimal
value for $\kexp$ is underestimated rather than overestimated.
Compared to the effect which $\kexp$ has on the error,
the effect of increasing the basis is much more significant.
Even though the green and the red curve both use a $\kexp$
which is off by $0.2$,
the red curve following a $(5,1,1)$-basis
stays below a relative error of $0.05$
over the full depicted range of distances.
On the other hand, the green one, a $(3,1,1)$-basis,
starts to become rather inaccurate from distances
of $7.5$ Bohr and larger.

Very similar conclusions can be drawn
from figure \vref{fig:LocalEnergyCS},
which shows the local energy versus relative distance.
Comparing this plot to the local energy obtained for the \cGTO
discretisations in figure \vref{fig:LocalEnergyCgto},
one notices how the \cGTO local energy has much more wiggles
and overall deviations from the exact value of $0.5$.
Even though the \CS discretisations depicted
in figure \ref{fig:LocalEnergyCS} are not perfect eigenfunctions
of the hydrogen atom,
the local energy is still mostly close to $0.5$,
thus they encode most of the physics.
Even with a too small value $\kexp = 0.8$,
the $(5,1,1)$ basis produces an acceptable eigenfunction over the full depicted
range --- except the nucleus.
This is illustrated in more detail in figure
\ref{fig:LocalEnergyCSZoom},
which is a close-up of the local energies of a $(3,1,1)$, a $(5,1,1)$
and a $(7,1,1)$ discretisation for $\kexp = 0.8$
around the nucleus.
Whilst the $(3,1,1)$ and the $(5,1,1)$ both decay visibly to $-\infty$ at the origin,
the $(7,1,1)$ discretisation already mostly corrects for this.
Even though it still goes to $-\infty$ in theory,
the resolution of the plot is no longer good enough to show this properly.
From the illustrated trends it is clear that
\CS discretisations are able to represent
both the exponential decay as well as the electron-nuclear cusp
up to any desired accuracy if the basis is chosen large enough.
More examples discussing the convergence behaviour of
\CS discretisations can be found in chapter \vref{ch:CSQChem}.

Apart from the ability of a basis function type to properly
represent the physics of a chemical system, we also need to be able
to solve the arising numerical problems
in order to make it useful for practical quantum-chemical calculations.
Similar to the other basis function types discussed so far,
we will therefore now turn our attention to the Fock matrix $\matFnfull$,
both its structure as well as its diagonalisation.
For this we first consider the computation of the integrals
\eqref{eqn:Tbas} to \eqref{eqn:Sbas},
starting with the overlap matrix.
Its elements $S_{\mu\mu'}$ can be computed
for any two Coulomb-Sturmians%
\footnote{The ``CS'' superscripts for basis functions
and radial parts are dropped in the remainder of this section for simplicity.}
$\varphi_{\mu}$ and $\varphi_{\mu'}$
by treating radial and angular part separately~\cite{Avery2015}
\begin{equation}
	\begin{aligned}
	S_{\mu\mu'} &= \int_{\R^3} \cc{\varphi_{\mu}}(\vec{r}) \, \varphi_{\mu'}(\vec{r})  \D\vec{r} \\
	&= \int_0^\infty R_{nl}(r)\, R_{n'l'}(r) \, r^2 \D r \cdot
		\int_{\set{S}^2} \cc{\left(Y^m_l\right)}\!\!(\uvec{r}) \,\,
			Y^{m'}_{l'}(\uvec{r}) \D \uvec{r} \\
			&= \delta_{m m'} \delta_{l l'}
				\underbrace{\int_0^\infty R_{nl}(r)\, R_{n'l}(r) r^2 \D r}_{= s^{(l)}_{n n'}}.
	\end{aligned}
	\label{eqn:OverlapCS}
\end{equation}
Normalisation implies that $s^{(l)}_{n n} = 1$
and the potential-weighted orthonormality \eqref{eqn:CoulombSturmianWeightedOrtho}
implies that $s^{(l)}_{n n'} = 0$ iff $\abs{n -n'} > 1$.
By following the algebra one can further show~\cite{Avery2015} that
\[
	s^{(l)}_{n,n+1} = s^{(l)}_{n+1,n} = - \frac12 \sqrt{\frac{(n-l)(n+l+1)}{n (n+1)}},
\]
making $\mat{S}$ both sparse and trivial to compute.
Similarly one can directly employ the potential-weighted orthonormality \eqref{eqn:CoulombSturmianWeightedOrtho}
to show that the nuclear attraction matrix is diagonal, namely
\begin{equation}
	\begin{aligned}
	\left(V_0\right)_{\mu\mu'} &=
	- \int_{\R^3} \cc{\varphi_{\mu}}(\vec{r})\, \frac{Z}{r}\,
		\varphi_{\mu'}(\vec{r})  \D\vec{r} \\
	&= - \frac{Z\kexp}{n'} \int_{\R^3} \cc{\varphi_{\mu}}(\vec{r})\,
		\frac{n'}{r\kexp}\, \varphi_{\mu'}(\vec{r})  \D\vec{r} \\
	&= - \delta_{\mu \mu'} \frac{Z\kexp}{n}.
	\end{aligned}
	\label{eqn:NucAttrCS}
\end{equation}
From \eqref{eqn:CSbeta} to \eqref{eqn:CSenergy} we get
\[ \left(- \frac12 \Delta - \frac{n \kexp}{r} + \frac{\kexp^2}{2} \right) \varphi_{\mu}(\vec{r}) = 0, \]
which implies for the kinetic energy matrix elements
\begin{equation}
	\begin{aligned}
		T_{\mu\mu'} &=
		\int_{\R^3} \cc{\varphi_{\mu}}(\vec{r}) \left( - \frac12 \Delta  \right)
		\varphi_{\mu'}(\vec{r})  \D\vec{r} \\
		&= \int_{\R^3} \cc{\varphi_{\mu}}(\vec{r})
		\left(\frac{n' \kexp}{r} - \frac{\kexp^2}{2} \right)
		\varphi_{\mu'}(\vec{r})  \D\vec{r} \\
		&= k^2 \left( \delta_{\mu\mu'} - \frac12 S_{\mu\mu'} \right) \\
		&= k^2 \delta_{ll'} \delta_{mm'} \left( \delta_{nn'} - \frac{1}{2} s^{(l)}_{n n'} \right),
	\end{aligned}
	\label{eqn:KineticCS}
\end{equation}
such that they follow the same advantageous sparsity pattern as the overlap matrix.
The one-electron integrals thus contain at most 3 non-zeros per row
and are tridiagonal in $lmn$ or $mln$ index order.
Due to the simplicity of the expressions of the matrix elements,
storing these matrix terms in memory --- even in a compressed tridiagonal form ---
is not needed,
since recomputing the values takes a neglible number of flops.

Unsurprisingly, treating the two-electron integrals is more involved.
We follow \cite{Avery2015},
which describes the treatment in a more general context
and the specialised arguments presented in the documentaton of
\sturmint~\cite{sturmintWeb}.
Due to the structure of the radial part $R_{nl}(r)$
one may write the product of two Coulomb-Sturmians as a
sum over Coulomb-Sturmians with twice the exponent, \ie
\newcommand{\csC}{\mathcal{C}}
\begin{equation}
	\cc{\varphi}_{\mu_1}(\vec{r}) \, \varphi_{\mu_2}(\vec{r})
	= \sum_{\mu} \csC^\mu_{\mu_1,\mu_2} \, \varphi_{\mu}(2\kexp, \vec{r}),
	\label{eqn:CSshellpairExpansion}
\end{equation}
where $\varphi_{\mu}(2\kexp, \vec{r})$ denotes a \CS function
with twice the exponent.
This expansion looks familiar to the density-fitting approximation
in the context of \cGTO basis sets,
but is in fact \emph{exact} in the case of Coulomb-Sturmians.
Since
\[ \cc{\left( \cc{\varphi}_{\mu_1}(\vec{r}) \varphi_{\mu_2}(\vec{r}) \right)} =
\cc{\varphi}_{\mu_2}(\vec{r}) \varphi_{\mu_1}(\vec{r}) \]
it follows that the conjugated product requires the related
expansion coefficients $\csC^\mu_{\mu_2,\mu_1}$.
With this the electron repulsion integral tensor in Mulliken index \eqref{eqn:ERI}
ordering may be written as the contraction
\begin{equation}
	\eriMu{\mu_1\mu_2}{\mu_3\mu_4}
		= \sum_{\mu\mu'} \cc{\left(\csC^\mu_{\mu_1,\mu_2}\right)} \,
		I_{\mu\mu'} \, \csC^{\mu'}_{\mu_3,\mu_4}
		= \sum_{\mu\mu'} \csC^\mu_{\mu_2,\mu_1} \,
		I_{\mu\mu'} \, \csC^{\mu'}_{\mu_3,\mu_4}
	\label{eqn:CSeriContraction}
\end{equation}
where $I_{\mu\mu'}$ is the electron repulsion kernel in terms of the $2\kexp$-functions
\begin{equation}
	I_{\mu\mu'} \equiv
	\int_{\R^3} \int_{\R^3}
	\frac{
	\cc{\varphi_{\mu'}}(2k,\vec{r}_1) \,
	\varphi_{\mu}(2k,\vec{r}_2)
	}{r_{12}}
	\D\vec{r}_1 \D\vec{r}_2.
	\label{eqn:CSeriKernelDef}
\end{equation}
Using the expansion of the Coulomb operator
in terms of spherical harmonics~\cite{Avery2018}
\todoil{Citation?}
\[ \frac{1}{r_{12}} = \sum_{l''=0}^\infty
	\frac{r^{l''}_<}{r_>^{l'' + 1}} \frac{4\pi}{2l'' + 1}
	\sum_{m''=-{l''}}^{l''} Y_{l''}^{m''}(\uvec{r}_1)
	\cc{\left(Y_{l''}^{m''}(\uvec{r}_2)\right)}
\]
where
\begin{align*}
	r_< &\equiv \min(r_1, r_2) & r_> &\equiv \max(r_1, r_2)
\end{align*}
equation \eqref{eqn:CSeriKernelDef} may be rewritten as
\begin{equation}
\begin{aligned}
	I_{\mu\mu'}
	&= \sum_{l''=0}^{\infty}\frac{4\pi}{2l''+1}
	\sum_{m''=-l''}^{l''}
	\int_0^\infty \int_0^\infty
	r_1^2 R_{nl}(2\kexp,r_1) \, r_2^2 R_{n'l'}(2\kexp,r_2) \frac{r^{l''}_<}{r_>^{l'' + 1}}
	\, \D r_1  \D r_2 \\
	&\hspace{30pt}
	\cdot \int_{\set{S}^2}
		\underbrace{\cc{\left(Y_l^m(\uvec{r}_1)\right)} \, Y_{l''}^{m''}(\uvec{r}_1)}
		_{=\delta_{l,l''}\delta_{m,m''}}
		\D \uvec{r}_1
	\cdot \int_{\set{S}^2}
		\underbrace{Y_{l'}^{m'}(\uvec{r}_2) \cc{\left(Y_{l''}^{m''}(\uvec{r}_2)\right)}}
		_{=\delta_{l',l''}\delta_{m',m''}}
		\, \D \uvec{r}_2 \\
	&= \delta_{ll'} \delta_{mm'} I^{(l)}_{nn'}
\end{aligned}
\label{eqn:CSeriKernel}
\end{equation}
where
\begin{equation}
	I^{(l)}_{nn'}
	= \frac{4\pi}{2l+1}
	\int_0^\infty \int_0^\infty
	r_1^2 R_{nl}(2\kexp,r_1) \, r_2^2 R_{n'l}(2\kexp,r_2) \frac{r^{l}_<}{r_>^{l + 1}}
	\, \D r_1  \D r_2.
	\label{eqn:CSeriKernelSmall}
\end{equation}
It is not immediately obvious from the form of equation \eqref{eqn:CSeriKernelSmall},
but the dependency on $\kexp$ can be factored out of this expression,
such that it only depends on $n$, $n'$ and $l$.
Assuming for the principle quantum number $n \leq 20$, which is rather typical,
the tensor $I^{(l)}_{nn'}$ has only about $20^3 = 8000$ elements,
which can be pre-evaluated and stored inside the program.
In fact even more simplifications are possible if one inserts
the definition of the radial parts and splits the integration kernel by powers of
$r_1$ and $r_2$.
For the required polynomial powers $\alpha, \beta$ the integrals
\[ \int_0^\infty \int_0^\infty r_1^\alpha r_2^\beta \exp(-r_1) \exp(-r_2)
	\frac{r^{l}_<}{r_>^{l + 1}} \, \D r_1  \D r_2 \]
can then be precomputed and stored as a vector.
At runtime one only needs to form the dot product of the precomputed vector
with the appropriate vector of polynomial coefficients
to yield the value for $I^{(l)}_{nn'}$.

\noindent
Let us now return to equation \eqref{eqn:CSshellpairExpansion}, \ie
\[
	\cc{\varphi}_{\mu_1}(\vec{r}) \, \varphi_{\mu_2}(\vec{r})
	= \sum_{\mu} \csC^\mu_{\mu_1,\mu_2} \, \varphi_{\mu}(2\kexp, \vec{r}).
\]
To obtain an expression for the coefficients $\csC^\mu_{\mu_1,\mu_2}$
we multiply this equation with
$\cc{\varphi}_{\mu'}(2\kexp, \vec{r})$
from the right and integrate over $\R^3$.
Using the potential-weighted orthonormality \eqref{eqn:CoulombSturmianWeightedOrtho}
for the $2\kexp$ Coulomb-Sturmians
this yields
\begin{equation}
	\begin{aligned}
	\csC^\mu_{\mu_1,\mu_2}
	&= \frac{n}{2k}
	\int_{\R^3}
	\cc{\varphi}_{\mu_1}(\vec{r}) \varphi_{\mu_2}(\vec{r}) \,
		\frac{1}{r} \, \varphi_{\mu}(2\kexp, \vec{r})
	\D \vec{r}. \\
	&= \frac{n}{2k}
		\int_0^\infty R_{n_1,l_1}(r) R_{n_2,l_2}(r) \, R_{n,l}(2k, r) r \D r \\
	&\hspace{30pt}
		\cdot \int_{\set{S}^2}
			\cc{\left(Y_l^m(\uvec{r})\right)} \,
			\cc{\left(Y_{l_1}^{m_1}(\uvec{r})\right)} \,
			Y_{l_2}^{m_2}(\uvec{r})
		\D \uvec{r}.
	\end{aligned}
	\label{eqn:CScoefficientExpression}
\end{equation}
The angular part of the latter expression can be written in terms of
\todoil{Citation}
Clebsch-Gordan coefficients, which are precomputed and stored%
\footnote{Due to the sparsity and symmetry properties
	of the Clebsch-Gordan coefficients even for a large value
	maximal principle quantum number like $n=20$,
	no more than a few
	hundred thousand such coefficients need to be stored.
	If some recursion relations are taken into account as well,
	it is far less.
}.
The properties of the Clebsch-Gordan coefficients imply that
$\csC^\mu_{\mu_1,\mu_2}$ can only be non-zero if
\begin{align*}
	m &= m_2 - m_1 &&\text{and}&
	l &\in \Big[\abs{l_1-l_2}, l_1+l_2\Big],
\end{align*}
such that $\csC^\mu_{\mu_1,\mu_2}$ is again a sparse tensor.
The radial part is computed similar to \eqref{eqn:CSeriKernelSmall},
\ie as a dot product between polynomial coefficients
and precomputed kernels over polynomial powers.

Due to the outlined sparsity of the $2\kexp$-kernel $I_{\mu\mu'}$
and the expansion coefficients $\csC^\mu_{\mu_1,\mu_2}$
the contraction in equation \eqref{eqn:CSeriContraction} can be written
more effectively as
\begin{equation}
	\begin{aligned}
	\eriMu{\mu_1\mu_2}{\mu_3\mu_4}
		&= \sum_{\mu\mu'} \csC^\mu_{\mu_2,\mu_1}
		% note: order has to be 2, then 1 since this implies a complex conjugation
		\, I_{\mu\mu'} \, \csC^{\mu'}_{\mu_3,\mu_4} \\
		&= \sum_{n,l,m} \sum_{n',l',m'}
			\csC^{(n,l,m)}_{\mu_2,\mu_1}
			\, \delta_{ll'} \delta_{mm'} I^{(l)}_{nn'} \,
			\csC^{(n',l',m')}_{\mu_3,\mu_4} \\
		&= \sum_{n'} \sum_{n,l,m}
			\csC^{(n,l,m)}_{\mu_2,\mu_1}
			\, I^{(l)}_{nn'} \,
			\csC^{(n',l,m)}_{\mu_3,\mu_4} \\
		&= \delta_{m_1 - m_2, m_4 - m_3}
		\sum_{l=\lmin}^{\lmax} \sum_{n=l+1}^{n_1+n_2-1} \sum_{n'=l+1}^{n_3+n_4-1}
			\csC^{(n,l,m_2-m_1)}_{\mu_2,\mu_1}
			\, I^{(l)}_{nn'} \,
			\csC^{(n',l,m_2-m_1)}_{\mu_3,\mu_4}
	\end{aligned}
	\label{eqn:CSeriContractionEffective}
\end{equation}
where
\begin{align}
	\lmin &= \max(\abs{l_1-l_2},\abs{l_3-l_4}) &
	\lmax &= \min(l_1+l_2, l_3+l_4).
	\label{eqn:CSminmax}
\end{align}
Because of the selection rules in the quantum numbers $l$ and $m$
the \ERI tensor is thus a sparse quantity
with far less than $\Nbas^4$ non-zeros.
When contracting it with the occupied coefficients $\mat{C}$
to form the Coulomb and exchange matrices, \ie computing the elements
\begin{align}
	\label{eqn:CScoulomb}
	J_{\mu_3\mu_4}\!\left[\mat{C}^{(n)}\left(\mat{C}^{(n)}\right)^\dagger\right]
	&= \sum_{i\in\Iocc} \, \sum_{\mu_1,\mu_2 \in\Ibas} \, \sum_{\mu,\mu'\in\Ibas}
		C^{(n)}_{\mu_1 i} C^{(n)\ast}_{\mu_2 i} \,
		\csC^{\mu}_{\mu_2,\mu_1}
		I_{\mu\mu'} \csC^{\mu'}_{\mu_3,\mu_4} \\
\intertext{and}
	\label{eqn:CSexchange}
	K_{\mu_3\mu_4}\!\left[\mat{C}^{(n)}\left(\mat{C}^{(n)}\right)^\dagger\right]
	&= \sum_{i\in\Iocc} \, \sum_{\mu_1,\mu_2 \in\Ibas} \, \sum_{\mu,\mu'\in\Ibas}
		C^{(n)}_{\mu_1 i} C^{(n)\ast}_{\mu_2 i} \,
		\csC^{\mu}_{\mu_2,\mu_3}
		I_{\mu\mu'} \csC^{\mu'}_{\mu_1,\mu_4},
\end{align}
the sparsity is partially lost.
The reason is that the sum over the occupied orbital index $i$ implies that
each element $J_{\mu_3\mu_4}$ or $K_{\mu_3\mu_4}$
becomes a linear combination of contributions from
different angular quantum number pairs $(l_1, m_1)$ and $(l_2, m_2)$.
Thus, a Coulomb or exchange matrix element is only a known zero
if \emph{all} of the possible combinations of the indices $\mu_3$, $\mu_4$
with the pairs $(l_1, m_1)$ and $(l_2, m_2)$ are guaranteed to be zero
--- a much weaker selection rule.
For forming the matrix-vector products of $\mat{J}$ and $\mat{K}$
with other vectors
therefore most elements of $J_{\mu_3\mu_4}$ and $K_{\mu_3\mu_4}$
need to be touched.
For the exchange matrix $\mat{K}$ in fact all elements may be non-zero,
giving rise to a full quadratic scaling of a matrix-vector product
in the number of basis functions.
On the other hand, avoiding the storage of $\eriMu{\mu_1\mu_2}{\mu_3\mu_4}$
and $\mat{K}$ in favour of directly computing
the matrix-vector product expression
\begin{equation}
	\left(\matK \vec{x}\right)_{\mu_3}
	= \sum_{i\in\Iocc} \, \sum_{\mu_1,\mu_2,\mu_4 \in\Ibas} \, \sum_{\mu,\mu'\in\Ibas}
		C^{(n)}_{\mu_1 i} C^{(n)\ast}_{\mu_2 i} \,
		\csC^{\mu}_{\mu_2,\mu_3}
		I_{\mu\mu'} \csC^{\mu'}_{\mu_1,\mu_4}
		\, x_{\mu_4},
	\label{eqn:ApplicationKcs}
\end{equation}
whenever the contraction of $\mat{K}$ with a vector $\vec{x}$ is needed,
one may fully exploit all angular momentum selection rules during the evaluation.
With this one may achieve the best possible scaling, certainly below quadratic.
Notice that an efficient contraction scheme for computing \eqref{eqn:ApplicationKcs}
will carry out the contraction over occupied orbitals (index $i$) at the very end.
In other words the improved scaling originating from \eqref{eqn:ApplicationKcs}
can only be achieved if $\mat{K}$ is not in memory
and if the occupied coefficients $\mat{C}$ are available as separate quantities
and not already contracted into a density matrix.

\begin{figure}
	\centering
	\includeimage[width=\textwidth]{4_solving_hf/fock_sturmian}
	\caption[Structure of the Fock matrix for a Coulomb-Sturmian based SCF]
		{Structure of the Fock matrix for a Coulomb-Sturmian based SCF
		for the beryllium atom starting from using a $(5,1,1)$
		Coulomb-Sturmian basis in $mln$ order
		and a Sturmian exponent of $\kexp = 1.99$.
		The three figures show left to right the Fock matrix
		at an SCF step with a Pulay error Frobenius norm of
		$0.13$, $0.0079$, $6.7 \cdot 10^{-8}$.
		The colouring depends on the absolute value
		of the respective Fock matrix entry
		with white indicating entries below $10^{-8}$.
		}
		% 14 out of 17 are diagonal-dominant in each case.
		% Matrix is symmetric
	\label{fig:StructureSturmianFock}
\end{figure}
Both the very simple form of the one-electron matrices,
given by the expressions
\eqref{eqn:OverlapCS},
\eqref{eqn:NucAttrCS} and
\eqref{eqn:KineticCS},
as well as the previous discussion about the angular momentum selection
rules in the case of the Coulomb and exchange matrices
suggests to employ a \contraction-based scheme
for a Coulomb-Sturmian-based \SCF.
Looking at the structure of the Fock matrix $\matF$
in figure \vref{fig:StructureSturmianFock},
we notice that it is very similar to the \cGTO case~(figure \vref{fig:StructureGaussianFock}).
Most notably it is almost diagonal dominant and of a similar
size than the \cGTO Fock matrix.
In other words a dense diagonalisation method could in theory be employed
for the Fock matrix $\matF$ as well.
The downside of a dense scheme would be the higher
storage requirement as well as the larger
computational scaling of the matrix-vector product.
Whilst \CS discretisations on the one hand
do not require \contraction-based methods to be feasible,
they still allow for improved contraction if such methods are employed.

While Coulomb-Sturmians can only be used for simulating atoms,
a range of more generalised Sturmian-type basis functions exist%
~\cite{Hoggan2009,Avery2011},
which can be applied, for example, to molecular systems as well.
Especially when it comes to evaluating the two-electron integrals,
these share some of the properties of the Coulomb-Sturmians,
but both the mathematical machinery as well as the numerics are more involved.
\CS functions can thus be seen as a first step towards these
more general Sturmian-type basis functions.
Generalised Sturmian-type orbitals are an active field of research%
~\cite{%
	Gruzdev1990,Avery2003,Avery2006,Hoggan2009,%
	Randazzo2010,Mitnik2011,%
	Avery2011,Avery2011PhD,%
	Avery2013,Avery2015,Randazzo2015,Granados2016,Abdouraman2016,%
	Morales2016,Avery2017,Avery2018%
}.
Some recent works include efforts to develop schemes for the
fast evaluation of the resulting \ERI tensor~\cite{Avery2013,Avery2017,Avery2018}
as well as the application of Sturmian-type functions
for evaluating \STO integrals more efficiently~\cite{Morales2016,Avery2017}.
Other methods include the combination of
Sturmians and some numerical methods to yield ionising Sturmians
to simultaneously model bound states as well as the continuum
in a single basis~\cite{Randazzo2010,Mitnik2011,Randazzo2015,Granados2016,Abdouraman2016}.

% TODO OPTIONAL
%In the most general sense Sturmian-type orbitals are the solution
%to the $\Nelec$-body Schrödinger-like equation
%\begin{equation}
%	\left( -\frac12 \Delta + \beta_{\mu} V_0(\vec{x}) - E \right) \Phi^\text{St}_\mu(\vec{x}) = 0
%	\label{eqn:SturmianGeneral}
%\end{equation}
%where
%\[ V_0(\vec{x}) = V_0(\vec{r}_1, \vec{r}_2, \ldots, \vec{r}_{\Nelec}) \]
%is a good zeroth order approximation to the actual potential
%\[ V(\vec{x}) = \sum_{A=1}^M \sum_{i=1}^{\Nelec} \frac{Z_A}{\norm{\vec{R}_A - \vec{r}_i}}
%+ \sum_{i=1}^{\Nelec} \sum_{j=i+1}^{\Nelec} \frac{1}{\norm{\vec{r}_i - \vec{r}_j}_2} \]
%such that \eqref{eqn:SturmianGeneral} can be analytically solved
%and $\beta_{\mu}$ is chosen to make the solutions isoenergetic.
%For many-electron atoms Goscinski for example suggested
%$V_0(\vec{x}) = \sum_i \frac{Z}{r_i}$.
%Many-centre Sturmians are possible, too~\cite{Avery2003,Avery2013}.
%In this broad sense Sturmian-type orbitals are an active
%field of research
