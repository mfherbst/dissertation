\section{Guess methods}
A good guess for an iterative procedure like the \SCF
is characterised by two things.
Firstly it should be already reasonably close to the expected solution,
because otherwise a totally random random guess will do.
Secondly it should be cheap to obtain,
at least considerably cheaper than the \SCF itself.
Otherwise again a totally random guess is fine.
Notice, that random guesses might seem rather ridiculous from a practical point of view,
but for investigating the stability of an \SCF procedure
or for investigating the stationary points of the manifolds $\mathcal{C}$
and $\mathcal{D}$ away from the actual minimum,
they are very helpful.



Secondly it should be already reasonably close to the expected solution,
because otherwise a totally random 

and secondly
it should already be very close to the solution.
Otherwise one might as well start with a completely random
guess and perform the full iterative procedure starting from there.
Even though a random guess might seem ridiculous on first thought,
such 



A good guess for an \SCF procedure is both cheap to obtain


Good guess is cheap to obtain and still captures most of the physics of the system.
Inside the basin of attraction for SCF minimum
for the chosen SCF algorithm

one could always start random.

\paragraph{Core Hamiltonian guess}
The expensive parts of the Fock matrix are the computation
of the ERI tensor as well the resulting J and K matrices as contraction.
In terms of absolute energy the Coulomb and Exchange energy,
however, contribute very little.
Plus their terms lead to the self-consistency problem.
A good first approximation on the other hand is to consider
totally non-interacting electrons, ie to only consider the core hamiltonian
$\mat{T} + \mat{V}_0$.

Typically far too contracted electrons.
Non-bonding electrons!
Leads to issues with certain basis types.
For example with Coulomb-Strumians it often yields a non-physical
stationary state on the SCF manifold.

\paragraph{Extended Hückel guess}
Basic description of bonding 


\paragraph{Superposition of atomic densities}
Typically a very good guess
Works on fitted parameters
Especially simple for atom-centered basis functions
(essenitally you place the guess with the basis functions onto the grid)


\paragraph{Projection from previous result}
