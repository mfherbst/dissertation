\section{Guess methods}
A good guess for an iterative procedure like the \SCF
is characterised by two things.
Firstly it should of course already be close to the expected solution.
Otherwise one might as well from a totally random random guess instead.
Secondly it should be cheap to obtain,
at least considerably cheaper than the \SCF itself.
Otherwise again a totally random guess does just as well.

Notice that in general random guesses have not much application
from a practical point of view,
but for investigating the stability of an \SCF procedure
they are really helpful.
For example one could check whether a combination of guess method
and \SCF algorithm yields the true minimum or just a stationary
point of the \HF problem \eqref{eqn:HFOptCoeff}
by trying a few random guesses and checking the resulting energies.

The next sections present a non-exhaustive list of
commonly used guess methods for starting \SCF procedures.

\subsection{Core Hamiltonian guess}
\label{sec:CoreHamiltonian}
Only the Coulomb and exchange matrix terms of the Fock matrix
expression \eqref{eqn:FockMatrix}
\[ \matFfull = \mat{T} + \mat{V}_0 + \matJfull + \matKfull \]
depend on the coefficients $\mat{C}$.
Furthermore the entries of the kinetic matrix $\mat{T}$
and the nuclear attraction matrix $\mat{V}_0$
are typically larger than the entries of the Coulomb and exchange matrices.
A reasonable approximation,
which avoids the \SCF procedure as a whole
is therefore to find an initial guess $\mat{C}^{(0)}$
by diagonalising the core Hamiltonian $\mat{T} + \mat{V}_0$
and keeping the $\Nelec$ lowest eigenvalue solutions.

Since electrons in this model do not repel each other
the resulting approximate orbitals are typically too contracted
and thus not extremely physical.
In my experiments with Coulomb-Sturmian-type
basis functions~(see section \ref{sec:BasisCS})
for example I found core Hamiltonian guesses to often
converge to stationary points in the \SCF process,
which are \emph{not} minima of the \HF problem.

Such issues become worse with large basis sets or larger molecules.
An ad-hoc way to fix this is to scale the nuclear attraction
matrix by a factor $0 < \alpha \leq 1$
in order to mimic the shielding of the nuclear charge somewhat.
Nevertheless this guess method is typically only used
if other options are not available.
It can, however, always be done.

\subsection{Superposition of atomic densities}
The idea of the superposition of atomic densities~(SAD)~\cite{Lenthe2006}
is that molecules are to a very large extend just a collection of atoms,
such that the molecular electron density
can be obtained approximately
just by adding the densities of all constituting atoms together.
If atom-centered basis functions are used this process is almost trivial.
Let us illustrate the procedure by
a chemical system made up of $M$ atoms labelled $1, 2, \ldots, M$.
We first perform atomic \ROHF
calculations on each atom
using the same basis set we want to employ for the molecular calculation,
but only the basis functions of the atom in question.
This yields converged atomic \SCF density matrices
$\mat{D}_1, \mat{D}_2, \ldots \mat{D}_M$.
In the SAD guess method as described in \cite{Lenthe2006}
the trial density matrix $\tilde{\mat{D}}$
is the sum of all density matrix $\mat{D}_1^\alpha$, $\mat{D}_1^\beta$,
$\mat{D}_2^\alpha$ after they have been projected from the atomic
basis onto the basis used for the molecular calculation.
If we compose the basis of the molecular system
in the usual manner, \ie
by pasting together
all basis functions a basis set defines for each atom,
in the order atom by atom,
$\tilde{\mat{D}}$ would be block-diagonal
\[
	\tilde{\mat{D}} = \mm{
		\mat{D}_1^\alpha + \mat{D}_1^\beta & 0         & \cdots & 0 \\
		0         & \mat{D}_2^\alpha + \mat{D}_2^\beta & \cdots & 0 \\
		\vdots    & \vdots    & \ddots & \vdots \\
		0         & 0         & \cdots & \mat{D}_M^\alpha + \mat{D}_M^\beta
	}.
\]
Replicating $\tilde{\mat{D}}$ twice on the $\alpha$ and the $\beta$
block we can construct the trial density matrix
\[
	\mat{D}^\text{t} =
	\mm{\tilde{\mat{D}} & 0 \\ 0 & \tilde{\mat{D}}}
\]
and with it a trial Fock matrix
$\mat{F}^\text{t} = \mat{F}\!\left[\mat{D}^\text{t}\right]$.
A diagonalisation
\[
	\mat{F}^\text{t} \mat{C}^{(0)} = \mat{C}^{(0)} \mat{E}^\text{t}
\]
finally yields the initial coefficients $\mat{C}^{(0)}$
along with some trial energies along the diagonal matrix $\mat{E}^\text{t}$.

\noindent
A few remarks about the SAD guess method:
\begin{itemize}
	\item 
	This whole procedure costs roughly as much as a single \SCF step
	plus the time needed for the atomic calculation,
	which is typically negligible.
	\item Furthermore many quantum-chemistry programs
		store precomputed atomic densities $\mat{D}_1, \mat{D}_2, \ldots$
		for their supported basis functions
		and all relevant elements of the periodic table,
		such that the cost of the SAD guess procedure
		is typically even lower in practice.
	\item The quality of the guess is in general rather good~\cite{Lenthe2006}.
	\item
	Since $\mat{D}^\text{t} \not\in \mathcal{P}$,
	the orbital energies $\mat{E}^\text{t}$
	obtained from the diagonalisation
	of $\mat{F}^\text{t}$ are not variational with respect to the
	overall \HF problem.
	At least one further \SCF step is therefore required.
	\item For molecular \UHF and \ROHF calculations,
		one might want to perturb the $\alpha$ and $\beta$
		parts of $\mat{D}^\text{t}$ slightly in order to enforce
		breaking the spin symmetry in the $\alpha$ and $\beta$ blocks.
\end{itemize}

\subsection{Guesses by projection}
\label{sec:GuessProject}
A common procedure in many discretisation approaches
is to first obtain a quick and crude solution
using only a small basis set and to refine the result later in a larger basis.
Ideally as much of the information gained in the crude result
is used for starting the large calculation.

In the context of the \SCF procedure one would for example
like to use the final coefficients $\tilde{\mat{c}}_0$ from a calculation
in the small basis $\{\chi_{\tilde{\nu}}\}_{\tilde{\nu}=1,\ldots,N_g}$
with only $N_g$ functions
to obtain a guess for the more refined calculation in the basis
$\{\varphi_\mu\}_{\mu=1,\ldots,N_b}$ with $N_b \leq N_g$.
Conceptionally this requires to operate on the coefficients
with the transformation matrix
$\mat{U} \in \C^{N_b\times N_g}$
consisting of elements
\[ U_{\mu\tilde{\nu}} = \braket{\varphi_\mu}{\chi_{\tilde{\nu}}}_1. \]
Since $U_{\mu\tilde{\nu}}$ is not necessarily unitary,
the simple matrix-matrix product $\mat{U} \mat{c}_0$
will in general not be orthonormal
with respect to the overlap matrix $\mat{S} \in \R^{N_b \times N_b}$
in the new basis and is thus not directly usable.
Instead a more involved treatment taking the overlap matrices
both in the old and the new basis into account.
If we use $\tilde{\mat{s}} \in \R^{\N_g \times \N_g}$
to denote the overlap matrix in the new basis,
a properly orthonormalised guess would be~\cite{Polly2004}
\begin{align}
	\mat{C}^{(0)} &= \mat{S}^{-1} \mat{U} \mat{c}_0 \mat{N}^{-1/2} \\
	\intertext{where the matrix}
	\mat{N} &= \tp{\tilde{\mat{c}}_0} \tp{\mat{U}} \mat{S}^{-1} \mat{U} \mat{c}_0
\end{align}
takes care of proper normalisation.
The computation (and diagonalisation) of $\mat{N}$
can be avoided if other techniques
for orthogonalising the $\Norb$ column vectors $\mat{S}^{-1} \mat{U} \mat{c}_0$
are used, like Gram-Schmidt procedure or a singular-value decomposition.

A modification of this procedure would be to alternatively build
\begin{align}
	\label{eqn:GuessMatF}
	\mat{f} &= \mat{U} \mat{c}_0 \tilde{\mat{E}} \tp{\mat{c}_0} \tp{\mat{U}}, \\
	\intertext{where}
	\nonumber
	\tilde{\mat{E}} &= \diag\left(\tilde{\varepsilon}_1, \tilde{\varepsilon}_2,
			\ldots \tilde{\varepsilon}_{\Norb}\right)
\end{align}
are the orbital energies obtained from the \SCF in the old basis.
Diagonalisation of this matrix with respect to $\mat{S}$
yields a set of initial guess coefficients $\mat{C}^{(0)}$ as the eigenvectors
and the modified orbital energies as the eigenvalues.
For example the quantum-chemistry program ORCA uses the latter
approach by default~\cite{ORCA}.

\subsection{Extended Hückel guess}
The extended Hückel~(EH) procedure for obtaining estimates
of molecular orbitals was developed in the 1960s
by Hoffmann~\cite{Hoffmann1963}
based on the extended Hückel Hamiltonian matrix defined
in the earlier work by Wolfsberg and Helmholtz~\cite{Wolfsberg1952}.
Sometimes this procedure is called
\textbf{Generalised Wolfsberg-Helmholtz procedure}
for this reason as well.

The idea is here to start from a minimal set of orbitals
$\{\phi_i\}_{i=1,\ldots,N_\text{trial}}$,
originally exponential-type orbitals,
and build the model Hamiltonian
\[
	H_{ij}^\text{EH}
		= \frac12 K S^\text{EH}_{ij} \left( H_{ii}^\text{EH} + H_{jj}^\text{EH} \right),
\]
from the EH overlap matrix $\mat{S}^\text{EH}$ with elements
\[ S^\text{EH}_{ij} = \int_{\R^3} \phi_i(\vec{r}) \phi_j(\vec{r}) \D\vec{r}, \]
an empirical parameter $K$ typically set to $1.75$
and the diagonal elements $H_{ii}^\text{EH}$,
which should be a rough estimate for the trial orbital energies $\phi_i$.
For this a range of methodologies are employed in practice,
including the diagonal elements of the core Hamiltonian matrix
of the trial basis,
experimental atomic ionisation energies~\cite{Yamehmop}
or even the results from a cheap \SCF procedure~\cite{ORCA}.
The obtained matrix $H_{ij}^\text{EH}$ is diagonalised with respect
to the EH overlap matrix $\mat{S}^\text{HF}$
yielding trial coefficients $\mat{C}^\text{HF}$.
Following the procedures of the previous section \ref{sec:GuessProject}
one may project these onto the basis set of the problem of interest
and use them as an initial guess $\mat{C}^{(0)}$ for the \SCF procedure.

Despite its age the EH method is still subject to active research.
For example \citet{Lee2015} have constructed a scheme combining
the extended Hückel method and Slater's rules~\cite{Slater1930}
by which decent guesses for Finite-Element based density-functional theory
calculations may be obtained.

Typically the EH method works reasonably well
for small basis sets and small molecular systems.
This drawback is overcome if an approach related to the
superposition of atomic densities is used for obtaining the diagonal
elements of $\mat{H}^\text{EH}$.
The quantum-chemistry ORCA~\cite{ORCA} for example
uses both the atomic orbitals as well as the orbital energies
from pre-calculated atomic STO-3g~\cite{Hehre1969} calculations
to drive the EH guess:
The trial basis set $\{\phi_i\}_{i=1,\ldots,N_\text{trial}}$
in their approach is just the combination of all atomic STO-3g orbitals
and the diagonal elements $H_{ii}^\text{EH}$ the corresponding STO-3g orbital energies.
