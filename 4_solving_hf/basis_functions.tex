\defineabbr{CS}{CS\xspace}{Coulomb-Sturmian or Coulomb-Sturmian basis, see section \vref{sec:BasisCS}}
\defineabbr{cGTO}{cGTO\xspace}{Contracted Gaussian-type orbitals, see section \vref{sec:cGTO}}
\defineabbr{STO}{STO\xspace}{Slater-type orbitals, see section \vref{sec:STO}}
\defineabbr{ETO}{ETO\xspace}{Exponential-type orbitals, orbitals with radial part of the form Polynomial in $r$ times $\exp{-\alpha r}$ with some parameter $\alpha$.}
\defineabbr{FE}{FE\xspace}{Finite elements, see section \vref{sec:FE}}
\defineabbr{ACO}{ACO\xspace}{atom-centered orbitals}

\section{Basis function types}
This section tries to address the question,
which classes of functions can be used
in order to build a basis set $\{\varphi_\mu\}_{\mu\in\Ibas}$
for solving the \HF problem in an \SCF procedure.
For this we will first introduce some desirable properties for a basis set,
both motivated from the desire to represent the physics of the electronic Schrödinger
equation as good as possible
as well as requirements from the numerical side.
In the light of this,
we will discuss four types of basis functions in depth,
namely the historic Slater-type orbitals~({\STO}s),
the most commonly employed contracted Gaussian-type orbitals~({\cGTO}s),
a finite-element based discretisation method
as an example of a fully numerical approach
as well as so-called Coulomb-Sturmian-type orbitals.

Even though we mostly concentrate on the \HF problem in this section,
quite a few of the observations made here
apply to \DFT or methods going beyond Hartree-Fock as well.
In this sense the outlined discussion
can be seen as an example case for the use of the mentioned
basis function types in electronic structure theory as a whole.

\subsection{Desirable properties}
Why can we not use every kind of function?
Issue with evaluating the integrals
get the correct physics



\begin{rem}
	This remark provides a short recap of some
	fundamental physical properties of the electronic Schrödinger equation.

Kato's cusp condition~\cite{Kato1957}
Energy-dependent exponential decay at infinity
(easy to rationalise physically)
\end{rem}


Apart from relative and absolute errors a concept,
which is very helpful for investigating how well a basis represents
the wavefunction is the \newterm{local energy}
\begin{defn}
	Local energy
	Trial wavefunction $\Psi$
	\[
		E_L(\vec{x}) \equiv \frac{\Op{H}_{\Nelec} \Phi(\vec{x})}{\Phi(\vec{x})}
	\]
	approximately constant for good approximations.
\end{defn}
This concept originates from the quantum Monte Carlo community%
~\cite{Foulkes2001,Ma2005}
where the 
% Ma paper has a good explanation for local energy and what it means
% physically




\subsection{Slater-type orbitals}
\label{sec:STO}

\subsection{Contracted Gaussian-type orbitals}
\label{sec:cGTO}
\todo[inline,caption={}]{
	\begin{itemize}
		\item Brief rationale
		\item Problems
		\item E.g. Local energy => what happens if $\psi = 0$
	\end{itemize}
}
% Convention:
%    Use \chi_\mu for primitives


\begin{figure}
	\centering
	\includeimage{4_solving_hf/local_energy_cgto}
	\caption{Local energy of contracted Gaussian basis sets for hydrogen.}
	\label{fig:LocalEnergyCgto}
\end{figure}

\begin{figure}
	\centering
	\includeimage{4_solving_hf/local_energy_cgto_zoom}
	\caption{Zoom-in for local energy of contracted Gaussian basis sets for hydrogen.}
	\label{fig:LocalEnergyCgtoZoom}
\end{figure}

\begin{figure}
	\centering
	\includeimage{4_solving_hf/relative_error_cgto}
	\caption{Relative error of a selection of contracted Gaussian basis sets
		for hydrogen.}
	\label{fig:LocalEnergyCgto}
\end{figure}



\subsection{Finite-element based discretisation}
\label{sec:FE}
\todo[inline,caption={}]{
	\begin{itemize}
		\item Show what needs to be done
		\item Show problems $\Rightarrow$ Lazy matrices
		\item Hint at the underlying linear algebra (eigensolvers, linear solvers)
		\item Multigrid methods
		\item Refinement
		\item How does it connect to the SCF methods
	\end{itemize}
}

% Use as an example to explain the ideas of finite elements
An important equation from classical electrodynamics which
involves the Laplace operator is Poisson's equation
\begin{equation}
	\Delta V_H(\vec{r}) = \rho(\vec{r})
	\label{eqn:Poisson}
\end{equation}
relating the potential to the charge density $\rho$ which generates it.

% TODO 
%  Note that norm wrt fock operator F is equivalent to H1 norm due to Laplace part
%  Lipschitz continuity
%  well-posedness

% TODO Talk about convergence rates?



\begin{figure}
	\centering
	\includeimage{4_solving_hf/fock_fe}
	\caption{Structure of the Finite-Element Fock matrix in beryllium
		with an error of about $0.1$}
	\label{fig:StructureFiniteElementFock}
\end{figure}

% Reduce margin size to fit the matrix plots and switch to landscape
\begin{landscape}
\begin{figure}
	\centering
	\includeimage{4_solving_hf/fock_gaussian}
	\caption{Structure of the Fock matrix for a cGTO-based SCF
		for the beryllium atom
		using a pc-2 basis set.
		The three figures show the SCF with a Pulay error
		Frobenius norm of $0.18$, $0.0063$ and $4.1 \cdot 10^{-7}$
		from left to right.}
		% 12 out of 15 rows are diagonal-dominant in each case.
		% Matrix is symmetric
	\label{fig:StructureGaussianFock}
\end{figure}

\begin{figure}
	\centering
	\includeimage{4_solving_hf/fock_sturmian}
	\caption{Structure of the Fock matrix for a Coulomb-Strumian based SCF
		for the beryllium atom starting from using a $(5,1,1)$
		Coulomb-Strumian basis in $mln$ order
		and a Sturmian exponent of $\kexp = 1.99$.
		The three figures show left to right the Fock matrix
		at an SCF step with a Pulay error Frobenius norm of
		$0.13$, $0.0079$, $6.7 \cdot 10^{-8}$.}
		% 14 out of 17 are diagonal-dominant in each case.
		% Matrix is symmetric
	\label{fig:StructureSturmianFock}
\end{figure}
\end{landscape}

\todoil{Talk about contraction-based methods for the first time.
Allows to pick this up in the next section}

\subsection{Coulomb-Sturmian-type orbitals}
\label{sec:BasisCS}

\begin{figure}
	\centering
	\includeimage{4_solving_hf/local_energy_cs}
	\caption{Local energy of a Coulomb-Sturmian basis sets for hydrogen.}
	\label{fig:LocalEnergyCS}
\end{figure}

\begin{figure}
	\centering
	\includeimage{4_solving_hf/local_energy_cs_zoom}
	\caption{Zoom-in for local energy of Coulomb-Sturmian basis sets for hydrogen.}
	\label{fig:LocalEnergyCSZoom}
\end{figure}

\begin{figure}
	\centering
	\includeimage{4_solving_hf/relative_error_cs}
	\caption{Relative error for Coulomb-Sturmian basis sets for hydrogen.}
	\label{fig:LocalEnergyCS}
\end{figure}



% A few historic words and references
\cite{Rotenberg1970}
\cite{Gruzdev1990}
\cite{Avery2006} % Blue book
\cite{Avery2011} % Purple book
\cite{Morales2016}
\cite{Avery2017} % Fast evaluation of STOs



\begin{equation}
	bla
	\label{eqn:CSEquation}
\end{equation}

% Sturmians form a complete basis for H^2(R^3) and for H^1(R^1) (only the radial part)

% complete but not over-complete

In previous work we have investigated the use of so-called generalised Sturmians as basis functions
in electronic structure theory.
%
\newcommand{\rpack}{\vec{r}_1, \ldots, \vec{r}_N}
Generalised Sturmians $\Phi_\nu(\rpack)$ are the solutions to $N$-body Schrödinger-like equation
\begin{equation}
	\left( -\frac12 \sum_{j=1}^N \Delta_j + \beta_\nu V_0(\rpack) - E \right) \Phi_\nu(\rpack) = 0
	\label{eqn:GenSturm}
\end{equation}
where $V_0(\rpack)$ is
a zero-th order potential.

In case of atoms a good choice is the nuclear attraction:
The electrostatic electron-nucleus interaction
\[
	V_0(\rpack) = - \sum_{j=1}^N \frac{Z}{r_j}.
\]
\todo[inline]{Is many nuclei also possible here?}
Note that the electron-nucleus interaction is scaled by the factor $\beta_\nu$ in \eqref{eqn:GenSturm}
such that the solutions $\Phi_\nu(\rpack)$ all become isoenergetic.
This makes Sturmians reproduce the correct long-range decay behaviour of the electron density
as well as properly represent the nuclear cusp at the core.
As such the functional form of generalised Sturmians is very much related to STOs.
Unlike STOs the two-electron integrals can, however, be reformulated in a particular way
to make computing them less demanding.
This requires, however, that the mathematical properties of the Sturmians
can be fully exploited during the computation,
which in turn requires the Fock matrix to be arranged in a specific way.
Moreover it is advantageous to not build the coulomb and exchange matrix in memory at all,
but much rather only use these matrices in the form of matrix-vector products,
since this overall saves an order of magnitude in the computational scaling.


Coulomb-Sturmian basis function is composed of
\[ \varphi_\mu(\vec{r}) = \]

\todo[inline,caption={}]{
	\begin{itemize}
		\item What are Coulomb-Sturmians and where do they come from
		\item Show what needs to be done
		\item Show problems $\Rightarrow$ Lazy matrices
	\end{itemize}
}

% TODO Things to show:
%    - The larger n, the larger the energy under the action of the fock operator
%    - HF Kinetic energy scales with k: Lower if k lower
%    - HF Nuclear attraction energy scales with k: Higher if k lower
%    - HF electron-electron interaction scales with k: Lower if k lower
% Refer to these results in the chapter about Sturmian calculations

\subsection{Other types of basis functions}
The selection of basis function types presented so far
represents a fair amount of what is used for electronic structure theory
calculations nowadays.
Nevertheless there are few more basis function types,
which should not go unmentioned.

This first of all applies to
plane-wave and projector-augmented wave approaches
~\cite{Kresse1996,Kresse1999,Mortensen2005,Enkovaara2010},
which are both extremely popular as well as extremely suitable for performing
electronic structure calculations
on extended periodic systems or systems in the solid state.
Over the years there has also been an enormous amount of development
into the direction of numerical basis functions.
\citet{Frediani2015} provides an excellent review.
The approaches range from a so-called fully numerical treatment,
where similar to the finite-element method as mentioned above,
the complete problem is treated by grid-based methods.
This includes employing clever
numerical integration grids~\cite{Losilla2012DCRsp,Toivanen2015,Enkovaara2010}
or discretisation schemes
based on finite-differences~\cite{Kobus2013}
or finite-elements~\cite{Tsuchida1995,Briggs1996,Pask05,Lehtovaara2009,Alizadegan2010,Avery2011PhD,Davydov2015,Boffi2016}.
Some approaches~\cite{Soler2002} use a dual representation,
where the same orbitals are both represented on a real-space grid
as well as in the form of orbitals
or they only treat
part of the electronic wavefunction numerically~\cite{Fischer1978,LUCAS}.
Such methods for example employ a factorisation of the one-particle functions
into a numerical radial part and a spherical harmonic function.
Last but not least one should also mention
wavelet-based methods~\cite{Bischoff2011,Bischoff2012,Bischoff2013,Bischoff2014,Bischoff2014a,Bischoff2017},
where quite some progress has been made in recent years.

\subsection{Mixed bases}
Already done in some or another sense in practice.
For example the SIESTA method~\cite{Soler2002} uses
a dual representation on a real-space grid as well as
an atomic orbital basis


Suggest an ansatz basis $\{\varphi_i + \chi_i\}_{i\in\Ibas}$
where $\chi_i$ is a fixed, predetermined Sturmian solution
and $\varphi_i$ are corrections to be found from FE.
Derive the HF energy functional expression for such an ansatz
where $\chi_i$ are fixed and only $\varphi_i$ are the parameters.
(Maybe do this in outlook?)




Guess methods:
Do a calculation in Coulomb-Sturmians for atoms.
Use the result for EHT and project overall orbitals onto FE grid
for fine structure.
