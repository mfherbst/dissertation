\chapter{Numerical solution of the HF equations}
\chapquote{%
I believe there is no philosophical high-road in science,
with epistemological signposts.
No, we are in a jungle and find our way by trial and error,
building our road behind us as we proceed.
}{Max Born~(1882--1970)}
\newcommand{\kexp}{\ensuremath k_\text{exp}}


\section{A brief review of Optimisation theory}
\todo[inline,caption={}]{
	\begin{itemize}
		\item Elaborate on Optimisation theory
		\item See Erics notes
		\item Keep it short, just what is needed for HF
		\item Move to QChem chapter?
	\end{itemize}
}


\section{Overview of standard SCF algorithms}
\todo[inline,caption={}]{
	\begin{itemize}
		\item Talk about the common structures
		\item Introduce a few schematically
		\item Hint at the underlying linear algebra (eigensolvers, linear solvers)
		\item Coefficient vs. density matrix based
		\item DIIS
		\item ODA
		\item tODA
	\end{itemize}
}

\section{Employing Slater-type orbitals}
\defineabbr{STO}{STO\xspace}{Slater-type orbtital}

\section{Employing a contracted Gaussian basis}
\todo[inline,caption={}]{
	\begin{itemize}
		\item Brief rationale
		\item Problems
		\item E.g. Local energy => what happens if $\psi = 0$
	\end{itemize}
}

\section{Employing a Finite Element basis}
\todo[inline,caption={}]{
	\begin{itemize}
		\item Show what needs to be done
		\item Show problems $\Rightarrow$ Lazy matrices
		\item Hint at the underlying linear algebra (eigensolvers, linear solvers)
		\item Multigrid methods
		\item Refinement
		\item How does it connect to the SCF methods
	\end{itemize}
}

\subsection{Introduction to Finite Elements}
\todo[inline,caption={}]{
	\begin{itemize}
		\item Basic ideas and properties
	\end{itemize}
}

% Use as an example to explain the ideas of finite elements
An important equation from classical electrodynamics which
involves the Laplace operator is Poisson's equation
\begin{equation}
	\Delta V_H(\vec{r}) = \rho(\vec{r})
	\label{eqn:Poisson}
\end{equation}
relating the potential to the charge density $\rho$ which generates it.

% TODO 
%  Note that norm wrt fock operator F is equivalent to H1 norm due to Laplace part
%  Lipschitz continuity
%  well-posedness

% TODO Talk about convergence rates?



\begin{figure}
	\centering
	\includeimage{4_solving_hf/fock_fe}
	\caption{Structure of the Finite-Element Fock matrix in beryllium
		with an error of about $0.1$}
	\label{fig:StructureFiniteElementFock}
\end{figure}

% Reduce margin size to fit the matrix plots and switch to landscape
\begin{landscape}
\begin{figure}
	\centering
	\includeimage{4_solving_hf/fock_gaussian}
	\caption{Structure of the Fock matrix for a cGTO-based SCF
		for the beryllium atom
		using a pc-2 basis set.
		The three figures show the SCF with a Pulay error
		Frobenius norm of $0.18$, $0.0063$ and $4.1 \cdot 10^{-7}$
		from left to right.}
		% 12 out of 15 rows are diagonal-dominant in each case.
		% Matrix is symmetric
	\label{fig:StructureGaussianFock}
\end{figure}

\begin{figure}
	\centering
	\includeimage{4_solving_hf/fock_sturmian}
	\caption{Structure of the Fock matrix for a Coulomb-Strumian based SCF
		for the beryllium atom starting from using a $(5,1,1)$
		Coulomb-Strumian basis in $mln$ order
		and a Sturmian exponent of $\kexp = 1.99$.
		The three figures show left to right the Fock matrix
		at an SCF step with a Pulay error Frobenius norm of
		$0.13$, $0.0079$, $6.7 \cdot 10^{-8}$.}
		% 14 out of 17 are diagonal-dominant in each case.
		% Matrix is symmetric
	\label{fig:StructureSturmianFock}
\end{figure}
\end{landscape}

\section{Employing a Coulomb-Sturmian basis}
% A few historic words and references
\cite{Rotenberg1970}
\cite{Gruzdev1990}
\cite{Avery2006} % Blue book
\cite{Avery2011} % Purple book
\cite{Morales2016}
\cite{Avery2017} % Fast evaluation of STOs


\label{sec:BasisCS}
\defineabbr{CS}{CS\xspace}{Coulomb-Strumian or Coulomb-Strumian basis}

\begin{equation}
	bla
	\label{eqn:CSEquation}
\end{equation}

% complete but not over-complete

In previous work we have investigated the use of so-called generalised Sturmians as basis functions
in electronic structure theory.
%
\newcommand{\rpack}{\vec{r}_1, \ldots, \vec{r}_N}
Generalised Sturmians $\Phi_\nu(\rpack)$ are the solutions to $N$-body Schrödinger-like equation
\begin{equation}
	\left( -\frac12 \sum_{j=1}^N \Delta_j + \beta_\nu V_0(\rpack) - E \right) \Phi_\nu(\rpack) = 0
	\label{eqn:GenSturm}
\end{equation}
where $V_0(\rpack)$ is
a zero-th order potential.

In case of atoms a good choice is the nuclear attraction:
The electrostatic electron-nucleus interaction
\[
	V_0(\rpack) = - \sum_{j=1}^N \frac{Z}{r_j}.
\]
\todo[inline]{Is many nuclei also possible here?}
Note that the electron-nucleus interaction is scaled by the factor $\beta_\nu$ in \eqref{eqn:GenSturm}
such that the solutions $\Phi_\nu(\rpack)$ all become isoenergetic.
This makes Sturmians reproduce the correct long-range decay behaviour of the electron density
as well as properly represent the nuclear cusp at the core.
As such the functional form of generalised Sturmians is very much related to STOs.
Unlike STOs the two-electron integrals can, however, be reformulated in a particular way
to make computing them less demanding.
This requires, however, that the mathematical properties of the Sturmians
can be fully exploited during the computation,
which in turn requires the Fock matrix to be arranged in a specific way.
Moreover it is advantageous to not build the coulomb and exchange matrix in memory at all,
but much rather only use these matrices in the form of matrix-vector products,
since this overall saves an order of magnitude in the computational scaling.


Coulomb-Sturmian basis function is composed of
\[ \chi_\mu(\vec{r}) = \]

\todo[inline,caption={}]{
	\begin{itemize}
		\item What are Coulomb-Sturmians and where do they come from
		\item Show what needs to be done
		\item Show problems $\Rightarrow$ Lazy matrices
	\end{itemize}
}

% TODO Things to show:
%    - The larger n, the larger the energy under the action of the fock operator
%    - HF Kinetic energy scales with k: Lower if k lower
%    - HF Nuclear attraction energy scales with k: Higher if k lower
%    - HF electron-electron interaction scales with k: Lower if k lower
% Refer to these results in the chapter about Sturmian calculations

\subsection{UHF and its issues}


\section{Basis-independence of SCF algorithms}
\todo[inline,caption={}]{
	\begin{itemize}
		\item Make sure it comes across that SCF has very similar structure for each basis type
		\item SCF algorithm may be formulated independent of basis type
		\item DFT very similar
		\item[$\Rightarrow$] Basis-function independent quantum chemical modelling
		\item Maybe move to flexible qchem program chapter
	\end{itemize}
}
