\chapter{Numerical approaches for solving the Hartree-Fock problem}
\chaptermark{Numerical approaches for solving \HF}
\label{ch:NumSolveHF}
\chapquote{%
I believe there is no philosophical high-road in science,
with epistemological signposts.
No, we are in a jungle and find our way by trial and error,
building our road behind us as we proceed.
}{Max Born~(1882--1970)}
\newcommand{\kexp}{\ensuremath k_\text{exp}}

This chapter is devoted to an in-depth discussion of
numerical approaches for solving the \HF problem
both when it comes to the basis function type used for the discretisation
and the algorithms for solving the discretised problem.
We will discuss how different basis function types
lead to numerical problems of vastly different structure
and how therefore not every algorithmic ansatz works
for every type of basis function.

In section \ref{sec:DiscreteHF} we noted that there are
roughly three ways to view the discretised \HF problem.
One way would be to think of it as a minimisation of the energy
with respect to the orbital coefficients,
another as a minimisation with respect to the density matrix
and yet a third as a non-linear eigenproblem,
which needs to solved self-consistently.
Our discussion here will generally take the third viewpoint
and only switch to the others when this aids our argument.
Furthermore we will implicitly assume a real-valued
\UHF ansatz in this chapter.
The adaption of the presented results to
\RHF or \ROHF is usually straightforward%
\footnote{
To go from \UHF to \RHF one just needs to consider both blocks of the relevant
Fock, coefficient, and density matrices to be equivalent.
Going from \UHF to \ROHF only amounts to replacing the \UHF Fock matrix
by the appropriately constructed \ROHF Fock matrix before performing
the diagonalisation for getting the new coefficients.
}.

\section{Overview of the \SCF procedure}
In remark \ref{rem:PropertiesDiscretised}
of the previous chapter
we suggested a simple procedure for iteratively solving the \HF equations.
The idea was to start from an initial guess $\mat{C}^{(0)}$
from the Stiefel manifold $\mathcal{C}$ as defined in \eqref{eqn:HFStiefel}
and repetitively construct occupied coefficient
matrices $\mat{C}^{(1)}, \mat{C}^{(2)}, \ldots, \mat{C}^{(n)} \in \mathcal{C}$
by solving the discretised \HF equations \eqref{eqn:HFiterated}
and considering the Aufbau principle.
Since the minimiser of the discretised \HF equations \eqref{eqn:HFOptCoeff}
is unique, there is no need to diagonalise exactly $\matFnfull$ in each iteration.
Instead we could diagonalise an arbitrary matrix $\matFnt$
for obtaining the new coefficients $\mat{C}^{(n+1)}$.
It is only important that the final coefficients $\mat{C}^{(n)}$ are situated
in the Stiefel manifold $\mathcal{C}$
and that the Pulay error \eqref{eqn:PulayError}
vanishes at the end of our iterative self-consistent procedure.
This leads to the following general approach.

\begin{rem}[\SCF procedure]
	\label{rem:SCFcoeff}
	Pick a \newterm{convergence threshold} $\epsilonconv \in \R$,
	a \textbf{basis set} $\{\varphi_\mu\}_{\mu\in\Ibas} \subset H^1(\R^3, \R^2)$
	and an \newterm{initial guess} $\mat{C}^{(0)} \in \mathcal{C}$
	of occupied coefficients.
	From this build an initial Fock matrix
	$\tilde{\mat{F}}^{(0)} = \mat{F}\!\!\left[\mat{C}^{(0)} \tp{\left(\mat{C}^{(0)}\right)}\right].$

	\noindent
	For $n=1,2,3,\ldots$
	\begin{itemize}
		\item Diagonalise
			\[ \tilde{\mat{F}}^{(n-1)} \mat{C}_F^{(n)} = \mat{S} \mat{C}_F^{(n)} \mat{E}^{(n)} \]
			under the condition
			\[ \tp{\left(\mat{C}_F^{(n)}\right)} \mat{S}  \mat{C}_F^{(n)} = \mat{I}_{\Norb} \]
			where
			\[
				\mat{E}^{(n)}
				= \text{diag}\left(\varepsilon_1^{(n)},
				\varepsilon_2^{(n)}, \ldots,
				\varepsilon_{\Norb}^{(n)}\right)
			\]
			is the diagonal matrix of orbital eigenvalues.
		\item Construct the occupied matrix $\mat{C}^{(n)}$
			from the full matrix $\mat{C}_F^{(n)}$ by the Aufbau principle.
		\item Build the Fock matrix $\matFnfull$.
		\item Compute $\mat{e}^{(n)}$ according to \eqref{eqn:PulayError}
			\[
			\mat{e}^{(n)}
			= \matFnfull \mat{C}^{(n)} \tp{\left(\mat{C}^{(n)}\right)} \, \mat{S}
			- \mat{S} \, \mat{C}^{(n)} \tp{\left(\mat{C}^{(n)}\right)} \matFnfull
			\]
			If $\norm{ \mat{e}^{(n)} }_\text{frob} \leq  \epsilonconv$
			the procedure is converged%
			\footnote{In finite dimensions all norms are equivalent,
				so the choice of the Frobenius norm is arbitrary here}
			with final coefficients $\mat{C}_0 \equiv \mat{C}^{(n)}$.
		\item Build a Fock matrix $\tilde{\mat{F}}^{(n)}$ somehow
			using $\mat{C}^{(n)}$ and all insight into the problem gathered so far.
	\end{itemize}
	The final \HF energy is given by $\mathcal{E}_C(\mat{C}_0)$
	according to \eqref{eqn:HFEnergyFunctionalCoeff} and the final
	\SCF orbitals $\Theta^0$ by \eqref{eqn:HFDiscretisationAnsatz}.
\end{rem}
This scheme still leaves a couple of important questions unanswered,
which we will address in the following sections, namely:
\begin{itemize}
	\item What is a suitable method for choosing the initial guess $\mat{C}^{(0)}$?
	\item What type of basis function is suitable?
	\item What algorithms are sensible for building the next
		Fock matrix guess $\tilde{\mat{F}}^{(n)}$?
\end{itemize}

The scheme we presented in remark \ref{rem:SCFcoeff}
considers the \HF problem to be parametrised
in terms of the occupied coefficients $\mat{C}^{(n)}$
and solves it by producing a sequence of coefficients
$\mat{C}^{(1)}, \mat{C}^{(2)}, \ldots, \mat{C}^{(n)} \in \mathcal{C}$
until convergence.
By the arguments discussed in section \ref{sec:DiscreteHF}
one can alternatively parametrise the \HF problem in terms of
density matrices $\mat{D}^{(n)}$.
In this light some \SCF algorithms are better understood
if one thinks about them as schemes producing
a sequence of density matrices
$\mat{D}^{(0)}, \mat{D}^{(1)}, \ldots, \mat{D}^{(n)} \in \mathcal{P}$
instead.
See the optimal damping algorithm in section \vref{sec:ODA} for an example.
To distinguish both approaches,
the first kinds of algorithms iterating $\mat{C}^{(n)}$
will be called \newterm{coefficient-based \SCF} schemes
whilst the second kind of algorithms iterating $\mat{D}^{(n)}$
we will call \newterm{density-based \SCF} algorithms.

\noindent
The identification
\[ \mat{D}^{(n)} = \mat{C}^{(n)} \tp{\left(\mat{C}^{(n)}\right)}, \]
which we already presented in \eqref{eqn:DensityReplacement} in section \ref{sec:DiscreteHF},
allows to build the density matrix from the coefficients by a matrix-matrix product
and in the reverse direction we can find matching coefficients for each density
matrix by a factorisation,
\eg a diagonalisation or a singular-value decomposition%
\footnote{Thanks to Eric Cancès for pointing me to the factorisation ansatz
for the reverse direction.}.
This allows --- at least theoretically --- to convert
every density-based algorithms
into a coefficient-based scheme like remark \ref{rem:SCFcoeff}
and vice versa.
In practice the factorisation from density matrices to coefficients
could become rather costly and might not be always applicable.

\section{Guess methods}
% Intertwined with Basis choice


\defineabbr{CS}{CS\xspace}{Coulomb-Sturmian or Coulomb-Sturmian basis, see section \vref{sec:BasisCS}}
\defineabbr{cGTO}{cGTO\xspace}{Contracted Gaussian-type orbitals, see section \vref{sec:cGTO}}
\defineabbr{STO}{STO\xspace}{Slater-type orbitals, see section \vref{sec:STO}}
\defineabbr{ETO}{ETO\xspace}{Exponential-type orbitals, orbitals with radial part of the form Polynomial in $r$ times $\exp{-\alpha r}$ with some parameter $\alpha$.}
\defineabbr{FE}{FE\xspace}{Finite elements, see section \vref{sec:FE}}
\defineabbr{ACO}{ACO\xspace}{atom-centered orbitals}

\section{Basis function types}
This section tries to address the question,
which classes of functions can be used
in order to build a basis set $\{\varphi_\mu\}_{\mu\in\Ibas}$
for solving the \HF problem in an \SCF procedure.
For this we will first introduce some desirable properties for a basis set,
both motivated from the desire to represent the physics of the electronic Schrödinger
equation as good as possible
as well as requirements from the numerical side.
In the light of this,
we will discuss four types of basis functions in depth,
namely the historic Slater-type orbitals~({\STO}s),
the most commonly employed contracted Gaussian-type orbitals~({\cGTO}s),
a finite-element based discretisation method
as an example of a fully numerical approach
as well as so-called Coulomb-Sturmian-type orbitals.

Even though we mostly concentrate on the \HF problem in this section,
quite a few of the observations made here
apply to \DFT or methods going beyond Hartree-Fock as well.
In this sense the outlined discussion
can be seen as an example case for the use of the mentioned
basis function types in electronic structure theory as a whole.

\subsection{Desirable properties}
Why can we not use every kind of function?
Issue with evaluating the integrals
get the correct physics



\begin{rem}
	This remark provides a short recap of some
	fundamental physical properties of the electronic Schrödinger equation.

Kato's cusp condition~\cite{Kato1957}
Energy-dependent exponential decay at infinity
(easy to rationalise physically)
\end{rem}


Apart from relative and absolute errors a concept,
which is very helpful for investigating how well a basis represents
the wavefunction is the \newterm{local energy}
\begin{defn}
	Local energy
	Trial wavefunction $\Psi$
	\[
		E_L(\vec{x}) \equiv \frac{\Op{H}_{\Nelec} \Phi(\vec{x})}{\Phi(\vec{x})}
	\]
	approximately constant for good approximations.
\end{defn}
This concept originates from the quantum Monte Carlo community%
~\cite{Foulkes2001,Ma2005}
where the 
% Ma paper has a good explanation for local energy and what it means
% physically




\subsection{Slater-type orbitals}
\label{sec:STO}

\subsection{Contracted Gaussian-type orbitals}
\label{sec:cGTO}
\todo[inline,caption={}]{
	\begin{itemize}
		\item Brief rationale
		\item Problems
		\item E.g. Local energy => what happens if $\psi = 0$
	\end{itemize}
}
% Convention:
%    Use \chi_\mu for primitives


\begin{figure}
	\centering
	\includeimage{4_solving_hf/local_energy_cgto}
	\caption{Local energy of contracted Gaussian basis sets for hydrogen.}
	\label{fig:LocalEnergyCgto}
\end{figure}

\begin{figure}
	\centering
	\includeimage{4_solving_hf/local_energy_cgto_zoom}
	\caption{Zoom-in for local energy of contracted Gaussian basis sets for hydrogen.}
	\label{fig:LocalEnergyCgtoZoom}
\end{figure}

\begin{figure}
	\centering
	\includeimage{4_solving_hf/relative_error_cgto}
	\caption{Relative error of a selection of contracted Gaussian basis sets
		for hydrogen.}
	\label{fig:LocalEnergyCgto}
\end{figure}



\subsection{Finite-element based discretisation}
\label{sec:FE}
\todo[inline,caption={}]{
	\begin{itemize}
		\item Show what needs to be done
		\item Show problems $\Rightarrow$ Lazy matrices
		\item Hint at the underlying linear algebra (eigensolvers, linear solvers)
		\item Multigrid methods
		\item Refinement
		\item How does it connect to the SCF methods
	\end{itemize}
}

% Use as an example to explain the ideas of finite elements
An important equation from classical electrodynamics which
involves the Laplace operator is Poisson's equation
\begin{equation}
	\Delta V_H(\vec{r}) = \rho(\vec{r})
	\label{eqn:Poisson}
\end{equation}
relating the potential to the charge density $\rho$ which generates it.

% TODO 
%  Note that norm wrt fock operator F is equivalent to H1 norm due to Laplace part
%  Lipschitz continuity
%  well-posedness

% TODO Talk about convergence rates?



\begin{figure}
	\centering
	\includeimage{4_solving_hf/fock_fe}
	\caption{Structure of the Finite-Element Fock matrix in beryllium
		with an error of about $0.1$}
	\label{fig:StructureFiniteElementFock}
\end{figure}

% Reduce margin size to fit the matrix plots and switch to landscape
\begin{landscape}
\begin{figure}
	\centering
	\includeimage{4_solving_hf/fock_gaussian}
	\caption{Structure of the Fock matrix for a cGTO-based SCF
		for the beryllium atom
		using a pc-2 basis set.
		The three figures show the SCF with a Pulay error
		Frobenius norm of $0.18$, $0.0063$ and $4.1 \cdot 10^{-7}$
		from left to right.}
		% 12 out of 15 rows are diagonal-dominant in each case.
		% Matrix is symmetric
	\label{fig:StructureGaussianFock}
\end{figure}

\begin{figure}
	\centering
	\includeimage{4_solving_hf/fock_sturmian}
	\caption{Structure of the Fock matrix for a Coulomb-Strumian based SCF
		for the beryllium atom starting from using a $(5,1,1)$
		Coulomb-Strumian basis in $mln$ order
		and a Sturmian exponent of $\kexp = 1.99$.
		The three figures show left to right the Fock matrix
		at an SCF step with a Pulay error Frobenius norm of
		$0.13$, $0.0079$, $6.7 \cdot 10^{-8}$.}
		% 14 out of 17 are diagonal-dominant in each case.
		% Matrix is symmetric
	\label{fig:StructureSturmianFock}
\end{figure}
\end{landscape}

\todoil{Talk about contraction-based methods for the first time.
Allows to pick this up in the next section}

\subsection{Coulomb-Sturmian-type orbitals}
\label{sec:BasisCS}

\begin{figure}
	\centering
	\includeimage{4_solving_hf/local_energy_cs}
	\caption{Local energy of a Coulomb-Sturmian basis sets for hydrogen.}
	\label{fig:LocalEnergyCS}
\end{figure}

\begin{figure}
	\centering
	\includeimage{4_solving_hf/local_energy_cs_zoom}
	\caption{Zoom-in for local energy of Coulomb-Sturmian basis sets for hydrogen.}
	\label{fig:LocalEnergyCSZoom}
\end{figure}

\begin{figure}
	\centering
	\includeimage{4_solving_hf/relative_error_cs}
	\caption{Relative error for Coulomb-Sturmian basis sets for hydrogen.}
	\label{fig:LocalEnergyCS}
\end{figure}



% A few historic words and references
\cite{Rotenberg1970}
\cite{Gruzdev1990}
\cite{Avery2006} % Blue book
\cite{Avery2011} % Purple book
\cite{Morales2016}
\cite{Avery2017} % Fast evaluation of STOs



\begin{equation}
	bla
	\label{eqn:CSEquation}
\end{equation}

% Sturmians form a complete basis for H^2(R^3) and for H^1(R^1) (only the radial part)

% complete but not over-complete

In previous work we have investigated the use of so-called generalised Sturmians as basis functions
in electronic structure theory.
%
\newcommand{\rpack}{\vec{r}_1, \ldots, \vec{r}_N}
Generalised Sturmians $\Phi_\nu(\rpack)$ are the solutions to $N$-body Schrödinger-like equation
\begin{equation}
	\left( -\frac12 \sum_{j=1}^N \Delta_j + \beta_\nu V_0(\rpack) - E \right) \Phi_\nu(\rpack) = 0
	\label{eqn:GenSturm}
\end{equation}
where $V_0(\rpack)$ is
a zero-th order potential.

In case of atoms a good choice is the nuclear attraction:
The electrostatic electron-nucleus interaction
\[
	V_0(\rpack) = - \sum_{j=1}^N \frac{Z}{r_j}.
\]
\todo[inline]{Is many nuclei also possible here?}
Note that the electron-nucleus interaction is scaled by the factor $\beta_\nu$ in \eqref{eqn:GenSturm}
such that the solutions $\Phi_\nu(\rpack)$ all become isoenergetic.
This makes Sturmians reproduce the correct long-range decay behaviour of the electron density
as well as properly represent the nuclear cusp at the core.
As such the functional form of generalised Sturmians is very much related to STOs.
Unlike STOs the two-electron integrals can, however, be reformulated in a particular way
to make computing them less demanding.
This requires, however, that the mathematical properties of the Sturmians
can be fully exploited during the computation,
which in turn requires the Fock matrix to be arranged in a specific way.
Moreover it is advantageous to not build the coulomb and exchange matrix in memory at all,
but much rather only use these matrices in the form of matrix-vector products,
since this overall saves an order of magnitude in the computational scaling.


Coulomb-Sturmian basis function is composed of
\[ \varphi_\mu(\vec{r}) = \]

\todo[inline,caption={}]{
	\begin{itemize}
		\item What are Coulomb-Sturmians and where do they come from
		\item Show what needs to be done
		\item Show problems $\Rightarrow$ Lazy matrices
	\end{itemize}
}

% TODO Things to show:
%    - The larger n, the larger the energy under the action of the fock operator
%    - HF Kinetic energy scales with k: Lower if k lower
%    - HF Nuclear attraction energy scales with k: Higher if k lower
%    - HF electron-electron interaction scales with k: Lower if k lower
% Refer to these results in the chapter about Sturmian calculations

\subsection{Other types of basis functions}
The selection of basis function types presented so far
represents a fair amount of what is used for electronic structure theory
calculations nowadays.
Nevertheless there are few more basis function types,
which should not go unmentioned.

This first of all applies to
plane-wave and projector-augmented wave approaches
~\cite{Kresse1996,Kresse1999,Mortensen2005,Enkovaara2010},
which are both extremely popular as well as extremely suitable for performing
electronic structure calculations
on extended periodic systems or systems in the solid state.
Over the years there has also been an enormous amount of development
into the direction of numerical basis functions.
\citet{Frediani2015} provides an excellent review.
The approaches range from a so-called fully numerical treatment,
where similar to the finite-element method as mentioned above,
the complete problem is treated by grid-based methods.
This includes employing clever
numerical integration grids~\cite{Losilla2012DCRsp,Toivanen2015,Enkovaara2010}
or discretisation schemes
based on finite-differences~\cite{Kobus2013}
or finite-elements~\cite{Tsuchida1995,Briggs1996,Pask05,Lehtovaara2009,Alizadegan2010,Avery2011PhD,Davydov2015,Boffi2016}.
Some approaches~\cite{Soler2002} use a dual representation,
where the same orbitals are both represented on a real-space grid
as well as in the form of orbitals
or they only treat
part of the electronic wavefunction numerically~\cite{Fischer1978,LUCAS}.
Such methods for example employ a factorisation of the one-particle functions
into a numerical radial part and a spherical harmonic function.
Last but not least one should also mention
wavelet-based methods~\cite{Bischoff2011,Bischoff2012,Bischoff2013,Bischoff2014,Bischoff2014a,Bischoff2017},
where quite some progress has been made in recent years.

\subsection{Mixed bases}
Already done in some or another sense in practice.
For example the SIESTA method~\cite{Soler2002} uses
a dual representation on a real-space grid as well as
an atomic orbital basis


Suggest an ansatz basis $\{\varphi_i + \chi_i\}_{i\in\Ibas}$
where $\chi_i$ is a fixed, predetermined Sturmian solution
and $\varphi_i$ are corrections to be found from FE.
Derive the HF energy functional expression for such an ansatz
where $\chi_i$ are fixed and only $\varphi_i$ are the parameters.
(Maybe do this in outlook?)




Guess methods:
Do a calculation in Coulomb-Sturmians for atoms.
Use the result for EHT and project overall orbitals onto FE grid
for fine structure.

\section{Self-consistent field algorithms}
\label{sec:SCFAlgorithms}

In this section we want to discuss a few standard self-consistent field algorithms
in the light of the various types of basis functions we discussed in the previous section.
Even though it is my hope that the selection of algorithms discussed here
is representative,
the vast number of methods,
which has been developed over the years,
makes it impossible to be exhaustive.

Most \SCF algorithms are designed only with a \cGTO-based
discretisation of the \HF and Kohn-Sham \DFT problem in mind.
The deviating numerical properties of the finite-element method
or a \CS-based discretisation,
therefore often call for minor modifications of the schemes.
For example both finite elements as well as Coulomb-Sturmians
favour \contract-based methods
due to the better scaling of equations like \eqref{eqn:ExchangeApply}
and \eqref{eqn:ApplicationKcs}
compared to building the full matrix.
Therefore the Fock matrix might not be built in memory any more,
which implies that a linear combination of Fock matrices
cannot be computed in memory either.
This does not imply that \SCF schemes,
which form linear combinations of Fock matrices
are completely ruled out,
but they might become less favourable compared to other schemes.
See section \ref{sec:DIIS} below for details.

On the other hand in \FE-based approaches all quantities,
which scale quadratically in $\Nbas$ cannot be stored in memory.
This applies not only to the iterated Fock matrix $\mat{F}^{(n)}$,
but to the density matrix $\mat{D}^{(n)} \in \R^{\Nbas\times \Nbas}$ as well.
Even though clever low-rank approximation methods
like hierarchical matrices~\cite{Hackbusch1999,Hackbusch2002,Grasedyck2003,Hackbusch2015}
or tensor decomposition methods
\todo{cite}
could reduce the memory footprint of the density matrix,
this work will try to indicate ways
by which building the density matrix in an \SCF can be avoided.
Naturally this implies
a focus on coefficient-based \SCF schemes as well,
where the number of iterated parameters
--- the coefficient matrix $\mat{C} \in \R^{\Nbas\times\Norb}$ ---
scales only linearly in $\Nbas$.
Furthermore coefficient-based \SCF schemes have the advantage,
that iterating the density matrix
destroys the possibility to perform the
linear-scaling application of $\mat{K}$ for \CS-based methods,
see equation \eqref{eqn:ApplicationKcs} for details.

\todoil{Pagebreak here?}
\pagebreak
It was already pointed out in section \vref{sec:OverviewSCF}
that focusing on coefficient-based schemes
is hardly a restriction in terms of the number of possible approaches,
since coefficient-based and density-based schemes can be interconverted,
at least approximately.
We will suggest modifications of the
optimal damping algorithm~(ODA)~\cite{Cances2000a} in section \ref{sec:tODA}
and Pulay's direct inversion of the iterative subspace~\cite{Pulay1982}
in section \ref{sec:DIIS},
which bring these methods to the coefficient-based setting.

Most of the \SCF algorithms we will consider here
only converge the \HF equations \eqref{eqn:HFiterated}
until the Pulay error \eqref{eqn:PulayError} vanishes
following our general description in remark \vref{rem:SCFcoeff}.
Regarding the \HF optimisation problem \eqref{eqn:HFOptCoeff}
this is only the necessary condition for a stationary point on the Stiefel
manifold $\mathcal{C}$.
Only some \SCF algorithms
termed \newterm{second-order self-consistent field method}s
guarantee (usually only appropriately)
that the stationary point they find is truly a minimum.
They are briefly considered in section \ref{sec:SOSCF}.

\subsection{Roothaan repeated diagonalisation}
\label{sec:RoothaanRepeatedDiag}

Roothaan's repeated diagonalisation~\cite{Roothaan1951}
approach to the \HF problem \eqref{eqn:HFiterated} is by far the simplest.
In the formalism of remark \vref{rem:SCFcoeff}
this algorithm can be described by building the next Fock matrix
$\tilde{\mat{F}}^{(n)}$ only by considering
the current occupied coefficients $\mat{C}^{(n)}$,
\ie $\tilde{\mat{F}}^{(n)} = \matFnfull$.
The two-step iteration procedure of figure \vref{fig:RoothanRepeatedDiag} results.
\begin{figure}
	\centering
	\missingfigure{Roothaan algorithm diagram}
	\caption{Roothaan algorithm diagram}
	\label{fig:RoothanRepeatedDiag}
\end{figure}

Even though Roothaan's algorithm already works for a few simple cases,
it is far from being reliable.
For example one can show~\cite{Cances2000,Cances2000b} that it
either converges to a stationary point of the
discretised \HF problem \eqref{eqn:HFOptCoeff}
or alternatively it numerically oscillates between two states,
where none of them are a stationary point of \eqref{eqn:HFOptCoeff}.
In practice which of these cases occurs is dependent on the system
as well as the basis set.
Furthermore there is no guarantee that the resulting stationary point
the Roothaan's algorithm finds is the \HF ground state.
All these cases are already found for \HF calculations
on the first three periods of the periodic table~\cite{Cances2000}.

\subsection{Level-shifting modification}
If one uses essentially the same \SCF scheme as \ref{fig:RoothanRepeatedDiag}
but instead diagonalises the matrix
\[ \tilde{\mat{F}}^{(n)} = \matFnfull
	- b \mat{S} \, \mat{C}^{(n)}\!\tp{\left(\mat{C}^{(n)}\right)} \, \mat{S} \]
where $b > 0$,
already a much better convergence is achieved.
This modification is called \newterm{level shifting}%
~\cite{Saunders1973,Guest1974},
where $b$ is the \textbf{level-shifting parameter},
typically chosen in the range between $0.1$ and $0.5$.
Effectively this modification increases the energy gap between occupied and virtual
orbital energies.
To see this, let us consider the converged case, where
\[ \mat{F} \mat{C}_F = \mat{S} \mat{C}_F \mat{E} \]
exactly and let us partition the full coefficient matrix%
\footnote{We assume \RHF here and furthermore only consider the $\alpha$ block.
	For \UHF the analysis is exactly the same with
	the relevant equations just replicated in $\alpha$ and $\beta$ block.}
\[
	\mat{C}_F = \mm{\mat{C} & \mat{C}_\text{virt}}
\]
into occupied and virtual parts.
Now let $\tilde{\mat{F}} = \mat{F} - b \mat{S} \, \mat{C}\tp{\mat{C}} \, \mat{S}$
such that
\begin{align*}
	\tilde{\mat{F}} \mat{C}_F
	&= \left( \mat{F} - b \mat{S} \, \mat{C}\tp{\mat{C}} \, \mat{S} \right) \mat{C}_F \\
	&= \mat{S} \mat{C}_F \mat{E} - b \mat{S} \mm{
		\mat{C}\tp{\mat{C}} \mat{S} \mat{C} &
		\mat{C}\tp{\mat{C}} \mat{S} \mat{C}_\text{virt}
	} \\
	&= \mat{S} \mat{C}_F \mat{E} - b \mat{S} \mm{ \mat{C} & 0 } \\
	&= \mat{S} \mat{C}_F \mat{E} + \mat{S} \mat{C}_F \mm{ -b  & 0 } \\
	&= \mat{S} \mat{C}_F \tilde{\mat{E}}
\end{align*}
where
\[ \tilde{\mat{E}} = \diag\left(\varepsilon_1 - b, \varepsilon_2 - b, \ldots,
	\varepsilon_{\Nelec} - b, \varepsilon_{\Nelec+1}, \ldots, \varepsilon_{\Norb}\right).
\]
In other words the virtual orbitals are unaffected
whereas the occupied orbitals are shifted downwards in energy by an amount $b$.

The effect of this is that coupling between both orbital spaces is reduced,
which tends to lead to faster convergence
especially if the gap between $\varepsilon_{\Nelec}$ and $\varepsilon_{\Nelec+1}$ is small.
This empirical observation is backed up by more sophisticated
mathematical analysis by \citet{Cances2000b}.
Their result shows that for large enough $b$,
the level-shifted Roothaan procedure is guaranteed to converge to a stationary
point of the \HF problem \eqref{eqn:HFOptCoeff}.
The also provide an expression for the lower bound of $b$.
In this manner convergence to a stationary point
can be forced even for cases where the original \HF equations \eqref{eqn:HFMO}
have no solution (like the negative ions with $N > 2Z + M$).
In such a case the result is no physical ground state, however.

One can show~\cite{Saunders1973}
that the level-shifting modification is mathematically equivalent
to another modification of Roothaan's repeated diagonalisation,
called \newterm{damping}.
In this procedure one chooses a \newterm{damping factor} $0 < \alpha < 1$
and sets
\begin{equation}
	\tilde{\mat{F}}^{(n)} = (1-\alpha) \tilde{\mat{F}}^{(n-1)} + \alpha \matFnfull,
	\label{eqn:FockDamping}
\end{equation}
such that the new Fock matrix to diagonalise contains still a share
of the old Fock matrix.

\defineabbr{ODA}{ODA\xspace}{Optimal damping algorithm}
\subsection{Optimal damping algorithm}
\label{sec:ODA}
The optimal damping algorithm~(\ODA) was proposed by \citet{Cances2000a}
based on their analysis of the Roothaan algorithm
including the level-shifting modification.

\begin{figure}
	\centering
	\missingfigure{Optimal damping algorithm}
	\caption{Optimal damping algorithm}
	\label{fig:OptimalDampingAlgorithm}
\end{figure}
In unmodified form~\cite{Cances2000,Cances2000a} it is a density-based \SCF algorithm.
Starting from an initial density $\tilde{\mat{D}}^{(0)} = \mat{D}^{(0)}$,
the procedure is roughly~(compare figure \ref{fig:OptimalDampingAlgorithm})
for $n=1, 2, 3, \ldots$
\begin{itemize}
	\item Build the Fock matrix
	\begin{equation}
	\tilde{\mat{F}}^{(n-1)} = \mat{F}\left[ \tilde{\mat{D}}^{(n)} \right]
		\label{eqn:ODAupFock}
	\end{equation}
	and diagonalise it to obtain the new coefficient $\mat{C}_F^{(n)}$.
	Form the new density $\mat{D}^{(n)}$ according to the Aufbau principle from these
	as
	\[ \mat{D}^{(n)} = \mat{C}^{(n)} \tp{\left( \mat{C}^{(n)} \right)}. \]
	%
	\item Evaluate the Pulay error $\mat{e}^{(n)}$ \eqref{eqn:PulayError}
		and end the process if $\norm{\mat{e}^{(n)}}_\text{frob} \leq \epsilonconv$.
	%
	\item Set
		\begin{equation}
			\tilde{\mat{D}}^{(n+1)}
			= \arginf_{\tilde{\mat{D}} \in
			\text{Seg}\left[\tilde{\mat{D}}^{(n)}, \mat{D}^{(n+1)}\right]}
			\mathcal{E}_D^\text{HF}[\tilde{\mat{D}}]
			\label{eqn:ODAupDens}
		\end{equation}
		where
		\[
			\text{Seg}\left[\mat{D}_1, \mat{D}_2\right]
				= \left\{ (1-\lambda) \mat{D}_1 + \lambda \mat{D}_2 \, \Big|\,
				\lambda \in [0,1] \right\}
		\]
		is a line segment of density matrices
		and the energy functional $\mathcal{E}_D^\text{HF}$
		is defined as in \eqref{eqn:HFEnergyFunctionalDens}
		and repeat the process.
\end{itemize}
The remaining question to complete the picture of the \ODA
is how to obtain minimal density $\tilde{\mat{D}}^{(n+1)}$ along the density segment.
First notice that in general the density matrix segment
\[\text{Seg}\left[\mat{D}_1, \mat{D}_2\right] \not\subset \mathcal{P}\]
even if $\mat{D}_1, \mat{D}_2 \in \mathcal{P}$.
Much rather this line segment is fully contained only in a superset
$\tilde{\mathcal{P}} \supset \mathcal{P}$,
where we relax the constraint $\mat{D}^2 = \mat{D}$ to%
\footnote{Let $\mat{A}, \mat{B} \in \R^{n\times n}$,
then $\mat{A} \leq \mat{B}
\Leftrightarrow \forall \vec{x} \in \R^n \
\tp{\vec{x}}\mat{A} \vec{x} \leq \tp{\vec{x}}\mat{B}\vec{x}$}
$\mat{D}^2 \leq \mat{D}$.
See \cite{Cances2000} for details.
For ease of notation let us define
%
\newcommand{\Gfct}[1]{  \mat{G}\left[ #1 \right]   }
\newcommand{\Pet}[1]{\textcolor{dkred}{\tilde{\mat{D}}^{(#1)}}}
\newcommand{\Pe}[1]{\textcolor{dkblue}{\mat{D}^{(#1)}}}
\newcommand{\Fot}[1]{\textcolor{dkred}{\tilde{\mat{F}}^{(#1)}}}
\newcommand{\Fo}[1]{\textcolor{dkblue}{\mat{F}^{(#1)}}}
\newcommand{\la}[1]{\lambda^{(#1)}}
\newcommand{\ila}[1]{\left( 1 - \la{#1} \right)}
\newcommand{\EHFD}[1]{\mathcal{E}_D^\text{HF}\left[#1\right]}
%
\newcommand{\Pn}{\Pet{n}}
\newcommand{\Pnn}{\Pe{n+1}}
\begin{align*}
	E_1 \left[\mat{D}\right] &\equiv \tr\left( \mat{T} \mat{D} + \mat{V}_0 \mat{D} \right), \\
	\mat{G}\left[ \mat{D} \right] &\equiv
		\mat{F}\left[ \mat{D} \right] + \mat{K}\left[ \mat{D} \right] \\
\intertext{and}
	E_2 \left[\mat{D}\right] &\equiv
		\frac12 \tr\left( \mat{D} \, \mat{G}\left[ \mat{D} \right] \right).
\end{align*}
For all matrices $\mat{D}_1, \mat{D}_2 \in \tilde{\mathcal{P}}$
we can show the properties~\cite{Cances2000a}
\begin{align}
	\label{eqn:TermGsymmetry}
	\tr \left( \mat{D}_1 \mat{G}\left[ \mat{D}_2 \right] \right)
		&= \tr \left( \mat{D}_2 \mat{G}\left[ \mat{D}_1 \right] \right) \\
	\label{eqn:GmatTrace}
	\tr\left( \mat{F}\left[ \mat{D}_1 \right] \mat{D}_2 \right)
	&= E_1[\mat{D}_2] + \tr\left( \mat{D}_1 \Gfct{\mat{D}_2} \right)
\end{align}
These imply for $E_2$ and arbitrary $\alpha, \beta \in \R$
\begin{equation}
	\begin{aligned}
	E_2[\alpha \mat{D}_1 + \beta\mat{D}_2]
	&= \frac 1 2 \tr\left( \alpha^2  \mat{D}_1 \Gfct{\mat{D_1}} \right)
		+ \frac 1 2 \tr\left( \alpha\beta \mat{D_1} \Gfct{\mat{D}_2} \right) \\
		\nonumber
		&\hspace{30pt}
		+ \frac 1 2 \tr\left( \alpha\beta \mat{D}_2 \Gfct{\mat{D_1}} \right)
		+ \frac 1 2 \tr\left( \beta^2 \mat{D}_2 \Gfct{\mat{D}_2} \right) \\
	 &\stackrel{\eqref{eqn:GmatTrace}}{=}
	 \alpha^2 E_2[\mat{D_1}] + \beta^2 E_2[\mat{D}_2]
	 + \alpha\beta \tr\left( \mat{D_1} \Gfct{\mat{D}_2} \right),
	\end{aligned}
	\label{eqn:HFE2}
\end{equation}
whereas $E_1$ is linear
\begin{equation}
	E_1[\alpha \mat{D}_1 +\beta \mat{D}_2] = \alpha E_1[\mat{D}_1] + \beta E_1[\mat{D}_2].
	\label{eqn:HFE1}
\end{equation}
These results allow to expand
the \HF energy for a member $\Pet{n+1}$
of the density matrix segment
$\text{Seg}\left[\Pet{n}, \Pe{n+1}\right]$
as
\begin{equation}
\begin{aligned}
	\EHFD{\Pet{n+1}} &= \EHFD{\ila{n+1} \Pet{n} + \la{n+1} \Pe{n+1}} \\
	&= \EHFD{\Pet{n} + \la{n+1} \left( \Pe{n+1}  - \Pet{n} \right)} \\
	&= E_1\left[ \Pet{n} + \la{n+1} \left( \Pe{n+1}  - \Pet{n} \right) \right] \\
		&\hspace{50pt}+E_2\left[\Pet{n} + \la{n+1} \left( \Pe{n+1}  - \Pet{n} \right) \right] \\
	&= E_1[\Pet{n}] + \la{n+1} E_1\left[\Pe{n+1}  - \Pet{n}\right] + E_2[\Pet{n}]\\
		&\hspace{50pt} + \la{n+1} \tr \left( \Pet{n} \Gfct{\Pe{n+1}-\Pet{n}} \right) \\
		&\hspace{50pt}
		+ \left( \la{n+1} \right)^2 E_2[\Pe{n}-\Pet{n}] \\
	&= \EHFD{\Pet{n}}
	+ \la{n+1} \underbrace{\tr\left( \Pet{n} \mat{F} \left[ \Pe{n+1}-\Pet{n} \right] \right)}_{=s} \\
	&\hspace{50pt}
	+ \left( \la{n+1} \right)^2 \underbrace{E_2[\Pe{n}-\Pet{n}]}_{=c} \\
	&= \EHFD{ \Pet{n} } + \la{n+1} s + \left( \la{n+1} \right)^2 c
\end{aligned}
\label{eqn:ODAquadratic}
\end{equation}
The coefficients $s$ and $c$ can alternatively be written as
\begin{equation}
\begin{aligned}
	s &= \tr\left( \Fot{n} \left( \Pe{n+1} - \Pet{n} \right) \right) \\
	&= \tr \left( \Fot{n} \Pe{n+1} \right) - E_\text{HF}[\Pet{n}] - E_2 [\Pet{n}] \\
	&= \tr \left( \Fot{n} \Pe{n+1} \right) - E_1[\Pet{n}] - 2 E_2 [\Pet{n}]  \\
\end{aligned}
\label{eqn:ODAs}
\end{equation}
and\footnote{Note that the original paper \cite{Cances2000a} uses a deviating formalism %
which causes an extra factor of $2$ to appear in their expression for $c$.}
\begin{equation}
\begin{aligned}
	c &= E_2\left[ \Pe{n+1} - \Pet{n} \right] \\
	&\stackrel{\eqref{eqn:HFE2}}{=}
		E_2 [\Pe{n+1}] - \tr\left( \Gfct{\Pet{n}} \Pe{n+1} \right) + E_2 [\Pet{n}] \\
	&= E_2 [\Pe{n+1}] - \tr \left( \Fot{n} \Pe{n+1} \right) + E_1[\Pe{n+1}] + E_2 [\Pet{n}]
\end{aligned}
\label{eqn:ODAc}
\end{equation}
Now the stationary point along the density matrix segment
can be determined by differentiating \eqref{eqn:ODAquadratic} resulting in
\begin{align*}
	\frac{\partial \EHFD{\Pet{n+1}}}{\partial \la{n+1}}
	&= s + 2 \la{n+1} c
	&&\text{and}&
	\frac{\partial^2 \EHFD{\Pet{n+1}}}{\partial \left(\la{n+1}\right)^2} &= 2c
\end{align*}
Due to $E_2[\mat{D}] \geq 0$~\cite{Cances2000a} for all $\mat{D} \in \tilde{\mathcal{P}}$
one easily deduces $c \geq 0$,
such that the stationary point of the above expression is always a minimum.
Since $\la{n+1} \in [0,1]$ the minimiser is
\begin{equation}
	\label{eqn:ODALambdaOpt}
	\la{n+1}_\text{min} = \left\{
	\begin{array}{cl}
		1 & \text{if $2c \leq -s$} \\
		- \frac{s}{2c} & \text{else}
	\end{array}
	\right.,
\end{equation}
where the cases $c=0$ and $s=0$ have been ignored, since they only occur at convergence.
This closes the missing link and allows to implement a \ODA
in as a density-based \SCF.

Let $\alpha, \beta \in \R$ and $\mat{D}_1, \mat{D}_2 \in \tilde{\mathcal{P}}$.
Since $\matFfullD = \mat{T} + \mat{V}_0 + \matJfullD + \matKfullD$
and the two-electron terms are linear in the density matrix, we have
\begin{equation}
	\mat{F}\!\left[\alpha \mat{D}_1 + \beta\mat{D}_2\right]
	= \alpha \mat{F}\!\left[\mat{D}_1\right] + \beta \mat{F}\!\left[\mat{D}_2\right]
	\label{eqn:FockLinearCombination}
\end{equation}
iff $\alpha + \beta = 1$. Defining
\begin{align*}
	\Fot{n} &\equiv  \mat{F}\left[ \Pet{n} \right] &
	\Fo{n} &\equiv \mat{F}\left[ \Pe{n} \right]
\end{align*}
this allows to rewrite \eqref{eqn:ODAupFock} as
\begin{equation}
	\label{eqn:ODAdamping}
	\Fot{n}
	= \mat{F}\left[ \ila{n} \Pet{n-1} + \la{n} \Pe{n} \right]
	= \ila{n} \Fot{n-1} + \la{n} \Fo{n},
\end{equation}
where the ``$\text{min}$'' subscripts were dropped.
Comparing with equation \eqref{eqn:FockDamping}
one can identify with $\la{n}$ the damping factor $\alpha$.
Since $\la{n}$ is optimal in the sense of minimising the energy along the line segment
spanned by $\Pe{n}$ and $\Pet{n-1}$,
the optimal damping algorithm can be described by repetitively finding the optimal damping
parameter from \SCF step to \SCF step.
Notice that its construction guarantees that the \SCF energy will always decrease.
It is hence guaranteed to converge to a local minimum of the \HF problem
\eqref{eqn:HFOptDens}~\cite{Cances2000,Cances2000a}.
The \ODA is only a particularly simple example from a whole family
of density-based \SCF algorithms called relaxed constraints algorithms,
which are discussed in detail in \cite{Cances2000a}.

\noindent
Using \eqref{eqn:ODAdamping} one can show by induction that
\begin{align}
	\label{eqn:ODAFock}
	\Fot{n} &= \sum_{j = 0}^n \Fo{j} \la{j} \prod_{i=j+1}^n \ila{i}, \\
	\label{eqn:ODADens}
	\Pet{n} &= \sum_{j = 0}^n \Pe{j} \la{j} \prod_{i=j+1}^n \ila{i},
\end{align}
where we set $\la{0} \equiv 1$. Since
\begin{align*}
	\Fo{j} &= \matFnfull \\
	\Pe{j} &= \mat{C}^{(j)} \tp{\left(\mat{C}^{(j)}\right)} \\
\end{align*}
these results in theory allow to express the complete \ODA
in terms of the coefficients
such that expressions like \eqref{eqn:ExchangeApply}
or \eqref{eqn:ApplicationKcs} could be used for a \FE-based
or a \CS-based discretisation respectively.

In practice this is usually not a fruitful approach for two reasons.
Firstly it requires to store a growing list of coefficients,
namely one for each \SCF step.
Especially for a \FE approach this becomes increasingly costly in terms of memory.
Secondly for a \contract-based ansatz
we especially want to avoid storing the Fock matrices $\Fo{j}$
in favour of \contract expression like \eqref{eqn:ExchangeApply}
and \eqref{eqn:ApplicationKcs}.
In other words each application of $\Fot{n}$ to a vector $\vec{x}$
would need to be performed by first computing $\Fo{j} \vec{x}$ for each $j$
and then adding the results.
This procedure is roughly $n$ times as expensive as a single apply.
Even though the \contract expressions formally scale better,
the increasing number of times they need to be invoked
should make this ansatz rather expensive.

Overall the \ODA is very suitable for {\cGTO} and \CS-based discretisations,
since for these density-based \SCF schemes are fine,
but this algorithm is not suitable for solving the \HF problem
with a \FE-based discretisation without further modifications.

\defineabbr{tODA}{tODA\xspace}{Truncated optimal damping algorithm}
\subsection{Truncated optimal damping algorithm}
\label{sec:tODA}

Let us again consider \eqref{eqn:ODAFock}.
Due to $\la{i} \in [0,1]$ the Fock matrix prefactor
\begin{equation}
	\label{eqn:ODAFockCoeff}
	\lambda^{(j)} \prod_{i=j+1}^{n} \left(1- \lambda^{(i)}\right) \in [0,1]
\end{equation}
is a product of factors,
which are all between $0$ and $1$.
Therefore this prefactor may become
rather small for small values of $j$ as $n$ increases.
In other words in the later \SCF steps the $\Fo{j}$ terms
which were produced at the beginning of the \SCF procedure
may be accompanied with a small prefactor and hence can at some point
be neglected in \eqref{eqn:ODAFock}.
This is the justification for the truncated optimal damping algorithm~(\tODA),
which approximates the \ODA by artificially restricting the number
of terms in \eqref{eqn:ODAFock} to the $m$ most recently
obtained Fock matrices.
If we define
\[
	j_0(n) \equiv n-m+1
\]
this allows to write the approximated sums as
\begin{align}
	\label{eqn:ODAFockApprox}
	\Fot{n} &= \frac{1}{\la{j_0(n)}} \sum_{j = j_0(n)}^n \Fo{j} \la{j} \prod_{i=j+1}^n \ila{i}, \\
\intertext{and analogously for the density matrices}
	\label{eqn:ODADensApprox}
	\Pet{n} &= \frac{1}{\la{j_0(n)}} \sum_{j = j_0(n)}^n \Pe{j} \la{j} \prod_{i=j+1}^n \ila{i}.
\end{align}
The factor $1/\la{j_0(n)}$ is required to make sure that
the Fock matrix prefactors sum to $1$,
\ie to make sure that the condition for
the linear combination of Fock matrices
\eqref{eqn:FockLinearCombination} is fulfilled.

The simplest case of this class of approximations is $m=1$.
This implies $j_0(n) = n$ such that \eqref{eqn:ODAFockApprox}
and \eqref{eqn:ODADensApprox} simplify to read
\begin{align*}
	\Fot{n} &= \Fo{n} & \Pet{n} &= \Pe{n}
\end{align*}
In other words this 2-step \tODA is equivalent to an \textit{adhoc} modification
of the exact ODA where we replace
$\tilde{\mat{D}}^{(n)}$ by $\mat{D}^{(n)}$,
the density of the previous \SCF step.
Taking this into account the expressions \eqref{eqn:ODAs} and \eqref{eqn:ODAc}
may be written as
\begin{align}
	\label{eqn:tODAs}
	\nonumber
	s &= \tr \left( \Fot{n} \Pe{n+1} \right) - E_1[\Pet{n}] - 2 E_2 [\Pet{n}] \\
	  &= \tr \left( \tp{\left(\mat{C}^{(n+1)}\right)} \Fo{n} \mat{C}^{(n+1)} \right)
	  - E_1\!\left[\mat{C}^{(n)} \tp{\left(\mat{C}^{(n)}\right)}\right]
	  - 2 E_2\! \left[\mat{C}^{(n)} \tp{\left(\mat{C}^{(n)}\right)}\right] \\
\intertext{and}
	\label{eqn:tODAc}
	\nonumber
c &= E_2[\Pe{n+1}] - \tr \left( \Fot{n} \Pe{n+1} \right) + E_1[\Pe{n+1}]
		+ E_2 [\Pet{n}] \\
	\nonumber
	&= E_2\!\left[\mat{C}^{(n+1)} \tp{\left(\mat{C}^{(n+1)}\right)}\right]
		- \tr \left( \tp{\left(\mat{C}^{(n+1)}\right)} \Fo{n} \mat{C}^{(n+1)} \right) \\
		&\hspace{50pt}
		+ E_1\!\left[\mat{C}^{(n+1)} \tp{\left(\mat{C}^{(n+1)}\right)}\right]
		+ E_2\!\left[\mat{C}^{(n)} \tp{\left(\mat{C}^{(n)}\right)}\right]
\end{align}
In contrast to the exact \ODA
this yields a coefficient-based \SCF algorithm.
Starting from an initial set of coefficients $\mat{C}^{(0)}$
with corresponding initial Fock matrix $\Fot{0} =
\mat{F}\!\!\left[\mat{C}^{(0)}\!\tp{\left(\mat{C}^{(0)}\right)}\right]$
we proceed for $n=1,2,3,\ldots$ as follows.
\begin{itemize}
	\item Diagonalise $\Fot{n-1}$ in order to obtain coefficients
		$\mat{C}_F^{(n)}$.
	\item According to the Aufbau principle select $\mat{C}^{(n)}$
		and build $\Fo{n} = \matFnfull$.
	\item Evaluate the Pulay error $\mat{e}^{(n)}$ \eqref{eqn:PulayError}
		and end the process if $\norm{\mat{e}^{(n)}}_\text{frob} \leq \epsilonconv$.
	\item Compute $s$, $c$ and $\lambda^{(n)}$
		according to \eqref{eqn:tODAs}, \eqref{eqn:tODAc}
		and \eqref{eqn:ODALambdaOpt}.
	\item Set
		\[ \Fot{n} = \ila{n} \Fo{n-1} + \la{n} \Fo{n} \]
		and repeat.
\end{itemize}
In this process one only needs the history of two Fock matrices $\Fo{n-1}$ and
$\Fo{n}$, such that $\Fot{n}$ can be applied when needed.
This in turn implies that only the coefficient matrices $\mat{C}^{(n-1)}$
and $\mat{C}^{(n)}$ are required, such that $\Fo{n}$ and $\Fo{n-1}$
can be applied whenever needed.

Compared to the Roothaan algorithm~(see section \ref{sec:RoothaanRepeatedDiag})
the \tODA only roughly doubles
the cost of each diagonalisation, since two Fock matrices need to be applied.
Additionally one needs to evaluate the trace
\[
	\tr \left( \tp{\left(\mat{C}^{(n+1)}\right)} \Fo{n} \mat{C}^{(n+1)} \right)
\]
and compute the energies
$E_1[\mat{D}^{(n)}]$ and $E_2[\mat{D}^{(n)}]$ in order to obtain $c$ and $s$
for each iteration.
The former step costs about as much as a single matrix-vector product
and the latter is usually done during the \SCF anyways
to display the progress to the user,
thus represents no extra cost.

Even though about twice as expensive as the Roothaan algorithm
if a \contract-based \SCF is performed,
the advantage of the \tODA is,
that it automatically finds the damping coefficient $\la{n}$,
which reduces the energy at each iteration as much as possible.
This amounts to break the oscillatory behaviour
of the standard Roothaan repeated diagonalisation scheme
in a slightly improved manner than the default damping
or level-shifting modifications.

One should mention, however, that the \tODA
does not inherit all of the nice mathematical properties from the \ODA.
For example it is no longer guaranteed that the \tODA
converges to a stationary point of the \HF problem \eqref{eqn:HFOptCoeff}%
~\cite{CancesODAprivate}.
Especially close to convergence it may for example happen,
that $\lambda^{(n)} \not\in [0,1]$ since both $c$ and $s$
become rather small, thus $2c \leq s$ ill-defined.
One can get around this by explicitly setting $\lambda^{(n)} = 1$ in the cases,
where $\abs{c}$ and $\abs{s}$ become small.
The \tODA is thus best used in the initial \SCF steps in order to
effectively prevent the Roothaan oscillations from happening.

\subsection{Direct inversion of the iterative subspace}
\label{sec:DIIS}



Standard commutator DIIS by Pulay~\cite{Pulay1982,Hamilton1986}

Other ways~\cite{Shepard2007}
LSIIS (least-squares commutator DIIS)~\cite{Li2016}



$\tilde{\mat{F}}^{(n)}$ gradient of $\mathcal{E}_D(\mat{D})$
wrt. $\mat{D}$~\cite{Lions1988,Cances2000}.
Can define~\cite{Lions1988,Cances2000}
\[
	\mat{C}^{(n+1)} = \arginf_{\mat{C} \in \mathcal{C}}
\left\{ \tr \tilde{\mat{F}}^{(n)} \mat{C} \tp{\mat{C}} \right\} \]



\subsection{Geometric direct minimisation}
Only brief

\subsection{Second-order \SCF algorithms}
\label{sec:SOSCF}
Only brief
mention that approximate methods exist, too

% https://www.sciencedirect.com/science/article/pii/0301010481851567

Quadratically-convergent scf \cite{Ochsenfeld1997}
Augmented Roothan-Hall \cite{Hoest2008}

\todoil{Check out: Linear scaling SCF as minimisation \cite{Salek2007} --- this is second order
and based on the density, preconditioned CG-like minimisation}


\subsection{Combinations of algorithms}
Ideally want to switch every now and then


DIIS and ODA


\section{Basis-type independence of SCF algorithms}
% Show schematic with Fock update - coefficient update
%     in red and blue as in paper
% Refer to appropriate equations from more detailed considerations in SCF
% algorithms chapter.
% TODO This could fit into the introductory part of the SCF algorithm
% section or in the conclusion part of it as well


\todo[inline,caption={}]{
	\begin{itemize}
		\item Make sure it comes across that SCF has very similar structure for each basis type
		\item SCF algorithm may be formulated independent of basis type
		\item DFT very similar
		\item[$\Rightarrow$] Basis-function independent quantum chemical modelling
		\item Maybe move to flexible qchem program chapter
	\end{itemize}
}
