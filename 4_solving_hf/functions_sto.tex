\subsection{Slater-type orbitals}
\label{sec:STO}

Motivate with cusp condition.

\begin{rem}
	This remark provides a short recap of some
	fundamental physical properties of the electronic Schrödinger equation.

Kato's cusp condition~\cite{Kato1957}
Energy-dependent exponential decay at infinity
(easy to rationalise physically)
\end{rem}



Starting from the exponential form of the analytical solutions of the
Schrödinger equation for hydrogen,
\citet{Slater1930} started introducing exponentially decaying basis functions
in order to compute atomic properties,
the nowadays well-known Slater-type orbitals~(STOs).

Applying the STOs to molecules turned out to be much more
challenging due to the demanding structure of the electron repulsion integrals~(ERIs)
for such types of basis functions.
\todoil{I cannot download that paper so im not too sure what is in it}
In \citeyear{Boys1950} \citeauthor{Boys1950} realised that employing
atom-centered Gaussian type basis functions~(GTOs) leads to integrals
which are much easier to evaluate numerically.


