\subsection{Slater-type orbitals}
\label{sec:STO}

In section \vref{sec:HydrogenAtom}
we discussed the analytical solution of the simplest chemical systems,
namely the hydrogen-like atoms or ions with only a single nucleus and a single electron.
Their solutions where functions
\[ \Psi_{nlm}(\vec{r}) = N_{nl} \tilde{P}_{nl}\left(\frac{2Zr}{n}\right)
	Y_l^m(\theta, \phi) \exp\left(-\frac{Zr}{n} \right) \]
where $\tilde{P}_{nl}$ is polynomial of degree $n-1$ in $\frac{2Zr}{n}$,
see \eqref{eqn:HydrogenRadialSolution} for details.
Characteristic for the functional form of these solutions
is both the exponential decay as $r \to \infty$ as well as
the discontinuity at the origin, \ie the position of the nucleus.
These two fundamental observations can be generalised to the setting of the
full electronic Hamiltonian $\Op{H}_{\Nelec}$ as summarised in the following remark.

\begin{rem}
	\label{rem:PhysicalProperties}
	Let $\Psi_i(\vec{x})$ be an exact eigenstate of the electronic
	Hamiltonian $\Op{H}_{\Nelec}$ with eigenenergy $E^{\Nelec}_i$.
	It holds:
	\begin{itemize}
		\item Kato's electron-nucleus cusp condition~\cite{Kato1957}:
			\[
				\left. \frac{\partial \langle\Psi(\vec{x})\rangle}
				{\partial r_i} \right|_{\vec{r}_i = \vec{R}_A}
				= -Z_A \left. \langle\Psi(\vec{x})\rangle \right|_{\vec{r}_i = \vec{R}_A}
			\]
			where $\left. \langle\Psi(\vec{x})\rangle \right|_{\vec{r}_i = \vec{R}_A}$
			denotes the average value on a hypersphere with $\vec{r}_i = \vec{R}_A$ fixed.
			Notice, that this expression can be reformulated to yield
			the more well-known result
			\[
				\left. \frac{\partial \rho(\vec{r})}{\partial \vec{r}} \right|_{\vec{r} = \vec{R}_A}
				= -2 Z_A \rho(\vec{R}_A).
			\]
		\item At large distances $r_1 \to \infty$,
			where all other electronic and nuclear coordinates stay finite,
			the first electron is essentially decoupled from the motion
			of the other particles and only sees them as a point charge
			of value
			\[ Z^\text{net} = \sum_{A=1}^M Z_A - \Nelec\]
			of the full system.
			The remaining system is effectively a $\Nelec-1$-electron system.
			This allows to approximately write
			\[ \Psi_i(\vec{x}) = \tilde{\Psi}^\text{rest}_j(\vec{x}) \tilde{\Psi}^1_i(\vec{r}_1), \]
			where $\tilde{\Psi}^\text{rest}_j$ has only parametric dependence%
			\footnote{Notice that this ansatz is somewhat related to the Born-Oppenheimer
			approximation.}
			on $\vec{r}_1$.
			Approximately it is an eigenfunction to $\Op{H}_{\Nelec-1}$.
			With this ansatz the electronic Schrödinger equation at large $r_1$
			reduces for $\tilde{\Psi}^1_i(\vec{r}_1)$ to
			\begin{equation}
				\left( -\frac12 \Delta_{\vec{r}_1} - \frac{Z^\text{net}}{r_1} - E^\text{net}_i \right)
				\tilde{\Psi}_i^1(\vec{r}_1) \simeq 0,
				\label{eqn:ExponentialDecay}
			\end{equation}
			where
			\[ Z^\text{net} = \sum_{A=1}^M Z_A - \Nelec - 1\]
			is the net charge and $E^\text{net}_i$ is the net energy eigenvalue
			roughly equal to \linebreak
			$E^{\Nelec}_i - E^{\Nelec-1}_j < 0$.
			For a neutral $\Nelec$-system $Z_\text{net} = 1$,
			such that the solution of \eqref{eqn:ExponentialDecay} is
			\[ \tilde{\Psi}_i^1(\vec{r}_1) \simeq \exp\left( - \sqrt{-2E^\text{net}_i} \, r_1 \right), \]
			\ie an energy-dependent exponential decay.
	\end{itemize}
\end{rem}
Both these results motivate the use of exponential-type atomic orbitals~(\ETO)
involving a factor $\exp\left(-\zeta \norm{\vec{r} - \vec{R}_A} \right)$
as basis functions for molecular calculations,
since such a basis will give rise to solutions,
which satisfy both conditions if the factor $\zeta$ is chosen correctly.

The first attempt to do this predates the rigorous results by Kato~\cite{Kato1951,Kato1957}
by over two decades.
In \citeyear{Slater1930} \citet{Slater1930}
obtained approximate solutions to the electronic Schrödinger equation
for many atoms of the periodic table.
He employed basis sets $\{\varphi_\mu\}_{\mu\in\Ibas} \subset H^1(\R^3,\R)$
made out of exponentially decaying functions.
Motivated from the solution to the Schrödinger equation of the Hydrogen atom,
his ansatz was to write each basis function as
\begin{equation}
	\varphi^\text{STO}_\mu(\vec{r})
	= R^\text{STO}_\mu(r_\mu) \, Y_{l_\mu}^{m_\mu}(\theta_\mu, \phi_\mu)
	\label{eqn:ACproduct}
\end{equation}
\ie as a product of radial part $R^\text{STO}_\mu$
and real-valued%
\footnote{Keeping in mind that all $2l+1$ spherical harmonics
	$Y_l^m$ with the same value for $l$ correspond to the same eigenspace
	one can always find an alternative representation to the functional
	form given in \eqref{eqn:SphericaHarmonics},
	where all spherical harmonics are real-valued functions.
	See \cite{Avery2018} for details.
} spherical harmonic $Y_{l_\mu}^{m_\mu}$, where
\[ \left(r_\mu, \theta_\mu, \phi_\mu\right) \equiv \vec{r}_\mu = \vec{r} - \vec{R}_\mu \]
is the distance vector to the nucleus located at $\vec{R}_\mu$.
For the radial part he used a polynomial times exponential form
\begin{equation}
	R_\mu(r) = N_\mu r^{n_{\mu}-1} \exp(-\zeta_\mu r),
	\label{eqn:STOradial}
\end{equation}
where $\zeta_\mu$
is a constant depending on the nuclear charge as well as
the orbital in question and
$N_\mu$ is the normalisation factor
\[ N_\mu = \left( 2\zeta_\mu \right)^{n_\mu} \sqrt{\frac{2\zeta_\mu}{(2n_\mu)!}}. \]
He was able to construct rules of thumb for obtaining the exponents $\zeta_\mu$
for many elements by introducing
a concept now known as \newterm{shielding}~\cite{Slater1930}.
A more detailed discussion about shielding constants
can be found in section \vref{sec:ValuesKopt}.
In his honour {\ETO}s of the form \eqref{eqn:ACproduct}
with radial part \eqref{eqn:STOradial}
are called Slater-type orbitals~({\STO}s).

As mentioned above {\STO}s are physically rather sound
and as such in many cases only few of them are required to achieve good results
as errors are generally small and convergence fast%
~\cite{Shepard2007,Guell2008,Hoggan2009,Hoggan2011}.
Their big drawback, however,
is that evaluating the electron repulsion tensor
$\eriMu{\varphi_\mu \varphi_\nu}{\varphi_\kappa \varphi_\lambda}$
is challenging,
such that \STO-based methods
are not amongst the most commonly used quantum-chemistry methods nowadays.
Nevertheless their promising properties and fast convergence
has motivated many people to work on optimising \STO expansions
and on designing efficient evaluation schemes for the \ERI tensor%
~\cite{Weniger1983,Hoggan2009,Hoggan2011,Avery2013,Avery2017}.
As a result a number of packages
like STOP~\cite{Bouferguene1996},
SMILES~\cite{FernandezRico2001}
and ADF~\cite{Velde2001}
have become available,
which employ basis sets composed of {\STO}s.
