\subsection{Finite-element based discretisation}
\label{sec:FE}
\todo[inline,caption={}]{
	\begin{itemize}
		\item Show what needs to be done
		\item Show problems $\Rightarrow$ Lazy matrices
		\item Hint at the underlying linear algebra (eigensolvers, linear solvers)
		\item Multigrid methods
		\item Refinement
		\item How does it connect to the SCF methods
	\end{itemize}
}

A completely different ansatz to so-called
atom-centered basis functions,
\ie basis functions, which are associated to a particular atom,
are grid-based basis functions.
The idea here is to partition three-dimensional space
by employing a grid
and place piecewise polynomials on these.
As such the basis functions are not related to any particular atom.

truely black box
discretisation adapts to problem


% Use as an example to explain the ideas of finite elements
An important equation from classical electrodynamics which
involves the Laplace operator is Poisson's equation
\begin{equation}
	\Delta V_H(\vec{r}) = \rho(\vec{r})
	\label{eqn:Poisson}
\end{equation}
relating the potential to the charge density $\rho$ which generates it.

% TODO 
%  Note that norm wrt fock operator F is equivalent to H1 norm due to Laplace part
%  Lipschitz continuity
%  well-posedness

% TODO Talk about convergence rates?



\begin{figure}
	\centering
	\includeimage{4_solving_hf/fock_fe}
	\caption{Structure of the Finite-Element Fock matrix in beryllium
		with an error of about $0.1$}
	\label{fig:StructureFiniteElementFock}
\end{figure}


\todoil{Talk about contraction-based methods for the first time.
Allows to pick this up in the next section}
