\subsection{Finite-element based discretisation}
\label{sec:FE}
\newcommand{\Nquadc}{\ensuremath N_\text{quadc}}
\newcommand{\Ncell}{\ensuremath N_\text{cell}}

The Slater-type orbitals and Gaussian-type orbitals
we introduced in the previous sections
are examples for so-called atom-centered basis functions
or \newterm{atom-centered orbital}s~({\ACO}s).
A different ansatz in many respects
are grid-based methods,
where the underlying idea is to partition three-dimensional real space
into smaller parts using a structured grid
solve the problem by interpolation and numerical integration.
The example we want to consider in this work are \newterm{finite elements},
which are specially constructed piecewise polynomials
often employed in structural mechanics or engineering
for solving partial differential equations~\cite{Johnson1987}.
Multiple approaches for
solving the \HF problem or the Kohn-Sham equations approximately
using finite-element based discretisations
have been performed over the years as well%
~\cite{Tsuchida1995,Soler2002,Lehtovaara2009,Alizadegan2010,Avery2011PhD,Davydov2015,Lee2015,Boffi2016},
This section will only give a short overview of the finite-element method
in the light of the \HF problem.
For more details the reader is referred
to the rich literature~\cite{Johnson1987,Grossmann1992,Bangerth2003,Brenner2008}.

Compared to the atom-centered basis functions,
where a discretisation based on the complete domain $\R^3$ is possible,
any grid-based method can only achieve this
on a subset $\Omega \subset \R^3$,
which is taken to be open.
At the boundary $\partial\Omega$ one needs
to impose a boundary condition for the solution to be unique.
There are a couple of options, which will be contrasted later.
For now let us assume that $\Omega$ is large enough,
such that the \SCF orbitals are essentially zero at the boundary $\partial\Omega$
and we can impose a homogeneous Dirichlet boundary%
\footnote{This implies that the \HF eigenfunctions are forced
	to be exactly zero at the boundary $\partial\Omega$.}.
Using this approximation
as well as the inner product
\[ \braket{\psi}{\chi}_1 \equiv \int_\Omega \psi(\vec{r}) \chi(\vec{r})  \D \vec{r} \]
the spin-free, real-valued \HF equations \eqref{eqn:HFequations} can be adapted to
%
\begin{equation}
\label{eqn:HFequationsFE}
\begin{aligned}
	\Op{F}_{\Theta^0} \psi_i^0(\vec{r}) &= \varepsilon_i \psi_i^0(\vec{r}) && \vec{r} \in \Omega \\
	\psi_i^0(\vec{r}) &=0 &&\vec{r} \in \partial \Omega \\
	\text{where}\qquad \braket{\psi_i^0}{\psi_j^0}_1 &= \delta_{ij},
\end{aligned}
\end{equation}
for $\Theta^0 = (\psi_1^0, \psi_2^0, \ldots, \psi_{\Nelec}^0) \in \left( H^2(\Omega,\R) \right)^{\Nelec}$
being the minimiser to the \HF problem \eqref{eqn:HFMO}.
The corresponding bilinear form
\[ a_{\Theta^0}(\psi, \chi)
	\equiv \int_\Omega \psi(\vec{r}) \Op{F}_{\Theta^0} \chi(\vec{r}) \D \vec{r} \]
is defined in analogy to \eqref{eqn:SesquilinearFormFock}.
By partial integration it can be seen that this form
is defined on the domain $Q(\Theta^0) = H^1(\Omega, \R)$.
In the \newterm{finite-element method}
the aim is to solve \eqref{eqn:HFequationsFE} variationally
in the sense of remark \vref{rem:DiscreteFormulation}
employing a hierarchy of approximation spaces $S_n$
consisting of piecewise polynomials.
Such an attempt is only sensible
is these spaces are more and more accurate
approximations of the form domain $H^1(\Omega, \R)$
in the sense of \eqref{eqn:CondSubspaces}.

To outline the construction of such spaces,
let us consider at first a (fictitious) one-dimensional case
where $\Omega = (a,b)$ with $a,b \in \R$.
This domain can be subdivided into $\Ncell$ parts
\[ a = x_0 < x_1 < x_2 < \cdots < x_{\Ncell} = b, \]
which do not need to be of equal size.
The open intervals $c_j = (x_j, x_{j+1})$ for $j=0, 1, \ldots, \Ncell-1$
are called grid \newterm{cells}
and the set $\mathcal{M}_h = \{c_j \, | \, j=0, 1, \ldots, \Ncell-1 \}$
of all grid cells is called a \newterm{mesh} or a \newterm{triangulation}.
In this set the index $h$ stands for the maximal size of a grid cell
defined as
\[ h \equiv \max_{c \in \mathcal{M}_h} \abs{\max(c) - \min(c)} \]
in one dimension. Using the vector space
\[
	\set{P}_k^1 \equiv \left\{ u \in C^\infty(\R) \,\middle|\, u(x) = \sum_{i=0}^k c_i x^i, c_i \in \R \right\}
\]
of all real polynomials of order at most $k$ in one dimension,
we can define
\begin{equation}
	P_k(\mathcal{M}_h)
	\equiv \left\{ u \in C^0(\overline{\Omega}) \ \middle| \
	\forall c \in \mathcal{M}_h: \
	u|_{\bar{c}} \in \set{P}_k^1 \right\},
	\label{eqn:PiecewisePolynomialsOne}
\end{equation}
the set of piecewise polynomials of at most degree $k$.
The elements of $P_k(\mathcal{M}_h)$
are on the complete domain $\Omega$ at least continuous
and inside the grid cells
they reduce to being polynomials, thus smooth.
It can be shown~\cite[Lemma 4.1]{Grossmann1992} that this implies
$P_k(\mathcal{M}_h) \subset H^1(\Omega, \R)$.
As $h \to 0$ such approximations become more exact,
which make $P_k(\mathcal{M}_h)$ the desired approximation
spaces of $H^1(\Omega, \R)$ for a one-dimensional problem.

\begin{figure}
	\centering
	\missingfigure{grid}
	\caption{One D case for linear FEs}
	\label{fig:FEoneDimLin}
\end{figure}
\begin{figure}
	\centering
	\missingfigure{grid}
	\caption{One D case for quadratic FEs}
	\label{fig:FEoneDimQuad}
\end{figure}
For representing $P_k(\mathcal{M}_h)$
\newcommand{\Ibash}{\mathcal{I}_{\text{bas},h}}
one typically chooses a \newterm{Lagrange basis} $\{\varphi_\mu\}_{\mu\in \Ibash}$
consisting of basis functions $\varphi_\mu$ with $0 \leq \mu \leq k \Ncell$
defined as
\begin{align*}
	\varphi_\mu &\in P_k(\mathcal{M}_h), &
	\varphi_\mu(\tilde{x}_\nu) &= \delta_{\mu\nu}
\end{align*}
where%
\footnote{In equation \eqref{eqn:FEnodalPointsOneD} $\nu/k$ denotes integer division
without remainder.}
\begin{equation}
	\tilde{x}_\nu = x_{\nu/k} + \frac{\nu \!\! \mod k}{k} \left( x_{(\nu/k) +1} - x_{\nu/k} \right)
	\label{eqn:FEnodalPointsOneD}
\end{equation}
for $0 \leq \nu \leq k \Ncell$ are the \newterm{nodal points}.
An alternative term to refer to such
basis functions $\varphi_\mu$
is \textbf{finite element of order $k$},
the order $k$ being a reference to the maximal polynomial degree inside the cells $c$.
Examples for linear and quadratic finite elements are
illustrated in figures \ref{fig:FEoneDimLin} and \ref{fig:FEoneDimQuad} respectively.
In each case the finite element functions
are either $1$ or $0$ at the nodal points and only
have support on a few cells, which are furthermore direct neighbours.

In the following we want to generalise this construction
for a three-dimensional domain $\Omega$.
In the most general form a mesh can be defined as
\begin{defn}[Mesh]
	Let $\Omega$ be a domain in $\R^3$.
	A mesh is a finite set $\mathcal{M}_h = {c_0, c_1, \ldots, c_{\Ncell-1}}$
	of $\Ncell$ bounded domains $c_i$ with sufficiently regular boundary%
	\footnote{More precisely the boundary has to be Lipschitz.},
	such that
	\begin{align*}
		\overline{\Omega} &= \bigcup_{i=0}^{\Ncell-1} \overline{c_i}
		&c_i \cap c_j = \emptyset \quad \forall i \neq j,
	\end{align*}
	\ie such that these domains fully partition $\Omega$.
	Furthermore we set for each $c \in \mathcal{M}_h$
	the \newterm{cell diameter}
	\[ h(c) = \max_{\vec{x},\vec{y} \in bar{c}} \norm{\vec{x}-\vec{y}}_2 \]
	and call
	\[ h = \max_{c \in \mathcal{M}_h} h(c) \]
	the \newterm{mesh size}.
\end{defn}
In many cases one does, however, only consider
so-called \newterm{affine} meshes.
\begin{defn}
	A mesh is called affine if a reference cell $c_0$
	and affine transformations%
	\footnote{A transformation $\tau$ is affine iff $\tau(\vec{x}) = \mat{A}\vec{x} + \vec{b}$ with $\mat{A}$ being a transformation matrix and $b$ being a constant shift vector.}
	$\tau_{c_i}$ exist for each cell $c_i \in \mathcal{M}_h$,
	such that $\overline{c_i} = \tau_{c_i}(c_0)$.
\end{defn}
In other words a mesh is affine exactly if each
grid cell can be generated from the reference cell $c_0$
by a linear transformation followed by a shift to the correct position.
This work only considers \textbf{cuboidal meshes},
where the reference cell $c_0 = [0,1]^3$ is the unit cube.
Similar to the construction of the grid cells,
the finite elements of order $k$
on each cuboidal grid cell $c \in \mathcal{M}_h$
can be constructed by applying the affine transformation $\tau_c$
to a set of template polynomials of order $k$
defined on the reference cell $c_0$.
These template polynomials are called
\newterm{shape function}s.
%
\begin{figure}
	\centering
	\missingfigure{shape functions}
	\caption{Shape functions}
	\label{fig:ShapeFunctions}
\end{figure}
%
A few examples in one and two dimensions are illustrated
in figure \ref{fig:ShapeFunctions}.
Notice how the shape functions in two (and higher) dimensions
can be constructed as mere tensor products
of the shape functions in one dimension.
This is a special property of the so-called $Q_k$ finite elements,
which are typically used in cuboidal meshes.

Let us denote with $S_h$ the
space spanned by all the $Q_k$ finite elements $\{\varphi_\mu\}_{\mu\in\Ibash}$
on a cuboidal mesh $\mathcal{M}_h$,
constructed by the means of appropriate affine maps and shape functions.
Even though we always have $S_h \subset H^1(\Omega, \R)$,
condition \eqref{eqn:CondSubspaces}
is not always satisfied as the mesh size $h \to 0$.
In other words vanishing mesh size is not enough
to ensure convergence of the Ritz-Galerkin procedure in three dimensions.
The missing aspect is the regularity of the mesh.
Roughly speaking one needs to make sure that going to finer and finer meshes
one does not construct needle-like cells.
Much rather all grid cells $c\in \mathcal{M}_h$ should
be uniformly refined and their shape should be as close to being
a ball as possible%
\footnote{The mathematical terms are \textit{uniform}
and \textit{shape-regular} for such meshes.}
in all meshes.

%
% ---------------------------------------------------
%

If these conditions are satisfied
convergence in higher dimensions is possible, too.
Higher dimensions: FEs still only have support at a  few cells,
which are direct neighbours.
Just more neighbours than in 2 dimensions.

Take away:
Truely black box
error estimates
grid refinement strategies
discretisation adapts to problem
Locality built into basis
Need large number of basis functions


With this approach get $Q_k$ finite elements $\{\varphi_\mu\}_{\mu\in\Ibash}$
which can be used to discretise \eqref{eqn:HFequationsFE}.
Following the procedure in section \vref{sec:DiscreteHF}
we obtain a non-linear eigenproblem
\[
	\begin{aligned}
		\matFnfull \mat{C}_F^{(n+1)} &= \mat{S} \mat{C}_F^{(n+1)} \mat{E}^{(n+1)} \\
		\tp{\mat{C}} \mat{S} \mat{C} &= \mat{I}_{\Nelec}
	\end{aligned}
\]
with the eigenvalues
\[
	\mat{E}^{(n+1)}
	= \text{diag}\left(\varepsilon_1^{(n+1)},
	\varepsilon_2^{(n+1)}, \ldots,
	\varepsilon_{\Norb}^{(n+1)}\right) \in \R^{\Norb \times \Norb}.
\]
in analogy to remark \vref{rem:PropertiesDiscretised}.


Expression of terms still \eqref{eqn:Tbas} to \eqref{eqn:Sbas}
just with the replacement
\[ \int_{\R^3} \to \int_\Omega. \]
Still an SCF

Indicate Relationship between polynomial order, number of cells
and number of finite elements

Let us consider in detail the evaluation of a local operator
like the one leading to $S_{\mu\nu}$, $(V_0)_{\mu\nu}$ or $T_{\mu\nu}$.
For computation one can thus
think of each finite-element function
as being constructed from a reference cell
by affine mapping.
This allows to evaluate integrals cell-wise
See presentation molsturm
Linear in FE

Evaluation of Coulomb needs poisson equation
\begin{equation}
	\Delta V_H(\vec{r}) = \rho(\vec{r})
	\label{eqn:Poisson}
\end{equation}
relating the potential to the charge density $\rho$ which generates it.
% other boundary conditions
% large enough domain ... Coulomb falls off slowly -> a problem
% Robin BC

Evaluation of exchange problematic.
Naive approach quadratic
Evaluation leads full fock matrix
\begin{figure}
	\centering
	\includeimage{4_solving_hf/fock_fe}
	\caption{Structure of the Finite-Element Fock matrix in beryllium
		with an error of about $0.1$}
	\label{fig:StructureFiniteElementFock}
\end{figure}
\todoil{Talk about contraction-based methods for the first time.
Allows to pick this up in the next section}


Mention as summary:
fe exact apart from polynomial interpolation
Eigensolver techniques
Multigrid preconditioning
Need coefficient-based SCF

