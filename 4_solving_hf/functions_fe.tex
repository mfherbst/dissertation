\subsection{Finite-element based discretisation}
\label{sec:FE}
\todo[inline,caption={}]{
	\begin{itemize}
		\item Show what needs to be done
		\item Show problems $\Rightarrow$ Lazy matrices
		\item Hint at the underlying linear algebra (eigensolvers, linear solvers)
		\item Multigrid methods
		\item Refinement
		\item How does it connect to the SCF methods
	\end{itemize}
}

% Use as an example to explain the ideas of finite elements
An important equation from classical electrodynamics which
involves the Laplace operator is Poisson's equation
\begin{equation}
	\Delta V_H(\vec{r}) = \rho(\vec{r})
	\label{eqn:Poisson}
\end{equation}
relating the potential to the charge density $\rho$ which generates it.

% TODO 
%  Note that norm wrt fock operator F is equivalent to H1 norm due to Laplace part
%  Lipschitz continuity
%  well-posedness

% TODO Talk about convergence rates?



\begin{figure}
	\centering
	\includeimage{4_solving_hf/fock_fe}
	\caption{Structure of the Finite-Element Fock matrix in beryllium
		with an error of about $0.1$}
	\label{fig:StructureFiniteElementFock}
\end{figure}

% Reduce margin size to fit the matrix plots and switch to landscape
\begin{landscape}
\begin{figure}
	\centering
	\includeimage{4_solving_hf/fock_gaussian}
	\caption{Structure of the Fock matrix for a cGTO-based SCF
		for the beryllium atom
		using a pc-2 basis set.
		The three figures show the SCF with a Pulay error
		Frobenius norm of $0.18$, $0.0063$ and $4.1 \cdot 10^{-7}$
		from left to right.}
		% 12 out of 15 rows are diagonal-dominant in each case.
		% Matrix is symmetric
	\label{fig:StructureGaussianFock}
\end{figure}

\begin{figure}
	\centering
	\includeimage{4_solving_hf/fock_sturmian}
	\caption{Structure of the Fock matrix for a Coulomb-Strumian based SCF
		for the beryllium atom starting from using a $(5,1,1)$
		Coulomb-Strumian basis in $mln$ order
		and a Sturmian exponent of $\kexp = 1.99$.
		The three figures show left to right the Fock matrix
		at an SCF step with a Pulay error Frobenius norm of
		$0.13$, $0.0079$, $6.7 \cdot 10^{-8}$.}
		% 14 out of 17 are diagonal-dominant in each case.
		% Matrix is symmetric
	\label{fig:StructureSturmianFock}
\end{figure}
\end{landscape}

\todoil{Talk about contraction-based methods for the first time.
Allows to pick this up in the next section}
