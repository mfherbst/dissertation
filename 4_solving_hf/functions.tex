\defineabbr{CS}{CS\xspace}{Coulomb-Sturmian or Coulomb-Sturmian basis}
\defineabbr{cGTO}{cGTO\xspace}{Contracted Gaussian-type orbital}
\defineabbr{GTO}{GTO\xspace}{Gaussian-type orbital}
\defineabbr{STO}{STO\xspace}{Slater-type orbital}
\defineabbr{ETO}{ETO\xspace}{Exponential-type orbital, orbital with radial part of the form Polynomial in $r$ times $\exp(-\alpha r)$ with some parameter $\alpha$.}
\defineabbr{FE}{FE\xspace}{Finite element}
\defineabbr{ACO}{ACO\xspace}{Atom-centered orbital}

\section{Basis function types}
\label{sec:BasisTypes}
This section tries to address the question,
which classes of functions can be used
in order to build a basis set $\{\varphi_\mu\}_{\mu\in\Ibas}$
for solving the \HF problem in an \SCF procedure.
For this we will first discuss some desirable properties for a basis set,
both motivated from the aim to represent the physics of the electronic Schrödinger
equation as good as possible
as well as requirements from the numerical side.
In the light of this,
we will discuss four types of basis functions in depth,
namely the historic Slater-type orbitals~({\STO}s),
the most commonly employed contracted Gaussian-type orbitals~({\cGTO}s),
a finite-element based discretisation method
as an example of a fully numerical approach
as well as so-called Coulomb-Sturmian-type orbitals.

Even though we mostly concentrate on the \HF problem in this section,
quite a few of the observations made here
apply to \DFT or methods going beyond Hartree-Fock as well.
In this sense the outlined discussion
can be seen as an example case for the use of the mentioned
basis function types in electronic structure theory as a whole.

\subsection{Desirable properties}
\label{sec:BasisDesiredProperties}
The central aspect of the Ritz-Galerkin procedure for
approximately solving a spectral problem
is the evaluation of the $a(\slot,\slot)$
corresponding to the operator for all pairs of basis functions,
compare with remark \vref{rem:ChoiceBasisFunction} for details.
For this procedure to be mathematically meaningful at all,
this requires the basis functions $\{\varphi_\mu\}_{\mu\in\Ibas}$
to be taken from a dense subspace of the form domain of the operator.
For the real-valued \HF problem this is the Sobolev space $H^1(\R^3, \R)$,
thus a hard requirement for all types of basis functions
used for Hartree-Fock and quantum chemistry is that
they originate from $H^1(\R^3, \R)$.
Furthermore, in some or another sense we will need compute the elements
of the Fock matrix $\matFfull$ \eqref{eqn:FockMatrix},
which in turn boils down to computing the integrals of the constituent
matrix expressions \eqref{eqn:Tbas} to \eqref{eqn:Kbas},
as well as the overlap matrix \eqref{eqn:Sbas}.
The challenging step for this is typically the evaluation
of the electron repulsion tensor \eqref{eqn:ERI}
\[
	\eriMu{\varphi_\mu \varphi_\nu}{\varphi_\kappa \varphi_\lambda}
		= \int_{\R^3} \int_{\R^3}
			\frac{\cc{\varphi_\mu}(\vec{r}_1) \varphi_\nu(\vec{r}_1)
				\,\cc{\varphi_\kappa}(\vec{r}_2) \varphi_\lambda(\vec{r}_2)}
			{\norm{\vec{r}_1 - \vec{r}_2}_2}
			\D \vec{r}_1 \D \vec{r}_2.
\]
as it involves a double integral over space
incorporating a singularity at the origin
as well as the product over four basis functions.
Additionally, the discretised \HF equations \eqref{eqn:HFDiscreteEquations}
need to be solvable numerically as well.
We will see in the next sections
that the main reason why contracted Gaussian-type orbitals
have become so popular in quantum chemistry
is that both evaluating the ERI tensor
as well as solving the resulting eigenproblem
is rather easy compared to the other cases.

Apart from the mathematical and numerical feasibility
we would like to get meaningful results with as little effort as possible,
\ie a good description of a chemical system should already be achievable
with small basis sets.
Usually this goes hand in hand with a basis function
which by itself represents the physics
of the chemical system very well already,
such that as much prior knowledge and chemical intuition as possible
could be incorporated already into the basis.
Ideally this would not bias the solution procedure,
such that unexpected or unintuitive results can still be found.

Last but not least we would like to be able to know
how wrong our \HF results are compared to the exact electronic ground state,
possible even with a pointer how to increase the basis,
such that results can be systematically improved.
The aspired scenario would be a rigorous and tight
\textit{a priori} or even better \textit{a posteriori} error estimate
for the chosen basis function type in the context of \HF.

Of course this just sketches an ideal scenario.
In reality one needs a good compromise,
typically even a different compromise for different applications.
Especially the \textit{a priori} and \textit{a posteriori} error estimates
are not easy to derive rigorously for \HF
and I am not aware of any work achieving this for the basis function
types I will discuss here in detail.

\subsection{Local energy}
\label{sec:LocalEnergy}
\todo{Rephrase the introductory paragraphs \ldots less we}
Before we start discussing individual basis types,
let us briefly pause and think about ways to quantitatively
judge a particular basis function type.
A natural choice is to consider a model system,
where the analytical solution can be found, and compare it
with the Ritz-Galerkin \HF result produced by a particular basis on the same system.
In this chapter, we will compare against the hydrogen atom.
Without a doubt this does not probe all aspects of the physical interactions
happening inside the electronic structure.
Most importantly it does miss an evaluation how a basis set deals with electron correlation.
All results therefore need to be taken with care:
In more complex systems the situation might be deviating.

For comparing our numerical answers in the form of the \HF
ground-state Slater determinant $\Phi_0$
to the exact electronic Schrödinger equation solution $\Psi_0$,
we will use absolute errors and relative errors in the ground-state wave function
as well as the ground-state energy.
Additionally, we will consider a quantity called \newterm{local energy},
which is defined below.
\begin{defn}
	Let $\Phi_0$ be an approximation to the ground state
	of the operator $\Op{H}_{\Nelec}$.
	The local energy is defined by the quotient
	\begin{equation}
		E_L(\vec{x}) \equiv \frac{\Op{H}_{\Nelec} \Phi_0(\vec{x})}{\Phi_0(\vec{x})},
		\label{eqn:LocalEnergy}
	\end{equation}
	which is constant for an exact eigenstate of $\Op{H}_{\Nelec}$
	and approximately constant for good approximations.
	Since the potential energy operator terms are only multiplicative,
	this expression can be alternatively written as
	\[
		E_L(\vec{x})
		= -\frac12 \sum_{i=1}^{\Nelec} \frac{\Delta_{\vec{r}_i} \Phi_0(\vec{x})}{\Phi_0(\vec{x})}
		- \sum_{i=1}^{\Nelec} \sum_{A=1}^M \frac{Z_a}{\norm{\vec{r} - \vec{R}_A}_2}
		+ \sum_{i=1}^{\Nelec} \sum_{j=1+1}^{\Nelec} \frac{1}{r_{ij}}.
	\]
\end{defn}
The concept of local energy originates from the
quantum Monte Carlo community~\cite{Foulkes2001,Ma2005},
where its sampling by a Monte Carlo procedure plays a central role
for obtaining the correlation energy.
It is related to the relative residual
\[
	\frac{1}{\Phi_0(\vec{x})} \, \left(\Op{H}_{\Nelec} - E_0\right) \Phi_0(\vec{x}) =
	\frac{\Op{H}_{\Nelec} \Phi_0(\vec{x}) - E_0 \Phi_0(\vec{x})}{\Phi_0(\vec{x})}
	= E_L(\vec{x}) - E_0,
\]
where $E_0$ is the \emph{exact} ground-state energy of $\Op{H}_{\Nelec}$.
This implies first of all
that $E_L(\vec{x}) = E_0$ is necessary for $\Phi_0(\vec{x})$ being the exact ground state.
Furthermore the fluctuations of $E_L(\vec{x})$ around the exact constant value $E_0$
provide a measure how far $\Phi_0(\vec{x})$ is off from being an
exact eigenstate of $\Op{H}_{\Nelec}$ at a particular point $\vec{x}$.
In this sense $E_L(\vec{x})$ can thus be seen as a \emph{local} measure
for the accuracy of $\Phi_0(\vec{x})$.
Inside regions where $E_L(\vec{x})$ is close to being constant,
the basis $\{\varphi_\mu\}_{\mu\in\Ibas}$ provides a sensible description
of an eigenstate of $\Op{H}_{\Nelec}$.
$E_L(\vec{x})$ is without a doubt conceptionally related to the
relative error in the ground-state wave function $1 - \Phi_0(\vec{r}) / \Psi_0(\vec{r})$.
Compared to the latter quantity, $E_L(\vec{x})$
has the additional advantage that one is able to notice
which eigenstate $\Phi_0(\vec{r})$ approximates in each region of space.
For example, if it fluctuates around $E_0$ in some areas and around $E_1$ in others,
we can see that $\Phi_0(\vec{r})$ sometimes represents the first excited state
better than the ground state.
Additionally, $E_L(\vec{x})$ can be applied even for cases where
the exact solution is not known and thus the relative error cannot be found.

\subsection{Slater-type orbitals}
\label{sec:STO}

In section \vref{sec:HydrogenAtom}
we discussed the analytical solution of the simplest chemical systems,
namely the hydrogen-like atoms or ions with only a single nucleus and a single electron.
Their solutions were functions
\[ \Psi_{nlm}(\vec{r}) = N_{nl} \tilde{P}_{nl}\left(\frac{2Zr}{n}\right)
	Y_l^m(\theta, \phi) \exp\left(-\frac{Zr}{n} \right) \]
where $\tilde{P}_{nl}$ is polynomial of degree $n-1$ in $\frac{2Zr}{n}$,
see \eqref{eqn:HydrogenRadialSolution} for details.
Characteristic for the functional form of these solutions
is both the exponential decay as $r \to \infty$ as well as
the discontinuity at the origin, \ie the position of the nucleus.
These two fundamental observations can be generalised to the setting of the
full electronic Hamiltonian $\Op{H}_{\Nelec}$ as summarised in the following remark.

\begin{rem}
	\label{rem:PhysicalProperties}
	Let $\Psi_i(\vec{x})$ be an exact eigenstate of the electronic
	Hamiltonian $\Op{H}_{\Nelec}$ with eigenenergy $E^{\Nelec}_i$.
	It holds:
	\begin{itemize}
		\item Kato's electron-nucleus cusp condition~\cite{Kato1957}:
			\[
				\left. \frac{\partial \langle\Psi(\vec{x})\rangle}
				{\partial r_i} \right|_{\vec{r}_i = \vec{R}_A}
				= -Z_A \left. \langle\Psi(\vec{x})\rangle \right|_{\vec{r}_i = \vec{R}_A}
			\]
			where $\left. \langle\Psi(\vec{x})\rangle \right|_{\vec{r}_i = \vec{R}_A}$
			denotes the average value on a hypersphere with $\vec{r}_i = \vec{R}_A$ fixed.
			Notice, that this expression can be reformulated to yield
			the more well-known result
			\[
				\left. \frac{\partial \rho(\vec{r})}{\partial \vec{r}} \right|_{\vec{r} = \vec{R}_A}
				= -2 Z_A \rho(\vec{R}_A).
			\]
		\item Taking the limit $r_1 \to \infty$,
			while keeping all other electronic and nuclear coordinates finite,
			the first electron is essentially decoupled from the motion
			of the other particles
			and only sees them as a point charge of value
			\begin{equation}
				Z^\text{net} = \sum_{A=1}^M Z_A - (\Nelec -1).
				\label{eqn:ChargeNet}
			\end{equation}
			The other particles of the system, \ie excluding electron 1,
			effectively forms a $\Nelec-1$-electron system.
			This allows to approximately write
			\[ \Psi_i(\vec{x}) \simeq \tilde{\Psi}^\text{rest}_j(\vec{x}) \tilde{\Psi}^1_i(\vec{r}_1), \]
			where $\tilde{\Psi}^\text{rest}_j$ has only parametric dependence%
			\footnote{Notice that this ansatz is somewhat related to the Born-Oppenheimer
			approximation.}
			on $\vec{r}_1$.
			Approximately it is an eigenfunction to $\Op{H}_{\Nelec-1}$.
			With this ansatz the electronic Schrödinger equation at large $r_1$
			reduces for $\tilde{\Psi}^1_i(\vec{r}_1)$ to
			\begin{equation}
				\left( -\frac12 \Delta_{\vec{r}_1} - \frac{Z^\text{net}}{r_1} - E^\text{net}_i \right)
				\tilde{\Psi}_i^1(\vec{r}_1) \simeq 0,
				\label{eqn:ExponentialDecay}
			\end{equation}
			where $Z^\text{net}$ is the net charge as in \eqref{eqn:ChargeNet}
			and $E^\text{net}_i$ is the net energy eigenvalue
			roughly equal to
			$E^{\Nelec}_i - E^{\Nelec-1}_j < 0$.
			For a neutral $\Nelec$-system $Z_\text{net} = 1$,
			such that the solution of \eqref{eqn:ExponentialDecay} is
			\begin{equation}
				\tilde{\Psi}_i^1(\vec{r}_1) \simeq \exp\left( - \sqrt{-2E^\text{net}_i} \, r_1 \right),
				\label{eqn:EnergyDependentDecay}
			\end{equation}
			\ie an energy-dependent exponential decay.
	\end{itemize}
\end{rem}
Both these results motivate the use of exponential-type atomic orbitals
involving a factor $\exp\left(-\zeta \norm{\vec{r} - \vec{R}_A} \right)$
as basis functions for molecular calculations,
since such a basis will give rise to solutions,
which satisfy both conditions if the factor $\zeta$ is chosen correctly.

The first attempt to do this predates the rigorous results by Kato~\cite{Kato1951,Kato1957}
by over two decades.
In \citeyear{Slater1930} \citet{Slater1930}
obtained approximate solutions to the electronic Schrödinger equation
for many atoms of the periodic table.
He employed basis sets $\{\varphi_\mu\}_{\mu\in\Ibas} \subset H^1(\R^3,\R)$
made out of exponentially decaying functions.
Motivated from the solution to the Schrödinger equation of the Hydrogen atom,
his ansatz was to write each basis function as
\begin{equation}
	\varphi^\text{STO}_\mu(\vec{r})
	= R^\text{STO}_\mu(r_\mu) \, Y_{l_\mu}^{m_\mu}(\theta_\mu, \phi_\mu)
	\label{eqn:ACproduct}
\end{equation}
\ie as a product of radial part $R^\text{STO}_\mu$
and real-valued%
\footnote{Keeping in mind that all $2l+1$ spherical harmonics
	$Y_l^m$ with the same value for $l$ correspond to the same eigenspace
	one can always find an alternative representation to the functional
	form given in \eqref{eqn:SphericaHarmonics},
	where all spherical harmonics are real-valued functions.
	See \cite{Avery2018} for details.
} spherical harmonic $Y_{l_\mu}^{m_\mu}$, where
\[ \left(r_\mu, \theta_\mu, \phi_\mu\right) \equiv \vec{r}_\mu = \vec{r} - \vec{R}_\mu \]
is the distance vector to the nucleus located at $\vec{R}_\mu$.
For the radial part he used a polynomial times exponential form
\begin{equation}
	R_\mu(r) = N_\mu r^{n_{\mu}-1} \exp(-\zeta_\mu r),
	\label{eqn:STOradial}
\end{equation}
where $\zeta_\mu$
is a constant depending on the nuclear charge as well as
the orbital in question and
$N_\mu$ is the normalisation factor
\[ N_\mu = \left( 2\zeta_\mu \right)^{n_\mu} \sqrt{\frac{2\zeta_\mu}{(2n_\mu)!}}. \]
He was able to construct rules of thumb for obtaining the exponents $\zeta_\mu$
for many elements by introducing
a concept now known as \newterm{shielding}~\cite{Slater1930}.
A more detailed discussion about shielding constants
can be found in section \vref{sec:ValuesKopt}.
In his honour exponential-type atomic orbitals of the form \eqref{eqn:ACproduct}
with radial part \eqref{eqn:STOradial}
are called Slater-type orbitals~({\STO}s).

As mentioned above {\STO}s are physically rather sound
and as such in many cases only few of them are required to achieve good results
as errors are generally small and convergence fast%
~\cite{Shepard2007,Guell2008,Hoggan2009,Hoggan2011}.
Their big drawback, however,
is that evaluating the electron-repulsion tensor
$\eriMu{\varphi_\mu \varphi_\nu}{\varphi_\kappa \varphi_\lambda}$
is challenging,
such that \STO-based methods
are not amongst the most commonly used quantum-chemistry methods nowadays.
Nevertheless, their promising properties and fast convergence
has motivated many people to work on optimising \STO expansions
and on designing efficient evaluation schemes for the \ERI tensor%
~\cite{Weniger1983,Hoggan2009,Hoggan2011,Avery2013,Avery2017}.
As a result, a number of packages
like STOP~\cite{Bouferguene1996},
SMILES~\cite{FernandezRico2001}
and ADF~\cite{ADF}
have become available,
which employ basis sets composed of {\STO}s.

\subsection{Contracted Gaussian-type orbitals}
\label{sec:cGTO}

In \citeyear{Boys1950} \citeauthor{Boys1950} suggested to replace
the exponential factor $\exp(- \zeta r)$ in the radial part \eqref{eqn:STOradial}
by a Gaussian factor $\exp(-\alpha r^2)$,
resulting in the so-called Gaussian-type orbitals~({\GTO}s).
Such \GTO basis functions still follow the ansatz radial part times real-valued spherical harmonic
\eqref{eqn:ACproduct}
\begin{equation}
	\varphi^\text{GTO}_\mu(\vec{r}) = R^\text{GTO}_\mu\!\left(\norm{\vec{r} - \vec{R}_\mu}\right)
	\, Y_{l_\mu}^{m_\mu}(\theta, \phi),
	\label{eqn:ACproduct}
\end{equation}
but their radial part is now given as
\begin{equation}
	R^\text{GTO}_\mu(r) = N_\mu r^{l_\mu} \exp(-\alpha_\mu r^2)
	\label{eqn:GTOradial}
\end{equation}
with Gaussian exponent $\alpha_\mu$ and normalisation constant
\[
	N_\mu =  \sqrt{\frac{2^{l+2}}{(2l+1)!!}} \, \sqrt[4]{\frac{(2\alpha)^{2l+3}}{\pi}}
\]
This replacement allows to perform the
evaluation of the integrals involved in building the Fock matrix $\mat{F}$
much more efficiently.
Because of the \newterm{Gaussian product theorem}%
~\cite{Boys1950,Szabo1996,Besalu2011},
the product of two Gaussians may be expressed \emph{exactly} as
\[
	R^\text{GTO}_\mu\!\left( \norm{\vec{r} - \vec{R}_\mu}\right)
	\,
	R^\text{GTO}_\nu\!\left( \norm{\vec{r} - \vec{R}_\mu}\right)
	= R^\text{GTO}_\kappa\!\left( \norm{\vec{r} - \vec{R}_\kappa}\right)
\]
where $l_\kappa$, $\alpha_\kappa$ and $\vec{R}_\kappa$ are chosen appropriately.
With this result the evaluation of certain \ERI integrals \eqref{eqn:ERI}
can be done analytically~\cite{Boys1950}.
An example would be those involving four basis functions with $l_\mu = 0$.
All other \ERI integrals, potentially involving higher angular momentum $l_\mu$,
can be computed from the initial ones employing
a set of recursion formulas~\cite{Gill1994}.
Similar strategies can be found for the one-electron integrals
in order to build $\mat{T}$ and $\mat{V}_0$.
Overall the construction of $\mat{F}$ therefore becomes much more
feasible for larger basis sets of Gaussians compared to large sets of {\STO}s.

Unfortunately certain physical aspects like the exponential decay
or the cusp are no longer directly built into the basis set
if such \GTO basis functions are used.
Since $\varphi^\text{GTO}_\mu(\vec{r}) \in C^\infty(\R^3, \R)$,
which is a dense subset of $H^1(\R^3, \R)$
this is not \textit{per se} a problem:
The denseness ensures that we can still represent every function
from $H^1(\R^3, \R)$ up to arbitrary accuracy if we use enough {\GTO}s.
In other words Ritz-Galerkin ansatz still allows us to
solve problems like \HF or \FCI up to arbitrary accuracy.
Since the {\GTO}s, however,
span only a smaller subset of $H^1(\R^3, \R)$ than the {\STO}s,
we will need more \GTO basis functions to achieve the same accuracy.

As a remedy \citet{Hehre1969} introduced so-called
\newterm{contracted Gaussian-type orbitals}~({\cGTO}s),
where the radial part of a basis function $\varphi_\mu$
is expressed as a fixed linear combination of $N_\text{contr}$
\newterm{primitive Gaussians}%
\footnote{Here we follow the usual convention to include the normalisation constant
	inside the contraction coefficients.}
\[ R^\text{cGTO}_\mu(r) = r^{l_\mu} \sum_i^{N_\text{contr}} c_{\mu,i} \exp(-\alpha_{\mu,i} r^2). \]
The idea is to get the best out of both worlds:
The easily solvable integrals in terms of primitive {\GTO}s
and an accurate description of the wave function by using
special sets of \newterm{contraction coefficient}s
$c_{\mu,i}$ and exponents $\alpha_{\mu,i}$
predetermined to give a good set {\cGTO}s $\{\varphi_\mu\}_{\mu\in\Ibas}$.
By the means of this trick
one is able to effectively split the parameter space of the variational problem
\eqref{eqn:HFOptCoeff} into two parts.
One --- the contraction coefficients ---
is fitted once and for all in order to fit a large range of problems
and another --- the coefficient matrix \eqref{eqn:HFCoeffMatrix} ---
is the search space over which one minimises during the actual calculation.

Out of the pragmatic desire to perform molecular calculations
on systems larger than what was feasible with \STO basis sets at that time,
\citet{Hehre1969} initially focused on contracting primitive Gaussians
in a way that they most closely resembled a particular \STO function.
This resulted in the famous STO-$n$G family of basis sets.
Later it was realised that more accurate basis sets could be constructed
by trying to minimise the energy,
which is resulted from an actual \HF or a MP2 calculation.
Other strategies included a rigorous construction of the basis set
in order to obtain convergence in the amount of recovered correlation energy,
or to be consistent in certain computed properties.
These deviating approaches have lead to a number of
different basis set families over the years,
most of which share common concepts, however.
Our discussion here should remain rather brief.
Interested readers are referred to the excellent reviews by \citet{Hill2013}
and \citet{Jensen2013}.

All basis sets, which are considered state-of-the-art nowadays,
are so-called \newterm{split-valence basis set}s,
which is meant to indicate that multiple contracted Gaussians are available
for describing the valence shell of an atom.
How many are used is typically referred to by the $\zeta$-level,
\eg a double-$\zeta$ basis set contains two \emph{contracted} Gaussians
for each valence orbital,
a triple-$\zeta$ basis set three and so on.
For this characters like \texttt{D}, \texttt{T}, \texttt{Q}, \texttt{5}, \ldots
--- for double, triple, quadruple, quintuple level ---
may be found in the name of the basis set.
Notice, that each contracted Gaussian inside such basis sets
is typically in turn made up from multiple primitives.
For a particular basis set family the error generally decreases
going to higher zeta levels.
For some families like Dunning's correlation-consistent
basis sets~\cite{Dunning1989}
empirical formulas for estimating the error at a particular zeta
level exist~\cite{Jensen2005}, see appendix \vref{apx:CbsLimit} for details.
To the best of my knowledge none of these results have been
backed up by rigorous analysis yet.
One should, however, mention a recent work by \citet{Bachmayr2014} in this context,
where error estimates
in the relevant $H^1(\R^3,\R)$-norm are derived for Gaussian basis sets,
which only consists of primitive \GTO functions.
Such a result should be be generalisable towards {\cGTO} basis sets as well.

\begin{figure}
	\centering
	\includeimage{4_solving_hf/relative_error_cgto}
	\caption[Relative error in the hydrogen \HF ground state
		for a range of \cGTO basis sets]
	{Relative error in the hydrogen \HF ground state
		employing various \cGTO basis sets~%
		\cite{Hehre1969,Dunning1989,Jensen2001,Wilson1996}.
		The error is plotted against
		the relative distance of electron and proton.
	}
	\label{fig:RelativeErrorCgto}
\end{figure}
A large range of \cGTO basis sets are available nowadays,
which offer a spectrum of compromises between accuracy and computational cost.
Nevertheless some systematic issues related to the non-physical shape
of the primitive {\GTO}s can not be fully accounted for,
even in the largest basis sets.
To illustrate this, consider figure \ref{fig:RelativeErrorCgto}.
In this plot the
relative error of the hydrogen \HF ground state $\Phi_0$
with respect to the exact
electronic ground state $\Psi_{1s}$ \eqref{eqn:FunctionGShydrogen}%
\footnote{For hydrogen \HF is equivalent to solving the full Schrödinger equation.}
\[ \frac{\Psi_0(\vec{r}) - \Psi_{1s}(\vec{r})}{\Psi_{1s}(\vec{r})} \]
is shown at various electron-proton distances.
We show this quantity for a range of standard \cGTO basis sets,
namely the minimal basis set STO-3G~\cite{Hehre1969},
the double-$\zeta$ basis sets cc-pVDZ~\cite{Dunning1989}
and pc-1~\cite{Jensen2001},
the quadruple-$\zeta$ basis set pc-3~\cite{Jensen2001}
as well as the sextuple-$\zeta$ basis set cc-pV6Z~\cite{Wilson1996}.
In each case the error is smallest at intermediate
electron-proton distances, but increases both at the origin
as well as larger distances.
The former feature can be explained due to the faster fall-off
of the Gaussians, $\exp(-\alpha r^2)$,
compared to the exact solution,
which goes as $\exp(-\zeta r)$.
The latter feature originates from the failure of Gaussians
to represent the electron-nuclear cusp.
Larger basis sets like pc-3 or cc-pV6Z amount to recover the
correct decay behaviour as well as the cusp somewhat,
such that the error stays below $0.02 \equiv 2\%$
in the complete inner part of the plot up to distances
of about $7.5$ Bohr.
Eventually all relative errors tend towards $-\infty$ as $r \to \infty$,
however.
This includes the case of pc-3,
where the relative error has a local maximum around $r = 10$
and then follows a downhill slope similar to the pc-1 error around $r = 5$.
Overall the plots agree with the rule of thumb
that results become more accurate at higher zeta levels:
Both the relative errors get smaller as well as the region
where the wave function is well-represented becomes larger
as we proceed from STO-3G to double-$\zeta$ and higher $\zeta$ levels.

\begin{figure}[p]
	\centering
	\includeimage{4_solving_hf/local_energy_cgto}
	\caption[Local energy of the hydrogen \HF ground state]{
		Local energy $E_L(\vec{r})$ of the hydrogen atom \HF ground state
		obtained using the indicated contracted Gaussian basis sets%
		~\cite{Hehre1969,Dunning1989,Jensen2001,Wilson1996}.
		$E_L(\vec{r})$ is plotted against the relative distance
		of electron and nucleus.
	}
	\label{fig:LocalEnergyCgto}
\end{figure}

\begin{figure}[p]
	\centering
	\includeimage{4_solving_hf/local_energy_cgto_zoom}
	\caption[Local energy of the hydrogen \HF ground state (magnified)]{
		Magnified version of figure \vref{fig:LocalEnergyCgto}
		around the origin.
	}
	\label{fig:LocalEnergyCgtoZoom}
\end{figure}

In figures \ref{fig:LocalEnergyCgto} and \ref{fig:LocalEnergyCgtoZoom}
the local energies \eqref{eqn:LocalEnergy}
of the aforementioned basis sets are depicted
--- again as a function of relative distance.
These plots not only diverge to $-\infty$ as $r \to \infty$,
but at the origin as well, see particularly figure \ref{fig:LocalEnergyCgtoZoom}.
At intermediate electron-proton distances
the local energies of all basis sets
fluctuate around the exact ground state energy of $0.5$,
where the amplitude of the fluctuations are lowest for cc-pV6Z and pc-3.
Recall, that the local energy is related to the relative residual error
and that ideally it should be a constant.
At intermediate distances, where the fluctuations are small,
the \HF ground state thus agrees well with the exact ground state.
Unsurprisingly the parts of figure \ref{fig:LocalEnergyCgto},
where $E_L(\vec{r})$ is almost constant,
agree roughly the parts of figure \ref{fig:RelativeErrorCgto},
where the relative error is small.
Similarly the wrongful decay behaviour of the \cGTO solutions
is observed in both the plot of the relative error as well as
the local energy plot.
The most notable discrepancy of both error metrics
is close to the nucleus, see \ref{fig:LocalEnergyCgtoZoom}.
Whilst the relative error gets smaller and smaller for the larger
pc-3 and cc-pV6Z basis sets close to the core as well,
these show rather vivid fluctuations in $E_L(\vec{r})$ as $r \to 0$.
Eventually they diverge to $-\infty$ exactly like the result
employing any other basis set.
In other words, whilst these basis sets amount to produce
a very good description of the ground state from distances around
$0.5$ Bohrs up to $7.5$ Bohrs,
they fail to do so close to the core in a rather misbehaving manner.
Since the relative error is small,
the issue is not that the function value
of the exact ground state is missed.
Much rather the culprit is the gradient of the approximated
\HF ground state.

This can be explained following \cite{Ma2005}.
The potential term in the local energy \eqref{eqn:LocalEnergy} diverges
as $-Z_A/r$ close to the nucleus $A$,
such that the kinetic energy term inside \eqref{eqn:LocalEnergy}
needs to provide an equal and opposite
divergence in order for the resulting local energy to be constant.
Since the gradient of every {\cGTO} basis functions is zero at the origin,
so is the gradient of the final \HF ground state,
thus the local energy goes to $-\infty$.
Furthermore the gradient of each individual primitive Gaussian
goes to zero at a different rate
depending on its exponent $\alpha_{\mu,i}$.
Overall this leads to an overcompensation
of the diverging potential in the kinetic term at some points
and an undercompensation at others,
giving rise to the oscillatory behaviour.
This oscillatory feature close to the nucleus is well-known
in the quantum Monte-Carlo community~\cite{Foulkes2001,Ma2005},
since it can lead to problems when sampling the local energy,
especially in diffusion Monte-Carlo.

In \HF, \DFT and Post-\HF methods
the failure of the {\cGTO}s to represent the nuclear cusp
or the long-range behaviour is typically only an issue,
if either parts of the wave functions are especially important
for a particular property.
Examples would be the determination of Rydberg-like excited states,
the computation electron affinities
or the computation of nuclear-magnetic resonance properties.
In such cases specific basis sets
are required~\cite{Hill2013,Jensen2013},
which include further \cGTO basis functions
to either sample the core region or the long-range tail
more accurately.
Is such basis sets are not employed
it may happen that the desired features are completely missed
or described very inaccurately.
In this sense {\cGTO}s are not fully black-box and
require some sense of chemical intuition to pick a sensible basis set
and get the right number.
Conversely if one already has a clear idea about the electronic structure
of a chemical system,
it is usually possible to find a basis set,
which promises to represent the required features both efficiently and accurately.

\begin{landscape}
\begin{figure}
	\centering
	\includeimage{4_solving_hf/fock_gaussian}
	\caption[Structure of the Fock matrix for a \cGTO-based \SCF]
		{Structure of the Fock matrix for a \cGTO-based \SCF
		of the beryllium atom
		in a pc-2~\cite{Jensen2007} basis set.
		The three figures show the Fock matrix
		at different convergence stages during the \SCF.
		From left to right the Pulay error
		Frobenius norm is $0.18$, $0.0063$ and $4.1 \cdot 10^{-7}$.
	}
	\label{fig:StructureGaussianFock}
\end{figure}
\end{landscape}
We already mentioned that the Gaussian product theorem
allows to evaluate the integrals
required for building the Fock matrix $\mat{F}$
rather efficiently.
Furthermore the resulting Fock matrix is comparatively small:
Even for systems with hundreds of atoms one typically only needs
in the order of thousands of basis functions.
In other words both building the Fock matrix
as well as diagonalising it can be performed using dense methods%
\todoil{put citations to those methods in footnote 9}\noindent
\footnote{%
	A standard procedure would be to reduce the matrix to
	tridiagonal form using Householder reflections
	and then use Cuppen's divide and conquer
	or an algorithm based on relatively robust representations.
}.
On top of that the numerical structure of a \cGTO-based Fock matrix
is rather advantageous.
Figure \vref{fig:StructureGaussianFock}, for example,
shows some Fock matrices from an \SCF calculation
of a beryllium atom
in the pc-2~\cite{Jensen2007} basis set.
The matrices are taken as snap shots during the \SCF procedure.
From left to right the Pulay error Frobenius norm decreases
from $0.18$ to $0.0063$ and finally $4.1\E{-7}$.
As the error gets smaller the matrix becomes more and more diagonal
as the off-diagonal elements in the occupied-virtual block of the
Fock matrix all have to vanish%
\footnote{This is another way equivalent to \eqref{eqn:PulayError}
to express \SCF convergence.}.
Already the leftmost matrix is almost diagonal-dominant with 12
out of 15 rows $\mu$
satisfying the condition for \newterm{diagonal-dominance}
\[
	\sum_{\nu=1}^{\Nbas} F_{\mu\nu} < 2 F_{\mu\mu}.
\]
For larger example systems,
the structure gets worse due to interactions between
the atoms, but still stays almost diagonal-dominant.
This allows to additionally employ
iterative eigensolver methods like Davidson's method~(see section \vref{sec:Davidson})
to efficiently obtain eigenpairs of the Fock matrix if not all are required.
In summary both obtaining and diagonalising Fock matrices
from a \cGTO discretisation is numerically no problem.

These numerically rather nice properties
have made \cGTO-based methods
historically the most feasible route
to describing a chemical systems within a decent accuracy.
Even though they are physically not the most sensible basis functions,
the rather pragmatic ansatz of Boys, Hehre, Pulay and others
has overall made them the most popular discretisation method in quantum-chemistry,
which is implemented in countless quantum-chemistry program packages.
In light of this it is remarkable,
that only in \citeyear{Bachmayr2014}
error bounds were developed by \citet{Bachmayr2014} for some special
kinds of Gaussian basis sets,
which furthermore do not see any application in practice
to the best of my knowledge.
I personally consider at a problem
that selecting the \cGTO basis set for performing a calculation is often done
based on habit rather than proper scientific evaluation
of the discretisation error in light of the chemical problem at hand.
I do not dare to estimate how many published works
draw conclusions based on calculations,
where a more appropriately chosen basis
could have given a totally different picture.

\subsection{Finite-element based discretisation}
\label{sec:FE}
\todo[inline,caption={}]{
	\begin{itemize}
		\item Show what needs to be done
		\item Show problems $\Rightarrow$ Lazy matrices
		\item Hint at the underlying linear algebra (eigensolvers, linear solvers)
		\item Multigrid methods
		\item Refinement
		\item How does it connect to the SCF methods
	\end{itemize}
}

% Use as an example to explain the ideas of finite elements
An important equation from classical electrodynamics which
involves the Laplace operator is Poisson's equation
\begin{equation}
	\Delta V_H(\vec{r}) = \rho(\vec{r})
	\label{eqn:Poisson}
\end{equation}
relating the potential to the charge density $\rho$ which generates it.

% TODO 
%  Note that norm wrt fock operator F is equivalent to H1 norm due to Laplace part
%  Lipschitz continuity
%  well-posedness

% TODO Talk about convergence rates?



\begin{figure}
	\centering
	\includeimage{4_solving_hf/fock_fe}
	\caption{Structure of the Finite-Element Fock matrix in beryllium
		with an error of about $0.1$}
	\label{fig:StructureFiniteElementFock}
\end{figure}


\subsection{Coulomb-Sturmian-type orbitals}
\label{sec:BasisCS}
Coulomb Sturmians~(CS) are another type of atom-centred basis functions,
which so far have seen little attention in electronic structure theory.
Similar to Slater-type orbitals
they were introduced~\cite{Shull1959}
as a generalisation to the solutions of the Schrödinger equation for hydrogen-like atoms.
\CS functions cannot be used for molecules, only for atoms,
but closely related functions exist,
which are more generally applicable.
The main motivation for \citet{Shull1959} to look into alternative
exponential functions was that they wanted to construct one-electron basis functions,
which could be used to compute the spectra of many-electron atoms.
From previous approaches it was known that a
proper representation of the cusps of the wave function
required the inclusion of the continuum
if hydrogen-like orbital functions were used.
This was, however, exceedingly difficult in practice~\cite{Avery2006}.
To avoid this dilemma \citet{Shull1959} artificially modified
the Schrödinger equation \eqref{eqn:HydrogenComplete}
for hydrogen-like atoms,
such that it was on the one hand still analytically solvable,
but on the other hand the spectrum of Helium could be modelled
up to a rather good level of accuracy,
even without explicit inclusion of the continuum.
Effectively their trick was to multiply the Coulomb term
in \eqref{eqn:HydrogenComplete} by a prefactor
\begin{equation}
	\beta_n = \frac{\kexp n}{Z}
	\label{eqn:CSbeta}
\end{equation}
with $\kexp \in \R$ arbitrary to yield
\begin{equation}
	\left( - \frac12 \Delta - \beta_n \frac{Z}{r} - E \right) \varphi^\text{CS}_\mu(\vec{r}) = 0.
	\label{eqn:CS}
\end{equation}
This equation has a countably infinite number of solutions
$\varphi^\text{CS}_\mu \in H^1(\R^3, \C)$,
which are the so-called \newterm{Coulomb Sturmians}.
They are \newterm{isoenergetic},
\ie all have the identical energy eigenvalue
\begin{equation}
	E = -\frac{\kexp^2}{2},
	\label{eqn:CSenergy}
\end{equation}
such that the underlying self-adjoint operator
\[ \Op{H}^\text{CS} = - \frac12 \Delta - \frac{n\kexp}{r} \]
has the very simple point spectrum
\[ \sigma_P(\Op{H}^\text{CS}) = \left\{ -\frac{\kexp^2}{2} \right\}, \]
but an empty discrete spectrum,
thus $\sigma_P(\Op{H}^\text{CS}) \subset \sigma_\text{ess}(\Op{H}^\text{CS})$,
which allows to capture continuum effects in a truncated basis.

Since \eqref{eqn:CS} and the hydrogen-like Schrödinger equation
\eqref{eqn:HydrogenComplete} are very similar,
we can apply the solution approach discussed in section \vref{sec:HydrogenAtom}
to equation \eqref{eqn:CS} as well.
Inserting a product ansatz of radial part and spherical harmonic
\begin{equation}
	\varphi^\text{CS}_\mu(\vec{r}) \equiv
	\varphi^\text{CS}_{nlm}(\vec{r}) = R^{\text{CS}}_{nl}(r) Y_l^m(\uvec{r})
	\equiv
	R^{\text{CS}}_{nl}(r) Y_l^m(\theta, \phi)
	\label{eqn:CSproduct}
\end{equation}
into \eqref{eqn:CS} allows to obtain the Coulomb-Sturmian radial equation
\begin{equation}
	\left( - \frac{1}{2r^2} \laplaceRadial + \frac{l (l+1)}{2 r^2}
	- \frac{n\kexp}{r} - E \right) R_{nl}(r) = 0.
	\label{eqn:CSRadial}
\end{equation}
Its solutions have the form
\begin{equation}
	R^\text{CS}_{nl}(r) = N_{nl} (2\kexp r)^l e^{-\kexp r}
	\;_1F_1\left(l+1-n \middle| 2l+2 \middle|2\kexp r\right)
	\label{eqn:CSradialSolution}
\end{equation}
with normalisation constant
\[ N_{nl} = \frac{2 \kexp^{3/2}}{(2l+1)!} \sqrt{ \frac{(l+n)!}{n (n-l-1)!}}. \]
Unsurprisingly this functional form is closely related to the radial part of the
hydrogen-like orbitals \eqref{eqn:HydrogenRadialSolution}.
In fact the Coulomb Sturmians can be constructed
from the equivalent hydrogen-like orbitals
just by replacing the factors $Z/r$ by $\kexp$.
In analogy one therefore commonly uses the spectroscopic terminology
$1s$, $2s$, $2p$, $\ldots$
to describe the respective triples
of quantum numbers $(n, l, m)$ for Coulomb Sturmians as well.
Given that both {\STO}s as well as \CS functions are exponential type orbitals
of the form radial part times spherical harmonic,
their radial parts \eqref{eqn:CSradialSolution} and \eqref{eqn:STOradial}
are related as well%
\footnote{In fact some recent work~\cite{Avery2017} exploits this to evaluate
\STO \ERI integrals with Coulomb-Sturmians.}.
The important difference between both types of orbitals is
that{\STO} basis sets may use a different Slater exponent $\zeta_\mu$ for each
\STO basis function,
whereas all \CS functions share the same exponent $\kexp$
as a commonly modified parameter.
Even though this difference is subtle,
it is the key ingredient to derive the efficient evaluation schemes
of the \CS-\ERI tensor discussed further down this section.

In their original work \citet{Shull1959} did not yet use the term ``Coulomb Sturmians''
to refer to the functions $\varphi^\text{CS}_\mu$.
This name was only introduced a few years later
by Rotenberg~\cite{Rotenberg1962,Rotenberg1970},
who managed to find a link between the \CS radial equation \eqref{eqn:CSRadial}
and the special class of Sturm-Liouville differential equations.
Sturm-Liouville equations are second order differential equation of the form
\begin{equation}
	\left(
	\frac{\D}{\D r} \left( p(r) \frac{\D}{\D r} \right) + q(r) + \lambda_n w(r) \right) u_n(r) = 0,
	\label{eqn:SturmLiouville}
\end{equation}
where $p(r) \in C^1(\Omega, \R)$ and $q(r), w(r) \in C^0(\Omega, \R)$ are all positive
functions and $\Omega = (a,b) \subset \R$ is an open interval.
Provided that on $a$ and $b$ suitable boundary conditions
\begin{align*}
	u_i(a) \cos \alpha - p(a) u_i'(a) \sin \alpha &= 0 & 0 &< \alpha < \pi \\
	u_i(b) \cos \beta - p(b) u_i'(b) \sin \beta &= 0 & 0 &< \beta < \pi
\end{align*}
are chosen,
the eigenvalues $\lambda_i$ are real and non-degenerate
\[ \lambda_1 < \lambda_2 < \lambda_3 < \cdots < \lambda_n < \cdots \to \infty \]
and the eigenfunctions $u_i$ can be normalised
to satisfy the weighted orthonormality condition
\begin{equation}
	\int_a^b u_i^\ast(r) w(r) u_j(r) \D r = \delta_{ij}.
	\label{eqn:SturmLiouvilleWeightedOrtho}
\end{equation}
Following Rotenberg~\cite{Rotenberg1962,Rotenberg1970} we can use the ansatz
\[ R_{nl}(r) = \frac{u_{nl}(r)}{r} \]
as well as \eqref{eqn:CSenergy} to rewrite the Coulomb-Sturmian radial equation
\eqref{eqn:CSRadial} as
\[ \left(\frac{\partial^2}{\partial r^2} - \frac{l (l+1)}{r^2} - \frac{k^2}{2} + \frac{kn}{r} \right)
u_{nl} = 0, \]
which is of Sturm-Liouville form with
\begin{align*}
	p(r) &= 1 & q(r) &= \frac{\kexp^2}{2} + \frac{l (l+1)}{r^2} & \lambda_n w(r) = \frac{n\kexp}{r}
\end{align*}
One consequence of this is that Coulomb Sturmians satisfy
the \newterm{potential-weighted orthonormality} condition~\cite{Avery2006}
\begin{equation}
	\int_{\R^3} \cc{\left(\varphi^\text{CS}_{nlm}(\vec{r})\right)}
		\frac{n}{r\kexp}  \varphi^\text{CS}_{n'l'm'}(\vec{r})
		\D \vec{r} = \delta_{n n'} \delta_{l l'} \delta_{m m'}.
	\label{eqn:CoulombSturmianWeightedOrtho}
\end{equation}
Most importantly, however, one is able to show
that the countably infinite set of all Coulomb Sturmians $\{\varphi^\text{CS}_\mu\}_{\mu}$
is a complete basis for $H^1(\R^3, \R)$~\cite[Theorem 2.3.4]{Avery2008}.
In the original context of Shull and Löwdin this implies that Coulomb Sturmians
are not only able to represent the bound states
of any atomic Schrödinger operator $\Op{H}_{\Nelec}$,
but the continuum as well.
When it comes to the discretised \HF problem~(see section \ref{sec:DiscreteHF})
or the \FCI problem~(see remark \ref{rem:FCI}),
this makes \CS basis functions rather promising as well,
since the completeness property
provides a mathematical guarantee that the exact solution can be approximated
arbitrarily closely
if more and more \CS functions are included.

\begin{figure}[t]
	\centering
	\includeimage{4_solving_hf/relative_error_cs}
	\caption[
		Relative error in the hydrogen \HF ground state
		for selected \CS basis sets.
	]
	{Relative error in the hydrogen \HF ground state
		for selected \CS basis sets.
		The error is plotted against
		the relative distance of electron and proton.
		The optimal value for $\kexp$ for hydrogen is $1.0$.
	}
	\label{fig:RelativeErrorCS}
\end{figure}

\noindent
Since \CS functions contain the term
\[ \exp(-\kexp r) = \exp\left( - \sqrt{-2E}r \right), \]
which both gives rise to a cusp at $r=0$ as well as
an energy-dependent exponential decay at $r\to\infty$,
they reflect the physical properties summarised in remark \vref{rem:PhysicalProperties}
already at the level of basis functions.
In general both the slope at the cusp as well as the decay at infinity
of the individual basis functions will \emph{not} be exactly the physically
required values, however.
Due to the completeness of the \CS basis
this is not an issue, since a large enough basis will recover the error
for each $\kexp$.%
%
\begin{figure}[p]
	\centering
	\includeimage{4_solving_hf/local_energy_cs}
	\caption[Local energy of the hydrogen \HF ground state for {\CS} bases]{
		Local energy $E_L(r)$ of the hydrogen atom \HF ground state
		of selected Coulomb-Sturmian basis sets.
		$E_L(r)$ is plotted against the relative distance
		of electron and nucleus.
		The optimal value for $\kexp$ for hydrogen is $1.0$.
	}
	\label{fig:LocalEnergyCS}
\end{figure}
%
\begin{figure}[p]
	\centering
	\includeimage{4_solving_hf/local_energy_cs_zoom}
	\caption[Local energy of the hydrogen \HF ground state for {\cGTO}s (magnified)]{
		Magnified version of figure \vref{fig:LocalEnergyCS}
		around the origin.
		The orange curve theoretically goes to $-\infty$ as well,
		but the slope is so large that this is not visible
		at the resolution level of the plot.
	}
	\label{fig:LocalEnergyCSZoom}
\end{figure}
%
To illustrate this for an example consider figure \vref{fig:RelativeErrorCS},
which shows the relative error in the hydrogen ground state
versus the relative electron-nucleus distance
for a few selected Coulomb-Sturmian basis sets.
The labels of the plots both indicate the $\kexp$ value as well as the
triple  $(\nmax, \lmax, \mmax)$, which is a short hand for indicating
the finite basis set
\[ \left\{ \varphi^\text{CS}_{nlm} \, \middle| \, n \leq \nmax, l \leq \lmax, \abs{m} \leq \mmax \right\} \]
of \CS functions.
The best $\kexp$ value for hydrogen is $1.0$,
which would in fact give the exact ground state in the $\varphi^\text{CS}_{1s}$ function.
Figure \vref{fig:RelativeErrorCS} shows in agreement with our previous discussion
that both the size of the basis as well as the value for $\kexp$ has an influence
on the relative error.
Since the slope at which the \CS functions decay
at infinity depends on $\kexp$ ---
with larger values leading to faster decay ---
it is not surprising to find that
a too large value for $\kexp$ leads to a negative relative error at $r = \pm \infty$,
whilst a too small value for $\kexp$ leads to a positive error.
Similarly larger deviations of $\kexp$ from $1.0$
cause the relative error to become larger in magnitude throughout the curve:
Compare the blue and the orange curve with $\kexp = 1.4$ and $\kexp = 1.2$
for example.
The relative error does, however, not scale linearly with $\kexp$.
Yet furthermore it is not even symmetric with respect to the direction
into which $\kexp$ deviates from the optimal value.
In this case the orange curve is less steep as $r\to\infty$ and has a lower value
at the cusp than the green one,
even though both miss the best exponent by $0.2$.
In all systems I investigated so far,
I made the similar observation that the error is more pronounced if the optimal
value for $\kexp$ is underestimated rather than overestimated.
Compared to the effect which $\kexp$ has on the error
the effect of increasing the basis is much more significant.
Even though the green and the red curve both use a $\kexp$,
which is off by $0.2$,
the red curve following a $(5,1,1)$-basis
stays below a relative error of $0.05$
over the full depicted range of distances.
On the other hand the green one, a $(3,1,1)$-basis,
starts to become rather inaccurate from distances
of $7.5$ Bohr and larger.

Very similar conclusions can be drawn
from figure \vref{fig:LocalEnergyCS},
which shows the local energy versus relative distance.
Comparing this plot to the local energy obtained for the \cGTO
discretisations in figure \vref{fig:LocalEnergyCgto}
one notices how the \cGTO local energy has much more wiggles
and overall deviations from the exact value of $0.5$.
Even though the \CS discretisations depicted
in figure \ref{fig:LocalEnergyCS} are not perfect eigenfunctions
of the hydrogen atom,
the local energy is still mostly close to $0.5$,
thus they encode most of the physics.
Even with a too small value $\kexp = 0.8$,
the $(5,1,1)$ basis produces an acceptable eigenfunction over the full depicted
range --- except the nucleus.
This is illustrated in more detail in figure
\ref{fig:LocalEnergyCSZoom},
which is a close-up of the local energies of a $(3,1,1)$, a $(5,1,1)$
and a $(7,1,1)$ discretisation for $\kexp = 0.8$
around the nucleus.
Whilst the $(3,1,1)$ and the $(5,1,1)$ both decay visibly to $-\infty$ at the origin,
the $(7,1,1)$ discretisation already mostly corrects for this.
Even though it still goes to $-\infty$ in theory,
the resolution of the plot is no longer good enough to show this properly.
From the illustrated trends it is clear that
\CS discretisations are able to represent
both the exponential decay as well as the electron-nuclear cusp
up to any desired accuracy if the basis is chosen large enough.
More examples discussing the convergence behaviour of
\CS discretisations can be found in chapter \vref{ch:CSQChem}.

Next to the ability of a basis function type to properly
represent the physics of a chemical system, we also need to be able
to solve the arising numerical problems
in order to make it useful for practical quantum-chemical calculations.
Similar to the other basis function types discussed so far,
we will therefore now turn our attention to the Fock matrix $\matFnfull$,
both its structure as well as its diagonalisation.
For this we first consider the computation of the integrals
\eqref{eqn:Tbas} to \eqref{eqn:Sbas},
starting with the overlap matrix.
Its elements $S_{\mu\mu'}$ can be computed
for any two Coulomb Sturmians%
\footnote{We will drop the ``CS'' superscripts for basis functions
and radial parts in the remainder of this section for simplicity.}
$\varphi_{\mu}$ and $\varphi_{\mu'}$
by treating radial and angular part separately~\cite{Avery2015}
\begin{equation}
	\begin{aligned}
	S_{\mu\mu'} &= \int_{\R^3} \cc{\varphi_{\mu}}(\vec{r}) \varphi_{\mu'}(\vec{r})  \D\vec{r} \\
	&= \int_0^\infty R_{nl}(r) R_{n'l'}(r) r^2 \D r \cdot
		\int_{\set{S}^2} \cc{\left(Y^m_l\right)}\!(\uvec{r}) \,
			Y^{m'}_{l'}(\uvec{r}) \D \uvec{r} \\
			&= \delta_{m m'} \delta_{l l'}
				\underbrace{\int_0^\infty R_{nl}(r) R_{n'l}(r) r^2 \D r}_{= s^{(l)}_{n n'}},
	\end{aligned}
	\label{eqn:OverlapCS}
\end{equation}
Normalisation implies that $s^{(l)}_{n n} = 1$
and the potential-weighted orthonormality \eqref{eqn:CoulombSturmianWeightedOrtho}
implies that $s^{(l)}_{n n'} = 0$ iff $\abs{n -n'} > 1$.
By following the algebra one can further show~\cite{Avery2015} that
\[
	s^{(l)}_{n,n+1} = s^{(l)}_{n+1,n} = - \frac12 \sqrt{\frac{(n-l)(n+l+1)}{n (n+1)}},
\]
making $\mat{S}$ both sparse and trivial to compute.
Similarly one can directly employ the potential-weighted orthonormality \eqref{eqn:CoulombSturmianWeightedOrtho}
to show that the nuclear attraction matrix is diagonal, namely
\begin{equation}
	\begin{aligned}
	\left(V_0\right)_{\mu\mu'} &=
	\int_{\R^3} \cc{\varphi_{\mu}}(\vec{r}) \frac{Z}{r} \varphi_{\mu'}(\vec{r})  \D\vec{r} \\
	&= \frac{Z\kexp}{n'} \int_{\R^3} \cc{\varphi_{\mu}}(\vec{r}) \frac{n'}{r\kexp} \varphi_{\mu'}(\vec{r})  \D\vec{r} \\
	&= \delta_{\mu \mu'} \frac{Z\kexp}{n}.
	\end{aligned}
	\label{eqn:NucAttrCS}
\end{equation}
From \eqref{eqn:CSbeta} to \eqref{eqn:CSenergy} we get
\[ \left(- \frac12 \Delta - \frac{n \kexp}{r} + \frac{\kexp^2}{2} \right) \varphi_{\mu}(\vec{r}) = 0, \]
which implies for the kinetic energy matrix elements
\begin{equation}
	\begin{aligned}
		\left(V_0\right)_{\mu\mu'} &=
		\int_{\R^3} \cc{\varphi_{\mu}}(\vec{r}) \left( - \frac12 \Delta  \right)
		\varphi_{\mu'}(\vec{r})  \D\vec{r} \\
		&= \int_{\R^3} \cc{\varphi_{\mu}}(\vec{r})
		\left(\frac{n' \kexp}{r} - \frac{\kexp^2}{2} \right)
		\varphi_{\mu'}(\vec{r})  \D\vec{r} \\
		&= k^2 \left( \delta_{\mu\mu'} - \frac12 S_{\mu\mu'} \right) \\
		&= k^2 \delta_{ll'} \delta_{mm'} \left( \delta_{nn'} - \frac{1}{2} s^{(l)}_{n n'} \right),
	\end{aligned}
	\label{eqn:KineticCS}
\end{equation}
such that they follow the same advantageous sparsity pattern as the overlap matrix.
The one-electron integrals thus contain at most 3 non-zeros per row
and are tridiagonal in $lmn$ or $mln$ index order.
Due to the simplicity of the expressions of the matrix elements,
storing these matrix terms in memory --- even in a compressed tridiagonal form ---
is not needed,
since recomputing the values takes a neglible number of flops.

Unsurprisingly treating the two-electron integrals is more involved.
We follow \cite{Avery2015},
which describes the treatment in a more general context
and the specialised arguments presented in the documentaton of
\sturmint~\cite{sturmintWeb}.
Due to the structure of the radial part $R_{nl}(r)$
one may write the product of two Coulomb Sturmians as a
sum over Coulomb Sturmians with twice the exponent, \ie
\newcommand{\csC}{\mathcal{C}}
\begin{equation}
	\cc{\varphi_{\mu_1}(\vec{r})} \varphi_{\mu_2}(\vec{r})
	= \sum_{\mu} \csC^\mu_{\mu_1,\mu_2} \varphi_{\mu}(2\kexp, \vec{r})
	\label{eqn:CSshellpairExpansion}
\end{equation}
This expansion looks familiar to the density-fitting approximation
in the context of \cGTO basis sets,
but is in fact \emph{exact} in the case of Coulomb Sturmians.
Since
\[ \cc{\left( \cc{\varphi_{\mu_1}(\vec{r})} \varphi_{\mu_2}(\vec{r}) \right)} =
\cc{\varphi_{\mu_2}(\vec{r})} \varphi_{\mu_1}(\vec{r}) \]
it follows that the conjugated product requires the related
expansion coefficients $\csC^\mu_{\mu_2,\mu_1}$.
With this the electron repulsion integral tensor in Mullikan index \eqref{eqn:ERI}
ordering may be written as the contraction
\begin{equation}
	\eriMu{\mu_1\mu_2}{\mu_3\mu_4}
		= \sum_{\mu\mu'} \cc{\left(\csC^\mu_{\mu_1,\mu_2}\right)} \,
		I_{\mu\mu'} \, \csC^{\mu'}_{\mu_3,\mu_4}
		= \sum_{\mu\mu'} \csC^\mu_{\mu_2,\mu_1} \,
		I_{\mu\mu'} \, \csC^{\mu'}_{\mu_3,\mu_4}
	\label{eqn:CSeriContraction}
\end{equation}
where $I_{\mu\mu'}$ is the electron repulsion kernel in terms of the $2\kexp$-functions
\begin{equation}
	I_{\mu\mu'} \equiv
	\int_{\R^3} \int_{\R^3}
	\frac{
	\cc{\varphi_{\mu'}}(2k,\vec{r}_1) \,
	\varphi_{\mu}(2k,\vec{r}_2)
	}{r_{12}}
	\D\vec{r}_1 \D\vec{r}_2.
	\label{eqn:CSeriKernelDef}
\end{equation}
Using the well-known expansion of the Coulomb operator
in terms of spherical harmonics
\[ \frac{1}{r_{12}} = \sum_{l''=0}^\infty
	\frac{r^{l''}_<}{r_>^{l'' + 1}} \frac{4\pi}{2l'' + 1}
	\sum_{m''=-{l''}}^{l''} Y_{l''}^{m''}(\uvec{r}_1)
	\cc{\left(Y_{l''}^{m''}(\uvec{r}_2)\right)}
\]
where
\begin{align*}
	r_< &\equiv \min(r_1, r_2) & r_> &\equiv \max(r_1, r_2)
\end{align*}
equation \eqref{eqn:CSeriKernelDef} may be rewritten as
\begin{equation}
\begin{aligned}
	I_{\mu\mu'}
	&= \sum_{l''=0}^{\infty}\frac{4\pi}{2l''+1}
	\sum_{m''=-l''}^{l''}
	\int_0^\infty \int_0^\infty
	r_1^2 R_{nl}(2\kexp,r_1) \, r_2^2 R_{n'l'}(2\kexp,r_2) \frac{r^{l''}_<}{r_>^{l'' + 1}}
	\, \D r_1  \D r_2 \\
	&\hspace{30pt}
	\cdot \int_{\set{S}^2}
		\underbrace{\cc{\left(Y_l^m\right)}\!(\uvec{r}_1) \, Y_{l''}^{m''}(\uvec{r}_1)}
		_{=\delta_{l,l''}\delta_{m,m''}}
		\D \uvec{r}_1
	\cdot \int_{\set{S}^2}
		\underbrace{Y_{l'}^{m'}(\uvec{r}_2) \cc{\left(Y_{l''}^{m''}\right)}\!\!(\uvec{r}_2)}
		_{=\delta_{l',l''}\delta_{m',m''}}
		\, \D \uvec{r}_2 \\
	&= \delta_{ll'} \delta_{mm'} I^{(l)}_{nn'}
\end{aligned}
\label{eqn:CSeriKernel}
\end{equation}
where
\begin{equation}
	I^{(l)}_{nn'}
	= \frac{4\pi}{2l+1}
	\int_0^\infty \int_0^\infty
	r_1^2 R_{nl}(2\kexp,r_1) \, r_2^2 R_{n'l}(2\kexp,r_2) \frac{r^{l}_<}{r_>^{l + 1}}
	\, \D r_1  \D r_2.
	\label{eqn:CSeriKernelSmall}
\end{equation}
It is not immediately obvious from the form of equation \eqref{eqn:CSeriKernelSmall},
but the dependency on $\kexp$ can be factored out of this expression,
such that it only depends on $n$, $n'$ and $l$.
If we assume for the principle quantum number $n \leq 20$, which is rather typical,
the tensor $I^{(l)}_{nn'}$ has only about $20^3 = 8000$ elements,
which can be pre-evaluated and stored inside the program.
In fact even more simplifications are possible if one inserts
the definition of the radial parts and splits the integration kernel by powers of
$r_1$ and $r_2$.
For the required polynomial powers $\alpha, \beta$ the integrals
\[ \int_0^\infty \int_0^\infty r_1^\alpha r_2^\beta \exp(-r_1) \exp(-r_2)
	\frac{r^{l}_<}{r_>^{l + 1}} \, \D r_1  \D r_2 \]
can then be precomputed and stored as a vector.
At runtime one only needs to form the dot product of the precomputed vector
with the appropriate vector of polynomial coefficients
to yield the value for $I^{(l)}_{nn'}$.

\noindent
Let us now return to equation \eqref{eqn:CSshellpairExpansion}, \ie
\[
	\cc{\varphi_{\mu_1}(\vec{r})} \varphi_{\mu_2}(\vec{r})
	= \sum_{\mu} \csC^\mu_{\mu_1,\mu_2} \varphi_{\mu}(2\kexp, \vec{r}).
\]
To obtain an expression for the coefficients $\csC^\mu_{\mu_1,\mu_2}$
we multiply this equation with
$\cc{\left(\varphi_{\mu'}\right)}(2\kexp, \vec{r})$
from the right and integrate over $\R^3$.
Using the potential-weighted orthonormality \eqref{eqn:CoulombSturmianWeightedOrtho}
for the $2\kexp$ Coulomb Sturmians
this yields
\begin{equation}
	\begin{aligned}
	\csC^\mu_{\mu_1,\mu_2}
	&= \frac{n}{2k}
	\int_{\R^3}
	\cc{\varphi_{\mu_1}(\vec{r})} \varphi_{\mu_2}(\vec{r})
		\frac{1}{r} \varphi_{\mu}(2\kexp, \vec{r})
	\D \vec{r}. \\
	&= \frac{n}{2k}
		\int_0^\infty R_{n_1,l_1}(r) R_{n_2,l_2}(r) R_{n,l}(2k, r) r \D r \\
	&\hspace{30pt}
		\cdot \int_{\set{S}^2}
			\cc{\left(Y_l^m\right)}\!(\uvec{r}) \,
			\cc{\left(Y_{l_1}^{m_1}\right)}\!(\uvec{r}) \,
			Y_{l_2}^{m_2}(\uvec{r})
		\D \uvec{r}.
	\end{aligned}
	\label{eqn:CScoefficientExpression}
\end{equation}
The angular part of the latter expression can be written in terms of
Clebsch-Gordan coefficients, which are precomputed and stored%
\footnote{Due to the sparsity and symmetry properties
	of the Clebsch-Gordan coefficients even for a large value
	maximal principle quantum number like $n=20$
	no more than a few
	hundred thousand such coefficients need to be stored.
	If some recursion relations are taken into account as well,
	it's far less.
}.
The properties of the Clebsch-Gordan coefficients imply that
$\csC^\mu_{\mu_1,\mu_2}$ can only be non-zero if
\begin{align*}
	m &= m_2 - m_1 &&\text{and}&
	l &\in \big[\abs{l_1-l_2}, l_1+l_2\big],
\end{align*}
such that $\csC^\mu_{\mu_1,\mu_2}$ is again a sparse tensor.
The radial part is computed similar to \eqref{eqn:CSeriKernelSmall},
\ie as a dot product between polynomial coefficients
and precomputed kernels over polynomial powers.

Due to the outlined sparsity of the $2\kexp$-kernel $I_{\mu\mu'}$
and the expansion coefficients $\csC^\mu_{\mu_1,\mu_2}$
the contraction in equation \eqref{eqn:CSeriContraction} can be written
more effectively as
\begin{equation}
	\begin{aligned}
	\eriMu{\mu_1\mu_2}{\mu_3\mu_4}
		&= \sum_{\mu\mu'} \csC^\mu_{\mu_2,\mu_1}
		% note: order has to be 2, then 1 since this implies a complex conjugation
		\, I_{\mu\mu'} \, \csC^{\mu'}_{\mu_3,\mu_4} \\
		&= \sum_{n,l,m} \sum_{n',l',m'}
			\csC^{(n,l,m)}_{\mu_2,\mu_1}
			\, \delta_{ll'} \delta_{mm'} I^{(l)}_{nn'} \,
			\csC^{(n',l',m')}_{\mu_3,\mu_4} \\
		&= \sum_{n'} \sum_{n,l,m}
			\csC^{(n,l,m)}_{\mu_2,\mu_1}
			\, I^{(l)}_{nn'} \,
			\csC^{(n',l,m)}_{\mu_3,\mu_4} \\
		&= \delta_{m_1 - m_2, m_4 - m_3}
		\sum_{l=\lmin}^{\lmax} \sum_{n=l+1}^{n_1+n_2-1} \sum_{n'=l+1}^{n_3+n_4-1}
			\csC^{(n,l,m_2-m_1)}_{\mu_2,\mu_1}
			\, I^{(l)}_{nn'} \,
			\csC^{(n',l,m_2-m_1)}_{\mu_3,\mu_4}
	\end{aligned}
	\label{eqn:CSeriContractionEffective}
\end{equation}
where
\begin{align}
	\lmin &= \max(\abs{l_1-l_2},\abs{l_3-l_4}) &
	\lmax &= \min(l_1+l_2, l_3+l_4).
	\label{eqn:CSminmax}
\end{align}
Because of the selection rules in the quantum numbers $l$ and $m$
the \ERI tensor is thus a sparse quantity
with far less than $\Nbas^4$ non-zeros.
When contracting it with the occupied coefficients $\mat{C}$
to form the Coulomb and exchange matrices, \ie compute the elements
\begin{align}
	\label{eqn:CScoulomb}
	J_{\mu_3\mu_4}\!\left[\mat{C}^{(n)}\left(\mat{C}^{(n)}\right)^\dagger\right]
	&= \sum_{i\in\Iocc} \, \sum_{\mu_1,\mu_2 \in\Ibas} \, \sum_{\mu,\mu'\in\Ibas}
		C^{(n)}_{\mu_1 i} C^{(n)\ast}_{\mu_2 i} \,
		\csC^{\mu}_{\mu_2,\mu_1}
		I_{\mu\mu'} \csC^{\mu'}_{\mu_3,\mu_4} \\
\intertext{and}
	\label{eqn:CSexchange}
	K_{\mu_3\mu_4}\!\left[\mat{C}^{(n)}\left(\mat{C}^{(n)}\right)^\dagger\right]
	&= \sum_{i\in\Iocc} \, \sum_{\mu_1,\mu_2 \in\Ibas} \, \sum_{\mu,\mu'\in\Ibas}
		C^{(n)}_{\mu_1 i} C^{(n)\ast}_{\mu_2 i} \,
		\csC^{\mu}_{\mu_2,\mu_3}
		I_{\mu\mu'} \csC^{\mu'}_{\mu_1,\mu_4},
\end{align}
the sparsity is partially lost.
The reason is that the sum over the occupied orbital index $i$ implies that
each element $J_{\mu_3\mu_4}$ or $K_{\mu_3\mu_4}$
becomes a linear combination of contributions from
different angular quantum number pairs $(l_1, m_1)$ and $(l_2, m_2)$.
Thus a Coulomb or exchange matrix element is only a known zero
if \emph{all} of the possible combinations of the indices $\mu_3$, $\mu_4$
with the pairs $(l_1, m_1)$ and $(l_2, m_2)$ are guaranteed to be zero
--- a much weaker selection rule.
For forming the matrix-vector products of $\mat{J}$ and $\mat{K}$
with other vectors
therefore most elements of $J_{\mu_3\mu_4}$ and $K_{\mu_3\mu_4}$
need to be touched.
For the exchange matrix $\mat{K}$ in fact all elements may be non-zero,
giving rise to a full quadratic scaling of a matrix-vector product
in the number of basis functions.
On the other hand avoiding the storage of $\eriMu{\mu_1\mu_2}{\mu_3\mu_4}$
and $\mat{K}$ in favour of directly computing
the matrix-vector product expression
\begin{equation}
	\left(\matK \vec{x}\right)_{\mu_3}
	= \sum_{i\in\Iocc} \, \sum_{\mu_1,\mu_2,\mu_4 \in\Ibas} \, \sum_{\mu,\mu'\in\Ibas}
		C^{(n)}_{\mu_1 i} C^{(n)\ast}_{\mu_2 i} \,
		\csC^{\mu}_{\mu_2,\mu_3}
		I_{\mu\mu'} \csC^{\mu'}_{\mu_1,\mu_4}
		\, x_{\mu_4}
	\label{eqn:ApplicationKcs}
\end{equation}
whenever the contraction of $\mat{K}$ with a vector $\vec{x}$ is needed,
allows to fully exploit all angular momentum
selection rules during the evaluation.
With this one may achieve the best possible scaling, certainly below quadratic.
Notice, that an efficient contraction scheme for computing \eqref{eqn:ApplicationKcs}
will carry out the contraction over occupied orbitals (index $i$) at the very end.
In other words the improved scaling originating from \eqref{eqn:ApplicationKcs}
can only be achieved if $\mat{K}$ is not in memory
and if the occupied coefficients $\mat{C}$ are available as separate quantities
and not already contracted into a density matrix.

\begin{figure}
	\centering
	\includeimage[width=\textwidth]{4_solving_hf/fock_sturmian}
	\caption[Structure of the Fock matrix for a Coulomb-Sturmian based SCF]
		{Structure of the Fock matrix for a Coulomb-Sturmian based SCF
		for the beryllium atom starting from using a $(5,1,1)$
		Coulomb-Sturmian basis in $mln$ order
		and a Sturmian exponent of $\kexp = 1.99$.
		The three figures show left to right the Fock matrix
		at an SCF step with a Pulay error Frobenius norm of
		$0.13$, $0.0079$, $6.7 \cdot 10^{-8}$.
		The colouring depends on the absolute value
		of the respective Fock matrix entry
		with white indicating entries below $10^{-8}$.
		}
		% 14 out of 17 are diagonal-dominant in each case.
		% Matrix is symmetric
	\label{fig:StructureSturmianFock}
\end{figure}
Both the very simple form of the one-electron matrices,
given by the expressions
\eqref{eqn:OverlapCS},
\eqref{eqn:NucAttrCS} and
\eqref{eqn:KineticCS},
as well as the previous discussion about the angular momentum selection
rules in the case of the Coulomb and exchange matrices
suggests to employ a \contraction-based scheme
for a Coulomb-Sturmian-based \SCF.
Looking at the structure of the Fock matrix $\matF$
in figure \vref{fig:StructureSturmianFock},
we notice that it is very similar to the \cGTO case~(figure \vref{fig:StructureGaussianFock}).
Most notably it is almost diagonal dominant and of a similar
size than the \cGTO Fock matrix.
In other words a dense diagonalisation method could in theory be employed
for the Fock matrix $\matF$ as well.
The downside of a dense scheme would be the higher
storage requirement as well as the larger
computational scaling of the matrix-vector product.
Whilst \CS discretisations on the one hand
do not require \contraction-based methods to be feasible,
they still allow for improved contraction if such methods are employed.

While Coulomb Sturmians can only be used for simulating atoms,
a range of more generalised Sturmian-type basis functions exist%
~\cite{Hoggan2009,Avery2011},
which can be applied, for example, to molecular systems as well.
Especially when it comes to evaluating the two-electron integrals,
these share some of the properties of the Coulomb Sturmians,
but both the mathematical machinery as well as the numerics is more involved.
\CS functions can thus be seen as a first step towards these
more general Sturmian-type basis functions.
Generalised Sturmian-type orbitals are an active field of research%
~\cite{%
	Gruzdev1990,Avery2003,Avery2006,Hoggan2009,%
	Randazzo2010,Mitnik2011,%
	Avery2011,Avery2011PhD,%
	Avery2013,Avery2015,Randazzo2015,Granados2016,Abdouraman2016,%
	Morales2016,Avery2017,Avery2018%
}.
Some recent works include efforts to develop schemes for the
fast evaluation of the resulting \ERI tensor~\cite{Avery2013,Avery2017,Avery2018}
as well as the application of Sturmian-type functions
for evaluating \STO integrals more efficiently~\cite{Morales2016,Avery2017}.
Other methods include the combination of
Sturmians and some numerical methods to yield ionising Sturmians
to simultaneously model bound states as well as the continuum
in a single basis~\cite{Randazzo2010,Mitnik2011,Randazzo2015,Granados2016,Abdouraman2016}.

% TODO OPTIONAL
%In the most general sense Sturmian-type orbitals are the solution
%to the $\Nelec$-body Schrödinger-like equation
%\begin{equation}
%	\left( -\frac12 \Delta + \beta_{\mu} V_0(\vec{x}) - E \right) \Phi^\text{St}_\mu(\vec{x}) = 0
%	\label{eqn:SturmianGeneral}
%\end{equation}
%where
%\[ V_0(\vec{x}) = V_0(\vec{r}_1, \vec{r}_2, \ldots, \vec{r}_{\Nelec}) \]
%is a good zeroth order approximation to the actual potential
%\[ V(\vec{x}) = \sum_{A=1}^M \sum_{i=1}^{\Nelec} \frac{Z_A}{\norm{\vec{R}_A - \vec{r}_i}}
%+ \sum_{i=1}^{\Nelec} \sum_{j=i+1}^{\Nelec} \frac{1}{\norm{\vec{r}_i - \vec{r}_j}_2} \]
%such that \eqref{eqn:SturmianGeneral} can be analytically solved
%and $\beta_{\mu}$ is chosen to make the solutions isoenergetic.
%For many-electron atoms Goscinski for example suggested
%$V_0(\vec{x}) = \sum_i \frac{Z}{r_i}$.
%Many-centre Sturmians are possible, too~\cite{Avery2003,Avery2013}.
%In this broad sense Sturmian-type orbitals are an active
%field of research


\subsection{Other types of basis functions}
The selection of basis function types presented so far
represents a fair amount of what is used for electronic structure theory
calculations nowadays.
Nevertheless there are few more basis function types,
which should not go unmentioned.

This first of all applies to
plane-wave and projector-augmented wave approaches
~\cite{Kresse1996,Kresse1999,Mortensen2005,Enkovaara2010},
which are both extremely popular as well as extremely suitable for performing
electronic structure calculations
on extended periodic systems or systems in the solid state.
Over the years there has also been an enormous amount of development
into the direction of numerical basis functions.
\citet{Frediani2015} provides an excellent review.
The approaches range from a so-called fully numerical treatment,
where similar to the finite-element method as mentioned above,
the complete problem is treated by grid-based methods.
This includes employing clever
numerical integration grids~\cite{Losilla2012DCRsp,Toivanen2015,Enkovaara2010}
or discretisation schemes
based on finite-differences~\cite{Kobus2013}
or finite-elements~\cite{Tsuchida1995,Briggs1996,Pask05,Lehtovaara2009,Alizadegan2010,Avery2011PhD,Davydov2015,Boffi2016}.
Some approaches~\cite{Soler2002} use a dual representation,
where the same orbitals are both represented on a real-space grid
as well as in the form of orbitals
or they only treat
part of the electronic wave function numerically~\cite{Fischer1978,LUCAS}.
Such methods for example employ a factorisation of the one-particle functions
into a numerical radial part and a spherical harmonic function.
Last but not least one should also mention
wavelet-based methods~\cite{Bischoff2011,Bischoff2012,Bischoff2013,Bischoff2014,Bischoff2014a,Bischoff2017},
where quite some progress has been made in recent years.
To the best of my knowledge wavelet-based electronic structure theory
is the only methodology where guaranteed precision in the solution to the
respective problems can be achieved.

\subsection{Mixed bases}
In theory nothing stops us from using more than a single type of basis function
in a discretisation.
In fact some of the methods mentioned before like
the SIESTA method~\cite{Soler2002} or the projector-augmented
plane-wave approaches~\cite{Kresse1996,Kresse1999,Mortensen2005,Enkovaara2010}
are in fact cases, where more than one type of basis function is employed.
For example SIESTA uses two discretisations at once
--- one in terms of {\cGTO}s and a numerical grid ---
and projects back and force between them,
whilst projector-augmented plane-wave methods use a plane-wave basis
augmented with other types of basis functions close to the atom cores.

In practice not all combinations of basis functions are feasible or sensible.
This can be rationalised by looking at the {\cGTO} or the \CS discretisations,
which both heavily rely on basis-specific properties for efficiently
computing the electron-repulsion integrals
in order to make the computation of the Fock matrix $\mat{F}$
or its matrix-vector product fast.
In a fully mixed basis one not only needs to compute \ERI integrals
between the same type of basis functions,
but also between all combinations of four basis functions involving different types.
In a combination of the two aforementioned basis function types,
this would destroy their advantageous properties.
This is not meant to say that mixture basis sets involving
{\cGTO}s or {\CS} basis functions are not possible or helpful,
but that getting the integrals fast requires some clever schemes.

\subsection{Takeaway}
\label{sec:BasisTakeaway}
In the previous sections we saw that different basis function types
can lead to rather different numerical properties in the discretised \HF problem.
Just considering the three figures
\vref{fig:StructureGaussianFock},
\vref{fig:StructureFiniteElementFock}
and \vref{fig:StructureSturmianFock}
illustrating the structures of the Fock matrices
the overall differences are apparent.
Whilst the number of basis functions of the atom-centred
\cGTO and \CS discretisations depends on the number of atoms
in the chemical system,
\FE discretisations need very similar number of basis functions
for atoms and molecules.
In contrast to \ACO approaches \FE-based discretisations
need many more basis functions, in the order of millions
compared to hundreds for \ACO discretisations.
For this reason only iterative methods
are feasible for a \FE discretisation,
where a \contraction-based scheme can theoretically
lead to linear scaling in the number of basis functions.
On the other hand the finite-element method
does not rely on the intuition of the user very much.
Initial grids can be easily auto-generated and
while the calculation is running adaptively refined.
Nevertheless prior knowledge of the physics can be incorporated
into the initial grid generation.
Leaving the numerical issues aside
the \FE approach in theory comes very close to the ideal
basis function type we sketched in \vref{sec:BasisDesiredProperties}.

From a practical point of view
the \ACO approaches are less black-box,
since more choices about the particular basis set need to be made before the calculation,
but they are numerically much more feasible.
Especially for \cGTO-based methods the evaluation of the integrals
is considerably less challenging compared to {\STO}s, Coulomb-Sturmians
or {\FE}s and is well-understood by now.
Coulomb-Sturmians on the other hand are physically much more sound than
{\cGTO}s allowing to represent the wave function
much better.
Contrast figures \vref{fig:LocalEnergyCgto} and \vref{fig:LocalEnergyCS} for example.
Unlike the \STO based approaches
the integrals in \CS discretisations can be evaluated rather efficiently
due to the restriction to a single exponent $k$.
In a \contraction-based ansatz linear scaling in the basis set size
is theoretically possible.
As figure \ref{fig:LocalEnergyCS} suggests the convergence properties
can be expected to be rather decent and predictable
going to larger and larger basis sets.
Originating from the completeness of the \CS functions
with respect to the form domain $Q(\Op{F}) = H^1(\R^3,\R)$
eventually both the long-range part as well as the cusp can be represented
perfectly with larger and larger basis sets.
The convergence properties of \CS basis sets in the context of
quantum-chemical calculations will be investigated
in chapter \vref{ch:CSQChem}.
