\defineabbr{CS}{CS\xspace}{Coulomb-Sturmian or Coulomb-Sturmian basis, see section \vref{sec:BasisCS}}
\defineabbr{cGTO}{cGTO\xspace}{Contracted Gaussian-type orbitals, see section \vref{sec:cGTO}}
\defineabbr{STO}{STO\xspace}{Slater-type orbitals, see section \vref{sec:STO}}
\defineabbr{ETO}{ETO\xspace}{Exponential-type orbitals, orbitals with radial part of the form Polynomial in $r$ times $\exp{-\alpha r}$ with some parameter $\alpha$.}
\defineabbr{FE}{FE\xspace}{Finite elements, see section \vref{sec:FE}}
\defineabbr{ACO}{ACO\xspace}{atom-centered orbitals}

\section{Basis function types}
This section tries to address the question,
which classes of functions can be used
in order to build a basis set $\{\varphi_\mu\}_{\mu\in\Ibas}$
for solving the \HF problem in an \SCF procedure.
For this we will first discuss some desirable properties for a basis set,
both motivated from the aim to represent the physics of the electronic Schrödinger
equation as good as possible
as well as requirements from the numerical side.
In the light of this,
we will discuss four types of basis functions in depth,
namely the historic Slater-type orbitals~({\STO}s),
the most commonly employed contracted Gaussian-type orbitals~({\cGTO}s),
a finite-element based discretisation method
as an example of a fully numerical approach
as well as so-called Coulomb-Sturmian-type orbitals.

Even though we mostly concentrate on the \HF problem in this section,
quite a few of the observations made here
apply to \DFT or methods going beyond Hartree-Fock as well.
In this sense the outlined discussion
can be seen as an example case for the use of the mentioned
basis function types in electronic structure theory as a whole.

\subsection{Desirable properties}
\label{sec:BasisDesiredProperties}
The central aspect of the Ritz-Galerkin procedure for
approximately solving a spectral problem
is the evaluation of the $a(\slot,\slot)$
corresponding to the operator for all pairs of basis functions,
compare with remark \vref{rem:ChoiceBasisFunction} for details.
For this procedure to be mathematically meaningful at all
this requires the basis functions $\{\varphi_\mu\}_{\mu\in\Ibas}$
to be taken from a dense subspace of the form domain of the operator.
For the real-valued \HF problem this is the Sobolev space $H^1(\R^3, \R)$,
thus a hard requirement for all types of basis functions
used for Hartree-Fock and quantum chemistry is that
they originate from $H^1(\R^3, \R)$.
Furthermore in some or another sense we will need compute the elements
of the Fock matrix $\matFfull$ \eqref{eqn:FockMatrix},
which in turn boils down to computing the integrals of the constituent
matrix expressions \eqref{eqn:Tbas} to \eqref{eqn:Kbas},
as well as the overlap matrix \eqref{eqn:Sbas}.
The challenging step for this is typically the evaluation
of the electron repulsion tensor \eqref{eqn:ERI}
\[
	\eriMu{\psi_i \psi_j}{\psi_k \psi_l}
		= \int_{\R^3} \int_{\R^3}
			\frac{\cc{\psi_i}(\vec{r}_1) \psi_j(\vec{r}_1)
				\,\cc{\psi_k}(\vec{r}_2) \psi_l(\vec{r}_2)}
			{\norm{\vec{r}_1 - \vec{r}_2}_2}
			\D \vec{r}_1 \D \vec{r}_2.
\]
as it involves a double integral over space
incorporating a singularity at the origin
as well as the product over four basis functions.
Additionally the discretised \HF equations \eqref{eqn:HFDiscreteEquations}
need to be solvable numerically as well.
We will see in the next sections,
that the main reason why contracted Gaussian-type orbitals
have become so popular in quantum chemistry
is that both evaluating the ERI tensor
as well as solving the resulting eigenproblem
is rather easy compared to the other cases.

Apart from the mathematical and numerical feasibility
we would like to get meaningful results with as little effort as possible,
\ie a good description of a chemical system should already be achievable
with small basis sets.
Usually this goes hand in hand with a basis function,
which by itself represents the physics
of the chemical system very well already
such that as much prior knowledge and chemical intuition as possible
could be incorporated already into basis.
Ideally this would not bias the solution procedure,
such that unexpected or unintuitive results can still be found.

Last but not least we would like to be able to know
how wrong our \HF results are compared to the exact electronic ground state,
possible even with a pointer how to increase the basis,
such that results can systematically improve.
The aspired scenario would be a rigorous and tight
\textit{a priori} or even better \textit{a posteriori} error estimate
for the chosen basis function type in the context of \HF.

Of course this just sketches an ideal scenario.
In reality one needs a good compromise,
typically even a different compromise for different applications.
Especially the \textit{a priori} and \textit{a posteriori} error estimates
are not easy to derive rigorously for \HF
and I am not aware of any work achieving this for the basis function
types I will discuss here in detail.

\subsection{Local energy}
Before we start discussing individual basis types,
let us briefly pause and think about ways to quantitatively
judge a particular basis function type.
A natural choice is to consider a model system,
where the analytical solution can be found and compare
with the Ritz-Galerkin \HF result produced by a particular basis on the same system.
In this chapter, we will typically compare against the hydrogen atom.
Without a doubt this does not probe all aspects of electronic structure theory.
Most importantly it does miss an evaluation how a basis set deals with electron correlation.
All results therefore need to be taken with care:
In more complex systems the situation might be different.

For comparing our numerical answers in the form of the \HF
ground state Slater determinant $\Phi_0$
to the exact electronic Schrödinger equation solution $\Psi_0$,
we will use absolute errors and relative errors in the ground state wavefunction
as well as the ground state energy.
Additionally we will consider a quantity called \newterm{local energy},
which is defined below.
\begin{defn}
	Let $\Phi_0$ be an approximation to the ground state
	of the operator $\Op{H}_{\Nelec}$.
	The local energy is defined by the quotient
	\begin{equation}
		E_L(\vec{x}) \equiv \frac{\Op{H}_{\Nelec} \Phi_0(\vec{x})}{\Phi_0(\vec{x})},
		\label{eqn:LocalEnergy}
	\end{equation}
	which is constant for an exact eigenstate of $\Op{H}_{\Nelec}$
	and approximately constant for good approximations.
	Since the potential operator terms are only multiplicative,
	this expression can be alternatively written as
	\[
		E_L(\vec{x})
		= -\frac12 \sum_{i=1}^{\Nelec} \frac{\Delta_{\vec{r}_i} \Phi_0(\vec{x})}{\Phi_0(\vec{x})}
		- \sum_{i=1}^{\Nelec} \sum_{A=1}^M \frac{Z_a}{\norm{\vec{r} - \vec{R}_A}_2}
		+ \sum_{i=1}^{\Nelec} \sum_{j=1+1}^{\Nelec} \frac{1}{r_{ij}}.
	\]
\end{defn}
The concept of local energy originates from the
quantum Monte Carlo community~\cite{Foulkes2001,Ma2005},
where its sampling by a Monte Carlo procedure plays a central role
for obtaining the correlation energy.
It is related to the relative residual
\[
	\frac{1}{\Phi_0(\vec{x})} \, \left(\Op{H}_{\Nelec} - E_0\right) \Phi_0(\vec{x}) =
	\frac{\Op{H}_{\Nelec} \Phi_0(\vec{x}) - E_0 \Phi_0(\vec{x})}{\Phi_0(\vec{x})}
	= E_L(\vec{x}) - E_0,
\]
where $E_0$ is the \emph{exact} ground state energy of $\Op{H}_{\Nelec}$.
This implies first of all
that $E_L(\vec{x}) = E_0$ is necessary for $\Phi_0(\vec{x})$ being the exact ground state.
Furthermore the fluctuations of $E_L(\vec{x})$ around the exact constant value $E_0$
provide a measure how far $\Phi_0(\vec{x})$ is off from being an
exact eigenstate of $\Op{H}_{\Nelec}$ at a particular point $\vec{x}$.
In this sense $E_L(\vec{x})$ can thus be seen as a \emph{local} measure
for the accuracy of $\Phi_0(\vec{x})$.
Inside regions where $E_L(\vec{x})$ is close to being constant,
the basis $\{\varphi_\mu\}_{\mu\in\Ibas}$ provides a sensible description
of an eigenstate of $\Op{H}_{\Nelec}$.
$E_L(\vec{x})$ is without a doubt conceptionally related to the
relative error in the ground state wave function $1 - \Phi_0(\vec{r}) / \Psi_0(\vec{r})$.
Compared to the latter quantity $E_L(\vec{x})$
has the additional advantage that one is able to notice,
which eigenstate $\Phi_0(\vec{r})$ approximates in each region of space.
For example if it fluctuates around $E_0$ in some areas and around $E_1$ in others,
we can see that $\Phi_0(\vec{r})$ sometimes represents the first excited state
better than the ground state.
Additionally $E_L(\vec{x})$ can be applied even for cases where
the exact solution is not known and thus the relative error cannot be found.

\subsection{Slater-type orbitals}
\label{sec:STO}

In section \vref{sec:HydrogenAtom}
we discussed the analytical solution of the simplest chemical systems,
namely the hydrogen-like atoms or ions with only a single nucleus and a single electron.
Their solutions where functions
\[ \Psi_{nlm}(\vec{r}) = N_{nl} \tilde{P}_{nl}\left(\frac{2Zr}{n}\right)
	Y_l^m(\theta, \phi) \exp\left(-\frac{Zr}{n} \right) \]
where $\tilde{P}_{nl}$ is polynomial of degree $n-1$ in $\frac{2Zr}{n}$,
see \eqref{eqn:HydrogenRadialSolution} for details.
Characteristic for the functional form of these solutions
is both the exponential decay as $r \to \infty$ as well as
the discontinuity at the origin, \ie the position of the nucleus.
These two fundamental observations can be generalised to the setting of the
full electronic Hamiltonian $\Op{H}_{\Nelec}$ as summarised in the following remark.

\begin{rem}
	\label{rem:PhysicalProperties}
	Let $\Psi_i(\vec{x})$ be an exact eigenstate of the electronic
	Hamiltonian $\Op{H}_{\Nelec}$ with eigenenergy $E^{\Nelec}_i$.
	It holds:
	\begin{itemize}
		\item Kato's electron-nucleus cusp condition~\cite{Kato1957}:
			\[
				\left. \frac{\partial \langle\Psi(\vec{x})\rangle}
				{\partial r_i} \right|_{\vec{r}_i = \vec{R}_A}
				= -Z_A \left. \langle\Psi(\vec{x})\rangle \right|_{\vec{r}_i = \vec{R}_A}
			\]
			where $\left. \langle\Psi(\vec{x})\rangle \right|_{\vec{r}_i = \vec{R}_A}$
			denotes the average value on a hypersphere with $\vec{r}_i = \vec{R}_A$ fixed.
			Notice, that this expression can be reformulated to yield
			the more well-known result
			\[
				\left. \frac{\partial \rho(\vec{r})}{\partial \vec{r}} \right|_{\vec{r} = \vec{R}_A}
				= -2 Z_A \rho(\vec{R}_A).
			\]
		\item At large distances $r_1 \to \infty$,
			where all other electronic and nuclear coordinates stay finite,
			the first electron is essentially decoupled from the motion
			of the other particles and only sees them as a point charge
			of value
			\[ Z^\text{net} = \sum_{A=1}^M Z_A - \Nelec\]
			of the full system.
			The remaining system is effectively a $\Nelec-1$-electron system.
			This allows to approximately write
			\[ \Psi_i(\vec{x}) = \tilde{\Psi}^\text{rest}_j(\vec{x}) \tilde{\Psi}^1_i(\vec{r}_1), \]
			where $\tilde{\Psi}^\text{rest}_j$ has only parametric dependence%
			\footnote{Notice that this ansatz is somewhat related to the Born-Oppenheimer
			approximation.}
			on $\vec{r}_1$.
			Approximately it is an eigenfunction to $\Op{H}_{\Nelec-1}$.
			With this ansatz the electronic Schrödinger equation at large $r_1$
			reduces for $\tilde{\Psi}^1_i(\vec{r}_1)$ to
			\begin{equation}
				\left( -\frac12 \Delta_{\vec{r}_1} - \frac{Z^\text{net}}{r_1} - E^\text{net}_i \right)
				\tilde{\Psi}_i^1(\vec{r}_1) \simeq 0,
				\label{eqn:ExponentialDecay}
			\end{equation}
			where
			\[ Z^\text{net} = \sum_{A=1}^M Z_A - \Nelec - 1\]
			is the net charge and $E^\text{net}_i$ is the net energy eigenvalue
			roughly equal to \linebreak
			$E^{\Nelec}_i - E^{\Nelec-1}_j < 0$.
			For a neutral $\Nelec$-system $Z_\text{net} = 1$,
			such that the solution of \eqref{eqn:ExponentialDecay} is
			\[ \tilde{\Psi}_i^1(\vec{r}_1) \simeq \exp\left( - \sqrt{-2E^\text{net}_i} \, r_1 \right), \]
			\ie an energy-dependent exponential decay.
	\end{itemize}
\end{rem}
Both these results motivate the use of exponential-type atomic orbitals~(\ETO)
involving a factor $\exp\left(-\zeta \norm{\vec{r} - \vec{R}_A} \right)$
as basis functions for molecular calculations,
since such a basis will give rise to solutions,
which satisfy both conditions if the factor $\zeta$ is chosen correctly.

The first attempt to do this predates the rigorous results by Kato~\cite{Kato1951,Kato1957}
by over two decades.
In \citeyear{Slater1930} \citet{Slater1930}
obtained approximate solutions to the electronic Schrödinger equation
for many atoms of the periodic table.
He employed basis sets $\{\varphi_\mu\}_{\mu\in\Ibas} \subset H^1(\R^3,\R)$
made out of exponentially decaying functions.
Motivated from the solution to the Schrödinger equation of the Hydrogen atom,
his ansatz was to write each basis function as
\begin{equation}
	\varphi^\text{STO}_\mu(\vec{r})
	= R^\text{STO}_\mu\!\left(\norm{\vec{r} - \vec{R}_\mu}\right)\, Y_{l_\mu}^{m_\mu}(\theta, \phi),
	\label{eqn:ACproduct}
\end{equation}
\ie as a product of radial part $R^\text{STO}_\mu$ centred at the nuclear
coordinate $\vec{R}_\mu$
and real-valued%
\footnote{Keeping in mind that all $2l+1$ spherical harmonics
	$Y_l^m$ with the same value for $l$ correspond to the same eigenspace
	one can always find an alternative representation to the functional
	form given in \eqref{eqn:SphericaHarmonics},
	where all spherical harmonics are real-valued functions.
	See \cite{Avery2018} for details.
} spherical harmonic $Y_{l_\mu}^{m_\mu}$.
For the radial part he used a polynomial times exponential form
\begin{equation}
	R_\mu(r) = N_\mu r^{n_{\mu}-1} \exp(-\zeta_\mu r),
	\label{eqn:STOradial}
\end{equation}
where $\zeta_\mu$
is a constant depending on the nuclear charge as well as
the orbital in question and
$N_\mu$ is the normalisation factor
\[ N_\mu = \left( 2\zeta_\mu \right)^{n_\mu} \sqrt{\frac{2\zeta_\mu}{(2n_\mu)!}}. \]
He was able to construct rules of thumb for obtaining the exponents $\zeta_\mu$
for many elements by introducing
a concept now known as \newterm{shielding}~\cite{Slater1930}.
A more detailed discussion about shielding constants
can be found in section \vref{sec:ValuesKopt}.
In his honour {\ETO}s of the form \eqref{eqn:ACproduct}
with radial part \eqref{eqn:STOradial}
are called Slater-type orbitals~({\STO}s).

As mentioned above {\STO}s are physically rather sound
and as such in many cases only few of them are required to achieve good results
as errors are generally small and convergence fast%
~\cite{Shepard2007,Guell2008,Hoggan2009,Hoggan2011}.
Their big drawback, however,
is that evaluating the electron repulsion tensor
$\eriMu{\varphi_\mu \varphi_\nu}{\varphi_\kappa \varphi_\lambda}$
is challenging,
such that \STO-based methods
are not amongst the most commonly used quantum-chemistry methods nowadays.
Nevertheless their promising properties and fast convergence
has motivated many people to work on optimising \STO expansions
and on designing efficient evaluation schemes for the \ERI tensor%
~\cite{Weniger1983,Hoggan2009,Hoggan2011,Avery2013,Avery2017}.
As a result a number of packages
like STOP~\cite{Bouferguene1996},
SMILES~\cite{FernandezRico2001}
and ADF~\cite{Velde2001}
have become available,
which employ basis sets composed of {\STO}s.

\subsection{Contracted Gaussian-type orbitals}
\label{sec:cGTO}

In \citeyear{Boys1950} \citeauthor{Boys1950} \cite{Boys1950} suggested to replace
the exponential factor $\exp(- \zeta r)$ in the radial part \eqref{eqn:STOradial}
by a Gaussian factor $\exp(-\alpha r^2)$,
resulting in the so-called Gaussian-type orbitals~({\GTO}s).
Such \GTO basis functions still follow the ansatz radial part times real-valued spherical harmonic
\begin{equation}
	\varphi^\text{GTO}_\mu(\vec{r}) =
	R^\text{GTO}_\mu(r_\mu) \, Y_{l_\mu}^{m_\mu}(\theta_\mu, \phi_\mu)
	\label{eqn:ACproductCgto}
\end{equation}
but their radial part is now given as
\begin{equation}
	R^\text{GTO}_\mu(r) = N_\mu r^{l_\mu} \exp(-\alpha_\mu r^2)
	\label{eqn:GTOradial}
\end{equation}
with Gaussian exponent $\alpha_\mu$ and normalisation constant
\[
	N_\mu =  \sqrt{\frac{2^{l+2}}{(2l+1)!!}} \, \sqrt[4]{\frac{(2\alpha)^{2l+3}}{\pi}}
\]
This replacement allows to perform the
evaluation of the integrals involved in building the Fock matrix $\mat{F}$
much more efficiently.
Because of the \newterm{Gaussian product theorem}%
~\cite{Boys1950,Szabo1996,Besalu2011},
the product of two Gaussians may be expressed \emph{exactly} as
\[
	R^\text{GTO}_\mu\!\left( \norm{\vec{r} - \vec{R}_\mu}\right)
	\,
	R^\text{GTO}_\nu\!\left( \norm{\vec{r} - \vec{R}_\mu}\right)
	= R^\text{GTO}_\kappa\!\left( \norm{\vec{r} - \vec{R}_\kappa}\right)
\]
where $l_\kappa$, $\alpha_\kappa$ and $\vec{R}_\kappa$ are chosen appropriately.
With this result the evaluation of certain \ERI integrals \eqref{eqn:ERI}
can be done analytically~\cite{Boys1950}.
An example would be those involving four basis functions with $l_\mu = 0$.
All other \ERI integrals, potentially involving higher angular momentum $l_\mu$,
can be computed from the initial ones employing
a set of recursion formulas~\cite{Gill1994}.
Similar strategies can be found for the one-electron integrals
in order to build $\mat{T}$ and $\mat{V}_0$.
Overall the construction of $\mat{F}$ therefore becomes much more
feasible for larger basis sets of Gaussians compared to large sets of {\STO}s.

Unfortunately, certain physical aspects like the exponential decay
or the cusp are no longer directly built into the basis set
if such \GTO basis functions are used.
Since $\varphi^\text{GTO}_\mu(\vec{r}) \in C^\infty(\R^3, \R)$,
which is a dense subset of $H^1(\R^3, \R)$
this is not \textit{per se} a problem:
The denseness ensures that we can still represent every function
from $H^1(\R^3, \R)$ up to arbitrary accuracy if we use enough {\GTO}s.
In other words, the Ritz-Galerkin ansatz still allows us to
solve problems like \HF or \FCI up to arbitrary accuracy.
Since the {\GTO}s, however,
span only a smaller subset of $H^1(\R^3, \R)$ than the {\STO}s,
we will need more \GTO basis functions to achieve the same accuracy.

As a remedy \citet{Hehre1969} introduced so-called
\newterm{contracted Gaussian-type orbitals}~({\cGTO}s),
where the radial part of a basis function $\varphi_\mu$
is expressed as a fixed linear combination of $N_\text{contr}$
\newterm{primitive Gaussians}%
\footnote{Here we follow the usual convention to include the normalisation constant
	inside the contraction coefficients.}
\[ R^\text{cGTO}_\mu(r) = r^{l_\mu} \sum_i^{N_\text{contr}} c_{\mu,i} \exp(-\alpha_{\mu,i} r^2). \]
The idea is to get the best out of both worlds:
The easily solvable integrals in terms of primitive {\GTO}s
and an accurate description of the wave function by using
predetermined sets of \newterm{contraction coefficient}s
$c_{\mu,i}$ and exponents $\alpha_{\mu,i}$,
known to give a good basis set $\{\varphi^{\text{cGTO}}_\mu\}_{\mu\in\Ibas}$.
By the means of this trick
one is able to effectively split the parameter space of the variational problem
\eqref{eqn:HFOptCoeff} into two parts.
One --- the contraction coefficients ---
is fitted once and for all in order to fit a large range of problems
and another --- the coefficient matrix \eqref{eqn:HFCoeffMatrix} ---
is the search space over which one minimises during the actual calculation.

Out of the pragmatic desire to perform molecular calculations
on systems larger than what was feasible with \STO basis sets at that time,
\citet{Hehre1969} initially focused on contracting primitive Gaussians
in a way that they most closely resembled a particular \STO function.
This resulted in the famous STO-$n$G family of basis sets.
Later it was realised that more accurate basis sets could be constructed
by trying to minimise the energy,
which is resulted from an actual \HF or an {\MP}2 calculation.
Other strategies included a rigorous construction of the basis set
in order to obtain convergence in the amount of recovered correlation energy,
or to be consistent in certain computed properties.
These deviating approaches have led to a number of
different basis set families over the years,
most of which share common concepts, however.
Our discussion here should remain rather brief.
Interested readers are referred to the excellent reviews by \citet{Hill2013}
and \citet{Jensen2013}.

\defineabbr{CBS}{CBS\xspace}{Complete basis set}
All basis sets, which are considered state-of-the-art nowadays,
are so-called \newterm{split-valence basis set}\textbf{s},
which is meant to indicate that multiple contracted Gaussians are available
for describing the valence shell of an atom.
How many are used is typically referred to by the $\zeta$-level,
\eg a double-$\zeta$ basis set contains two \emph{contracted} Gaussians
for each valence orbital,
a triple-$\zeta$ basis set three and so on.
For this characters like \texttt{D}, \texttt{T}, \texttt{Q}, \texttt{5}, \ldots
--- for double, triple, quadruple, quintuple level ---
may be found in the name of the basis set.
Notice that each contracted Gaussian inside such basis sets
is typically in turn made up from multiple primitives.
For a particular basis set family the error generally decreases
going to higher zeta levels.
For some families like Dunning's correlation-consistent basis sets~\cite{Dunning1989}
empirical formulas for estimating the error at a particular zeta
level exist~\cite{Jensen2005}.
These results have been used for many years
to estimate properties at the so-called \newterm{complete basis set}~(\CBS) limit,
\ie the theoretical value obtained if an infinitely large basis set of {\cGTO}s
were employed for the calculation.
% TODO OPTIONAL See appendix \vref{apx:CbsLimit} for more details.
A recent work by \citet{Bachmayr2014}
provides some mathematical support for such formulas
by rigorously deriving error estimates in the relevant $H^1(\R^3,\R)$-norm.
One should note, however, that these results strictly speaking
only apply to a basis of uncontracted even-tempered {\GTO}s.
The authors point out, however, that a generalisation towards
{\cGTO}s should be possible.

\begin{figure}
	\centering
	\includeimage{4_solving_hf/relative_error_cgto}
	\caption[Relative error in the hydrogen \HF ground state
		for selected \cGTO bases]
	{Relative error in the hydrogen \HF ground state
		employing selected \cGTO basis sets.
		% \cite{Hehre1969,Dunning1989,Jensen2001,Wilson1996}.
		The error is plotted against
		the relative distance of electron and proton.
	}
	\label{fig:RelativeErrorCgto}
\end{figure}
A large range of \cGTO basis sets are available nowadays,
which offer a spectrum of compromises between accuracy and computational cost.
Nevertheless, some systematic issues related to the non-physical shape
of the primitive {\GTO}s cannot be fully accounted for,
even in the largest basis sets.
To illustrate this, consider figure \ref{fig:RelativeErrorCgto}.
In this plot the
relative error of the hydrogen \HF ground state $\Phi_0$
with respect to the exact
electronic ground state $\Psi_{1s}$ \eqref{eqn:FunctionGShydrogen}%
\footnote{For hydrogen \HF is equivalent to solving the full Schrödinger equation.}
\[ \frac{\Phi_0(\vec{r}) - \Psi_{1s}(\vec{r})}{\Psi_{1s}(\vec{r})} \]
is shown at various electron-proton distances.
The plots for multiple standard \cGTO basis sets are depicted,
namely the minimal basis set STO-3G~\cite{Hehre1969},
the double-$\zeta$ basis sets cc-pVDZ~\cite{Dunning1989}
and pc-1~\cite{Jensen2001},
the quadruple-$\zeta$ basis set pc-3~\cite{Jensen2001}
as well as the sextuple-$\zeta$ basis set cc-pV6Z~\cite{Wilson1996}.
In each case the error is smallest at intermediate
electron-proton distances, but increases both at the origin
as well as larger distances.
The former feature originates from the failure of Gaussians
to represent the electron-nuclear cusp.
The latter feature can be explained due to the faster fall-off
of the Gaussians, $\exp(-\alpha r^2)$,
compared to the exact solution,
which goes as $\exp(-\zeta r)$.
Larger basis sets like pc-3 or cc-pV6Z amount to recover the
correct decay behaviour as well as the cusp somewhat,
such that the error stays below $0.02 \equiv 2\%$
in the complete inner part of the plot up to distances
of about $7.5$ Bohr.
Eventually all relative errors tend towards $-\infty$ as $r \to \infty$,
however.
Even though this cannot be seen in figure \ref{fig:RelativeErrorCgto},
this includes the case of pc-3,
where the relative error has a local maximum around $r = 10$
and then follows a downhill slope as well.
Overall the plots agree with the rule of thumb
that results become more accurate at higher $\zeta$-levels:
Both the relative errors get smaller as well as the region
where the wave function is well-represented becomes larger
as we proceed from STO-3G to double-$\zeta$ and higher $\zeta$ levels.

\begin{figure}[p]
	\centering
	\includeimage{4_solving_hf/local_energy_cgto}
	\caption[Local energy of the hydrogen \HF ground state for {\cGTO} bases]{
		Local energy $E_L(r)$ of the hydrogen atom \HF ground state
		obtained using the indicated contracted Gaussian basis sets
		%~\cite{Hehre1969,Dunning1989,Jensen2001,Wilson1996}.
		$E_L(r)$ is plotted against the relative distance
		of electron and nucleus.
	}
	\label{fig:LocalEnergyCgto}
\end{figure}

\begin{figure}[p]
	\centering
	\includeimage{4_solving_hf/local_energy_cgto_zoom}
	\caption[Local energy of the hydrogen \HF ground state for {\cGTO} bases (magnified)]{
		Magnified version of figure \vref{fig:LocalEnergyCgto}
		around the origin.
	}
	\label{fig:LocalEnergyCgtoZoom}
\end{figure}

In figures \ref{fig:LocalEnergyCgto} and \ref{fig:LocalEnergyCgtoZoom}
the local energies \eqref{eqn:LocalEnergy}
of the aforementioned basis sets are depicted
--- again as a function of relative distance.
These plots not only diverge to $-\infty$ as $r \to \infty$,
but at the origin as well, see particularly figure \ref{fig:LocalEnergyCgtoZoom}.
At intermediate electron-proton distances
the local energies of all basis sets
fluctuate around the exact ground-state energy of $0.5$ Hartree,
where the amplitude of the fluctuations are lowest for cc-pV6Z and pc-3.
Recall that the local energy is related to the relative residual error
and that ideally it should be a constant.
At intermediate distances, where the fluctuations are small,
the \HF ground state thus agrees well with the exact ground state.
Unsurprisingly, the parts of figure \ref{fig:LocalEnergyCgto},
where $E_L(\vec{r})$ is almost constant,
agree roughly with the parts of figure \ref{fig:RelativeErrorCgto}
where the relative error is small.
Similarly, the wrongful decay behaviour of the \cGTO solutions
is observed in both the plot of the relative error as well as
the local energy plot.
The most notable discrepancy of both error metrics
is close to the nucleus, see figure \ref{fig:LocalEnergyCgtoZoom}.
Whilst the relative error gets smaller and smaller for the larger
pc-3 and cc-pV6Z basis sets close to the core as well,
these show rather vivid fluctuations in $E_L(\vec{r})$ as $r \to 0$.
Eventually they diverge to $-\infty$ exactly like the result
employing any other basis set.
In other words, whilst these basis sets amount to produce
a very good description of the ground state from distances around
$0.5$ Bohr up to $7.5$ Bohr,
they fail to do so close to the core in a rather misbehaving manner.
Since the relative error is small,
the issue is not that the function value
of the exact ground state is missed.
Much rather the culprit is the gradient of the approximated
\HF ground state.

This can be explained following \cite{Ma2005}.
The potential term in the local energy \eqref{eqn:LocalEnergy} diverges
as $-Z_A/r$ close to the nucleus $A$,
such that the kinetic energy term inside \eqref{eqn:LocalEnergy}
needs to provide an equal and opposite
divergence in order for the resulting local energy to be constant.
Since the gradient of every {\cGTO} basis functions is zero at the origin,
so is the gradient of the final \HF ground state,
thus the local energy goes to $-\infty$.
Furthermore, the gradient of each individual primitive Gaussian
goes to zero at a different rate
depending on its exponent $\alpha_{\mu,i}$.
Overall this leads to an overcompensation
of the diverging potential in the kinetic term at some points
and an undercompensation at others,
giving rise to the oscillatory behaviour.
This oscillatory feature close to the nucleus is well-known
in the quantum Monte-Carlo community~\cite{Foulkes2001,Ma2005},
since it can lead to problems when sampling the local energy,
especially in diffusion Monte-Carlo.

In \HF, \DFT and Post-\HF methods
the failure of the {\cGTO}s to represent the nuclear cusp
or the long-range behaviour is typically only an issue
if either parts of the wave functions are especially important
for a particular property.
Examples would be the determination of Rydberg-like excited states,
the computation electron affinities,
the computation of X-ray absorption spectra
or the computation of nuclear-magnetic resonance properties.
In such cases specific basis sets
are required~\cite{Hill2013,Jensen2013},
which include further \cGTO basis functions
to either sample the core region or the long-range tail more accurately.
If such basis sets are not employed
it may happen that the desired features are completely missed or described very inaccurately.
In this sense {\cGTO}s are not fully black-box and
picking a reasonable basis set for a particular problem
usually requires some idea of the electronic structure already.
On the other hand, if such special basis sets are employed,
one may encounter numerical instabilities.
The reason is such basis sets
tend to be amended with {\cGTO}s of either very small
or very similar exponents.
This implies that the basis functions $\varphi_\mu$ may
become almost linearly dependent,
yielding large off-diagonal
overlap matrix elements $\braket{\varphi_\mu}{\varphi_\nu}_1$
and a near-singular overlap matrix.

\begin{figure}
	\centering
	\includeimage[width=\textwidth]{4_solving_hf/fock_gaussian}
	\caption[Structure of the Fock matrix for a \cGTO-based \SCF]
		{Structure of the Fock matrix for a \cGTO-based \SCF
		of the beryllium atom
		in a pc-2~\cite{Jensen2007} basis set.
		The three figures show the matrix
		at different convergence stages during the \SCF.
		From left to right the Pulay error
		Frobenius norm is $0.18$, $0.0063$ and $4.1 \cdot 10^{-7}$.
		The colouring depends on the absolute value
		of the respective Fock matrix entry
		with white indicating entries below $10^{-8}$.
	}
	\label{fig:StructureGaussianFock}
\end{figure}
It was already mentioned that the Gaussian product theorem
allows for an efficient evaluation of the integrals
required for building the Fock matrix $\mat{F}$.
Furthermore the resulting Fock matrix is comparatively small:
Even for systems with hundreds of atoms one typically only needs
in the order of thousands of basis functions.
In other words, both building the Fock matrix
as well as diagonalising it can be performed using direct methods%
\footnote{%
	A standard procedure would be to reduce the matrix to
	tridiagonal form using Householder reflections
	and then use Cuppen's divide and conquer~\cite{Arbenz2010}
	or multiple relatively robust representations~\cite{Dhillon2006}.
}.
Ignoring the basis sets suffering from over-completeness for a second,
the numerical structure of a \cGTO-based Fock matrix
is rather advantageous in most cases.
Figure \vref{fig:StructureGaussianFock}, for example,
shows some Fock matrices from an \SCF calculation
of a beryllium atom
in the pc-2~\cite{Jensen2007} basis set.
The matrices are taken as snapshots during the \SCF procedure.
From left to right the Pulay error Frobenius norm decreases
from $0.18$ to $0.0063$ and finally $4.1\E{-7}$.
As the error gets smaller the matrix becomes more and more diagonal
as the off-diagonal elements in the occupied-virtual block of the
Fock matrix all have to vanish%
\footnote{This is another way equivalent to \eqref{eqn:PulayError}
to express \SCF convergence.}.
Already the leftmost matrix is almost diagonal-dominant with 12
out of 15 rows $\mu$
satisfying the condition for \newterm{diagonal-dominance}
\[
	\sum_{\nu=1}^{\Nbas} F_{\mu\nu} < 2 F_{\mu\mu}.
\]
For larger systems, the structure generally gets
worse due to interactions between the atoms,
but if a proper description of the core region or the tail is not required
$\mat{F}$ stays numerically manageable and almost diagonal-dominant.
This allows to additionally employ
iterative eigensolver methods like Davidson's method~(see section \vref{sec:Davidson})
to efficiently obtain eigenpairs of the Fock matrix if not all are required.

Since for most cases in chemistry the core-like region
and the long-range tail are not extremely important
both obtaining and diagonalising Fock matrices
from a \cGTO discretisation is straightforward.
Even though {\cGTO}s are physically not the most sensible basis function type
this has historically made \cGTO-based methods
the most predominant approach to describe
a chemical systems within decent accuracy
such that these methods are now implemented in countless quantum-chemistry packages.
In light of this it is remarkable,
that only in \citeyear{Bachmayr2014}
error bounds were rigorously derived by \citet{Bachmayr2014} for some special
kinds of Gaussian basis sets
and these results are not employed on a daily basis.

I personally consider it a problem
that selecting the \cGTO basis set for performing a calculation is often done
based on habit rather than proper scientific evaluation
in light of the chemical problem at hand.
I do not dare to estimate how many published works
draw conclusions based on calculations,
where a more appropriately chosen basis
could have given a totally different picture.

\subsection{Finite-element based discretisation}
\label{sec:FE}
\newcommand{\Nquadc}{\ensuremath N_\text{quadc}}
\newcommand{\Ncell}{\ensuremath N_\text{cell}}

The Slater-type orbitals and Gaussian-type orbitals
we introduced in the previous sections
are examples for so-called atom-centered basis functions
or \newterm{atom-centered orbital}s~({\ACO}s).
A different ansatz in many respects
are grid-based methods,
where the underlying idea is to partition three-dimensional real space
into smaller parts using a structured grid
solve the problem by interpolation and numerical integration.
The example we want to consider in this work are \newterm{finite elements},
which are specially constructed piecewise polynomials
often employed in structural mechanics or engineering
for solving partial differential equations~\cite{Johnson1987}.
Multiple approaches for
solving the \HF problem or the Kohn-Sham equations approximately
using finite-element based discretisations
have been performed over the years as well%
~\cite{Tsuchida1995,Soler2002,Lehtovaara2009,Alizadegan2010,Avery2011PhD,Davydov2015,Lee2015,Boffi2016},
This section will only give a short overview of the finite-element method
in the light of the \HF problem.
For more details the reader is referred
to the rich literature~\cite{Johnson1987,Grossmann1992,Bangerth2003,Brenner2008}.

\subsubsection{Construction of a \FE grid}
Compared to the atom-centered basis functions,
where a discretisation based on the complete domain $\R^3$ is possible,
any grid-based method can only achieve this
on a subset $\Omega \subset \R^3$,
which is taken to be open.
At the boundary $\partial\Omega$ one needs
to impose a boundary condition for the solution to be unique.
There are a couple of options, which will be contrasted later.
For now let us assume that $\Omega$ is large enough,
such that the \SCF orbitals are essentially zero at the boundary $\partial\Omega$
and we can impose a homogeneous Dirichlet boundary%
\footnote{This implies that the \HF eigenfunctions are forced
	to be exactly zero at the boundary $\partial\Omega$.}.
Using this approximation
as well as the inner product
\[ \braket{\psi}{\chi}_1 \equiv \int_\Omega \psi(\vec{r}) \chi(\vec{r})  \D \vec{r} \]
the spin-free, real-valued \HF equations \eqref{eqn:HFequations} can be adapted to
%
\begin{equation}
\label{eqn:HFequationsFE}
\begin{aligned}
	\Op{F}_{\Theta^0} \psi_i^0(\vec{r}) &= \varepsilon_i \psi_i^0(\vec{r}) && \vec{r} \in \Omega \\
	\psi_i^0(\vec{r}) &=0 &&\vec{r} \in \partial \Omega \\
	\text{where}\qquad \braket{\psi_i^0}{\psi_j^0}_1 &= \delta_{ij},
\end{aligned}
\end{equation}
for $\Theta^0 = (\psi_1^0, \psi_2^0, \ldots, \psi_{\Nelec}^0) \in \left( H^2(\Omega,\R) \right)^{\Nelec}$
being the minimiser to the \HF problem \eqref{eqn:HFMO}.
The corresponding sesquilinear form
\[ a_{\Theta^0}(\psi, \chi)
	\equiv \int_\Omega \psi(\vec{r}) \Op{F}_{\Theta^0} \chi(\vec{r}) \D \vec{r} \]
is defined in analogy to \eqref{eqn:SesquilinearFormFock}.
By partial integration it can be seen that this form
is defined on the domain $Q(\Theta^0) = H^1(\Omega, \R)$.
In the \newterm{finite-element method}
the aim is to solve \eqref{eqn:HFequationsFE} variationally
in the sense of remark \vref{rem:DiscreteFormulation}
employing a hierarchy of approximation spaces $S_n$
consisting of piecewise polynomials.
Such an attempt is only sensible
is these spaces are more and more accurate
approximations of the form domain $H^1(\Omega, \R)$
in the sense of \eqref{eqn:CondSubspaces}.

To outline the construction of such spaces,
let us consider at first a (fictitious) one-dimensional case
where $\Omega = (a,b)$ with $a,b \in \R$.
This domain can be subdivided into $\Ncell$ parts
\[ a = x_0 < x_1 < x_2 < \cdots < x_{\Ncell} = b, \]
which do not need to be of equal size.
The open intervals $c_j = (x_j, x_{j+1})$ for $j=0, 1, \ldots, \Ncell-1$
are called grid \newterm{cells}
and the set $\mathcal{M}_h = \{c_j \, | \, j=0, 1, \ldots, \Ncell-1 \}$
of all grid cells is called a \newterm{mesh} or a \newterm{triangulation}.
In this set the index $h$ stands for the maximal size of a grid cell
defined as
\[ h \equiv \max_{c \in \mathcal{M}_h} \abs{\max(c) - \min(c)} \]
in one dimension. Using the vector space
\[
	\set{P}_k^1 \equiv \left\{ u \in C^\infty(\R) \,\middle|\, u(x) = \sum_{i=0}^k c_i x^i, c_i \in \R \right\}
\]
of all real polynomials of order at most $k$ in one dimension,
we can define
\begin{equation}
	P_k(\mathcal{M}_h)
	\equiv \left\{ u \in C^0(\overline{\Omega}) \ \middle| \
	\forall c \in \mathcal{M}_h: \
	u|_{\bar{c}} \in \set{P}_k^1 \right\},
	\label{eqn:PiecewisePolynomialsOne}
\end{equation}
the set of piecewise polynomials of at most degree $k$.
The elements of $P_k(\mathcal{M}_h)$
are on the complete domain $\Omega$ at least continuous
and inside the grid cells
they reduce to being polynomials, thus smooth.
It can be shown~\cite[Lemma 4.1]{Grossmann1992} that this implies
$P_k(\mathcal{M}_h) \subset H^1(\Omega, \R)$.
As $h \to 0$ such approximations become more exact,
which make $P_k(\mathcal{M}_h)$ the desired approximation
spaces of $H^1(\Omega, \R)$ for a one-dimensional problem.

\begin{figure}
	\centering
	\missingfigure{1D grid with linear}
	\caption{One D case for linear FEs}
	\label{fig:FEoneDimLin}
\end{figure}
\begin{figure}
	\centering
	\missingfigure{1D grid with quadratic}
	\caption{One D case for quadratic FEs}
	\label{fig:FEoneDimQuad}
\end{figure}
For representing $P_k(\mathcal{M}_h)$
\newcommand{\Ibash}{\mathcal{I}_{\text{bas},h}}
one typically chooses a \newterm{Lagrange basis} $\{\varphi_\mu\}_{\mu\in \Ibash}$
consisting of basis functions $\varphi_\mu$ with $0 \leq \mu \leq k \Ncell$
defined as
\begin{align*}
	\varphi_\mu &\in P_k(\mathcal{M}_h), &
	\varphi_\mu(\tilde{x}_\nu) &= \delta_{\mu\nu}
\end{align*}
where%
\footnote{In equation \eqref{eqn:FEnodalPointsOneD} $\nu/k$ denotes integer division
without remainder.}
\begin{equation}
	\tilde{x}_\nu = x_{\nu/k} + \frac{\nu \!\! \mod k}{k} \left( x_{(\nu/k) +1} - x_{\nu/k} \right)
	\label{eqn:FEnodalPointsOneD}
\end{equation}
for $0 \leq \nu \leq k \Ncell$ are the \newterm{nodal points}.
An alternative term to refer to such
basis functions $\varphi_\mu$
is \textbf{finite element of order $k$},
the order $k$ being a reference to the maximal polynomial degree inside the cells $c$.
Examples for linear and quadratic finite elements are
illustrated in figures \ref{fig:FEoneDimLin} and \ref{fig:FEoneDimQuad} respectively.
In each case the finite element functions
are either $1$ or $0$ at the nodal points and only
have support on a few cells, which are furthermore direct neighbours.

In the following we want to generalise this construction
for a three-dimensional domain $\Omega$.
In the most general form a mesh can be defined as
\begin{defn}[Mesh]
	Let $\Omega$ be a domain in $\R^3$.
	A mesh is a finite set $\mathcal{M}_h = {c_0, c_1, \ldots, c_{\Ncell-1}}$
	of $\Ncell$ bounded domains $c_i$ with sufficiently regular boundary%
	\footnote{More precisely the boundary has to be Lipschitz.},
	such that
	\begin{align*}
		\overline{\Omega} &= \bigcup_{i=0}^{\Ncell-1} \overline{c_i}
		&c_i \cap c_j = \emptyset \quad \forall i \neq j,
	\end{align*}
	\ie such that these domains fully partition $\Omega$.
	Furthermore we set for each $c \in \mathcal{M}_h$
	the \newterm{cell diameter}
	\[ h(c) = \max_{\vec{x},\vec{y} \in bar{c}} \norm{\vec{x}-\vec{y}}_2 \]
	and call
	\[ h = \max_{c \in \mathcal{M}_h} h(c) \]
	the \newterm{mesh size}.
\end{defn}
In many cases one does, however, only consider
so-called \newterm{affine} meshes.
\begin{defn}
	A mesh is called affine if a reference cell $c_0$
	and affine transformations%
	\footnote{A transformation $\tau$ is affine iff $\tau(\vec{x}) = \mat{A}\vec{x} + \vec{b}$ with $\mat{A}$ being a transformation matrix and $b$ being a constant shift vector.}
	$\tau_{c_i}$ exist for each cell $c_i \in \mathcal{M}_h$,
	such that $\overline{c_i} = \tau_{c_i}(c_0)$.
\end{defn}
In other words a mesh is affine exactly if each
grid cell can be generated from the reference cell $c_0$
by a linear transformation followed by a shift to the correct position.
This work only considers \textbf{cuboidal meshes},
where the reference cell $c_0 = [0,1]^3$ is the unit cube.
Similar to the construction of the grid cells,
the finite elements of order $k$
on each cuboidal grid cell $c \in \mathcal{M}_h$
can be constructed by applying the affine transformation $\tau_c$
to a set of template polynomials of order $k$
defined on the reference cell $c_0$.
These template polynomials are called
\newterm{shape function}s.
%
\begin{figure}
	\centering
	\missingfigure{shape functions}
	\caption{Shape functions}
	\label{fig:ShapeFunctions}
\end{figure}
%
A few examples in one and two dimensions are illustrated
in figure \ref{fig:ShapeFunctions},
where the shape functions in two dimensions
have been constructed as tensor products from the one-dimensional ones.
This construction is a special property of so-called $Q_k$ finite elements,
which are typically used in cuboidal meshes.
This tensor product ansatz allows to construct $Q_k$ elements
in three and higher dimensions as well.

Let us denote with $S_h$ the
space spanned by all the $Q_k$ finite elements $\{\varphi_\mu\}_{\mu\in\Ibash}$
on a cuboidal mesh $\mathcal{M}_h$,
constructed by the means of appropriate affine maps and shape functions.
Even though we always have $S_h \subset H^1(\Omega, \R)$,
condition \eqref{eqn:CondSubspaces}
is not always satisfied as the mesh size $h \to 0$.
In other words to ensure convergence of the Ritz-Galerkin procedure in three dimensions
vanishing mesh size is not sufficient.
On top of that the mesh needs to stay
what is called \textbf{uniform} and \textbf{shape-regular}.
Roughly speaking these conditions ensure that
all grid cells have roughly the same size
and their shape is closer to being a ball than to being a needle.

If those conditions are taken into account,
an initial mesh can be refined more and more
until the \HF problem \eqref{eqn:HFequationsFE}
is solved up to the desired accuracy.
Furthermore \textit{a priori} and \textit{a posteriori} error estimates
can be derived for regular meshes and
those grid cells, which contribute most
to the estimated error, can be identified.
Usually the error in a grid cell $c$
is related to its diameter $h(c)$,
which suggest to refine a mesh adaptively at those cells,
where the error is largest.
Since \FE grids do not need to be equally spaced,
this suggests a scheme where starting from a crude initial grid
the problem \eqref{eqn:HFequationsFE} is solved,
whilst adaptively refining the grid only at some places
until the desired accuracy is achieved.
In this manner the density of grid points
is lower where the electron density does not change a lot
and is higher,
where more grid cells are needed to represent the problem properly.
Notice, how this can be entirely automatised.
Compared to a \cGTO-based discretisation
the finite-element method is therefore truly back box.

There are a couple of caveats to the finite-element method,
which should not go unmentioned.
First of all the tensor product construction of the $Q_k$ finite elements
in two and three dimensions implies
that the three dimensional \FE basis $\{\varphi_\mu\}_{\mu\in\Ibash}$
has extremely local support as well.
The difference is that instead of 2 neighbours like in 1D,
most cells now have $2^3 = 8$ neighbours instead.
Consequently a $Q_k$ finite-element in 3D can have support on at most 8 cells,
\ie stays highly localised.
This is an advantage for evaluating the integrals
involved in the sesquilinear form $a_\Theta(\slot,\slot)$
as such become embarrassingly parallel for local operators.
See the discussion below for details.
It implies further, however, that many finite elements
are typically required for a proper description of the \HF orbitals
on the full domain $\Omega$,
typically in the order of millions.
Thus all algorithmic approaches for solving the numerical problems
arising from a \FE discretisation need to
scale linearly or sub-linearly in the number of finite-element basis functions.

Secondly the electron-nuclear cusp tends to be an issue.
Even the simplest cell-wise error estimators employed in the finite-element method,
like the Kelly error estimator~\cite{KellyError},
contain a term involving the gradient of the approximate solution.
In other words a larger amount of grid points
and thus a larger amount of finite-element basis functions
is required around regions where the gradient is large.
Both the wave function as well as the \HF orbitals
have large gradients around the electron-nuclear cusp~\cite{Kato1957}.
Furthermore the only discontinuity occurs at the nuclear position~\cite{Kato1951}.
Even though the region around the core
is not very interesting from a chemical point of view,
it thus consumes a large number of \FE basis functions
for proper representation.
In the light of the previous paragraph this is not at all ideal.
As a remedy most \FE-based approaches to quantum chemistry employ pseudo-potentials
to represent the core region~\cite{Davydov2015},
leaving only the regions of smaller gradients
to be represented by finite elements.
Overall this significantly reduces the number of finite elements required.
For simplicity we will not consider pseudo-potentials
in the remaining discussion about finite-elements,
but our expressions can be easily modified to incorporate such.

\subsubsection{Evaluating the discretised Fock matrix}
Let us now consider a particular cuboidal mesh $\mathcal{M}_h$
at some stage during
the process of solving \eqref{eqn:HFequationsFE} up to desired accuracy.
On $\mathcal{M}_h$ we can construct
a set of $\Nbas$ $Q_k$ finite elements $\{\varphi_\mu\}_{\mu\in\Ibash}$
following the procedure outlined above
and use these to discretise \eqref{eqn:HFequationsFE}
in a completely analogous procedure to section \vref{sec:DiscreteHF}.
This again results in a non-linear eigenproblem
\begin{equation}
	\begin{aligned}
		\matFnfull \mat{C}_F^{(n+1)} &= \mat{S} \mat{C}_F^{(n+1)} \mat{E}^{(n+1)} \\
		\tp{\mat{C}} \mat{S} \mat{C} &= \mat{I}_{\Nelec},
	\end{aligned}
	\label{eqn:SCFproblemFE}
\end{equation}
where
\begin{align*}
	\matFnfull &= \mat{T} + \mat{V}_0 + \matJnfull + \matKnfull, \\
	\mat{E}^{(n+1)}
	&= \text{diag}\left(\varepsilon_1^{(n+1)},
	\varepsilon_2^{(n+1)}, \ldots,
	\varepsilon_{\Norb}^{(n+1)}\right) \in \R^{\Norb \times \Norb}
\end{align*}
and the occupied coefficients $\mat{C}^{(n+1)} \in \R^{\Nbas \times \Nelec}$
are the first $\Nelec$ columns of the full coefficient matrix
$\mat{C}_F^{(n+1)} \in \R^{\Nbas \times \Norb}$
according to Aufbau principle.
The individual terms of the Fock matrix and the overlap matrix
are given by expressions \eqref{eqn:Tbas} to \eqref{eqn:Sbas}
just with the replacement of the integration over $\R^3$
by an integration over $\Omega$.
Naturally problem \eqref{eqn:SCFproblemFE} can be solved
by a self-consistent field procedure of remark \vref{rem:SCFcoeff}.

For evaluating the Fock matrix $\matFnfull$,
let us first consider the easy cases,
namely the terms $\mat{T}$ and $\mat{V}_0$ as well as the overlap matrix $\mat{S}$.
This amounts to evaluating integrals
\begin{align*}
O_{\mu\nu} &=  \int_\Omega \varphi_\mu(\vec{r}) \, \Op{O} \, \varphi_\nu(\vec{r}) \D\vec{r}
&&\text{where} \quad \Op{O} = \Op{T}, \Op{V}_0 \text{ or } \id_{H^1(\Omega,\R)}.
\intertext{
With reference to the grid $\mathcal{M}_h$ we can write this as a
sum of cell contributions $O^c_{\mu\nu}$
}
O_{\mu\nu} &= \sum_{c\in\mathcal{M}_h} O^c_{\mu\nu}
&&\text{where} \quad O^c_{\mu\nu} = \int_c \varphi_\mu(\vec{r})\, \Op{O}\, \varphi_\nu(\vec{r}) \D\vec{r}.
\end{align*}
All of the operators $\Op{T}$, $\Op{V}_0$ or $\id$
are so-called \newterm{local operators},
which implies
\[ \forall \nu\in\Ibash: \quad \text{Supp}\left(\Op{O}\, \varphi_\nu\right) \subseteq \text{Supp}\left(\varphi_\nu\right), \]
where
\[ \text{Supp}(\chi) \equiv \left\{ \vec{r} \in \Omega \,\middle|\, \chi(\vec{r}) \neq 0 \right\} \]
denotes the \newterm{support} of a function $\chi$.
In other words $\left(\Op{O} \varphi_\nu\right)(\vec{r})$
is non-zero only if $\varphi_\nu(\vec{r})$ is non-zero,
which implies
\[ c \not\in \text{Supp}(\varphi_\mu) \cap \text{Supp}(\varphi_\nu)
	\quad \Rightarrow \quad
	O^c_{\mu\nu} = 0.
\]
Conversely to build the matrix $\mat{O}$
we only need to consider those elements $O_{\mu\nu}$
where $\text{Supp}(\varphi_\mu) \cap \text{Supp}(\varphi_\nu) \neq \emptyset$.
A particular $\varphi_\mu$ only has support in 8 cells as indicated above.
In each cell at most $(k+1)^3$ finite elements have support,
such that for a particular $\mu\in\Ibash$,
$O_{\mu\nu}$ can only be non-zero
for at most $8 (k+1)^3$ values of $\nu\in\Ibash$.
Using a clever ordering of the finite-element functions
one can determine the set of finite-elements $\varphi_\nu$,
which couple with a given element $\varphi_\mu$ immediately~\cite{CuthillMcKee},
such that $\mat{O}$ can be evaluated by only considering
a number of pairs $(\mu,\nu) \in \left(\Ibash\right)^2$,
which scales linearly with the number of finite elements $\Nbas$.
Furthermore the cell contributions $O^c_{\mu\nu}$ to $\mat{O}$
are independent from another,
such that $\mat{O}$ can be determined by an embarrassingly parallel MapReduce step.

By construction for each finite-element
$\varphi_\mu$ one can find a shape function $e_i$
such that on a particular cell $c \in \mathcal{M}_h$
\[
	\varphi_\mu\big|_c(\vec{r}) = e_i\left( \tau_c^{-1}(\vec{r}) \right).
\]
Let similarly $e_j$ be the shape function corresponding to $\varphi_\nu$
and further let $J_c(\vec{\xi})$ denote the Jacobian matrix
of the mapping $\vec{r} = \tau_c(\vec{\xi})$,
defined as
\[
	\forall \alpha,\beta \in \{x,y,z\}:
	\left( J_c (\vec{\xi}) \right)_{\alpha\beta} = \frac{\partial \left( \tau_c(\vec{\xi})  \right)_\alpha}{\partial \xi_\beta}.
\]
Then we can evaluate $O^c_{\mu\nu}$ as
\begin{align*}
	O^c_{\mu\nu} &= \int_c \varphi_\mu(\vec{r})\, \Op{O} \, \varphi_\nu(\vec{r}) \D\vec{r} \\
	&= \int_c e_i\left( \tau_c^{-1}(\vec{r}) \right) \, \Op{O} \,
		e_j\left( \tau_c^{-1}(\vec{r}) \right) \\
		&= \int_{c_0} e_i(\vec{\xi}) \, \Op{O} \, e_j(\vec{\xi}) \det\left( J_c(\vec{\xi}) \right) \D \vec{\xi} \\
	&= \sum_{q=1}^{\Nquadc} e_i(\vec{\xi}_q) \, \Op{O} \, e_j(\vec{\xi}_q) \det\left( J_c(\vec{\xi}_q) \right) w_q,
\end{align*}
where in the last step we introduced a quadrature for the integration
using $\Nquadc$ quadrature points
$\vec{\xi}_1, \vec{\xi}_2, \ldots, \vec{\xi}_{\Nquadc} \in c_0$
with quadrature weights $w_1, w_2, \ldots, w_{\Nquadc}$.
If we assume $\Nquadc$ is large enough,
this quadrature can be made exact,
since $e_i$ and $e_j$ are only polynomials of order $k$.
Notice that both the quadrature as well as the Jacobian
only need to be computed on the reference cell $c_0$
and can be re-used for all cells by the means of the affine transformation $\tau_c$.
Together with the guaranteed linear scaling
in the number of matrix elements $O_{\mu\nu}$
which need to be computed
as well as the embarrassingly parallel procedure
this makes the computation of the matrices $\mat{T}$,
$\mat{V}_0$ and $\mat{S}$ extremely efficient despite
the large number of basis functions $\Nbas$ for a
finite-element based discretisation.
Since we know already \emph{before} any computation
the pairs of finite elements $(\varphi_\mu, \varphi_\nu)$
which do not have common support,
we can already set-up sensible storage schemes
for these matrices,
which do not store these zeros explicitly.
This leads to linear scaling in storage with respect to the number
of finite elements as well.

For the evaluation of the Coulomb term $\matJ$
and the exchange term $\matK$ this naive approach does not work,
unfortunately, since neither $\Op{J}$ nor $\Op{K}$ are local operators.
Let us first treat the Coulomb term.
With reference to \eqref{eqn:OperatorCoulomb}
we can write
for all $\mu, \nu \in \Ibash$
\begin{equation}
	\label{eqn:CoulombFE}
	J_{\mu\nu}\!\left[\mat{C}^{(n)}
		\left(\mat{C}^{(n)}\right)^\dagger\right]
	= \int_\Omega \varphi_\mu(\vec{r}_1)
	\left(\int_\Omega \frac{\rho^{(n)}(\vec{r}_2)}
		{ r_{12} }\D \vec{r}_2 \right)
	\varphi_\nu(\vec{r}_1) \D \vec{r}_1,
\end{equation}
where we introduced the discretised electron density
\begin{equation}
	\rho^{(n)}(\vec{r}) \equiv \sum_{i\in\Iocc} \abs{\sum_{\mu\in\Ibash} C^{(n)}_{\mu i} \varphi_\mu(\vec{r})}^2.
	\label{eqn:ElectronDensityFE}
\end{equation}
Following classical electrostatics~\cite{Jackson1999}
such an electron density gives rise to a potential $V^{(n)}_H(\vec{r})$,
defined by a Poission equation
\begin{equation}
\label{eqn:PoissionFE}
\begin{aligned}
	-\Delta V^{(n)}_H(\vec{r}) &= 4\pi \rho^{(n)}(\vec{r}) &&\vec{r} \in \Omega \\
	V^{(n)}_H(\vec{r}) &= 0 && \vec{r} \in \partial\Omega.
\end{aligned}
\end{equation}
In this case $V^{(n)}_H(\vec{r})$ is called the \newterm{Hartree potential} as well.
Assuming \eqref{eqn:PoissionFE} can be solved,
this allows to rewrite \eqref{eqn:CoulombFE} to yield
\[
	J_{\mu\nu}\!\left[\mat{C}^{(n)}\left(\mat{C}^{(n)}\right)^\dagger\right]
	= \int_\Omega \varphi_\mu(\vec{r}_1) V^{(n)}_H(\vec{r}_1) \varphi_\nu(\vec{r}_1) \D \vec{r}_1.
\]
Since the Hatree potential $V^{(n)}_H$ is a local operator,
this latter integral can be evaluated in $\bigO(\Nbas)$
time and space using the cell-wise numerical integration scheme discussed above.

Solving the Poission equation \eqref{eqn:PoissionFE}
is a well-understood problem in numerical mathematics.
Using a combination of multigrid preconditioning~\cite{Hackbusch1985}
and a conjugate-gradient linear solver~\cite{Grossmann1992},
this problem can be solved in $\bigO(\Nbas)$%
\footnote{Try this yourself using the
	example program \url{http://dealii.org/developer/doxygen/deal.II/step_16.html}
	distributed alongside
	the \texttt{deal.ii} finite-element library~\cite{Arndt2017,Bangerth2007}.
}.

A question worth addressing in this context is which boundary condition to choose.
In \eqref{eqn:SCFproblemFE} as well as \eqref{eqn:PoissionFE}
we used a  homogeneous Dirichlet boundary in both cases.
Whilst this is still somewhat sensible for the \SCF orbitals,
this can lead to issues for the Hartree potential, which only falls off as $-1/r$.
So even at distances of $10^6$ Bohr from the nucleus the potential
will still be around $10^{-6}$.
The situation can be improved by approximating the density $\rho^{(n)}(\vec{r})$
at large distances by a point charge in the sense of a multipole expansion.
The solution to the Poisson equation in this case is trivial,
yielding the Coulomb potential
\[ V_P(\vec{r}) = \frac{\Nelec -1}{r}. \]
For the complete \SCF problem \eqref{eqn:SCFproblemFE}
one could similarly employ a multipole approximation
to yield an approximate solution at the boundary,
related to what we already did in remark
\vref{rem:PhysicalProperties}.
In both cases such approximate solutions can be enforced
using appropriate boundary conditions.
The options are
\begin{itemize}
	\item Dirichlet boundary conditions:
		\begin{align}
			V_H^{(n)}(\vec{r}) &= V_P(\vec{r}) \qquad \vec{r} \in \partial\Omega
			\label{eqn:PoissonHartreeBCDirichlet}
		\end{align}
	\item Neumann boundary conditions:
		\begin{align}
			\partial_n V_H^{(n)}(\vec{r}) &= \partial_n V_P(\vec{r}) \qquad \vec{r} \in \partial\Omega,
			\label{eqn:PoissonHartreeBCNeumann}
		\end{align}
		where $\partial_n V_H^{(n)}$ denotes the normal derivative at the boundary $\partial\Omega$.
	\item Robin boundary conditions:
		\begin{align}
			\label{eqn:PoissonHartreeBCRobin}
			\alpha(\vec{r}) \tilde{V}_H^{(n)}(\vec{r})
				&= \partial_n \tilde{V}_H(\vec{r}) \qquad \vec{r} \in \partial\Omega \\
		\intertext{where $\alpha(\vec{r})$ is determined from}
			\nonumber
			\alpha(\vec{r}) V_P(\vec{r}) &= \partial_n  V_P(\vec{r}) 
		\end{align}
\end{itemize}
For the Poisson equation Robin boundary conditions usually
\eqref{eqn:PoissonHartreeBCRobin} work best in practice,
since they somewhat enforce resemblance of the gradient and the value
of $V_P(\vec{r}$ at the same time.

In theory there is no reason why
one should use the same discretisation
for solving the Poission equation \eqref{eqn:PoissionFE}
and for solving the \HF equations \eqref{eqn:HFequationsFE}.
Using different meshes is possible,
but leads to complications when projecting the Hartree potential $V_H^{(n)}$
onto the grid used for solving the \HF equations.
The use of different polynomial orders for example
has been investigated by \citet{Davydov2015}
in the context of the related Kohn-Sham equations.
Their results suggest to use twice the polynomial order for solving the Poission equation
compared to the polynomials used for the \HF problem.
This can be rationalised by looking at the expression \eqref{eqn:ElectronDensityFE}
for the discretised density.
If $\varphi_\mu$ and $\varphi_\nu$ denote two $Q_k$ finite elements,
which are used for the discretisation of \eqref{eqn:HFequationsFE},
solving the Poisson equation \eqref{eqn:PoissionFE} requires to represent
the density $\rho^{(n)}(\vec{r})$,
which consists of products $\varphi_\mu, \varphi_\nu$.
These can only be represented exactly if at least $Q_{2k}$ elements
are used to discretise \eqref{eqn:PoissionFE}.

\subsubsection{Evaluating the \HF exchange contribution}
Similar to the Coulomb term we
use \eqref{eqn:OperatorExchange} and \eqref{eqn:Kbas} to
write the exchange matrix elements for all $\mu, \nu \in \Ibash$ as
\begin{equation}
	\label{eqn:ExchangeFE}
	K^{(n)}_{\mu\nu} \equiv
	K_{\mu\nu}\!\left[\mat{C}^{(n)} \left(\mat{C}^{(n)}\right)^\dagger\right]
	%
	= \int_\Omega \int_\Omega
		\varphi_\mu(\vec{r}_1) \frac{\gamma^{(n)}(\vec{r}_1, \vec{r}_2)}
		{ r_{12} }\varphi_\nu(\vec{r}_2) \D \vec{r}_2 \D \vec{r}_1,
\end{equation}
where we introduced the discretised one-particle reduced density matrix
\begin{equation}
	\gamma^{(n)}(\vec{r}_1, \vec{r}_2)
	\equiv \sum_{i\in\Iocc} \, \sum_{\mu,\nu\in\Ibash}
	C^{(n)}_{\mu i} \varphi_\mu(\vec{r}_1) \, C^{(n)}_{\nu i} \varphi_\nu(\vec{r}_2).
	\label{eqn:DensityMatrixFE}
\end{equation}
The double integral \eqref{eqn:ExchangeFE}
can be split into a sum of contributions from each grid cell pair
$(c,d) \in \left( \mathcal{M}_h \right)^2$
\begin{align*}
	K^{(n)}_{\mu\nu}
	&= \sum_{c,d \in \mathcal{M}_h}
		\int_{c} \int_{d}
		\frac{\varphi_\mu(\vec{r}_1) \, \gamma^{(n)}(\vec{r}_1, \vec{r}_2)
			\,\varphi_\nu(\vec{r}_2)}
		{r_{12}} \D \vec{r}_2 \D \vec{r}_1 \\
	&= \sum_{c,d \in \mathcal{M}_h}
		\int_{c_0} \int_{c_0}
		\frac{e_i(\vec{\xi}_1)\,
			\gamma^{(n)}\!\left(\tau_c(\vec{\xi}_1), \tau_d(\vec{\xi}_2)\right)
			\,e_j(\vec{\xi}_2)}
		{\norm{\tau_c(\vec{\xi}_1) - \tau_d(\vec{\xi}_2)}_2}
		\D \vec{\xi}_1 \D \vec{\xi}_2,
\end{align*}
where $e_i$ is the shape function corresponding to $\varphi_\mu$
and $e_j$ the one corresponding to $\varphi_\nu$.
Introducing two quadratures $\mathcal{Q}$ and $\mathcal{Q}'$
with quadrature points $\vec{\xi}_q$, $\vec{\xi}_r'$
and corresponding weights $w_q$ and $w_r'$ yields
\begin{align*}
	K^{(n)}_{\mu\nu}
		&\simeq \sum_{c,d \in \mathcal{M}_h}
		\sum_{q=1}^{\Nquadc}
		\sum_{r=1}^{\Nquadc'}
		\frac{
			e_i(\vec{\xi}_q) \,
			\gamma^{(n)}\!\left(\tau_c(\vec{\xi}_q), \tau_d(\vec{\xi}'_r)\right)
			\,e_j(\vec{\xi}'_r)}
		{\norm{\tau_c(\vec{\xi}_q) - \tau_d(\vec{\xi}'_r)}_2}
		w_q w_r'
\end{align*}

\todoil{Rephrase from here}
In this expression the non-local nature of \HF exchange can be clearly observed:
Unlike the previous Fock matrix terms
no immediate criterion for excluding some
pairs of finite element indices $(\mu, \nu)$ can be found
from the derived expression.
It is, however, well-known that the density matrix $\gamma^{(n)}(\vec{r}_1, \vec{r}_2)$
decays exponentially with distance
originating from the exponential decay behaviour of the wave function.
Furthermore $1/r_{12}$ decays with increasing distance of the cells from another,
such that overall some distance cut-off can most likely be found
in order to exclude at least some pairs of cells in the sum.

For evaluating the computational complexity,
let us first consider the evaluation of
$\gamma^{(n)}\!\left(\tau_c(\vec{\xi}_q), \tau_d(\vec{\xi}'_r)\right)$
on one cell pair $(c,d)$ and for one pair
of quadrature points $(\vec{\xi}_q, \vec{\xi}'_r)$.
Let $N_\text{shape}$ denote the number of shape functions per grid cell.
Due to the local support of the finite element functions
we discussed before,
there will be at most $\Nelec N_\text{shape}^2$
terms in \eqref{eqn:DensityMatrixFE},
which are non-zero.
In other words evaluating
$\gamma^{(n)}\!\left(\tau_c(\vec{\xi}_q), \tau_d(\vec{\xi}'_r)\right)$
takes $\bigO(\Nelec N_\text{shape}^2)$.
Overall this makes the evaluation of $\matK$ scale as
\[ \bigO(\Ncell^2 \Nquadc \Nquadc' \Nelec N_\text{shape}^4) = \bigO(\Nbas^2 \Nquadc \Nquadc' \Nelec N_\text{shape}^2), \]
which is quadratic in the number of finite elements.
Since nothing about the sparsity structure is known the memory
requirements for storing $\mat{K}$ are $\bigO(\Nbas^2)$ as well,
which is highly undesirable.

Luckily in many cases storing this matrix is not required.
The reason for this is that iterative diagonalisation methods
like the Arnoldi procedure~(see section \ref{sec:Arnoldi})
or Davidson's method~(see section \ref{sec:Davidson})
only need to be able to apply the matrix to be diagonalised to
a test vector $\vec{x}$.
In the case of the exchange matrix this allows to change
the order of the summation in a very favourable way.
Using expression \eqref{eqn:ExchangeFE} we can write
for the application of $\matK$ to a test vector $\vec{x}$:
\todoil{Use some colour to make things more clear}
\begin{equation}
\label{eqn:ExchangeApply}
\begin{aligned}
	\left( \mat{K}^{(n)} \vec{x}  \right)_\mu
	&= \sum_{\nu\in\Ibash} K^{(n)}_{\mu\nu} x_\nu \\
	&= \sum_{\nu\in\Ibash} \sum_{i\in\Iocc} \sum_{\kappa\in\Ibash} \sum_{\lambda\in\Ibash}
		x_\nu \\
		&\hspace{70pt}
		\int_\Omega \int_\Omega
		\varphi_\mu(\vec{r}_1)
		\frac{
			C^{(n)}_{\kappa i} \varphi_\kappa(\vec{r}_1) \,
			C^{(n)}_{\lambda i} \varphi_\lambda(\vec{r}_2)
		}{r_{12}}
		\varphi_\nu(\vec{r}_2) \D \vec{r}_2 \D \vec{r}_1 \\
	&= \sum_{i\in\Iocc} \sum_{\kappa\in\Ibash}
		\int_\Omega
			\varphi_\mu(\vec{r}_1) \, C^{(n)}_{\kappa i} \varphi_\kappa(\vec{r}_1) \\
		&\hspace{70pt}
		\left(
		\int_\Omega
		\sum_{\nu,\lambda\in\Ibash}
			\frac{C^{(n)}_{\lambda i} \varphi_\lambda(\vec{r}_2) \, x_\nu \varphi_\nu(\vec{r}_2)}
			{r_{12}}
		\D \vec{r}_2
		\right) \D \vec{r}_1 \\
	&= \sum_{i\in\Iocc} \sum_{\kappa\in\Ibash}
		\int_\Omega
			\varphi_\mu(\vec{r}_1) \, C^{(n)}_{\kappa i} \varphi_\kappa(\vec{r}_1)
		\left(
		\int_\Omega \frac{\rho_{\vec{x},i}^{(n)}(\vec{r}_2)}{r_{12}} \D \vec{r}_2
		\right) \D \vec{r}_1,
\end{aligned}
\end{equation}
where we introduced the \newterm{exchange contraction densities}
\begin{equation}
	\label{eqn:ExchangeContrDens}
	\rho_{\vec{x},i}^{(n)}(\vec{r}) = \sum_{\nu,\lambda\in\Ibash}
	C^{(n)}_{\lambda i} \varphi_\lambda(\vec{r}_2) \, x_\nu \varphi_\nu(\vec{r}_2).
\end{equation}
For each $i\in\Iocc$ we can solve a Poisson equation
\begin{equation}
\label{eqn:ExchangePoissonEqns}
\begin{aligned}
	- \Delta V_{\vec{x}, i}(\vec{r}) &= 4 \pi \rho_{\vec{x},i}^{(n)}(\vec{r})
		&& \vec{r} \in \Omega \\
	V^{(n)}_{\vec{x}, i}(\vec{r}) &= 0 && \vec{r} \in \partial\Omega
\end{aligned}
\end{equation}
defining implicity the \newterm{exchange contraction potentials} $V^{(n)}_{\vec{x}, i}$.
With these potentials \eqref{eqn:ExchangeApply} becomes
\begin{align*}
	\left( \mat{K}^{(n)} \vec{x}  \right)_\mu
	&= \sum_{i\in\Iocc} \sum_{\kappa\in\Ibash}
		\int_\Omega
		\varphi_\mu(\vec{r}) \, C^{(n)}_{\kappa i} \varphi_\kappa(\vec{r})
		V^{(n)}_{\vec{x}, i}(\vec{r}) \D \vec{r}.
\end{align*}
Since $V^{(n)}_{\vec{x}, i}$ is a local operator,
the integral can be evaluated in $\bigO(\Nbas)$
and thus the complete expression in $\bigO(\Nelec \Nbas)$
once the  $V^{(n)}_{\vec{x}, i}$ are known.

%
% TODO complete scaling
%


\subsection{Coulomb-Sturmian-type orbitals}
\label{sec:BasisCS}

% https://www.math.tu-berlin.de/fachgebiete_ag_modnumdiff/fg_modellierung_simulation_und_optimierung_in_natur_und_ingenieurswissenschaften/v_menue/mitarbeiter/prof_dr_reinhold_schneider/publikationen/



\begin{figure}
	\centering
	\includeimage{4_solving_hf/local_energy_cs}
	\caption{Local energy of a Coulomb-Sturmian basis sets for hydrogen.}
	\label{fig:LocalEnergyCS}
\end{figure}

\begin{figure}
	\centering
	\includeimage{4_solving_hf/local_energy_cs_zoom}
	\caption{Zoom-in for local energy of Coulomb-Sturmian basis sets for hydrogen.}
	\label{fig:LocalEnergyCSZoom}
\end{figure}

\begin{figure}
	\centering
	\includeimage{4_solving_hf/relative_error_cs}
	\caption{Relative error for Coulomb-Sturmian basis sets for hydrogen.}
	\label{fig:LocalEnergyCS}
\end{figure}



% A few historic words and references
\cite{Rotenberg1970}
\cite{Gruzdev1990}
\cite{Avery2006} % Blue book
\cite{Avery2011} % Purple book
\cite{Morales2016}
\cite{Avery2017} % Fast evaluation of STOs



\begin{equation}
	bla
	\label{eqn:CSEquation}
\end{equation}

% Sturmians form a complete basis for H^2(R^3) and for H^1(R^1) (only the radial part)

% complete but not over-complete

In previous work we have investigated the use of so-called generalised Sturmians as basis functions
in electronic structure theory.
%
\newcommand{\rpack}{\vec{r}_1, \ldots, \vec{r}_N}
Generalised Sturmians $\Phi_\nu(\rpack)$ are the solutions to $N$-body Schrödinger-like equation
\begin{equation}
	\left( -\frac12 \sum_{j=1}^N \Delta_j + \beta_\nu V_0(\rpack) - E \right) \Phi_\nu(\rpack) = 0
	\label{eqn:GenSturm}
\end{equation}
where $V_0(\rpack)$ is
a zero-th order potential.

In case of atoms a good choice is the nuclear attraction:
The electrostatic electron-nucleus interaction
\[
	V_0(\rpack) = - \sum_{j=1}^N \frac{Z}{r_j}.
\]
\todo[inline]{Is many nuclei also possible here?}
Note that the electron-nucleus interaction is scaled by the factor $\beta_\nu$ in \eqref{eqn:GenSturm}
such that the solutions $\Phi_\nu(\rpack)$ all become isoenergetic.
This makes Sturmians reproduce the correct long-range decay behaviour of the electron density
as well as properly represent the nuclear cusp at the core.
As such the functional form of generalised Sturmians is very much related to STOs.
Unlike STOs the two-electron integrals can, however, be reformulated in a particular way
to make computing them less demanding.
This requires, however, that the mathematical properties of the Sturmians
can be fully exploited during the computation,
which in turn requires the Fock matrix to be arranged in a specific way.
Moreover it is advantageous to not build the coulomb and exchange matrix in memory at all,
but much rather only use these matrices in the form of matrix-vector products,
since this overall saves an order of magnitude in the computational scaling.


Coulomb-Sturmian basis function is composed of
\[ \varphi_\mu(\vec{r}) = \]

\todo[inline,caption={}]{
	\begin{itemize}
		\item What are Coulomb-Sturmians and where do they come from
		\item Show what needs to be done
		\item Show problems $\Rightarrow$ Lazy matrices
	\end{itemize}
}

% TODO Things to show:
%    - The larger n, the larger the energy under the action of the fock operator
%    - HF Kinetic energy scales with k: Lower if k lower
%    - HF Nuclear attraction energy scales with k: Higher if k lower
%    - HF electron-electron interaction scales with k: Lower if k lower
% Refer to these results in the chapter about Sturmian calculations

\todoil{Talk about contraction-based methods for the first time.
Allows to pick this up in the next section}


\subsection{Other types of basis functions}
The selection of basis function types presented so far
represents a fair amount of what is used for electronic structure theory
calculations nowadays.
Nevertheless there are few more basis function types,
which should not go unmentioned.

This first of all applies to
plane-wave and projector-augmented wave approaches
~\cite{Kresse1996,Kresse1999,Mortensen2005,Enkovaara2010},
which are both extremely popular as well as extremely suitable for performing
electronic structure calculations
on extended periodic systems or systems in the solid state.
Over the years there has also been an enormous amount of development
into the direction of numerical basis functions.
\citet{Frediani2015} provides an excellent review.
The approaches range from a so-called fully numerical treatment,
where similar to the finite-element method as mentioned above,
the complete problem is treated by grid-based methods.
This includes employing clever
numerical integration grids~\cite{Losilla2012DCRsp,Toivanen2015,Enkovaara2010}
or discretisation schemes
based on finite-differences~\cite{Kobus2013}
or finite-elements~\cite{Tsuchida1995,Briggs1996,Pask05,Lehtovaara2009,Alizadegan2010,Avery2011PhD,Davydov2015,Boffi2016}.
Some approaches~\cite{Soler2002} use a dual representation,
where the same orbitals are both represented on a real-space grid
as well as in the form of orbitals
or they only treat
part of the electronic wavefunction numerically~\cite{Fischer1978,LUCAS}.
Such methods for example employ a factorisation of the one-particle functions
into a numerical radial part and a spherical harmonic function.
Last but not least one should also mention
wavelet-based methods~\cite{Bischoff2011,Bischoff2012,Bischoff2013,Bischoff2014,Bischoff2014a,Bischoff2017},
where quite some progress has been made in recent years.
To the best of my knowledge wavelet-based electronic structure theory
is the only methodology where guaranteed precision in the solution to the
respective problems can be achieved.

\subsection{Mixed bases}
Already done in some or another sense in practice.
For example the SIESTA method~\cite{Soler2002} uses
a dual representation on a real-space grid as well as
an atomic orbital basis


Suggest an ansatz basis $\{\varphi_i + \chi_i\}_{i\in\Ibas}$
where $\chi_i$ is a fixed, predetermined Sturmian solution
and $\varphi_i$ are corrections to be found from FE.
Derive the HF energy functional expression for such an ansatz
where $\chi_i$ are fixed and only $\varphi_i$ are the parameters.
(Maybe do this in outlook?)




Guess methods:
Do a calculation in Coulomb-Sturmians for atoms.
Use the result for EHT and project overall orbitals onto FE grid
for fine structure.


\subsection{Takeaway}
\todoil{
	Maybe go a bit into contrasting the basis function types
	if not already done before properly.
	
	Some summarising remarks for this probably rather long section.
}
