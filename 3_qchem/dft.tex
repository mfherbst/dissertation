\section{Density-functional theory}
\label{sec:DFT}
\defineabbr{DFT}{DFT\xspace}{Density-functional theory}
In this section we want to briefly look at a different
approach towards modelling the electronic structure.
Instead of solving for the wave function $\Psi_0$ associated to the
ground state of the electronic Hamiltonian $\Op{H}_{\Nelec}$,
the idea behind \newterm{density-functional theory}
is to solve for the state's electronic density $\rho_0$ instead.

The rationale for this is twofold.
Firstly the density contains all information about the chemical system.
The integral
$\int_{\R^3} \rho(\vec{r}) \D\vec{r}$
evaluates to the number of electrons $\Nelec$
and via Kato's cusp condition~\cite{Kato1951} one may obtain the nuclear
charges $Z_A$ via the derivatives of the electron density at the cusp points.
Secondly the Hohnberg-Kohn theorems~\cite{Hohenberg1964}
as well as the Levy constrained search ansatz~\cite{Levy1979}
provide a unique identification between a particular ground state electron density
and the potential, which generates this density.
Even from a mathematical point of view
solving for the ground state density $\rho_0(\vec{r})$ is thus
sufficient to characterise all properties of the ground state of a system.

The Levy constrained search ansatz~\cite{Levy1979}
provides a conceptionally rather
intuitive route to obtain the ground state density,
namely by a constrained minimisation of the energy
with respect to all possible densities.
The issue with this procedure is that a closed-form expression
for the energy functional $\mathcal{E}(\rho)$,
which returns the energy of a given density,
is not known for any relevant chemical system.
In other words Levy constrained search in the form presented so far
cannot be applied to chemical systems.

Further progress can be made with the Kohn-Sham ansatz~\cite{Kohn1965}, however.
The idea is to consider a fictitious system of
$\Nelec$ non-interacting electrons,
which still has the property that it reproduces the exact
ground state density of the full, interacting system.
In this model system the exact wave function is a single determinant
\begin{align*}
	\Psi &= \Phi_\Theta = \bigwedge_{i=1}^{\Nelec} \psi_i
	&&\text{where}&
	\Theta \equiv \left(\psi_1, \psi_2, \ldots, \psi_{\Nelec}\right)
	&\in \left(H^1(\R^3, \C)\right)^{\Nelec}
\end{align*}
is a tuple of $\Nelec$ single-particle functions.
Ignoring spin the resulting ground state density is
\[
	\rho_\Theta(\vec{r}) = \sum_{i=1}^{\Nelec} \abs{\psi_i(\vec{r})}^2,
\]
which allows to write the Kohn-Sham energy functional as
\begin{equation}
	\begin{aligned}
	\mathcal{E}^\text{KS}(\Theta)
	&= \frac12 \sum_{i=1}^{\Nelec} \int_{\R^3} \norm{\nabla \psi_i}_2^2 \D\vec{r}
	+ \int_{\R^3} \sum_{A=1}^M
		\frac{Z_A \, \rho_\Theta(\vec{r})}{\norm{\vec{r} - \vec{R}_A}_2} \D\vec{r} \\
	&\hspace{20pt}
	+ \frac12 \int_{\R^3}\int_{\R^3}
		\frac{\rho_\Theta(\vec{r}_1) \rho_\Theta(\vec{r}_2)}
			{\norm{\vec{r}_1 - \vec{r}_2}_2} \D\vec{r}_1 \D\vec{r}_2
	+ E_{xc}(\rho_\Theta).
	\end{aligned}
	\label{eqn:KSEnergyFunctional}
\end{equation}
In this expression $E_{xc}$ is the \newterm{exchange-correlation functional},
which depends only on the density function $\rho$.
This term is supposed to describe the non-local
many-body effects not yet contained in the other terms,
which is both the exchange interaction as well as correlation effects.
The crux with Kohn-Sham \DFT is that its exact functional form is again unknown,
such that one has to live with approximations.
Which exchange-correlation functional is sensible for a particular
problem depends very much on the context of the chemical system,
the property one is interested in and is still subject of debate
in quantum-chemical literature.

Following the original Levy constrained search,
we want to find the density corresponding to the minimal energy,
which in the Kohn-Sham picture implies the minimisation
of $\mathcal{E}^\text{KS}(\Theta)$ with respect to the orbitals,
thus the problem
\begin{equation}
	E_0 \leq E_0^\text{KS}
	= \inf \left\{
		\mathcal{E}^\text{KS}(\Theta)
		\, \middle| \,
		\Theta \in \left(H^1(\R^3, \C)\right)^{\Nelec}, \,
		\forall i,j \,
		\braket{\psi_i}{\psi_j}_1 = \delta_{ij}.
	\right\}.
	\label{eqn:KSMO}
\end{equation}
Both the energy functional \eqref{eqn:KSEnergyFunctional}
as well as the Kohn-Sham minimisation problem \eqref{eqn:KSMO} are closely
related to the \HF problem \eqref{eqn:HFMO}.
In fact the only difference is the substitution of the exchange energy term
by the exchange-correlation functional.
As such it should not be very surprising that the methods employed to solve
\eqref{eqn:KSMO} is very similar to \HF as well.
The conditions to obtain the stationary points of \eqref{eqn:KSMO},
the Euler-Lagrange equations,
can be reformulated as
\begin{align}
	\label{eqn:KSequations}
	\Op{F}^\text{KS}_{\Theta^0} \psi_i^0 &= \varepsilon_i \psi_i^0
	&&\text{and}&
	\braket{\psi_i^0}{\psi_j^0} &= \delta_{ij}
\end{align}
where $\Theta^0$ is the minimiser of \eqref{eqn:KSMO} and
\begin{equation}
	\Op{F}_{\Theta^0}^\text{KS} = \Op{T} + \Op{V}_0 + \Op{J}_{\Theta^0} + V_{xc}
	\label{eqn:KohnShamOperator}
\end{equation}
is the Kohn-Sham operator.
Its difference to the Fock operator \eqref{eqn:FockOperator}
is again simply the replacement of the exchange operator $\op{K}_{\Theta^0}$
by the \newterm{exchange-correlation potential} $V_{xc}(\vec{r})$,
which is the derivative of the exchange-correlation energy $E_{xc}(\rho)$
with respect to the density function $\rho$.
Equation \eqref{eqn:KSequations} as well as the minimisation problem
\eqref{eqn:KSMO} can now be discretised
similar to the procedure outlined in section \vref{sec:DiscreteHF}
for Hartree-Fock,
which leads to an iterative self-consistent field procedure,
which is very similar to the Hartree-Fock \SCF outlined in
remark \vref{rem:PropertiesDiscretised}.
Algorithmically both for Kohn-Sham \DFT as well as \HF the same type of problem
needs to be solved, such that all of the numerical procedures
discussed in the next chapters for \HF could be applied to Kohn-Sham \DFT
with only very few changes.
