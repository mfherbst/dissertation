\section{Density-functional theory}
\defineabbr{DFT}{DFT\xspace}{Density-functional theory}
In this section we want to briefly look at a slightly different
approach towards modelling the electronic structure.
Instead of solving for the wave function $\Psi_I$ associated to the
$I$-th eigenstate of the electronic Hamiltonian $\Op{H}_{\Nelec}$,
we will solve for the state's electronic density $\rho_I$ directly instead.

The rationale for this is twofold.
Firstly the density contains all information about the chemical system.
The integral
\[ \int_{\R^3} \rho(\vec{r}) \D\vec{r} \]
evaluates to the number of electrons $\Nelec$
and via Kato's cusp condition~\cite{Kato1951} one may obtain the nuclear
charges $Z_A$ via the derivatives of the electron density at the cusp points.
Secondly the Hohnberg-Kohn theorems~\cite{Hohenberg1964}
provide a unique identification between a particular electron density
and the potential, which generates this density.

% See C6 notes about this matter


Generalised proof by Levy~\cite{Levy1979}

Kohn-Sham equations~\cite{Kohn1965}



\todo[inline,caption={}]{
	\begin{itemize}
		\item Brief rationale
		\item Show physical and mathematical relationship to HF
		\item Extremely brief
		\item Mention TDDFT as excited states method
	\end{itemize}
}

