\section{Single-determinant ansatz}
\label{sec:HFIntro}

\nomenclature{$\NelecA, \NelecB$}{Number of $\alpha$/$\beta$ electrons.
Note $\NelecA + \NelecB = \Nelec$ and $\NelecA \ge \NelecB$ by convention.}
\nomenclature{$\Iorb$}{Index set of computed SCF orbitals, typically $\{1, \ldots \Norb\}$}
\nomenclature{$\Norb$}{The number of computed SCF orbitals.
Note, that $\Norb \le \Nbas$.}

\defineabbr{HF}{HF\xspace}{Hartree-Fock}
\nomenclature{$\OpFock, \OpFockFull$}{$1$-electron Fock operator (dependent on its eigenfunctions $\{ \varphi_i \}_{i \in \Iorb}$)}
\nomenclature{$\Op{\rho}$}{Density operator}
\nomenclature{$\IoccA$, $\IoccB$}{Index set of occupied SCF orbitals of $\alpha$ or $\beta$ spin, respectively. Typically $\{1, \ldots, \NelecA\}$ and similar for $\IoccB$.}
\nomenclature{$\epsilonconv$}{Convergence tolerance for an iterative process.
	If the error is below this value, the process should be considered converged.}

\nomenclature{$\Op{K}$}{One-electron exchange operator}
\nomenclature{$\Op{J}$}{Effective Coulomb operator}

In the previous section we noted that even an approximate
solution to the electronic Schrödinger equation \eqref{eqn:ElectronicSchrödinger}
via the full CI ansatz
is hardly feasible.
Even if comparatively small one-electron basis sets
$\{ \varphi_\mu \}_{\mu\in\Ibas} \subset H^1(\R^3,\C)$
are used,
the dimensionality
of the matrix $\mat{A}_\text{FCI}$ becomes simply too large.
In this section we discuss the opposite end of the scale and
only consider one-dimensional subspaces of the form domain
\[ Q(\Op{H}_{\Nelec}) = \bigwedge_{i=1}^{\Nelec} H^1(\R^3, \C) \cap L^2(\R^3, \C) \]
for solving the electronic problem.
Formally by the Courant-Fischer theorem \eqref{thm:CourantFischer}
the ground state electronic energy $E_0$ can be obtained by
a variational minimisation over all subspaces of dimension $1$.
In other words
\begin{align}
	\label{eqn:GroundStateVariational}
	E_0 &= \inf_{\Psi \in \mathcal{W}_{\Nelec}} \braket{\Psi}{\Op{H}_{\Nelec} \Psi}_{\Nelec} \\
\intertext{where}
	\label{eqn:NormalisedFormDomain}
	\mathcal{W}_{\Nelec} &= \left\{ \Psi \in Q(\Op{H}_{\Nelec})
	\, \middle| \,
	\norm{\Psi}_{L^2(\R^{3 \Nelec}, \C)} = 1 \right\}.
\end{align}
denotes the subspace of all normalised functions from $Q(\Op{H}_{\Nelec})$.
If we restrict the search to only run over the space
\begin{equation}
	\mathcal{R}^1_{\Nelec} = \left\{ \bigwedge_{i=1}^{\Nelec} \psi_i
	\, \middle| \,
	\psi_i \in H^1(\R^3, \C),
	\braket{\psi_i}{\psi_j}_1 = \delta_{ij}
	\quad \forall \, 1\leq i,j\leq \Nelec
	\right\}
	\label{eqn:RankOneSubspace}
\end{equation}
of all normalised Slater determinants,
which is a proper subspace of $\mathcal{W}_{\Nelec}$,
we no longer yield the exact energy.
According to corollary \vref{cor:Convergence} we merely obtain an upper bound
\begin{equation}
	E_0 \leq E_0^\text{HF} = \inf_{\Phi \in \mathcal{R}^1_{\Nelec}}
	\braket{\Phi}{\Op{H}_{\Nelec} \Phi}_{\Nelec}.
	\label{eqn:HFDeterminant}
\end{equation}
The implied procedure,
where an approximation to the electronic ground state
is computed by minimising the sesquilinear form of $\Op{H}_{\Nelec}$
over the space spanned by all normalised Slater determinants,
is the celebrated \newterm{Hartree-Fock}~(\HF) approximation~\cite{Fock1930}.
The resulting minimal energy $E_0^\text{HF}$
is the \HF ground state energy
and the corresponding minimising determinant
$\Phi_0$ the \HF ground state.

Notice, that $\Phi_0$ is --- by construction ---
the best possible single Slater determinant to
approximate the electronic ground state.
Mathematically speaking the set of Slater determinants
$\mathcal{R}^1_{\Nelec}$
is exactly the set of all elements from $\mathcal{W}_{\Nelec}$,
which are of tensor rank 1.
For this reason one sometimes refers to the Hartree-Fock
ground state $\Phi_0$
as a \textbf{rank-1 approximation} to the exact
electronic ground state.

\begin{rem}[Molecular orbital formulation of \HF]
	Let
	\[  \Theta \equiv \left(\psi_1, \psi_2, \ldots, \psi_{\Nelec}\right)
		\in \left(H^1(\R^3, \C)\right)^{\Nelec} \]
	denote an arbitrary tuple of single-particle functions.
	It uniquely defines a Slater determinant
	$\Phi_\Theta = \bigwedge_{i=1}^{\Nelec} \psi_i.$
	Inserting this ansatz into the energy expression
	\eqref{eqn:EnergySlaterDeterminant} for a single determinant
	yields the \newterm{Hartree-Fock energy functional}
	\begin{equation}
		\begin{aligned}
		\mathcal{E}^\text{HF}(\Theta)
		&= \frac12 \sum_{i=1}^{\Nelec} \int_{\R^3} \norm{\nabla \psi_i}_2^2 \D\vec{r}
		+ \int_{\R^3} \sum_{A=1}^M
			\frac{\rho_\Theta(Z_A \, \vec{r})}{\norm{\vec{r} - \vec{R}_A}_2} \D\vec{r} \\
		&\hspace{20pt}
		+ \frac12 \int_{\R^3}\int_{\R^3}
			\frac{\rho_\Theta(\vec{r}_1) \rho_\Theta(\vec{r}_2)}
				{\norm{\vec{r}_1 - \vec{r}_2}_2} \D\vec{r}_1 \D\vec{r}_2
		- \frac12 \int_{\R^3}\int_{\R^3}
			\frac{\abs{\gamma_\Theta(\vec{r}_1, \vec{r}_2)}^2}
				{\norm{\vec{r}_1 - \vec{r}_2}_2} \D\vec{r}_1 \D\vec{r}_2
		\end{aligned}
		\label{eqn:HFEnergyFunctional}
	\end{equation}
	where
	\begin{align}
		\rho_\Theta(\vec{r}) &= \sum_{i=1}^{\Nelec} \abs{\psi_i(\vec{r})}^2
		&\text{and} &&
		\gamma_\Theta(\vec{r}_1,\vec{r}_2)
			&= \sum_{i=1}^{\Nelec} \cc{\psi}_i(\vec{r}_1) \psi_i(\vec{r}_2)
	\end{align}
	are the \newterm{electron density}
	and the \newterm{one-particle reduced density matrix}, respectively.
	The HF ansatz \eqref{eqn:RankOneSubspace}
	thus becomes
	\begin{equation}
		E_0 \leq E_0^\text{HF}
		= \inf \left\{
			\mathcal{E}^\text{HF}(\Theta)
			\, \middle| \,
			\Theta \in \left(H^1(\R^3, \C)\right)^{\Nelec}, \,
			\forall i,j \,
			\braket{\psi_i}{\psi_j}_1 = \delta_{ij}.
		\right\}
		\label{label:HFMO}
	\end{equation}
	The minimiser $\Theta_0$, \ie the tuple for which the minimum
	energy $E_0^\text{HF} = \mathcal{E}^\text{HF}(\Theta_0)$
	is exactly obtained,
	defines the HF ground state
	$\Phi_0 \equiv \Phi_{\Theta_0}$.
\end{rem}

Before we discuss the mathematical properties of the \HF ansatz,
let us first generalise our setting slightly.

\begin{rem}[Spin degree of freedom]
	\label{rem:Spin}
Recall that in section \vref{sec:ElectronicSchrödinger}
we needed the exterior power in order to construct a domain for
our electronic Hamiltonian,
which complies both with the
spin statistics theorem as well as experimental observations.

comply with the
spin statistics theorem as well as experimental results
for electrons.


for spin-$\sfrac12$ particl


This was needed, since electrons are spin-$\sfrac12$ particles,
thus fermions.
What our current treatment lacks, however,
is the description of this intrin


\end{rem}

thus may obey two spin states


So far we have included the antisymmetry properties
resulting from the spin statistics theorem
in section \vref{sec:ElectronicSchrödinger}.
We have not included the spin degree of freedom, however.


treated the 

neglected electron spin entirely
and treated quantum-mechanics only 



% ---------------------------

\begin{rem}[Spin]
	Spin as spinor!
\todoil{
In our discussion so far we neglected the spin component
of the wavefunction
$L^2(\R^{3\Nelec}, \C)$ to $L^2(\R^{3\Nelec}, \C^{2\Nelec})$

two-component spinor representation

	Everything works the same.
	Especially if one thinks of $\alpha$-spin orbitals
	as those which only produce non-zero in the first component
	of the $\C^2$ vector and $\beta$-spin as only producing
	values on the second one.
	Then we also have $\Nelec$ orbitals to occupy
	$\Nelec$ electrons, but their spatial part may be the same or different.
}
\end{rem}


Mathematical results

\begin{rem}[Invariance under orbital rotations]
	One can easily show


\end{rem}

The following theorem summarises some mathematical
results regarding \HF equations~\cite{Lieb1977,Lions1987,Lions1989}.
All results shown hold for the general case with spin as well.
\begin{thm}
	\todoil{Write it up.}
\end{thm}
\todoil{Show these results as well as the invariance remark
in the discrete case (see MWM example sheet)}

% define occupied and virtual

% mention term SCF
% spin generalisation?

\begin{rem}[discretisation]
	\todo[inline,caption={}]{
		\begin{itemize}
			\item 
	$\varphi$ \newterm{one-particle basis} for HF
	\item
	$\psi$ HF orbitals or (SCF) orbital basis,
	typically in FCI one uses the HF orbital basis as the one-particle basis functions for FCI
		\end{itemize}
}
\end{rem}


\todoil{Present explicit expressions for the individual
	terms. Perhaps already introduce the J and K operators
and use the functional form and formalism of Cances}

\todoil{
	Quick note that real is possible: \\
For an atomic system, we can consider the nucleus to be located at the origin
of the coordinate system, which allows us to simplify
the Hamiltonian to \\
Argue that we can do the mathematical background in real function
arithmetic only and still use the results for complex input due to linearity.
$\Psi \in H^2(\R^{3d}, \R)$
}


% Coefficient matrix

\newterm{Hartree-Fock}
\todo[inline,caption={}]{
	\begin{itemize}
		\item Idea
		\item Discretisation
		\item use Formalism of Cances
		\item Derivation of HF in discretised space
		\item Summary of the numerical task
		\item Summary of the result (SCF orbitals)
		\item Explain properties of the individual terms of Fock operator (locality, ..)
		\item Show what we miss compared to FCI
	\end{itemize}
}

The discretised Hartree-Fock equations in weak form read
\begin{equation}
	\forall i, j \in \Iorb :  \mbra{\varphi_i} \OpFockFull \mket{\varphi_j} = \lambda_j \braket{\varphi_i}{\varphi_j}
	\label{eqn:HF}
\end{equation}

\begin{equation}
	C^\alpha_{\mu i}, C^\beta_{\mu i}
	\label{eqn:SCFalphaBetaCoeffs}
\end{equation}

% Proove the Fock operator (for fixed orbitals) is linear in the function
% it acts upon

% Explain the term Hartree-Fock and SCF (idea from Hartree, but spin adaption from Fock and Slater)

