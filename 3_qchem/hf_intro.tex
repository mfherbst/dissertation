\section{Single-determinant ansatz}
\nomenclature{$\NelecA, \NelecB$}{Number of $\alpha$/$\beta$ electrons.
Note $\NelecA + \NelecB = \Nelec$ and $\NelecA \ge \NelecB$ by convention.}
\nomenclature{$\Iorb$}{Index set of computed SCF orbitals, typically $\{1, \ldots \Norb\}$}
\nomenclature{$\Norb$}{The number of computed SCF orbitals.
Note, that $\Norb \le \Nbas$.}

\defineabbr{HF}{HF\xspace}{Hartree-Fock}
\nomenclature{$\OpFock, \OpFockFull$}{$1$-electron Fock operator (dependent on its eigenfunctions $\{ \varphi_i \}_{i \in \Iorb}$)}
\nomenclature{$\Op{\rho}$}{Density operator}
\nomenclature{$\IoccA$, $\IoccB$}{Index set of occupied SCF orbitals of $\alpha$ or $\beta$ spin, respectively. Typically $\{1, \ldots, \NelecA\}$ and similar for $\IoccB$.}
\nomenclature{$\epsilonconv$}{Convergence tolerance for an iterative process.
	If the error is below this value, the process should be considered converged.}

\nomenclature{$\Op{K}$}{One-electron exchange operator}
\nomenclature{$\Op{J}$}{Effective Coulomb operator}

\todoil{
	Spin: Replace $H^1(\R^3, \C)$ by $H^1(\R^3, \C^2)$.
	Everything works the same.
	Especially if one thinks of $\alpha$-spin orbitals
	as those which only produce non-zero in the first component
	of the $\C^2$ vector and $\beta$-spin as only producing
	values on the second one.
	Then we also have $\Nelec$ orbitals to occupy
	$\Nelec$ electrons, but their spatial part may be the same or different.
}


\todoil{Present explicit expressions for the individual
	terms. Perhaps already introduce the J and K operators
and use the functional form and formalism of Cances}

% Note that we need a more general setting here.
% Discuss how spin is included
% Slater determinants with spin
\todoil{
	Quick note that real is possible: \\
For an atomic system, we can consider the nucleus to be located at the origin
of the coordinate system, which allows us to simplify
the Hamiltonian to \\
Argue that we can do the mathematical background in real function
arithmetic only and still use the results for complex input due to linearity.
$\Psi \in H^2(\R^{3d}, \R)$
}



\todo[inline,caption={}]{
\begin{equation}
	\mathcal{W}_{\Nelec} \equiv \left\{ \Psi \in H^1(R^{3\Nelec}, \C) \bigwedge^{\Nelec} L^2(\R^3, \C)
	\, \middle| \,
	\norm{\Psi}_{L^2(\R^{3 \Nelec}, \C)} = 1 \right\}.
	\label{eqn:DefSetSlaterDeterminant}
\end{equation}
}


% Density matrix
% Density operator
% Coefficient matrix

\newterm{Hartree-Fock}
\todo[inline,caption={}]{
	\begin{itemize}
		\item Idea
		\item Discretisation
		\item use Formalism of Cances
		\item Derivation of HF in discretised space
		\item Summary of the numerical task
		\item Summary of the result (SCF orbitals)
		\item Explain properties of the individual terms of Fock operator (locality, ..)
		\item Show what we miss compared to FCI
	\end{itemize}
}

The discretised Hartree-Fock equations in weak form read
\begin{equation}
	\forall i, j \in \Iorb :  \mbra{\varphi_i} \OpFockFull \mket{\varphi_j} = \lambda_j \braket{\varphi_i}{\varphi_j}
	\label{eqn:HF}
\end{equation}

\begin{equation}
	C^\alpha_{\mu i}, C^\beta_{\mu i}
	\label{eqn:SCFalphaBetaCoeffs}
\end{equation}

% Proove the Fock operator (for fixed orbitals) is linear in the function
% it acts upon


