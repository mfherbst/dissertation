\section{Single-determinant ansatz}


\defineabbr{HF}{HF\xspace}{Hartree-Fock}
\nomenclature{$\OpFock, \OpFockFull$}{$1$-electron Fock operator (dependent on its eigenfunctions $\{ \varphi_i \}_{i \in \Iorb}$)}
\nomenclature{$\Op{K}$}{One-electron exchange operator}
\nomenclature{$\Op{J}$}{Effective Coulomb operator}
\defineabbr{ERI}{ERI\xspace}{Electron repulsion integrals}
\nomenclature{$(ij|kl)$}{Electron repulsion integrals in chemist's or Mullikan notation, see definition at TODO}
\nomenclature{$\Op{\rho}$}{Density operator}
\nomenclature{$\IoccA$, $\IoccB$}{Index set of occupied SCF orbitals of $\alpha$ or $\beta$ spin, respectively. Typically $\{0, \ldots, \NelecA\}$ and similar for $\IoccB$.}
\nomenclature{$\epsilonconv$}{Convergence tolerance for an iterative process.
	If the error is below this value, the process should be considered converged.}

% Density matrix
% Density operator
% Coefficient matrix

\newterm{Hartree-Fock}
\todo[inline,caption={}]{
	\begin{itemize}
		\item Idea
		\item Discretisation
		\item use Formalism of Cances
		\item Derivation of HF in discretised space
		\item Summary of the numerical task
		\item Summary of the result (SCF orbitals)
		\item Explain properties of the individual terms of Fock operator (locality, ..)
		\item Show what we miss compared to FCI
	\end{itemize}
}

The discretised Hartree-Fock equations in weak form read
\begin{equation}
	\forall i, j \in \Iorb :  \mbra{\varphi_i} \OpFockFull \mket{\varphi_j} = \lambda_j \braket{\varphi_i}{\varphi_j}
	\label{eqn:HF}
\end{equation}

\begin{equation}
	C^\alpha_{\mu i}, C^\beta_{\mu i}
	\label{eqn:SCFalphaBetaCoeffs}
\end{equation}

% Proove the Fock operator (for fixed orbitals) is linear in the function
% it acts upon


