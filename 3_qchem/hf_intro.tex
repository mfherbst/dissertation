\section{Single-determinant ansatz}
\label{sec:HFIntro}

\nomenclature{$\NelecA, \NelecB$}{Number of $\alpha$/$\beta$ electrons.
Note $\NelecA + \NelecB = \Nelec$ and $\NelecA \ge \NelecB$ by convention.}
\nomenclature{$\Iorb$}{Index set of computed SCF orbitals, typically $\{1, \ldots \Norb\}$}
\nomenclature{$\Norb$}{The number of computed SCF orbitals.
Note, that $\Norb \le \Nbas$.}

\defineabbr{HF}{HF\xspace}{Hartree-Fock}
\defineabbr{SCF}{SCF\xspace}{Self-consistent field approach}
%\nomenclature{$\Op{\rho}$}{Density operator}
\nomenclature{$\IoccA$, $\IoccB$}{Index set of occupied SCF orbitals of $\alpha$ or $\beta$ spin, respectively. Typically $\{1, \ldots, \NelecA\}$ and similar for $\IoccB$.}
\nomenclature{$\epsilonconv$}{Convergence tolerance for an iterative process.
	If the error is below this value, the process should be considered converged.}

\nomenclature{$\Op{K}$}{One-electron exchange operator}
\nomenclature{$\Op{J}$}{Effective Coulomb operator}

In the previous section we noted that even an approximate
solution to the electronic Schrödinger equation \eqref{eqn:ElectronicSchrödinger}
via the full CI ansatz
is hardly feasible.
Even if comparatively small one-electron basis sets
$\{ \varphi_\mu \}_{\mu\in\Ibas} \subset H^1(\R^3,\C)$
are used,
the dimensionality
of the matrix $\mat{A}_\text{FCI}$ becomes simply too large.
In this section we discuss the opposite end of the scale and
only consider one-dimensional subspaces of the form domain
\[ Q(\Op{H}_{\Nelec}) = \bigwedge_{i=1}^{\Nelec} H^1(\R^3, \C) \cap L^2(\R^3, \C) \]
for solving the electronic problem.
Formally by the Courant-Fischer theorem \eqref{thm:CourantFischer}
the ground state electronic energy $E_0$ can be obtained by
a variational minimisation over all subspaces of dimension $1$.
In other words
\begin{align}
	\label{eqn:GroundStateVariational}
	E_0 &= \inf_{\Psi \in \mathcal{W}_{\Nelec}} \braket{\Psi}{\Op{H}_{\Nelec} \Psi}_{\Nelec} \\
\intertext{where}
	\label{eqn:NormalisedFormDomain}
	\mathcal{W}_{\Nelec} &= \left\{ \Psi \in Q(\Op{H}_{\Nelec})
	\, \middle| \,
	\norm{\Psi}_{L^2(\R^{3 \Nelec}, \C)} = 1 \right\}.
\end{align}
denotes the subspace of all normalised functions from $Q(\Op{H}_{\Nelec})$.
If we restrict the search to only run over the space
\begin{equation}
	\mathcal{R}^1_{\Nelec} = \left\{ \bigwedge_{i=1}^{\Nelec} \psi_i
	\, \middle| \,
	\psi_i \in H^1(\R^3, \C),
	\braket{\psi_i}{\psi_j}_1 = \delta_{ij}
	\quad \forall \, 1\leq i,j\leq \Nelec
	\right\}
	\label{eqn:RankOneSubspace}
\end{equation}
of all normalised Slater determinants,
which is a proper subspace of $\mathcal{W}_{\Nelec}$,
we no longer yield the exact energy.
According to corollary \vref{cor:Convergence} we merely obtain an upper bound
\begin{equation}
	E_0 \leq E_0^\text{HF} = \inf_{\Phi \in \mathcal{R}^1_{\Nelec}}
	\braket{\Phi}{\Op{H}_{\Nelec} \Phi}_{\Nelec}.
	\label{eqn:HFDeterminant}
\end{equation}
The implied procedure,
where an approximation to the electronic ground state
is computed by minimising the sesquilinear form of $\Op{H}_{\Nelec}$
over the space spanned by all normalised Slater determinants,
is the celebrated \newterm{Hartree-Fock}~(\HF) approximation~\cite{Fock1930}.
The resulting minimal energy $E_0^\text{HF}$
is the \HF ground state energy
and the corresponding minimising determinant
$\Phi_0$ the \HF ground state.

The \HF approach is named both
after Douglas Hartree and Vladimir Fock.
Historically \citet{Hartree1928} first proposed
an ansatz for the electronic problem
based on symmetric products of one-electron functions,
which is algorithmically very similar to the procedure followed
nowadays~(see remark~\vref{rem:SCFcoeff}).
\citet{Slater1930a}
and \citet{Fock1930} then both noted
the issue arising from the use of symmetric
products in the context of spin~(see section \vref{sec:ElectronicSchrödinger})
and subsequently Fock reformulated the procedure
using Slater determinants.

Notice, that $\Phi_0$ is --- by construction ---
the best possible single Slater determinant to
approximate the electronic ground state.
Mathematically speaking the set of Slater determinants
$\mathcal{R}^1_{\Nelec}$
is exactly the set of all elements from $\mathcal{W}_{\Nelec}$,
which are of tensor rank 1.
For this reason one sometimes refers to the Hartree-Fock
ground state $\Phi_0$
as a \textbf{rank-1 approximation} to the exact
electronic ground state.

\begin{rem}[Molecular orbital formulation of \HF]
	Let
	\[  \Theta \equiv \left(\psi_1, \psi_2, \ldots, \psi_{\Nelec}\right)
		\in \left(H^1(\R^3, \C)\right)^{\Nelec} \]
	denote an arbitrary tuple of single-particle functions.
	It uniquely defines a Slater determinant
	$\Phi_\Theta = \bigwedge_{i=1}^{\Nelec} \psi_i.$
	Inserting this ansatz into the energy expression
	\eqref{eqn:EnergySlaterDeterminant} for a single determinant
	yields the \newterm{Hartree-Fock energy functional}
	\begin{equation}
		\begin{aligned}
		\mathcal{E}^\text{HF}(\Theta)
		&= \frac12 \sum_{i=1}^{\Nelec} \int_{\R^3} \norm{\nabla \psi_i}_2^2 \D\vec{r}
		+ \int_{\R^3} \sum_{A=1}^M
			\frac{Z_A \, \rho_\Theta(\vec{r})}{\norm{\vec{r} - \vec{R}_A}_2} \D\vec{r} \\
		&\hspace{20pt}
		+ \frac12 \int_{\R^3}\int_{\R^3}
			\frac{\rho_\Theta(\vec{r}_1) \rho_\Theta(\vec{r}_2)}
				{\norm{\vec{r}_1 - \vec{r}_2}_2} \D\vec{r}_1 \D\vec{r}_2
		- \frac12 \int_{\R^3}\int_{\R^3}
			\frac{\abs{\gamma_\Theta(\vec{r}_1, \vec{r}_2)}^2}
				{\norm{\vec{r}_1 - \vec{r}_2}_2} \D\vec{r}_1 \D\vec{r}_2
		\end{aligned}
		\label{eqn:HFEnergyFunctional}
	\end{equation}
	where
	\begin{align}
		\label{eqn:HFDensityNoSpin}
		\rho_\Theta(\vec{r}) &= \sum_{i=1}^{\Nelec} \abs{\psi_i(\vec{r})}^2
		&\text{and} &&
		\gamma_\Theta(\vec{r}_1,\vec{r}_2)
			&= \sum_{i=1}^{\Nelec} \cc{\psi}_i(\vec{r}_1) \psi_i(\vec{r}_2)
	\end{align}
	are the \newterm{electron density}
	and the \newterm{one-particle reduced density matrix}, respectively.
	The HF ansatz \eqref{eqn:RankOneSubspace}
	thus becomes
	\begin{equation}
		E_0 \leq E_0^\text{HF}
		= \inf \left\{
			\mathcal{E}^\text{HF}(\Theta)
			\, \middle| \,
			\Theta \in \left(H^1(\R^3, \C)\right)^{\Nelec}, \,
			\forall i,j \,
			\braket{\psi_i}{\psi_j}_1 = \delta_{ij}.
		\right\}
		\label{eqn:HFMO}
	\end{equation}
	The minimiser $\Theta^0$, \ie the tuple for which the minimum
	energy $E_0^\text{HF} = \mathcal{E}^\text{HF}(\Theta^0)$
	is exactly obtained,
	defines the HF ground state
	$\Phi_0 \equiv \Phi_{\Theta^0}$.
\end{rem}

Before we discuss the mathematical properties of the \HF ansatz,
let us first pick up on our discussion
of spin in section \vref{sec:ElectronicSchrödinger}
and generalise the formalism.

\begin{rem}[Spin-adapted formulation of \HF]
	\label{rem:Spin}
	The mathematical treatment up to this point
	only includes one property resulting from the spin-$\sfrac12$ nature
	of electrons,
	namely the antisymmetry of the wave function.
	The missing property is the explicit inclusion of the spin degree of freedom.
	For a single spin-$\sfrac12$ particle the spin degree of freedom
	spans the two-dimensional Hilbert space $\C^2$,
	which can be probed by the spin operator
	\[ \op{\vec{s}} \equiv \left( \op{s}_x, \op{s}_y, \op{s}_z \right)
		= \frac12 \left( \sigma_x, \sigma_y, \sigma_z \right). \]
	In this expression we used the \newterm{Pauli matrices} defined as
	\begin{align}
		\label{eqn:PauliMatrices}
		\sigma_x &= \mm{0&1\\1&0} &
		\sigma_y &= \mm{0&-\I\\\I&0} &
		\sigma_z &= \mm{1&0\\0&-1}.
	\end{align}
	The operator $\op{s}_z$ has two eigenstates
	\begin{align*}
		\uparrow &\equiv \mm{1\\0} & \downarrow &\equiv \mm{0\\1}
	\end{align*}
	which are called \newterm{spin-up} and \newterm{spin-down}, respectively.

	One way to incorporate spin into our present treatment
	is the \newterm{spinor} formalism,
	where a one-particle function is now written as a function of two
	\newterm{spin component}s
	\begin{equation}
		\psi(\vec{r}) \equiv \mm{\psi^\alpha(\vec{r}) \\ \psi^\beta(\vec{r})}
		\label{eqn:OneParticleSpinor}
	\end{equation}
	mapping each real coordinate $\vec{r}$ to a complex spinor from $\C^2$.
	All results from functional analysis and spectral theory
	which we derived for the spin-free case can be adapted to the
	spinor formalism,
	simply moving from the function space $L^2(\R^3, \C)$
	(and its subspaces) to $L^2(\R^3, \C^2)$ (and equivalent subspaces).
	For example for two spin-adapted one-particle functions
	\begin{align*}
		\psi &\equiv \smallmm{\psi^\alpha\\ \psi^\beta} \in L^2(\R^3, \C^2)
		&&\text{and} &
		\varphi &\equiv \smallmm{\varphi^\alpha\\ \varphi^\beta} \in L^2(\R^3, \C^2)
	\end{align*}
	the one-particle inner product $\braket{\slot}{\slot}$ becomes
	the $L^2(\R^3, \C^2)$ inner product
	\[ \braket{\psi}{\varphi}_1 \equiv
		\int_{\R^3} \braket{\psi(\vec{r})}{\varphi(\vec{r})}_{2} \D \vec{r}
		= \int_{\R^3}
			\cc{\left(\psi^\alpha(\vec{r})\right)} \varphi^\alpha(\vec{r})
			+
			\cc{\left(\psi^\beta(\vec{r})\right)} \varphi^\beta(\vec{r})
		\D \vec{r}
	\]
	in analogy to the spin-free case.
	In a similar fashion one may construct the exterior power
	$\bigwedge_{i=1}^{\Nelec} L^2(\R^3, \C^2)$
	and Slater determinants $\bigwedge_{i=1}^{\Nelec} \psi_i$
	from functions $\psi_i \in H^1(\R^3, \C^2)$ as well.
	Notice that the tensor product nature of the exterior power implies
	\[ \bigwedge_{i=1}^{\Nelec} L^2(\R^3, \C^2) \subset L^2(\R^{3\Nelec}, \C^{2\Nelec}) \]
	and
	\begin{equation}
		\bigwedge_{i=1}^{\Nelec} \psi_i \in \bigwedge_{i=1}^{\Nelec} H^1(\R^3, \C^2)
		\subset H^1(\R^{3\Nelec}, \C^{2\Nelec}).
		\label{eqn:SpinAnsatzSpace}
	\end{equation}
	In this sense the derived expressions
	from sections \ref{sec:ManyBodyTISE} and \ref{sec:FCI}
	can be brought forward to the spin-adapted case with only minor modifications.
	For example, the expression of the \HF energy functional
	$\mathcal{E}^\text{HF}(\Theta)$ can be used exactly as stated in
	\eqref{eqn:HFEnergyFunctional}
	for a tuple
	\[ \Theta \equiv \left(\psi_1, \psi_2, \ldots, \psi_{\Nelec}\right)
		\in \left(H^1(\R^3, \C^2)\right)^{\Nelec} \]
	of spin-adapted one-particle functions as well.
	We only need to understand the gradient
	\[ \nabla \psi_1 \equiv \mm{ \nabla \psi_1^\alpha \\ \nabla \psi_1^\beta } \in \C^6 \]
	as a vector from $\C^6$
	and define the density as
	\begin{align}
		\rho_\Theta(\vec{r}) &= \sum_{i=1}^{\Nelec} \norm{\psi_i(\vec{r})}_2^2
		= \sum_{i=1}^{\Nelec} \abs{\psi_i^\alpha(\vec{r})}^2 + \abs{\psi_i^\beta(\vec{r})}^2 \\
	\intertext{and the one-particle density matrix as}
		\gamma_\Theta(\vec{r}_1,\vec{r}_2)
		&= \sum_{i=1}^{\Nelec} \braket{\psi_i(\vec{r}_1)}{\psi_i(\vec{r}_2)}_2
		= \sum_{i=1}^{\Nelec} \cc{\left(\psi_i^\alpha\right)}\!\!(\vec{r}_1)\  \psi_i^\alpha(\vec{r}_2)
			+ \cc{\left(\psi_i^\beta\right)}\!\!(\vec{r}_1)\  \psi_i^\beta(\vec{r}_2).
	\end{align}

	The extra spin component introduces another complication
	into the \HF procedure.
	Analogously to the one-particle spin operator $\op{\vec{s}}$
	and component operators $\op{s}_x, \op{s}_y, \op{s}_z$
	for a single particle,
	one can define the total spin operator $\Op{\vec{S}}$
	as well as Cartesian spin components $\Op{S}_x, \Op{S}_y, \Op{S}_z$
	for an $\Nelec$-electron system.
	Originating from $\Op{\vec{S}}$ one may define
	\[ \Op{S}^2 = \Op{\vec{S}} \cdot \Op{\vec{S}} = \left(\Op{S}^2_x, \Op{S}^2_y, \Op{S}^2_z\right). \]
	One can show~\cite{Shankar1994} that
	\[ \comm{\Op{S}^2}{\Op{S}_z} = 0, \]
	\ie that the total spin squared operator commutes with $\Op{S}_z$.
	Since the Hamiltonian $\Op{H}_{\Nelec}$ makes no explicit reference
	to any particular spin component,
	we necessarily have
	\begin{equation}
		\comm{\Op{S}_z}{\Op{H}_{\Nelec}} = \comm{\Op{S}^2}{\Op{H}_{\Nelec}} = 0
		\label{eqn:CommutationSpinHamlitonian}
	\end{equation}
	as well.
	This implies\footnote{%
		Since $\Op{A}\Op{B} f = \Op{B}\Op{A} f \Leftrightarrow \comm{A}{B} = 0$.%
	}
	that one is able to find simultaneous eigenfunctions of $\Op{H}_{\Nelec}$,
	$\Op{S}_z$ and $\Op{S}^2$.
	For many applications of electronic structure theory,
	\eg the interpretation of certain spectroscopic results
	or for understanding the aforementioned Stern-Gerlach experiment,
	the determination of simultaneous eigenstates of these three operators at once
	is crucial.
	\eqref{eqn:CommutationSpinHamlitonian}
	naturally implies that it is possible to obtain the ground state
	or ground states of $\Op{H}_{\Nelec}$
	in a way that they are eigenfunctions of the spin operators.
	There is, however, no guarantee that the \HF ansatz
	\eqref{eqn:HFMO} gives rise to a \HF ground state $\Phi_{\Theta^0}$,
	which is an eigenfunction of $\Op{S}_z$ or $\Op{S}^2$~\cite{Fukutome1981,McWeeny1985}.
	In fact it is rather easy to construct Slater determinants,
	which are neither an eigenstate of $\Op{S}_z$ nor of $\Op{S}^2$.
	\todoil{Give an example}

	There are two common approaches to deal with this issue~\cite{McWeeny1985}.
	One is to minimise according to \eqref{eqn:HFMO}
	and then use appropriate projections in order to yield the required
	eigenstates with respect to $\Op{S}_z$ and $\Op{S}^2$.
	We will not consider this approach in this work any further.
	The other ansatz is to impose conditions
	on the ansatz space \eqref{eqn:SpinAnsatzSpace},
	such that the resulting \HF ground states
	are eigenfunctions of $\Op{S}_z$ and $\Op{S}^2$.
	This is what we will discuss when we move to a discretised
	treatment of the \HF ansatz in \vref{rem:HFDiscreteVariational}.
\end{rem}

Even though the Hartree-Fock ansatz was already proposed
by \citet{Fock1930} in \citeyear{Fock1930},
its fundamental mathematical properties were
only rigorously characterised and proved by Lieb~\cite{Lieb1977}
and Lions~\cite{Lions1987,Lions1989} in the 70s and 80s
for the general spin-adapted case.
These are summarised in the following.

\begin{rem}[Invariance under orbital rotations]
	\label{rem:HFInvariance}
	Let
	\[ \Theta = \left(\psi_1, \psi_2, \ldots, \psi_{\Nelec}\right)
		\in \left(H^1(\R^3, \C^2)\right)^{\Nelec} \]
	be a tuple, which satisfies the orthonormality condition
	\begin{equation}
		\forall i,j \in 1,\ldots,\Nelec: \quad \braket{\psi_i}{\psi_j}_1 = \delta_{ij}.
		\label{eqn:OrthonormalityCodition}
	\end{equation}
	One can easily show~\cite{Lions1989},
	that for any unitary matrix $\mat{U} \in \C^{\Nelec\times\Nelec}$ it holds:
	\begin{itemize}
		\item $\Theta' = \Theta \mat{U}$ satisfies \eqref{eqn:OrthonormalityCodition} as well.
		\item $\mathcal{E}^\text{HF}(\Theta \mat{U}) = \mathcal{E}(\mat{U})$
	\end{itemize}
	In other words all properties of \HF can only be stated up to
	a unitary rotation amongst the constituents of the ground state Slater
	determinant $\Phi_0$.
\end{rem}

\begin{thm}[Mathematical properties of \HF]
	\label{thm:MathematicalHF}
	Assume $\Nelec \leq \sum_{A=1}^M Z_A$, \ie a neutral or positively charged chemical system.
	\begin{enumerate}
		\item A minimiser
			\begin{equation}
				\Theta^0 = \left(\psi_1^0, \psi_2^0, \ldots, \psi_{\Nelec}^0 \right)
					\in \left( H^1(\R^3, \C^2) \right)^{\Nelec}
				\label{eqn:HFTheoremMinimiser}
			\end{equation}
			to $\mathcal{E}^\text{HF}$ exists~\cite{Lieb1977},
			\ie the \HF model \eqref{eqn:HFMO} has a ground state.
		\item Let us define the \newterm{Fock operator}
			\begin{equation}
				\Op{F}_{\Theta^0} = \Op{T} + \Op{V}_0 + \Op{J}_{\Theta^0} + \Op{K}_{\Theta^0}
				\label{eqn:FockOperator}
			\end{equation}
			consisting of the kinetic energy operator $\Op{T}$ and
			the nuclear attraction operator $\Op{V}_0$ as defined in \eqref{eqn:HCore}
			as well as the effective \newterm{Coulomb operator}
			\begin{align}
				\label{eqn:OperatorCoulomb}
					\Op{J}_{\Theta^0} &= \int_{\R^3} \frac{\rho_{\Theta^0}(\vec{r}_2)}
						{\norm{\slot - \vec{r}_2}_2} \D \vec{r}_2 \\
				\intertext{and the \newterm{exchange operator}, implicitly defined by}
					\left( \Op{K}_{\Theta^0} \chi \right)(\vec{r})
					&= - \int_{\R^3} \frac{\gamma_{\Theta^0}(\vec{r}, \vec{r}_2)}
						{\norm{\vec{r} - \vec{r}_2}_2} \chi(\vec{r}_2) \D \vec{r}_2.
				\label{eqn:OperatorExchange}
			\end{align}

			To find the minimiser of $\mathcal{E}^\text{HF}$ one needs
			to solve the Euler-Lagrange equations corresponding to the minimisation
			problem \eqref{eqn:HFMO}.
			They state, that $\Theta^0$ as defined in \eqref{eqn:HFTheoremMinimiser}
			is a minimiser to $\mathcal{E}^\text{HF}$ if and only if
			there exists a Hermitian matrix $\mat{\lambda} \in \C^{\Nelec\times\Nelec}$
			such that for all $i,j \in \{1,\ldots,\Nelec\}$
			\begin{align}
				\label{eqn:EulerLagrangeHF}
				\Op{F}_{\Theta^0} \psi_i^0 &= \sum_{ij} \lambda_{ij} \psi_j^0
				&&\text{and}&
				\braket{\psi_i^0}{\psi_j^0} &= \delta_{ij}
			\end{align}
			hold.

			Once we found the ground state, the application of the Fock operator
			will thus only rotate us around the space spanned by the minimising
			functions from $\Theta^0$.
		\item Due to the elliptic regularity theorem~\cite{Lieb1977}
			\[ \psi_i^0 \in H^2(\R^3,\C^2) \cap C^\infty(\R^3 \backslash \{\vec{R}_A\}_{A=1,\ldots,M},\C^2), \]
			which implies that a solution to \eqref{eqn:EulerLagrangeHF}
			will always be sooth everywhere but the nuclei and globally in $H^2(\R^3,\C^2)$.
		\item The Fock operator $\Op{F}_{\Theta^0}$ as defined in \eqref{eqn:FockOperator}
			is a self-adjoint operator on $L^2(\R^3,\C^2)$
			with domain $D(\Op{F}_{\Theta^0}) = H^2(\R^3,\C^2)$
			and form domain $Q(\Op{F}_{\Theta^0}) = H^1(\R^3,\C^2)$~\cite{Lions1987}.
			It is bounded below and $\sigma_\text{ess} =[0, +\infty)$.
		\item Up to replacing $\Theta^0$ by $\Theta^0 \mat{U}$
			for some unitary matrix $\mat{U} \in \C^{\Nelec\times\Nelec}$,
			the canonical \newterm{Hartree-Fock equations} hold
			\begin{align}
				\label{eqn:HFequations}
				\Op{F}_{\Theta^0} \psi_i^0 &= \varepsilon_i \psi_i^0
				&&\text{and}&
				\braket{\psi_i^0}{\psi_j^0} &= \delta_{ij}
			\end{align}
			with
			\[ \varepsilon_1 \leq \varepsilon_2 \leq \cdots \leq \varepsilon_{\Nelec} < 0. \]
		\item The \newterm{Aufbau principle} is satisfied:
			The $\{\varepsilon_1, \ldots,\varepsilon_{\Nelec} \}$
			are the lowest $\Nelec$ eigenvalues of $\Op{F}_{\Theta^0}$.
		\item Let $\varepsilon_{\Nelec+1}$ be the $(\Nelec+1)$-th eigenvalue of $\Op{F}_{\Theta^0}$
			if the Fock matrix has $(\Nelec+1)$ negative eigenvalue (counting multiplicities)
			otherwise set $\varepsilon_{\Nelec+1} = 0$.
			The \textbf{no unfilled-shell} property
			\[ \varepsilon_{\Nelec} < \varepsilon_{\Nelec+1} \]
			is satisfied~\todo{cite Bach, Lieb, Loss, Solovej 1994}.
	\end{enumerate}
\end{thm}
The proofs for these results in the general setting are somewhat involved
and can be found in the cited works.
To sketch the required steps appendix \vref{apx:HFTheoremFinite}
proves some of the results of remark \ref{rem:HFInvariance}
and theorem \ref{thm:MathematicalHF} in a finite-dimensional setting.
Theorem \ref{thm:MathematicalHF} provides the mathematical
justification for the \HF procedure as it is performed
in almost every quantum-chemistry program these days
for neutral or positively charged system.
I am not aware of a mathematical statement making
the guarantee that \HF gives a sensible ground state for negatively charged systems.
In fact one can even show that no solution to the \HF problem \eqref{eqn:HFMO}
exists for negative ions with $\Nelec \geq 2Z_\text{tot} + M$~\cite{Cances2000}.
This holds for example for \ce{H^{2-}}.
Notice that up to today in the infinite-dimensional setting
there is furthermore no uniqueness proof for the solution \eqref{eqn:HFTheoremMinimiser},
not even for the resulting ground state density~\cite{Cances2000}.

\subsection{Discretised Hartree-Fock}
\label{sec:DiscreteHF}
\defineabbr{UHF}{UHF\xspace}{unrestricted Hartree-Fock}
\defineabbr{GUHF}{GUHF\xspace}{generalised unrestricted Hartree-Fock}
\defineabbr{RHF}{RHF\xspace}{restricted Hartree-Fock}
\defineabbr{ROHF}{ROHF\xspace}{restricted open-shell Hartree-Fock}
\nomenclature{$\mat{C}$}{Matrix of occupied molecular orbital coefficients of
	size $2\Nbas \times \Nelec$}
\nomenclature{$\mat{C}^\alpha$}{Matrix of occupied spin-up molecular orbital
	coefficients of size $\Nbas \times \NelecA$}
\nomenclature{$\mat{C}_F$}{Matrix of all molecular orbital coefficients.
	Result of the diagonalisation of $\mat{F}$. 
	Size $2\Nbas \times \Norb$}
\nomenclature{$\mat{C}_F^\alpha$}{Matrix of all spin-up
	molecular orbital coefficients.
	Result of the diagonalisation of $\mat{F}^\alpha$.
	Size $2\Nbas \times \NorbA$}
\nomenclature{$\mat{I}_N$}{Identity matrix in $\C^{N\times N}$}

A central result of theorem \ref{thm:MathematicalHF}
is that the \HF ansatz \eqref{eqn:HFMO}
can be seen as a variational problem towards finding
the best molecular orbitals for a single Slater determinant,
which at the optimal point reduces to
the spectral problem of the Fock operator $\Op{F}$.
This section deals with the discretisation of both
representations of \HF.

\begin{rem}[Discretised \HF variational problem]
	\label{rem:HFDiscreteVariational}
	Assume a one-particle basis $\{\varphi_\mu\}_{\mu\in\Ibas}$
	consisting of $\Nbas = \abs{\Ibas}$
	basis functions taken from a dense subspace of $H^1(\R^3,\C)$.
	The space
	\[
		\mathcal{S}_1 = \spacespan \left\{
			\mm{\varphi_\mu \\ 0 }, \mm{0\\\varphi_\mu}
		\, \middle| \, \mu \in \Ibas \right\}
	\]
	spanned by spin-adapted linear combinations from
	$\{\varphi_\mu\}_{\mu\in\Ibas}$
	is a subspace of $H^1(\R^3,\C^2)$.
	Consequently
	\[ \left\{ \bigwedge_{i=1}^{\Nelec} \psi_i \,\middle|\, \psi_i \in \mathcal{S}_1,
	\braket{\psi_i}{\psi_j}_1 = \delta_{ij} \right\} \subset \mathcal{R}^1_{\Nelec}, \]
	which implies
	\begin{equation}
		E_0 \leq E_0^\text{HF} \leq \tilde{E}_0^\text{HF} = \inf \left\{
		\Ehf(\Theta)
		\, \middle| \,
		\Theta = \left(\psi_1, \ldots, \psi_{\Nelec}\right) \in (\mathcal{S}_1)^{\Nelec},
			\braket{\psi_i}{\psi_j}_1 = \delta_{ij}
		\right\}.
		\label{eqn:DiscreteMinimisation}
	\end{equation}
	Let us denote with
	\[ \bm{\varphi} = \mm{\varphi_1 & \varphi_2 & \cdots & \varphi_{\Nbas}} \]
	the row vector of all basis functions written one after another.
	Then each element $\psi_i$ of the tuple
	\begin{equation}
		\Theta = \left(\psi_1, \ldots, \psi_{\Nelec}\right) \in (\mathcal{S}_1)^{\Nelec}
		\label{eqn:HFTuple}
	\end{equation}
	can be expanded as
	\begin{equation}
		\psi_i = \mm{\psi_i^\alpha\\\psi_i^\beta}
		= \sum_{\mu\in\Ibas} \varphi_\mu \, \mm{
			\tilde{C}^\alpha_{\mu i} \\
			\tilde{C}^\beta_{\mu i}
		}
		= \underbrace{\mm{\bm{\varphi} & 0 \\ 0 & \bm{\varphi}}}_{
			\substack{\text{$2$ rows,} \\ \text{$2\Nbas$ columns}}}
			\mm{%
				\tilde{C}^\alpha_{1i} \\ \vdots \\ \tilde{C}^\alpha_{\Nbas, i} \\[5pt]
				\tilde{C}^\beta_{1i} \\ \vdots \\ \tilde{C}^\beta_{\Nbas, i}
			}.
		\label{eqn:HFDiscretisationAnsatzPhi}
	\end{equation}
	This allows to write
	\begin{align}
		\Theta &= \mm{\bm{\varphi} & 0 \\ 0 & \bm{\varphi}} \mat{C}
		&&\text{where}&
		\mat{C} &= \mm{\tilde{\mat{C}}^\alpha \\ \tilde{\mat{C}}^\beta}
			\in \C^{2 \Nbas \times \Nelec}
		\label{eqn:HFDiscretisationAnsatz}
	\end{align}
	is the \newterm{coefficient matrix},
	which is built by pasting
	the row matrices at the right hand side of \eqref{eqn:HFDiscretisationAnsatz}
	one after another.

	It is not hard to imagine
	that one could insert \eqref{eqn:HFDiscretisationAnsatz}
	into the expression for the \HF energy functional $\Ehf$
	in order to obtain an expression of the \HF energy
	in terms of $\mat{C}$.
	This expression could be minimised
	with respect to the coefficients $\mat{C}$
	in order to obtain an approximate
	\HF ground state energy and a corresponding approximate \HF ground state.
	Even though this could be done
	such a \newterm{generalised unrestricted Hartree-Fock}~(\GUHF)
	procedure is hardly ever performed in practice~\cite{McWeeny1985}.
	The reason is that it suffers
	exactly from the issues raised at the end of remark \vref{rem:Spin},
	namely that the resulting \HF ground state
	is neither an eigenfunction of $\Op{S}^2$ nor $\Op{S}_z$.

	Instead one typically selects a target eigenvalue $M_S$
	of the projected spin operator $\Op{S}_z$ \emph{before}
	performing the \HF procedure.
	From this value and $\Nelec$ one determines two parameters
	$\NelecA$ and $\NelecB$,
	the number of spin-up and the number of spin-down electrons,
	such that
	\begin{align*}
		\NelecA + \NelecB &= \Nelec & M_S = \frac12 \left( \NelecA - \NelecB\right).
	\end{align*}
	One can show~\cite{Szabo1996,Helgaker2013},
	that any Slater determinant $\Phi_\Theta$
	made from a tuple like \eqref{eqn:HFTuple}
	is an eigenfunction of $\Op{S}_z$ with eigenvalue $M_S$
	if it consists of $\NelecA$ single-particle functions with zero $\beta$
	component and $\NelecB$ single-particle functions with zero $\alpha$ component.
	Invoking remark \vref{rem:HFInvariance}
	we can always reorder the single-particle functions
	such that the $\NelecA$ functions with zero $\beta$ component are first
	and the other functions with zero $\alpha$-component follow thereafter
	or
	\begin{align*}
		\psi_i^\beta  &= 0 \ \forall i \in \{1,\ldots,\NelecA\}
		&&\text{and}&
		\psi_i^\alpha  &= 0 \ \forall i \in \{\NelecA+1,\ldots,\Nelec\}.
	\end{align*}
	Applying these conditions
	to the generalised unrestricted Hartree-Fock ansatz \eqref{eqn:HFDiscretisationAnsatz}
	leads to the \newterm{unrestricted Hartree-Fock}~(\UHF) method%
	\footnote{
		Beware that even though the \UHF ansatz is termed \emph{unrestricted}
		it implies a restriction of the search space due to spin symmetry.
		This naming is an unfortunate historic consequence.
	}~\cite{Pople1954}.
	In \UHF the coefficient matrix $\mat{C}$ of \eqref{eqn:HFDiscretisationAnsatz}
	becomes block-diagonal
	\begin{equation}
		\label{eqn:HFCoeffMatrix}
		\mat{C} = \mm{\mat{C}^\alpha & 0 \\
				0 & \mat{C}^\beta}
			\in \C^{2\Nbas \times \Nelec},
	\end{equation}
	where $\mat{C}^\alpha \in \C^{\Nbas \times \NelecA}$
	and $\mat{C}^\beta \in \C^{\Nbas \times \NelecB}$
	are the spin-up and spin-down occupied coefficient matrices, respectively.
	Inserting \eqref{eqn:HFDiscretisationAnsatz} and \eqref{eqn:HFCoeffMatrix}
	into \eqref{eqn:HFEnergyFunctional}
	we obtain the \HF energy functional in terms of
	of the coefficient matrix $\mat{C}$
	\begin{equation}
		\Ehf_C(\mat{C})
			= \tr \Big( \tp{\mat{C}} \left(\mat{T} + \mat{V}_0 \right) \mat{C} \Big)
			+ \tr \Big( \tp{\mat{C}} \left(\matJfull + \matKfull \right) \mat{C} \Big),
		\label{eqn:HFEnergyFunctionalCoeff}
	\end{equation}
	where all involved matrices are $\alpha$-$\beta$ block-diagonal,
	just like the coefficient matrix \eqref{eqn:HFCoeffMatrix}.
	Furthermore
	\begin{itemize}
		\item the \newterm{kinetic energy matrix}
			$\mat{T}$ has identical $\alpha$ and $\beta$ blocks
			with elements
			\begin{equation}
				T^\alpha_{\mu\nu} = T^\beta_{\mu\nu}
				= \frac12 \int_{\R^3} \cc{\left(\nabla \varphi_\mu\right)}
					\cdot \nabla \varphi_\nu \D\vec{r}.
				\label{eqn:Tbas}
			\end{equation}
		\item the \newterm{nuclear attraction matrix} $\mat{V}_0$
			has identical $\alpha$ and $\beta$ blocks of elements
			\begin{equation}
				\left(V^\alpha_0\right)_{\mu\nu} = \left(V^\beta_0\right)_{\mu\nu}
				= \int_{\R^3} \sum_{A=1}^M Z_A
				\frac{\cc{\varphi_\mu(\vec{r})} \varphi_\nu(\vec{r})}{\norm{\vec{r} - \vec{R}_A}_2}
				\D \vec{r}.
				\label{eqn:Vbas}
			\end{equation}
		\item the \newterm{Coulomb matrix} $\matJfull$
			depends explicitly on the coefficient matrix $\mat{C}$
			as expressed by the term in the square brackets.
			It has an identical $\alpha$ and $\beta$ block with elements
			\begin{equation}
				J_{\mu\nu}^\alpha\!\left[\mat{C}\mat{C}^\dagger\right]
					= J_{\mu\nu}^\beta\!\left[\mat{C}\mat{C}^\dagger\right]
					= \sum_{\sigma\in\{\alpha\beta\}}
						\sum_{i=1}^{N_\text{elec}^\sigma}
						C^\sigma_{\lambda i}
						\cc{\left(C^\sigma_{\kappa i}\right)}
						\eriMu{\varphi_{\mu} \varphi_{\nu}}
							{\varphi_{\kappa} \varphi_{\lambda}}.
				\label{eqn:Jbas}
			\end{equation}
			Here as usual $\eriMu{\slot \slot}{\slot \slot}$
			denotes the electron repulsion integrals defined like \eqref{eqn:ERI}.
		\item the \newterm{Exchange matrix} $\matKfull$ has
			deviating $\alpha$ and $\beta$ blocks,
			both depending on the coefficients.
			For $\sigma \in \{\alpha\beta\}$ their elements are
			\begin{equation}
				K_{\mu\nu}^\sigma\!\left[\mat{C}\mat{C}^\dagger\right]
					= \sum_{i=1}^{N_\text{elec}^\sigma}
						C^\sigma_{\lambda i}
						\cc{\left(C^\sigma_{\kappa i}\right)}
						\eriMu{\varphi_{\kappa} \varphi_{\nu}}
							{\varphi_{\mu} \varphi_{\lambda}}.
				\label{eqn:Kbas}
			\end{equation}
	\end{itemize}
	Let us further define a block-diagonal \newterm{overlap matrix}
	\[ \mat{S} = \mm{ \mat{S}^\alpha & 0 \\ 0 & \mat{S}^\beta} \]
	with elements
	\begin{equation}
		S^\alpha_{\mu\nu} = S^\beta_{\mu\nu} = \int_{\R^3} \cc{\varphi}_\mu(\vec{r})
			\varphi_{\nu}(\vec{r})\D\vec{r}.
		\label{eqn:Sbas}
	\end{equation}
	Alltogether definitions \eqref{eqn:HFEnergyFunctionalCoeff}
	to \eqref{eqn:Sbas} allow to rewrite \eqref{eqn:DiscreteMinimisation}
	as an optimisation problem with respect to the coefficients $\mat{C}$
	\begin{equation}
		\tilde{E}_0^\text{HF} = \inf \left\{
			\Ehf_C(\mat{C}) \, \middle| \, \mat{C} \in \mathcal{C}
		\right\}
		\label{eqn:HFOptCoeff}
	\end{equation}
	where
	\begin{equation}
		\mathcal{C} = \left\{
			\mm{\mat{C}^\alpha & 0 \\ 0 & \mat{C}^\beta}
			\, \middle| \,
			\mat{C}^\alpha \in \C^{\Nbas\times\NelecA},
			\mat{C}^\beta \in \C^{\Nbas\times\NelecB},
			\tp{\mat{C}} \mat{S} \mat{C} = \mat{I}_{\Nelec}
			\right\}.
		\label{eqn:HFStiefel}
	\end{equation}
	One can show~\cite{Cances2000} in the discrete case,
	that the minimiser $\mat{C}_0$ is even unique
	up to unitary rotations, \ie up to multiplications with unitary matrices
	$\mat{U}^\alpha \in \C^{\NelecA\times\NelecA}$,
	$\mat{U}^\beta \in \C^{\NelecB\times\NelecB}$ with
	\[ \mm{\mat{C}^\alpha & 0 \\ 0 & \mat{C}^\beta}
		\to \mm{\mat{C}^\alpha\mat{U}^\alpha & 0 \\ 0 & \mat{C}^\beta\mat{U}^\beta} \]

	\noindent
	In many cases an alternative formulation of \eqref{eqn:HFOptCoeff} in terms
	of the \newterm{density matrix}
	\begin{align}
		\label{eqn:Dbas}
		\mat{D} &= \mat{C} \tp{\mat{C}} \in \C^{2\Nbas\times2\Nbas}
		&&\text{where}&
		\mat{D}^\alpha &= \mat{C}^\alpha\tp{\left(\mat{C}^\alpha\right)}, &
		\mat{D}^\beta  &= \mat{C}^\beta \tp{\left(\mat{C}^\beta \right)},
	\end{align}
	is desirable.
	The coefficient matrices from $\mathcal{C}$ span
	\begin{equation}
		\mathcal{P} = \left\{
			\mm{\mat{D}^\alpha & 0 \\ 0 & \mat{D}^\beta}
			\, \middle| \,
			\forall \sigma \in \{\alpha, \beta\} \,
			\mat{D}^\sigma \in \C^{\Nbas\times\Nbas},
			\tr\left(\mat{D}^\sigma\right) = N_\text{elec}^\sigma,
			\left(\mat{D}^\sigma\right)^2 = \mat{D}^\sigma
		\right\}.
		\label{eqn:HFGrassmann}
	\end{equation}
	With these definitions we can recast \eqref{eqn:HFOptCoeff} as
	\begin{equation}
		\tilde{E}_0^\text{HF} = \inf \left\{
			\Ehf_D(\mat{D})
				\, \middle| \,
				\mat{D} \in \mathcal{P}
		\right\},
		\label{eqn:HFOptDens}
	\end{equation}
	where the energy functional in terms of the density matrix is
	\begin{equation}
		\Ehf_D(\mat{D})
			= \tr \big( \left(\mat{T} + \mat{V}_0 \right) \mat{D} \big)
			+ \tr \big( \left(\matJfullD + \matKfullD \right) \mat{D} \big).
		\label{eqn:HFEnergyFunctionalDens}
	\end{equation}
	The respective expressions for $\matJfullD$ and $\matKfullD$
	can be obtained from \eqref{eqn:Jbas} and \eqref{eqn:Kbas}
	by replacing
	\begin{align}
		\label{eqn:DensityReplacement}
		\mat{C}\tp{\mat{C}} &\to \mat{D}
		&&\text{and}&
	\sum_{i=1}^{N_\text{elec}^\sigma}
	C^\sigma_{\lambda i}
	\cc{\left(C^\sigma_{\kappa i}\right)} &\to D^\sigma_{\lambda \kappa}.
	\end{align}
	Notice, that the first trace term of $\Ehf_D(\mat{D})$
	is linear in the density matrix, whereas the second trace term
	is quadratic in the density matrix.
	Again the minimiser $\mat{D}_0$ of \eqref{eqn:HFOptDens}
	is unique~\cite{Cances2000} if the Aufbau principle ordering of orbital
	energies is chosen when building $\mat{D}$ from $\mat{C}$.
\end{rem}

\nomenclature{$\norm{\slot}_\text{frob}$}{Frobenius norm of a matrix,
which is the square root of the sum of all elements squared.}
\begin{rem}
	All matrices arising from a discretisation of the \HF ansatz \eqref{eqn:HFMO}
	in the sense of \UHF give rise to block-diagonal matrices,
	with the $\alpha$-block describing the
	spin-up component
	and the $\beta$-block describing the spin-down component.
	Apart from the exchange matrix $\matK$,
	the density matrix $\mat{D}$ as well as the coefficient matrix $\mat{C}$
	all matrices arising in remark \ref{rem:HFDiscreteVariational}
	have identical entries in both blocks.
	Inside the \HF energy functional
	it is the exchange term $\tr \left(\tp{\mat{C}} \matK \mat{C}\right)$
	where the $\alpha$ and $\beta$ block lead to non-symmetrical
	energy contributions.
	In a minimisation it is thus this term, which distinguishes
	spin-up and spin-down electrons and makes them become subject to deviating physics.
	In other words this term gives rise to the non-classical effects
	inside the Hartree-Fock approximation.

	The \UHF procedure automatically assures that the minimiser $\mat{C}_0$
	gives rise to a Slater determinant,
	which is an eigenfunctions of $\Op{S}_z$.
	It is not assured, however, that it is an eigenfunction of $\Op{S}^2$.
	In fact one can show~\cite{Szabo1996},
	that the value obtained for the total spin squared
	for the discretised \HF ground state $\Phi_0$ is
	\begin{equation}
		\label{eqn:UHFS2}
		S^2 = \braket{\Phi_0}{\Op{S}^2 \Phi_0}
		=
		S^2_\text{exact} + \NelecB -
		\norm{
		\tp{\left(\mat{C}^\alpha\right)} \mat{S}^\beta \mat{C}^\beta
		}_\text{frob}^2
	\end{equation}
	where
	\[ S^2_\text{exact}
		= \left(\frac{\NelecA - \NelecB}{2}\right)
			\left(\frac{\NelecA - \NelecB}{2} +1\right).
	\]
	The observed deviation typically becomes larger if the basis gets larger.

	The mathematical structure of the minimisation problems
	\eqref{eqn:HFOptCoeff} and \eqref{eqn:HFOptDens} are comparatively complex.
	One reason for this is that the spaces $\mathcal{C}$ and $\mathcal{P}$,
	spanned by the coefficent or the density matrix parameters sets,
	are not vector spaces.
	Much rather they are \textbf{manifolds},
	\ie geometrical objects which locally look like vector spaces,
	but globally show less structure.
	More precisely $\mathcal{C}$ is a subset of a Stiefel manifold
	and $\mathcal{P}$ is a subset of a Grassmann manifold.
	This aspect becomes apparent
	when designing rigorous algorithms for solving the \HF problem
	since the
	topological properties of the \HF parameter spaces
	imply that intuitive approaches to the problem may not always work.
\end{rem}

\begin{rem}[Discretised \HF equations]
Theorem \vref{thm:MathematicalHF} allows to recast
the \HF ansatz \eqref{eqn:HFMO} into an equivalent
spectral problem \eqref{eqn:HFequations}
for the Fock operator $\Op{F}$ at the minimal point.
It guarantees further that $\Op{F}$
shows the spectral requirements for applying the Ritz-Galerkin ansatz
of remark \vref{rem:DiscreteFormulation},
namely a discrete spectrum below the essential spectrum.
Choosing the same basis $\{\varphi_\mu\}_{\mu\in\Ibas}$
as in remark \ref{rem:HFDiscreteVariational}
and projecting problem \eqref{eqn:HFequations} onto this basis
yields the discretised \HF equations
\begin{equation}
	\forall i,j \in \{1,\ldots,\Nelec\} \quad
	\left\{
	\begin{aligned}
	\, \matFfull \vec{c}_i &= \varepsilon_i \mat{S} \vec{c}_i \\
	   \tp{\vec{c}}_i \vec{c}_j &= \delta_{ij}
	\end{aligned}
	\right. ,
	\label{eqn:HFDiscreteEquations}
\end{equation}
where the elements of the \newterm{Fock matrix} $\matFfull$
are computed by applying the sesquilinear form
\[
	a(\phi,\chi)
		= \int_{\R^3} \braket{\phi(\vec{r})}{ \left(\Op{F}_\Theta \chi\right)\!(\vec{r})}_2
			\D \vec{r}
		= \int_{\R^3}
			\cc{\left(\phi^\alpha(\vec{r})\right)}
				\left(\Op{F}_\Theta\chi^\alpha\right)\!(\vec{r})
			+ \cc{\left(\phi^\beta(\vec{r})\right)}
				\left(\Op{F}_\Theta\chi^\beta\right)\!(\vec{r})
\]
to all pairs of basis spinors $(\phi, \chi)$ with
\[ \phi, \chi \in \left\{ \mm{\varphi\\ 0}, \mm{0\\ \varphi} \,\middle|\,
	\varphi \in \{\varphi_\mu\}_{\mu\in\Ibas} \right\}.
\]
This matrix is Hermitian and block-diagonal%
\footnote{This is true even in the case of generalised unrestricted Hartree-Fock.}
\[ \matF = \mm{\mat{F}^\alpha & 0 \\ 0 & \mat{F}^\beta} \]
and can be written as
\begin{equation}
	\matFfull = \mat{T} + \mat{V}_0 + \matJfull + \matKfull \in \C^{2\Nbas\times 2\Nbas},
	\label{eqn:FockMatrix}
\end{equation}
where the matrix terms are defined as \eqref{eqn:Tbas} to \eqref{eqn:Kbas}.

The generalised eigenvalue problem \eqref{eqn:HFDiscreteEquations}
can be solved for up $\Norb \leq \Nbas$ eigenpairs
$(\varepsilon_i, \vec{c}_i)$
using one of the algorithms described in section \vref{sec:DiagAlgos}
incorporating the modifications discussed in
section \ref{sec:GeneralisedEigenvalueProblem}.
Let us assume the usual ordering
\[ \varepsilon_1 \leq \varepsilon_2 \leq \cdots \leq \varepsilon_{\Nelec}. \]
By the Aufbau principle of theorem \ref{thm:MathematicalHF}
the coefficient matrix is
\[ \mat{C} = \mm{ \vec{c}_1 & \vec{c}_2 & \cdots & \vec{c}_{\Nelec}}, \]
\ie the first $\Nelec$ eigenvectors pasted together.
In analogy we define a full coefficient matrix
\begin{equation}
	\mat{C}_F = \mm{ \vec{c}_1 & \vec{c}_2 & \cdots & \vec{c}_{\Nelec} & \cdots & \vec{c}_{\Norb}},
	\label{eqn:DefinitionCoefficient}
\end{equation}
which contains all eigenvectors we solved \eqref{eqn:HFDiscreteEquations} for.

Applying the spin restrictions of unrestricted \HF
exactly as in remark \ref{rem:HFDiscreteVariational},
we know that that we expect $\NelecA$ eigenvectors
with only $\alpha$ components and $\NelecB$ eigenvectors with only $\beta$ components.
Let us take $\Norb$ to be even and such that $\NelecA, \NelecB > \Norb/2$.
Since both $\mat{F}$ and $\mat{S}$ are block-diagonal
we can solve \eqref{eqn:HFDiscreteEquations} blockwise,
\ie we solve
\[
	\mat{F}^\sigma\!\!\left[\mat{C}\tp{\mat{C}}\right] \vec{c}_i^\sigma
	= \varepsilon_i \mat{S}^\alpha \vec{c}_i^\sigma
\]
for $\sigma = \alpha$ and $\sigma = \beta$
separately for $\Norb/2$ eigenpairs each.
In analogy to \eqref{eqn:DefinitionCoefficient} we proceed to define
\begin{align*}
	 \mat{C}_F^\sigma &= \mm{ \vec{c}_1^\sigma & \vec{c}_2^\sigma &
		\cdots & \vec{c}_{\Norb}^\sigma} \\
\intertext{and consequently get a block-diagonal full coefficients matrix}
	\mat{C}_F &= \mm{ \mat{C}_F^\alpha & 0 \\ 0 & \mat{C}_F^\beta }
		\in \C^{2\Nbas\times \Norb}.
\end{align*}
To adhere with the restriction to $\NelecA$ spin-up
and $\NelecB$ spin-down orbitals
the \newterm{occupied coefficient matrix}
$\mat{C}^\alpha$ is obtained as the first $\NelecA$ columns of $\mat{C}_F^\alpha$
and likewise $\mat{C}^\beta$ as the first $\NelecB$ columns of $\mat{C}_F^\beta$.

The eigenfunctions $\vec{c}_i$ of $\matF$ are called \newterm{HF orbitals}
or \newterm{SCF orbitals}%
\footnote{The term \SCF will defined in the next remark.}.
Those orbitals, which are part of $\mat{C}$ according to the Aufbau principle,
are called \newterm{occupied orbitals}
as they are in some sense occupied by electrons.
Conversely all other orbitals are called unoccupied or \newterm{virtual orbitals}.
For later convenience let us define the index sets
\begin{align*}
	\Iorb &\equiv \{1, \ldots, \Norb \} \\
	\IoccA &\equiv \{1, \ldots, \NelecA \} &
		\IvirtA &\equiv \{\NelecA + 1, \ldots, \Norb/2 \} \\
	\IoccB &\equiv \{\Norb/2 + 1, \ldots, \Norb \} &
		\IvirtB &\equiv \{\NelecA + \Norb/2 + 1, \ldots, \Norb \} \\
	\Iocc &\equiv  \IoccA \cup \IoccB &
		\Ivirt &\equiv  \Iorb \backslash \Iocc,
\end{align*}
whose meaning should be self-explanatory.
\end{rem}

\begin{rem}[Properties of the discretised HF ansatz]
	\label{rem:PropertiesDiscretised}
By theorem \vref{thm:MathematicalHF}
the minimiser $\mat{C}_0 \in \mathcal{C}$ of \eqref{eqn:HFStiefel}
satisfies exactly the discretised \HF equations \eqref{eqn:HFDiscreteEquations}
such that
\newcommand{\matFzfull}{\mat{F}\!\!\left[\mat{C}_0\tp{\mat{C}_0}\right]}
\[
	\matFzfull \mat{C}_0 = \mat{S} \mat{C}_0 \mat{E}
\]
where
\[
	\mat{E} = \text{diag}(\varepsilon_1, \varepsilon_2, \ldots,
		\varepsilon_{\Nelec}).
\]
This condition can be equivalently expressed as~\cite{Pulay1982}
\begin{equation}
	\matFzfull \mat{C}_0\tp{\mat{C}_0} \
	\mat{S} - \mat{S} \ \mat{C}_0\tp{\mat{C}_0} \ \matFzfull
	= \mat{0}
	\label{eqn:HFconvergenceCondition}
\end{equation}
and is only satisfied at the minimal point.

Nothing stops us, however, to pick any other $\mat{C}^{(n)} \in \mathcal{C}$
and build the Fock matrix $\matFnfull$
according to \eqref{eqn:FockMatrix}.
We can solve for its eigenpairs, \ie find a matrix $\mat{C}_F^{(n+1)}$
of eigenvectors and corresponding eigenvalues
\[
	\mat{E}^{(n+1)}
	= \text{diag}\left(\varepsilon_1^{(n+1)},
	\varepsilon_2^{(n+1)}, \ldots,
	\varepsilon_{\Norb}^{(n+1)}\right).
\]
such that
\begin{equation}
	\matFnfull \mat{C}_F^{(n+1)} = \mat{S} \mat{C}_F^{(n+1)} \mat{E}^{(n+1)}.
	\label{eqn:HFiterated}
\end{equation}
In such a case, we can in general not expect the expression
\begin{equation}
	\mat{e}^{(n)}
	= \matFnfull \! \mat{C}^{(n)}\!\tp{\left(\mat{C}^{(n)}\right)} \, \mat{S}
	- \mat{S} \, \mat{C}^{(n)}\!\tp{\left(\mat{C}^{(n)}\right)} \! \matFnfull
	\label{eqn:PulayError}
\end{equation}
to be zero.
Typically, however, $\mat{C}^{(n+1)}$ is
much closer to the minimum $\mat{C}_0$ than $\mat{C}^{(n)}$ was,
which suggests an iterative approach to find the minimum $\mat{C}$
starting from a guess $\mat{C}^{(0)}$.
This approach is called \newterm{self-consistent field}~(\SCF) procedure
and we will discuss it in more detail in the next chapter.
In the context of such an iterative approach
to solve \eqref{eqn:HFDiscreteEquations}
expression \eqref{eqn:PulayError} is called the \newterm{Pulay error}
after Peter Pulay~\cite{Pulay1982}.

There are two more aspects of theorem \ref{thm:MathematicalHF} worth pointing out.
From the unfilled shell property (7.) $\varepsilon_{\Nelec} < \varepsilon_{\Nelec+1}$,
we can deduce that there is always a gap between the
highest occupied orbital~(HOMO) and the lowest unoccupied orbital~(LUMO)
in a converged \HF result.
Together with (5.), we see that only eigenfunctions $\psi_i$
with negative orbital energy eigenvalue $\varepsilon_i$ may be part of $\mat{C}$
in the converged case.
Secondly the smoothness property (3.) makes sure that
all solutions of the \HF equations are numerically easy to model,
such that even numerical basis functions like finite elements
should be able to model the \HF equations with good convergence
provided that appropriate care is taken with respect to the positions of the nuclei.
\end{rem}

In this section we discussed roughly three ways to view the \HF
problem in the discrete setting of a finite basis set $\{\varphi_i\}_{i\in\Ibas}$.
Firstly as a minimisation on a Stiefel manifold \eqref{eqn:HFOptCoeff},
where the energy is optimised with respect to the occupied orbital coefficients
$\mat{C}$.
The second option is to minimise on a Grassmann manifold \eqref{eqn:HFOptDens}
and optimise the energy with respect to the density matrix $\mat{D}$.
The last option is to solve the non-linear \HF equations
\eqref{eqn:HFDiscreteEquations}
in a self-consistent field approach,
which would construct a sequence $\mat{C}^{(n)}$
of orbital coefficients or of density matrices $\mat{D}^{(n)}$
until \eqref{eqn:HFDiscreteEquations} is satisfied within a certain error.
In section \vref{sec:SCFAlgorithms} we will discuss algorithms
for all these approaches.

\subsection{Restricted Hartree-Fock}
By the means of equation \eqref{eqn:UHFS2}
we already stated
that the unrestricted \HF ansatz does not always
yield an \HF ground state $\Psi_0$,
which is an eigenfunction of $\Op{S}^2$.
Following our discussion in remark \vref{rem:Spin}
we could fix this either by projection onto the space of eigenstates
of $\Op{S}^2$ or by imposing extra conditions on the \HF ansatz.

In the case of closed-shell chemical systems the latter is in fact rather simple.
For closed-shell atoms $\NelecA = \NelecB$,
which implies that \eqref{eqn:UHFS2}
gives the correct value $S^2_\text{exact}$ if we enforce $\mat{C}^\alpha = \mat{C}^\beta$.
The condition $\mat{C}^\alpha = \mat{C}^\beta$ not only yields
an eigenfunction of  $\Op{S}^2$,
but furthermore implies that $\mat{F}^\alpha = \mat{F}^\beta$ as well.
In other words in this
\newterm{restricted Hartree-Fock}~(\RHF)~\cite{Roothaan1951}
ansatz one only needs to solve one block of the Fock matrix
in \eqref{eqn:HFDiscreteEquations} and may use the result for both
spin-up and spin-down functions.

Enforcing the correct $S^2$ value \eqref{eqn:UHFS2}
for open-shell electronic systems
with%
\footnote{By convention there are always more spin-up than spin-down electrons}
$\NelecA > \NelecB$
is considerably more involved.
A first approach was published by \citeauthor{Roothaan1960}
~\cite{Roothaan1960,Alder1963}.
In this celebrated work he
distinguishes doubly occupied singly occupied and virtual orbitals.
He then replaces the
block-diagonal Fock matrix from \eqref{eqn:HFDiscreteEquations}
by a specially crafted Fock matrix
consisting of nine blocks,
each block modelling the interaction between two of the aforementioned
orbital subspaces.
Each of the block is a linear combination
of certain parts of $\mat{F}^\alpha$ and $\mat{F}^\beta$,
made in such a way to ensure,
that the resulting SCF minimum is an eigenfunction of $S^2$
with exactly the desired eigenvalue.
We do not consider this approach further in this thesis.
Instead we will treat all open-shell systems with the \UHF procedure,
since \UHF is simpler to implement, computationally cheaper
and overall more widespread.
Nevertheless some issues,
which appeared in our convergence analysis of our
Coulomb-Sturmian based Hartree-Fock ansatz~(see chapter \vref{ch:CSQChem})
could be appointed to \UHF.
We will discuss this further in section \vref{sec:IssuesUHF}.

\subsection{Real-valued Hartree-Fock}
Our discussion of Hartree-Fock up to this point
leads to numerical problems taking place in complex arithmetic,
since both the coefficient matrix $\mat{C}$
as well as the Fock matrix $\mat{F}$ are complex-valued.
Whilst doing this is possible,
it is typically nevertheless
avoided by reducing
the \HF problem to a problem of equivalent structure,
but situated in real Hilbert spaces%
\footnote{Notice, that the approach we will sketch in the following
is not restricted to \HF, but can be applied to many problems of
quantum physics in order to yield equivalent real-valued problems.}.
The major motivation for this is that computations
amongst complex numbers are slower since effectively
more floating point operations need to be performed
in order to treat the real and the imaginary part as required.

\noindent
Let $(\varepsilon_i, \psi_i) \in \R \times H^2(\R^3, \C^2)$ be an eigenpair
of the \HF problem, \ie
\begin{equation}
	 0 = \left(\Op{F}_{\Theta^0} - \varepsilon_i \right) \psi_i
	\label{eqn:FockRealStep0}
\end{equation}
Choosing appropriate functions $\psi_i^R, \psi_i^I \in H^2(\R^3, \R^2)$
we can write $\psi_i = \psi_i^R + \I \psi_i^I$ such that
by linearity of the Fock operator
\begin{equation}
0 = \left(\Op{F}_{\Theta^0} - \varepsilon_i \right) \psi_i^R
	+ \I \left(\Op{F}_{\Theta^0} - \varepsilon_i \right) \psi_i^I.
	\label{eqn:FockRealStep1}
\end{equation}
For \eqref{eqn:FockRealStep1} to be satisfied,
we need the real and the complex part of the right hand side
to be equal to the zero function.

Since all terms of the Fock operator only contain real factors
or real differential operators,
it is clear that the Fock operator maps real-valued functions
to real-valued functions.
In other words
\[ \chi \in H^2(\R^3, \R^2) \quad \Rightarrow \quad \left(\Op{F} \chi\right) \in L^2(\R^3, \R^2). \]
This implies that \eqref{eqn:FockRealStep1} is true iff simultaneously
\begin{align}
	0 &= \left(\Op{F}_{\Theta^0} - \varepsilon_i \right) \psi_i^R \\
	0 &= \left(\Op{F}_{\Theta^0} - \varepsilon_i \right) \psi_i^I.
\end{align}
If $\psi_i$ is not already real-valued and thus $\psi_i^I = 0$,
its real part $\psi_i^R$ and its imaginary part $\psi_i^I$ must both be
solutions to the \HF equations as well.
Furthermore $\psi_i^R$ and $\psi_i^I$ are associated to the same
eigenvalue $\varepsilon_i$ as $\psi$.

As a result one can obtain the solutions for the complex-valued problem
\eqref{eqn:FockRealStep0} by only looking for eigenpairs
$(\varepsilon_i, \psi_i) \in \R \times H^2(\R^3, \R^2)$.
This completely avoids the need for complex arithmetic
as $\mat{C}_F$, $\mat{D}$, $\mat{F}$ and all other matrices
we defined previously in this section will only consist of real elements in this case.
Apart from this simplification no extra care needs to be taken,
since the real eigenfunctions corresponding to an eigenvalue $\varepsilon_i$
still span the full complex eigenspace
one would obtain from solving the original complex-valued problem
--- provided that complex coefficients are used.
This makes sure that we neither miss anything nor get
spurious results by using a real-valued ansatz,
thus still get exactly the physical eigenstates we are after.
