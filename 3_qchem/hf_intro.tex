\section{Single-determinant ansatz}
\label{sec:HFIntro}

\nomenclature{$\NelecA, \NelecB$}{Number of $\alpha$/$\beta$ electrons.
Note $\NelecA + \NelecB = \Nelec$ and $\NelecA \ge \NelecB$ by convention.}
\nomenclature{$\Iorb$}{Index set of computed SCF orbitals, typically $\{1, \ldots \Norb\}$}
\nomenclature{$\Norb$}{The number of computed SCF orbitals.
Note, that $\Norb \le \Nbas$.}

\defineabbr{HF}{HF\xspace}{Hartree-Fock}
\nomenclature{$\OpFock, \OpFockFull$}{$1$-electron Fock operator (dependent on its eigenfunctions $\{ \varphi_i \}_{i \in \Iorb}$)}
\nomenclature{$\Op{\rho}$}{Density operator}
\nomenclature{$\IoccA$, $\IoccB$}{Index set of occupied SCF orbitals of $\alpha$ or $\beta$ spin, respectively. Typically $\{1, \ldots, \NelecA\}$ and similar for $\IoccB$.}
\nomenclature{$\epsilonconv$}{Convergence tolerance for an iterative process.
	If the error is below this value, the process should be considered converged.}

\nomenclature{$\Op{K}$}{One-electron exchange operator}
\nomenclature{$\Op{J}$}{Effective Coulomb operator}

In the previous section we noted that even an approximate
solution to the electronic Schrödinger equation \eqref{eqn:ElectronicSchrödinger}
via the full CI ansatz
is hardly feasible.
Even if comparatively small one-electron basis sets
$\{ \varphi_\mu \}_{\mu\in\Ibas} \subset H^1(\R^3,\C)$
are used,
the dimensionality
of the matrix $\mat{A}_\text{FCI}$ becomes simply too large.
In this section we discuss the opposite end of the scale and
only consider one-dimensional subspaces of the form domain
\[ Q(\Op{H}_{\Nelec}) = \bigwedge_{i=1}^{\Nelec} H^1(\R^3, \C) \cap L^2(\R^3, \C) \]
of the electronic Hamiltonian.
Formally by the Courant-Fischer theorem \eqref{thm:CourantFischer}
the ground state electronic energy can be obtained by
a variational minimisation over all subspaces of dimension $1$,
or more formally
\begin{align}
	\label{eqn:GroundStateVariational}
	E_0 &= \inf_{\Psi \in \mathcal{W}_{\Nelec}} \braket{\Psi}{\Op{H}_{\Nelec} \Psi}_{\Nelec} \\
\intertext{where}
	\label{eqn:NormalisedFormDomain}
	\mathcal{W}_{\Nelec} &= \left\{ \Psi \in Q(\Op{H}_{\Nelec})
	\, \middle| \,
	\norm{\Psi}_{L^2(\R^{3 \Nelec}, \C)} = 1 \right\}.
\end{align}
denotes the subspace of all normalised functions from $Q(\Op{H}_{\Nelec})$.
If we restrict the search to run only over the subspace
\begin{equation}
	\mathcal{R}^1_{\Nelec} = \left\{ \bigwedge_{i=1}^{\Nelec} \psi_i
	\, \middle| \,
	\left( \psi_1, \psi_2, \ldots, \psi_{\Nelec} \right) \in H^1(\R^3, \C),
	\braket{\psi_i}{\psi_j}_1 = \delta_{ij}
	\right\}
	\label{eqn:RankOneSubspace}
\end{equation}
of all normalised Slater determinants we do not get the exact
energy much rather we obtain an upper bound
\begin{equation}
	E_0 \leq E_0^\text{HF} = \inf_{\Phi \in \mathcal{R}^1_{\Nelec}}
	\braket{\Phi}{\Op{H}_{\Nelec} \Phi}_{\Nelec}
	\label{eqn:HFDeterminant}
\end{equation}
according to corollary \vref{cor:Convergence}.
The implied procedure
to obtain an approximation to the electronic ground state
by minimising the sesquilinear form of $\Op{H}_{\Nelec}$
over the space spanned by all normalised Slater determinants
is the celebrated \newterm{Hartree-Fock}~(\HF) approximation.
The resulting minimal energy $E_0^\text{HF}$
is the \HF ground state energy
and the corresponding minimising determinant
$\Phi_0$ the \HF ground state.

Notice that $\Phi_0$ is --- by construction ---
the best possible single Slater determinant to
approximate the electronic ground state.
Mathematically speaking the set of Slater determinants
$\mathcal{R}^1_{\Nelec}$
is exactly the set of all elements from $\mathcal{W}_{\Nelec}$,
which are of tensor rank 1.
For this reason one sometimes refers to the Hartree-Fock
ground state $\Phi_0$
as a \textbf{rank-1 approximation} to the exact
electronic ground state.

\begin{rem}[Molecular orbital formulation of \HF]
	Using equations \eqref{eqn:EnergySlaterDeterminant}
	and \eqref{eqn:RankOneSubspace}
	the \HF problem \eqref{eqn:HFDeterminant}
	can be re-formulated in terms of the molecular orbital
	basis $\{\psi_i\}_{i\in\Iorb}$
\end{rem}

The following theorem summarises some mathematical
results regarding \HF.
\begin{thm}
	\todoil{Seek literature for Cances stuff}
\end{thm}
\todoil{Show these results in the discrete case (see MWM example sheet) in appendix.}

% define occupied and virtual

% mention term SCF
% spin generalisation?

\begin{rem}[Spin]
	bla
\end{rem}

% TODO Mention some theoretical result regarding well-posedness,
%      solutions and so on

\begin{rem}[discretisation]
	\todo[inline,caption={}]{
		\begin{itemize}
			\item 
	$\varphi$ \newterm{one-particle basis} for HF
	\item
	$\psi$ HF orbitals or (SCF) orbital basis,
	typically in FCI one uses the HF orbital basis as the one-particle basis functions for FCI
		\end{itemize}
}
\end{rem}


% ----------------------------------ü



\todoil{
In our discussion so far we neglected the spin component
of the wavefunction
$L^2(\R^{3\Nelec}, \C)$ to $L^2(\R^{3\Nelec}, \C^{2\Nelec})$

two-component spinor representation

	Everything works the same.
	Especially if one thinks of $\alpha$-spin orbitals
	as those which only produce non-zero in the first component
	of the $\C^2$ vector and $\beta$-spin as only producing
	values on the second one.
	Then we also have $\Nelec$ orbitals to occupy
	$\Nelec$ electrons, but their spatial part may be the same or different.
}


\todoil{Present explicit expressions for the individual
	terms. Perhaps already introduce the J and K operators
and use the functional form and formalism of Cances}

% Note that we need a more general setting here.
% Discuss how spin is included
% Slater determinants with spin
\todoil{
	Quick note that real is possible: \\
For an atomic system, we can consider the nucleus to be located at the origin
of the coordinate system, which allows us to simplify
the Hamiltonian to \\
Argue that we can do the mathematical background in real function
arithmetic only and still use the results for complex input due to linearity.
$\Psi \in H^2(\R^{3d}, \R)$
}




% Density matrix
% Density operator
% Coefficient matrix

\newterm{Hartree-Fock}
\todo[inline,caption={}]{
	\begin{itemize}
		\item Idea
		\item Discretisation
		\item use Formalism of Cances
		\item Derivation of HF in discretised space
		\item Summary of the numerical task
		\item Summary of the result (SCF orbitals)
		\item Explain properties of the individual terms of Fock operator (locality, ..)
		\item Show what we miss compared to FCI
	\end{itemize}
}

The discretised Hartree-Fock equations in weak form read
\begin{equation}
	\forall i, j \in \Iorb :  \mbra{\varphi_i} \OpFockFull \mket{\varphi_j} = \lambda_j \braket{\varphi_i}{\varphi_j}
	\label{eqn:HF}
\end{equation}

\begin{equation}
	C^\alpha_{\mu i}, C^\beta_{\mu i}
	\label{eqn:SCFalphaBetaCoeffs}
\end{equation}

% Proove the Fock operator (for fixed orbitals) is linear in the function
% it acts upon


