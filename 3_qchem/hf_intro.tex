\section{Single-determinant ansatz}
\label{sec:HFIntro}

\nomenclature{$\NelecA, \NelecB$}{Number of $\alpha$/$\beta$ electrons.
Note $\NelecA + \NelecB = \Nelec$ and $\NelecA \ge \NelecB$ by convention.}
\nomenclature{$\Iorb$}{Index set of computed SCF orbitals, typically $\{1, \ldots \Norb\}$}
\nomenclature{$\Norb$}{The number of computed SCF orbitals.
Note, that $\Norb \le \Nbas$.}

\defineabbr{HF}{HF\xspace}{Hartree-Fock}
\nomenclature{$\OpFock, \OpFockFull$}{$1$-electron Fock operator (dependent on its eigenfunctions $\{ \varphi_i \}_{i \in \Iorb}$)}
\nomenclature{$\Op{\rho}$}{Density operator}
\nomenclature{$\IoccA$, $\IoccB$}{Index set of occupied SCF orbitals of $\alpha$ or $\beta$ spin, respectively. Typically $\{1, \ldots, \NelecA\}$ and similar for $\IoccB$.}
\nomenclature{$\epsilonconv$}{Convergence tolerance for an iterative process.
	If the error is below this value, the process should be considered converged.}

\nomenclature{$\Op{K}$}{One-electron exchange operator}
\nomenclature{$\Op{J}$}{Effective Coulomb operator}

In the previous section we noted that even an approximate
solution to the electronic Schrödinger equation \eqref{eqn:ElectronicSchrödinger}
via the full CI ansatz
is hardly feasible.
Even if comparatively small one-electron basis sets
$\{ \varphi_\mu \}_{\mu\in\Ibas} \subset H^1(\R^3,\C)$
are used,
the dimensionality
of the matrix $\mat{A}_\text{FCI}$ becomes simply too large.
In this section we discuss the opposite end of the scale and
only consider one-dimensional subspaces of the form domain
\[ Q(\Op{H}_{\Nelec}) = \bigwedge_{i=1}^{\Nelec} H^1(\R^3, \C) \cap L^2(\R^3, \C) \]
for solving the electronic problem.
Formally by the Courant-Fischer theorem \eqref{thm:CourantFischer}
the ground state electronic energy $E_0$ can be obtained by
a variational minimisation over all subspaces of dimension $1$.
In other words
\begin{align}
	\label{eqn:GroundStateVariational}
	E_0 &= \inf_{\Psi \in \mathcal{W}_{\Nelec}} \braket{\Psi}{\Op{H}_{\Nelec} \Psi}_{\Nelec} \\
\intertext{where}
	\label{eqn:NormalisedFormDomain}
	\mathcal{W}_{\Nelec} &= \left\{ \Psi \in Q(\Op{H}_{\Nelec})
	\, \middle| \,
	\norm{\Psi}_{L^2(\R^{3 \Nelec}, \C)} = 1 \right\}.
\end{align}
denotes the subspace of all normalised functions from $Q(\Op{H}_{\Nelec})$.
If we restrict the search to only run over the space
\begin{equation}
	\mathcal{R}^1_{\Nelec} = \left\{ \bigwedge_{i=1}^{\Nelec} \psi_i
	\, \middle| \,
	\psi_i \in H^1(\R^3, \C),
	\braket{\psi_i}{\psi_j}_1 = \delta_{ij}
	\quad \forall \, 1\leq i,j\leq \Nelec
	\right\}
	\label{eqn:RankOneSubspace}
\end{equation}
of all normalised Slater determinants,
which is a proper subspace of $\mathcal{W}_{\Nelec}$,
we no longer yield the exact energy.
According to corollary \vref{cor:Convergence} we merely obtain an upper bound
\begin{equation}
	E_0 \leq E_0^\text{HF} = \inf_{\Phi \in \mathcal{R}^1_{\Nelec}}
	\braket{\Phi}{\Op{H}_{\Nelec} \Phi}_{\Nelec}.
	\label{eqn:HFDeterminant}
\end{equation}
The implied procedure,
where an approximation to the electronic ground state
is computed by minimising the sesquilinear form of $\Op{H}_{\Nelec}$
over the space spanned by all normalised Slater determinants,
is the celebrated \newterm{Hartree-Fock}~(\HF) approximation~\cite{Fock1930}.
The resulting minimal energy $E_0^\text{HF}$
is the \HF ground state energy
and the corresponding minimising determinant
$\Phi_0$ the \HF ground state.

Notice, that $\Phi_0$ is --- by construction ---
the best possible single Slater determinant to
approximate the electronic ground state.
Mathematically speaking the set of Slater determinants
$\mathcal{R}^1_{\Nelec}$
is exactly the set of all elements from $\mathcal{W}_{\Nelec}$,
which are of tensor rank 1.
For this reason one sometimes refers to the Hartree-Fock
ground state $\Phi_0$
as a \textbf{rank-1 approximation} to the exact
electronic ground state.

\begin{rem}[Molecular orbital formulation of \HF]
	Let
	\[  \Theta \equiv \left(\psi_1, \psi_2, \ldots, \psi_{\Nelec}\right)
		\in \left(H^1(\R^3, \C)\right)^{\Nelec} \]
	denote an arbitrary tuple of single-particle functions.
	It uniquely defines a Slater determinant
	$\Phi_\Theta = \bigwedge_{i=1}^{\Nelec} \psi_i.$
	Inserting this ansatz into the energy expression
	\eqref{eqn:EnergySlaterDeterminant} for a single determinant
	yields the \newterm{Hartree-Fock energy functional}
	\begin{equation}
		\begin{aligned}
		\mathcal{E}^\text{HF}(\Theta)
		&= \frac12 \sum_{i=1}^{\Nelec} \int_{\R^3} \norm{\nabla \psi_i}_2^2 \D\vec{r}
		+ \int_{\R^3} \sum_{A=1}^M
			\frac{\rho_\Theta(Z_A \, \vec{r})}{\norm{\vec{r} - \vec{R}_A}_2} \D\vec{r} \\
		&\hspace{20pt}
		+ \frac12 \int_{\R^3}\int_{\R^3}
			\frac{\rho_\Theta(\vec{r}_1) \rho_\Theta(\vec{r}_2)}
				{\norm{\vec{r}_1 - \vec{r}_2}_2} \D\vec{r}_1 \D\vec{r}_2
		- \frac12 \int_{\R^3}\int_{\R^3}
			\frac{\abs{\gamma_\Theta(\vec{r}_1, \vec{r}_2)}^2}
				{\norm{\vec{r}_1 - \vec{r}_2}_2} \D\vec{r}_1 \D\vec{r}_2
		\end{aligned}
		\label{eqn:HFEnergyFunctional}
	\end{equation}
	where
	\begin{align}
		\label{eqn:HFDensityNoSpin}
		\rho_\Theta(\vec{r}) &= \sum_{i=1}^{\Nelec} \abs{\psi_i(\vec{r})}^2
		&\text{and} &&
		\gamma_\Theta(\vec{r}_1,\vec{r}_2)
			&= \sum_{i=1}^{\Nelec} \cc{\psi}_i(\vec{r}_1) \psi_i(\vec{r}_2)
	\end{align}
	are the \newterm{electron density}
	and the \newterm{one-particle reduced density matrix}, respectively.
	The HF ansatz \eqref{eqn:RankOneSubspace}
	thus becomes
	\begin{equation}
		E_0 \leq E_0^\text{HF}
		= \inf \left\{
			\mathcal{E}^\text{HF}(\Theta)
			\, \middle| \,
			\Theta \in \left(H^1(\R^3, \C)\right)^{\Nelec}, \,
			\forall i,j \,
			\braket{\psi_i}{\psi_j}_1 = \delta_{ij}.
		\right\}
		\label{eqn:HFMO}
	\end{equation}
	The minimiser $\Theta^0$, \ie the tuple for which the minimum
	energy $E_0^\text{HF} = \mathcal{E}^\text{HF}(\Theta^0)$
	is exactly obtained,
	defines the HF ground state
	$\Phi_0 \equiv \Phi_{\Theta^0}$.
\end{rem}

Before we discuss the mathematical properties of the \HF ansatz,
let us first pick up on our discussion
of spin in section \vref{sec:ElectronicSchrödinger}
and generalise our formalism slightly.

\begin{rem}[Spin-adapted formulation of \HF]
	\label{rem:Spin}
	The mathematical treatment up to this point
	only includes one property resulting from the spin-$\sfrac12$ nature
	of electrons,
	namely the antisymmetry of the wave function.
	The missing property is the explicit inclusion of the spin degree of freedom.
	For a single spin-$\sfrac12$ particle the spin degree of freedom
	by itself spans the two-dimensional Hilbert space $\C^2$,
	which can be probed by the spin operator
	\[ \Op{\vec{S}} \equiv \left( \Op{S}_x, \Op{S}_y, \Op{S}_z \right)
		= \frac12 \left( \sigma_x, \sigma_y, \sigma_z \right). \]
	In this expression we used the \newterm{Pauli matrices} defined as
	\begin{align}
		\label{eqn:PauliMatrices}
		\sigma_x &= \mm{0&1\\1&0} &
		\sigma_y &= \mm{0&-\I\\\I&0} &
		\sigma_z &= \mm{1&0\\0&-1}.
	\end{align}
	The operator $\Op{S}_z$ has two eigenvectors
	\begin{align*}
		\uparrow &\equiv \mm{1\\0} & \downarrow &\equiv \mm{0\\1}
	\end{align*}
	which are called \newterm{spin-up} and \newterm{spin-down}, respectively.

	One way to incorporate this into our present formalism
	is the \newterm{spinor} formalism,
	where a one-particle function is now written as a \newterm{two-component} function
	\begin{equation}
		\psi(\vec{r}) \equiv \mm{\psi^\alpha(\vec{r}) \\ \psi^\beta(\vec{r})}
		\label{eqn:OneParticleSpinor}
	\end{equation}
	mapping each real coordinate $\vec{r}$ to a complex spinor from $\C^2$.
	All results from functional analysis and spectral theory
	which we derived for the spin-free case can be adapted to the
	spinor formalism,
	simply moving from the function space $L^2(\R^3, \C)$
	(and its subspaces) to $L^2(\R^3, \C^2)$.
	For example for two spin-adapted one-particle functions
	\begin{align*}
		\psi &\equiv \smallmm{\psi^\alpha\\ \psi^\beta} \in L^2(\R^3, \C^2)
		&&\text{and} &
		\varphi &\equiv \smallmm{\varphi^\alpha\\ \varphi^\beta} \in L^2(\R^3, \C^2)
	\end{align*}
	the one-particle inner product $\braket{\slot}{\slot}$ becomes
	the $L^2(\R^3, \C^2)$ inner product
	\[ \braket{\psi}{\varphi}_1 \equiv
		\int_{\R^3} \braket{\psi(\vec{r})}{\varphi(\vec{r})}_{2} \D \vec{r}
		= \int_{\R^3}
			\cc{\left(\psi^\alpha\right)}\!\!(\vec{r}) \  \varphi^\alpha(\vec{r})
			+
			\cc{\left(\psi^\beta\right)}\!\!(\vec{r})\  \varphi^\beta(\vec{r})
		\D \vec{r}
	\]
	in analogy to the spin-free case.
	In a similar fashion one may construct the exterior power
	$\bigwedge_{i=1}^{\Nelec} L^2(\R^3, \C^2)$
	and Slater determinants $\bigwedge_{i=1}^{\Nelec} \psi_i$
	from functions $\psi_i \in H^1(\R^3, \C)$.
	Notice that the tensor product nature of the exterior power
	implies
	\[ \bigwedge_{i=1}^{\Nelec} L^2(\R^3, \C^2) \subset L^2(\R^{3\Nelec}, \C^{2\Nelec}) \]
	and
	\[ \bigwedge_{i=1}^{\Nelec} \psi_i \in \bigwedge_{i=1}^{\Nelec} H^1(\R^3, \C^2)
		\subset H^1(\R^{3\Nelec}, \C^{2\Nelec}). \]
	In this sense most of the derivations from this chapter can be brought forward
	to the spin-adapted case as well.
	For example the expression of the \HF energy functional
	$\mathcal{E}^\text{HF}(\Theta)$ can be used exactly as stated in
	\eqref{eqn:HFEnergyFunctional}
	for a tuple
	\[ \Theta \equiv \left(\psi_1, \psi_2, \ldots, \psi_{\Nelec}\right)
		\in \left(H^1(\R^3, \C^2)\right)^{\Nelec} \]
	of spin-adapted one-particle functions as well,
	provided that we understand the gradient
	\[ \nabla \psi_1 \equiv \mm{ \nabla \psi_1^\alpha \\ \nabla \psi_1^\beta } \in \C^6 \]
	as a vector from $\C^6$
	and define the density as
	\begin{align}
		\rho_\Theta(\vec{r}) &= \sum_{i=1}^{\Nelec} \norm{\psi_i(\vec{r})}_2^2
		= \sum_{i=1}^{\Nelec} \abs{\psi_i^\alpha(\vec{r})}^2 + \abs{\psi_i^\beta(\vec{r})}^2 \\
	\intertext{and the one-particle density matrix as}
		\gamma_\Theta(\vec{r}_1,\vec{r}_2)
		&= \sum_{i=1}^{\Nelec} \braket{\psi_i(\vec{r}_1)}{\psi_i(\vec{r}_2)}_2
		= \sum_{i=1}^{\Nelec} \cc{\left(\psi_i^\alpha\right)}\!\!(\vec{r}_1)\  \psi_i^\alpha(\vec{r}_2)
			+ \cc{\left(\psi_i^\beta\right)}\!\!(\vec{r}_1)\  \psi_i^\beta(\vec{r}_2).
	\end{align}
\end{rem}

Even though the Hartree-Fock ansatz was already proposed
by \citet{Fock1930} in \citeyear{Fock1930},
its fundamental mathematical properties were
only rigorously characterised and proved by Lieb~\cite{Lieb1977}
and Lions~\cite{Lions1987,Lions1989} in the 70s and 80s
for the general spin-adapted case.
The are summarised in the following.

\begin{rem}[Invariance under orbital rotations]
	\label{rem:HFInvariance}
	Let
	\[ \Theta = \left(\psi_1, \psi_2, \ldots, \psi_{\Nelec}\right)
		\in \left(H^1(\R^3, \C^2)\right)^{\Nelec} \]
	be a tuple, which satisfies the orthonormality condition
	\begin{equation}
		\forall i,j \in 1,\ldots,\Nelec: \quad \braket{\psi_i}{\psi_j}_1 = \delta_{ij}.
		\label{eqn:OrthonormalityCodition}
	\end{equation}
	One can easily show~\cite{Lions1989},
	that for any unitary matrix $\mat{U} \in \C^{\Nelec\times\Nelec}$ it holds:
	\begin{itemize}
		\item $\Theta' = \Theta \mat{U}$ satisfies \label{eqn:OrthonormalityCodition} as well.
		\item $\mathcal{E}^\text{HF}(\Theta \mat{U}) = \mathcal{E}(\mat{U})$
	\end{itemize}
	In other words all properties of \HF can only be stated up to
	a unitary rotation amongst the constituents of the ground state Slater
	determinant $\Phi_0$.
\end{rem}

\begin{thm}[Mathematical properties of \HF]
	\label{thm:MathematicalHF}
	Assume $\Nelec \leq \sum_{A=1}^M Z_A$, \ie a neutral or positively charged chemical system.
	\begin{enumerate}
		\item A minimiser
			\begin{equation}
				\Theta^0 = \left(\psi_1^0, \psi_2^0, \ldots, \psi_{\Nelec}^0 \right)
					\in \left( H^1(\R^3, \C^2) \right)^{\Nelec}
				\label{eqn:HFTheoremMinimiser}
			\end{equation}
			to $\mathcal{E}^\text{HF}$ exists~\cite{Lieb1977},
			\ie the \HF model \eqref{eqn:HFMO} has a ground state.
		\item Let us define the \newterm{Fock operator}
			\begin{equation}
				\Op{F}_{\Theta^0} = \Op{T} + \Op{V}_0 + \Op{J}_{\Theta^0} + \Op{K}_{\Theta^0}
				\label{eqn:FockOperator}
			\end{equation}
			consisting of the kinetic energy operator $\Op{T}$ and
			the nuclear attraction operator $\Op{V}_0$ as defined in \eqref{eqn:HCore}
			as well as the classical \newterm{Coulomb operator}
			\begin{align}
				\label{eqn:OperatorCoulomb}
					\Op{J}_{\Theta^0} &= \int_{\R^3} \frac{\rho_{\Theta^0}(\vec{r}_2)}
						{\norm{\slot - \vec{r}_2}_2} \D \vec{r}_2 \\
				\intertext{and the \newterm{exchange operator}, implicitly defined by}
					\left( \Op{K}_{\Theta^0} \chi \right)(\vec{r})
					&= - \int_{\R^3} \frac{\gamma_{\Theta^0}(\vec{r}, \vec{r}_2)}
						{\norm{\vec{r} - \vec{r}_2}_2} \chi(\vec{r}_2) \D \vec{r}_2.
				\label{eqn:OperatorExchange}
			\end{align}

			To find the minimiser of $\mathcal{E}^\text{HF}$ one needs
			to solve the Euler-Lagrange equations corresponding to the minimisation
			problem \eqref{eqn:HFMO}.
			They state, that $\Theta^0$ as defined in \eqref{eqn:HFTheoremMinimiser}
			is a minimiser to $\mathcal{E}^\text{HF}$ if and only if
			there exists a Hermitian matrix $\mat{\lambda} \in \C^{\Nelec\times\Nelec}$
			such that for all $i,j \in \{1,\ldots,\Nelec\}$
			\begin{align}
				\label{eqn:EulerLagrangeHF}
				\Op{F}_{\Theta^0} \psi_i^0 &= \sum_{ij} \lambda_{ij} \psi_j^0
				&&\text{and}&
				\braket{\psi_i^0}{\psi_j^0} &= \delta_{ij}
			\end{align}
			hold.

			Once we found the ground state, the application of the Fock operator
			will thus only rotate us around the space spanned by the minimising
			functions from $\Theta^0$.
		\item Due to the elliptic regularity theorem~\cite{Lieb1977}
			\[ \psi_i^0 \in H^2(\R^3,\C^2) \cap C^\infty(\R^3 \backslash \{\vec{R}_A\}_{A=1,\ldots,M},\C^2), \]
			which implies that a solution to \eqref{eqn:EulerLagrangeHF}
			will always be sooth everywhere but the nuclei and globally in $H^2(\R^3,\C^2)$.
		\item The Fock operator $\Op{F}_{\Theta^0}$ as defined in \eqref{eqn:FockOperator}
			is a self-adjoint operator on $L^2(\R^3,\C^2)$
			with domain $D(\Op{F}_{\Theta^0}) = H^2(\R^3,\C^2)$
			and form domain $Q(\Op{F}_{\Theta^0}) = H^1(\R^3,\C^2)$~\cite{Lions1987}.
			It is bounded below and $\sigma_\text{ess} =[0, +\infty)$.
		\item Up to replacing $\Theta^0$ by $\Theta^0 \mat{U}$
			for some unitary matrix $\mat{U} \in \C^{\Nelec\times\Nelec}$,
			the canonical \newterm{Hartree-Fock equations} hold
			\begin{align}
				\label{eqn:HFequations}
				\Op{F}_{\Theta^0} \psi_i^0 = \varepsilon_i \psi_i^0
				&&\text{and}&
				\braket{\psi_i^0}{\psi_j^0} &= \delta_{ij}
			\end{align}
			with
			\[ \varepsilon_1 \leq \varepsilon_2 \leq \cdots \leq \varepsilon_{\Nelec} < 0. \]
		\item The \newterm{Aufbau principle} is satisfied:
			The $\{\varepsilon_1, \ldots,\varepsilon_{\Nelec} \}$
			are the lowest $\Nelec$ eigenvalues of $\Op{F}_{\Theta^0}$.
		\item Let $\varepsilon_{\Nelec+1}$ be the $(\Nelec+1)$-th eigenvalue of $\Op{F}_{\Theta^0}$
			if the Fock matrix has $(\Nelec+1)$ negative eigenvalue (counting multiplicities)
			otherwise set $\varepsilon_{\Nelec+1} = 0$.
			The \textbf{no unfilled-shell} property
			\[ \varepsilon_{\Nelec} < \varepsilon_{\Nelec+1} \]
			is satisfied~\todo{cite Bach, Lieb, Loss, Solovej 1994}.
	\end{enumerate}
\end{thm}
The proofs for these results in the general setting are somewhat involved
and can be found in the cited works.
To sketch the required steps appendix \vref{apx:HFTheoremFinite}
proves some of the results of remark \ref{rem:HFInvariance}
and theorem \ref{thm:MathematicalHF} in a finite-dimensional setting.

We will return to some of the results of theorem \ref{thm:MathematicalHF}
further down this section.
Let us just note, that theorem \ref{thm:MathematicalHF} allows to recast
the \HF ansatz \eqref{eqn:HFMO} into an equivalent spectral problem \eqref{eqn:HFequations}
for the Fock operator.
This problem satisfies all the spectral requirements for applying the Ritz-Galerkin ansatz
of remark \vref{rem:DiscreteFormulation},
such that a Ritz-Galerkin ansatz can be applied.

\newcommand{\Ehf}{\mathcal{E}^\text{HF}}
\begin{rem}[Discretised \HF equations]
	Assume a one-particle basis $\{\varphi_\mu\}_{\mu\in\Ibas}$
	with $L^2(\R^3,\C)$-orthonormal
	basis functions $\varphi_\mu$ taken from a dense subspace of $H^1(\R^3,\C)$.
	The set
	\begin{equation}
		\mathcal{S} = \left\{
		\mm{
			\sum_{\mu\in\Ibas} C^\alpha_{\mu i} \varphi_\mu \\
			\sum_{\mu\in\Ibas} C^\beta_{\mu i} \varphi_\mu
		} \, \middle|\, C^\alpha_{\mu i}, C^\beta_{\mu i} \in \C \right\}
		\label{eqn:HFDiscretisationAnsatz}
	\end{equation}
	spanned by spin-adapted linear combinations from
	$\{\varphi_\mu\}_{\mu\in\Ibas}$
	is a subspace of $H^1(\R^3,\C^2)$.
	Consequently
	\[ \left\{ \bigwedge_{i=1}^{\Nelec} \psi_i \,\middle|\, \psi_i \in \mathcal{S},
	\braket{\psi_i}{\psi_j}_1 = \delta_{ij} \right\} \subset \mathcal{R}^1_{\Nelec}, \]
	which implies
	\[
		E_0 \leq E_0^{HF} \leq \tilde{E}_0^\text{HF} = \inf \left\{
		\mathcal{E}^\text{HF}(\Theta)
		\, \middle| \,
		\Theta = \left(\psi_1, \ldots, \psi_{\Nelec}\right) \in \mathcal{S}^{\Nelec},
			\braket{\psi_i}{\psi_j}_1 = \delta_{ij}
		\right\}.
	\]
	Inserting \eqref{eqn:HFDiscretisationAnsatz} into the \HF energy functional
	\eqref{eqn:HFEnergyFunctional}
	we obtain the energy functional for the discrete case
	\begin{equation}
		\Ehf(\mat{C})
		\label{eqn:HFEnergyFunctionalDiscrete}
	\end{equation}







	\begin{itemize}
			\item 
	$\varphi$ \newterm{one-particle basis} for HF
	\item
	$\psi$ HF orbitals or (SCF) orbital basis,
	typically in FCI one uses the HF orbital basis as the one-particle basis functions for FCI
		\end{itemize}
\end{rem}

% --------------------------------





% define occupied and virtual
With the elliptic regularity at hand we can even expect convergence of such a discretised
scheme to be fast
at all places except the nuclei,
since the solution is infinitely differentiable almost everywhere.


fast, since the functions are smooth everywhere 




% mention term SCF

equivalent problem of diagonalisation
to find MOs from AOs
allows to define concept of occupied and virtual orbitals
No unfilled shell means Homo lumo gap

\todoil{
	Quick note that real is possible: \\
For an atomic system, we can consider the nucleus to be located at the origin
of the coordinate system, which allows us to simplify
the Hamiltonian to \\
Argue that we can do the mathematical background in real function
arithmetic only and still use the results for complex input due to linearity.
$\Psi \in H^2(\R^{3d}, \R)$
}


\newterm{Hartree-Fock}
\todo[inline,caption={}]{
	\begin{itemize}
		\item Summary of the numerical task
		\item Summary of the result (SCF orbitals)
		\item Explain properties of the individual terms of Fock operator (locality, ..)
	\end{itemize}
}

The discretised Hartree-Fock equations in weak form read
\begin{equation}
	\forall i, j \in \Iorb :  \mbra{\varphi_i} \OpFockFull \mket{\varphi_j} = \lambda_j \braket{\varphi_i}{\varphi_j}
	\label{eqn:HF}
\end{equation}

\begin{equation}
	C^\alpha_{\mu i}, C^\beta_{\mu i}
	\label{eqn:SCFalphaBetaCoeffs}
\end{equation}

% Explain the term Hartree-Fock and SCF (idea from Hartree, but spin adaption from Fock and Slater)


% UHF ground state (derived here) no eigenfunction of S^2
