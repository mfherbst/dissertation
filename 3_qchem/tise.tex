\section{Many-body Schrödinger equation}
\label{sec:ManyBodyTISE}
\defineabbr{BO}{BO\xspace}{Born-Oppenheimer approximation, see section \vref{sec:BO}}
\nomenclature{$\Psi$}{State of a (many body) quantum system,
	typically the \emph{exact} ground state solution to the many-electron,
	electronic TISE, see section \vref{sec:BO}}
\nomenclature{$\Nelec$}{Total number of electrons}
\nomenclature{$\Op{H}_{\Nelec}$}{Electronic Hamiltonian
	for a $\Nelec$-electron system, see section \vref{sec:BO}}
\nomenclature{$\vec{x}$}{Vector in $\R^{3 \Nelec}$,
which typically specifies the positions of all electrons of the system.}
\nomenclature{$\vec{r}$}{Vector in $\R^3$, which specifies the position of an electron in space}

Let us consider a chemical system consisting of $M$ nuclei and $\Nelec$ electrons.
We take the nuclei to be located at mass-scaled\footnote{
	If $\vec{\tilde{R}}_A$ is the Cartesian coordinate
	of the $A$-th nucleus with mass $M_A$, then
	the mass-scaled coordinates are given as
	$\vec{R}_A = \sqrt{M_A} \, \vec{\tilde{R}}_A$.
} coordinates
$\{\vec{R}_A\}_{A = 1, 2, \ldots, M} \subset \R^3$
with corresponding charges
$\{Z_A\}_{A = 1, 2, \ldots, M}$.
The electron positions are denoted by the (Cartesian) coordinates
$\{\vec{r}_i\}_{i = 1, 2, \ldots, \Nelec} \subset \R^3$.
Following the correspondence to classical mechanics (see section \vref{sec:IntroQM})
we can construct the many-body Hamiltonian
on the Hilbert space $L^2(\R^L, \C)$ with dimensionality $L = 3M+3\Nelec$ as
\begin{equation}
	\Op{H}^\text{MB} = \Op{T}_e + \Op{T}_n + \Op{V}_{nn} + \Op{V}_{ne} + \Op{V}_{ee}.
	\label{eqn:ManyBodyHamiltonian}
\end{equation}
In this expression we introduced the
nuclear-nuclear, electronic-electronic and nuclear-electronic
Coulombic interaction potentials
\begin{align}
	\label{eqn:ManyBodyHamiltonianPotential}
	\Op{V}_{nn} &= \sum_{A=1}^M \sum_{B=A}^M
		\frac{Z_A Z_B}{\norm{\vec{R}_A - \vec{R}_B}_2} &
	\Op{V}_{ee} &= \sum_{i=1}^{\Nelec} \sum_{j=i+1}^{\Nelec}
		\frac{1}{\norm{\vec{r}_i - \vec{r}_j}_2} \\
	\nonumber
	\Op{V}_{ne} &= \sum_{A=1}^M \sum_{i=0}^{\Nelec} \frac{Z_A}{\norm{\vec{R}_A - \vec{r}_i}_2},
\end{align}
respectively.
Furthermore we used the electronic and nuclear kinetic energy operators
\begin{align}
	\label{eqn:ManyBodyHamiltonianKinetic}
	\Op{T}_e &= \sum_{i=1}^{\Nelec} \Delta_{\vec{r}_i} &
	\Op{T}_n &= \sum_{A=1}^M \Delta_{\vec{R}_A},
\end{align}
with the shorthand
\[ \Delta_{\vec{q}} = \sum_{\alpha=1}^3 \frac{\partial^2}{\partial q_\alpha^2} \]
for the Laplace operator with respect to particle coordinates $\vec{q}$.
If we take the domain $D(\Op{H}_\text{MB}) = H^2(\R^L, \C)$
this operator can be made self-adjoint~\cite{Kato1951}.
It is furthermore bounded below~\cite{Kato1951}
with a couple of discrete states below the essential spectrum.

The operator $\Op{H}_\text{MB}$ is the fundamental object
the field of \newterm{quantum chemistry} investigates.
Its properties allow for a full
(non-relativistic) quantum-mechanical description
of a chemical system.
This includes important properties like stable chemical structures
or reactivity, with respect to other molecules as well as external potentials.
As discussed in section \vref{sec:SpectralTakeAway}
a consequence of the laws of thermodynamics is,
that in many cases one already gets
a reasonable idea about the chemical properties of matter
if only the lowest-energy, discrete eigenstates of the relevant
many-body Hamiltonian $\Op{H}_\text{MB}$ are determined.

Let us use the vectors $\vec{x} \in \R^{3\Nelec}$ and $\vec{X} \in \R^{3M}$, defined as
\begin{align}
	\rtp{\vec{x}} &\equiv \left(\rtp{\vec{r}_1}, \rtp{\vec{r}_2}, \ldots, \rtp{\vec{r}_{\Nelec}}\right)
	&
	\rtp{\vec{X}} &\equiv \left(\rtp{\vec{R}_1}, \rtp{\vec{R}_2}, \ldots, \rtp{\vec{R}_{M}}\right),
	\label{eqn:DefAllCoords}
\end{align}
to refer to all electronic or all nuclear coordinates, respectively.
Taking $I$ to denote an appropriate multi-index of quantum numbers,
our problem from the previous paragraph can be reformulated
as finding those eigenstates $\Psi^\text{MB}_I \in H^2(\R^L, \C)$
with lowest corresponding energies $E^\text{MB}_I \in \R$
by the means of solving the time-independent Schrödinger equation
\begin{equation}
	\Op{H}^\text{MB} \Psi_I^\text{MB}(\vec{X}, \vec{x})
	= E_I^\text{MB} \Psi_I^\text{MB}(\vec{X}, \vec{x}).
	\label{eqn:TISEManyBody}
\end{equation}
olving this equation analytically is not possible in general.
Already for the Helium atom, a 3-body problem, clever approximations are needed
to get anywhere~\cite{Hylleras1929}.
But even numerically \eqref{eqn:TISEManyBody} is in general impossible to solve
without further approximations.

Let us illustrate this claim by an example.
In water, \ce{H2O}, we have $3$ nuclei and $10$ electrons.
The dimensionality of the problem is thus $L = 3 \cdot 13 = 39$.
In the numerical approach we introduces in section \vref{sec:Projection}
the evaluation of the inner product
\begin{equation}
	\braket{\Psi}{\Phi} \equiv \int_{\R^{3M}} \int_{\R^{3\Nelec}}
		\cc{\Psi}(\vec{x}, \vec{X}) \Phi(\vec{x}, \vec{X})
	\D \vec{x} \D \vec{X}
	\label{eqn:MBInnerProduct}
\end{equation}
between two functions $\Psi$ and $\Phi$ from the
underlying Hilbert space $L^2(\R^L, \C)$ appears rather prominently.
Most notably the computation of the sesquilinear form $a(\slot,\slot)$
in order to build the discretisation matrix in \eqref{eqn:DiscretisedEigenproblem}
boils down to computing such integrals.
The numerical evaluation of \eqref{eqn:MBInnerProduct}
implies a sampling of the $L$-dimensional space $\R^L$ in some way or another.
Even for an extremely sophisticated discretisation method
or a well-designed quadrature scheme we will probably need of the order of
$10$ sampling points per dimension.
For a 39-dimensional problem, like our water molecule,
this makes on the order of $10^{39}$ sampling points overall.
If we want a single integration to finish within the lifetime of a human being,
say $100$ years,
the evaluation of the
integration kernel $\Psi(\vec{x}, \vec{X}) \Phi(\vec{x}, \vec{X})$
may take no more than some $10^{-30}$ seconds,
which is impossible due to the physical limitations inside a general purpose computer.

Certainly one could probably find even more clever methods in some cases,
but the example illustrates the so-called \newterm{curse of dimensionality} rather well.
For a general quantum-chemical investigation of matter
one needs to develop approximate methodologies.

\section{Born-Oppenheimer approximation}
\label{sec:BO}
The masses of electrons and nuclei differ by orders of magnitude.
The ratio between the mass of a proton and the electron massi
is already around $1836$ and this ratio increases
further across the table.
Already for the elements oft he first period,
this value is at least of the order of $10^4$.
This justifies an approximative treatment,
where we assume the motion of the electrons
and the motion of the nuclei to happen at different timescales.

For this let us consider at first
a simplified version of \eqref{eqn:ManyBodyHamiltonian},
namely the \newterm{electronic Hamiltonian}
\[
	\Op{H}^\text{elec} \equiv \Op{H}^\text{MB} - \Op{T}_n
	= \Op{T}_e + \Op{V}_{ne} + \Op{V}_{ee} + \Op{V}_{nn}.
\]
This operator is constructed form the full many-body Hamiltonian
by neglecting the nuclear kinetic energy operator $\Op{T}_n$ completely.
Introducing the short hand notation
\begin{align*}
r_{AB} &\equiv \norm{\vec{R}_A - \vec{R}_B}_2, &
r_{iA} &\equiv \norm{\vec{r}_i - \vec{R}_A}_2, &
r_{ij} &\equiv \norm{\vec{r}_i - \vec{r}_j}_2, &
\end{align*}
we can write it as
\begin{equation}
	\Op{H}^\text{elec}
	= \sum_{i=1}^{\Nelec} \Delta_{\vec{r}_i}
	+ \sum_{A=1}^M \sum_{i=0}^{\Nelec} \frac{Z_A}{r_{iA}}
	+ \sum_{i=1}^{\Nelec} \sum_{j=i+1}^{\Nelec} \frac{1}{r_{ij}}
	\  + \  %
	\sum_{A=1}^{M} \sum_{B=A+1}^{M} \frac{1}{r_{AB}}.
	\label{eqn:ElectronicFullHamiltonian}
\end{equation}
Even though $\Op{H}^\text{elec}$ still depends on the nuclear coordinates $\vec{X}$,
one could interpret the elements of the vector
$\vec{X}$ not as coordinates,
but much rather as parameters for the potential operators $\Op{V}_{ne}$ and $\Op{V}_{nn}$.
Physically this means that $\Op{H}^\text{elec}$ describes a chemical system
where the nuclei are clamped at well-defined points in space.
Sometimes we will write $\Op{H}^\text{elec}(\vec{X})$
in order to make the parametrisation of $\Op{H}^\text{elec}$ with respect to $\vec{X}$
visible.

\newcommand{\Iel}{I_\text{e}}
\newcommand{\Inu}{I_\text{n}}
Without going into details at the moment,
let us assume that $\Op{H}^\text{elec}$ becomes self-adjoint
inside a suitable domain.
With appropriate multi-indices $\Iel$
we can thus find its eigenpairs $(E_{\Iel}, \Psi^\text{elec}_{\Iel})$
via the \newterm{electronic Schrödinger equation}
\begin{equation}
	\Op{H}^\text{elec}(\vec{X}) \Psi^\text{elec}_{\Iel}(\vec{X}, \vec{x})
	= E^\text{elec}_{\Iel}(\vec{X})
		\Psi^\text{elec}_{\Iel}(\vec{X}, \vec{x}).
	\label{eqn:ElectronicSchrödinger}
\end{equation}
Originating from the dependence of $\Op{H}^\text{elec}(\vec{X})$
towards the nuclear coordinates,
we can think of the resulting \newterm{electronic energies}
$E^\text{elec}_{\Iel}(\vec{X})$
and \newterm{electronic wavefunction}s $\Psi^\text{elec}_{\Iel}(\vec{X}, \vec{x})$
to be dependent on $\vec{X}$ as well.
Sometimes one uses the term \newterm{electronic state} or just \newterm{stat}
to refer to $\Psi^\text{elec}_{\Iel}(\vec{X}, \vec{x})$.

With the electronic states at hand we are able to formulate
the framework of the \newterm{Born-Oppenheimer approximation},
which consists of the following two assumptions:
\begin{itemize}
	\item Each eigenstate of \eqref{eqn:ManyBodyHamiltonian} may be written
		by a factorisation
	\begin{equation}
		\Psi^\text{MB}_I(\vec{X}, \vec{x}) \equiv \Psi^\text{MB}_{\Iel\Inu}(\vec{X}, \vec{x})
		\simeq \Psi^\text{elec}_{\Iel}(\vec{X}, \vec{x})
		\Psi^\text{nuc}_{\Inu}(\vec{x}),
		\label{eqn:BOFactorisation}
	\end{equation}
	where the multi-indices are related by $I \equiv (\Iel, \Inu)$.
	$\Psi^\text{elec}_{\Iel}(\vec{X}, \vec{x})$ is a solution
	to the electronic Schrödinger equation \eqref{eqn:ElectronicSchrödinger}
	and the \newterm{nuclear wavefunction}
	$\Psi^\text{nuc}_{\Inu}(\vec{x})$ is yet to be determined.
	%
	\item The factorisation \eqref{eqn:BOFactorisation} satisfies the property%
	\footnote{%
		More precisely what we assume is that the nuclear kinetic energy
		operator $\Op{T}_n$ projected onto the basis formed by
		all electronic states $\Psi^\text{elec}_{\Iel}(\vec{X}, \vec{x})$
		is diagonal with all elements equal to $1$.
		See \cite{Baer2006} or \cite{WikipediaBornOppenheimer} for details.
	}
	\begin{equation}
		\Op{T}_n \Psi^\text{MB}_{I}(\vec{X}, \vec{x})
		\simeq \Op{T}_n \left( \Psi^\text{elec}_{\Iel}(\vec{X}, \vec{x})
		\Psi^\text{nuc}_{\Inu}(\vec{X}) \right)
		\simeq \Psi^\text{elec}_{\Iel}(\vec{X}, \vec{x})
		\left(\Op{T}_n \Psi^\text{nuc}_{\Inu}(\vec{X}) \right).
		\label{eqn:BONuclearDerivative}
	\end{equation}
\end{itemize}
By plugging ansatz \eqref{eqn:BOFactorisation} into \eqref{eqn:TISEManyBody}
we can simplify
\begin{align*}
	0 &=
	\left( \Op{H}^\text{MB} - E^\text{MB}_{I} \right)
	\Psi^\text{MB}_{I}(\vec{X}, \vec{x}) \\
	&\stackrel{\eqref{eqn:BOFactorisation}}{\simeq}
	\left( \Op{H}^\text{elec} + \Op{T}_n - E^\text{MB}_I \right)
	\Psi^\text{elec}_{\Iel}(\vec{X}, \vec{x}) \Psi^\text{nuc}_{\Inu}(\vec{X}) \\
	&\stackrel{\eqref{eqn:BONuclearDerivative}}{\simeq}
	\left( \Op{H}^\text{elec} \Psi^\text{elec}_{\Iel}(\vec{X}, \vec{x}) \Psi^\text{nuc}_{\Inu}(\vec{X}) \right)
	+ \Psi^\text{elec}_{\Iel}(\vec{X}, \vec{x}) \left(\Op{T}_n \Psi^\text{nuc}_{\Inu}(\vec{X}) \right)
	- E^\text{MB}_{I} \Psi^\text{MB}_I(\vec{X}, \vec{x})
	\\
	&\stackrel{\eqref{eqn:ElectronicSchrödinger}}{=}
	\Psi^\text{elec}_{\Iel}(\vec{X}, \vec{x})
	\left(E^\text{elec}_{\Iel}(\vec{X}) \Psi^\text{nuc}_{\Inu}(\vec{X})
	+ \Op{T}_n \Psi^\text{nuc}_{\Inu}(\vec{X})
	- E^\text{MB}_I \Psi^\text{nuc}_{\Inu}(\vec{X}) \right).
\end{align*}
This statement is thus satisfied provided that the nuclear wavefunction
$\Psi^\text{nuc}_{\Inu}(\vec{X})$
follows the \newterm{nuclear Schrödinger equation}
\begin{equation}
	\left( \Op{T}_n + E^\text{elec}_{\Iel}(\vec{X}) \right)
	\Psi^\text{nuc}_{\Inu}(\vec{X})
	= E^\text{MB}_I \Psi^\text{nuc}_{\Inu}(\vec{X}).
	\label{eqn:NuclearSchrödinger}
\end{equation}

Overall the Born-Oppenheimer approximation
allows to solve the many-body Schrödinger equation \eqref{eqn:TISEManyBody} in two steps.
First we limit ourselves to the point of view of the electrons
under the electric field induced by fixed, motionless nuclei.
This leads to \eqref{eqn:ElectronicSchrödinger},
which is solved for the
electronic states $\Psi^\text{elec}_{\Iel}(\vec{X}, \vec{x})$
along with corresponding electronic energies $E^\text{elec}_{\Iel}(\vec{X})$.
In the second step we consider nuclear motion by solving
\eqref{eqn:NuclearSchrödinger}.
In this equation the electronic energies $E^\text{elec}_{\Iel}(\vec{X})$
depending on the nuclear coordinates act as the electrostatic potential
in which the nuclei move.
For this reason $E^\text{elec}_{\Iel}(\vec{X})$ is sometimes
called a \newterm{potential energy surface} as well.
Note, that each electronic state characterised by quantum numbers $\Iel$
gives rise to a different potential energy surface.

Employing a more detailed treatment of the Born-Oppenheimer approximation,
like in the original paper \cite{Born1927} or \citet{Baer2006},
allows to gain more insight regarding the range of applicability
of the Born-Oppenheimer approximation.
Loosely speaking it is a valid approximation
as long as the potential energy surfaces $E^\text{elec}_{\Iel}(\vec{X})$
are well-separated from another.

From a numerical point of view this approximation allows to reduce
the dimensionality of the problem somewhat.
To illustrate this let us return to the water molecule,
which was already discussed at the end of section \vref{sec:ManyBodyTISE}.
In the exact problem we need to solve one equation,
namely the many-body Schrödinger equation \eqref{eqn:TISEManyBody}
of dimensionality $L = 39$.
Within the Born-Oppenheimer approximation
this is replaced by solving two equations,
the electronic one \eqref{eqn:ElectronicSchrödinger} of dimensionality $3 \Nelec = 30$
and the nuclear TISE \eqref{eqn:NuclearSchrödinger} of dimensionality $3 M = 9$.
In the rough estimate we presented in section \vref{sec:ManyBodyTISE}
for the $L^2$ inner products,
this would roughly provide a speed-up factor of $10^9$.

\subsection{Electronic Schrödinger equation}
\label{sec:ElectronicSchrödinger}
