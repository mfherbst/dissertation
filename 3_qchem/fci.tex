\section{Full configuration interaction}
\label{sec:FCI}
\defineabbr{FCI}{FCI\xspace}{Full configuration interaction}
\nomenclature{$\Phi$}{Slater determinant $\bigwedge_{i=1}^{\Nelec} \psi_i$
or many-electron basis functions}
\nomenclature{$\psi_i$}{One-particle function,
typically $i$-th eigenfunction of the Fock operator, i.e.~a Hartree-Fock orbital}
\nomenclature{$\Ibas$}{Index set of the one-particle basis functions.
	Typically a set of multi-indices of quantum numbers.}
\nomenclature{$\Nbas$}{Cardinality of $\Ibas$, i.e.~the number of one-particle basis functions.}
\nomenclature{$\varphi_\mu$}{$\mu$-th one-particle basis function of the one-particle basis $\{\varphi_\mu\}_{\mu \in \Ibas}$}

\nomenclature{$\Iocc$}{Index set of occupied SCF orbitals.}
\nomenclature{$\Ivirt$}{Index set of virtual, i.e. unoccupied, SCF orbitals.}

In this section we want to develop a numerical treatment
for solving the electronic Schrödinger equation \eqref{eqn:ElectronicSchrödinger}
under the Ritz-Galerkin projection ansatz of section \vref{sec:RitzGalerkin}.
In the previous section we analysed the mathematical implications
of the spin statistics theorem for electrons as fermionic systems,
which lead us to choose the form domain
\[ Q(\Op{H}_{\Nelec}) = H^1(\R^{3\Nelec}, \C) \cap \bigwedge^{\Nelec} L^2(\R^3, \C). \]
for the electronic Schrödinger operator $\Op{H}_{\Nelec}$.


For simplifying our treatment
we will not try to discretise this domain in the Ritz-Galerkin ansatz
of definition \ref{defn:RitzGalerkin},
much rather we will develop methods to sample only the subspace
\[ \tilde{Q}(\Op{H}_{\Nelec}) = \bigwedge^{\Nelec} H^1(\R^3, \C) \subset Q(\Op{H}_{\Nelec}) \]
due to its simpler structure.
Since this subspace is dense this implies no potential loss of numerical accuracy.
By definition of the exterior power
\begin{equation}
	\tilde{Q}(\Op{H}_{\Nelec}) = \spacespan \left\{
		\bigwedge_{i=1}^{\Nelec} \psi_i
		\, \middle| \,
		\psi_i \in H^1(\R^3, \C) \, \forall i = 1, \ldots, \Nelec
	\right\}.
	\label{eqn:FormDomainAllFunctions}
\end{equation}
Since $H^1(\R^{3\Nelec}, \C)$ is separable, we can find a countable basis set
\begin{equation}
	\set{B}_1 \equiv \{\psi_i\}_{i \in \N} \qquad \text{with }
	\braket{\psi_i}{\psi_j}_1 = \delta_{ij}
	\quad \text{and} \quad \spacespan \set{B}_1 = H^1(\R^{3\Nelec}, \C),
	\label{eqn:OneParticleBasis}
\end{equation}
where we used the abbreviated notation
$\braket{\slot}{\slot}_1 = \braket{\slot}{\slot}_{L^2(\R^3, \C)}$.
$\set{B}_1$ is sometimes called the \newterm{one-particle basis}.
Taking the properties of the wedge product \eqref{eqn:PropertiesExteriorProduct}
into account allows to deduce the equivalent construction
\begin{equation}
	\label{eqn:FormDomainSlaterDeterminants}
	\tilde{Q}(\Op{H}_{\Nelec}) = \spacespan
	\left\{ \bigwedge_{i=1}^{\Nelec} \psi_i
	, \middle| \, \psi_i \in \set{B}_1 \, \forall i=1,\ldots,\Nelec
	\right\},
\end{equation}
which builds $\tilde{Q}(\Op{H}_{\Nelec})$ as the span
over all Slater determinants built by selecting
$\Nelec$ functions from $\set{B}_1$.
Since nothing stops us from selecting the same basis function
twice from $\set{B}_1$ in this construction,
many of the constructed determinants $\bigwedge_{i=1}^{\Nelec} \psi_i$
are zero.
In other words these determinants amount to span $\tilde{Q}(\Op{H}_{\Nelec})$,
but they are not a basis for this space.
In the following we want to fix this and construct an orthonormal basis
of suitable Slater determinants.
This requires an appropriate inner product.
\begin{defn}
	Let $\tilde{Q}(\Op{H}_{\Nelec})$ be defined as in \eqref{eqn:FormDomainAllFunctions}.
	We define an inner product on $\tilde{Q}(\Op{H}_{\Nelec})$ by
	requiring for any two arbitrary Slater determinants
	\begin{align*}
		\Psi &= \bigwedge_{i=1}^{\Nelec} \psi_i
		&&\text{and}& \Xi &= \bigwedge_{i=1}^{\Nelec} \xi_i
	\end{align*}
	with $\psi_i, \xi_i \in H^1(\R^3, \C)$ for all $i \in 1,\ldots,\Nelec$:
	\begin{align}
		\braket{\Psi}{\Xi}_{\Nelec} \equiv \det \mat{G}
		\qquad
		\text{where } G_{ij} = \braket{\psi_i}{\xi_j}_{L^2(\R^3, \C)} \forall i,j \in 1,\ldots,\Nelec.
		\label{eqn:InnerProductFormDomain}
	\end{align}
	The inner product for other elements from $\tilde{Q}(\Op{H}_{\Nelec})$
	is then constructed in accordance
	with the axioms shown in definition \vref{def:InnerProduct}.
	\todoil{A more formal proof would be nice,
		especially for the completeness under this inner product.
		Or showing equivalence with respect to the inner product
		on $H^1(\R^{3\Nelec}, \C$ for the elements of $\tilde{Q}(\Op{H}_{\Nelec})$
		would be nice.}
\end{defn}

\noindent
With this inner product at hand we can construct an orthonormal basis for
$\tilde{Q}(\Op{H}_{\Nelec})$.

\begin{rem}[Orthonormal basis for $\tilde{Q}(\Op{H}_{\Nelec})$]
	\label{rem:Determinants}
	Let $\tilde{Q}(\Op{H}_{\Nelec})$ be defined as in \eqref{eqn:FormDomainAllFunctions}
	and let $\set{B}_1$ be an arbitrary one-particle basis for $H^1(\R^3, \C)$.
	We take one arbitrary, non-trivial Slater determinant
	$0 \neq \Phi_0 \in \tilde{Q}(\Op{H}_{\Nelec})$,
	such that
	\[
		\Phi_0 = \varphi_1 \wedge \varphi_2 \wedge \cdots \varphi_i \cdots \wedge \varphi_{\Nelec}.
	\]
	This determinant can always be found due to the alternative
	construction for $\tilde{Q}(\Op{H}_{\Nelec})$ sketched
	in \eqref{eqn:FormDomainSlaterDeterminants}.
	Let us call $\Phi_0$ the \newterm{reference determinant}
	in the following.

	The functions of the (countable) basis set $\{\psi_i\}_{i \in \N} = \set{B}$
	can be indexed in such a way
	that the first $\Nelec$ functions coincide with $\{\varphi\}_{i \in \Nelec}$.
	In other words
	\[
		\Phi_0 = \psi_1 \wedge \psi_2 \wedge \cdots \psi_i \cdots \wedge \psi_{\Nelec}
	\]
	as well.
	We further define the index sets%
	\footnote{%
		The subscript ``occ'' stands for \textit{occupied}
		and ``virt'' for \textit{virtual}.
		These terms will become clear when we discuss the Hartree-Fock ansatz
		in the next section.
	}
	\begin{align*}
		\Iocc &= \{1, \ldots, \Nelec\}
		&&\text{and} &
		\Ivirt &= \{ i \in \N \, | \, i > \Nelec \}.
	\end{align*}
	With reference to $\Phi_0$ we can construct
	for each $i \in \Iocc$ and each $a \in \Ivirt$
	a so-called singly \newterm{excited determinant}
	\[
		\Phi_i^a = \psi_1 \wedge \psi_2 \wedge \cdots \psi_a \cdots \wedge \psi_{\Nelec}
	\]
	by replacing the $i$-th function of the Slater determinant
	wedge string
	by the $a$-th function of $\set{B}$
	without changing the order.
	Analogously one may define doubly or higher excited determinants
	\begin{align*}
		\Phi_{ij}^{ab} &= \psi_1 \wedge \psi_2 \wedge \cdots \wedge \psi_a \cdots \psi_b
			\wedge \cdots \wedge \psi_{\Nelec} \\
		\Phi_{ijk}^{abc} &= \psi_1 \wedge \psi_2 \wedge \cdots \wedge \psi_a \cdots \psi_b
			\cdots \psi_c \wedge \cdots \wedge \psi_{\Nelec}
	\end{align*}
	where%
	\footnote{%
		This is the typical indexing convention in quantum chemistry.
		Indices $i,j,k,l,m, \ldots$ stand for occupied indices
		and $a,b,c,d,e, \ldots$ for virtual indices.
	} $i,j,k \in \Iocc$ and $a,b,c \in \Ivirt$
	In this case one has to additionally
	require that $i < j < k < \cdots$ and $a < b < c < \cdots$,
	because otherwise no new determinants are generated (if $i=j$ or $i=k$ or \ldots)
	or a zero determinant is generated (if $a=b$ or similar).
	Constructed in this way all determinants in the set
	\[
		\set{B}_{\Nelec} \equiv
		\left\{
			\Phi_0, \Phi_i^a, \Phi_{ij}^{ab}, \Phi_{ijk}^{abc}, \cdots \right\}
	\]
	are unique.
	Still it is not hard to see
	that $\spacespan \set{B}_{\Nelec} = \tilde{Q}(\Op{H}_{\Nelec})$,
	since we only took away those determinants adding redundant information
	in the construction \eqref{eqn:FormDomainSlaterDeterminants}.

	With the inner product defined in \eqref{eqn:InnerProductFormDomain} we notice
	for all $r,s \in \N$
	\[
		\braket{\Phi_0}{\Phi_r^s}_{\Nelec}
			= \braket{\psi_r}{\psi_s}_1 = \delta_{rs},
	\]
	since by \eqref{eqn:OneParticleBasis}
	all functions in $\set{B}_1$ are orthonormal to each other.
	In other words $\set{B}_{\Nelec}$ is an orthonormal
	basis for $\tilde{Q}(\Op{H}_{\Nelec})$.
	\todo[inline, caption={}]{
		This following statement might not be true,
		since $\braket{\slot}{\slot}_{\Nelec}$
		might not be equivalent to the Hilbertian inner product!
		\begin{center}
		Since it is countable, $\tilde{Q}(\Op{H}_{\Nelec})$ is separable.
		\end{center}
	}
\end{rem}
The set $\set{B}_{\Nelec}$ is sometimes called the \newterm{$\Nelec$-particle basis}
or \newterm{many-particle basis}
corresponding to $\set{B}_1$ and the reference determinant $\Phi_0$.
Notice that any choice of $\Phi_0$ or $\set{B}_1$ would work for its construction
and albeit the precise entries in $\set{B}_{\Nelec}$ might differ from case to case
the end result $\spacespan \set{B}_{\Nelec} = \tilde{Q}(\Op{H}_{\Nelec})$
is always true.

\begin{rem}
	Given a many-particle basis $\set{B}_{\Nelec}$ consisting
	of normalised Slater determinants,
	any function $\Psi \in \tilde{Q}(\Op{H}_{\Nelec})$
	can be expanded as such
	\begin{align}
		\label{eqn:ExpansionSlaterDeterminant}
		\Psi &= \sum_\mu c_\mu \Phi_\mu
		&\text{where } \forall \mu \in \N: \quad
		\Phi_\mu \in \set{B}_{\Nelec}, c_\mu \in \C.
	\end{align}
	If one is interested in emphasising the particular
	one-particle basis $\set{B}_1$ and the particular reference determinants $\Phi_0$
	this can be written equivalently as
	\begin{equation}
		\label{eqn:CIExpansion}
		\Psi
		= c_0 \Phi_0 + \sum_{ia} c_i^a \Phi_i^a
		+ \sum_{\substack{i<j \\a<b}} c_{ij}^{ab} \Phi_{ij}^{ab}
		+ \sum_{\substack{i<j<k \\ a<b<c}} c_{ijk}^{abc} \Phi_{ijk}^{abc}
		 + \cdots,
	\end{equation}
	where $i,j,k, \ldots \in \Iocc$ and $a,b,c, \ldots \in \Ivirt$.

	This expansion is commonly referred to as the \newterm{CI expansion},
	where CI stands for configuration interaction,
	a term which will become more clear
	after the next remark.
\end{rem}

\defineabbr{FCI}{FCI\xspace}{Full configuration interaction,
	see section \vref{sec:FCI}}
\newcommand{\Nfci}{N_\text{FCI}}
\begin{rem}[Full CI]
	\label{rem:FCI}
	The discrete formulation of the Ritz-Galerkin scheme
	of remark \vref{rem:DiscreteFormulation}
	can now be applied rather easily to the electronic Schrödinger equation.
	This leads to a procedure called \newterm{full CI}
	or full configuration interaction~(\FCI).

	\begin{itemize}
	\item Take a finite-sized one-electron basis set
	\[ \set{B}_1^{(n)} \equiv \{\varphi_\mu\}_{\mu \in \Ibas} \]
	of orthonormal functions $\varphi_\mu \in U$,
	where $U \subset H^1(\R^3, \C)$ is dense.
	%
	\item Define a reference determinant
	\[ \Phi_0 = \varphi_1 \wedge \varphi_2 \wedge \ldots \wedge \varphi_{\Nelec} \]
	where $\{\varphi_i\}_{i=1,\ldots,\Nelec} \subset \{\varphi_\mu\}_{\mu \in \Ibas}$
	arbitrarily and construct the finite $\Nelec$-electron basis
	\[ \set{B}_{\Nelec}^{(n)} \equiv \{ \Phi_0, \Phi_i^a, \Phi_{ij}^{ab}, \ldots \} \]
	according to the procedure described in remark \vref{rem:Determinants}.
	As usual we take
	\begin{align}
		\label{eqn:CondOccIndex}
		i, j, k, l, \ldots &\in \Iocc &&\text{with}& i&<j<k<l<\cdots \\
		\label{eqn:CondVirtIndex}
		a, b, c, d, \ldots &\in \Ivirt &&\text{with}& a&<b<c<d<\cdots
	\end{align}
	where in the finite case
	\begin{align*}
		\Iocc &= \{1, \ldots, \Nelec\} &&\text{and}& \Ivirt &= \{ \Nelec + 1, \ldots, \Nbas \}.
	\end{align*}
	%
	\item Construct the full CI matrix $\mat{A}_\text{FCI} \in \C^{\Nfci \times \Nfci}$
		consisting of elements
		\begin{equation}
			a(\Phi_1, \Phi_2)
			= \mbra{\Phi_1}
			\Op{T}_e + \Op{V}_{ne} + \Op{V}_{ee}
			\mket{\Phi_2}_{\Nelec}
			\label{eqn:FCIMatrixElements}
		\end{equation}
		for all combinations $\Phi_1, \Phi_2 \in \set{B}_{\Nelec}^{(n)}$.
		There are
		\[ \Nfci = \binom{\Nbas}{\Nelec} \leq \Nbas^{\Nelec} \]
		such Slater determinants.
	\item Diagonalise $\mat{A}_\text{FCI}$ to find a few energy eigenvalues
		$E_i^{(n)} \in \R$ and corresponding \newterm{CI vectors}
		$\vec{c}^{(n)}_i \in \C^{\Nfci}$.
	\item Repeat with a larger basis $\set{B}_1^{(n+1)}$ until convergence
		of eigenstates up to desired accuracy has been achieved.
		In many cases one already selects a suitable basis set
		$\set{B}_1^{(n)}$ and only performs the calculation top-to-bottom once.
	\end{itemize}
	Notice that the subspace sequence
	$\spacespan \set{B}_{\Nelec}^{(n)} \subset \tilde{Q}(\Op{H}_{\Nelec})$
	satisfies the required condition \eqref{eqn:CondSubspaces}
	since $U$ is dense in $H^1(\R^3, \C)$,
	which makes $\spacespan \set{B}_{\Nelec}^{(n)}$
	dense in $\tilde{Q}(\Op{H}_{\Nelec})$
	and thus transitively dense in $Q(\Op{H}_{\Nelec})$.
	If $\Op{H}_{\Nelec}$ thus has a discrete ground and some discrete
	excited states below the essential spectrum,
	we can approximate it by this procedure up to arbitrary accuracy.
	This is satisfied for all neutral or positively charged systems
	and some negatively charged systems.
	Recall remark \vref{rem:ElectronicTISENumerical} for details.
\end{rem}
Equation \eqref{eqn:FCIMatrixElements} helps to understand
where the term full configuration interaction for the sketched
method comes from.
In some sense a basis function in the single-particle
basis $\{\varphi_\mu\}_{\mu\in\Ibas}$
describes the behaviour of a single electron.
In turn a Slater determinant can be interpreted physically
as one sensible configuration of $\Nelec$ electrons
amongst the available single-particle functions.
The full CI matrix \eqref{eqn:FCIMatrixElements}
now couples these configurations
via the electronic Hamiltonian $\Op{H}_{\Nelec}$,
which describes the interaction of the electrons in the chemical
system with another.
By diagonalising the matrix $\mat{A}_\text{FCI}$ we thus determine
the electronic eigenstates taking the full range of interactions between
all configurations into account, explaining the name full CI.

Even though \FCI allows to compute the solution of the electronic
Schrödinger equation up to arbitrary accuracy,
it is not employed a lot in practice.
The main reason for this is its enormous computational cost.
Already for a small molecules like water with only
$\Nelec = 10$ electrons and a rather small def2-sv(p)\todo{cite} basis set
$\set{B}_1^{(n)}$ with $\Nbas = 18$ basis functions this makes $\Nfci = 43758$
and thus around $\Nfci^2 = 2\E{9}$ entries in $\mat{A}_\text{FCI}$
in an extremely naive implementation.
Of course this can be improved by exploiting some symmetries
or the rather sparse structure of $\mat{A}_\text{FCI}$,
which we will discuss in the next remark.
Nevertheless the computational cost scales exponentially
and allows only treatment of extremely small systems.

\defineabbr{ERI}{ERI\xspace}{Electron repulsion integrals}
\newcommand{\eriAsym}[2]{\left\langle #1 \middle|\middle| #2 \right\rangle}
\newcommand{\eriPh}[2]{\left\langle #1 \middle| #2 \right\rangle}
\newcommand{\eriMu}[2]{\left( #1 \middle| #2 \right)}
\nomenclature{$(ij|kl)$}{%
	Electron repulsion integrals in chemist's or Mullikan notation,
	see definition in remark \vref{rem:ERI}.
}
\begin{rem}[Structure and evaluation of the \FCI matrix]
	Recall the expression
	\[
		\Op{H}_{\Nelec}
		= \Op{T}_e + \Op{V}_{en} + \Op{V}_{ee}
		= \sum_{i=1}^{\Nelec}
		  \left( -\frac12 \Delta_{\vec{r}_i} + \sum_{A=1}^M \frac{Z_A}{r_{iA}} \right)
		+ \sum_{i=1}^{\Nelec} \sum_{j=i+1}^{\Nelec} \frac{1}{r_{ij}}
	\]
	for the electronic Hamiltonian.
	The goal of this remark will be to write the many electron
	integrals $\braket{\Phi_1}{\Op{H}_{\Nelec}\, \Phi_2}$ between
	two Slater determinants $\Phi_1, \Phi_2 \in \set{B}_{\Nelec}$
	in terms of integrals over the one-electron functions $\psi_i$
	these determinants are composed of.

	\todoil{Derivation of Slater-Condon rules in appendix?}
	For this we will make use of the so-called Slater-Condon rules~\cite{Szabo1996}.
	For applying these rules
	we need to differentiate between so-called \newterm{one-electron operator}s
	and \newterm{two-electron operator}s.
	One-particle operators like $\Op{T}_e$ and $\Op{V}_{en}$
	can be written as a sum of operators like $\Delta_{\vec{r}_i}$ or $r_{iA}^{-1}$,
	which act only on the coordinate $\vec{r}_i$ of a single electron at once.
	Two-particle operators like $\Op{V}_{ee}$, however,
	are built as a sum of terms $r_{ij}^{-1}$ making reference
	to the coordinates of two electrons.

	For our discussion here, let us take $\Phi_0$ to be an arbitrary
	reference determinant constructed from the single-particle basis $\set{B}_1$.
	We construct excited determinants
	$\Phi_i^a$, $\Phi_{ij}^{ab}$, \ldots
	under the index conventions
	\eqref{eqn:CondOccIndex} and \eqref{eqn:CondVirtIndex}.

	For the one-electron operator $\Op{T}_e + \Op{V}_{en}$
	the Slater-Condon rules yield
	\begin{align*}
		\braket{\Phi_0}{\left(\Op{T}_e + \Op{V}_{en}\right) \,\Phi_0}_{\Nelec}
		&= \sum_{i\in\Iocc} \braket{\psi_i}{\Op{H}_{\text{core}} \, \psi_i}_1 \\
		%
		\braket{\Phi_0}{\left(\Op{T}_e + \Op{V}_{en}\right) \,\Phi_{i}^{a}}_{\Nelec}
		&= \braket{\psi_i}{\Op{H}_\text{core} \, \psi_a}_1 \\
		%
		\braket{\Phi_0}{\left(\Op{T}_e + \Op{V}_{en}\right) \,\Phi_{ij}^{ab}}_{\Nelec}
		&= 0.
	\end{align*}
	In this result we made use of the \newterm{core Hamiltonian} operator
	\begin{equation}
		\Op{H}_\text{core}
		= \Op{T} + \Op{V}_0
		= -\frac12 \Delta + \sum_{A=1}^M \frac{Z_A}{\norm{\vec{r} - \vec{r}_A}_2},
		\label{eqn:HCore}
	\end{equation}
	which is just the sum of the kinetic operator $\Op{T}$
	and the nuclear attraction operator $\Op{V}_0$
	contribution from a single electron.
	More generally we can state
	that the matrix element $\braket{\Phi_1}{\Op{A}_1 \Phi_2}_{\Nelec}$
	of a one-particle operator $\Op{A}_1$
	is only non-zero for determinants $\Phi_1$, $\Phi_2$,
	which differ in at most one single-particle function.

	On the other hand for
	two-electron operators like $\Op{V}_{ee}$
	the Slater-Condon rules yield
	\begin{align*}
		\braket{\Phi_0}{\Op{V}_{ee}\, \Phi_0}_{\Nelec}
		&= \frac12 \sum_{i\in\Iocc}
			  \eriMu{\psi_i\psi_i}{\psi_j\psi_j}
			- \eriMu{\psi_i\psi_j}{\psi_i\psi_j} \\
		%
		\braket{\Phi_0}{\Op{V}_{ee}\, \Phi_i^a}_{\Nelec}
		&= \sum_{j\in\Iocc}
			  \eriMu{\psi_i\psi_a}{\psi_j\psi_j}
			- \eriMu{\psi_j\psi_a}{\psi_i\psi_j} \\
		%
		\braket{\Phi_0}{\Op{V}_{ee}\, \Phi_{ij}^{ab}}_{\Nelec}
		&= \eriMu{\psi_i\psi_j}{\psi_a\psi_b}
		 - \eriMu{\psi_a\psi_j}{\psi_i\psi_b} \\
		%
		\braket{\Phi_0}{\Op{V}_{ee}\, \Phi_{ijk}^{abc}}_{\Nelec} &= 0.
	\end{align*}
	where the \newterm{electron repulsion integrals}~(ERIs)
	in Mulliken's indexing convention are given by the expression
	\begin{equation}
		\eriMu{\psi_i \psi_j}{\psi_k \psi_l}
			= \int_{\R^3} \int_{\R^3}
				\frac{\cc{\psi_i}(\vec{r}_1) \psi_j(\vec{r}_1)
					\,\cc{\psi_k}(\vec{r}_2) \psi_l(\vec{r}_2)}
				{\norm{\vec{r}_1 - \vec{r}_2}_2}
				\D \vec{r}_1 \D \vec{r}_2.
		\label{eqn:ERI}
	\end{equation}
	Again as a general result,
	determinants which differ in more than two one-particle functions
	vanish according to the Slater-Condon rules.

	For the full CI matrix $\mat{A}_\text{FCI}$
	the Slater-Condon rules imply that it has a rather sparse structure
	despite its size~(see fig. \vref{fig:StructureFCIMatrix}).
	For this reason the diagonalisation
	step of remark \vref{rem:FCI} is typically performed
	using one of the iterative methods sketched
	in section \vref{sec:DiagAlgos}.
\end{rem}
\begin{figure}
	\centering
	\missingfigure{FCI matrix structure}
	\caption{FCI matrix structure}
	\label{fig:StructureFCIMatrix}
\end{figure}

The electron repulsion integral tensor introduced in \eqref{eqn:ERI}
is a very important integral quantity in computational chemistry,
which will used at many occasions throughout the thesis.
A couple of deviating indexing conventions
and definitions exist, the following remark gives an overview.

\begin{rem}[Formulation of the repulsion integrals]
	\label{rem:ERI}
	\todo[inline,caption={}]{
	State integral definition again
	and associate it with the Mullikan and phys convention
	}

	Multiple conventions exist for the repulsion integrals.
	In \eqref{eqn:ERI} we introduced the shell pair notation
	or Chemists' notation.
	Alternatively there is the physicists' convention
	\[
		\eriPh{\psi_i \psi_j}{\psi_k \psi_l}
		\equiv \eriMu{\psi_i\psi_k}{\psi_j\psi_l}
	\]
	With this we can define an antisymmetrised electron repulsion
	integral
	\[ \eriAsym{\psi_i \psi_j}{\psi_k \psi_l}
		\equiv \eriPh{\psi_i \psi_j}{\psi_k \psi_l}
		- \eriPh{\psi_j \psi_i}{\psi_k \psi_l}
		= \eriMu{\psi_i\psi_k}{\psi_j\psi_l} - \eriMu{\psi_j\psi_k}{\psi_i\psi_l}
	\]
	if the one-particle basis and its indexing is clear from context
	one often just writes
	\[ \eriMu{ij}{kl}, \eriPh{ij}{kl}, \eriAsym{ij}{kl} \]

	For a list of all index symmetries between electron
	repulsion integrals see appendix \vref{apx:ERIProps}.
\end{rem}
