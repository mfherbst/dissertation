\section{Full-CI}
\label{sec:FCI}

\nomenclature{$\NelecA, \NelecB$}{Number of $\alpha$/$\beta$ electrons.
Note $\NelecA + \NelecB = \Nelec$ and $\NelecA \ge \NelecB$ by convention.}
\nomenclature{$\Phi$}{Slater determinant or many-electron basis functions}
\nomenclature{$\psi_i$}{One-particle function,
typically $i$-th eigenfunction of the Fock operator, i.e.~a Hartree-Fock orbital}
\nomenclature{$\Ibas$}{Index set of the one-particle basis functions.
	Typically a set of multi-indices of quantum numbers.}
\nomenclature{$\Nbas$}{Cardinality of $\Ibas$, i.e.~the number of one-particle basis functions.}
\nomenclature{$\varphi_\mu$}{$\mu$-th one-particle basis function of the one-particle basis $\{\varphi_\mu\}_{\mu \in \Ibas}$}
\nomenclature{$\Iorb$}{Index set of computed SCF orbitals, typically $\{0, \ldots \Norb\}$}
\nomenclature{$\Norb$}{The number of computed SCF orbitals.
Note, that $\Norb \le \Nbas$.}

In this section we want to develop a numerical treatment
for solving the electronic Schrödinger equation \eqref{eqn:ElectronicSchrödinger}.

Form domain
\[ Q(\Op{H}_{\Nelec}) = H^1(\R^{3\Nelec}, \C) \cap \bigwedge^{\Nelec} L^2(\R^3, \C). \]

Dense subspace is
\[ \tilde{Q}(\Op{H}_{\Nelec}) =  \bigwedge^{\Nelec} H^1(\R^3, \C) \]
By definition
\[
	\tilde{Q}(\Op{H}_{\Nelec}) = \spacespan \left\{
		\bigwedge_{i=1}^{\Nelec} \psi_i
		\, \middle| \,
		\psi_i \in H^1(\R^{3\Nelec}, \C) \, \forall i = 1, \ldots, \Nelec
	\right\}
\]
By definition of the wedge product
\begin{align*}
	\psi_1 \wedge \psi_1 &= 0 & \psi_1 \wedge (\psi_1 + \psi_2) &= \psi_1 \wedge \psi_2 + \psi_1 \wedge \psi_2.
\end{align*}
In other words we can easily restrict ourselves to orthonormal
functions from $H^1(\R^{3\Nelec}, \C)$
and write equivalently
\[
	\tilde{Q}(\Op{H}_{\Nelec}) = \spacespan \left\{
		\bigwedge_{i=1}^{\Nelec} \psi_i
		\, \middle| \,
		\psi_i \in H^1(\R^{3\Nelec}, \C),
		\braket{\psi_i}{\psi_j} = \delta_{ij}
		\, \forall i,j = 1, \ldots, \Nelec
	\right\}.
\]
Note, that from the construction of $\tilde{Q}(\Op{H}_{\Nelec})$
as an exterior power of $H^1(\R^3, \C)$
it also follows that for two determinants
\begin{align*}
	\Phi &= \bigwedge_{i=1}^{\Nelec} \psi_i &
	\tilde{\Phi} &= \bigwedge_{i=1}^{\Nelec} \tilde{\psi}_i
\end{align*}
we have
\[
	\braket{\Phi}{\tilde{\Phi}}_{\tilde{Q}(\Op{H}_{\Nelec})}
	= \prod_{i=1}^{\Nelec} \braket{\psi_i}{\tilde{\psi}_i}_{H^1(\R^{3\Nelec}, \C)}
\]

Following the Ritz-Galerkin approach
as sketched in remark \vref{rem:DiscreteFormulation}
one of our tasks is to evaluate the sesquilinar form
$a(\Phi_1, \Phi_2)$ for $\Phi_1, \Phi_2 \in \tilde{Q}(\Op{H}_{\Nelec})$ or a subspace of $\tilde{Q}(\Op{H}_{\Nelec})$.
We now want to construct a general evaluation scheme
for this form for the operator $\Op{H}_{\Nelec}$.
For this let us consider the determinant
\[
	\Phi_0 \equiv \psi_1 \wedge \psi_2 \wedge \cdots \wedge \psi_i \wedge \cdots \wedge \psi_{\Nelec}
\]
which we will also call \newterm{reference determinant}
and for arbitrary $i,j \in \{1,\ldots,\Nelec\}$
with $i \neq j$
and arbitrary%
\footnote{We choose the symbols $\psi_a$, $\psi_b$ and $\psi_c$ to refer to these functions in order to resemble the notation known from quantum chemistry. It should be noted however, that $a$ and $b$ are not necessarily integer numbers, since the dimensionality of $H^1(R^{3\Nelec}$ is not countable.}
$\psi_a, \psi_b, \psi_c \in H^1(R^{3\Nelec}, \C)$
with
\begin{align*}
	\forall \alpha,\beta \in \{a,b,c\}: \quad
	\braket{\psi_\alpha}{\psi_\beta} &= 0
	&
	\forall \alpha \in \{a,b,c\}, i \in \{1,\ldots,\Nelec\}: \quad
	\braket{\psi_\alpha}{\psi_i} &= 0
\end{align*}
the determinants
\begin{align*}
	\Phi_i^a &= \psi_1 \wedge \psi_2 \wedge \cdots \wedge \psi_a \wedge \cdots \wedge \psi_{\Nelec} \\
	\Phi_{ij}^{ab} &= todo \\
	\Phi_{ijk}^{abc} &= todo
\end{align*}
i.e. where one, two or three single particle functions are replaced
compared to the reference determinant.

According to the Slater-Condon rules
\todoil{State that only up to two differences contribute}
more precisely
\todoil{Decompose $\Op{H}_{\Nelec}$ into terms and
show for them separately (one-electron and two-electron terms)}
\todoil{Present explicit expressions for the individual
	terms. Perhaps already introduce the J and K operators
and use the functional form and formalism of Cances}


\todoil{Sketch treatment if one-particle basis is introduced}
\todoil{Sketch treatment if $H^1(\R^3, \C)$ truncated to finite size.}
\todoil{Explain where the name FCI comes from}

Exact treatment as of section \vref{sec:ElectronicSchrödinger}




% Note that we need a more general setting here.
% Discuss how spin is included
% Slater determinants with spin
% FCI ansatz


% Slater-determinant is a way to construct many-particle basis set
% from one-particle basis set

% If all Slater determinants are included FCI results
% Need to compute certain integrals
% So-called Slater-Condon rules simplify problem
% Won't go into details here


\todoil{
	Quick note that real is possible: \\
For an atomic system, we can consider the nucleus to be located at the origin
of the coordinate system, which allows us to simplify
the Hamiltonian to \\
Argue that we can do the mathematical background in real function
arithmetic only and still use the results for complex input due to linearity.
$\Psi \in H^2(\R^{3d}, \R)$
}
