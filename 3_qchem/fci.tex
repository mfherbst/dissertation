\section{Full-CI}
\label{sec:FCI}

\nomenclature{$\NelecA, \NelecB$}{Number of $\alpha$/$\beta$ electrons.
Note $\NelecA + \NelecB = \Nelec$ and $\NelecA \ge \NelecB$ by convention.}
\nomenclature{$\Phi$}{Slater determinant or many-electron basis functions}
\nomenclature{$\psi_i$}{One-particle function,
typically $i$-th eigenfunction of the Fock operator, i.e.~a Hartree-Fock orbital}
\nomenclature{$\Ibas$}{Index set of the one-particle basis functions.
	Typically a set of multi-indices of quantum numbers.}
\nomenclature{$\Nbas$}{Cardinality of $\Ibas$, i.e.~the number of one-particle basis functions.}
\nomenclature{$\varphi_\mu$}{$\mu$-th one-particle basis function of the one-particle basis $\{\varphi_\mu\}_{\mu \in \Ibas}$}
\nomenclature{$\Iorb$}{Index set of computed SCF orbitals, typically $\{0, \ldots \Norb\}$}
\nomenclature{$\Norb$}{The number of computed SCF orbitals.
Note, that $\Norb \le \Nbas$.}

% Note that we need a more general setting here.
% Discuss how spin is included
% Slater determinants with spin
% FCI ansatz


\todo[inline,caption={}]{
which allows to define a set of normalised $\Nelec$-electron functions
\begin{equation}
	\mathcal{W}_{\Nelec} \equiv \left\{ \Psi \in \bigwedge^{\Nelec} L^2(\R^3, \C)
	\, \middle| \,
	\norm{\Psi}_{L^2(\R^{3 \Nelec}, \C)} = 1 \right\}.
	\label{eqn:DefSetSlaterDeterminant}
\end{equation}
These all satisfy the constraint imposed by the spin statistics theorem.
We will often use the symbol $\Phi$ to refer to Slater determinant from
from $\mathcal{W}_{\Nelec}$,
\ie an element in this set, which can be written as a wedge product string
$\wedgestring$ of single particle functions $\psi_i$.
}

\todo[inline,caption={}]{
	\begin{itemize}
		\item 
	Define reference determinant
	\[ \Phi_0 = \psi_1 \wedge \psi_2 \wedge \cdots \wedge \psi_i \wedge \cdots \wedge \psi_{\Nelec} \]
	and introduce excitation formalism
	\[ \Phi_i^a = \psi_1 \wedge \psi_2 \wedge \cdots \wedge \psi_a \wedge \cdots \wedge \psi_{\Nelec} \]
	\item
	Define
	\[ \Phi = \psi_1 \wedge \psi_2 \wedge \cdots \psi_{\Nelec} \]
	and
	\[ \tilde{\Phi} = \tilde{\psi}_1 \wedge \tilde{\psi}_2 \wedge \cdots \tilde{\psi}_{\Nelec} \]
	Discuss elements
	\[ \braket{\Phi}{\tilde{\Phi}}_{space} = \braket{\psi_1}{\tilde{\psi}_1}_{L^2(\R^3, \C)} \cdots \]
	(Follows by construction of tensor product Hilbert space)
	\item Discuss Slater-Condon rules and thus the terms of $a(\Phi, \tilde{\Phi})$
	\item Take basis functions $\varphi_\mu \in H^1(\R^3, \C)$
		to make sure the resulting determinants are in $Q(\Op{H})$.
		This is fine since all is dense.
\end{itemize}
}


% Slater-determinant is a way to construct many-particle basis set
% from one-particle basis set

% If all Slater determinants are included FCI results
% Need to compute certain integrals
% So-called Slater-Condon rules simplify problem
% Won't go into details here


\todoil{
	Quick note that real is possible: \\
For an atomic system, we can consider the nucleus to be located at the origin
of the coordinate system, which allows us to simplify
the Hamiltonian to \\
Argue that we can do the mathematical background in real function
arithmetic only and still use the results for complex input due to linearity.
$\Psi \in H^2(\R^{3d}, \R)$
}
