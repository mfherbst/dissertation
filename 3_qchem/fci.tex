\section{Full-CI}
\label{sec:FCI}

\nomenclature{$\NelecA, \NelecB$}{Number of $\alpha$/$\beta$ electrons.
Note $\NelecA + \NelecB = \Nelec$ and $\NelecA \ge \NelecB$ by convention.}
\nomenclature{$\Phi$}{Slater determinant or many-electron basis functions}
\nomenclature{$\psi_i$}{One-particle function,
typically $i$-th eigenfunction of the Fock operator, i.e.~a Hartree-Fock orbital}
\nomenclature{$\Ibas$}{Index set of the one-particle basis functions.
	Typically a set of multi-indices of quantum numbers.}
\nomenclature{$\Nbas$}{Cardinality of $\Ibas$, i.e.~the number of one-particle basis functions.}
\nomenclature{$\varphi_\mu$}{$\mu$-th one-particle basis function of the one-particle basis $\{\varphi_\mu\}_{\mu \in \Ibas}$}
\nomenclature{$\Iorb$}{Index set of computed SCF orbitals, typically $\{0, \ldots \Norb\}$}
\nomenclature{$\Norb$}{The number of computed SCF orbitals.
Note, that $\Norb \le \Nbas$.}


% Argue that we can do the mathematical background in real function
% arithmetic only and still use the results for complex input due to linearity.

$\Psi \in H^2(\R^{3d}, \R)$





\todo[inline,caption={}]{
	\begin{itemize}
		\item Idea
		\item Ansatz
	\end{itemize}
}


