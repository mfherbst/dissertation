\section{Full-CI}
\label{sec:FCI}
\nomenclature{$\Phi$}{Slater determinant $\bigwedge_{i=1}^{\Nelec} \psi_i$
or many-electron basis functions}
\nomenclature{$\psi_i$}{One-particle function,
typically $i$-th eigenfunction of the Fock operator, i.e.~a Hartree-Fock orbital}
\nomenclature{$\Ibas$}{Index set of the one-particle basis functions.
	Typically a set of multi-indices of quantum numbers.}
\nomenclature{$\Nbas$}{Cardinality of $\Ibas$, i.e.~the number of one-particle basis functions.}
\nomenclature{$\varphi_\mu$}{$\mu$-th one-particle basis function of the one-particle basis $\{\varphi_\mu\}_{\mu \in \Ibas}$}

\nomenclature{$\Op{K}$}{One-electron exchange operator}
\nomenclature{$\Op{J}$}{Effective Coulomb operator}
\defineabbr{ERI}{ERI\xspace}{Electron repulsion integrals}
\nomenclature{$(ij|kl)$}{Electron repulsion integrals in chemist's or Mullikan notation, see definition at \textcolor{red}{TODO}}

In this section we want to develop a numerical treatment
for solving the electronic Schrödinger equation \eqref{eqn:ElectronicSchrödinger}
under the Ritz-Galerkin projection ansatz of section \vref{sec:RitzGalerkin}.
In the previous section we developed the form domain of the electronic
Schrödinger operator $\Op{H}_{\Nelec}$ to be
\[ Q(\Op{H}_{\Nelec}) = H^1(\R^{3\Nelec}, \C) \cap \bigwedge^{\Nelec} L^2(\R^3, \C). \]
For simplifying our treatment in this section,
we will develop methods to sample only the dense subspace
\[ \tilde{Q}(\Op{H}_{\Nelec}) = \bigwedge^{\Nelec} H^1(\R^3, \C) \subset Q(\Op{H}_{\Nelec}). \]
Because of remark \vref{rem:ApproxBottomDiscrete}
this implies no loss of numerical accuracy.
By definition
\[
	\tilde{Q}(\Op{H}_{\Nelec}) = \spacespan \left\{
		\bigwedge_{i=1}^{\Nelec} \psi_i
		\, \middle| \,
		\psi_i \in H^1(\R^3, \C) \, \forall i = 1, \ldots, \Nelec
	\right\}.
\]
Since $H^1(\R^{3\Nelec}, \C)$ is separable, we can find a countable Hilbertian
basis set
\[ \set{B}_1 \equiv \{\psi_i\}_{i \in \N} \quad \text{with }
	\braket{\psi_i}{\psi_j}_{H^1(\R^{3\Nelec}, \C)} = \delta_{ij}
	\text{ and } \spacespan \set{B}_1 = H^1(\R^{3\Nelec}, \C),
\]
which we will sometimes call the \newterm{one-particle basis}.

Taking the properties of the wedge product \eqref{eqn:PropertiesExteriorProduct}
into account this allows us to deduce that $\tilde{Q}(\Op{H}_{\Nelec})$ can be equivalently
constructed by taking the span over all Slater determinants built by selecting
$\Nelec$ functions from $\set{B}_1$
\begin{equation}
	\label{eqn:FormDomainSlaterDeterminants}
	\tilde{Q}(\Op{H}_{\Nelec}) = \spacespan
	\left\{ \bigwedge_{i=1}^{\Nelec} \psi_i
	, \middle| \, \psi_i \in \set{B}_1 \, \forall i=1,\ldots,\Nelec
	\right\}.
\end{equation}
Let
\begin{align*}
	\Psi &= \bigwedge_{i=1}^{\Nelec} \psi_i &&\text{and}& \Xi &= \bigwedge_{i=1}^{\Nelec} \xi_i
\end{align*}
denote two arbitrary Slater determinants with
$\psi_i, \xi_i \in H^1(\R^{3\Nelec}, \C)$ for all $i \in 1,\ldots,\Nelec$.
We can define a rather natural inner product
on $\tilde{Q}(\Op{H}_{\Nelec})$ by requiring%
\begin{align}
	\braket{\Psi}{\Xi}_{\tilde{Q}(\Op{H}_{\Nelec})} = \det \mat{G}
	\qquad
	\text{with } G_{ij} = \braket{\psi_i}{\xi_j}_{H^1(\R^3, \C)} \forall i,j \in 1,\ldots,\Nelec
	\label{eqn:InnerProductFormDomain}
\end{align}
for all such Slater determinants.
The inner product value for other elements from $\tilde{Q}(\Op{H}_{\Nelec})$
is then constructed by the axioms shown in definition \vref{def:InnerProduct}.
\todoil{A more formal proof would be nice, especially for the completeness under this inner product.}

\begin{rem}[Reference and excited determinants]
	\label{rem:Determinants}
	In this remark we will rationalise why $\tilde{Q}(\Op{H}_{\Nelec})$ is separable
	and at the same time introduce a lot of terminology often used
	in the community of quantum chemistry.

	Let $\set{B}_1)^{\Nelec}$ be an arbitrary one-particle basis for $H^1(\R^3, \C)$
	and let $0 \neq \Phi_0 \in \tilde{Q}(\Op{H}_{\Nelec})$ be an arbitrary Slater determinant
	constructed from $\Nelec$ elements $\{\varphi\}_{i \in \Nelec} \subset \set{B}_1$
	such that
	\[
		\Phi_0 = \varphi_1 \wedge \varphi_2 \wedge \cdots \varphi_i \cdots \wedge \varphi{\Nelec}.
	\]
	Note that we can find this determinant due to \eqref{eqn:FormDomainSlaterDeterminants}.
	We shall refer to $\Phi_0$ as the \newterm{reference determinant}
	in the following.
	Let us now index the functions $\{\psi_i\}_{i \in \N} = \set{B}$ in such a way
	that the first $\Nelec$ coincide with $\{\varphi\}_{i \in \Nelec}$
	or in other words such that
	\[
		\Phi_0 = \psi_1 \wedge \psi_2 \wedge \cdots \psi_i \cdots \wedge \psi_{\Nelec}.
	\]
	We further define the index sets
	\begin{align*}
		\mathcal{I}_\text{occ} &= \{1, \ldots, \Nelec\} \\
		\mathcal{I}_\text{virt} &= \{ i \in \N | i > \Nelec \}.
	\end{align*}
	With reference to $\Psi_0$ we can define
	for each $i \in \mathcal{I}_\text{occ}$ and each $a \in \mathcal{I}_\text{virt}$
	a so-called singly \newterm{excited determinant}
	\[
		\Phi_i^a = \psi_1 \wedge \psi_2 \wedge \cdots \psi_a \cdots \wedge \psi_{\Nelec}
	\]
	by replacing the $i$-th function of $\set{B}$ by the $a$-th function of $\set{B}$.
	Analogously one may define doubly or higher excited determinants
	\begin{align*}
		\Phi_{ij}^{ab} &= \psi_1 \wedge \psi_2 \wedge \cdots \wedge \psi_a \cdots \psi_b
			\wedge \cdots \wedge \psi_{\Nelec} \\
		\Phi_{ijk}^{abc} &= \psi_1 \wedge \psi_2 \wedge \cdots \wedge \psi_a \cdots \psi_b
			\cdots \psi_c \wedge \cdots \wedge \psi_{\Nelec}
	\end{align*}
	where $i,j,k \in \mathcal{I}_\text{occ}$ and $a,b,c \in \mathcal{I}_\text{virt}$.
	In order to prevent an excited determinant to vanish~(two one-particle functions identical)
	or be insensible~(replaced the same one-particle function twice),
	we enforce further that $i < j < k < \cdots$ and $a < b < c < \cdots$.

	With the inner product defined in \eqref{eqn:InnerProductFormDomain} we notice
	for all $r,s \in \N$
	\[
		\braket{\Phi_0}{\Phi_r^s}_{\tilde{Q}(\Op{H}_{\Nelec})}
			= \braket{\psi_r}{\psi_s}_{\Op{H}_{\Nelec}} = \delta_{ia},
	\]
	since by construction all functions in $\set{B}_1$ are orthonormal to each other.
	In other words the set
	\[
		\set{B}_{\Nelec} \equiv
		\left\{
			\Phi_0, \Phi_i^a, \Phi_{ij}^{ab}, \Phi_{ijk}^{abc}, \cdots \right\}
	\]
	with the conditions on $i,j,k$ and $a,b,c$ sketched in this remark
	is an orthonormal basis for $\tilde{Q}(\Op{H}_{\Nelec})$.
	This makes $\tilde{Q}(\Op{H}_{\Nelec})$ separable.
\end{rem}
The set $\set{B}_{\Nelec}$ is sometimes called the \newterm{$\Nelec$-particle basis}
corresponding to $\set{B}_1$ or just the \newterm{many-particle basis}.
Notice that any choice of $\Phi_0$ or $\set{B}_1$ works for its construction.

\begin{rem}
	A consequence of the previous remark is that any
	function $\Psi \in \tilde{Q}(\Op{H}_{\Nelec})$
	can be expanded in normalised Slater determinants
	from the many-particle basis
	\begin{align}
		\label{eqn:ExpansionSlaterDeterminant}
		\Psi &= \sum_\mu c_\mu \Phi_\mu &\text{with } \Phi_\mu \in \set{B}_{\Nelec}\, \forall \mu.
	\end{align}
	If one is in the context of a particular reference determinants $\Phi_0$
	and a particular one-particle basis $\set{B}_1$ this can be written
	equivalently as
	\begin{equation}
		\label{eqn:CIExpansion}
		\Psi = \Phi_0 + c_i^a \Phi_i^a + c_{ij}^{ab} \Phi_{ij}^{ab} + c_{ijk}^{abc} \Phi_{ijk}^{abc}
		 + \cdots
	\end{equation}
	This expansion is commonly referred to as the \newterm{CI expansion},
	where CI stands for configuration interaction.
	This term will become more clear when we discuss electron
	correlation in section \vref{eqn:CIExpansion}.
\end{rem}

One main task when following the Ritz-Galerkin approach
as sketched in remark \vref{rem:DiscreteFormulation}
s to evaluate the sesquilinar form
$a(\Phi_1, \Phi_2)$ for $\Phi_1, \Phi_2 \in \set{B}_{\Nelec}$
for the $\Nelec$-electron basis we use.
What this boils down to is to discuss the  application of the sesquilinar form
onto pairs of Slater determinants,
either the reference or the excited determinants.
For this let us first discuss the terms in turn.

\begin{rem}[Slater-Condon rules]

% Introduce induvidual terms
% Introduce Slater-Condon rules

According to the Slater-Condon rules
\todoil{State that only up to two differences contribute}
more precisely
\todoil{Decompose $\Op{H}_{\Nelec}$ into terms and
show for them separately (one-electron and two-electron terms)}
\todoil{Present explicit expressions for the individual
	terms. Perhaps already introduce the J and K operators
and use the functional form and formalism of Cances}

\todoil{Do not go into details, do not proof them, refer to Szabo}
\end{rem}

% -------------------------

\begin{rem}[Full-CI]
\todoil{Sketch treatment if basis set for one-particle basis is introduced}
\todoil{Sketch treatment if $H^1(\R^3, \C)$ truncated to finite size.}
\todoil{Explain where the name FCI comes from}
Exact treatment as of section \vref{sec:ElectronicSchrödinger}
\todoil{Go into computational cost and algorithmic considerations}
\end{rem}
