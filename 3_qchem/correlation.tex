\section{Capturing electronic correlation}
\label{sec:Correlation}

\subsection{What does Hartree-Fock miss?}
% Think in terms of SCF procedure

% Averaging approach



% effective one-particle problem
%	-> show picture of cloud of interactions
%	-> show planetary system picture
% averaged interactions
% show dsa pictures



%still a good ansatz
\begin{itemize}
\item
$\psi$ HF orbitals or (SCF) orbital basis,
typically in FCI one uses the HF orbital basis as the one-particle basis functions for FCI
	\end{itemize}


\todo[inline,caption={}]{
	\begin{itemize}
		\item What is it and where does HF fail
		\item A couple of sentences about truncated CI
			\begin{itemize}
				\item Brief explanation what
				\item Mention size-consistency issues
			\end{itemize}
	\end{itemize}
}

Correlation energy: Difference HF to FCI
(even though HF not fully uncorrelated)

dynamic vs. static correlation


We will discuss so-called \newterm{truncated CI methods},
which only include determinants up to a certain
degree of excitations (double, triple, quadruple, \ldots)
in the context of so-called correlation methods in \vref{sec:Correlation}.


\subsection{Second order Møller-Plesset perturbation theory}
\todo[inline,caption={}]{
	\begin{itemize}
		\item Sketch Hylleras ansatz
		\item Show resulting expression
		\item No derivation, brief
	\end{itemize}
}

\subsection{Coupled-cluster theory}
\todo[inline,caption={}]{
	\begin{itemize}
		\item Explain ansatz
		\item Mention size-consistency, size-extensivity
		\item Show working exquations (needed later)
		\item No derivation, brief
	\end{itemize}
}

