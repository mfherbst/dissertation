\section{Capturing electronic correlation}
\label{sec:Correlation}

In the previous section we talked at length about the \HF approximation for solving
the non-relativistic electronic Schrödinger equation.
Since the search space for the variational minimisation is much more restricted compared
to the \FCI ansatz,
we necessarily make a larger approximation error in the \HF case.
Nevertheless it should be noted that \HF yields a rather good rank-1
approximation to the full Schrödinger problem,
where up to $99\%$ of the
\FCI energy~\cite{Jensen2007book} in a respective basis is obtained
at a fraction of the cost.
Unfortunately chemistry is about energy differences and not absolute energies.
For example to a good approximation,
chemical reactivity can be determined by looking at the
energy barrier between reactants and products,
\ie the difference in energy between the reactants
and the maximal energy,
which is obtained along the reaction path transforming them to products.
As the difference matters
getting $99\%$ of the absolute energy
typically still leads to much larger errors than $1\%$ for the reaction barrier.
One might therefore wonder
what part of the exact physical picture \HF is missing and how one could improve on that.

\subsection{What does Hartree-Fock miss?}
\label{sec:FailureHF}
Even though the
Fock operator \eqref{eqn:FockOperator} describes a many-electron system
it is a one-particle operator,
since it only acts on single-electron functions.
The many-particle aspect is only treated via the Coulomb term \eqref{eqn:OperatorCoulomb}
and the exchange term \eqref{eqn:OperatorExchange},
where the interaction with
other electrons is included in the form of integrals over
the electron density $\rho_{\Theta^0}$ or the density matrix $\gamma_{\Theta^0}$.
Overall an electron thus does not see the exact position of all its neighbours
via the Fock matrix,
but only some kind of an average electron field.
In this sense the \HF ansatz is sometimes called a \newterm{mean-field approximation}.
In the light of this the \SCF can be thought of as an adjustment procedure,
where the electronic arrangement in the form of the occupied \SCF orbitals $\{\psi_i\}_{i\in\Iocc}$
is adjusted until their generated mean field
is no longer changing this arrangement,
\ie is self-consistent.

\begin{figure}
	\centering
	\includeimage{3_qchem/planetary_systems}
	\caption[Real planetary system and mean-field model of it shown side-by-side]{
		Real planetary system and mean-field model
		in the spirit of the \HF approximation
		shown side-by-side.
		The mean-field picture on the right-hand side
		is shown from the point of view of the red planet,
		such that its neighbours are smeared out as thick black
		circles over their respective orbits.
		Adapted from \cite{Jensen2007book}.
	}
	\label{fig:HFplanets}
\end{figure}
To visualise this issue better,
let us consider in analogy a planetary system%
\footnote{The idea is taken from \cite{Jensen2007book}.},
where multiple planets are revolving around a central sun.
In the real system,
which is depicted on the left-hand side of figure \vref{fig:HFplanets},
the individual planets feel each other at all times
at their exact positions.
As a result their orbit around the sun is not a perfect circle but shows pronounced
wiggles due to the interaction between the planets.
In other words the motion of the planets around the sun is highly correlated.
In contrast to this the right-hand side
depicts the scenario drawn for the red planet in a \HF-like mean-field model.
Its neighbours are no longer visible at their exact positions
and the red planet thus only amounts to interact with
some sort of smeared out particle density,
where their position is averaged over their complete orbits.
This interaction is almost as strong at all points
and thus the mean-field orbit of the red planet is much more smooth.

In the electronic system the situation is similar
in sense that the behaviour of individual electrons is indeed very much correlated.
Due to its mean-field nature the \HF ansatz
largely misses the description of this so-called \newterm{electron correlation}%
\footnote{%
	Conventionally one calls the \HF treatment of a chemical system
	the \emph{uncorrelated} treatment of the electronic structure.
	This is not perfectly sound in my opinion,
	as for example the Pauli principle is fulfilled in \HF.
	This implies for example that two electrons of the same spin
	cannot occupy the same orbital,
	which implies in turn that the motion
	of electrons of the same spin is at least to this extend correlated.
}.
In fact one typically refers to the difference
\begin{equation}
	E_0^{\text{corr}} = E_0^{\text{FCI}} - E_0^{\text{HF}}
	\label{eqn:EnergyCorrelation}
\end{equation}
between the \HF and \FCI energies in a particular basis set
as the \newterm{correlation energy}.
As mentioned before $E_0^{\text{corr}}$ is typically rather small
compared to $E_0^{\text{HF}}$.
Nevertheless the effects of the electron-electron interaction are very important
for a proper description of the electronic structure of a chemical
system and can therefore not be
neglected~\cite{Helgaker2013,Szabo1996,Jensen2007book}.

In practice one sometimes divides correlation effects
into two subclasses.
The first kind, the so-called \newterm{dynamic correlation},
is the aforementioned failure of the \HF approximation
to treat the communal, correlated motion of electrons properly.
The second kind, \newterm{static correlation},
occurs if the number of Slater determinants,
which is available for the description of a degenerate or near-degenerate
state is not sufficient.
For the \HF approximation,
where only one determinant for the description of the ground state
is available,
this defect becomes apparent in situations
with a low-lying excited state, for example.
A classic example would be a molecule close to bond breaking.
In such a case the ground state resulting from a full CI treatment
has relevant contributions from more than one determinant.
As a result even the best restricted \HF ground state determinant misses a substantial
part of the full CI ground state and thus represents
a wrong description of the physics.
In the remainder of this discussion about electron correlation
we will ignore static correlation and assume that a single determinant
\HF ground state is already a pretty decent description of the electronic structure.
Detailed discussions of so-called multi-reference
or multi-configurational methods tackling static correlation
can be found for example in \cite{McWeeny1985,Fischer1986,Jensen2007book,Helgaker2013}.

\subsection{Truncated configuration interaction}
\label{sec:TruncatedCI}
In section \vref{sec:FCI} about the \FCI method we already
mentioned that the exact ground state $\Psi_0$ to the electronic Schrödinger equation
can be expressed as a CI expansion \eqref{eqn:CIExpansion}%
\footnote{
	In this section about correlation methods
	we will adhere to the usual index conventions
	where occupied indices are denoted with letters
	$i,j,k,l \in \Iocc$ and virtual indices with letters $a,b,c,d \in \Ivirt$,
	see remark \vref{rem:FCI} for details.
	To avoid clutter we will usually not indicate the index set in sums explicitly,
	\eg write $\sum_a$ instead of $\sum_{a \in \Ivirt}$.
}
\[
	\Psi
	= c_0 \Phi_0 + \sum_{ia} c_i^a \Phi_i^a
	+ \sum_{\substack{i<j \\a<b}} c_{ij}^{ab} \Phi_{ij}^{ab}
	+ \sum_{\substack{i<j<k \\ a<b<c}} c_{ijk}^{abc} \Phi_{ijk}^{abc}
	 + \cdots,
\]
starting from an arbitrary reference determinant $\Phi_0$.
A very natural choice for this is to take the reference
determinant to be the \HF ground state,
\ie the best single determinant for describing the electronic ground state.
In this way the
remaining contributions of the excited determinants $\Phi_i^a$,
$\Phi_{ij}^{ab}$, $\Phi_{ijk}^{abc}$, $\ldots$
can be expected to be small,
which makes it numerically more feasible to diagonalise
the \FCI matrix $\mat{A}_\text{FCI}$.
Furthermore this justifies truncating the CI expansion \eqref{eqn:CIExpansion}
prematurely to yield some sort of an intermediate approximation
between \HF and \FCI.
For example CISD, configuration interaction singles-doubles~\cite{Sherrill1999},
truncates the above expansion in a way that only singles and doubles
excitations are taken into account.
This would lead to the ansatz wave function
\[
	\Psi_0^\text{CISD}
	= \Phi_0 + \sum_{ia} c_i^a \Phi_i^a
	+ \sum_{\substack{i<j \\a<b}} c_{ij}^{ab} \Phi_{ij}^{ab}
\]
where one assumes the individual determinants
are normalised in a way that $c_0 = 1$.

Even though truncated CI methods are conceptionally very simple,
they are not used much any more for capturing dynamic correlation%
\footnote{%
	In contrast for dealing with statically correlated systems,
the multi-reference CI ansatz~\cite{Shamasundar2011} is used excessively.}.
The main reason for this is the so-called \newterm{size-consistency problem}.
Unlike \FCI the CISD energy of two molecular fragments
is in general not additive,
even if these fragments do not interact.
Put more mathematically one can show~\cite{Helgaker2013} the following:
If $E_A$ is the energy corresponding to the CISD ground state $\Psi_0^\text{CISD}$
for a molecule $A$
and $E_B$ is the analogous energy for another molecule $B$,
then the CISD ground-state energy $E_{AB}$
for a system consisting of both $A$ \emph{and} $B$
separated by an infinite distance is \emph{not} $E_A + E_B$.
One refers to this unphysical behaviour as \newterm{size-inconsistent}.
Including higher excitations
does not fix this problem,
such that all canonical truncated CI methods are size-inconsistent.
For the modelling of chemical reactions or even large molecules,
size-inconsistency is a major problem.
Nowadays better, size-consistent alternatives like the
coupled-cluster ansatz~(see below) exist
and are usually preferred.

\defineabbr{MP}{MP\xspace}{Møller-Plesset perturbation theory treatment of electron-electron correlation.
Typically followed by a number to indicate the order.}
\subsection{Second order Møller-Plesset perturbation theory}
\label{sec:MP}

Starting from the reasonable
assumption that the \HF ground state determinant $\Phi_0$
is a very good approximation to the exact electronic ground state
it is a sensible ansatz to
employ Rayleigh-Schrödinger perturbation theory~\cite{Szabo1996,Helgaker2013}
and correct perturbatively for the missing correlation contribution
to the energy as well as the wave function.
one partitions the electronic Schrödinger Hamlitonian \eqref{eqn:ElectronicHamiltonian}
into
\[ \Op{H}_{\Nelec} = \Op{H}^0 + \Op{H}^1, \]
\ie a zeroth order Hamiltonian $\Op{H}^0$,
which is easy to compute,
and the perturbation $\Op{H}^1$,
which is the part missed in $\Op{H}^0$,
assumed to be small.

One way this partitioning can be achieved is Møller-Plesset
perturbation theory~\cite{Moeller1934},
where the unperturbed operator is taken to be the
direct sum of $\Nelec$ Fock operators
at the orbital configuration corresponding to the \HF ground state $\Phi_0$,
\begin{align*}
	\Op{H}^0 &= \bigoplus_{i=1}^{\Nelec} \Op{F}_{\Theta^0},
	\intertext{
and the perturbation is}
	\Op{H}^1 &= \Op{V}_{ee} - \bigoplus_{i=1}^{\Nelec} \left( \Op{J}_{\Theta^0} + \Op{K}_{\Theta^0} \right),
\end{align*}
\ie whatever the \HF operator misses.
In the discretised setting of a finite-dimensional one-particle basis
$\{\varphi_\mu\}_{\mu\in\Nbas}$
one may easily derive the zeroth to second order
energy contributions~\cite{Szabo1996}
\begin{align*}
	E_0^0 &= \sum_i \varepsilon_i \\
	E_0^1 &= - \frac12 \sum_{ij} \eriAsym{ij}{ij}  \\
	E_0^2 &= \frac14 \sum_{ijab} \frac{\abs{\eriAsym{ij}{ab}}^2}
						{\varepsilon_i + \varepsilon_j - \varepsilon_a - \varepsilon_b}
\end{align*}
to the ground-state energy.
In other words up to zeroth order we obtain the sum of the orbital energies.
The first order correction accounts for the double counting
of the electron-electron interactions and recovers the \HF energy expression.
The first real improvement to \HF results
at \newterm{second order Møller-Plesset perturbation theory}~({\MP}2).
For reasons, which will become clear in the next section
one often introduces the so-called \newterm{$T_2$ amplitude}
\begin{equation}
	t_{ij}^{ab} \equiv \frac{\eriAsym{ij}{ab}}
		{\varepsilon_i + \varepsilon_j - \varepsilon_a - \varepsilon_b}
	\label{eqn:amplitudesMPtwo}
\end{equation}
and writes the {\MP}2 energy as
\begin{equation}
	E_0^\text{MP2} = E_0^\text{HF} + \frac14 \sum_{ijab} \cc{\eriAsym{ij}{ab}} t_{ij}^{ab}.
	\label{eqn:energyMPtwo}
\end{equation}
The \MP methods do in general have some issues as well.
Most notably the perturbation expansion of energies does in general
not converge~\cite{Helgaker2013},
making it hard to properly justify these methods from a mathematical basis.
In practice {\MP}2 is still vividly employed,
mainly because it gives a decent guess
towards the exact energy of the electronic ground state
at manageable computational cost%
\footnote{Using sensible approximations linear-scaling {\MP}2 is possible%
~\cite{Zalesny2011}.}.
The other \MP methods on the other hands
are nowadays used only rarely.

One should mention that due to its perturbative nature,
there is no guarantee that $E_0^\text{MP2} \geq E_0$,
the exact ground-state energy of the electronic Schrödinger equation \eqref{eqn:ElectronicSchrödinger}.
In the community of quantum chemistry one often refers
to this fact as {\MP}2 being non-variational.
This saying is, however, a bit inaccurate,
since the method {\MP}2 is indeed variational
in the sense of the Courant-Fischer theorem \eqref{eqn:CourantFischer},
namely that larger basis sets will always lead a lower-energy {\MP}2 ground state,
which is furthermore closer to the exact {\MP}2 ground-state wave function.
This is not that much apparent in the outlined derivation,
but can be seen from an alternative route
employing the Hylleraas functional~\cite{Helgaker2013}.
In contrast it is not variational in the sense that
a larger basis yields an {\MP}2 energy
which approaches $E_0$ from above.

\subsection{Coupled-cluster theory}
\defineabbr{CC}{CC\xspace}{Coupled cluster}
\label{sec:CC}

The main idea of coupled-cluster theory
is to employ a more elaborate ansatz for the ground-state wave function
with the overall aim to
reach a size-consistent method.
In this work coupled-cluster only plays a minor role.
This section will therefore be limited to the absolutely necessary
steps to get the rough idea.
For a more thorough introduction the reader is directed
to the excellent review by \citet{Crawford2007}
as well as numerous other works~\cite{Helgaker2013,Hodecker2016}
dealing with the topic.

\noindent
In coupled-cluster theory one starts from the so-called exponential ansatz
\begin{equation}
	\Psi^\text{CC} = \exp(\op{T}) \Phi_0
	\label{eqn:AnsatzCC}
\end{equation}
to generate the coupled-cluster wave function $\Psi^\text{CC}$
from a \HF ground state reference determinant $\Phi_0$.
In this equation
\begin{equation}
	\op{T} = \op{T}_1 + \op{T}_2 + \cdots + \op{T}_{\Nelec}
	\label{eqn:ExcitationCC}
\end{equation}
is the \newterm{excitation operator}
consisting of all singles excitations
\begin{align*}
	\op{T}_1 &= \sum_{ia} t_i^a \op{\tau}_i^a
\intertext{with $\op{\tau}_i^a$ defined, such that $\Phi_i^a = \op{\tau}_i^a \Phi_0$,
all doubles excitations}
	\op{T}_2 &= \sum_{\substack{i<j \\a<b}} t_{ij}^{ab} \op{\tau}_{ij}^{ab}
		&&\text{with}& \Phi_{ij}^{ab} &= \op{\tau}_{ij}^{ab} \Phi_0, \\
\intertext{all triples}
	\op{T}_3 &= \sum_{\substack{i<j<k \\ a<b<c}} t_{ijk}^{abc} \op{\tau}_{ijk}^{abc}
		&&\text{with}& \Phi_{ijk}^{abc} &= \op{\tau}_{ijk}^{abc} \Phi_0, \\
\end{align*}
and so forth.
In these sums the
coefficients $t_i^a$, $t_{ij}^{ab}$, $t_{ijk}^{abc}$ and so forth
are called \newterm{cluster amplitudes}.
In a similar notation to \eqref{eqn:ExpansionSlaterDeterminant}
the excitation operator is often directly written as a sum of the operators
$\op{\tau}_i^a$, $\op{\tau}_{ij}^{ab}$, $\op{\tau}_{ijk}^{abc}$ \ldots, namely as
\begin{equation}
	\op{T} = \sum_\mu t_\mu \op{\tau}_\mu,
	\label{eqn:ExcitationCCsum}
\end{equation}
where $\mu$ is an appropriately chosen multi-index and the sum
is implicitly taken to have sensible limits.

If we allow all possible excitations in \eqref{eqn:ExcitationCC},
\ie do not truncate the sum,
the space spanned by all possible coupled-cluster wave functions $\Psi^\text{CC}$
is exactly equivalent to the space of all Slater determinants,
namely the form domain%
\footnote{Recall the definition in \eqref{eqn:FormDomainAllFunctions}.}
$\tilde{Q}(\Op{H}_{\Nelec})$.
Without truncation \CC is thus equivalent to \FCI,
moreover the exponential ansatz in this case just provides
an alternative to the standard parametrisation of $\tilde{Q}(\Op{H}_{\Nelec})$
in terms of the CI expansion~(see remark \vref{rem:Determinants}).

In the corresponding discretised setting,
$\Phi_0$ is the solution to the discretised
\HF problem~(section \vref{sec:DiscreteHF}).
In a similar fashion to full CI one would expect
a good ansatz for obtaining a \CC approximation to
the ground state of the electronic Schrödinger equation
to use the Ritz-Galerkin ansatz of remark \vref{rem:DiscreteFormulation}.
In other words, one would attempt to solve the variational minimisation problem
\begin{equation}
E_0 \leq E_0^\text{CC} = \inf_{\{t_\mu\}_{\mu}} \frac{\braket{\exp(\op{T}) \Phi_0}{\Op{H}_{\Nelec} \exp(\op{T}) \Phi_0}_{\Nelec}}
	{\braket{\exp(\op{T}) \Phi_0}{\exp(\op{T}) \Phi_0}_{\Nelec}},
	\label{eqn:VariationalCC}
\end{equation}
where there resulting minimising amplitudes give the \CC ground state
wave function corresponding to the minimal ground-state energy $E_0^\text{CC}$.
Without truncation of \eqref{eqn:ExcitationCC}
this is again equivalent to discretised full CI.
Even with truncation to, for example, $\op{T} = \op{T}_1 + \op{T}_2$,
equation \eqref{eqn:VariationalCC} is intractable to solve.
The reason for this is the number of parameters in the problem.
Even with truncation the exponential ansatz
$\exp(\op{T}) \Phi_0$ generates \emph{every} Slater determinant,
such that \eqref{eqn:VariationalCC} yields
a high-dimensional, non-linear problem,
where products of the individual amplitudes $\{t_\mu\}_{\mu}$ often occur
in the resulting system of equations.

For this reason one usually employs a different projection approach,
which shall only be sketched here%
\footnote{Notice, that some mathematical rigour is dropped here.
	Since $\Op{H}_{\Nelec} \exp(\op{T}) \Phi_0$ is only defined properly
	if $\Phi_0 \in H^2(\C^{3\Nelec}, \C^{2\Nelec})$,
	which is not true in general.
	Even in the discretised case one may choose a one-particle basis $\{\varphi_\mu\}_{\mu\in\Ibas}$,
	where some functions are not members of $H^2(\C^{3\Nelec}, \C)$,
	such that $\Phi_0 \not\in H^2(\C^{3\Nelec}, \C^{2\Nelec})$ is possible.
}.
If one plugs the exponential ansatz directly into the electronic Schrödinger equation
\eqref{eqn:ElectronicSchrödinger} for the ground state one obtains
\begin{equation}
	\Op{H}_{\Nelec} \exp(\op{T}) \Phi_0 = E_0^\text{CC} \exp(\op{T}) \Phi_0,
	\label{eqn:CCschrödinger}
\end{equation}
where $E_0^\text{CC}$ is the coupled-cluster ground-state energy.
By a simple rearrangement this can be written as
\begin{equation}
	E_0^\text{CC} = \mbra{\Phi_0} \Op{H}_T \mket{\Phi_0}_{\Nelec}
	\label{eqn:CCenergy}
\end{equation}
where we introduced the similarity-transformed Hamiltonian
\[ \Op{H}_T = \exp(-\op{T}) \Op{H}_{\Nelec} \exp(\op{T}). \]
For making use of equation \eqref{eqn:CCenergy}
at all the unknown amplitudes $\{t_\mu\}_{\mu}$ still need to be found.
This is done by projecting \eqref{eqn:CCschrödinger}
onto determinants $\exp(-\op{T}) \Phi_\mu = \exp(-\op{T}) \op{\tau}_\mu \Phi_0$,
which yields equations
\begin{equation}
	\mbra{\Phi_\mu} \Op{H}_T \mket{\Phi_0}_{\Nelec} = 0
	\label{eqn:CCamplitudes}
\end{equation}
one for each $\mu$.
In truncated \CC methods, where only some of the terms of \eqref{eqn:ExcitationCC} are kept,
we can use \eqref{eqn:ExcitationCCsum} to generate exactly one equation
for each amplitude $\mu$.
In other words, the $\mu$ in \eqref{eqn:CCamplitudes} is just taken to run over the same
index range as in the expansion \eqref{eqn:ExcitationCCsum} for the truncated
excitation operator $\op{T}$.

\defineabbr{CCD}{CCD\xspace}{Coupled-cluster doubles}
Numerically solving for the \CC amplitudes in \eqref{eqn:CCamplitudes}
amounts to a root-finding problem,
where the parameters are the set of all amplitudes $\{t_\mu\}_{\mu}$.
This is typically approached by minimising the residuals
\begin{equation}
	r_\mu = \mbra{\Phi_\mu} \Op{H}_T \mket{\Phi_0}_{\Nelec}
	\label{eqn:CCresidual}
\end{equation}
iteratively until numerically $r_\mu = 0$ for all $\mu$.
Even though this problem is easier compared to the
variational \CC ansatz,
the working equations resulting from the expressions \eqref{eqn:CCresidual}
are typically all but simple.
For example, let us consider one of the simplest coupled-cluster approaches, where
\[ \op{T} = \op{T}_2 = \sum_{\substack{i<j \\a<b}} t_{ij}^{ab} \op{\tau}_{ij}^{ab}, \]
called coupled-cluster doubles~(\CCD).
A proper derivation~\cite{Hurley1976,Bartlett1978,Crawford2007,Hodecker2016}
starting from \eqref{eqn:CCamplitudes}
yields the equations
\begin{equation}
\begin{aligned}
	r_{ij}^{ab}
		&= \eriAsym{ab}{ij} \\
		%
		&+ \sum_e f_{ae} \, t_{ij}^{eb}
		 - \sum_e f_{be} \, t_{ij}^{ea}
		 - \sum_m f_{mi} \, t_{mj}^{ab}
		 + \sum_m f_{mj} \, t_{mi}^{ab} \\
		%
		&+ \frac12 \sum_{mn} \eriAsym{mn}{ij} \, t_{mn}^{ab}
		+ \frac12 \sum_{ef} \eriAsym{ab}{ef} \, t_{ij}^{ef} \\
		%
		&+ \sum_{me} \eriAsym{mb}{ej} \, t_{im}^{ae}
		 - \sum_{me} \eriAsym{mb}{ei} \, t_{jm}^{ae} \\
		&- \sum_{me} \eriAsym{ma}{ej} \, t_{im}^{be}
		 + \sum_{me} \eriAsym{ma}{ei} \, t_{jm}^{be} \\
		%
		&- \frac12 \sum_{mnef} \eriAsym{mn}{ef} \, t_{mn}^{af} \, t_{ij}^{eb}
		 + \frac12 \sum_{mnef} \eriAsym{mn}{ef} \, t_{mn}^{bf} \, t_{ij}^{ea} \\
		&- \frac12 \sum_{mnef} \eriAsym{mn}{ef} \, t_{in}^{ef} \, t_{mj}^{ab}
		 + \frac12 \sum_{mnef} \eriAsym{mn}{ef} \, t_{jn}^{ef} \, t_{mi}^{ab} \\
		&+ \frac14 \sum_{mnef} \eriAsym{mn}{ef} \, t_{mn}^{ab} \, t_{ij}^{ef}
		 + \frac12 \sum_{mnef} \eriAsym{mn}{ef} \, t_{im}^{ae} \, t_{jn}^{bf} \\
		&- \frac12 \sum_{mnef} \eriAsym{mn}{ef} \, t_{jm}^{ae} \, t_{in}^{bf}
		 - \frac12 \sum_{mnef} \eriAsym{mn}{ef} \, t_{im}^{be} \, t_{jn}^{af} \\
		&+ \frac12 \sum_{mnef} \eriAsym{mn}{ef} \, t_{jm}^{be} \, t_{in}^{af}
\end{aligned}.
	\label{eqn:CCDworking}
\end{equation}
for the \CCD residual $r_{ij}^{ab}$.
They involve multiple contractions over the
antisymmetrised \ERI tensor from remark \vref{rem:ERI},
the amplitudes $t_{ij}^{eb}$ and elements of the Fock matrix $\mat{f}$ in the \SCF orbital basis.
This latter matrix is defined as
\[ \mat{f} = \tp{\mat{C}_F} \mat{F} \mat{C}_F \in \C^{\Norb\times\Norb}.\]
If the canonical \HF ansatz of \eqref{eqn:HFequations} is used, $\mat{f}$ will be diagonal
and equivalent to $\diag(\varepsilon_1, \varepsilon_2, \ldots, \varepsilon_{\Norb})$.
The corresponding \CCD energy expression
\begin{equation}
	E_\text{CCD} = \frac14 \sum_{ijab} \eriAsym{ij}{ab} t_{ij}^{ab}.
	\label{eqn:CCDenergy}
\end{equation}
can be obtained by simplifying \eqref{eqn:CCenergy}.
Since the rank-4 tensor $t_{ij}^{ab}$ occurs in the expression for the
$\op{T}_2$ excitation operator,
this tensor is usually called the $T_2$-amplitudes tensor as well.
Comparing the structure of \eqref{eqn:energyMPtwo} and \eqref{eqn:CCDenergy},
the name of the expression \eqref{eqn:amplitudesMPtwo} in {\MP}2 finally becomes apparent.

For higher-order methods like CCSD, where $\op{T} = \op{T}_1 + \op{T}_2$,
or CCSDT, were $\op{T}_3$ is considered on top,
the expressions for the working equations \eqref{eqn:CCresidual}
are even more involved.
In turn these methods become rather expensive as well,
\eg CCSD scales as $\bigO(\Nbas^6)$ and CCSDT as $\bigO(\Nbas^8)$.
Nevertheless, \CC methods are very popular and widely adopted in quantum chemistry.
Firstly because they converge systematically towards the \FCI energy
as higher and higher excitations are considered in \eqref{eqn:ExcitationCC}
and secondly because \emph{all} \CC methods are size-consistent
--- unlike the truncated CI methods we mentioned above.
One particular approach named CCSD(T),
where the triples excitations are perturbatively added on top of CCSD,
has been named the \emph{gold standard} of chemistry
as it generally yields highly accurate results with an expensive,
but an acceptable scaling of $\bigO(\Nbas^7)$,
where the most costly $\bigO(\Nbas^7)$ is not iterative.
Recent improvements~\cite{Riplinger2013} within the framework
of pair-natural orbital approaches,
has brought down the apparent scaling of CCSD(T) to linear,
allowing a CCSD(T) treatment of complete proteins.

\defineabbr{ADC}{ADC\xspace}{Algebraic-diagrammatic construction}
\subsection{Excited states methods}
\label{sec:ExcitedStates}
In most of our discussion up to this point we have only focused
on obtaining an approximation to the ground state of the electronic Schrödinger
equation.
In some applications of electronic structure theory, however,
electronic excitations play a role.
Examples include the interaction of UV photons or photons of visible light
with the electronic structure in a dye or a solar cell
or more generally any photo-activated chemical reaction.
Whenever this is the case the modelling of multiple electronic states on an equal footing
is required.

For \FCI or the truncated CI methods,
this can be achieved without additional modification
by solving the respective full or truncated CI matrix
for more than one eigenpair.
All but the lowest are excited states.
These are not the only excited states methods in existence.
In fact to each of the other methods we have discussed
so far one is able to appoint at least one analogue~\cite{Dreuw2005}.
For example for Hartree-Fock, there is configuration-interaction singles~(CIS)
or time-dependent \HF~(TDHF) and
for coupled-cluster there
are the equation-of-motion and linear-response coupled-cluster theories%
~\cite{Schirmer2010,Sekino1984}.
Last but not least, the algebraic-diagrammatic construction scheme~(\ADC)
for the polarisation propagator at various orders~\cite{Schirmer1982,Trofimov1999}
can be seen as a CI-like scheme on top of a Møller-Plesset ground state.
Its excited states are generally in good agreement with the \MP
description of the ground state.
