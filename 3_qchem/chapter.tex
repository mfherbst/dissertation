\chapter{Solving the many-body electronic Schrödinger equation}
\chaptermark{Solving the many-body elec. Schrödinger eqn.}
\label{ch:qchem}
\chapquote{%
The word ``reality'' is also a word,
a word which we must learn to use correctly.}
{Niels Bohr~(1885--1962)} \\
\chapquote{%
I am convinced that despite his slightly positivist language,
Bohr believes as much as we do in the reality of phenomena of
which he speaks, and then the difference between the views
of Bohr and mine is more a difference of language than a
difference of content.}{Vladimir Fock~(1898--1974)}

\noindent
This chapter is concerned with the generalisation
of the one-electron hydrogen-like Schrödinger Hamiltonian \eqref{eqn:OpHydrogen}
\[
	\Op{H}_H = -\frac12 \Delta - \frac{Z}{r},
\]
which we discussed in section \vref{sec:HydrogenAtom},
towards the many-body problems of quantum chemistry.

Even though the spectral properties are very similar to the hydrogen-like case,
solving the associated time-independent
Schrödinger equation \eqref{eqn:TISE} analytically for
any but the most trivial problems is impossible.
Most of this chapter will therefore be devoted to
discussing approximations to the exact \TISE
as well as numerical approaches for solving such approximations
in practice.

\section{Many-body Schrödinger equation}
\label{sec:ManyBodyTISE}
\defineabbr{BO}{BO\xspace}{Born-Oppenheimer approximation, see section \vref{sec:BO}}
\nomenclature{$\Psi$}{State of a (many body) quantum system,
	typically the \emph{exact} ground state solution to the many-electron,
	electronic TISE, see section \vref{sec:BO}}
\nomenclature{$\Nelec$}{Total number of electrons}
\nomenclature{$M$}{Number of nuclei}
\nomenclature{$\Op{H}_{\Nelec}$}{Electronic Hamiltonian
	for a $\Nelec$-electron system, see section \vref{sec:BO}}
\nomenclature{$\vec{x}$}{Vector in $\R^{3 \Nelec}$,
which typically specifies the positions of all electrons of the system.}
\nomenclature{$\vec{r}$}{Vector in $\R^3$, which specifies the position of an electron in space}

Let us consider a chemical system consisting of $M$ nuclei and $\Nelec$ electrons.
We take the nuclei to be located at mass-scaled\footnote{
	If $\vec{\tilde{R}}_A$ is the Cartesian coordinate
	of the $A$-th nucleus with mass $M_A$, then
	the mass-scaled coordinates are given as
	$\vec{R}_A = \sqrt{M_A} \, \vec{\tilde{R}}_A$.
} coordinates
$\{\vec{R}_A\}_{A = 1, 2, \ldots, M} \subset \R^3$
with corresponding charges
$\{Z_A\}_{A = 1, 2, \ldots, M}$.
The electron positions are denoted by the (Cartesian) coordinates
$\{\vec{r}_i\}_{i = 1, 2, \ldots, \Nelec} \subset \R^3$.
Following the correspondence to classical mechanics (see section \vref{sec:IntroQM})
we can construct the many-body Hamiltonian
on the Hilbert space $L^2(\R^L, \C)$ with dimensionality $L = 3M+3\Nelec$ as
\begin{equation}
	\Op{H}^\text{MB} = \Op{T}_e + \Op{T}_n + \Op{V}_{nn} + \Op{V}_{ne} + \Op{V}_{ee}.
	\label{eqn:ManyBodyHamiltonian}
\end{equation}
In this expression we introduced the
nuclear-nuclear, electronic-electronic and nuclear-electronic
Coulombic interaction potentials
\begin{align}
	\label{eqn:ManyBodyHamiltonianPotential}
	\Op{V}_{nn} &= \sum_{A=1}^M \sum_{B=A}^M
		\frac{Z_A Z_B}{\norm{\vec{R}_A - \vec{R}_B}_2} &
	\Op{V}_{ee} &= \sum_{i=1}^{\Nelec} \sum_{j=i+1}^{\Nelec}
		\frac{1}{\norm{\vec{r}_i - \vec{r}_j}_2} \\
	\nonumber
	\Op{V}_{ne} &= \sum_{A=1}^M \sum_{i=1}^{\Nelec} \frac{Z_A}{\norm{\vec{R}_A - \vec{r}_i}_2},
\end{align}
respectively.
Furthermore we used the electronic and nuclear kinetic energy operators
\begin{align}
	\label{eqn:ManyBodyHamiltonianKinetic}
	\Op{T}_e &= -\frac12 \sum_{i=1}^{\Nelec} \Delta_{\vec{r}_i} &
	\Op{T}_n &= -\frac12 \sum_{A=1}^M \Delta_{\vec{R}_A},
\end{align}
with the shorthand
\[ \Delta_{\vec{q}} = \sum_{\alpha=1}^3 \frac{\partial^2}{\partial q_\alpha^2} \]
for the Laplace operator with respect to particle coordinates $\vec{q}$.
If we take the domain $D(\Op{H}_\text{MB}) = H^2(\R^L, \C)$
this operator can be made self-adjoint~\cite{Kato1951}.
It is furthermore bounded below~\cite{Kato1951}
with a couple of discrete states below the essential spectrum.

The operator $\Op{H}_\text{MB}$ is the fundamental object
the field of \newterm{quantum chemistry} investigates.
Its properties allow for a full
(non-relativistic) quantum-mechanical description
of a chemical system.
This includes important properties like stable chemical structures
or reactivity, with respect to other molecules as well as external potentials.
As discussed in section \vref{sec:SpectralTakeAway}
a consequence of the laws of thermodynamics is,
that in many cases one already gets
a reasonable idea about the chemical properties of matter
if only the lowest-energy, discrete eigenstates of the relevant
many-body Hamiltonian $\Op{H}_\text{MB}$ are determined.

Let us use the vectors $\vec{x} \in \R^{3\Nelec}$ and $\vec{X} \in \R^{3M}$, defined as
\begin{align}
	\rtp{\vec{x}} &\equiv \left(\rtp{\vec{r}_1}, \rtp{\vec{r}_2}, \ldots, \rtp{\vec{r}_{\Nelec}}\right)
	&
	\rtp{\vec{X}} &\equiv \left(\rtp{\vec{R}_1}, \rtp{\vec{R}_2}, \ldots, \rtp{\vec{R}_{M}}\right),
	\label{eqn:DefAllCoords}
\end{align}
to refer to all electronic or all nuclear coordinates, respectively.
Taking $I$ to denote an appropriate multi-index of quantum numbers,
our problem from the previous paragraph can be reformulated
as finding those eigenstates $\Psi^\text{MB}_I \in H^2(\R^L, \C)$
with lowest corresponding energies $E^\text{MB}_I \in \R$
by the means of solving the time-independent Schrödinger equation
\begin{equation}
	\Op{H}^\text{MB} \Psi_I^\text{MB}(\vec{X}, \vec{x})
	= E_I^\text{MB} \Psi_I^\text{MB}(\vec{X}, \vec{x}).
	\label{eqn:TISEManyBody}
\end{equation}
Solving this equation analytically is not possible in general.
Already for the Helium atom, a 3-body problem, clever approximations are needed
to get anywhere~\cite{Hylleraas1929}.
But even numerically \eqref{eqn:TISEManyBody} is intractable to solve
without further approximations.

Let us illustrate this claim by an example.
In water, \ce{H2O}, we have $3$ nuclei and $10$ electrons.
The dimensionality of the problem is thus $L = 3 \cdot 13 = 39$.
In the numerical approach we introduces in section \vref{sec:Projection}
the evaluation of the inner product
\begin{equation}
	\braket{\Psi}{\Phi} \equiv \int_{\R^{3M}} \int_{\R^{3\Nelec}}
		\cc{\Psi}(\vec{x}, \vec{X}) \Phi(\vec{x}, \vec{X})
	\D \vec{x} \D \vec{X}
	\label{eqn:MBInnerProduct}
\end{equation}
between two functions $\Psi$ and $\Phi$ from the
underlying Hilbert space $L^2(\R^L, \C)$ appears rather prominently.
Most notably the computation of the sesquilinear form $a(\slot,\slot)$
in order to build the discretisation matrix in \eqref{eqn:DiscretisedEigenproblem}
boils down to computing such integrals.
The numerical evaluation of \eqref{eqn:MBInnerProduct}
implies a sampling of the $L$-dimensional space $\R^L$ in some way or another.
Even for an extremely sophisticated discretisation method
or a well-designed quadrature scheme we will probably need of the order of
$10$ sampling points per dimension.
For a 39-dimensional problem, like our water molecule,
this makes on the order of $10^{39}$ sampling points overall.
If we want a single integration to finish within the lifetime of a human being,
say $100$ years,
the evaluation of the
integration kernel $\Psi(\vec{x}, \vec{X}) \Phi(\vec{x}, \vec{X})$
may take no more than some $10^{-30}$ seconds,
which is impossible due to the physical limitations inside a general purpose computer.

Certainly one could probably find even more clever methods in some cases,
but the example illustrates the so-called \newterm{curse of dimensionality} rather well.
For a general quantum-chemical investigation of matter
one needs to develop approximate methodologies.

\section{Born-Oppenheimer approximation}
\label{sec:BO}
The masses of electrons and nuclei differ by orders of magnitude.
The ratio between the mass of a proton and the electron masses
is already around $1836$ and this ratio increases
further across the table.
Already for the elements of the first period,
this value is at least of the order of $10^4$.
This justifies an approximative treatment,
where we assume the motion of the electrons
and the motion of the nuclei to happen at different timescales.

For this let us consider at first
a simplified version of \eqref{eqn:ManyBodyHamiltonian},
namely the \newterm{electronic Hamiltonian}
\[
	\Op{H}^\text{elec} \equiv \Op{H}^\text{MB} - \Op{T}_n
	= \Op{T}_e + \Op{V}_{ne} + \Op{V}_{ee} + \Op{V}_{nn}.
\]
This operator is constructed form the full many-body Hamiltonian
by neglecting the nuclear kinetic energy operator $\Op{T}_n$ completely.
Introducing the short hand notation
\nomenclature{$r_{ij}$}{short hand for $\norm{\vec{r}_i - \vec{r}_j}_2$}
\begin{align*}
r_{AB} &\equiv \norm{\vec{R}_A - \vec{R}_B}_2, &
r_{iA} &\equiv \norm{\vec{r}_i - \vec{R}_A}_2, &
r_{ij} &\equiv \norm{\vec{r}_i - \vec{r}_j}_2, &
\end{align*}
we can write it as
\begin{equation}
	\Op{H}^\text{elec}
	= - \frac12 \sum_{i=1}^{\Nelec} \Delta_{\vec{r}_i}
	+ \sum_{A=1}^M \sum_{i=1}^{\Nelec} \frac{Z_A}{r_{iA}}
	+ \sum_{i=1}^{\Nelec} \sum_{j=i+1}^{\Nelec} \frac{1}{r_{ij}}
	\  + \  %
	\sum_{A=1}^{M} \sum_{B=A+1}^{M} \frac{1}{r_{AB}}.
	\label{eqn:ElectronicFullHamiltonian}
\end{equation}
Even though $\Op{H}^\text{elec}$ still depends on the nuclear coordinates $\vec{X}$,
one could interpret the elements of the vector
$\vec{X}$ not as coordinates,
but much rather as parameters for the potential operators $\Op{V}_{ne}$ and $\Op{V}_{nn}$.
Physically this means that $\Op{H}^\text{elec}$ describes a chemical system
where the nuclei are clamped at well-defined points in space.
Sometimes we will write $\Op{H}^\text{elec}(\vec{X})$
in order to make the parametrisation of $\Op{H}^\text{elec}$ with respect to $\vec{X}$
visible.

\newcommand{\Iel}{I_\text{e}}
\newcommand{\Inu}{I_\text{n}}
Without going into details at the moment,
let us assume that $\Op{H}^\text{elec}$ becomes self-adjoint
inside a suitable domain.
With appropriate multi-indices $\Iel$
we can thus find its eigenpairs $(E_{\Iel}, \Psi^\text{elec}_{\Iel})$
via the \newterm{electronic Schrödinger equation}
\begin{equation}
	\Op{H}^\text{elec}(\vec{X}) \Psi^\text{elec}_{\Iel}(\vec{X}, \vec{x})
	= E^\text{elec}_{\Iel}(\vec{X})
		\Psi^\text{elec}_{\Iel}(\vec{X}, \vec{x}).
	\label{eqn:ElectronicSchrödinger}
\end{equation}
Originating from the dependence of $\Op{H}^\text{elec}(\vec{X})$
towards the nuclear coordinates,
we can think of the resulting \newterm{electronic energies}
$E^\text{elec}_{\Iel}(\vec{X})$
and \newterm{electronic wave function}s $\Psi^\text{elec}_{\Iel}(\vec{X}, \vec{x})$
to be dependent on $\vec{X}$ as well.
Typically one uses the term \newterm{electronic state}
to refer to $\Psi^\text{elec}_{\Iel}(\vec{X}, \vec{x})$.

With the electronic states at hand we are able to formulate
the framework of the \newterm{Born-Oppenheimer approximation},
which consists of the following two assumptions:
\begin{itemize}
	\item Each eigenstate of \eqref{eqn:ManyBodyHamiltonian} may be written
		by a factorisation
	\begin{equation}
		\Psi^\text{MB}_I(\vec{X}, \vec{x}) \equiv \Psi^\text{MB}_{\Iel\Inu}(\vec{X}, \vec{x})
		\simeq \Psi^\text{elec}_{\Iel}(\vec{X}, \vec{x})
		\Psi^\text{nuc}_{\Inu}(\vec{x}),
		\label{eqn:BOFactorisation}
	\end{equation}
	where the multi-indices are related by $I \equiv (\Iel, \Inu)$.
	$\Psi^\text{elec}_{\Iel}(\vec{X}, \vec{x})$ is a solution
	to the electronic Schrödinger equation \eqref{eqn:ElectronicSchrödinger}
	and the \newterm{nuclear wave function}
	$\Psi^\text{nuc}_{\Inu}(\vec{x})$ is yet to be determined.
	%
	\item The factorisation \eqref{eqn:BOFactorisation} satisfies the property%
	\footnote{%
		More precisely what we assume is that the nuclear kinetic energy
		operator $\Op{T}_n$ projected onto the basis formed by
		all electronic states $\Psi^\text{elec}_{\Iel}(\vec{X}, \vec{x})$
		is diagonal with all elements equal to $1$.
		See \cite{Baer2006} or \cite{WikipediaBornOppenheimer} for details.
	}
	\begin{equation}
		\Op{T}_n \Psi^\text{MB}_{I}(\vec{X}, \vec{x})
		\simeq \Op{T}_n \left( \Psi^\text{elec}_{\Iel}(\vec{X}, \vec{x})
		\Psi^\text{nuc}_{\Inu}(\vec{X}) \right)
		\simeq \Psi^\text{elec}_{\Iel}(\vec{X}, \vec{x})
		\left(\Op{T}_n \Psi^\text{nuc}_{\Inu}(\vec{X}) \right).
		\label{eqn:BONuclearDerivative}
	\end{equation}
\end{itemize}
By plugging ansatz \eqref{eqn:BOFactorisation} into \eqref{eqn:TISEManyBody}
we can simplify
\begin{align*}
	0 &=
	\left( \Op{H}^\text{MB} - E^\text{MB}_{I} \right)
	\Psi^\text{MB}_{I}(\vec{X}, \vec{x}) \\
	&\stackrel{\eqref{eqn:BOFactorisation}}{\simeq}
	\left( \Op{H}^\text{elec} + \Op{T}_n - E^\text{MB}_I \right)
	\Psi^\text{elec}_{\Iel}(\vec{X}, \vec{x}) \Psi^\text{nuc}_{\Inu}(\vec{X}) \\
	&\stackrel{\eqref{eqn:BONuclearDerivative}}{\simeq}
	\left( \Op{H}^\text{elec} \Psi^\text{elec}_{\Iel}(\vec{X}, \vec{x}) \Psi^\text{nuc}_{\Inu}(\vec{X}) \right)
	+ \Psi^\text{elec}_{\Iel}(\vec{X}, \vec{x}) \left(\Op{T}_n \Psi^\text{nuc}_{\Inu}(\vec{X}) \right)
	- E^\text{MB}_{I} \Psi^\text{MB}_I(\vec{X}, \vec{x})
	\\
	&\stackrel{\eqref{eqn:ElectronicSchrödinger}}{=}
	\Psi^\text{elec}_{\Iel}(\vec{X}, \vec{x})
	\left(E^\text{elec}_{\Iel}(\vec{X}) \Psi^\text{nuc}_{\Inu}(\vec{X})
	+ \Op{T}_n \Psi^\text{nuc}_{\Inu}(\vec{X})
	- E^\text{MB}_I \Psi^\text{nuc}_{\Inu}(\vec{X}) \right).
\end{align*}
This statement is satisfied provided that the nuclear wave function
$\Psi^\text{nuc}_{\Inu}(\vec{X})$
follows the \newterm{nuclear Schrödinger equation}
\begin{equation}
	\left( \Op{T}_n + E^\text{elec}_{\Iel}(\vec{X}) \right)
	\Psi^\text{nuc}_{\Inu}(\vec{X})
	= E^\text{MB}_I \Psi^\text{nuc}_{\Inu}(\vec{X}).
	\label{eqn:NuclearSchrödinger}
\end{equation}

Overall the Born-Oppenheimer approximation
allows to solve the many-body Schrödinger equation \eqref{eqn:TISEManyBody} in two steps.
First we limit ourselves to the point of view of the electrons
under the electric field induced by fixed, motionless nuclei.
This leads to \eqref{eqn:ElectronicSchrödinger},
which is solved for the
electronic states $\Psi^\text{elec}_{\Iel}(\vec{X}, \vec{x})$
along with corresponding electronic energies $E^\text{elec}_{\Iel}(\vec{X})$.
In the second step we consider nuclear motion by solving
\eqref{eqn:NuclearSchrödinger}.
In this equation the electronic energies $E^\text{elec}_{\Iel}(\vec{X})$
depending on the nuclear coordinates act as the electrostatic potential
in which the nuclei move.
For this reason $E^\text{elec}_{\Iel}(\vec{X})$ is sometimes
called a \newterm{potential energy surface} as well.
Note, that each electronic state characterised by quantum numbers $\Iel$
gives rise to a different potential energy surface.

Employing a more detailed treatment of the Born-Oppenheimer approximation,
like in the original paper \cite{Born1927} or \citet{Baer2006},
allows to gain more insight regarding the range of applicability
of the Born-Oppenheimer approximation.
Loosely speaking it is a valid approximation
as long as the potential energy surfaces $E^\text{elec}_{\Iel}(\vec{X})$
are well-separated from another.

From a numerical point of view this approximation allows to reduce
the dimensionality of the problem somewhat.
To illustrate this let us return to the water molecule,
which was already discussed at the end of section \vref{sec:ManyBodyTISE}.
In the exact problem we need to solve one equation,
namely the many-body Schrödinger equation \eqref{eqn:TISEManyBody}
of dimensionality $L = 39$.
Within the Born-Oppenheimer approximation
this is replaced by solving two equations,
the electronic one \eqref{eqn:ElectronicSchrödinger} of dimensionality $3 \Nelec = 30$
and the nuclear TISE \eqref{eqn:NuclearSchrödinger} of dimensionality $3 M = 9$.
In the estimate we presented in section \vref{sec:ManyBodyTISE}
for the $L^2$ inner products,
this would roughly provide a speed-up factor of $10^9$.

\subsection{Electronic Schrödinger equation}
\label{sec:ElectronicSchrödinger}
By solving the electronic Schrödinger equation \eqref{eqn:ElectronicSchrödinger}
we get access to the electronic states $\Psi^\text{elec}_{\Iel}(\vec{X}, \vec{x})$
as well as the potential energy surface $E^\text{elec}_{\Iel}(\vec{X})$.
In many cases these quantities already provide
enough insight into a chemical system in order to address many
questions relevant to quantum chemistry.
For this reason the nuclear Schrödinger equation \eqref{eqn:NuclearSchrödinger}
will be neglected in this work from now on
and we will focus only
on approximation methods for solving \eqref{eqn:ElectronicSchrödinger} instead.

For ease of notation we will usually drop the indices ``e'',
superscripts ``elec'' from now on if we refer to electronic energies
or the electronic part of the wave function.
Similarly in the context of the electronic Schrödinger equation
the nuclei are motionless, which makes $\vec{X}$ a fixed quantity.
Thus we drop the nuclear coordinates ``$\vec{X}$''
from the function arguments, too.
In this convention we would for example
write the electronic Schrödinger equation \eqref{eqn:ElectronicSchrödinger} as
\[ \Op{H} \Psi_I(\vec{x}) = E_I \Psi_I(\vec{x}). \]
Another simplification we sometimes employ is to consider
the simplified electronic Hamiltonian
\begin{equation}
	\Op{H}_{\Nelec} \equiv \Op{H}^\text{elec} - \Op{V}_{nn}
	= -\frac12 \sum_{i=1}^{\Nelec} \Delta_{\vec{r}_i}
	+ \sum_{A=1}^M \sum_{i=1}^{\Nelec} \frac{Z_A}{r_{iA}}
	+ \sum_{i=1}^{\Nelec} \sum_{j=i+1}^{\Nelec} \frac{1}{r_{ij}}
	\label{eqn:ElectronicHamiltonian}
\end{equation}
instead of $\Op{H}^\text{elec}$.
This is possible, since the potential operator governing the Coulombic interaction
amongst the nuclei
\[
	\Op{V}_{nn} = \sum_{A=1}^M \sum_{B=A+1}^M \frac1{r_{AB}}
\]
only depends on $\vec{X}$, which makes it a constant value for one particular chemical system.
In many cases one can therefore work with $\Op{H}_{\Nelec}$
in a numerical treatment and only add the nuclear potential energy term $\Op{V}_{nn}$
afterwards.

In analogy to the many-body Hamiltonian \eqref{eqn:ManyBodyHamiltonian}
and the hydrogen-like Hamiltonian \eqref{eqn:OpHydrogen}
we choose the underlying Hilbert space of $\Op{H}_{\Nelec}$
to be $L^2(\R^{3\Nelec}, \C)$.
Due to Kato's theorem~\cite{Kato1951} $\Op{H}_{\Nelec}$
becomes self-adjoint if we set its domain to $D(\Op{H}_{\Nelec}) = H^2(\R^{3\Nelec}, \C)$.
Not all functions in $H^2(\R^{3\Nelec}, \C)$ are physical, however~\cite{Mueller2000,Shankar1994}.
This is due to the fact that electrons do not only show spatial degrees of freedom,
but furthermore an intrinsic angular momentum degree of freedom
called \newterm{spin}.
More precisely electrons are so-called spin-$\sfrac12$ particles.
By the spin statistics theorem~\cite{Shankar1994} of quantum field theory
this requires the electronic wave function
to be antisymmetric with respect to particle exchange.
More symbolically all eigenfunctions $\Psi_I$ of $\Op{H}_{\Nelec}$
need to satisfy the condition
\begin{equation}
	\forall i,j \in \{1, 2, \ldots, \Nelec\}: \qquad
	  \Psi_I(\ldots, \vec{r}_i, \ldots, \vec{r}_j, \ldots) =
	- \Psi_I(\ldots, \vec{r}_j, \ldots, \vec{r}_i, \ldots).
	\label{eqn:AsymElecWavefunction}
\end{equation}
It is easy to see that not all elements of $H^2(\R^{3\Nelec}, \C)$ satisfy this.

Given that the classical correspondence
of \vref{sec:QMCorrespondence} did not yield any kind of spin degree of freedom
for non-relativistic \QM,
one might wonder at this point
why we need to bother with spin and the resulting
antisymmetry of the wave function at all in our physical model.
As it turns out many fundamental experimental results
and observations made at the beginning of the $20^\text{th}$ century
can only be explained if proper spin statistics is taken into account.
This includes the Stern-Gerlach experiment~\cite{Gerlach1922b,Gerlach1922,Gerlach1922a},
the spectral properties of atoms~\cite{Pauli1925}
and Fermi-Dirac statistics~\cite{Dirac1926},
just to name a few.
Even though spin can only be rigorously derived
using more sophisticated theories like
relativistic \QM or quantum field theory~\cite{Shankar1994},
one still needs to include it \textit{ad hoc} in non-relativistic \QM as well
such that above observations can be explained~\cite{Pauli1925,Shankar1994,Straumann2004}.

Notice that a proper inclusion of spin in non-relativistic \QM
requires two modifications.
First we need each wave function to include an extra spin degree of freedom%
~\cite{Pauli1925}.
Secondly we need to make sure that \eqref{eqn:AsymElecWavefunction}
is always satisfied~\cite{Dirac1926}.
We will defer the first modification to remark \vref{rem:Spin}
in order to yield a simpler mathematical treatment for now.
Unfortunately we cannot ignore \eqref{eqn:AsymElecWavefunction}
due to its tremendous impact on the mathematical structure of the emerging problems%
~\cite{Dirac1926,Fock1930}.

\todoil{
	I really would like to know more about the relationship
	between Slater determinants and exterior products.
	Go and ask someone about this.
	The internet was not very helpful.
}
There are a couple of approaches which could be followed
to adhere to \eqref{eqn:AsymElecWavefunction}.
Typically one abstains from modifying the Hamiltonian $\Op{H}_{\Nelec}$
and instead restricts the search space for the eigenstates $\Psi_I$
to an appropriate subspace of $L^2(\R^{3\Nelec}, \C)$,
which is constructed in a way
to enforce the required
antisymmetry with respect to the electronic coordinates~\cite{Dirac1926,Fock1930}.
Such a space is the $\Nelec$-th \newterm{exterior power} of $L^2(\R^3, \C)$ defined as
\newcommand{\wedgestring}{\psi_1 \wedge \psi_2 \wedge \cdots \wedge \psi_{\Nelec}}

\begin{equation}
	\bigwedge^{\Nelec} L^2(\R^3, \C)
	\equiv \spacespan \Big\{ \wedgestring
	\, \Big| \, \psi_i \in L^2(\R^3, \C) \, \forall i = 1, \ldots \Nelec \Big\}.
	\label{eqn:DefExteriorPower}
\end{equation}
The key component of this definition is the \newterm{wedge product}
or \newterm{exterior product} $f \wedge g$.
This product is totally antisymmetric with respect to its operands
and is closely related to the tensor product $f \otimes g$.
For example if $f, g \in L^2(\R^3, \C)$, then $f \wedge g \in L^2(\R^6, \C)$.
In some sense one can think of the wedge product
as a generalisation of the  cross product $\vec{a} \times \vec{b}$
for vectors $\vec{a}, \vec{b} \in \R^3$.
This is somewhat apparent from its properties.
Notice for example
\begin{align}
	\label{eqn:PropertiesExteriorProduct}
	\psi_1 \wedge \psi_1 &= 0, &\psi_1 \wedge \psi_2 &= -\psi_2 \wedge \psi_1, &
	\psi_1 \wedge (c_1 \psi_1 + c_2 \psi_2) &= c_2 \psi_1 \wedge \psi_2
\end{align}
for any $\psi_1, \psi_l \in L^2(\R^3, \C)$ and any $c_1, c_2 \in \C$.
One may identify the application of a wedge product string like
\[ \bigwedge_{i=1}^{\Nelec} \psi_i \equiv \wedgestring \]
onto the electronic coordinates $\vec{x}$
with the evaluation of a determinant, \ie
\begin{align*}
	\left( \bigwedge_{i=1}^{\Nelec} \psi_i \right) (\vec{x})
	&\equiv \Big(\wedgestring\Big) (\vec{r}_1, \vec{r}_2, \ldots, \vec{r}_{\Nelec}) \\
	\equiv \frac{1}{\sqrt{\Nelec}} &\det
	\left(
	\begin{array}{*{6}{l}}
		\psi_1(\vec{r}_1) & \psi_2(\vec{r}_1) & \cdots & \psi_{\Nelec}(\vec{r}_1) \\
		\psi_1(\vec{r}_2) & \psi_2(\vec{r}_2) & \cdots & \psi_{\Nelec}(\vec{r}_2) \\
		\vdots & \vdots & \ddots & \vdots \\
		\psi_1(\vec{r}_{\Nelec}) & \psi_2(\vec{r}_{\Nelec}) & \cdots & \psi_{\Nelec}(\vec{r}_{\Nelec}) \\
	\end{array}
	\right).
\end{align*}
Because of this observation $\wedgestring$
is typically called a \newterm{Slater determinant}%
\footnote{After John~C.~Slater, who introduced it.~\cite{Slater1929,Slater1930a}.}
in standard quantum-chemistry textbooks~\cite{Szabo1996,Helgaker2013}.
The functions $\psi_i \in L^2(\R^3, \C)$
are usually called \newterm{single-particle functions},
since they only depend on a single electronic coordinate.
Another way of phrasing \eqref{eqn:DefExteriorPower}
is therefore that $\bigwedge^{\Nelec} L^2(\R^3, \C)$
is the space spanned by all Slater determinants
consisting of $\Nelec$ single-particle functions from $L^2(\R^3, \C)$.
Notice, that%
\footnote{%
	This observation is the reason why the single-particle functions
	need to be square-integrable, \ie from $L^2(\R^3, \C)$.
}
\[ \bigwedge^{\Nelec} L^2(\R^3, \C) \subset L^2(\R^{3\Nelec}, \C), \]
exterior power on the left hand side is even dense in the space on the right.

If we want to encode condition \eqref{eqn:AsymElecWavefunction}
into our problem an easy solution is to combine this with Kato's theorem
and employ the domain
\begin{equation}
	D(\Op{H}_{\Nelec}) = H^2(\R^{3\Nelec}, \C) \cap \bigwedge^{\Nelec} L^2(\R^3, \C)
	\label{eqn:ElectronicHamiltonianDomain}
\end{equation}
for the electronic Hamiltonian $\Op{H}_{\Nelec}$.
This makes both the operator self-adjoint and the electronic wave function
comply with the spin statistics theorem.
The analogous form domain in this case would be
\[ Q(\Op{H}_{\Nelec}) = H^1(\R^{3\Nelec}, \C) \cap \bigwedge^{\Nelec} L^2(\R^3, \C). \]

For establishing the spectral properties of $\Op{H}_{\Nelec}$,
there is first the important HVZ theorem~\cite{Zhislin1959,Zhislin1960,Reed1978,Teschl2014}
after Hunziker, van Winter and Zhislin.
\begin{thm}[HVZ]
	\label{thm:HVZ}
	Let $\Op{H}_{\Nelec}$ be the self-adjoint operator of \eqref{eqn:ElectronicHamiltonian}
	on the Hilbert space $L^2(\R^{3\Nelec}, \C)$
	with the domain given in \eqref{eqn:ElectronicHamiltonianDomain}.
	Then $\Op{H}_{\Nelec}$ is bounded from below and
	\[ \sigma_\text{ess}\left(\Op{H}_{\Nelec}\right) = [\Sigma_{\Nelec}, +\infty) \]
	with
	\[ \Sigma_{\Nelec} = \left\{
		\begin{array}{ll}
			0 & \text{ if $\Nelec = 1$} \\
			\inf \sigma\left(\Op{H}_{\Nelec-1}\right) < 0 & \text{ if $\Nelec \geq 2$}
		\end{array}
		\right. .
	\]
\end{thm}
This theorem establishes a link between
the lower bound of the essential spectrum
of the electronic Hamiltonian of a $\Nelec$-electron system
and the lower bound of the complete spectrum of
a corresponding $\Nelec - 1$ electron system
with the same nuclear arrangement.

For characterising the discrete spectrum of $\Op{H}_{\Nelec}$
we employ the important results by \citet{Zhislin1961} and \citet{Yafaev1976},
summarised by the following proposition.
\begin{prop}
	\label{prop:ZhislinYafaev}
	Let $\Op{H}_{\Nelec}$ be the $\Nelec$-electron electronic Hamiltonian
	operator of \eqref{eqn:ElectronicHamiltonian}
	with domain as stated in \eqref{eqn:ElectronicHamiltonianDomain}.
	Let further $Z_\text{tot} := \sum_{A=1}^M Z_A$ denote the total nuclear charge.
	\begin{itemize}
		\item If $\Nelec \leq Z_\text{tot}$, \ie we consider a \emph{neutral}
			or \emph{positively charged} system, then
			$\Op{H}_{\Nelec}$
			has an infinite number of discrete eigenvalues
			below the essential spectrum.
		\item If $\Nelec \geq 1 + Z_\text{tot}$ (\emph{negatively charged} system),
			then $\Op{H}_{\Nelec}$ has \emph{at most} a finite number of discrete
			states below the essential spectrum.
	\end{itemize}
	\begin{proof}
		See~\cite{Zhislin1961,Yafaev1976}.
	\end{proof}
\end{prop}
Before we discuss the physical interpretation of these results,
let us first introduce some terminology.
If we are either concerned with a neutral or positively charged
$\Nelec$-electron system or a negatively charged system
with at least a single discrete eigenvalue, we can define a \newterm{ground state energy}
\begin{equation}
	E^{\Nelec}_0 = \min \sigma\left(\Op{H}_{\Nelec}\right) = \Sigma_{\Nelec}
	\label{eqn:GroundStateEnergy}
\end{equation}
We order the energies of the discrete spectrum ordered as usual
\[ E^{\Nelec}_0 \leq E^{\Nelec}_1 \leq E^{\Nelec}_2 \leq \cdots \]
and associate with these eigenvalues
the corresponding (bound) eigenstates
\[ \Psi_0, \Psi_1, \Psi_2, \ldots. \]
All states $\Psi_i$ which have an energy eigenvalue $E^{\Nelec}_i = E^{\Nelec}_0$
are commonly refered to as the \newterm{ground state}.
If $E^{\Nelec}_0$ is not degenerate by construction
$\Psi_0$ is the ground state.
All other states $\Psi_i$ with $E^{\Nelec}_i \neq E^{\Nelec}_0$
are called \newterm{excited states}.

For negatively charged systems we similarly use the
term ``ground state'' to refer to the state or states corresponding
to the lowest eigenvalue of $\sigma_P\left(\Op{H}_{\Nelec}\right)$
and ``excited states'' for the other eigenfunctions of $\Op{H}_{\Nelec}$.
Note, however, that for negatively charged systems
the case \mbox{$\inf \sigma\left(\Op{H}_{\Nelec}\right) < E^{\Nelec}_0$} is possible.

\begin{rem}[Physical interpretation of the spectrum]
	In this remark we will summarise the results,
	which can be deduced from the HVZ theorem \vref{thm:HVZ}
	and from proposition \vref{prop:ZhislinYafaev}.

	% TODO OPTIONAL
	%\todoil{Figure ?}
	Let us first consider a neutral or positively charged
	system with $\Nelec$ electrons.
	It has a ground states as well as an infinite number
	of discrete and bound excited states
	until the ground state energy $E^{\Nelec-1}_0$
	of the corresponding $(\Nelec-1)$-electron
	with the same nuclear arrangement is hit.
	Note, that we can be sure that $E^{\Nelec-1}_0$
	exists, because the $(\Nelec-1)$-electron system
	is positively charged.
	This behaviour is easy to understand phyiscally.
	As soon as the energy $E^{\Nelec-1}_0$ is reached
	our $\Nelec$-electron system can always
	separate into a stable system with $\Nelec-1$
	bound electrons and an unbound $\Nelec$-th electron
	taking the excess energy into the continuum.
	From this we can easily understand the energy difference
	$E^{\Nelec-1}_0 - E^{\Nelec}_0$
	as the ionisation energy.
	Note, that embedded inside the emerging continuum
	at energies beyond $E^{\Nelec-1}_0$
	may still be bound states of the $(\Nelec-1)$-electron system.
	In other words in general we have
	\[ \sigma_C\left(\Op{H}_{\Nelec}\right)
		\subsetneq \sigma_\text{ess}\left(\Op{H}_{\Nelec}\right). \]

	If the $\Nelec$-system is of single negative charge
	and possesses no bound states below the essential spectrum,
	this implies
	\[ \inf \sigma\left(\Op{H}_{\Nelec}\right)
		= \inf \sigma\left(\Op{H}_{\Nelec-1}\right) = E^{\Nelec-1}_0 \]
	since the $\Nelec-1$-electron system is neutral,
	thus possesses a ground state.
	In other words all bound states of this system
	are embedded inside the continuum.
	The system thus happily separates into a bound $(\Nelec-1)$-electron system
	and an unbound electron at all energies:
	This negative ion is not stable.
	Conversely for stable negative ions
	we would expect at least a single bound state to exist.

	Unlike neutral or positively charged systems,
	negatively charged systems in each case only possess a
	finite number of bound states below
	the essential spectrum.

	To summarise this remark, let us note the following
	interesting observations,
	which are now backed by rigorous mathematical observations:
	\begin{itemize}
		\item The essential spectrum marks the energies
			at which a chemical system is unstable,
			because it can separate into (one or more) unbound electron
			plus a stable system with a
			reduced number of bound electrons.
		\item Forming a positive ion always costs energy.
		\item All systems with more than one electron
			will produce unbound electrons,
			\ie ionise, at large-enough energies (in vacuum).
		\item Not all negative ions possess a stable ground state (in vacuum).
		\item All positive ions possess a stable ground state (in vacuum).
	\end{itemize}
\end{rem}

\begin{rem}[Consequences for a numerical treatment]
	\label{rem:ElectronicTISENumerical}
	\todoil{Rephrase}
	From proposition \ref{prop:ZhislinYafaev}
	we can immediately deduce,
	that there are no problems with neutral or positively charged
	systems under a variational numerical treatment as discussed
	in section \vref{sec:Projection}.
	Both the ground state as well as the first few
	excited states are located below the
	essential spectrum and thus accessible for the treatment
	described in remark \vref{rem:ApproxBottomDiscrete}.

	For negative ions we might get into trouble.
	If the ion is stable,
	then at least its ground state can be approximated
	numerically via remark \ref{rem:ApproxBottomDiscrete}.
	If it is not, we might not even be able to get
	its ground state.
	The problem is, as usual, that in the numerical treatment
	we cannot easily distinguish between approximations
	to bound states and approximations to continuous states
	if they are located in the same energy range
	as they are in the essential spectrum.
	So our numerical treatment will always yield
	an apparent ground state.
	Whether it is physical or not.
\end{rem}

\section{Full configuration interaction}
\label{sec:FCI}
\defineabbr{FCI}{FCI\xspace}{Full configuration interaction}
\nomenclature{$\Phi$}{Slater determinant $\bigwedge_{i=1}^{\Nelec} \psi_i$
or many-electron basis functions}
\nomenclature{$\psi_i$}{One-particle function,
typically $i$-th eigenfunction of the Fock operator, i.e.~a Hartree-Fock orbital}
\nomenclature{$\Ibas$}{Index set of the one-particle basis functions.
	Typically a set of multi-indices of quantum numbers.}
\nomenclature{$\Nbas$}{Cardinality of $\Ibas$, i.e.~the number of one-particle basis functions.}
\nomenclature{$\varphi_\mu$}{$\mu$-th one-particle basis function of the one-particle basis $\{\varphi_\mu\}_{\mu \in \Ibas}$}

\nomenclature{$\Iocc$}{Index set of occupied SCF orbitals.}
\nomenclature{$\Ivirt$}{Index set of virtual, i.e. unoccupied, SCF orbitals.}

In this section we want to develop a numerical treatment
for solving the electronic Schrödinger equation \eqref{eqn:ElectronicSchrödinger}
under the Ritz-Galerkin projection ansatz of section \vref{sec:RitzGalerkin}.
In the previous section we analysed the mathematical implications
of the spin statistics theorem for electrons as fermionic systems,
which lead us to choose the form domain
\[ Q(\Op{H}_{\Nelec}) = H^1(\R^{3\Nelec}, \C) \cap \bigwedge^{\Nelec} L^2(\R^3, \C). \]
for the electronic Schrödinger operator $\Op{H}_{\Nelec}$.


For simplifying our treatment
we will not try to discretise this domain in the Ritz-Galerkin ansatz
of definition \ref{defn:RitzGalerkin},
much rather we will develop methods to sample only the subspace
\[ \tilde{Q}(\Op{H}_{\Nelec}) = H^1_a(\R^{3\Nelec}, \C) \equiv \bigwedge^{\Nelec} H^1(\R^3, \C) \subset Q(\Op{H}_{\Nelec}) \]
due to its simpler structure.
Since this subspace is dense we will not suffer from any loss of numerical
accuracy in the approximate treatment later on.
this implies no potential loss of numerical accuracy.
By definition of the exterior power
\begin{equation}
	\tilde{Q}(\Op{H}_{\Nelec}) = \spacespan \left\{
		\bigwedge_{i=1}^{\Nelec} \psi_i
		\, \middle| \,
		\psi_i \in H^1(\R^3, \C) \, \forall i = 1, \ldots, \Nelec
	\right\}.
	\label{eqn:FormDomainAllFunctions}
\end{equation}
Since $H^1(\R^{3\Nelec}, \C)$ is separable, we can find a countable basis set
\begin{equation}
	\set{B}_1 \equiv \{\psi_i\}_{i \in \N} \qquad \text{with }
	\braket{\psi_i}{\psi_j}_1 = \delta_{ij}
	\quad \text{and} \quad \spacespan \set{B}_1 = H^1(\R^{3\Nelec}, \C),
	\label{eqn:OneParticleBasis}
\end{equation}
where we used the abbreviated notation
$\braket{\slot}{\slot}_1 = \braket{\slot}{\slot}_{L^2(\R^3, \C)}$.
Taking the properties of the wedge product \eqref{eqn:PropertiesExteriorProduct}
into account allows to deduce the equivalent construction
\begin{equation}
	\label{eqn:FormDomainSlaterDeterminants}
	\tilde{Q}(\Op{H}_{\Nelec}) = \spacespan
	\left\{ \bigwedge_{i=1}^{\Nelec} \psi_i
	, \middle| \, \psi_i \in \set{B}_1 \, \forall i=1,\ldots,\Nelec
	\right\},
\end{equation}
which builds $\tilde{Q}(\Op{H}_{\Nelec})$ as the span
over all Slater determinants built by selecting
$\Nelec$ functions from $\set{B}_1$.
Since nothing stops us from selecting the same basis function
twice from $\set{B}_1$ in this construction,
many of the constructed determinants $\bigwedge_{i=1}^{\Nelec} \psi_i$
are zero.
In other words these determinants amount to span $\tilde{Q}(\Op{H}_{\Nelec})$,
but they are not a basis for this space.
In the following we want to fix this and construct an orthonormal basis
of suitable Slater determinants.
This requires an appropriate inner product.
\begin{defn}
	Let $\tilde{Q}(\Op{H}_{\Nelec})$ be defined as in \eqref{eqn:FormDomainAllFunctions}.
	We define an inner product on $\tilde{Q}(\Op{H}_{\Nelec})$ by
	requiring for any two arbitrary Slater determinants
	\begin{align*}
		\Psi &= \bigwedge_{i=1}^{\Nelec} \psi_i
		&&\text{and}& \Xi &= \bigwedge_{i=1}^{\Nelec} \xi_i
	\end{align*}
	with $\psi_i, \xi_i \in H^1(\R^3, \C)$ for all $i \in 1,\ldots,\Nelec$:
	\begin{align}
		\braket{\Psi}{\Xi}_{\Nelec} \equiv \det \mat{G}
		\qquad
		\text{where } G_{ij} = \braket{\psi_i}{\xi_j}_{L^2(\R^3, \C)} \forall i,j \in 1,\ldots,\Nelec.
		\label{eqn:InnerProductFormDomain}
	\end{align}
	The inner product for other elements from $\tilde{Q}(\Op{H}_{\Nelec})$
	is then constructed in accordance
	with the axioms shown in definition \vref{def:InnerProduct}.
	\todoil{A more formal proof would be nice,
		especially for the completeness under this inner product.
		Or showing equivalence with respect to the inner product
		on $H^1(\R^{3\Nelec}, \C$ for the elements of $\tilde{Q}(\Op{H}_{\Nelec})$
		would be nice.}
\end{defn}

\noindent
With this inner product at hand we can construct an orthonormal basis for
$\tilde{Q}(\Op{H}_{\Nelec})$.

\begin{rem}[Orthonormal basis for $\tilde{Q}(\Op{H}_{\Nelec})$]
	\label{rem:Determinants}
	Let $\tilde{Q}(\Op{H}_{\Nelec})$ be defined as in \eqref{eqn:FormDomainAllFunctions}
	and let $\set{B}_1$ be an arbitrary basis for $H^1(\R^3, \C)$.
	We take one arbitrary, non-trivial Slater determinant
	$0 \neq \Phi_0 \in \tilde{Q}(\Op{H}_{\Nelec})$,
	such that
	\[
		\Phi_0 = \tilde{\psi}_1 \wedge \tilde{\psi}_2 \wedge \cdots \tilde{\psi}_i \cdots \wedge \tilde{\psi}_{\Nelec}
	\]
	for appropriate $\tilde{\psi}_i \in \set{B}_1$.
	This determinant can always be found due to the alternative
	construction for $\tilde{Q}(\Op{H}_{\Nelec})$ sketched
	in \eqref{eqn:FormDomainSlaterDeterminants}.
	Let us call $\Phi_0$ the \newterm{reference determinant}
	in the following.

	The functions of the (countable) basis set $\set{B}_1 = \{\psi_i\}_{i \in \N}$
	can be indexed in such a way
	that the first $\Nelec$ functions coincide with $(\tilde{\psi}_1, \tilde{\psi}_2, \ldots, \tilde{\psi}_{\Nelec})$.
	In other words
	\[
		\Phi_0 = \psi_1 \wedge \psi_2 \wedge \cdots \psi_i \cdots \wedge \psi_{\Nelec}
	\]
	as well.
	We further define the index sets%
	\footnote{%
		The subscript ``occ'' stands for \textit{occupied}
		and ``virt'' for \textit{virtual}.
		These terms will become clear when we discuss the Hartree-Fock ansatz
		in the next section.
	}
	\begin{align*}
		\Iocc &= \{1, \ldots, \Nelec\}
		&&\text{and} &
		\Ivirt &= \{ i \in \N \, | \, i > \Nelec \}.
	\end{align*}
	With reference to $\Phi_0$ we can construct
	for each $i \in \Iocc$ and each $a \in \Ivirt$
	a so-called singly \newterm{excited determinant}
	\[
		\Phi_i^a = \psi_1 \wedge \psi_2 \wedge \cdots \psi_a \cdots \wedge \psi_{\Nelec}
	\]
	by replacing the $i$-th function of the Slater determinant
	wedge string
	by the $a$-th function of $\set{B}_1$
	without changing the order.
	Analogously one may define doubly or higher excited determinants
	\begin{align*}
		\Phi_{ij}^{ab} &= \psi_1 \wedge \psi_2 \wedge \cdots \wedge \psi_a \cdots \psi_b
			\wedge \cdots \wedge \psi_{\Nelec} \\
		\Phi_{ijk}^{abc} &= \psi_1 \wedge \psi_2 \wedge \cdots \wedge \psi_a \cdots \psi_b
			\cdots \psi_c \wedge \cdots \wedge \psi_{\Nelec}
	\end{align*}
	where%
	\footnote{%
		This is the typical indexing convention in quantum chemistry.
		Indices $i,j,k,l,m, \ldots$ stand for occupied indices
		and $a,b,c,d,e, \ldots$ for virtual indices.
	} $i,j,k \in \Iocc$ and $a,b,c \in \Ivirt$
	In this case one has to additionally
	require that $i < j < k < \cdots$ and $a < b < c < \cdots$,
	because otherwise no new determinants are generated (if $i=j$ or $i=k$ or \ldots)
	or a zero determinant is generated (if $a=b$ or similar).
	Constructed in this way all determinants in the set
	\[
		\set{B}_{\Nelec} \equiv
		\left\{
			\Phi_0, \Phi_i^a, \Phi_{ij}^{ab}, \Phi_{ijk}^{abc}, \cdots \right\}
	\]
	are unique.
	Still it is not hard to see
	that $\spacespan \set{B}_{\Nelec} = \tilde{Q}(\Op{H}_{\Nelec})$,
	since we only took away those determinants adding redundant information
	in the construction \eqref{eqn:FormDomainSlaterDeterminants}.

	With the inner product defined in \eqref{eqn:InnerProductFormDomain} we notice
	for all $r,s \in \N$
	\[
		\braket{\Phi_0}{\Phi_r^s}_{\Nelec}
			= \braket{\psi_r}{\psi_s}_1 = \delta_{rs},
	\]
	since by \eqref{eqn:OneParticleBasis}
	all functions in $\set{B}_1$ are orthonormal to each other.
	In other words $\set{B}_{\Nelec}$ is an orthonormal
	basis for $\tilde{Q}(\Op{H}_{\Nelec})$.
	\todo[inline, caption={}]{
		This following statement might not be true,
		since $\braket{\slot}{\slot}_{\Nelec}$
		might not be equivalent to the Hilbertian inner product!
		\begin{center}
		Since it is countable, $\tilde{Q}(\Op{H}_{\Nelec})$ is separable.
		\end{center}
	}
\end{rem}
The set $\set{B}_{\Nelec}$ is sometimes called the \newterm{$\Nelec$-particle basis}
or \newterm{many-particle basis}
corresponding to $\set{B}_1$ and the reference determinant $\Phi_0$.
Albeit the precise entries in $\set{B}_{\Nelec}$ might differ from case to case
the end result $\spacespan \set{B}_{\Nelec} = \tilde{Q}(\Op{H}_{\Nelec})$
is always true
regardless of the choice of $\Phi_0$ or $\set{B}_1$.

\begin{rem}
	Given a many-particle basis $\set{B}_{\Nelec}$ consisting
	of normalised Slater determinants,
	any function $\Psi \in \tilde{Q}(\Op{H}_{\Nelec})$
	can be expanded as such
	\begin{align}
		\label{eqn:ExpansionSlaterDeterminant}
		\Psi &= \sum_\mu c_\mu \Phi_\mu
		&\text{where } \forall \mu \in \N: \quad
		\Phi_\mu \in \set{B}_{\Nelec}, c_\mu \in \C.
	\end{align}
	If one is interested in emphasising the particular
	basis of one-particle functions
	$\set{B}_1$ and the particular reference determinants $\Phi_0$
	this can be written equivalently as
	\begin{equation}
		\label{eqn:CIExpansion}
		\Psi
		= c_0 \Phi_0 + \sum_{ia} c_i^a \Phi_i^a
		+ \sum_{\substack{i<j \\a<b}} c_{ij}^{ab} \Phi_{ij}^{ab}
		+ \sum_{\substack{i<j<k \\ a<b<c}} c_{ijk}^{abc} \Phi_{ijk}^{abc}
		 + \cdots,
	\end{equation}
	where $i,j,k, \ldots \in \Iocc$ and $a,b,c, \ldots \in \Ivirt$.

	This expansion is commonly referred to as the \newterm{CI expansion},
	where CI stands for configuration interaction,
	a term which will become more clear
	after the next remark.
\end{rem}

\defineabbr{FCI}{FCI\xspace}{Full configuration interaction,
	see section \vref{sec:FCI}}
\newcommand{\Nfci}{N_\text{FCI}}
\begin{rem}[Full CI]
	\label{rem:FCI}
	The discrete formulation of the Ritz-Galerkin scheme
	of remark \vref{rem:DiscreteFormulation}
	can now be applied rather easily to the electronic Schrödinger equation.
	This leads to a procedure called \newterm{full CI}
	or full configuration interaction~(\FCI).

	\begin{itemize}
	\item Take a finite-sized basis set of orthonormal one-particle functions
	\[ \set{B}_1^{(n)} \equiv \{\psi_i\}_{i \in \Ibas} \subset U, \]
	where $U \subset H^1(\R^3, \C)$ is dense.
	%
	\item Choose --- at random or using prior knowledge ---
		an arbitrary reference determinant
	\[ \Phi_0 = \psi_1 \wedge \psi_2 \wedge \ldots \wedge \psi_{\Nelec} \]
	where
	\[ (\psi_1, \psi_2, \ldots, \psi_{\Nelec}) \in \left(\set{B}_1^{(n)}\right)^{\Nelec} \]
	and construct the finite $\Nelec$-electron basis
	\begin{equation}
		\set{B}_{\Nelec}^{(n)} \equiv \{ \Phi_0, \Phi_i^a, \Phi_{ij}^{ab}, \ldots \}
		\label{eqn:NelectronBasisFinite}
	\end{equation}
	using substitutions of the functions from $\set{B}_1^{(n)}$
	according to the procedure described in remark \vref{rem:Determinants}.
	As usual we take
	\begin{align}
		\label{eqn:CondOccIndex}
		i, j, k, l, \ldots &\in \Iocc &&\text{with}& i&<j<k<l<\cdots \\
		\label{eqn:CondVirtIndex}
		a, b, c, d, \ldots &\in \Ivirt &&\text{with}& a&<b<c<d<\cdots
	\end{align}
	where in the finite case
	\begin{align*}
		\Iocc &= \{1, \ldots, \Nelec\} &&\text{and}& \Ivirt &= \{ \Nelec + 1, \ldots, \Nbas \}.
	\end{align*}
	%
	\item Construct the full CI matrix $\mat{A}_\text{FCI} \in \C^{\Nfci \times \Nfci}$
		consisting of elements
		\begin{equation}
			a(\Phi_1, \Phi_2)
			= \mbra{\Phi_1}
			\Op{T}_e + \Op{V}_{ne} + \Op{V}_{ee}
			\mket{\Phi_2}_{\Nelec}
			\label{eqn:FCIMatrixElements}
		\end{equation}
		for all combinations $\Phi_1, \Phi_2 \in \set{B}_{\Nelec}^{(n)}$.
		There are
		\[ \Nfci = \binom{\Nbas}{\Nelec} \leq \Nbas^{\Nelec} \]
		such Slater determinants.
	\item Diagonalise $\mat{A}_\text{FCI}$ to find a few energy eigenvalues
		$E_i^{(n)} \in \R$ and corresponding \newterm{CI vectors}
		$\vec{c}^{(n)}_i \in \C^{\Nfci}$.
	\item Repeat with a larger basis $\set{B}_1^{(n+1)}$ until convergence
		of eigenstates up to desired accuracy has been achieved.
		In many cases one already selects a suitable basis set
		$\set{B}_1^{(n)}$ and only performs the calculation top-to-bottom once.
	\end{itemize}
	Notice that the subspace sequence
	$\spacespan \set{B}_{\Nelec}^{(n)} \subset \tilde{Q}(\Op{H}_{\Nelec})$
	satisfies the required condition \eqref{eqn:CondSubspaces}
	since $U$ is dense in $H^1(\R^3, \C)$,
	which makes $\spacespan \set{B}_{\Nelec}^{(n)}$
	dense in $\tilde{Q}(\Op{H}_{\Nelec})$
	and thus transitively dense in $Q(\Op{H}_{\Nelec})$.
	If $\Op{H}_{\Nelec}$ thus has a discrete ground and some discrete
	excited states below the essential spectrum,
	we can approximate it by this procedure up to arbitrary accuracy.
	This is satisfied for all neutral or positively charged systems
	and some negatively charged systems.
	Recall remark \vref{rem:ElectronicTISENumerical} for details.
\end{rem}
Equation \eqref{eqn:FCIMatrixElements} helps to understand
where the term full configuration interaction for the sketched
method comes from.
In some sense a basis function in the single-particle
basis $\{\varphi_\mu\}_{\mu\in\Ibas}$
describes the behaviour of a single electron.
In turn a Slater determinant can be interpreted physically
as one sensible configuration of $\Nelec$ electrons
amongst the available single-particle functions.
The full CI matrix \eqref{eqn:FCIMatrixElements}
now couples these configurations
via the electronic Hamiltonian $\Op{H}_{\Nelec}$,
which describes the interaction of the electrons in the chemical
system with another.
By diagonalising the matrix $\mat{A}_\text{FCI}$ we thus determine
the electronic eigenstates taking the full range of interactions between
all configurations into account, explaining the name full CI.

Even though \FCI allows to compute the solution of the electronic
Schrödinger equation up to arbitrary accuracy,
it is not employed a lot in practice.
The main reason for this is its enormous computational cost.
Already for a small molecules like water with only
$\Nelec = 10$ electrons and a rather small def2-sv(p)\todo{cite} basis set
$\set{B}_1^{(n)}$ with $\Nbas = 18$ basis functions this makes $\Nfci = 43758$
and thus around $\Nfci^2 = 2\E{9}$ entries in $\mat{A}_\text{FCI}$
in an extremely naive implementation.
Of course this can be improved by exploiting some symmetries
or the rather sparse structure of $\mat{A}_\text{FCI}$,
which we will discuss in the next remark.
Nevertheless the computational cost scales exponentially
and allows only treatment of extremely small systems.

\defineabbr{ERI}{ERI\xspace}{Electron repulsion integrals}
\newcommand{\eriAsym}[2]{\left\langle #1 \middle|\middle| #2 \right\rangle}
\newcommand{\eriPh}[2]{\left\langle #1 \middle| #2 \right\rangle}
\newcommand{\eriMu}[2]{\left( #1 \middle| #2 \right)}
\nomenclature{$(ij|kl)$}{%
	Electron repulsion integrals in chemist's or Mullikan notation,
	see definition in remark \vref{rem:ERI}.
}
\begin{rem}[Structure of the \FCI matrix]
	\label{eqn:EvalFCIMatrix}
	Recall the expression
	\[
		\Op{H}_{\Nelec}
		= \Op{T}_e + \Op{V}_{en} + \Op{V}_{ee}
		= \sum_{i=1}^{\Nelec}
		  \left( -\frac12 \Delta_{\vec{r}_i} + \sum_{A=1}^M \frac{Z_A}{r_{iA}} \right)
		+ \sum_{i=1}^{\Nelec} \sum_{j=i+1}^{\Nelec} \frac{1}{r_{ij}}
	\]
	for the electronic Hamiltonian.
	The goal of this remark will be to write the many electron
	integrals $\braket{\Phi_1}{\Op{H}_{\Nelec}\, \Phi_2}$ between
	two Slater determinants $\Phi_1, \Phi_2 \in \set{B}_{\Nelec}$
	in terms of integrals over the one-electron functions $\psi_i$
	these determinants are composed of.

	\todoil{Derivation of Slater-Condon rules in appendix?}
	For this we will make use of the so-called Slater-Condon rules~\cite{Szabo1996}.
	For applying these rules
	we need to differentiate between so-called \newterm{one-electron operator}s
	and \newterm{two-electron operator}s.
	One-particle operators like $\Op{T}_e$ and $\Op{V}_{en}$
	can be written as a sum of operators like $\Delta_{\vec{r}_i}$ or $r_{iA}^{-1}$,
	which act only on the coordinate $\vec{r}_i$ of a single electron at once.
	Two-particle operators like $\Op{V}_{ee}$, however,
	are built as a sum of terms $r_{ij}^{-1}$ making reference
	to the coordinates of two electrons.

	For our discussion here, let us take $\Phi_0$ to be an arbitrary
	reference determinant constructed from the single-particle basis $\set{B}_1$.
	We construct excited determinants
	$\Phi_i^a$, $\Phi_{ij}^{ab}$, \ldots
	under the index conventions
	\eqref{eqn:CondOccIndex} and \eqref{eqn:CondVirtIndex}.

	For the one-electron operator $\Op{T}_e + \Op{V}_{en}$
	the Slater-Condon rules yield
	\begin{equation}
		\begin{aligned}
		\braket{\Phi_0}{\left(\Op{T}_e + \Op{V}_{en}\right) \,\Phi_0}_{\Nelec}
		&= \sum_{i\in\Iocc} \braket{\psi_i}{\Op{H}_{\text{core}} \, \psi_i}_1 \\
		%
		\braket{\Phi_0}{\left(\Op{T}_e + \Op{V}_{en}\right) \,\Phi_{i}^{a}}_{\Nelec}
		&= \braket{\psi_i}{\Op{H}_\text{core} \, \psi_a}_1 \\
		%
		\braket{\Phi_0}{\left(\Op{T}_e + \Op{V}_{en}\right) \,\Phi_{ij}^{ab}}_{\Nelec}
		&= 0.
		\label{eqn:SlaterCondonHcore}
		\end{aligned}
	\end{equation}
	In this result we made use of the \newterm{core Hamiltonian} operator
	\begin{equation}
		\Op{H}_\text{core}
		= \Op{T} + \Op{V}_0
		= -\frac12 \Delta + \sum_{A=1}^M \frac{Z_A}{\norm{\vec{r} - \vec{R}_A}_2},
		\label{eqn:HCore}
	\end{equation}
	which is just the sum of the kinetic operator $\Op{T}$
	and the nuclear attraction operator $\Op{V}_0$
	contribution from a single electron.
	Since the choice of the reference determinant $\Phi_0$
	was arbitrary, we can state more generally that the element
	$\braket{\Phi_1}{\Op{A}_1 \Phi_2}_{\Nelec}$
	of a one-particle operator $\Op{A}_1$
	is only non-zero for determinants $\Phi_1$, $\Phi_2$,
	which differ in at most one single-particle function.

	On the other hand for
	two-electron operators like $\Op{V}_{ee}$
	the Slater-Condon rules yield
	\begin{equation}
	\begin{aligned}
		\braket{\Phi_0}{\Op{V}_{ee}\, \Phi_0}_{\Nelec}
		&= \frac12 \sum_{i\in\Iocc} \sum_{j\in\Iocc}
			  \eriMu{\psi_i\psi_i}{\psi_j\psi_j}
			- \eriMu{\psi_i\psi_j}{\psi_i\psi_j} \\
		%
		\braket{\Phi_0}{\Op{V}_{ee}\, \Phi_i^a}_{\Nelec}
		&= \sum_{j\in\Iocc}
			  \eriMu{\psi_i\psi_a}{\psi_j\psi_j}
			- \eriMu{\psi_j\psi_a}{\psi_i\psi_j} \\
		%
		\braket{\Phi_0}{\Op{V}_{ee}\, \Phi_{ij}^{ab}}_{\Nelec}
		&= \eriMu{\psi_i\psi_j}{\psi_a\psi_b}
		 - \eriMu{\psi_a\psi_j}{\psi_i\psi_b} \\
		%
		\braket{\Phi_0}{\Op{V}_{ee}\, \Phi_{ijk}^{abc}}_{\Nelec} &= 0.
	\end{aligned}
		\label{eqn:SlaterCondonCoulomb}
	\end{equation}
	where the \newterm{electron repulsion integrals}~(ERIs)
	in Mulliken's indexing convention are given by the expression
	\begin{equation}
		\eriMu{\psi_i \psi_j}{\psi_k \psi_l}
			= \int_{\R^3} \int_{\R^3}
				\frac{\cc{\psi_i}(\vec{r}_1) \psi_j(\vec{r}_1)
					\,\cc{\psi_k}(\vec{r}_2) \psi_l(\vec{r}_2)}
				{\norm{\vec{r}_1 - \vec{r}_2}_2}
				\D \vec{r}_1 \D \vec{r}_2.
		\label{eqn:ERI}
	\end{equation}
	Again this result generalises in the sense
	that for a two particle operator $\Op{A}_2$
	and any determinants $\Phi_1$ and $\Phi_2$
	the elmement $\braket{\Phi_1}{\Op{A}_2 \Phi_2}_{\Nelec}$
	is only non-zero if the determinants
	differ in at most two single-particle functions.

	Both these observations combined allow to deduce
	that the full CI matrix $\mat{A}_\text{FCI}$ must be rather sparse.
	Originating from the two-electron Coulomb term $\Op{V}_{ee}$
	all entries $a(\Phi_1, \Phi_2)$ where the determinants
	differ by more than two functions vanish.
	If we pick an arbitrary reference determinant $\Op{Phi}_0$
	and order the $\Nelec$-electron basis as in equation \eqref{eqn:NelectronBasisFinite}
	a banded structure as in fig. \vref{fig:StructureFCIMatrix} results.
	Of course the dimensionality is still rather large,
	but a combination of the iterative methods sketched
	in section \vref{sec:DiagAlgos}
	and a \contract-based ansatz like the one sketched in section \vref{sec:ContractionAlgos}
	allow to obtain a few eigenvalues of $\mat{A}_\text{FCI}$
	exploiting the sparsity structure.
\end{rem}
\begin{figure}
	\centering
	\includeimage{3_qchem/fci_matrix}
	\caption{
		Schematic sketch of upper left part of the FCI matrix $\mat{A}_\text{FCI}$.
		The identified blocks denote the
		interaction of equivalent classes of excited determinants
		under the electronic Hamiltonian $\Op{H}_{\Nelec}$.
		The size of the blocks increases left to right and top to bottom
		and is not depicted to scale.
		Blocks with white background are identically zero
		and blocks with grey background may contain non-zero elements.
		Notice, that the grey blocks may show further
		sparsity, which is not fully depicted here.
	}
	\label{fig:StructureFCIMatrix}
	\todoil{More detailed: \url{https://arxiv.org/abs/1609.07757}}
\end{figure}

The electron repulsion integral tensor introduced in \eqref{eqn:ERI}
is a very important quantity in computational chemistry.
We will require it at various occasions throughout the thesis.
In the standard literature a number of deviating conventions are used
for denoting this tensor. The following remark provides a summary.

\begin{rem}[Formulation of the repulsion integrals]
	\label{rem:ERI}
	In equation \eqref{eqn:ERI} we already met
	the electron repulsion integal $\eriMu{\psi_i \psi_j}{\psi_k \psi_l}$
	in \textbf{Mulliken notation}.
	Alternative names for this indexing convention are
	\textbf{shell pair notation} or \textbf{chemists' notation}.
	If the one-particle basis and its indexing convention is clear
	from context one sometimes writes this integral
	abbreviated as $\eriMu{ij}{kl}$ as well.

	\noindent
	An alternative convention is physicists' notation
	\[
		\eriPh{i j}{k l} \equiv
		\eriPh{\psi_i \psi_j}{\psi_k \psi_l}
		= \int_{\R^3} \int_{\R^3}
				\frac{\cc{\psi_i}(\vec{r}_1) \cc{\psi_j}(\vec{r}_2)
				\, \psi_k(\vec{r}_1) \psi_l(\vec{r}_2)}
				{\norm{\vec{r}_1 - \vec{r}_2}_2}
				\D \vec{r}_1 \D \vec{r}_2.
	\]
	Both conventions are related by
	\begin{equation}
		\eriPh{i j}{k l} = \eriMu{i k}{j l}.
		\label{eqn:EriMulPhys}
	\end{equation}
	It is a rather common feature that the ERI integrals appear in pairs
	like in \eqref{eqn:SlaterCondonCoulomb},
	where the indices are only slightly permuted.
	For this reason one typically defines an
	\newterm{antisymmetrised electron repulsion tensor}
	with elements
	\[ \eriAsym{ij}{kl} \equiv
		\eriAsym{\psi_i \psi_j}{\psi_k \psi_l}
		\equiv \eriPh{\psi_i \psi_j}{\psi_k \psi_l}
		- \eriPh{\psi_j \psi_i}{\psi_k \psi_l}
		= \eriMu{\psi_i\psi_k}{\psi_j\psi_l} - \eriMu{\psi_j\psi_k}{\psi_i\psi_l}
	\]
	as well. With this quantity the element $a(\Phi, \Phi)$,
	where the quadratic form is applied to an arbitrary normalised
	determinant $\Phi = \psi_1 \wedge \psi_2 \wedge \cdots \wedge \psi_{\Nelec}$
	can be written as
	\begin{equation}
		a(\Phi, \Phi) = \braket{\Phi}{\Op{H}_{\Nelec} \, \Phi}_{\Nelec}
		= \sum_{i\in\Iocc} \braket{\psi_i}{\Op{H}_\text{core}\, \psi_i}_1
			+ \frac12 \sum_{i\in\Iocc} \sum_{j\in\Iocc} \eriAsym{ij}{ij}.
		\label{eqn:EnergySlaterDeterminant}
	\end{equation}

	Originating from the integral expression \eqref{eqn:ERI}
	both the ERI tensor as well as the antisymmetrised ERI tensor
	show a lot of symmetry with respect to index permutations.
	An overview of these symmetry properties provides appendix \vref{apx:ERIProps}.
\end{rem}

\section{Single-determinant ansatz}
\label{sec:HFIntro}

\nomenclature{$\NelecA, \NelecB$}{Number of $\alpha$/$\beta$ electrons.
Note $\NelecA + \NelecB = \Nelec$ and $\NelecA \ge \NelecB$ by convention.}
\nomenclature{$\Iorb$}{Index set of computed SCF orbitals, typically $\{1, \ldots \Norb\}$}
\nomenclature{$\Norb$}{The number of computed SCF orbitals.
Note, that $\Norb \le \Nbas$.}

\defineabbr{HF}{HF\xspace}{Hartree-Fock}
\nomenclature{$\OpFock, \OpFockFull$}{$1$-electron Fock operator (dependent on its eigenfunctions $\{ \varphi_i \}_{i \in \Iorb}$)}
\nomenclature{$\Op{\rho}$}{Density operator}
\nomenclature{$\IoccA$, $\IoccB$}{Index set of occupied SCF orbitals of $\alpha$ or $\beta$ spin, respectively. Typically $\{1, \ldots, \NelecA\}$ and similar for $\IoccB$.}
\nomenclature{$\epsilonconv$}{Convergence tolerance for an iterative process.
	If the error is below this value, the process should be considered converged.}

\nomenclature{$\Op{K}$}{One-electron exchange operator}
\nomenclature{$\Op{J}$}{Effective Coulomb operator}

In the previous section we noted that even an approximate
solution to the electronic Schrödinger equation \eqref{eqn:ElectronicSchrödinger}
via the full CI ansatz
is hardly feasible.
Even if comparatively small one-electron basis sets
$\{ \varphi_\mu \}_{\mu\in\Ibas} \subset H^1(\R^3,\C)$
are used,
the dimensionality
of the matrix $\mat{A}_\text{FCI}$ becomes simply too large.
In this section we discuss the opposite end of the scale and
only consider one-dimensional subspaces of the form domain
\[ Q(\Op{H}_{\Nelec}) = \bigwedge_{i=1}^{\Nelec} H^1(\R^3, \C) \cap L^2(\R^3, \C) \]
of the electronic Hamiltonian.
Formally by the Courant-Fischer theorem \eqref{thm:CourantFischer}
the ground state electronic energy can be obtained by
a variational minimisation over all subspaces of dimension $1$,
or more formally
\begin{align}
	\label{eqn:GroundStateVariational}
	E_0 &= \inf_{\Psi \in \mathcal{W}_{\Nelec}} \braket{\Psi}{\Op{H}_{\Nelec} \Psi}_{\Nelec} \\
\intertext{where}
	\label{eqn:NormalisedFormDomain}
	\mathcal{W}_{\Nelec} &= \left\{ \Psi \in Q(\Op{H}_{\Nelec})
	\, \middle| \,
	\norm{\Psi}_{L^2(\R^{3 \Nelec}, \C)} = 1 \right\}.
\end{align}
denotes the subspace of all normalised functions from $Q(\Op{H}_{\Nelec})$.
If we restrict the search to run only over the subspace
\begin{equation}
	\mathcal{R}^1_{\Nelec} = \left\{ \bigwedge_{i=1}^{\Nelec} \psi_i
	\, \middle| \,
	\left( \psi_1, \psi_2, \ldots, \psi_{\Nelec} \right) \in H^1(\R^3, \C),
	\braket{\psi_i}{\psi_j}_1 = \delta_{ij}
	\right\}
	\label{eqn:RankOneSubspace}
\end{equation}
of all normalised Slater determinants we do not get the exact
energy much rather we obtain an upper bound
\begin{equation}
	E_0 \leq E_0^\text{HF} = \inf_{\Phi \in \mathcal{R}^1_{\Nelec}}
	\braket{\Phi}{\Op{H}_{\Nelec} \Phi}_{\Nelec}
	\label{eqn:HFDeterminant}
\end{equation}
according to corollary \vref{cor:Convergence}.
The implied procedure
to obtain an approximation to the electronic ground state
by minimising the sesquilinear form of $\Op{H}_{\Nelec}$
over the space spanned by all normalised Slater determinants
is the celebrated \newterm{Hartree-Fock}~(\HF) approximation.
The resulting minimal energy $E_0^\text{HF}$
is the \HF ground state energy
and the corresponding minimising determinant
$\Phi_0$ the \HF ground state.

Notice that $\Phi_0$ is --- by construction ---
the best possible single Slater determinant to
approximate the electronic ground state.
Mathematically speaking the set of Slater determinants
$\mathcal{R}^1_{\Nelec}$
is exactly the set of all elements from $\mathcal{W}_{\Nelec}$,
which are of tensor rank 1.
For this reason one sometimes refers to the Hartree-Fock
ground state $\Phi_0$
as a \textbf{rank-1 approximation} to the exact
electronic ground state.

\begin{rem}[Molecular orbital formulation of \HF]
	Using equations \eqref{eqn:EnergySlaterDeterminant}
	and \eqref{eqn:RankOneSubspace}
	the \HF problem \eqref{eqn:HFDeterminant}
	can be re-formulated in terms of the molecular orbital
	basis $\{\psi_i\}_{i\in\Iorb}$
\end{rem}

The following theorem summarises some mathematical
results regarding \HF.
\begin{thm}
	\todoil{Seek literature for Cances stuff}
\end{thm}
\todoil{Show these results in the discrete case (see MWM example sheet) in appendix.}

% define occupied and virtual

% mention term SCF
% spin generalisation?

\begin{rem}[Spin]
	bla
\end{rem}

% TODO Mention some theoretical result regarding well-posedness,
%      solutions and so on

\begin{rem}[discretisation]
	\todo[inline,caption={}]{
		\begin{itemize}
			\item 
	$\varphi$ \newterm{one-particle basis} for HF
	\item
	$\psi$ HF orbitals or (SCF) orbital basis,
	typically in FCI one uses the HF orbital basis as the one-particle basis functions for FCI
		\end{itemize}
}
\end{rem}


% ----------------------------------ü



\todoil{
In our discussion so far we neglected the spin component
of the wavefunction
$L^2(\R^{3\Nelec}, \C)$ to $L^2(\R^{3\Nelec}, \C^{2\Nelec})$

two-component spinor representation

	Everything works the same.
	Especially if one thinks of $\alpha$-spin orbitals
	as those which only produce non-zero in the first component
	of the $\C^2$ vector and $\beta$-spin as only producing
	values on the second one.
	Then we also have $\Nelec$ orbitals to occupy
	$\Nelec$ electrons, but their spatial part may be the same or different.
}


\todoil{Present explicit expressions for the individual
	terms. Perhaps already introduce the J and K operators
and use the functional form and formalism of Cances}

% Note that we need a more general setting here.
% Discuss how spin is included
% Slater determinants with spin
\todoil{
	Quick note that real is possible: \\
For an atomic system, we can consider the nucleus to be located at the origin
of the coordinate system, which allows us to simplify
the Hamiltonian to \\
Argue that we can do the mathematical background in real function
arithmetic only and still use the results for complex input due to linearity.
$\Psi \in H^2(\R^{3d}, \R)$
}




% Density matrix
% Density operator
% Coefficient matrix

\newterm{Hartree-Fock}
\todo[inline,caption={}]{
	\begin{itemize}
		\item Idea
		\item Discretisation
		\item use Formalism of Cances
		\item Derivation of HF in discretised space
		\item Summary of the numerical task
		\item Summary of the result (SCF orbitals)
		\item Explain properties of the individual terms of Fock operator (locality, ..)
		\item Show what we miss compared to FCI
	\end{itemize}
}

The discretised Hartree-Fock equations in weak form read
\begin{equation}
	\forall i, j \in \Iorb :  \mbra{\varphi_i} \OpFockFull \mket{\varphi_j} = \lambda_j \braket{\varphi_i}{\varphi_j}
	\label{eqn:HF}
\end{equation}

\begin{equation}
	C^\alpha_{\mu i}, C^\beta_{\mu i}
	\label{eqn:SCFalphaBetaCoeffs}
\end{equation}

% Proove the Fock operator (for fixed orbitals) is linear in the function
% it acts upon



\section{Capturing electronic correlation}
\label{sec:Correlation}

In the previous section we talked at length about the \HF approximation for solving
the non-relativistic electronic Schrödinger equation.
Since the search space for the variational minimisation is much more restricted compared
to the \FCI ansatz,
we necessarily make a larger approximation error in the \HF case.
Nevertheless it should be noted that \HF yields a rather good rank-1
approximation to the full Schrödinger problem,
where up to $99\%$ of the
\FCI energy~\cite{Jensen2007book} in a respective basis is obtained
at a fraction of the cost.
Unfortunately chemistry is about energy differences and not about absolute energies.
For example to a good approximation,
chemical reactivity can be determined by looking at the
energy barrier between reactants and products,
\ie the difference in energy between the reactants
and the maximal energy,
which is obtained along the reaction path transforming them to products.
As the difference matters
getting $99\%$ of the absolute energy
typically still leads to much larger errors than $1\%$ for the reaction barrier.
One might therefore wonder
what part of the exact physical picture \HF is missing and how one could improve on that.

\subsection{What does Hartree-Fock miss?}
\label{sec:FailureHF}
Even though the
Fock operator \eqref{eqn:FockOperator} describes a many-electron system,
it is a one-particle operator,
since it only acts on single-electron functions.
The many-particle aspect is only treated via the Coulomb term \eqref{eqn:OperatorCoulomb}
and the exchange term \eqref{eqn:OperatorExchange},
where the interaction with
other electrons is included in the form of integrals over
the electron density $\rho_{\Theta^0}$ or the density matrix $\gamma_{\Theta^0}$.
Overall an electron thus does not see the exact position of all its neighbours
via the Fock matrix,
but only some kind of an average electron field.
In this sense the \HF ansatz is sometimes called a \newterm{mean-field approximation}.
In the light of this the \SCF can be thought of as an adjustment procedure,
where the electronic arrangement in the form of the occupied \SCF orbitals $\{\psi_i\}_{i\in\Iocc}$
is adjusted until their generated mean field
is no longer changing this arrangement,
\ie is self-consistent.

\begin{figure}
	\centering
	\includeimage{3_qchem/planetary_systems}
	\caption[Real planetary system and mean-field model of it shown side-by-side]{
		Real planetary system and mean-field model
		in the spirit of the \HF approximation
		shown side-by-side.
		The mean-field picture on the right-hand side
		is shown from the point of view of the red planet,
		such that its neighbours are smeared out as thick black
		circles over their respective orbits.
		Adapted from \cite{Jensen2007book}.
	}
	\label{fig:HFplanets}
\end{figure}
To visualise this issue better,
let us consider in analogy a planetary system%
\footnote{The idea is taken from \cite{Jensen2007book}.},
where multiple planets are revolving around a central sun.
In the real system,
which is depicted on the left-hand side of figure \vref{fig:HFplanets},
the individual planets feel each other at all times
at their exact positions.
As a result their orbit around the sun is not a perfect circle but shows pronounced
wiggles due to the interaction between the planets.
In other words the motion of the planets around the sun is highly correlated.
In contrast to this the right-hand side
depicts the scenario drawn for the red planet in a \HF-like mean-field model.
Its neighbours are no longer visible at their exact positions
and the red planet thus only amounts to interact with
some sort of smeared out particle density,
where their position is averaged over their complete orbits.
This interaction is almost as strong at all points
and thus the mean-field orbit of the red planet is much more smooth.

In the electronic system the situation is similar
in sense that the behaviour of individual electrons is indeed very much correlated.
Due to its mean-field nature the \HF ansatz
largely misses the description of this so-called \newterm{electron correlation}%
\footnote{%
	Conventionally one calls the \HF treatment of a chemical system
	the \emph{uncorrelated} treatment of the electronic structure.
	This is not perfectly sound in my opinion,
	as for example the Pauli principle is fulfilled in \HF.
	This implies for example that two electrons of the same spin
	cannot occupy the same orbital,
	which implies in turn that the motion
	of electrons of the same spin is at least to this extend correlated.
}.
In fact one typically refers to the difference
\begin{equation}
	E_0^{\text{corr}} = E_0^{\text{FCI}} - E_0^{\text{HF}}
	\label{eqn:EnergyCorrelation}
\end{equation}
between the \HF and \FCI energies in a particular basis set
as the \newterm{correlation energy}.
As mentioned before $E_0^{\text{corr}}$ is typically rather small
compared to $E_0^{\text{HF}}$.
Nevertheless the effects of the electron-electron interaction are very important
for a proper description of the electronic structure of a chemical
system and can therefore not be
neglected~\cite{Helgaker2013,Szabo1996,Jensen2007book}.

In practice one sometimes divides correlation effects
into two subclasses.
The first kind, the so-called \newterm{dynamic correlation},
is the aforementioned failure of the \HF approximation
to treat the communal, correlated motion of electrons properly.
The second kind, \newterm{static correlation},
occurs if the number of Slater determinants,
which is available for the description of a degenerate or near-degenerate
state is not sufficient.
For the \HF approximation,
where only one determinant for the description of the ground state
is available,
this defect becomes apparent in situations
with a low-lying excited state, for example.
A classic example would be a molecule close to bond breaking.
In such a case the ground state resulting from a full CI treatment
has relevant contributions from more than one determinant.
As a result even the best restricted \HF ground state determinant misses a substantial
part of the full CI ground state and thus provides
a wrong description of the physics.
In the remainder of this discussion about electron correlation
we will ignore static correlation and assume that a single determinant
\HF ground state is already a pretty decent description of the electronic structure.
Detailed discussions of so-called multi-reference
or multi-configurational methods tackling static correlation
can be found for example in \cite{McWeeny1985,Fischer1986,Jensen2007book,Helgaker2013}.

\subsection{Truncated configuration interaction}
\label{sec:TruncatedCI}
In section \vref{sec:FCI} about the \FCI method we already
mentioned that the exact ground state $\Psi_0$ to the electronic Schrödinger equation
can be expressed as a CI expansion \eqref{eqn:CIExpansion}%
\footnote{
	In this section about correlation methods
	we will adhere to the usual index conventions
	where occupied indices are denoted with letters
	$i,j,k,l \in \Iocc$ and virtual indices with letters $a,b,c,d \in \Ivirt$,
	see remark \vref{rem:FCI} for details.
	To avoid clutter we will usually not indicate the index set in sums explicitly,
	\eg write $\sum_a$ instead of $\sum_{a \in \Ivirt}$.
}
\[
	\Psi
	= c_0 \Phi_0 + \sum_{ia} c_i^a \Phi_i^a
	+ \sum_{\substack{i<j \\a<b}} c_{ij}^{ab} \Phi_{ij}^{ab}
	+ \sum_{\substack{i<j<k \\ a<b<c}} c_{ijk}^{abc} \Phi_{ijk}^{abc}
	 + \cdots,
\]
starting from an arbitrary reference determinant $\Phi_0$.
A very natural choice for this is to take the reference
determinant to be the \HF ground state,
\ie the best single determinant for describing the electronic ground state.
In this way the
remaining contributions of the excited determinants $\Phi_i^a$,
$\Phi_{ij}^{ab}$, $\Phi_{ijk}^{abc}$, $\ldots$
can be expected to be small,
which makes it numerically more feasible to diagonalise
the \FCI matrix $\mat{A}_\text{FCI}$.
Furthermore this justifies truncating the CI expansion \eqref{eqn:CIExpansion}
prematurely to yield some sort of an intermediate approximation
between \HF and \FCI.
For example CISD, configuration interaction singles-doubles~\cite{Sherrill1999},
truncates the above expansion in a way that only singles and doubles
excitations are taken into account.
This would lead to the ansatz wave function
\[
	\Psi_0^\text{CISD}
	= \Phi_0 + \sum_{ia} c_i^a \Phi_i^a
	+ \sum_{\substack{i<j \\a<b}} c_{ij}^{ab} \Phi_{ij}^{ab}
\]
where one assumes the individual determinants
are normalised in a way that $c_0 = 1$.

\pagebreak[1]
Even though truncated CI methods are conceptionally very simple,
they are not used much any more for capturing dynamic correlation%
\footnote{%
	In contrast to this statement the related
	multi-reference CI ansatz~\cite{Shamasundar2011}
	\emph{is} a state-of-the-art method
	for dealing with \emph{statically} correlated systems.
}
The main reason for this is the so-called \newterm{size-consistency problem}.
Unlike \FCI the CISD energy of two molecular fragments
is in general not additive,
even if these fragments do not interact.
Put more mathematically one can show~\cite{Helgaker2013} the following:
If $E_A$ is the energy corresponding to the CISD ground state $\Psi_0^\text{CISD}$
for a molecule $A$
and $E_B$ is the analogous energy for another molecule $B$,
then the CISD ground-state energy $E_{AB}$
for a system consisting of both $A$ \emph{and} $B$
separated by an infinite distance is \emph{not} $E_A + E_B$.
One refers to this unphysical behaviour as \newterm{size-inconsistent}.
Including higher excitations
does not fix this problem,
such that all canonical truncated CI methods are size-inconsistent.
For the modelling of chemical reactions or even large molecules,
size-inconsistency is a major problem.
Nowadays better, size-consistent alternatives like the
coupled-cluster ansatz~(see below) exist
and are usually preferred.

\defineabbr{MP}{MP\xspace}{Møller-Plesset perturbation theory treatment of electron-electron correlation.
Typically followed by a number to indicate the order.}
\subsection{Second order Møller-Plesset perturbation theory}
\label{sec:MP}

Starting from the reasonable
assumption that the \HF ground state determinant $\Phi_0$
is a very good approximation to the exact electronic ground state
it is a sensible ansatz to
employ Rayleigh-Schrödinger perturbation theory~\cite{Szabo1996,Helgaker2013}
and correct perturbatively for the missing correlation contribution
to the energy as well as the wave function.
The typical perturbation theory ansatz is to
partition the electronic Schrödinger Hamlitonian \eqref{eqn:ElectronicHamiltonian} into
\[ \Op{H}_{\Nelec} = \Op{H}^0 + \Op{H}^1, \]
\ie a zeroth order Hamiltonian $\Op{H}^0$,
which is easy to compute,
and the perturbation $\Op{H}^1$,
which is the part missed in $\Op{H}^0$,
assumed to be small.

One way this partitioning can be achieved is Møller-Plesset
perturbation theory~\cite{Moeller1934},
where the unperturbed operator is taken to be the
direct sum of $\Nelec$ Fock operators
at the orbital configuration corresponding to the \HF ground state $\Phi_0$,
\begin{align*}
	\Op{H}^0 &= \bigoplus_{i=1}^{\Nelec} \Op{F}_{\Theta^0},
	\intertext{
and the perturbation is}
	\Op{H}^1 &= \Op{V}_{ee} - \bigoplus_{i=1}^{\Nelec} \left( \Op{J}_{\Theta^0} + \Op{K}_{\Theta^0} \right),
\end{align*}
\ie whatever the \HF operator misses.
In the discretised setting of a finite-dimensional one-particle basis
$\{\varphi_\mu\}_{\mu\in\Nbas}$
one may easily derive the zeroth to second order
energy contributions~\cite{Szabo1996}
\begin{align*}
	E_0^0 &= \sum_i \varepsilon_i, \\
	E_0^1 &= - \frac12 \sum_{ij} \eriAsym{ij}{ij},  \\
	E_0^2 &= \frac14 \sum_{ijab} \frac{\abs{\eriAsym{ij}{ab}}^2}
						{\varepsilon_i + \varepsilon_j - \varepsilon_a - \varepsilon_b}
\end{align*}
to the ground-state energy.
In other words up to zeroth order we obtain the sum of the orbital energies.
The first order correction accounts for the double counting
of the electron-electron interactions and recovers the \HF energy expression.
The first real improvement to \HF results
at \newterm{second order Møller-Plesset perturbation theory}~({\MP}2).
For reasons, which will become clear in the next section
one often introduces the so-called \newterm{$T_2$ amplitude}
\begin{equation}
	t_{ij}^{ab} \equiv \frac{\eriAsym{ij}{ab}}
		{\varepsilon_i + \varepsilon_j - \varepsilon_a - \varepsilon_b}
	\label{eqn:amplitudesMPtwo}
\end{equation}
and writes the {\MP}2 energy as
\begin{equation}
	E_0^\text{MP2} = E_0^\text{HF} + \frac14 \sum_{ijab} \cc{\eriAsym{ij}{ab}} t_{ij}^{ab}.
	\label{eqn:energyMPtwo}
\end{equation}
The \MP methods do have some issues as well.
Most notably the perturbation expansion of energies does in general
not converge~\cite{Helgaker2013},
making it hard to properly justify these methods from a mathematical basis.
In practice {\MP}2 is still vividly employed,
mainly because it gives a decent guess
towards the exact energy of the electronic ground state
at manageable computational cost%
\footnote{Using sensible approximations linear-scaling {\MP}2 is possible%
~\cite{Zalesny2011}.}.
The other \MP methods on the other hands
are nowadays used only rarely.

One should mention that due to its perturbative nature,
there is no guarantee that $E_0^\text{MP2} \geq E_0$,
the exact ground-state energy of the electronic Schrödinger equation \eqref{eqn:ElectronicSchrödinger}.
In the community of quantum chemistry one often refers
to this fact as {\MP}2 being non-variational.
This saying is, however, a bit inaccurate,
since the method {\MP}2 is indeed variational
in the sense of the Courant-Fischer theorem \eqref{eqn:CourantFischer},
namely that larger basis sets will always lead a lower-energy {\MP}2 ground state,
which is furthermore closer to the exact {\MP}2 ground-state wave function.
This is not that much apparent in the outlined derivation,
but can be seen from an alternative route
employing the Hylleraas functional~\cite{Helgaker2013}.
In contrast it is not variational in the sense that
a larger basis yields an {\MP}2 energy
which approaches $E_0$ from above.

\subsection{Coupled-cluster theory}
\defineabbr{CC}{CC\xspace}{Coupled cluster}
\label{sec:CC}

The main idea of coupled-cluster theory
is to employ a more elaborate ansatz for the ground-state wave function
with the overall aim to
reach a size-consistent method.
In this work coupled-cluster only plays a minor role.
This section will therefore be limited to the absolutely necessary
steps to get the rough idea.
For a more thorough introduction the reader is directed
to the excellent review by \citet{Crawford2007}
as well as numerous other works~\cite{Helgaker2013,Hodecker2016}
dealing with the topic.

\noindent
In coupled-cluster theory one starts from the so-called exponential ansatz
\begin{equation}
	\Psi^\text{CC} = \exp(\op{T}) \Phi_0
	\label{eqn:AnsatzCC}
\end{equation}
to generate the coupled-cluster wave function $\Psi^\text{CC}$
from a \HF ground state reference determinant $\Phi_0$.
In this equation
\begin{equation}
	\op{T} = \op{T}_1 + \op{T}_2 + \cdots + \op{T}_{\Nelec}
	\label{eqn:ExcitationCC}
\end{equation}
is the \newterm{excitation operator}
consisting of all singles excitations
\begin{align*}
	\op{T}_1 &= \sum_{ia} t_i^a \op{\tau}_i^a
\intertext{with $\op{\tau}_i^a$ defined, such that $\Phi_i^a = \op{\tau}_i^a \Phi_0$,
all doubles excitations}
	\op{T}_2 &= \sum_{\substack{i<j \\a<b}} t_{ij}^{ab} \op{\tau}_{ij}^{ab}
		&&\text{with}& \Phi_{ij}^{ab} &= \op{\tau}_{ij}^{ab} \Phi_0, \\
\intertext{all triples}
	\op{T}_3 &= \sum_{\substack{i<j<k \\ a<b<c}} t_{ijk}^{abc} \op{\tau}_{ijk}^{abc}
		&&\text{with}& \Phi_{ijk}^{abc} &= \op{\tau}_{ijk}^{abc} \Phi_0, \\
\end{align*}
and so forth.
In these sums the
coefficients $t_i^a$, $t_{ij}^{ab}$, $t_{ijk}^{abc}$ and so forth
are called \newterm{cluster amplitudes}.
In a similar notation to \eqref{eqn:ExpansionSlaterDeterminant}
the excitation operator is often directly written as a sum of the operators
$\op{\tau}_i^a$, $\op{\tau}_{ij}^{ab}$, $\op{\tau}_{ijk}^{abc}$ \ldots, namely as
\begin{equation}
	\op{T} = \sum_\mu t_\mu \op{\tau}_\mu,
	\label{eqn:ExcitationCCsum}
\end{equation}
where $\mu$ is an appropriately chosen multi-index and the sum
is implicitly taken to have sensible limits.

If we allow all possible excitations in \eqref{eqn:ExcitationCC},
\ie do not truncate the sum,
the space spanned by all possible coupled-cluster wave functions $\Psi^\text{CC}$
is exactly equivalent to the space of all Slater determinants,
namely the form domain%
\footnote{Recall the definition in \eqref{eqn:FormDomainAllFunctions}.}
$\tilde{Q}(\Op{H}_{\Nelec})$.
Without truncation \CC is thus equivalent to \FCI,
moreover the exponential ansatz in this case just provides
an alternative to the standard parametrisation of $\tilde{Q}(\Op{H}_{\Nelec})$
in terms of the CI expansion~(see remark \vref{rem:Determinants}).

In the corresponding discretised setting,
$\Phi_0$ is the solution to the discretised
\HF problem~(section \vref{sec:DiscreteHF}).
In a similar fashion to full CI one would expect
a good ansatz for obtaining a \CC approximation to
the ground state of the electronic Schrödinger equation
to use the Ritz-Galerkin ansatz of remark \vref{rem:DiscreteFormulation}.
In other words, one would attempt to solve the variational minimisation problem
\begin{equation}
E_0 \leq E_0^\text{CC} = \inf_{\{t_\mu\}_{\mu}} \frac{\braket{\exp(\op{T}) \Phi_0}{\Op{H}_{\Nelec} \exp(\op{T}) \Phi_0}_{\Nelec}}
	{\braket{\exp(\op{T}) \Phi_0}{\exp(\op{T}) \Phi_0}_{\Nelec}},
	\label{eqn:VariationalCC}
\end{equation}
where there resulting minimising amplitudes give the \CC ground state
wave function corresponding to the minimal ground-state energy $E_0^\text{CC}$.
Without truncation of \eqref{eqn:ExcitationCC}
this is again equivalent to discretised full CI.
Even with truncation to, for example, $\op{T} = \op{T}_1 + \op{T}_2$,
equation \eqref{eqn:VariationalCC} is intractable to solve.
The reason for this is the number of parameters in the problem.
Even with truncation the exponential ansatz
$\exp(\op{T}) \Phi_0$ generates \emph{every} Slater determinant,
such that \eqref{eqn:VariationalCC} yields
a high-dimensional, non-linear problem,
where products of the individual amplitudes $\{t_\mu\}_{\mu}$ often occur
in the resulting system of equations.

For this reason one usually employs a different, so-called \textbf{projection} approach,
which shall only be sketched here%
\footnote{Notice, that some mathematical rigour is dropped here.
	The expression $\Op{H}_{\Nelec} \exp(\op{T}) \Phi_0$
	is only defined properly
	if $\Phi_0 \in H^2(\C^{3\Nelec}, \C^{2\Nelec})$.
	This is, however, not true in general.
	Even in the discretised case one may choose
	a one-particle basis $\{\varphi_\mu\}_{\mu\in\Ibas}$,
	where some functions are not members of $H^2(\C^{3\Nelec}, \C)$.
	As a result $\Phi_0 \not\in H^2(\C^{3\Nelec}, \C^{2\Nelec})$.
}.
If one plugs the exponential ansatz directly into the electronic Schrödinger equation
\eqref{eqn:ElectronicSchrödinger} for the ground state one obtains
\begin{equation}
	\Op{H}_{\Nelec} \exp(\op{T}) \Phi_0 = E_0^\text{CC} \exp(\op{T}) \Phi_0,
	\label{eqn:CCschrödinger}
\end{equation}
where $E_0^\text{CC}$ is the coupled-cluster ground-state energy.
By a simple rearrangement this can be written as
\begin{equation}
	E_0^\text{CC} = \mbra{\Phi_0} \Op{H}_T \mket{\Phi_0}_{\Nelec}
	\label{eqn:CCenergy}
\end{equation}
where we introduced the similarity-transformed Hamiltonian
\[ \Op{H}_T = \exp(-\op{T}) \Op{H}_{\Nelec} \exp(\op{T}). \]
For making use of equation \eqref{eqn:CCenergy}
at all, the unknown amplitudes $\{t_\mu\}_{\mu}$ still need to be found.
This is done by projecting \eqref{eqn:CCschrödinger}
onto determinants $\exp(-\op{T}) \Phi_\mu = \exp(-\op{T}) \op{\tau}_\mu \Phi_0$,
which yields equations
\begin{equation}
	\mbra{\Phi_\mu} \Op{H}_T \mket{\Phi_0}_{\Nelec} = 0
	\label{eqn:CCamplitudes}
\end{equation}
one for each $\mu$.
In truncated \CC methods, where only some of the terms of \eqref{eqn:ExcitationCC} are kept,
we can use \eqref{eqn:ExcitationCCsum} to generate exactly one equation
for each amplitude $\mu$.
In other words, the $\mu$ in \eqref{eqn:CCamplitudes} is just taken to run over the same
index range as in the expansion \eqref{eqn:ExcitationCCsum} for the truncated
excitation operator $\op{T}$.

\defineabbr{CCD}{CCD\xspace}{Coupled-cluster doubles}
Numerically solving for the \CC amplitudes in \eqref{eqn:CCamplitudes}
amounts to a root-finding problem,
where the parameters are the set of all amplitudes $\{t_\mu\}_{\mu}$.
This is typically approached by minimising the residuals
\begin{equation}
	r_\mu = \mbra{\Phi_\mu} \Op{H}_T \mket{\Phi_0}_{\Nelec}
	\label{eqn:CCresidual}
\end{equation}
iteratively until numerically $r_\mu = 0$ for all $\mu$.
Even though this problem is easier compared to the
variational \CC ansatz,
the working equations resulting from the expressions \eqref{eqn:CCresidual}
are typically all but simple.
For example, let us consider one of the simplest coupled-cluster approaches, where
\[ \op{T} = \op{T}_2 = \sum_{\substack{i<j \\a<b}} t_{ij}^{ab} \op{\tau}_{ij}^{ab}, \]
called coupled-cluster doubles~(\CCD).
A proper derivation~\cite{Hurley1976,Bartlett1978,Crawford2007,Hodecker2016}
starting from \eqref{eqn:CCamplitudes}
yields the equations
\begin{equation}
\begin{aligned}
	r_{ij}^{ab}
		&= \eriAsym{ab}{ij} \\
		%
		&+ \sum_e f_{ae} \, t_{ij}^{eb}
		 - \sum_e f_{be} \, t_{ij}^{ea}
		 - \sum_m f_{mi} \, t_{mj}^{ab}
		 + \sum_m f_{mj} \, t_{mi}^{ab} \\
		%
		&+ \frac12 \sum_{mn} \eriAsym{mn}{ij} \, t_{mn}^{ab}
		+ \frac12 \sum_{ef} \eriAsym{ab}{ef} \, t_{ij}^{ef} \\
		%
		&+ \sum_{me} \eriAsym{mb}{ej} \, t_{im}^{ae}
		 - \sum_{me} \eriAsym{mb}{ei} \, t_{jm}^{ae} \\
		&- \sum_{me} \eriAsym{ma}{ej} \, t_{im}^{be}
		 + \sum_{me} \eriAsym{ma}{ei} \, t_{jm}^{be} \\
		%
		&- \frac12 \sum_{mnef} \eriAsym{mn}{ef} \, t_{mn}^{af} \, t_{ij}^{eb}
		 + \frac12 \sum_{mnef} \eriAsym{mn}{ef} \, t_{mn}^{bf} \, t_{ij}^{ea} \\
		&- \frac12 \sum_{mnef} \eriAsym{mn}{ef} \, t_{in}^{ef} \, t_{mj}^{ab}
		 + \frac12 \sum_{mnef} \eriAsym{mn}{ef} \, t_{jn}^{ef} \, t_{mi}^{ab} \\
		&+ \frac14 \sum_{mnef} \eriAsym{mn}{ef} \, t_{mn}^{ab} \, t_{ij}^{ef}
		 + \frac12 \sum_{mnef} \eriAsym{mn}{ef} \, t_{im}^{ae} \, t_{jn}^{bf} \\
		&- \frac12 \sum_{mnef} \eriAsym{mn}{ef} \, t_{jm}^{ae} \, t_{in}^{bf}
		 - \frac12 \sum_{mnef} \eriAsym{mn}{ef} \, t_{im}^{be} \, t_{jn}^{af} \\
		&+ \frac12 \sum_{mnef} \eriAsym{mn}{ef} \, t_{jm}^{be} \, t_{in}^{af}
\end{aligned}.
	\label{eqn:CCDworking}
\end{equation}
for the \CCD residual $r_{ij}^{ab}$.
They involve multiple contractions over the
antisymmetrised \ERI tensor from remark \vref{rem:ERI},
the amplitudes $t_{ij}^{eb}$ and elements of the Fock matrix $\mat{f}$ in the \SCF orbital basis.
This latter matrix is defined as
\[ \mat{f} = \tp{\mat{C}_F} \mat{F} \mat{C}_F \in \C^{\Norb\times\Norb}.\]
If the canonical \HF ansatz of \eqref{eqn:HFequations} is used, $\mat{f}$ will be diagonal
and equivalent to $\diag(\varepsilon_1, \varepsilon_2, \ldots, \varepsilon_{\Norb})$.
The corresponding \CCD energy expression
\begin{equation}
	E_\text{CCD} = \frac14 \sum_{ijab} \eriAsym{ij}{ab} t_{ij}^{ab}.
	\label{eqn:CCDenergy}
\end{equation}
can be obtained by simplifying \eqref{eqn:CCenergy}.
Since the rank-4 tensor $t_{ij}^{ab}$ occurs in the expression for the
$\op{T}_2$ excitation operator,
this tensor is usually called the $T_2$-amplitudes tensor as well.
Comparing the structure of \eqref{eqn:energyMPtwo} and \eqref{eqn:CCDenergy},
the name of the expression \eqref{eqn:amplitudesMPtwo} in {\MP}2 finally becomes apparent.

For higher-order methods like CCSD, where $\op{T} = \op{T}_1 + \op{T}_2$,
or CCSDT, where $\op{T}_3$ is considered on top,
the expressions for the working equations \eqref{eqn:CCresidual}
are even more involved.
In turn these methods become rather expensive as well,
\eg CCSD scales as $\bigO(\Nbas^6)$ and CCSDT as $\bigO(\Nbas^8)$.
Nevertheless, \CC methods are very popular and widely adopted in quantum chemistry.
Firstly because they converge systematically towards the \FCI energy
as higher and higher excitations are considered in \eqref{eqn:ExcitationCC}
and secondly because \emph{all} \CC methods are size-consistent
--- unlike the truncated CI methods we mentioned above.
One particular approach named CCSD(T),
where the triples excitations are perturbatively added on top of CCSD,
has been named the \emph{gold standard} of chemistry
as it generally yields highly accurate results with an expensive,
but an acceptable scaling of $\bigO(\Nbas^7)$,
where the most costly $\bigO(\Nbas^7)$ step is not iterative.
Recent improvements~\cite{Riplinger2013} within the framework
of pair-natural orbital approaches,
has brought down the apparent scaling of CCSD(T) to linear,
allowing to compute the energies of complete proteins on the level of CCSD(T).

\defineabbr{ADC}{ADC\xspace}{Algebraic-diagrammatic construction}
\subsection{Excited states methods}
\label{sec:ExcitedStates}
In most of our discussion up to this point we have only focused
on obtaining an approximation to the ground state of the electronic Schrödinger
equation.
In some applications of electronic structure theory, however,
electronic excitations play a role.
Examples include the interaction of UV photons or photons of visible light
with the electronic structure in a dye or a solar cell
or more generally any photo-activated chemical reaction.
Whenever this is the case the modelling of multiple electronic states on an equal footing
is required.

For \FCI or truncated CI methods,
this can be achieved without additional modification
by solving the respective full or truncated CI matrix
for more than one eigenpair.
All but the lowest-energy eigenpair describe excited states.
These are not the only excited states methods in existence.
In fact to each of the other methods we have discussed
so far one is able to appoint at least one analogue~\cite{Dreuw2005}.
For example for Hartree-Fock, there is configuration-interaction singles~(CIS)
or time-dependent \HF~(TDHF) and
for coupled-cluster there
are the equation-of-motion and linear-response coupled-cluster theories%
~\cite{Schirmer2010,Sekino1984}.
Last but not least, the algebraic-diagrammatic construction scheme~(\ADC)
for the polarisation propagator at various orders~\cite{Schirmer1982,Trofimov1999}
can be seen as a CI-like scheme on top of a Møller-Plesset ground state.
Its excited states are generally in good agreement with the \MP
description of the ground state.

\section{Density-functional theory}
\label{sec:DFT}
\defineabbr{DFT}{DFT\xspace}{Density-functional theory}
In this section we want to briefly look at a different
approach towards modelling the electronic structure.
Instead of solving for the wave function $\Psi_0$ associated to the
ground state of the electronic Hamiltonian $\Op{H}_{\Nelec}$,
the idea behind \newterm{density-functional theory}
is to solve for the state's electronic density $\rho_0$ instead.

The rationale for this is twofold.
Firstly the density contains all information about the chemical system.
The integral
$\int_{\R^3} \rho(\vec{r}) \D\vec{r}$
evaluates to the number of electrons $\Nelec$
and via Kato's cusp condition~\cite{Kato1951} one may obtain the nuclear
charges $Z_A$ via the derivatives of the electron density at the cusp points.
Secondly the Hohnberg-Kohn theorems~\cite{Hohenberg1964}
as well as the Levy constrained search ansatz~\cite{Levy1979}
provide a unique identification between a particular ground state electron density
and the potential, which generates this density.
Even from a mathematical point of view
solving for the ground state density $\rho_0(\vec{r})$ is thus
sufficient to characterise all properties of the ground state of a system.

The Levy constrained search ansatz~\cite{Levy1979}
provides a conceptionally rather
intuitive route to obtain the ground state density,
namely by a constrained minimisation of the energy
with respect to all possible densities.
The issue with this procedure is that a closed-form expression
for the energy functional $\mathcal{E}(\rho)$,
which returns the energy of a given density,
is not known for any relevant chemical system.
In other words Levy constrained search in the form presented so far
cannot be applied to chemical systems.

Further progress can be made with the Kohn-Sham ansatz~\cite{Kohn1965}, however.
The idea is to consider a fictitious system of
$\Nelec$ non-interacting electrons,
which still has the property that it reproduces the exact
ground state density of the full, interacting system.
In this model system the exact wave function is a single determinant
\begin{align*}
	\Psi &= \Phi_\Theta = \bigwedge_{i=1}^{\Nelec} \psi_i
	&&\text{where}&
	\Theta \equiv \left(\psi_1, \psi_2, \ldots, \psi_{\Nelec}\right)
	&\in \left(H^1(\R^3, \C)\right)^{\Nelec}
\end{align*}
is a tuple of $\Nelec$ single-particle functions.
Ignoring spin the resulting ground state density is
\[
	\rho_\Theta(\vec{r}) = \sum_{i=1}^{\Nelec} \abs{\psi_i(\vec{r})}^2,
\]
which allows to write the Kohn-Sham energy functional as
\begin{equation}
	\begin{aligned}
	\mathcal{E}^\text{KS}(\Theta)
	&= \frac12 \sum_{i=1}^{\Nelec} \int_{\R^3} \norm{\nabla \psi_i}_2^2 \D\vec{r}
	+ \int_{\R^3} \sum_{A=1}^M
		\frac{Z_A \, \rho_\Theta(\vec{r})}{\norm{\vec{r} - \vec{R}_A}_2} \D\vec{r} \\
	&\hspace{20pt}
	+ \frac12 \int_{\R^3}\int_{\R^3}
		\frac{\rho_\Theta(\vec{r}_1) \rho_\Theta(\vec{r}_2)}
			{\norm{\vec{r}_1 - \vec{r}_2}_2} \D\vec{r}_1 \D\vec{r}_2
	+ E_{xc}(\rho_\Theta).
	\end{aligned}
	\label{eqn:KSEnergyFunctional}
\end{equation}
In this expression $E_{xc}$ is the \newterm{exchange-correlation functional},
which depends only on the density function $\rho$.
This term is supposed to describe the non-local
many-body effects not yet contained in the other terms,
which is threefold,
(1) the part of the kinetic energy missed by the non-interacting electrons,
(2) the exchange interaction as well as (3) correlation effects.
The crux with Kohn-Sham \DFT is that its exact functional form is unknown,
such that one has to live with approximations.
Which exchange-correlation functional is sensible for a particular
problem depends very much on the context of the chemical system,
the property one is interested in and is still subject of debate
in quantum-chemical literature.
Notice, however, that if the exact exchange-correlation functional was to be found,
\eqref{eqn:KSEnergyFunctional} would yield the exact ground-state energy.

Following the original Levy constrained search,
we want to find the density corresponding to the minimal energy,
which in the Kohn-Sham picture implies the minimisation
of $\mathcal{E}^\text{KS}(\Theta)$ with respect to the orbitals,
thus the problem
\begin{equation}
	E_0 \leq E_0^\text{KS}
	= \inf \left\{
		\mathcal{E}^\text{KS}(\Theta)
		\, \middle| \,
		\Theta \in \left(H^1(\R^3, \C)\right)^{\Nelec}, \,
		\forall i,j \,
		\braket{\psi_i}{\psi_j}_1 = \delta_{ij}.
	\right\}.
	\label{eqn:KSMO}
\end{equation}
Both the energy functional \eqref{eqn:KSEnergyFunctional}
as well as the Kohn-Sham minimisation problem \eqref{eqn:KSMO} are closely
related to the \HF problem \eqref{eqn:HFMO}.
In fact the only difference is the substitution of the exchange energy term
by the exchange-correlation functional.
As such it should not be very surprising that the methods employed to solve
\eqref{eqn:KSMO} is very similar to \HF as well.
The conditions to obtain the stationary points of \eqref{eqn:KSMO},
the Euler-Lagrange equations,
can be reformulated as
\begin{align}
	\label{eqn:KSequations}
	\Op{F}^\text{KS}_{\Theta^0} \psi_i^0 &= \varepsilon_i \psi_i^0
	&&\text{and}&
	\braket{\psi_i^0}{\psi_j^0} &= \delta_{ij}
\end{align}
where $\Theta^0$ is the minimiser of \eqref{eqn:KSMO} and
\begin{equation}
	\Op{F}_{\Theta^0}^\text{KS} = \Op{T} + \Op{V}_0 + \Op{J}_{\Theta^0} + V_{xc}
	\label{eqn:KohnShamOperator}
\end{equation}
is the Kohn-Sham operator.
Its difference to the Fock operator \eqref{eqn:FockOperator}
is again simply the replacement of the exchange operator $\op{K}_{\Theta^0}$
by the \newterm{exchange-correlation potential} $V_{xc}(\vec{r})$,
which is the derivative of the exchange-correlation energy $E_{xc}(\rho)$
with respect to the density function $\rho$.
Equation \eqref{eqn:KSequations} as well as the minimisation problem
\eqref{eqn:KSMO} can now be discretised
similar to the procedure outlined in section \vref{sec:DiscreteHF}
for Hartree-Fock,
which leads to an iterative self-consistent field procedure,
which is very similar to the Hartree-Fock \SCF outlined in
remark \vref{rem:PropertiesDiscretised}.
Algorithmically both for Kohn-Sham \DFT as well as \HF the same type of problem
needs to be solved, such that all of the numerical procedures
discussed in the next chapters for \HF could be applied to Kohn-Sham \DFT
with only very few changes.

Even though the mathematical problem of the Kohn-Sham \DFT ansatz is related
to \HF, one should mention that \DFT in combination with modern
exchange-correlation functionals~%
\cite{Tsuneda2014,Grimme2011,Perdew2005,Perdew1996,Becke1993,Lee1988}
is much more exact than \HF
for common applications of quantum-chemical calculations
and is thus by far the most widely used method of quantum chemistry.

