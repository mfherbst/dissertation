\chapter{Solving the many-body electronic Schrödinger equation}
\chaptermark{Solving the many-body \abbr{elec}{c} Schrödinger \abbr{eq}{n}}
\label{ch:qchem}
\chapquote{%
The word ``reality'' is also a word,
a word which we must learn to use correctly.}
{Niels Bohr~(1885--1962)} \\
\chapquote{%
I am convinced that despite his slightly positivist language,
Bohr believes as much as we do in the reality of phenomena of
which he speaks, and then the difference between the views
of Bohr and mine is more a difference of language than a
difference of content.}{Vladimir Fock~(1898--1974)}

\noindent
This chapter is concerned with the generalisation
of the one-electron hydrogen-like Schrödinger Hamiltonian \eqref{eqn:OpHydrogen}
\[
	\Op{H}_H = -\frac12 \Delta - \frac{Z}{r},
\]
which we discussed in section \vref{sec:HydrogenAtom},
towards the many-body problems of quantum chemistry.

Even though the spectral properties are very similar to the hydrogen-like case,
solving the associated time-independent
Schrödinger equation \eqref{eqn:TISE} analytically for
any but the most trivial problems is impossible.
Most of this chapter will therefore be devoted to
discussing approximations to the exact \TISE
as well as numerical approaches for solving such approximations
in practice.

\section{Many-body Schrödinger equation}
\defineabbr{BO}{BO\xspace}{Born-Oppenheimer approximation, see section \vref{sec:BO}}
\nomenclature{$\Psi$}{State of a (many body) quantum system,
	typically the \emph{exact} ground state solution to the many-electron,
	electronic TISE, see section \vref{sec:BO}}
\nomenclature{$\Nelec$}{Total number of electrons}
\nomenclature{$\Op{H}_{\Nelec}$}{Electronic Hamiltonian
	for a $\Nelec$-electron system, see section \vref{sec:BO}}
\nomenclature{$\vec{x}$}{Vector in $\R^{3 \Nelec}$,
which typically specifies the positions of all electrons of the system.}
\nomenclature{$\vec{r}$}{Vector in $\R^3$, which specifies the position of an electron in space}

Let us consider a chemical system consisting of $M$ nuclei
and $\Nelec$ electrons.
The nuclei shall have charges $\{Z_A\}_{A = 1, 2, \ldots, M}$
and mass-scaled\footnote{
	If $\vec{\tilde{R}}_A$ is the Cartesian coordinate
	of the $A$-th nucleus with mass $M_A$, then
	the mass-scaled coordinates are given as
	$\vec{R}_A = \sqrt{M_A} \, \vec{\tilde{R}}_A$.
} coordinates $\{\vec{R}_A\}_{A = 1, 2, \ldots, M} \in \R^3$,
whereas the electrons shall sit
at positions $\{\vec{r}_i\}_{i = 1, 2, \ldots, \Nelec} \in \R^3$.
Originating from the correspondence principle to classical mechanics
(see section \vref{sec:IntroQM})
we can construct the many-body Hamiltonian
\begin{equation}
	\Op{H}_\text{MB} = \Op{T}_e + \Op{T}_n + \Op{V}_{nn} + \Op{V}_{ne} + \Op{V}_{ee}
	\label{eqn:ManyBodyHamiltonian}
\end{equation}
with the Coulombic interaction potentials
\begin{align}
	\label{eqn:ManyBodyHamiltonianPotential}
	\Op{V}_{nn} &= \sum_{A=1}^M \sum_{B=A}^M
		\frac{Z_A Z_B}{\norm{\vec{R}_A - \vec{R}_B}_2} &
	\Op{V}_{ee} &= \sum_{i=1}^{\Nelec} \sum_{j=i+1}^{\Nelec}
		\frac{1}{\norm{\vec{r}_i - \vec{r}_j}_2} \\
	\nonumber
	\Op{V}_{ne} &= \sum_{A=1}^M \sum_{i=0}^{\Nelec} \frac{Z_A}{\norm{\vec{R}_A - \vec{r}_i}_2}
\end{align}
and the electronic and nuclear kinetic energy operators
\begin{align}
	\label{eqn:ManyBodyHamiltonianKinetic}
	\Op{T}_e &= \sum_{i=1}^{\Nelec} \Delta_{\vec{r}_i} &
	\Op{T}_n &= \sum_{A=1}^M \Delta_{\vec{R}_A},
\end{align}
where we used the shorthand
\[ \Delta_{\vec{q}} = \sum_{\alpha=1}^3 \frac{\partial^2}{\partial q_\alpha^2} \]
for the Laplace operator with respect to particle coordinates $\vec{q}$.
Defining this operator on the Hilbert space $L^2(\R^L, \C)$
with $L = 3M+3\Nelec$ and domain $D(\Op{H}_\text{MB}) = H^2(\R^L, \C)$
makes this operator self-adjoint~\cite{Kato1951}.
It is furthermore bounded below \cite{Kato1951}.

For ease of writing we define the vectors $\vec{x} \in \R^{3\Nelec}$
and $\vec{X} \in \R^{3M}$ with
\begin{align}
	\rtp{\vec{x}} &\equiv \left(\rtp{\vec{r}_1}, \rtp{\vec{r}_2}, \ldots, \rtp{\vec{r}_{\Nelec}}\right)
	&
	\rtp{\vec{X}} &\equiv \left(\rtp{\vec{R}_1}, \rtp{\vec{R}_2}, \ldots, \rtp{\vec{R}_{M}}\right).
	\label{eqn:DefAllCoords}
\end{align}
They combine either all electronic or all nuclear coordinates, respectively.

For finding the discrete eigenstates of $\Op{H}_\text{MB}$
we need to solve the time-independent Schrödinger equation
\begin{equation}
	\Op{H}_\text{MB} \Psi_\text{MB}(\vec{X}, \vec{x})
	= E_\text{MB} \Psi_\text{MB}(\vec{X}, \vec{x})
	\label{eqn:TISEManyBody}
\end{equation}
where $E_\text{MB} \in \R$ and $\tilde{\Psi} \in H^2(\R^L, \C)$.
Doing this both analytically as well as numerically as such is hardly feasible.
This can be understood by the so-called curse of dimensionality.
Even for comparatively small systems like water,
this equation has a large dimensionality,
which implies that the number of grid points grows very fast.

\section{Born-Oppenheimer approximation}
\label{sec:BO}

Starting from the curse of dimensionality
the Born-Oppenheimer approximation separates the time-independent
Schrödinger equation~\eqref{eqn:TISEManyBody} into
a coupled system of equations.
One describing the motion of the electrons in the field of statically
clamped nuclei
and one describing the motion of the nuclei in the potential generated
by the electrons.

The physical reasoning behind this is that due to the rather large deviation
between the mass of the electrons and the mass of the nuclei%
\footnote{The ratio between the mass of a proton and the mass of an electron is
	approximately 1836.}
leads to electron and nuclear motion to happen on different timescales.
To put this boldly, we regard the nuclei as clamped when treating
the electronic motion.
In other words $\Op{T}_n = 0$ and we yield the \newterm{electronic Schrödinger equation}
for a particular nuclear arrangement $\vec{X}$,
which is considered to be constant.
For this reason one may write a product ansatz
\[ \Psi_\text{MB}(\vec{X}, \vec{x}) = \Psi_\text{elec}(\vec{X}, \vec{x}) \Psi_\text{nuc}(\vec{X}) \]
where the electronic part solves this equation

thus the resulting nuclear equation


From \eqref{eqn:ManyBodyHamiltonian}
and \eqref{eqn:TISEManyBody}
we obtain the 



problem and one for the nuclear p


Time scale separation
Clamped-nuclei approximation
Ignore nuclear derivatives
Vnn constant



Side effect: Potential-energy surface


The Born-Oppenheimer approximation now boldly assumes
that the exact state can be represented as a product
\[ \Psi_\text{MB}(\vec{X}, \vec{x}) = \Psi(\vec{x}) \Psi_\text{nuc}(\vec{X}) \]


% https://www.tcm.phy.cam.ac.uk/%7Epdh1001/thesis/node13.html
% https://en.wikipedia.org/wiki/Born%E2%80%93Huang_approximation
% https://en.wikipedia.org/wiki/Born-Oppenheimer_Approximation

More details \cite{Baer2006}
\cite{Born1927}




Useful:
\[ r_{ij} \equiv \norm{\vec{r}_i - \vec{r}_j}_2 \]
\[ r_{iA} \equiv \norm{\vec{r}_i - \vec{R}_A}_2 \]

Drop $\text{elec}, \vec{X}$ index to call electronic wavefunction just $\Psi$.


% Argue that we can do the mathematical background in real function
% arithmetic only and still use the results for complex input due to linearity.

$\Psi \in H^2(\R^{3d}, \R)$



% show the equation
% rationalise it
% name the terms




\section{Full-CI}
\nomenclature{$\NelecA, \NelecB$}{Number of $\alpha$/$\beta$ electrons.
Note $\NelecA + \NelecB = \Nelec$ and $\NelecA \ge \NelecB$ by convention.}
\nomenclature{$\Phi$}{Slater determinant or many-electron basis functions}
\nomenclature{$\varphi_i$}{One-particle function,
typically solutions to the Hartree-Fock equation, i.e. a Hartree-Fock orbital}
\nomenclature{$\Ibas$}{Index set of the one-particle basis functions.
	Typically a set of multi-indices of quantum numbers.}
\nomenclature{$\Nbas$}{Cardinality of $\Ibas$, i.e. the number of one-particle basis functions.}
\nomenclature{$\chi_\mu$}{$\mu$-th one-particle basis function of the one-particle basis $\{\chi_\mu\}_{\mu \in \Ibas}$}
\nomenclature{$\Iorb$}{Index set of computed SCF orbitals, typically $\{0, \ldots \Norb\}$}
\nomenclature{$\Norb$}{The number of computed SCF orbitals.
Note, that $\Norb \le \Nbas$.}



\todo[inline,caption={}]{
	\begin{itemize}
		\item Idea
		\item Ansatz
	\end{itemize}
}


\section{Single-determinant ansatz}


\defineabbr{HF}{HF\xspace}{Hartree-Fock}
\nomenclature{$\OpFock, \OpFockFull$}{$1$-electron Fock operator (dependent on its eigenfunctions $\{ \varphi_i \}_{i \in \Iorb}$)}
\nomenclature{$\Op{K}$}{One-electron exchange operator}
\nomenclature{$\Op{J}$}{Effective Coulomb operator}
\defineabbr{ERI}{ERI\xspace}{Electron repulsion integrals}
\nomenclature{$(ij|kl)$}{Electron repulsion integrals in chemist's or Mullikan notation, see definition at TODO}
\nomenclature{$\Op{\rho}$}{Density operator}
\nomenclature{$\IoccA$, $\IoccB$}{Index set of occupied SCF orbitals of $\alpha$ or $\beta$ spin, respectively. Typically $\{0, \ldots, \NelecA\}$ and similar for $\IoccB$.}
\nomenclature{$\epsilonconv$}{Convergence tolerance for an iterative process.
	If the error is below this value, the process should be considered converged.}

% Density matrix
% Density operator
% Coefficient matrix

\newterm{Hartree-Fock}
\todo[inline,caption={}]{
	\begin{itemize}
		\item Idea
		\item Discretisation
		\item use Formalism of Cances
		\item Derivation of HF in discretised space
		\item Summary of the numerical task
		\item Summary of the result (SCF orbitals)
		\item Explain properties of the individual terms of Fock operator (locality, ..)
		\item Show what we miss compared to FCI
	\end{itemize}
}

The discretised Hartree-Fock equations in weak form read
\begin{equation}
	\forall i, j \in \Iorb :  \mbra{\varphi_i} \OpFockFull \mket{\varphi_j} = \lambda_j \braket{\varphi_i}{\varphi_j}
	\label{eqn:HF}
\end{equation}

\begin{equation}
	C^\alpha_{\mu i}, C^\beta_{\mu i}
	\label{eqn:SCFalphaBetaCoeffs}
\end{equation}

% Proove the Fock operator (for fixed orbitals) is linear in the function
% it acts upon


\section{Capturing electronic correlation}
\todo[inline,caption={}]{
	\begin{itemize}
		\item What is it and where does HF fail
		\item A couple of sentences about CI
	\end{itemize}
}

\subsection{MP2}
\todo[inline,caption={}]{
	\begin{itemize}
		\item Sketch perturbation theory (maybe leave that out fully)
		\item Use Hylleras functional derivation (fits better)
	\end{itemize}
}

\subsection{Coupled-Cluster}
\todo[inline,caption={}]{
	\begin{itemize}
		\item Explain ansatz
		\item Show diagrams for CCD and working exquations (needed later)
		\item No derivation, brief
	\end{itemize}
}

\section{Density-functional theory}
\todo[inline,caption={}]{
	\begin{itemize}
		\item Brief rationale
		\item Show physical and mathematical relationship to HF
		\item Extremely brief
	\end{itemize}
}

\section{Excited states methods}
\todo[inline,caption={}]{
	\begin{itemize}
		\item Why HF is not good for excited states
		\item Explain ADC ansatz
		\item Rationalise derivation
		\item Rationalise final equations
		\item No derivation, brief
	\end{itemize}
}

