\chapter{Numerical solution of the electronic Schrödinger equation}
\label{ch:qchem}
\chapquote{%
The word ``reality'' is also a word,
a word which we must learn to use correctly.}
{Niels Bohr~(1885--1962)} \\
\chapquote{%
I am convinced that despite his slightly positivist language,
Bohr believes as much as we do in the reality of phenomena of
which he speaks, and then the difference between the views
of Bohr and mine is more a difference of language than a
difference of content}{Vladimir Fock~(1898--1974)}

This chapter reviews the basic concepts of quantum mechanics,
which we require for our treatment.


\todoil{Explain coordinate system choices, convention for $\vec{r}$, $r$, $\uvec{r}$}

\section{Hydrogenic atoms}
\todoil{This should be done already in chapter 1}
\label{sec:HAtom}



note aufbau principle
discuss some properties of the solution

\todoil{Show that generally it makes sense that larger values of $n$ give rise to larger
	energy due to the kinetic energy operator. Maybe just generally take $\mbra{\Psi} \Delta \mket{\Psi}$ or argue it.}

In an analogous manor to section \vref{sec:HAtom}
one can argue for molecules that larger principle quantum numbers
$n$ give rise to an associated Lagrange polynomial of higher degree,
thus more sign changes in the basis function.
This in turn implies a larger curvature and thus a higher kinetic energy.
The potential energy of a basis function, however,
is not affected very much,
since the radial extent is pretty much unchanged by $n$
(exponential decay does not depend on $n$).

\subsection{The Aufbau principle for the periodic table}
\label{sec:PSEAufbau}
Mention the Madelung rule

\section{$N$-body Schrödinger equation}
% Argue that we can do the mathematical background in real function
% arithmetic only and still use the results for complex input due to linearity.

$ \Psi \in H^2(\R^{3d}, \R)$
\nomenclature{$\Psi$}{State of an $N$-body quantum system, typically the \emph{exact} ground state solution to the $N$-electron electronic TISE}
\defineabbr{TISE}{TISE\xspace}{Time-independent Schrödinger equation,
usually only the electronic part}
\nomenclature{$\Phi$}{Slater determinant or $N$-particle basis functions}
\nomenclature{$\varphi_i$}{One-particle function, typically solutions to the Hartree-Fock equation, i.e. a Hartree-Fock orbital}
\nomenclature{$\Op{H}$}{$N$-electron Schrödinger Hamiltonian}
\nomenclature{$\Delta$}{Laplace operator}
\nomenclature{$\Op{T}$}{Kinetic energy operator}
\nomenclature{$\Op{V}$}{Combined potential energy operator (nuclear attraction and Coulombic electron-electron repulsion)}

\nomenclature{$\Ibas$}{Index set of the one-particle basis functions.
	Typically a set of multi-indices of quantum numbers.}
\nomenclature{$\Nbas$}{Cardinality of $\Ibas$, i.e. the number of one-particle basis functions.}
\nomenclature{$\Iorb$}{Index set of computed SCF orbitals, typically $\{0, \ldots \Norb\}$}
\nomenclature{$\Norb$}{The number of computed SCF orbitals.
Note, that $\Norb \le \Nbas$.}
\nomenclature{$\Nelec, \NelecA, \NelecB$}{Number of electrons, number of $\alpha$ electrons, number of $\beta$ electrons. Note $\NelecA + \NelecB = \Nelec$ and $\NelecA \le \NelecB$ by convention.}
\nomenclature{$\chi_\mu$}{$\mu$-th one-particle basis function of the one-particle basis $\{\chi_\mu\}_{\mu \in \Ibas}$}

\nomenclature{$\vec{x}$}{Vector in $\R^{3 \Nelec}$, which typically specifies the positions of all electrons of the system.}
\nomenclature{$\vec{r}$}{Vector in $\R^3$, which specifies the position of a particle in space}



% show the equation
% rationalise it
% name the terms


\todo[inline,caption={}]{
	\begin{itemize}
		\item Define it
		\item TDSE and TISE
		\item Properties
		\item Show how difficult
	\end{itemize}
}

\section{BO separation}
\todo[inline,caption={}]{
	\begin{itemize}
		\item Eckard conditions
		\item Rotational, vibrational, translational, electronic separate
		\item PES and electronic schrödinger
	\end{itemize}

	% follow conical intersection talk of ivona
}

%\section{Second quantisation and Fock spaces}
%\todo[inline,caption={}]{
%	\begin{itemize}
%		\item Structure of the spaces
%		\item Properties
%	\end{itemize}
%}







\section{Full-CI}
\todo[inline,caption={}]{
	\begin{itemize}
		\item Idea
		\item Ansatz
	\end{itemize}
}


\section{Single-determinant ansatz}


\defineabbr{HF}{HF\xspace}{Hartree-Fock}
\nomenclature{$\OpFock, \OpFockFull$}{$1$-electron Fock operator (dependent on its eigenfunctions $\{ \varphi_i \}_{i \in \Iorb}$)}
\nomenclature{$\Op{K}$}{One-electron exchange operator}
\nomenclature{$\Op{J}$}{Effective Coulomb operator}
\defineabbr{ERI}{ERI\xspace}{Electron repulsion integrals}
\nomenclature{$(ij|kl)$}{Electron repulsion integrals in chemist's or Mullikan notation, see definition at TODO}
\nomenclature{$\Op{\rho}$}{Density operator}
\nomenclature{$\IoccA$, $\IoccB$}{Index set of occupied SCF orbitals of $\alpha$ or $\beta$ spin, respectively. Typically $\{0, \ldots, \NelecA\}$ and similar for $\IoccB$.}
\nomenclature{$\epsilonconv$}{Convergence tolerance for an iterative process.
	If the error is below this value, the process should be considered converged.}

% Density matrix
% Density operator
% Coefficient matrix

\newterm{Hartree-Fock}
\todo[inline,caption={}]{
	\begin{itemize}
		\item Idea
		\item Discretisation
		\item use Formalism of Cances
		\item Derivation of HF in discretised space
		\item Summary of the numerical task
		\item Summary of the result (SCF orbitals)
		\item Explain properties of the individual terms of Fock operator (locality, ..)
		\item Show what we miss compared to FCI
	\end{itemize}
}

The discretised Hartree-Fock equations in weak form read
\begin{equation}
	\forall i, j \in \Iorb :  \mbra{\varphi_i} \OpFockFull \mket{\varphi_j} = \lambda_j \braket{\varphi_i}{\varphi_j}
	\label{eqn:HF}
\end{equation}

\begin{equation}
	C^\alpha_{\mu i}, C^\beta_{\mu i}
	\label{eqn:SCFalphaBetaCoeffs}
\end{equation}

% Proove the Fock operator (for fixed orbitals) is linear in the function
% it acts upon


\section{Capturing electronic correlation}
\todo[inline,caption={}]{
	\begin{itemize}
		\item What is it and where does HF fail
		\item A couple of sentences about CI
	\end{itemize}
}

\subsection{MP2}
\todo[inline,caption={}]{
	\begin{itemize}
		\item Sketch perturbation theory (maybe leave that out fully)
		\item Use Hylleras functional derivation (fits better)
	\end{itemize}
}

\subsection{Coupled-Cluster}
\todo[inline,caption={}]{
	\begin{itemize}
		\item Explain ansatz
		\item Show diagrams for CCD and working exquations (needed later)
		\item No derivation, brief
	\end{itemize}
}

\section{Density-functional theory}
\todo[inline,caption={}]{
	\begin{itemize}
		\item Brief rationale
		\item Show physical and mathematical relationship to HF
		\item Extremely brief
	\end{itemize}
}

\section{Excited states methods}
\todo[inline,caption={}]{
	\begin{itemize}
		\item Why HF is not good for excited states
		\item Explain ADC ansatz
		\item Rationalise derivation
		\item Rationalise final equations
		\item No derivation, brief
	\end{itemize}
}

