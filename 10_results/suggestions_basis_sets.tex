\section{Suggestions for Coulomb-Sturmian basis sets}
% TODO Table of optimal values for k per basis set.
%      together with their energy and a comparison to literature and UHF values
%      create the UHF values via a CBL expansion from cc-pvNz calculations with Orca
%      N = 5, 6, 7 maybe. Do this for RHF and closed-shell as well to see the difference
%      This probably should be at the end of the chapter leading over to the comparison
%      with the Gaussians

\include{table_kopt}

See table \vref{tab:Kopt2} and table \vref{tab:Kopt3} % and table \vref{tab:Kopt4}


% Note that molsturm uses the fitted kopt values as a builtin ad hoc estimate
	% Explain procedure for finding the optimal k
	% explain justification for the "out of my hat formula"
	% comapare with various definitions of Z_eff
	% compare with solution to 1-electron hydrogenic equation energy scaling with Z



% Show and discuss the ideal exponent
% Maybe hint at or suggest good values for nlm

% Since the CS are complete in the radial part,
% we can essentially select sensible values for l and m and converge them
% with respect to these quantities thereby determining suggested values for l and m
% for other types of basis functions as well.
% => AM choice should apply to gaussians (and fe and \ldots), too!


Procedure:
\begin{enumerate}
	\item Find a not too small and not too large basis
	\item Optimise k
	\item Compute at optimal k a full-shell basis set
	\item find $\RMSOl$
	\item limit $\lmax$ and $\mmax$ properly
	\item Go to large-enough value for $\nmax$ to reach desired convergence
\end{enumerate}

\subsection{A word about the physics and the chemistry}
Let us briefly recall that Coulomb-Sturmians are the result of
solving equation \eqref{eqn:CS},
which is very much related to the Schrödinger equation for hydrogenic atoms.
Most importantly the angular part of both the Coulomb-Sturmian
as well as the solution to the hydrogenic problem (see section \vref{sec:HydrogenAtom})
is the same, namely a spherical harmonic $Y_l^m$.
Along this lines we would expect that the \CS basis functions
which are most important to describe the electronic structure
of an atom are related to the quantum numbers expected
for the occupied atomic orbitals in this atom by the Aufbau principle
or the Madelung rule.



A very remarkable features of the CS basis is how well it captures all the expectations
from the chemical and phyisical behaviour, even though this has not been taken
into account in the above consideration, all chemical trends of the
periodic table which are known are nicely reproduced and hence
we would guess, that they are very good at capturing chemical effects as well.

\todo[inline,caption={}]{
	\begin{itemize}
		\item Compute Sturmian k-dependence for more than one example
		\item Contrast the results for the periodic table?
	\end{itemize}

	% Explain shape of E-vs-k graph:
	% large k favour valance orbitals and disfavour core-like orbitals
	% kmall k vice versa
	% valance orbitals have smaller contribution to the HF energy
	% Hence steeper slope at small k side
	%
	% Hence more complete basis leads to smaller kopt
	%    - Since more basis functions the large-k contributions are represented
	%      in the extra basis functions with larger n, hence we can choose a smaller k
	%      to also represent the valence orbitals well (This is not true!)
	%    - Explains also why the valance orbitals only become bonding if basis is large enough
	%
	% Compute E-vs-k graph for a few examples.
	% Compute Orben vs k grapk

	% Larger lmax helps nothing fer the first 3 periods

	% Fit approximate formula for atomic number vs k_opt

	% Show convergence graph for eigenvalues (e.g. 311, 411, 511, 611, 711, ... for one example
	%       Go into largeness of the basis for proper description of the virtual orbitals
	% Show convergence graph for eigenvalues for k

	% m tuning not looked into very closely

	% TODO
	% FCI calculation of something to compare
	%       interpret FCI
	% Can one really make the argument that a smaller orbital energy value is a better approximation?

	% Smaller optimal exponent for the energy (physical argument):
	%   - Double counting of e-e interaction leads to electrons avoiding another
	%   - Hence smaller exponent (and higher effective energy, i.e. smaller magnitude)

	% Compare HF energy and HF orbital energies to CBS limit
}

