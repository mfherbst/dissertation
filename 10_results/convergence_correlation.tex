\section{Convergence at correlated level}
% TODO OPTIONAL I do not like the "fraction of correlation energy" plots
%      It should be more sensible to instead show relative errors in the *total*
%      energies.

% TODO
% Plot MP2 correlation energy vs. Nbas for the computations we did on manjula
% converge_correlation. Annotate the points with the actual basis set used.
% Use this plot to justify the choice of basis in the nlm sense for the
% different elements.

In the previous section we took a first look at the convergence
properties of Coulomb-Sturmian-based discretisations at Hartree-Fock level.
In practical quantum-chemical calculations Hartree-Fock is typically
not the final answer,
but only a first step,
such that our discussion of convergence should really not focus on Hartree-Fock alone.
In this spirit the aim of this section is to take our preliminary guidelines
for sensible basis sets at \HF level and describe some adaptions,
which could help to construct sensible \CS basis sets for correlated
quantum-chemical methods.
In line with what we discussed before, we will ignore the dependency
of the \CS basis on the exponent $\kexp$ and keep a sensibly chosen,
fixed exponent value for each atom throughout.
So far we have not carried out many calculations
for investigating the dependency of the correlated ground-state energy
with respect to altering the maximal quantum numbers $\nmax$, $\lmax$ and $\mmax$.
Our results in this section are therefore just exemplary
and should not be taken to be general for Coulomb-Sturmian discretisations
at correlated level.
Moreover we have only considered two Post-HF approaches,
namely {\MP}2 and to a much lesser extent full CI,
such that the behaviour might well deviate in other methods.

\begin{figure}
	\centering
	\includeimage{10_results/Be_fci_cc_vs_nbas}
	\caption[
		Fraction of beryllium correlation energy
		recovered with selected \CS bases
	]{
		Fraction of beryllium correlation energy recovered
		relative to a FCI reference calculation with a $(10, 2, 2)$ \CS basis.
		All calculations employ an exponent of $\kexp = 2.1$.
	}
	\label{fig:EccFciBe}
\end{figure}
A first impression regarding the dependency of the correlation energy
on the \CS basis set provides figure \vref{fig:EccFciBe}.
It shows the fraction of the beryllium atom correlation energy, which is recovered
by selected \CS basis sets and at \FCI and {\MP}2 level,
plotted against the size of the basis.
As the reference, \ie $100\%$ correlation energy,
we take the value obtained in a \FCI calculation employing
the rather large $(10,2,2)$ \CS basis.
When it comes to interpreting this figure one has to be a little careful.
First of all the ground-state energy at \HF level is
not necessarily constant for all of the basis sets employed.
In the depicted cases the changes in $E_\text{HF}$ are,
however, only very little and orders of magnitude
smaller than the changes in correlation energy.
The reason for this is that the selected basis sets only differ
in the maximal angular momentum quantum numbers $\lmax$ and $\mmax$,
whilst the beryllium \HF wave function is already
converged very well in the angular part for $\lmax = 0$.
The second issue with this plot
is that the blue curve somewhat compares apples and pears,
namely a variationally obtained reference correlation energy at \FCI level
with a perturbatively obtained correlation energy using {\MP}2.
Ignoring this fact for a moment,
we find that the \FCI and the {\MP}2 correlation energy curves
follow very similar trends.
Most notable are the two strong increases in the amount of correlation
energy recovered going from $(6,0,0)$ to $(6,1,0)$
and from $(6,2,0)$ to $(6,1,1)$.
Interestingly another increase of $\lmax$,
namely the transition $(6,1,1)$ to $(6,2,1)$
does not have such a pronounced effect.
In line with the arguments presented in the context of \HF
it seems that the angular momentum discretisation
of the {\MP}2 or \FCI ground-state wave functions
are largely converged as soon as $\lmax = \mmax = 1$,
such that further increases of angular momentum have much smaller effects.

Since the \FCI calculations on large basis sets such as $(10,2,2)$ become
extremely costly for larger atoms with more electrons,
we did not perform such calculations except for beryllium.
The only other correlation method,
which is currently available from \molsturm is {\MP}2.
Thus somewhat pragmatically we limited
our investigation of the convergence at correlated level
for the other atoms of the second and third period to {\MP}2 only,
arguing that at least for the case of beryllium we got the same trends.
The results are presented in the tables of appendix
\vref{apx:CSCorrelationConvergence} and graphically
in figures \ref{fig:EccVsNlmEasy} and \vref{fig:EccVsNlmHard}.
These show that the fraction of \emph{total} {\MP}2 energy
which is missed by a particular basis set compared to the most
accurate result we obtained in our calculations for a particular atom.
Again this value is plotted against the size of the basis set.
Similar to the case of beryllium sketched above,
we concentrate on capturing the effect of converging the discretisation
of the angular part of the {\MP}2 wave function
by varying $\lmax$ and $\mmax$.
In figure \ref{fig:EccVsNlmEasy} for the half-filled
and filled valence shells,
a convergence is visible.
For \ce{Li} and \ce{Be},
where $\lmax = \mmax = 0$ converges the angular part of the \HF ground state,
$\lmax = \mmax = 1$ does so pretty much for the {\MP}2 ground state.
For the other atoms shown in figure \ref{fig:EccVsNlmEasy} $\lmax = \mmax = 2$
is at least required.
In other words compared to converging the \HF ground state
we roughly speaking need one extra angular momentum.
For the cases of one and two unpaired electrons,
which are shown in figure \ref{fig:EccVsNlmHard},
the picture is not so conclusive.
Since already the \HF ground state requires $\lmax = 3$
for a decent modelling of the angular part,
this is of course at least required for the {\MP}2 wave function as well.
But figure \ref{fig:EccVsNlmHard} seems to suggest
that $\lmax = 4$ is important as well
since the change from $(6,3,3)$ to $(6,4,4)$ is much
steeper compared to the change from $(6,2,2)$ to $(6,3,3)$ in figure \ref{fig:EccVsNlmEasy}.
Whether even larger angular momentum is required cannot be said
with the currently available results.

\begin{figure}
	\centering
	\includeimage{10_results/Etot_vs_bas_easy}
	\caption[
		Missing fraction of {\MP}2 energy versus \CS basis size
	]{
		Plot of the
		missing fraction of total {\MP}2 energy
		compared to a calculation employing a $(6,3,3)$ \CS basis
		versus the basis size.
		Shown are the atoms of the second and third period
		with a full or half-full valence shell.
	}
	\label{fig:EccVsNlmEasy}
\end{figure}
\begin{figure}
	\centering
	\includeimage{10_results/Etot_vs_bas_hard}
	\caption[
		Missing fraction of {\MP}2 energy versus \CS basis size (plot 2)
	]{
		Plot of the
		missing fraction of total {\MP}2 energy
		compared to a calculation employing a $(6,4,4)$ \CS basis
		versus the basis size.
		Shown are the atoms of the second and third period
		with one or two unpaired electrons.
	}
	\label{fig:EccVsNlmHard}
\end{figure}

Following our discussion above it is probably a little far fetched to assume
that one can properly judge how well
a \CS basis set is able to capture correlation effects
just by looking at {\MP}2 correlation energies.
Nevertheless given how well the trends of {\MP}2 and \FCI agree for beryllium,
it seems likely that at least for the well-behaving cases
with closed or half-filled valence shell
the rule to take one extra angular momentum for the correlated calculation
captures the predominant effects.
On top of the few investigations towards converging the angular part
of the correlated wave function,
no further investigation regarding $\nmax$
and the convergence of the radial part was attempted so far.
