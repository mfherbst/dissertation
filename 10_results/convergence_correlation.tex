\section{Convergence at correlated level}
\todoil{TODO}

% Remove this
\newcommand{\limp}{\ensuremath l_\text{imp}}

Results questionable for cases with one or two unpaired electrons

% TODO
% Plot MP2 correlation energy vs. Nbas for the computations we did on manjula
% converge_correlation. Annotate the points with the actual basis set used.
% Use this plot to justify the choice of basis in the nlm sense for the
% different elements.


Another aspect to note is that hardly ever one is interested
in the exact description of the one-determinantal \HF solution.
Much of the chemistry which takes place happens due to effects originating
from electron-electron correlation.
In this section we will therefore try to hint at the convergence effects
in Full-CI or MP2 calculations with \CS basis sets.

Note, that we have so far not performed conclusive convergence studies
on these results, so especially for the difficult cases,
i.e. where $\limp > 2$,
the results are extremely preliminary.

\begin{figure}
	\centering
	\includeimage{10_results/Be_fci_cc_vs_nbas}
	\caption{Fraction of correlation energy recovered
		relative to a FCI reference calculation with a $(10, 2, 2)$ \CS basis
		and $\kexp = 2.1$ in beryllium.}
	\label{fig:EccFciBe}
\end{figure}
Table of results in appendix \vref{apx:CSCorrelationConvergence}

% TODO Plot MP2 and FCI correlation energy amount

Table \vref{tab:CStruncationEnergies} shows the \HF, MP2 and
Full-CI energies for various \CS basis sets.
As expected from our previous discussion,
the \HF energy does not depend very much on the choice
of $\lmax$ and $\mmax$ as long as both values
are large enough to incorporate the orbitals with $l \leq \limp$.
Since for beryllium $\limp$ is either $1$ or $0$
(for a large SCF convergence threshold $\epsilonconv$),
the HF energy is indifferent
to the values of $\lmax$ and $\mmax$.

For the methods including electron-electron correlation,
i.e. MP2 and Full-CI,
we see that they show largely the same trends
when it comes to the value of the electronic correlation obtained.
We will therefore assume from now onwards
that MP2 is a sensible description
for the correlation effects of \CS basis sets.
This observation should be taken with a grain
of salt
as we already found out,
that Be is one of the nicely-behaving atoms.

One should point out again that for the description
of correlation effects, the effect of increasing $\nmax$
is generally more significant than increasing
$\lmax$ once a certain value of $\lmax$ has been reached
(here about 2).

In appendix \vref{apx:CSCorrelationConvergence} we show
some computational results for the 2nd and 3rd
period of the periodic system of elements.

\todo[inline,caption={}]{
	\begin{itemize}
		\item I would like to say that FCI and MP2 agree for Be
		\item Hence we can use MP2 to characterise how CS capture correlation effects
		\item Kind of does not work except the well-behaving cases as the results jump around
		\item Still some suggestions that we can get away with basis sets of the kind
			$\mmax = \lmax-1$ in some cases.
		\item I feel we should show the results in a nicer manor
			emphasizing the important features,
			but I kind of fail to see trends.
	\end{itemize}
}

\begin{figure}
	\centering
	\includeimage{10_results/Ecc_vs_bas_easy}
	\caption{Fraction of MP2 correlation energy recovered
		relative to the calculation with a $(6,3,3)$ \CS basis for
		atoms with a full or half-full shell in the second and third period.}
	\label{fig:EccVsNlmEasy}
\end{figure}

\begin{figure}
	\centering
	\includeimage{10_results/Ecc_vs_bas_hard}
	\caption{Fraction of MP2 correlation energy recovered
		relative to the calculation with a $(6,4,4)$ \CS basis for
		atoms with one or two unpaired electrons.}
	\label{fig:EccVsNlmHard}
\end{figure}

For the full set of value see appendix \vref{apx:CSCorrelationConvergence}
