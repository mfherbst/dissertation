\section{Convergence at Hartree-Fock level}
\label{sec:CSconvergenceHF}
In chapter~\vref{ch:qchem} we mentioned that Hartree-Fock
is not only a very good approximative method for simulating chemical systems,
but it furthermore forms the reference for many more accurate Post-HF methods.
Because of this as well as its simplicity it is a very good
starting point for our investigation of the convergence
of Coulomb-Sturmian-based discretisations in quantum-chemical calculations.
To reduce the complexity further,
we will not yet consider variations of the \CS exponent $\kexp$ in this as well as
next few sections
and only discuss the effect of changing the maximal quantum numbers
$\nmax$, $\lmax$ and $\mmax$.
The reason for this is twofold.
First of all already our initial discussion
about the relative error and local energies of \CS discretisations
in section \vref{sec:BasisCS}
showed that the effect of varying the maximal quantum numbers
is much more pronounced compared to changing $\kexp$.
Secondly the completeness property of the Coulomb Sturmians
is satisfied regardless of the value of $\kexp$
and thus guarantees that any error resulting from a less ideal
value of $\kexp$ will be corrected with larger basis sets.
In other words $\kexp$ will at most have an effect on the speed of convergence.

The net effect of tuning the maximal quantum numbers $\nmax$, $\lmax$ and $\mmax$
is that one effectively selects which set of radial functions $R_{nl}$
and which set of angular functions $Y_l^m$ is available for modelling the wave function.
The completeness property of the Coulomb Sturmians
implies that both the set of all
radial parts $R_{nl}$ as well as all angular parts $Y_l^m$ are complete bases as well.
The former is furthermore apparent from the connection of the \CS radial equation
to Sturm-Liouville theory~(see section \vref{sec:BasisCS})
and the latter is a well-known property of the spherical harmonics.
Completeness in both these sets implies
that appropriate tuning the maximal quantum numbers $\nmax$, $\lmax$ and $\mmax$
of the discretisation basis
allows to converge the radial part and the angular part of the wave function separately.
In agreement with chemical intuition
$\lmax$ and $\mmax$ control the convergence with respect to the angular part
and $\nmax$ controls the convergence in the radial part,
which might seem a bit odd at first, given that the radial
part is indexed both in $n$ \emph{and} $l$.
The point is, however, that the recurrence relations between the
confluent hypergeometric functions
allow to write every radial part $R_{nl}$ with $l>0$
as a linear combination of the functions $R_{n'0}$ with $n' \leq n$,
such that convergence in the radial part is in fact independent from $\lmax$.

Related to this aspect is the scaling of \CS basis set size
with the maximal quantum numbers $\nmax$, $\lmax$ and $\mmax$.
For example a \CS basis consisting of complete shells with principle
quantum numbers up to and including $\nmax$ consists of
\begin{equation}
	\begin{aligned}
	\Nbas(\nmax) &= \sum_{n=1}^{\nmax} \sum_{l=0}^{n-1} \sum_{m=-l}^l 1
		= \sum_{n=1}^{\nmax} \sum_{l=0}^{n-1} 2l+1 \\
		&= \sum_{n=1}^{\nmax} n^2
		= \frac{(2\nmax+1)(\nmax+1)\nmax}{6}
		\in \bigO(\nmax^3),
	\end{aligned}
	\label{eqn:NbaxFullShellBasis}
\end{equation}
basis functions, \ie scales cubically with $\nmax$.
In contrast the size of a basis set, which is limited both by $\nmax$ as well as
the maximal angular momentum $\lmax$ scales as
\begin{equation}
	\begin{aligned}
	\Nbas(\nmax) &= \sum_{n=1}^{\nmax} \ \sum_{l=0}^{\min(\lmax, n-1)} 2l + 1 \\
	&= \sum_{n=1}^{\nmax} \Big( \min(\lmax+1, n) \Big)^2 \\
	&\le \sum_{n=1}^{\nmax} (\lmax+1)^2 = (\nmax-1) (\lmax+1)^2
	\in \bigO(\nmax \lmax^2).
	\end{aligned}
\end{equation}
In other words if we manage to find a sensible upper bound for $\lmax$,
which captures all of the angular part of the wave function,
we can converge the radial part thereafter by just increasing the basis set size
linearly.
A key aspect of the next few sections will therefore be to find a suitable
upper bound for $\lmax$ for a particular chemical system.
Notice, that the completeness of the radial part of the \CS functions
implies that this upper bound for $\lmax$ is not specific to \CS functions,
but can be applied to \emph{any} basis function type,
which is of the product form radial part times angular part.

\defineabbr{CBS}{CBS\xspace}{complete basis set limit}
To estimate errors and judge the quality of our \CS-based \HF results
we compare to the reference values given in table \vref{tab:HFReference}.
For the closed-shell atoms we use the very accurate numerical \RHF energies
obtained by \citet{Morgon1997}.
For open-shell atoms as well as the other systems
we employ the method of \citet{Jensen2005} to extrapolate
the \UHF complete basis set~(\CBS) limit from
\UHF calculations using the Dunning cc-pV$n$Z family
of \cGTO basis sets.
% TODO OPTIONAL
% See appendix \vref{apx:CbsLimit} for more details on the extrapolation
% procedure which lead to the tabulated results.
\input{table_reference.tex}

%
% --------------------------------------------------------------------------
%
\subsection{Basis sets without limiting angular momentum}
\begin{figure}
	\centering
	\includeimage{10_results/Delta_EHF_vs_shell}
	\caption
	[Plot of the absolute error in the \HF energy versus the size of the \CS basis]
	{
		Plot of the absolute error in the \HF energy versus the number of basis
		functions in a \CS basis containing complete shells
		up to and including principle quantum number $\nmax$.
		For the closed-shell atoms Be and Ne
		the restricted \HF procedure was used,
		whereas for the other systems \UHF was employed.
		The errors were computed against the reference
		values from table \vref{tab:HFReference}.
}
	\label{fig:ErrorHF_vs_shell}
\end{figure}
Without truncating the maximal angular momentum $\lmax$
the \CS basis set effectively consists only of full shells
of principle quantum numbers $n$ ranging from $1$ to $\nmax$.
Since the \CS functions are complete, increasing $\nmax$ is guaranteed
to reduce the error.
Figure \vref{fig:ErrorHF_vs_shell} shows this for the atoms of the second period
by plotting the absolute error in the \HF energy
versus the number of basis functions.
For each calculation of a particular atom the same value of $\kexp$ was used,
which was taken to be close to the optimal exponent of this atom at $(6,5,5)$ level
to exclude any influence on the behaviour originating from a
very unsuitable exponent.
Whilst we notice a clear convergence with increasing basis set size,
it is furthermore visible that the convergence rate drops
for larger values of $\nmax$.
Additionally absolute errors up to the order of $0.1$ are not great
for basis sets with more than $80$ atoms.

The question is now whether all employed
basis functions are actually required
in order to represent the wave function properly.
From a physical point of view,
we would not expect all angular momentum to be equally important
for the description of the electronic ground state of a particular atom.
In beryllium, for example, only the $1s$ and $2s$ atomic orbitals are occupied,
such that we would expect, that only angular momentum up to $l = 0$ is required.
In light of our discussion in the previous section,
we would therefore propose that a basis with $\lmax = 0$ is sufficient
to converge the angular part of the beryllium ground state.
Conversely we would expect all \CS functions with $l > 0$
to contribute only very little to the increase in accuracy
as we go to larger basis sets in figure \vref{fig:ErrorHF_vs_shell}.
To test this hypothesis, let us introduce the
\newterm{root mean square occupied coefficient} per angular momentum $l$,
formally defined as follows.

\defineabbr{RMS}{RMS\xspace}{root mean square}
\begin{defn}
	\label{defn:RMSOl}
	The root mean square~(\RMS) occupied coefficient per angular momentum $l$
	is the quantity
	\begin{equation}
	\RMSOl =
		\sqrt{
		\sum_{(n,l,m) \in \Ibas} \
		\sum_{i \in \IoccA} \frac{1}{\NelecA \ N_{\text{bas}, l}}
			\Big(C^\alpha_{nlm, i} \Big)^2
			+ \sum_{i \in \IoccB} \frac{1}{\NelecB \ N_{\text{bas}, l}}
			\Big( C^\beta_{nlm, i} \Big)^2
		}
		\label{eqn:DefRMSOl}
	\end{equation}
	where $C^\alpha_{\mu i}$ and $C^\beta_{\mu i}$
	are the orbital coefficients of the $\alpha$ and $\beta$ orbitals
	(see \eqref{eqn:HFCoeffMatrix})
	and
	\[
		N_{\text{bas}, l} := \Big| \left\{ (n',l',m') \, \big|\, (n',l',m') \in \Ibas
			\ \text{and} \ l' = l \right\} \Big|
	\]
	is the number of basis functions in the \CS basis which has angular momentum
	quantum number $l$.
\end{defn}

By construction $\RMSOl$ is the \RMS-averaged coefficient for a particular angular
momentum quantum number $l$ in the occupied SCF orbitals.
It therefore provides a measure which angular momentum quantum numbers $l$
are required in the current basis set for describing the ground state properly.
Conversely values of $\RMSOl$ below the convergence threshold $\epsilonconv$
of the \SCF procedure indicates that all \CS basis functions of this angular momentum
value $l$ can be safely removed from the \CS basis set without
influencing the accuracy of the \HF calculation significantly.
In many cases this property of $\RMSOl$ can assist in finding a good
value of $\lmax$ for truncating the orbital angular momentum.

\begin{figure}[p]
	\centering
	\includeimage{10_results/rmso_period2_vs_l}
	\caption
	[Plot $\RMSOl$ vs $l$ for the \HF ground state of the atoms of the second period]
	{
		Plot $\RMSOl$ vs $l$ for the \HF ground state
		of the atoms of the second period
		if a $(6,5,5)$ \CS basis is employed.
		In each case $\kexp$ was taken close to the optimal value.
		For \ce{Be} and \ce{Ne} a \RHF procedure was used,
		for the other cases \UHF.
	}
	\label{fig:RMSOl_period2}
\end{figure}
\begin{figure}[p]
	\centering
	\includeimage{10_results/rmso_period3_vs_l}
	\caption
	[Plot $\RMSOl$ vs $l$ for the \HF ground state of the atoms of the third period]
	{
		Plot $\RMSOl$ vs $l$ for the \HF ground state
		of the atoms of the third period
		if a $(6,5,5)$ \CS basis is employed.
		In each case $\kexp$ was taken close to the optimal value.
		For \ce{Mg} and \ce{Ar} a \RHF procedure was used,
		for the other cases \UHF.
	}
	\label{fig:RMSOl_period3}
\end{figure}

For example let us consider figures \ref{fig:RMSOl_period2}
and \vref{fig:RMSOl_period3},
which show the variation of $\RMSOl$ vs $l$ for the \HF ground state
for the atoms of the second and third period
if a $(6,5,5)$ Coulomb-Sturmian basis is employed.
In the plot roughly two kinds of behaviour can be identified.
The first kind applies to those atoms which are either closed-shell
like \ce{Be}, \ce{Ne}, \ce{Mg} or \ce{Ar}
or which have a half-filled valence sub-shell
like \ce{Li}, \ce{N}, \ce{Na} or \ce{P}.
For these a very pronounced drop in $\RMSOl$ occurs once a particular
angular momentum value $l$ has been reached.
For \ce{Li} and \ce{Be}, where only $s$-functions are occupied in the ground state,
this happens from $l=0$ to $l=1$
and for the other mentioned atoms from $l=1$ to $l=2$,
which in both cases is in perfect agreement with the behaviour expected
from the physical point of view.
For these atoms truncating at $\lmax = 0$ or $\lmax=1$, respectively,
will not introduce a noticeable error as we will see in the next section.
In contrast to this the other atoms
\ce{B}, \ce{C}, \ce{O}, \ce{F}, \ce{Al}, \ce{Si}, \ce{S} and \ce{Cl}
do not quite follow this trend.
Much rather their $\RMSOl$ value decreases only very moderately over the range
of angular momentum quantum numbers.

\begin{figure}
	\centering
	\includeimage{10_results/rms_lf_N}
	\caption[
		Root mean square coefficient value
		per angular momentum for nitrogen
	]
	{
		Root mean square coefficient value per
		basis function angular momentum quantum number $l$
		for selected orbitals of nitrogen.
		The atom is modelled
		in a $(6,5,5)$ \CS basis at \HF level.
	}
	\label{fig:RMSLF_N}
\end{figure}
\begin{figure}
	\centering
	\includeimage{10_results/rms_lf_C}
	\caption[
		Root mean square coefficient value
		per angular momentum for carbon
	]
	{
		Root mean square coefficient value per
		basis function angular momentum quantum number $l$
		for selected orbitals of carbon.
		The atom is modelled
		in a $(6,5,5)$ \CS basis at \HF level.
	}
	\label{fig:RMSLF_C}
\end{figure}
\begin{figure}
	\centering
	\includeimage{10_results/rms_lf_O}
	\caption[
		Root mean square coefficient value
		per angular momentum for oxygen
	]
	{
		Root mean square coefficient value per
		basis function angular momentum quantum number $l$
		for selected orbitals of oxygen.
		The atom is modelled
		in a $(6,5,5)$ \CS basis at \HF level.
	}
	\label{fig:RMSLF_O}
\end{figure}
A good hint to understand this behaviour provide figures
\ref{fig:RMSLF_N}, \vref{fig:RMSLF_C} and \vref{fig:RMSLF_O}.
These show the \RMS-averaged value of those orbital coefficients,
which share the same angular momentum quantum number $l$ in the basis function.
For the modelling of the atoms in each case
a $(6,5,5)$ Coulomb-Sturmian basis with a near-optimal value of $\kexp$ is employed.
Whilst in the case of nitrogen~(figure \ref{fig:RMSLF_N})
the $2s$ function mostly has significant coefficient values
associated to basis functions with $l=0$,
both for carbon as well as nitrogen
the basis functions with $l=2$ and $l=4$ are important as well.
Similar observations can be made for the $2p$ functions,
which for \ce{N} and \ce{C} have significant angular momenta
$l=1,3,5$ and the $3d$ functions which have basis function
components with angular momenta $l=2$ and $l=4$, sometimes $l=0$ as well.
Even though this observed mix-in of other angular momenta
is undoubtedly rather odd at first glance,
it provides an explanation why cutting the basis at $\lmax=2$ is not
sufficient for carbon and oxygen,
even though only $s$-type and $p$-type \SCF orbitals are occupied
in the respective ground states.
Similar mixing of basis functions with angular momenta differing in steps of $2$
are observed for the other atoms,
which are not closed-shell
or have a half-filled valence shell, as well.
Overall this feature therefore
explains the slow decrease of some of the
$\RMSOl$ plots of figure \ref{fig:RMSOl_period2} and \ref{fig:RMSOl_period3}.

% TODO OPTIONAL
%\begin{figure}[p]
%	\centering
%	\includeimage{10_results/rmso_ions_vs_l}
%	\caption
%	[Plot $\RMSOl$ vs $l$ for the \HF ground state of selected ions]
%	{
%		Plot $\RMSOl$ vs $l$ for the \HF ground state
%		of selected ions
%		if a $(7,6,6)$ \CS basis is employed.
%		In each case $\kexp$ was taken close to the optimal value.
%	}
%	\label{fig:RMSOl_ions}
%\end{figure}
%A first hint to understand this behaviour provides figure \vref{fig:RMSOl_ions}, where
%$\RMSOl$ plots for the related closed-shell or half-filled valence shell ions
%\ce{B+}, \ce{C-}, \ce{O+}, \ce{F-}, \ce{Al+}, \ce{Si-}, \ce{S+} and \ce{Cl-}
%are depicted.
%In these plots the expected drop between $l=0$ and $l=1$ or $l=1$ and $l=2$
%can be found as expected by chemical intuition,
%such that the electronic configuration most play a role in 

The remaining question to understand the $\RMSOl$ plots
is why the mixing of angular momenta
only occurs for those atoms,
which do not have a closed-shell or of half-filled valence shell electron configuration.
The answer is in fact not related to the \CS discretisation we employ.
One can 


as the plots depicted in appendix \vref{apx:GaussianUHF}
for a \cGTO discretisation of 




but much rather the \UHF treatment,
which we employ for the open-shell atoms.
See 


as for example the plots depicted in appendix \vref{apx:GaussianUHF}
demonstrate.


A similar problem was first observed by \citet{Cook1981}
in the context of the \UHF wave functions of carbon and fluorine
based on \cGTO discretisations.


that the $s$-type and $p$-type \SCF orbitals
were in fact not only composed of \cGTO basis functions with $l=0$ and $l=1$,
but much rather were mixtures of basis functions with
angular momenta $l$, $l+2$, $l+4$ and so on.
This artefact was later found to be a general issue of \UHF~\cite{Cook1984,McWeeny1985}.

Regarding the physical reasons why 
the $l+2$ behaviour and not for half-filled systems
see \cite{Fukutome1981}
for a detailed discussion of the symmetry breaking in HF

\citet{Fukutome1981} provides a very detailed discussion of the
effects of symmetry-breaking in HF wave functions like for example in UHF.
\label{sec:IssuesUHF}





Such that the \CS discretisation reproduces the exact
behaviour and $\RMSOl$ is a great tool to show what goes on


% coupling only to l+2 and the same m
% so we could shrink the basis by imposing that restriction!
% The states can be stationary or unstable points rather then minimaA
%
% Tho odd-even pattern can be explained by considering the parity
% of the wave function of that particular orbital. Even l have even parity
% and odd l have odd parity. Linear combination of functions of even
% and odd parity breks M_L symmetry (???)


% Sometimes we do not only have this issue, but the SCF also does not converge to
% a minimum but only a saddle point that way, so we need to force convergence using a DIIS.



















Loosely speaking $\RMSOl$ gives us the importance of the basis functions
with angular momentum quantum number $l$ in the \CS basis set.
This idea motivates the next definition

\begin{defn}
	The \newterm{largest important angular momentum quantum number} is the value
	\[ \limp = \sup \{ l \, | \, (n,l,m) \in \Ibas \ \text{and} \ \RMSOl > \epsilonconv \}, \]
	i.e. the largest $l$ for which the coefficients are still larger than the
	desired accuracy of the SCF procedure.
\end{defn}

If we look at \vref{fig:RMSO_period2_l} we notice
that for lithium and beryllium pretty much all angular momentum
except $l = 0$ and $l = 1$ is unimportant.
Similarly for nitrogen and neon.
For the remaining atoms of the second period,
boron, carbon, oxygen and fluorine, however,
even angular momentum $l=5$ is reasonably important and the
decay of $\RMSOl$ is much less pronounced.
In either case this hints that the rate of convergence might
be improved for cases like beryllium
if a restriction is imposed on the principle angular momentum
quantum number $l$ as well.




%
% --------------------------------------------------------------------------
%
\subsection{Basis sets bounded by $\nmax$ and $\lmax$}
% Hint at the general property James mentioned


From the discussion of the previous section
it becomes clear that at least in some cases it makes sense to
limit not only $\nmax$, but on top of that $\lmax$ as well.
For example for beryllium the error
in the Hartree-Fock energy is pretty much unchanged if the
restriction $\lmax = 1$ is imposed onto the full-shell \CS basis sets as well.
This can be seen in figure \vref{fig:ErrorHF_vs_nlm}
where we again show the error in HF energy compared to the reference
as specified in table \vref{tab:HFReference},
but we look at basis sets with a particular $\lmax < \nmax$ now.

\begin{figure}
	\centering
	\includeimage{10_results/ehf_vs_nlm}
	\caption{Plot relative error in $E_\text{HF}$ vs $\Nbas$.
		The reference is mentaioned in \vref{tab:HFReference}.}
	\label{fig:ErrorHF_vs_nlm}
\end{figure}


\begin{figure}
	\centering
	\includeimage{10_results/ehf_vs_nlm_O}
	\caption{Plot relative error in $E_\text{HF}$ vs $\Nbas$}
	\label{fig:ErrorHF_vs_nlm_O}
\end{figure}

As can be seen the convergence is greatly improved
for nitrogen, beryllium and phosphorus
if we limit ourselves to $\lmax = 1$
in accordance with the observation we made in fig.
\vref{fig:RMSO_period2_l}.
For nitrogen two convergence curves are shown,
one with the limit $\lmax = 2$ and one with $\lmax = 1$.
Note that the difference in \HF error in both cases
is really negligible, whilst the rate of convergence is
surely much faster for $\lmax = 1$.

\defineabbr{AM}{AM\xspace}{angular momentum}
This implies that at least for the well-behaving cases,
i.e. where the $\limp \le 2$,
the convergence pretty much only depends on the value
of $n$ and thus a systematic construction of basis
sets which solve the \HF equations up to a certain
desired accuracy should be possible for these cases.

For O where we know that higher angular momentum is important, too,
the effect of truncating the angular momentum can be observed.
For $\lmax = 1$ the convergence initially is almost linear,
but than quickly stagnates and the error does not improve for larger
values of $\nmax$ any more. The same thing happens for
$\lmax = 2$ just at a larger value of $n$.
Generally one should note that in the examples shown
the errors in HF energy are still comparatively large,
namely in the order of Millihartrees even for the largest
basis sets employed for oxygen.

The correlation between $\RMSOl$ and the lowest error which can be
achieved for a given $\lmax$ is interesting to note.
$\text{RMSO}_2$ and $\text{RMSO}_3$ are of a similar size for oxygen
whilst there is a decrease by about 2 orders of magnitude going
from $\text{RMSO}_3$ to $\text{RMSO}_4$.
This also reflected by the fact that we do not gain a much better
result choosing $\lmax = 2$ instead of $\lmax = 1$
the error values stagnate at about the same value.
For $\lmax = 3$ on the other hand linear convergence is obtained
up to a much smaller error value.


The overall basis set recommendation of HF level is thus
Pretty much $(n, 0, 0)$ up to Be, $(n, 1, 1)$ for N and Ne, Na, Mg, P, Ar.
For the rest $(n, 2, 2)$ at least for description up to error less than $1e-4$
$(n, 3, 3)$ if lower error (down to $1e-6$ is desired.


% TODO OPTIONAL
% TODO orbital energy convergence?
%	Maybe make a similar plot to fig. \vref{fig:ErrorHF_vs_nlm}
%	showing the relative errors of the orbital energies compared
%	to reference results in N or O or so.
%	Show how the orbital energies are approximated from above.
%	This would help in understanding the signifficance of the coefficients
%	wrt. correlation methods as well.
%
% TODO Not sure if this way of plotting is ideal.
%\begin{figure}
%	\centering
%	\includeimage{10_results/orben_vs_nlm_N}
%	\caption{Plot relative difference in the orbital energy vs $n$.
%		The relative difference is $1 - \varepsilon_i^{(n)} / \varepsilon_i^{(n+1)}$
%		for N.
%	}
%	\label{fig:OrbenNlmN}
%\end{figure}

% TODO OPTIONAL
%	I get the feeling the bad convergence for oxygen and the other
%	open-shell systems is because we use UHF,
%	which implicitly causes other determinants to be mixed into the HF
%	solution.
%	Some literature suggests that UHF destroys the rotational symmetry.
%	\begin{itemize}
%		\item Try to code a rudimentary ROHF and check weather things are better there
%	\end{itemize}
%
% Spin squared values:
%       should    is
%  Li   0.75       0.7500174679889692
%  Be   0
%  B    0.75       0.7535472111793353
%  C    2          2.00941178981773
%  N    3.75       3.755922043215328
%  O    2          2.006538441200304
%  F    1.75       0.7530439372523485

% TODO optional
% \subsection{$m$-tuning for oxygen and carbon}
% Do the calculation of 8.7 but truncate m=1, m=2
% Does not really seem to do the trick
