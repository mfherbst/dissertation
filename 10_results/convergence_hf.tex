\section{Convergence at Hartree-Fock level}

\todoil{Some introductory words}
\todoil{Concentrate first on nlm convergence since larger effect in this and next
section}


\subsection{Reference results}
\defineabbr{CBS}{CBS\xspace}{complete basis set limit}

In order to estimate errors as well as the quality of the \CS-based \HF
results obtained,
we compare against the reference values given in table \vref{tab:HFReference}.
For the closed-shell atoms we use the very accurate numerical RHF results
from \citet{Morgon1997}.
For the open-shell atoms we take the UHF complete basis set~(\CBS)
limit extrapolated from calculations using the Dunning cc-pV$n$Z family
of Gaussian basis sets and following the method of \citet{Jensen2005}.
% TODO OPTIONAL
% See appendix \vref{apx:CbsLimit} for more details on the extrapolation
% procedure which lead to the tabulated results.
\input{table_reference.tex}

\todoil{full-shell is a bad term, complete-shell is better}
\todoil{This stuff should go into the denoting section}
\subsection{Convergence of \HF for full-shell \CS basis sets}
Let us use the term \newterm{shell} to refer to the set
\[ \{ \chi_{n'l'm'} \, | \, n' = n \} \]
of \CS basis functions with the same principle quantum number
in analogy of the same term used for hydrogenic orbitals.
Naturally this allows us to build up a \CS basis set by considering
all the shells up to and including the $\nmax$-th shell.
The result is a \newterm{full-shell} \CS basis of the type
\[ \{ \chi_{n'l'm'} \, | \, n' \leq \nmax \}. \]

\begin{figure}
	\centering
	\includeimage{10_results/Delta_EHF_vs_shell}
	\caption{
		Plot of the error in the \HF energy vs the number of basis
		functions in the full-shell \CS basis containing
		the specified number of shells.
		As the error we use the absolute error between \CS-based \HF energy
		$E_\text{HF}^\text{CS}$ as well as the reference value
		$E_\text{HF}^\text{ref}$ of table \vref{tab:HFReference}.
		For the closed-shell atoms Be and Ne restricted \HF results are shown,
		whereas for all other open-shell systems unrestricted \HF results
		are plotted.
		All energies are given in Hartree
		and the $y$-axis has a logarithmic scaling.
}
	\label{fig:ErrorHF_vs_shell}
\end{figure}
By the means of the similarity between the \CS basis and the actual
hydrogenic orbitals,
we would expect such a basis to show convergence to the reference \HF result.
In fact this is more or less what is observed:
Figure \vref{fig:ErrorHF_vs_shell} shows the absolute error in the \HF energy
for the second period versus the number of basis functions.
Even though the convergence rate drops visibly
over the selected sequence of basis functions,
we still clearly note the systematic convergence over the progression
of basis functions.

The question is now if and how we could improve this convergence rate.
From a physical point of view,
we would not expect all angular momentum to be equally important
for the description of the electronic structure of a particular atom.
In beryllium for example we would imagine that only $l == 0$ is really required,
since a ground state beryllium atom only has the 1s and the 2s orbitals occupied,
which of course both have only $l == 0$.
To investigate this more quantitatively,
let us introduce the \newterm{root mean square occupied coefficient}
per angular momentum $l$:

\defineabbr{RMS}{RMS\xspace}{root mean square}
\begin{defn}
	\label{defn:RMSOl}
	The root mean square~(\RMS) occupied coefficient per angular momentum $l$
	is the quantity
	\begin{equation}
	\RMSOl =
		\sqrt{
		\sum_{(n,l,m) \in \Ibas} \
		\sum_{i \in \IoccA} \frac{1}{\NelecA \ N_{\text{bas}, l}}
			\Big(C^\alpha_{nlm, i} \Big)^2
			+ \sum_{i \in \IoccB} \frac{1}{\NelecB \ N_{\text{bas}, l}}
			\Big( C^\beta_{nlm, i} \Big)^2
		}
		\label{eqn:DefRMSOl}
	\end{equation}
	where $C^\alpha_{\mu i}$ and $C^\beta_{\mu i}$
	are the orbital coefficients of the $\alpha$ and $\beta$ orbitals
	(see \eqref{eqn:HFCoeffMatrix})
	and
	\[
		N_{\text{bas}, l} := \abs{ \left\{ (n',l',m') \, \big|\, (n',l',m') \in \Ibas
			\ \text{and} \ l' = l \right\} }
	\]
	is the number of basis functions in the \CS basis which has angular momentum
	quantum number $l$.
\end{defn}
Since $\RMSOl$ is the \RMS-averaged coefficient for a particular angular
momentum quantum number $l$ in the occupied SCF orbitals,
it provides a measure for the importance of the \CS basis functions
with quantum number $l$ for the description of this particular system.
Loosely speaking a drop in $\RMSOl$ to values below the
accuracy desired for the calculation indicates that basis functions with
this value of $l$ can be omitted from the basis without a change in \HF energy.

\begin{figure}
	\centering
	\includeimage{10_results/rmso_period2_vs_l}
	\caption{Plot $\RMSOl$ vs $l$ for $nlm = 655$ and period 2}
	\label{fig:RMSOl_period2}
\end{figure}

\begin{figure}
	\centering
	\includeimage{10_results/rmso_period3_vs_l}
	\caption{Plot $\RMSOl$ vs $l$ for $nlm = 655$ and period 3}
	\label{fig:RMSOl_period3}
\end{figure}
Let us now consider the neutral atoms of the second and third period of
the periodic table in a $(6,5,5)$ \CS basis set.
If we plot $\RMSOl$ vs $l$ like in figures \vref{fig:RMSOl_period2}
and \vref{fig:RMSOl_period3}, we observe two types of behaviour.

Those atoms which consist of a fully filled or half-filled valence subshell
, i.e. Li, Be, N, Ne and Na, Mg, P and Ar,
show a very distinct drop in $\RMSOl$ once a certain value
for $l$ is reached.
This suggests that for lithium and beryllium only $l=0$,
i.e. $s$-functions are required for a description of these atoms,
whilst for the others we only need $l \leq 1$, i.e. $s$- and $p$-functions.
This is in perfect agreement to what one would expected
considering the Aufbau principle and the Madelung rule
\todoil{Maybe briefly explain Madelung rule}
for atoms,
since these principles suggest that only $s$ and $p$-orbitals will be
occupied for the second and third period of the periodic table.

On the other hand the atoms which do not consist
of a closed or half-filled subshell,
i.e. B, C, O, F and Al, Si, S, Cl,
only show a rather moderate decrease in $\RMSOl$ over the range of the plot.
% TODO HERE



The first kind applies for closed-shell atoms or atoms
with a half-filled subshell


Half-filled 2p or 3p subshell
or closed-shell

% Look up in Keeler Wothers
Elements which have a closed-shell or half-full-shell 


If we plot $\RMSOl$ vs $l$ for the neutral atoms of the
second and the third period
we observe a clear drop in $\RMSOl$ only for the closed-shell
and the half-closed-shell systems,
that is Li, Be, N, Ne, Na, Mg, P and Ar.
For all other systems the drop in $\RMSOl$ is only comparatively slow.







For this let us consider at first the relationship of the number of basis
functions $\Nbas$ in a \CS basis including all shells up to $\nmax$.
We find
\begin{align}
	\nonumber
	\Nbas(\nmax) &= \sum_{n=1}^{\nmax} \sum_{l=0}^{n-1} \sum_{m=-l}^l 1
		= \sum_{n=1}^{\nmax} \sum_{l=0}^{n-1} 2l+1 \\
		&= \sum_{n=1}^{\nmax} n^2
		= \frac{(2\nmax+1)(\nmax+1)\nmax}{6}
		\in \bigO(\nmax^3),
	\label{eqn:NbaxFullShellBasis}
\end{align}
i.e. a dependency as the third power of $\nmax$.
In contrast a basis, which is not only limited by $\nmax$, but by $\lmax$ as well,
exhibits only a linear dependency of $\Nbas$ wrt. $\nmax$, namely
\begin{align}
	\nonumber
	\Nbas(\nmax) &= \sum_{n=1}^{\nmax} \ \sum_{l=0}^{\min(\lmax, n-1)} 2l + 1 \\
	&= \sum_{n=1}^{\nmax} \Big( \min(\lmax+1, n) \Big)^2 \\
	\nonumber
	&\le \sum_{n=1}^{\nmax} (\lmax+1)^2 = (\nmax-1) (\lmax+1)^2
	\in \bigO(\nmax \lmax^2).
\end{align}
In other words if we manage to find a particular value for $\lmax$,
for which





As we will show now,
for many atoms it is sensible to limit both $\nmax$ as well as $\lmax$
as the \CS basis functions with large values of $l$ are not selected
by the SCF procedure for the occupied MOs.
Leaving them out hence typically does not effect convergence.
What this implies is that for these atoms the convergence rate
will be increased by choosing a basis with $\nmax$ and $\lmax$ restriced.


% The argument is that we want to explain the reduced convergence
% rate by saying that the basis grows as n^3 whilst
% only some of the l values really play a role, i.e.
% if we restrict l as well we could get something growing only linearly in n






If we would leave the terrain of full-shell basis sets
and allow ourselves to limit the maximal angular momentum quantum
number as well by a constant $\lmax$ as introduced above,
then the number of basis functions scales as
\begin{align*}
\sum_{n=1}^{\nmax} \ \sum_{l=0}^{\min(\lmax, n-1)} 2l + 1
	&= \sum_{n=1}^{\nmax} \Big( \min(\lmax+1, n) \Big)^2 \\
	&\le \sum_{n=1}^{\nmax} (\lmax+1)^2 = (\nmax-1) (\lmax+1)^2
	\in \bigO(\nmax \lmax^2)
\end{align*}
in other words only increases linearly with $\nmax$.

It turns out that for some cases like beryllium and nitrogen
such an approach is very much justified.
In order to see this let us introduce the
\newterm{root mean square occupied coefficient}
per angular momentum $l$:

\begin{figure}
	\centering
	\includeimage{10_results/rmso_ions_vs_l}
	\caption{Plot $\RMSOl$ vs $l$ for $nlm = 766$ and the ions}
	\label{fig:RMSOl_ions}
\end{figure}

Loosely speaking $\RMSOl$ gives us the importance of the basis functions
with angular momentum quantum number $l$ in the \CS basis set.
This idea motivates the next definition

\begin{defn}
	The \newterm{largest important angular momentum quantum number} is the value
	\[ \limp = \sup \{ l \, | \, (n,l,m) \in \Ibas \ \text{and} \ \RMSOl > \epsilonconv \}, \]
	i.e. the largest $l$ for which the coefficients are still larger than the
	desired accuracy of the SCF procedure.
\end{defn}

If we look at \vref{fig:RMSO_period2_l} we notice
that for lithium and beryllium pretty much all angular momentum
except $l = 0$ and $l = 1$ is unimportant.
Similarly for nitrogen and neon.
For the remaining atoms of the second period,
boron, carbon, oxygen and fluorine, however,
even angular momentum $l=5$ is reasonably important and the
decay of $\RMSOl$ is much less pronounced.
In either case this hints that the rate of convergence might
be improved for cases like beryllium
if a restriction is imposed on the principle angular momentum
quantum number $l$ as well.

This behaviour of carbon and fluorine is known.
See \citet{Cook1981}
and this is a general problem of UHF~\citet{Cook1984}.
See also \citet{McWeeny1985}.
\citet{Fukutome1981} provides a very detailed discussion of the
effects of symmetry-breaking in HF wave functions like for example in UHF.
\label{sec:IssuesUHF}
% coupling only to l+2 and the same m
% so we could shrink the basis by imposing that restriction!
% The states can be stationary or unstable points rather then minimaA
%
% Tho odd-even pattern can be explained by considering the parity
% of the wave function of that particular orbital. Even l have even parity
% and odd l have odd parity. Linear combination of functions of even
% and odd parity breks M_L symmetry (???)

In fact analoguous calculations using Gaussian basis sets reveal,
that the same effect occurs if UHF is used.
See \vref{apx:GaussianUHF} for equivalent plots to figures \vref{fig:RMSO_period1},
\vref{fig:RMSO_period2},  \vref{fig:RMSLF_N}, \vref{fig:RMSLF_C}, \vref{fig:RMSLF_O}
using Dunning basis sets.

% TODO Show by similar analysis that we also have the problem for Gaussians
% Indeed that is true for gaussians as well!


% Sometimes we do not only have this issue, but the SCF also does not converge to
% a minimum but only a saddle point that way, so we need to force convergence using a DIIS.

\todo[inline,caption={}]{
	\begin{itemize}
		\item Also look at ions \ce{O+}, \ce{F-}, \ce{C-}, \ce{B+} to show that its
			to do with the occupation
	\end{itemize}
}

\begin{figure}
	\centering
	\includeimage{10_results/rms_lf_N}
	\caption{Plot root mean square coefficient vs $l$ and orbital index for N}
	\label{fig:RMSLF_N}
\end{figure}

\begin{figure}
	\centering
	\includeimage{10_results/rms_lf_C}
	\caption{Plot root mean square coefficient vs $l$ and orbital index for C}
	\label{fig:RMSLF_C}
\end{figure}

\begin{figure}
	\centering
	\includeimage{10_results/rms_lf_O}
	\caption{Plot root mean square coefficient vs $l$ and orbital index for O}
	\label{fig:RMSLF_O}
\end{figure}

\todo[inline,caption={}]{
	\begin{itemize}
		\item It seems that the full and half-full shells are ok and the $\RMSOl$
			decays rather early and in agreement with the chemical intution,
			but all others this is not the case
		\item Any clue why?
		\item \emph{Could} be something to do with the restricted vs unrestricted
			issue. See TODO in the next section.
	\end{itemize}
}

\subsection{Convergence of \HF for \CS basis sets with $\nmax$ and $\lmax$}
% Hint at the general property James mentioned


From the discussion of the previous section
it becomes clear that at least in some cases it makes sense to
limit not only $\nmax$, but on top of that $\lmax$ as well.
For example for beryllium the error
in the Hartree-Fock energy is pretty much unchanged if the
restriction $\lmax = 1$ is imposed onto the full-shell \CS basis sets as well.
This can be seen in figure \vref{fig:ErrorHF_vs_nlm}
where we again show the error in HF energy compared to the reference
as specified in table \vref{tab:HFReference},
but we look at basis sets with a particular $\lmax < \nmax$ now.

\begin{figure}
	\centering
	\includeimage{10_results/ehf_vs_nlm}
	\caption{Plot relative error in $E_\text{HF}$ vs $\Nbas$.
		The reference is mentaioned in \vref{tab:HFReference}.}
	\label{fig:ErrorHF_vs_nlm}
\end{figure}

% Plot the same elements or the same basis set kinds with the same color.

\begin{figure}
	\centering
	\includeimage{10_results/ehf_vs_nlm_O}
	\caption{Plot relative error in $E_\text{HF}$ vs $\Nbas$}
	\label{fig:ErrorHF_vs_nlm_O}
\end{figure}

As can be seen the convergence is greatly improved
for nitrogen, beryllium and phosphorus
if we limit ourselves to $\lmax = 1$
in accordance with the observation we made in fig.
\vref{fig:RMSO_period2_l}.
For nitrogen two convergence curves are shown,
one with the limit $\lmax = 2$ and one with $\lmax = 1$.
Note that the difference in \HF error in both cases
is really negligible, whilst the rate of convergence is
surely much faster for $\lmax = 1$.

\defineabbr{AM}{AM\xspace}{angular momentum}
This implies that at least for the well-behaving cases,
i.e. where the $\limp \le 2$,
the convergence pretty much only depends on the value
of $n$ and thus a systematic construction of basis
sets which solve the \HF equations up to a certain
desired accuracy should be possible for these cases.

For O where we know that higher angular momentum is important, too,
the effect of truncating the angular momentum can be observed.
For $\lmax = 1$ the convergence initially is almost linear,
but than quickly stagnates and the error does not improve for larger
values of $\nmax$ any more. The same thing happens for
$\lmax = 2$ just at a larger value of $n$.
Generally one should note that in the examples shown
the errors in HF energy are still comparatively large,
namely in the order of Millihartrees even for the largest
basis sets employed for oxygen.

The correlation between $\RMSOl$ and the lowest error which can be
achieved for a given $\lmax$ is interesting to note.
$\text{RMSO}_2$ and $\text{RMSO}_3$ are of a similar size for oxygen
whilst there is a decrease by about 2 orders of magnitude going
from $\text{RMSO}_3$ to $\text{RMSO}_4$.
This also reflected by the fact that we do not gain a much better
result choosing $\lmax = 2$ instead of $\lmax = 1$
the error values stagnate at about the same value.
For $\lmax = 3$ on the other hand linear convergence is obtained
up to a much smaller error value.

\todoil{Maybe make a similar plot to fig. \vref{fig:ErrorHF_vs_nlm}
showing the relative errors of the orbital energies compared
to reference results in N or O or so.
Show how the orbital energies are approximated from above.
This would help in understanding the signifficance of the coefficients
wrt. correlation methods as well.
}

% TODO orbital energy convergence?
% TODO Not sure if this way of plotting is ideal.
\begin{figure}
	\centering
	\includeimage{10_results/orben_vs_nlm_N}
	\caption{Plot relative difference in the orbital energy vs $n$.
		The relative difference is $1 - \varepsilon_i^{(n)} / \varepsilon_i^{(n+1)}$
		for N.
	}
	\label{fig:OrbenNlmN}
\end{figure}

\todo[inline,caption={}]{
	I get the feeling the bad convergence for oxygen and the other
	open-shell systems is because we use UHF,
	which implicitly causes other determinants to be mixed into the HF
	solution.
	Some literature suggests that UHF destroys the rotational symmetry.
	\begin{itemize}
		\item Try to code a rudimentary ROHF and check weather things are better there
	\end{itemize}
}
% Spin squared values:
%       should    is
%  Li   0.75       0.7500174679889692
%  Be   0
%  B    0.75       0.7535472111793353
%  C    2          2.00941178981773
%  N    3.75       3.755922043215328
%  O    2          2.006538441200304
%  F    1.75       0.7530439372523485

\subsection{$m$-tuning for oxygen and carbon}
Do the calculation of 8.7 but truncate m=1, m=2

Does not really seem to do the trick

\subsection{$n,l,m$ recommendations for HF}
Pretty much $(n, 0, 0)$ up to Be, $(n, 1, 1)$ for N and Ne, Na, Mg, P, Ar.
For the rest $(n, 2, 2)$ at least for description up to error less than $1e-4$
$(n, 3, 3)$ if lower error (down to $1e-6$ is desired.
