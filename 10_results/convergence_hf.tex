\section{Convergence at Hartree-Fock level}
\label{sec:CSconvergenceHF}
In chapter~\vref{ch:qchem} we mentioned that Hartree-Fock
is typically the first step for a quantum-chemical simulation
with many accurate Post-HF methods building on top of the \HF result.
Because of this as well as its simplicity it is a very good
starting point for our investigation of the convergence
of Coulomb-Sturmian-based discretisations in quantum-chemical calculations.
To reduce the complexity further,
we will not yet consider variations of the \CS exponent $\kexp$ in this as well as
next few sections
and only discuss the effect of changing the maximal quantum numbers
$\nmax$, $\lmax$ and --- to a lesser extend --- $\mmax$.
The reason for this is twofold.
First of all already our initial discussion
about the relative error and local energies of \CS discretisations
in section \vref{sec:BasisCS}
showed that the effect of varying the maximal quantum numbers
is much more pronounced compared to changing $\kexp$.
Secondly the completeness property of the Coulomb-Sturmians
is satisfied regardless of the value of $\kexp$
and thus guarantees that any error resulting from a less ideal
value of $\kexp$ will be corrected with larger basis sets.
In other words $\kexp$ will at most have an effect on the speed of convergence.

The net effect of tuning the maximal quantum numbers $\nmax$, $\lmax$ and $\mmax$
is that one effectively selects which set of radial functions $R_{nl}$
and which set of angular functions $Y_l^m$ is available for modelling the wave function.
The completeness property of the Coulomb-Sturmians
implies that both the set of all
radial parts $R_{nl}$ as well as all angular parts $Y_l^m$ are complete bases as well.
The former is furthermore apparent from the connection of the \CS radial equation
to Sturm-Liouville theory~(see section \vref{sec:BasisCS})
and the latter is a well-known property of the spherical harmonics.
Completeness in both these sets implies
that appropriate tuning the maximal quantum numbers $\nmax$, $\lmax$ and $\mmax$
of the discretisation basis
allows to converge the radial part and the angular part of the wave function separately.
In agreement with chemical intuition
$\lmax$ and $\mmax$ control the convergence with respect to the angular part
and $\nmax$ controls the convergence in the radial part,
which might seem a bit odd at first, given that the radial
part is indexed both in $n$ \emph{and} $l$.
The point is, however, that the recurrence relations between the
confluent hypergeometric functions
allow to write every radial part $R_{nl}$ with $l>0$
as a linear combination of the functions $R_{n'0}$ with $n' \leq n$,
such that convergence in the radial part is in fact independent from $\lmax$.

Related to this aspect is the scaling of \CS basis set size
with the maximal quantum numbers $\nmax$, $\lmax$ and $\mmax$.
For example a \CS basis consisting of complete shells with principle
quantum numbers up to and including $\nmax$ consists of
\begin{equation}
	\begin{aligned}
	\Nbas(\nmax) &= \sum_{n=1}^{\nmax} \sum_{l=0}^{n-1} \sum_{m=-l}^l 1
		= \sum_{n=1}^{\nmax} \sum_{l=0}^{n-1} 2l+1 \\
		&= \sum_{n=1}^{\nmax} n^2
		= \frac{(2\nmax+1)(\nmax+1)\nmax}{6}
		\in \bigO(\nmax^3),
	\end{aligned}
	\label{eqn:NbasFullShellBasis}
\end{equation}
basis functions, \ie scales cubically with $\nmax$.
In contrast the size of a basis set, which is limited both by $\nmax$ as well as
the maximal angular momentum $\lmax$ scales as
\begin{equation}
	\begin{aligned}
	\Nbas(\nmax) &= \sum_{n=1}^{\nmax} \ \sum_{l=0}^{\min(\lmax, n-1)} 2l + 1 \\
	&= \sum_{n=1}^{\nmax} \Big( \min(\lmax+1, n) \Big)^2 \\
	&\le \sum_{n=1}^{\nmax} (\lmax+1)^2 = (\nmax-1) (\lmax+1)^2
	\in \bigO(\nmax \lmax^2).
	\end{aligned}
	\label{eqn:NbasAmLimitedBasis}
\end{equation}
In other words if we manage to find a sensible upper bound for $\lmax$,
which captures all of the angular part of the \HF wave function,
we can converge the radial part thereafter by just increasing the basis set size
linearly.
A key aspect of the next few sections will therefore be to find a suitable
upper bound for $\lmax$ for a particular chemical system.
Notice, that the completeness of the radial part of the \CS functions
implies that this upper bound for $\lmax$ is not specific to \CS functions,
but can be applied to \emph{any} basis function type,
which is of the product form radial part times angular part.

To estimate errors and judge the quality of our \CS-based \HF results
we compare to the reference values given in table \vref{tab:HFReference}.
For the closed-shell atoms we use the very accurate numerical \RHF energies
obtained by \citet{Morgon1997}.
For open-shell atoms as well as the other systems
we employ the method of \citet{Jensen2005} to extrapolate
the \UHF complete basis set~(\CBS) limit from
\UHF calculations using the Dunning cc-pV$n$Z family
of \cGTO basis sets.
% TODO OPTIONAL
% See appendix \vref{apx:CbsLimit} for more details on the extrapolation
% procedure which lead to the tabulated results.
\input{table_reference.tex}

%
% --------------------------------------------------------------------------
%
\subsection{Basis sets without limiting angular momentum}
\begin{figure}
	\centering
	\includeimage{10_results/Delta_EHF_vs_shell}
	\caption
	[Plot of the absolute error in the \HF energy versus the size of the \CS basis]
	{
		Plot of the absolute error in the \HF energy versus the number of basis
		functions in a \CS basis containing complete shells
		up to and including principle quantum number $\nmax$.
		For the closed-shell atoms Be and Ne
		the restricted \HF procedure was used,
		whereas for the other systems \UHF was employed.
		The errors were computed against the reference
		values from table \vref{tab:HFReference}.
}
	\label{fig:ErrorHF_vs_shell}
\end{figure}
Without truncating the maximal angular momentum by limiting $\lmax$
the \CS basis set effectively consists only of full shells
of principle quantum numbers $n$ ranging from $1$ to $\nmax$.
Since the \CS functions are complete, increasing $\nmax$ is guaranteed
to reduce the error.
Figure \vref{fig:ErrorHF_vs_shell} shows this for the atoms of the second period
by plotting the absolute error in the \HF energy
versus the number of basis functions.
For each calculation of a particular atom the same value of $\kexp$ was used,
which was taken to be close to the optimal exponent of this atom at $(6,5,5)$ level
to exclude any influence on the behaviour originating from a
very unsuitable exponent.
Whilst we notice a clear convergence with increasing basis set size,
it is furthermore visible that the convergence rate drops
for larger values of $\nmax$.
Additionally absolute errors up to the order of $0.1$ are not great
for basis sets with more than $80$ atoms.

The question is now whether all employed
basis functions are actually required
in order to represent the \HF wave function properly.
From a physical point of view,
we would not expect all angular momentum to be equally important
for the description of the electronic ground state of a particular atom.
In beryllium, for example, only the $1s$ and $2s$ atomic orbitals are occupied,
such that we would expect, that only angular momentum up to $l = 0$ is required.
In light of our discussion in the previous section,
we would therefore propose that a basis with $\lmax = 0$ is sufficient
to converge the angular part of the beryllium ground state.
Conversely we would expect all \CS functions with $l > 0$
to contribute only very little to the increase in accuracy
as we go to larger basis sets in figure \vref{fig:ErrorHF_vs_shell}.
To test this hypothesis, let us introduce the
\newterm{root mean square occupied coefficient} per angular momentum $l$,
formally defined as follows.

\defineabbr{RMS}{RMS\xspace}{Root mean square}
\begin{defn}
	\label{defn:RMSOl}
	The root mean square~(\RMS) occupied coefficient per angular momentum $l$
	is the quantity
	\begin{equation}
	\RMSOl =
		\sqrt{
		\sum_{(n,l,m) \in \Ibas} \
		\sum_{i \in \IoccA} \frac{1}{\NelecA \ N_{\text{bas}, l}}
			\Big(C^\alpha_{nlm, i} \Big)^2
			+ \sum_{i \in \IoccB} \frac{1}{\NelecB \ N_{\text{bas}, l}}
			\Big( C^\beta_{nlm, i} \Big)^2
		}
		\label{eqn:DefRMSOl}
	\end{equation}
	where $C^\alpha_{\mu i}$ and $C^\beta_{\mu i}$
	are the orbital coefficients of the $\alpha$ and $\beta$ orbitals
	(see \eqref{eqn:HFCoeffMatrix})
	and
	\[
		N_{\text{bas}, l} := \Big| \left\{ (n',l',m') \, \big|\, (n',l',m') \in \Ibas
			\ \text{and} \ l' = l \right\} \Big|
	\]
	is the number of basis functions in the \CS basis which has angular momentum
	quantum number $l$.
\end{defn}

By construction $\RMSOl$ is the \RMS-averaged coefficient for a particular angular
momentum quantum number $l$ in the occupied SCF orbitals.
It therefore provides a measure which angular momentum quantum numbers $l$
are required in the current basis set for describing the ground state properly.
Conversely values of $\RMSOl$ below the convergence threshold $\epsilonconv$
of the \SCF procedure indicates that all \CS basis functions of this angular momentum
value $l$ can be safely removed from the \CS basis set without
influencing the accuracy of the \HF calculation significantly.
In many cases this property of $\RMSOl$ can assist in finding a good
value of $\lmax$ for truncating the orbital angular momentum.

\begin{figure}[p]
	\centering
	\includeimage{10_results/rmso_period2_vs_l}
	\caption
	[Plot $\RMSOl$ vs. $l$ for the \HF ground state of the atoms of the 2nd period]
	{
		Plot $\RMSOl$ vs. $l$ for the \HF ground state
		of the atoms of the second period
		if a $(6,5,5)$ \CS basis is employed.
		In each case $\kexp$ was taken close to the optimal value.
		For \ce{Be} and \ce{Ne} a \RHF procedure was used,
		for the other cases \UHF.
	}
	\label{fig:RMSOl_period2}
\end{figure}
\begin{figure}[p]
	\centering
	\includeimage{10_results/rmso_period3_vs_l}
	\caption
	[Plot $\RMSOl$ vs. $l$ for the \HF ground state of the atoms of the 3rd period]
	{
		Plot $\RMSOl$ vs. $l$ for the \HF ground state
		of the atoms of the third period
		if a $(6,5,5)$ \CS basis is employed.
		In each case $\kexp$ was taken close to the optimal value.
		For \ce{Mg} and \ce{Ar} a \RHF procedure was used,
		for the other cases \UHF.
	}
	\label{fig:RMSOl_period3}
\end{figure}

For example let us consider figures \ref{fig:RMSOl_period2}
and \vref{fig:RMSOl_period3},
which show the variation of $\RMSOl$ vs. $l$ for the \HF ground state
for the atoms of the second and third period
if a $(6,5,5)$ Coulomb-Sturmian basis is employed.
In the plot roughly two kinds of behaviour can be identified.
The first kind applies to those atoms which are either closed-shell
like \ce{Be}, \ce{Ne}, \ce{Mg} or \ce{Ar}
or which have a half-filled valence sub-shell
like \ce{Li}, \ce{N}, \ce{Na} or \ce{P}.
For these a very pronounced drop in $\RMSOl$ occurs once a particular
angular momentum value $l$ has been reached.
For \ce{Li} and \ce{Be}, where only $s$-functions are occupied in the ground state,
this happens from $l=0$ to $l=1$
and for the other mentioned atoms from $l=1$ to $l=2$,
which in both cases is in perfect agreement with the behaviour expected
from the physical point of view.
For these atoms truncating at $\lmax = 0$ or $\lmax=1$, respectively,
will not introduce a noticeable error as we will see in the next section.
In contrast to this the other atoms
\ce{B}, \ce{C}, \ce{O}, \ce{F}, \ce{Al}, \ce{Si}, \ce{S} and \ce{Cl}
do not quite follow this trend.
Much rather their $\RMSOl$ value decreases only very moderately over the range
of angular momentum quantum numbers.

\begin{figure}
	\centering
	\includeimage{10_results/rms_lf_N}
	\caption[
		Root mean square coefficient value
		per angular momentum for nitrogen
	]
	{
		Root mean square coefficient value per
		basis function angular momentum quantum number $l$
		for selected orbitals of nitrogen.
		The atom is modelled
		in a $(6,5,5)$ \CS basis using \UHF.
	}
	\label{fig:RMSLF_N}
\end{figure}
\begin{figure}
	\centering
	\includeimage{10_results/rms_lf_C}
	\caption[
		Root mean square coefficient value
		per angular momentum for carbon
	]
	{
		Root mean square coefficient value per
		basis function angular momentum quantum number $l$
		for selected orbitals of carbon.
		The atom is modelled
		in a $(6,5,5)$ \CS basis using \UHF.
	}
	\label{fig:RMSLF_C}
\end{figure}
\begin{figure}
	\centering
	\includeimage{10_results/rms_lf_O}
	\caption[
		Root mean square coefficient value
		per angular momentum for oxygen
	]
	{
		Root mean square coefficient value per
		basis function angular momentum quantum number $l$
		for selected orbitals of oxygen.
		The atom is modelled
		in a $(6,5,5)$ \CS basis using \UHF.
	}
	\label{fig:RMSLF_O}
\end{figure}
A good hint to understand this behaviour provide figures
\ref{fig:RMSLF_N}, \vref{fig:RMSLF_C} and \vref{fig:RMSLF_O}.
These show the \RMS-averaged value of those orbital coefficients,
which share the same angular momentum quantum number $l$ in the corresponding
basis function.
For the modelling of the atoms in each case
a $(6,5,5)$ Coulomb-Sturmian basis with a near-optimal value of $\kexp$ is employed.
Whilst for nitrogen~(figure \ref{fig:RMSLF_N})
the $2s$ function mostly has significant coefficient values
associated to basis functions with $l=0$,
both for carbon~(figure \ref{fig:RMSLF_C}) as well as oxygen~(figure \ref{fig:RMSLF_O})
the basis functions with $l=2$ and $l=4$ are important as well.
Similar observations can be made for the $2p$ functions,
which for \ce{N} and \ce{C} have significant coefficients at angular momenta
$l=1,3,5$ and the $3d$ functions, which require $l=2$ and $l=4$,
sometimes even $l=0$, for a proper description.
This explains why $\RMSOl$ plots for carbon
and oxygen do not show the expected drop from $l=1$ to $l=2$,
since the higher angular momenta play a role
for the occupied $s$-type and $p$-type \SCF orbitals as well.
Equivalent plots to figures \ref{fig:RMSLF_C} and \ref{fig:RMSLF_O}
for the other atoms,
which are not closed-shell
or have a half-filled valence shell,
show very similar features,
which overall explains the slow decrease
in the $\RMSOl$ plots for such atoms.

% TODO OPTIONAL
%\begin{figure}[p]
%	\centering
%	\includeimage{10_results/rmso_ions_vs_l}
%	\caption
%	[Plot $\RMSOl$ vs $l$ for the \HF ground state of selected ions]
%	{
%		Plot $\RMSOl$ vs $l$ for the \HF ground state
%		of selected ions
%		if a $(7,6,6)$ \CS basis is employed.
%		In each case $\kexp$ was taken close to the optimal value.
%	}
%	\label{fig:RMSOl_ions}
%\end{figure}
%A first hint to understand this behaviour provides figure \vref{fig:RMSOl_ions}, where
%$\RMSOl$ plots for the related closed-shell or half-filled valence shell ions
%\ce{B+}, \ce{C-}, \ce{O+}, \ce{F-}, \ce{Al+}, \ce{Si-}, \ce{S+} and \ce{Cl-}
%are depicted.
%In these plots the expected drop between $l=0$ and $l=1$ or $l=1$ and $l=2$
%can be found as expected by chemical intuition,
%such that the electronic configuration most play a role in 

\label{sec:IssuesUHF}
The pending question is now why angular momenta higher than the expected
$l=0$ and $l=1$ are needed for modelling the $s$-like and $p$-like orbitals
in some atoms in the first place.
Since very similar $\RMSOl$ and \RMS orbital coefficient plots
are observed if \cGTO discretisations are used, see appendix \vref{apx:GaussianUHF},
this effect cannot be due to the \CS discretisation we employ.
Much rather it is an artefact of our \UHF treatment of the open-shell systems.
For example \citet{Cook1981} described a similar behaviour
for a \UHF modelling of carbon and fluorine
based on \cGTO discretisations.
He noticed that the $s$-type and $p$-type \SCF orbitals
for both these systems were not only
composed of \cGTO basis functions with $l=0$ and $l=1$,
but much rather were linear combinations of basis functions with
angular momentum quantum numbers in steps of $2$ apart.
So for $s$-like \SCF orbitals $s$, $d$, $g$, \ldots basis functions
were combined in his calculations --- exactly what we observe in figure \ref{fig:RMSLF_C}.
Later it was found that the occurrence of higher angular momenta in
the ground state is a general issue of
\UHF~\cite{Fukutome1981,Cook1984,McWeeny1985}.
\citet{Fukutome1981} provides a very detailed analysis of the
underlying mechanisms including a discussion of the effect of
spin symmetry breaking and \HF instabilities in \UHF and \GUHF.
Roughly speaking for cases with an unevenly occupied electronic configuration,
like a single or two $p$ electrons,
the initial guess for the \UHF Fock operator is
no longer spherically but \emph{axially} symmetric~\cite{McWeeny1985}.
This implies that the obtained \SCF orbitals no longer represent
a spherically symmetric density, but an axially symmetric density instead.
Over the turn of the \UHF iteration
this broken spherical symmetry allows for different
angular momenta to be combined in a single orbital,
such that the resulting \SCF orbitals are no longer of pure
$s$, $p$, $d$, \ldots character.
Let us give a few examples for this.
If a spherically symmetric $s$ orbital is amend it with
a fraction of $d_{z^2}$, then this effectively causes a stretching
of the orbital along the $z$ axis,
which makes it axially symmetric.
Similarly the $p_x$, $p_y$ and $p_z$ orbitals
may be amended with $f_{xz^2}$, $f_{yz^2}$ and $f_{z^3}$
to cause the same stretching along the $z$-axis in each of these.
Even if all $p$ orbitals of this $p$-shell are evenly occupied in the final \HF
ground state, the wave function is then axially symmetric.
If such linear combinations with higher angular momenta
lowers the total \SCF energy
the \UHF procedure for unevenly occupied electron configurations
may well explore these due to the broken spherical symmetry.
In contrast for evenly occupied valence shells,
like the half-filled valence shell atoms \ce{N} and \ce{P},
such a symmetry breaking may not occur,
such that the pure angular momentum character of each of the orbitals is kept
even in an \UHF treatment.

% coupling only to l+2 and the same m
% so we could shrink the basis by imposing that restriction!
% The states can be stationary or unstable points rather then minimaA
%
% Tho odd-even pattern can be explained by considering the parity
% of the wave function of that particular orbital. Even l have even parity
% and odd l have odd parity. Linear combination of functions of even
% and odd parity breks M_L symmetry (???)


% Sometimes we do not only have this issue, but the SCF also does not converge to
% a minimum but only a saddle point that way, so we need to force convergence using a DIIS.

With the aforementioned detailed analysis of the \UHF procedure
we are finally in a position to explain all features of the $\RMSOl$ plots
in figures \ref{fig:RMSOl_period2} and \ref{fig:RMSOl_period3}.
Conversely our discussion shows that $\RMSOl$ is a good diagnostic measure
for understanding which angular momentum quantum numbers
are required for an accurate quantum-chemical modelling.
Since its value for a particular quantum number $l$ indicates
the \RMS-averaged coefficient value
it even provides a quantitative measure for the error,
which is introduced if the range of available angular momentum in a \CS basis set
is truncated to angular momenta belew this value.

% TODO OPTIONAL
%This realisation motivates the next definition
%of a quantity to estimate the largest required angular momentum
%for modelling a given chemical system at target accuracy $\epsilonconv$.
%
%\begin{defn}
%	The \newterm{largest important angular momentum quantum number} is the value
%	\[ \limp = \max \{ l \, | \, (n,l,m) \in \Ibas \ \text{and} \ \RMSOl > \epsilonconv \}, \]
%	\ie the largest quantum number $l$
%	for which the corresponding occupied
%	orbital coefficients are still larger than the
%	desired accuracy of the SCF procedure.
%\end{defn}
%
%As we will see in the next section
%this quantity should only be taken as a guideline for choosing sensible basis sets
%and does not represent a mathematically rigorous error bound at all
%as it sometimes underestimates the amount of angular momentum required.

%
% --------------------------------------------------------------------------
%
\subsection{Basis sets with truncated angular momentum}
From the discussion of the previous section
it becomes clear that at least in some cases it makes sense to
limit not only $\nmax$, but on top of that $\lmax$ as well.
For example for beryllium the clear drop in $\RMSOl$
in figure \ref{fig:RMSOl_period2} suggests that limiting the angular
momentum quantum numbers to $\lmax = 0$ is reasonable.
Similar arguments for \ce{N} and \ce{P} suggest taking $\lmax = 1$ for these atoms.
In other words we would expect the discretisation
of the angular part of the \HF wave function to be already well-converged
for \CS discretisations with $\lmax = 0$ or $\lmax = 1$, respectively.
Consequently we will only need to increase $\nmax$ further and further
in order to converge the remaining radial part of the wave function as well.
Since fixing $\lmax$ reduces the scaling of the basis set size
from cubic in $\nmax$~(see \eqref{eqn:NbasFullShellBasis}) to linear%
~(see \eqref{eqn:NbasAmLimitedBasis}),
we would expect to obtain a much faster convergence rate.

\begin{figure}
	\centering
	\includeimage{10_results/ehf_vs_nlm}
	\caption[
		Relative error in $E_\text{HF}$ versus the basis size
		for selected \CS discretisations
	]{
		Relative error in $E_\text{HF}$ versus the number of basis functions
		for selected \CS basis sets of the form $(\nmax, \lmax, \lmax)$.
		The connected points show basis set progressions
		in which the maximum principle quantum number 
		$\nmax$ is varied in steps of one and the maximum $\lmax$ is fixed.
		The first and last value for $\nmax$ are indicated as small numbers
		next to the plot.
		The same line type is used for all progressions of the same $\lmax$
		and the same colour for all progressions of the same atom.
	}
	\label{fig:ErrorHF_vs_nlm}
\end{figure}
To test our hypothesis figure \vref{fig:ErrorHF_vs_nlm} shows some
example calculations
of beryllium, nitrogen, carbon, oxygen and phosphorus
using progressions of \CS basis sets, where $\lmax$ is limited
to either $0$, $1$ or $2$,
but $\nmax$ is ranged between $4$ and $12$
In each case the relative error of the \HF energy
with respect to the reference values in table \ref{tab:HFReference}
are plotted against the size of the \CS basis
and those error values corresponding to the same atom and the same $\lmax$,
but different $\nmax$, are connected by lines.
We will refer to such a sequence of connected error values
by the term progression in the following.
As usual $\kexp$ is fixed to a sensible value for all calculations
of the same atom
and for beryllium we used \RHF, for the other atoms \UHF.

Even though the basis sets
are now additionally truncated in angular momentum quantum numbers,
the \HF energies for beryllium still converge steadily.
This applies both to the cases $\lmax = 0$ as well as $\lmax = 1$.
Compared to figure \ref{fig:ErrorHF_vs_shell} one notices,
however, a massive improvement in convergence rate.
For the $\lmax = 1$ progressions of nitrogen and phosphorus the same holds true.
Choosing $\lmax = 2$ for nitrogen does not improve the obtained
values very much
in accordance with our the $\RMSOl$ plot~(figure \ref{fig:RMSOl_period2}).
Since the basis now grows faster as $\nmax$ increases,
the convergence rate is now slower, however.
For oxygen and carbon the angular momentum values $l > 2$ are important
for a proper modelling of the ground state as well.
As such it is no surprise that the convergence of the \HF energy
for these two cases stagnates visibly for the $\nmax$-progressions
with $\lmax = 1$ and $\lmax = 2$.
Even though the convergence is initially linear as well,
the curves bend off at some point.
The reason for this is that the truncation of the available
set of angular momentum quantum numbers to at most $\lmax$ causes
an error in the discretisation of the angular part of the \HF wave function.
At some point this error completely dominates,
such that improving the radial part by increasing $\nmax$
does not improve the relative error by much any more.

\begin{figure}
	\centering
	\includeimage{10_results/ehf_vs_nlm_O}
	\caption[
		Relative error in $E_\text{HF}$ versus the number of basis functions
		for oxygen
	]{
		Relative error in $E_\text{HF}$ versus the number of basis functions
		for oxygen.
		The plot shows the oxygen progressions of figure \ref{fig:ErrorHF_vs_nlm}
		amended with further progressions using larger basis sets,
	}
	\label{fig:ErrorHF_vs_nlm_O}
\end{figure}
The results of our investigations on the oxygen atom are summarised in figure
\vref{fig:ErrorHF_vs_nlm_O}, which shows
the $(\nmax, 1,1)$ and $(\nmax, 2,2)$ progressions already depicted above
as well as one using $(\nmax, 3,3)$ \CS basis sets.
The effect of truncating the angular momentum is clearly visible.
The blue curve with $\lmax =1$ is not able to converge to relative errors
below around $7\E{-5}$,
whilst the orange curve with $\lmax = 2$ can take the error
down to $2\E{-5}$.
The green curve with $\lmax = 3$ on the other hand
converges almost linearly over the full range of $\nmax$ considered.
This can be explained if we take another look at the
$\RMSOl$ plot of oxygen in figure \ref{fig:RMSOl_period2}.
Comparing these findings with the $\RMSOl$ plot of oxygen
in figure \ref{fig:RMSOl_period2} indicates some noteworthy agreements.
Whilst $\text{RMSO}_2$ and $\text{RMSO}_3$ are of a similar size,
there is a larger decrease, about 2 orders of magnitude,
going from $\text{RMSO}_3$ to $\text{RMSO}_4$.
In other words selecting $\lmax = 2$ instead of $\lmax = 1$
does not improve the error in the angular part
as much as going to $\lmax = 3$ does.
This is reflected by the fact that for $\lmax = 3$
an almost linear convergence up to $\nmax = 12$ is obtained,
whilst for $\lmax < 3$, the error in the angular part starts to dominate
from around $\nmax = 10$, such that the convergence slows down.

For constructing \CS basis sets, which converge rapidly at \HF level,
a good balance between the error remaining in the discretisation of the angular
part as well as the error in the discretisation of the radial part is required.
As discussed the former aspect is controlled by selecting $\lmax$,
the latter by selecting $\nmax$.
We found in the case of oxygen that the magnitude of the error
in the discretisation of the angular part and the trends in the $\RMSOl$ versus $l$
plots are related.
We believe this finding to be general
in the sense that a significant drop of $\RMSOl$
from $l$ to $l+1$ indicates
that the \CS basis set progression with $\lmax = l$
will allow convergence to a lower error in the \HF energy
than $\lmax = l-1$ is able to.
Once a large enough value for $\lmax$ is chosen to converge the angular part,
the convergence in the radial part,
\ie the convergence in the progression of increasing $\nmax$,
is initially linear.
For large values of $\nmax$ the remaining error in the angular part
will start to dominate and yet a larger $\lmax$
has to be chosen to make further progress.

From the examples considered in this section we would expect to require
around $\nmax = 10$ to reach a target accuracy of 5 digits,
which equals a relative error of below $10^{-5}$.
For \ce{Li} and \ce{Be}, where $\lmax = 0$ is sufficient this equals
a $(10,0,0)$ basis consisting of only 10 \CS basis functions.
For \ce{N}, \ce{Ne}, \ce{Na}, \ce{Mg}, \ce{P} and \ce{Ar}
a $(10,1,1)$ basis would be required, which has 37 basis functions.
For the difficult cases like \ce{O}, \ce{C} and most other atoms
with a single or 2 unpaired electrions
at least $\lmax = 3$ is required.
A $(10,3,3)$ \CS basis has the enormous number of 126 basis functions,
which still only reaches 5 digits of accuracy in the \HF energy.

% TODO OPTIONAL
% TODO orbital energy convergence?
%	Maybe make a similar plot to fig. \vref{fig:ErrorHF_vs_nlm}
%	showing the relative errors of the orbital energies compared
%	to reference results in N or O or so.
%	Show how the orbital energies are approximated from above.
%	This would help in understanding the signifficance of the coefficients
%	wrt. correlation methods as well.
%
% TO DO Not sure if this way of plotting is ideal.
%\begin{figure}
%	\centering
%	\includeimage{10_results/orben_vs_nlm_N}
%	\caption{Plot relative difference in the orbital energy vs $n$.
%		The relative difference is $1 - \varepsilon_i^{(n)} / \varepsilon_i^{(n+1)}$
%		for N.
%	}
%	\label{fig:OrbenNlmN}
%\end{figure}

% TODO OPTIONAL
%	I get the feeling the bad convergence for oxygen and the other
%	open-shell systems is because we use UHF,
%	which implicitly causes other determinants to be mixed into the HF
%	solution.
%	Some literature suggests that UHF destroys the rotational symmetry.
%	\begin{itemize}
%		\item Try to code a rudimentary ROHF and check weather things are better there
%	\end{itemize}
%
% Spin squared values:
%       should    is
%  Li   0.75       0.7500174679889692
%  Be   0
%  B    0.75       0.7535472111793353
%  C    2          2.00941178981773
%  N    3.75       3.755922043215328
%  O    2          2.006538441200304
%  F    1.75       0.7530439372523485

% TODO optional
% \subsection{$m$-tuning for oxygen and carbon}
% Do the calculation of 8.7 but truncate m=1, m=2
% Does not really seem to do the trick
