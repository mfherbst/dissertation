\section{Denoting Coulomb-Sturmian basis sets}
In section \vref{sec:BasisCS}
we denoted a Coulomb-Sturmian basis function as the product
\begin{equation}
	\varphi_{nlm}(\vec{r}) = R_{nl}(r) Y_l^m(\uvec{r})
\end{equation}
of radial part
\begin{equation}
	\label{eqn:RadialPartCsRepeated}
	R_{nl}(r) = N_{nl} (2\kexp r)^l e^{-\kexp r}
	\;_1F_1\left(l+1-n \middle| 2l+2 \middle|2\kexp r\right)
\end{equation}
and spherical harmonic, compare equation \eqref{eqn:CSproduct}.
It is uniquely defined by specifying both the \CS exponent $\kexp$
as well as the quantum number triple $(n,l,m)$.
Since all basis functions share the same exponent $\kexp$
a truncated \CS basis is thus uniquely defined by specifying
the common exponent $\kexp$ as well as the set of all $(n, l, m)$
triples of all basis functions $\varphi^\text{CS}_{nlm}$.

Theoretically any selection of triples $(n, l, m)$ can be used to form a \CS basis.
From the similarity of the \CS functions to the hydrogen-like orbital functions
as well as the shape of the orbitals of other atoms
one would, however, expect Coulomb-Sturmians with smaller values of $n$
to be the most important.
In this work we have therefore restricted ourselves to \CS basis sets
of the form
\[ \big\{ (n, l, m) \, \big| \, n \le \nmax, l \le \lmax, -\mmax \le m \le \mmax \big\}, \]
\ie where all three quantum numbers are bound from above.
We will sometimes refer to such a \CS basis set by the triple
$(\nmax, \lmax, \mmax)$ of the three maximal quantum numbers itself.
In other words a $(3,2,2)$-basis set shall denote a basis set with
$\nmax = 3$, $\lmax = 2$ and $\mmax = 2$.
One should mention that a restriction to basis sets of this form
is entirely arbitrary
and mainly done for the sake of reducing the search space at hand
for an initial investigation.

In existing literature about Coulomb-Sturmians the existing terminology
to denote atomic orbitals as well as sets of atomic orbitals is often carried
forward to the \CS context as well.
For examples the spectroscopic terms $1s, 2s, 2p_{-1}$ and so on
are often used to denote the Coulomb-Sturmian functions
$\varphi_{100}$, $\varphi_{200}$, $\varphi_{2,1,-1}$.
Additionally we will use the term \newterm{shell} to refer to the set
\[ \{ \varphi_{n'l'm'} \, | \, n' = n \} \]
of \CS basis functions with the same principle quantum number $n$.
Similarly we call a \CS basis set a \newterm{complete-shell} basis set
if it has the form
\[ \{ \varphi_{n'l'm'} \, | \, n' \leq \nmax \}, \]
\ie contains all \CS functions with principle quantum number
below a certain maximum.
