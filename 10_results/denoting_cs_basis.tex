\section{Denoting Coulomb-Sturmian basis sets}


For the following discussion let us briefly recap
that in general a Coulomb-Sturmian basis function
\begin{equation}
\chi_\mu(\vec{r}) = \mathcal{N}_\mu \,
	(2 \kexp \, r)^l \,
	L^{2l+1}_{n-l-1}(2 \kexp \, r) \,
	\exp(- \kexp \, r) \,
	Y_l^m(\theta, \varphi)
	\label{eqn:FormCSBasisFunction}
\end{equation}
with $L^{2l+1}_{n-l-1}$ being an associated Legendre polynomial,
$Y_l^m$ being a spherical harmonic and $\mathcal{N}_\mu$ being a normalisation
constant,
can be uniquely described by its quantum numbers $\mu \equiv (n,l,m)$
as well as the its Coulomb-Sturmian exponent $\kexp$ (\cf section \vref{sec:BasisCS}).
Since the exponent $\kexp$ is shared amongst all Coulomb-Sturmian in a basis,
we can define a particular \CS basis
by specifying the common exponent $\kexp$
and the set of quantum number triples $(n, l, m)$ of all \CS basis functions
present in the basis set.

Theoretically any selection of quantum number triples for forming a basis
is possible.
In this work, however, we have restricted ourselves to basis sets of the form
\[ \big\{ (n, l, m) \, \big| \, n \le \nmax, l \le \lmax, -\mmax \le m \le \mmax \big\} \]
i.e. where all three quantum numbers are bound from above.
We will sometimes refer to such a \CS basis set by the triple
$(\nmax, \lmax, \mmax)$ of the three maximal quantum numbers itself.
In other words a $(3,2,2)$-basis set shall denote a basis set with
$\nmax = 3$, $\lmax = 2$ and $\mmax = 2$.

The selection to basis sets of this particular form is entirely arbitrary
and mainly done for the sake of reducing the search space at hand
and to get some initial insight into the properties \CS basis sets.
