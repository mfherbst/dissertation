\chapter{Coulomb-Sturmian-based quantum chemistry}
\label{ch:CSQChem}
\chapquote{%
The real problem is that programmers have spent far too much
time worrying about efficiency in the wrong places and at the
wrong times; premature optimization is the root of all evil
(or at least most of it) in programming.}
{Donald Knuth~(1928--present)}
\nomenclature{$\kopt$}{Optimal Coulomb-Sturmian exponent $\kexp$,
	which yields the best description of the system
	given the current Coulomb-Sturmian basis and
	the selected quantum-chemical method.}
\nomenclature{$\RMSOl$}{
	The root mean square occupied coefficient per angular momentum $l$,
	see definition \vref{defn:RMSOl}.
}

\noindent
In this chapter we will discuss preliminary computational results
obtained from quantum-chemical calculations using Coulomb-Sturmian
basis functions inside the \linebreak \molsturm framework.
The focus will be on discussing and understanding the convergence
properties of \CS-based calculations on atoms of
the second and the third period of the periodic table,
mostly at Hartree-Fock level.
Some exemplary full CI and {\MP}2 calculations have been performed
to get an idea how the picture changes
if correlation effects are taken into account as well.

Based on these results we will discuss some
preliminary guidelines for selecting \CS basis set parameters,
like the angular momentum restrictions or the Sturmian exponent $\kexp$,
where the overall aim is to yield rapid convergence
of the ground state energies at \HF or correlated level.
Finally we present the first results of a \CS-based
excited states calculation at {\ADC}(2) level.

\section{Denoting Coulomb-Sturmian basis sets}
\label{sec:DenotingCSbasis}
In section \vref{sec:BasisCS}
we denoted a Coulomb-Sturmian basis function as the product
\begin{equation}
	\varphi_{nlm}(\vec{r}) = R_{nl}(r) Y_l^m(\uvec{r})
\end{equation}
of radial part
\begin{equation}
	\label{eqn:RadialPartCsRepeated}
	R_{nl}(r) = N_{nl} \, (2\kexp r)^l e^{-\kexp r}
	\;_1F_1\left(l+1-n \middle| 2l+2 \middle|2\kexp r\right)
\end{equation}
and spherical harmonic, compare equation \eqref{eqn:CSproduct}.
It is uniquely defined by specifying both the \CS exponent $\kexp$
as well as the quantum number triple $(n,l,m) \in \Itpf$,
where $\Itpf$ is the index set defined in \eqref{eqn:AllCSTriples}.
Since all basis functions share the same exponent $\kexp$
a truncated \CS basis is thus uniquely defined by specifying
the common exponent $\kexp$ as well as the set of all $(n, l, m)$
triples of all basis functions $\varphi^\text{CS}_{nlm}$.

Theoretically any selection of triples $(n, l, m)$ can be used to form a \CS basis.
From the similarity of the \CS functions to the hydrogen-like orbital functions
as well as the shape of the orbitals of other atoms
one would, however, expect Coulomb-Sturmians with smaller values of $n$
to be the most important.
In this work we have therefore restricted ourselves to \CS basis sets
of the form
\[ \left\{ \varphi_{nlm} \, \Big| \,  (n, l, m) \in \Itpf, \   n \le \nmax,
	\  l \le \lmax,
	\  -\mmax \le m \le \mmax \right\},
\]
\ie where all three quantum numbers are bound from above.
We will sometimes refer to such a \CS basis set by the triple
$(\nmax, \lmax, \mmax)$ of the three maximal quantum numbers itself.
In other words a $(3,2,2)$-basis set shall denote a basis set with
$\nmax = 3$, $\lmax = 2$ and $\mmax = 2$.
One should mention that a restriction to basis sets of this form
is entirely arbitrary
and mainly done for the sake of reducing the search space at hand
for an initial investigation.

In existing literature about Coulomb-Sturmians the existing terminology
to denote atomic orbitals as well as sets of atomic orbitals is often carried
forward to the \CS context as well.
For examples the spectroscopic terms $1s, 2s, 2p_{-1}$ and so on
are often used to denote the Coulomb-Sturmian functions
$\varphi_{100}$, $\varphi_{200}$, $\varphi_{2,1,-1}$.
%Additionally we will use the term \newterm{shell} to refer to the set
%\[ \{ \varphi_{n'l'm'} \, | \, n' = n \} \]
%of \CS basis functions with the same principle quantum number $n$.
%Similarly we call a \CS basis set a \newterm{complete-shell} basis set
%if it has the form
%\[ \{ \varphi_{n'l'm'} \, | \, n' \leq \nmax \}, \]
%\ie contains all \CS functions with principle quantum number
%below a certain maximum.

\section{Convergence at Hartree-Fock level}
\label{sec:CSconvergenceHF}
In chapter~\vref{ch:qchem} we mentioned that Hartree-Fock
is typically the first step for a quantum-chemical simulation
with many accurate Post-HF methods building on top of the \HF result.
Because of this as well as its simplicity it is a very good
starting point for our investigation of the convergence
of Coulomb-Sturmian-based discretisations in quantum-chemical calculations.
To reduce the complexity further,
we will not yet consider variations of the \CS exponent $\kexp$ in this as well as
next few sections
and only discuss the effect of changing the maximal quantum numbers
$\nmax$, $\lmax$ and --- to a lesser extend --- $\mmax$.
The reason for this is twofold.
First of all already our initial discussion
about the relative error and local energies of \CS discretisations
in section \vref{sec:BasisCS}
showed that the effect of varying the maximal quantum numbers
is much more pronounced compared to changing $\kexp$.
Secondly the completeness property of the Coulomb-Sturmians
is satisfied regardless of the value of $\kexp$
and thus guarantees that any error resulting from a less ideal
value of $\kexp$ will be corrected with larger basis sets.
In other words $\kexp$ will at most have an effect on the speed of convergence.

The net effect of tuning the maximal quantum numbers $\nmax$, $\lmax$ and $\mmax$
is that one effectively selects which set of radial functions $R_{nl}$
and which set of angular functions $Y_l^m$ is available for modelling the wave function.
The completeness property of the Coulomb-Sturmians
implies that both the set of all
radial parts $R_{nl}$ as well as all angular parts $Y_l^m$ are complete bases as well.
The former is furthermore apparent from the connection of the \CS radial equation
to Sturm-Liouville theory~(see section \vref{sec:BasisCS})
and the latter is a well-known property of the spherical harmonics.
Completeness in both these sets implies
that appropriate tuning the maximal quantum numbers $\nmax$, $\lmax$ and $\mmax$
of the discretisation basis
allows to converge the radial part and the angular part of the wave function separately.
In agreement with chemical intuition
$\lmax$ and $\mmax$ control the convergence with respect to the angular part
and $\nmax$ controls the convergence in the radial part,
which might seem a bit odd at first, given that the radial
part is indexed both in $n$ \emph{and} $l$.
The point is, however, that the recurrence relations between the
confluent hypergeometric functions
allow to write every radial part $R_{nl}$ with $l>0$
as a linear combination of the functions $R_{n'0}$ with $n' \leq n$,
such that convergence in the radial part is in fact independent from $\lmax$.

Related to this aspect is the scaling of \CS basis set size
with the maximal quantum numbers $\nmax$, $\lmax$ and $\mmax$.
For example a \CS basis consisting of complete shells with principle
quantum numbers up to and including $\nmax$ consists of
\begin{equation}
	\begin{aligned}
	\Nbas(\nmax) &= \sum_{n=1}^{\nmax} \sum_{l=0}^{n-1} \sum_{m=-l}^l 1
		= \sum_{n=1}^{\nmax} \sum_{l=0}^{n-1} 2l+1 \\
		&= \sum_{n=1}^{\nmax} n^2
		= \frac{(2\nmax+1)(\nmax+1)\nmax}{6}
		\in \bigO(\nmax^3),
	\end{aligned}
	\label{eqn:NbasFullShellBasis}
\end{equation}
basis functions, \ie scales cubically with $\nmax$.
In contrast the size of a basis set, which is limited both by $\nmax$ as well as
the maximal angular momentum $\lmax$ scales as
\begin{equation}
	\begin{aligned}
	\Nbas(\nmax) &= \sum_{n=1}^{\nmax} \ \sum_{l=0}^{\min(\lmax, n-1)} 2l + 1 \\
	&= \sum_{n=1}^{\nmax} \Big( \min(\lmax+1, n) \Big)^2 \\
	&\le \sum_{n=1}^{\nmax} (\lmax+1)^2 = (\nmax-1) (\lmax+1)^2
	\in \bigO(\nmax \lmax^2).
	\end{aligned}
	\label{eqn:NbasAmLimitedBasis}
\end{equation}
In other words if we manage to find a sensible upper bound for $\lmax$,
which captures all of the angular part of the \HF wave function,
we can converge the radial part thereafter by just increasing the basis set size
linearly.
A key aspect of the next few sections will therefore be to find a suitable
upper bound for $\lmax$ for a particular chemical system.
Notice, that the completeness of the radial part of the \CS functions
implies that this upper bound for $\lmax$ is not specific to \CS functions,
but can be applied to \emph{any} basis function type,
which is of the product form radial part times angular part.

To estimate errors and judge the quality of our \CS-based \HF results
we compare to the reference values given in table \vref{tab:HFReference}.
For the closed-shell atoms we use the very accurate numerical \RHF energies
obtained by \citet{Morgon1997}.
For open-shell atoms as well as the other systems
we employ the method of \citet{Jensen2005} to extrapolate
the \UHF complete basis set~(\CBS) limit from
\UHF calculations using the Dunning cc-pV$n$Z family
of \cGTO basis sets.
% TODO OPTIONAL
% See appendix \vref{apx:CbsLimit} for more details on the extrapolation
% procedure which lead to the tabulated results.
\input{table_reference.tex}

%
% --------------------------------------------------------------------------
%
\subsection{Basis sets without limiting angular momentum}
\begin{figure}
	\centering
	\includeimage{10_results/Delta_EHF_vs_shell}
	\caption
	[Plot of the absolute error in the \HF energy versus the size of the \CS basis]
	{
		Plot of the absolute error in the \HF energy versus the number of basis
		functions in a \CS basis containing complete shells
		up to and including principle quantum number $\nmax$.
		For the closed-shell atoms Be and Ne
		the restricted \HF procedure was used,
		whereas for the other systems \UHF was employed.
		The errors were computed against the reference
		values from table \vref{tab:HFReference}.
}
	\label{fig:ErrorHF_vs_shell}
\end{figure}
Without truncating the maximal angular momentum by limiting $\lmax$
the \CS basis set effectively consists only of full shells
of principle quantum numbers $n$ ranging from $1$ to $\nmax$.
Since the \CS functions are complete, increasing $\nmax$ is guaranteed
to reduce the error.
Figure \vref{fig:ErrorHF_vs_shell} shows this for the atoms of the second period
by plotting the absolute error in the \HF energy
versus the number of basis functions.
For each calculation of a particular atom the same value of $\kexp$ was used,
which was taken to be close to the optimal exponent of this atom at $(6,5,5)$ level
to exclude any influence on the behaviour originating from a
very unsuitable exponent.
Whilst we notice a clear convergence with increasing basis set size,
it is furthermore visible that the convergence rate drops
for larger values of $\nmax$.
Additionally absolute errors up to the order of $0.1$ are not great
for basis sets with more than $80$ atoms.

The question is now whether all employed
basis functions are actually required
in order to represent the \HF wave function properly.
From a physical point of view,
we would not expect all angular momentum to be equally important
for the description of the electronic ground state of a particular atom.
In beryllium, for example, only the $1s$ and $2s$ atomic orbitals are occupied,
such that we would expect, that only angular momentum up to $l = 0$ is required.
In light of our discussion in the previous section,
we would therefore propose that a basis with $\lmax = 0$ is sufficient
to converge the angular part of the beryllium ground state.
Conversely we would expect all \CS functions with $l > 0$
to contribute only very little to the increase in accuracy
as we go to larger basis sets in figure \vref{fig:ErrorHF_vs_shell}.
To test this hypothesis, let us introduce the
\newterm{root mean square occupied coefficient} per angular momentum $l$,
formally defined as follows.

\defineabbr{RMS}{RMS\xspace}{root mean square}
\begin{defn}
	\label{defn:RMSOl}
	The root mean square~(\RMS) occupied coefficient per angular momentum $l$
	is the quantity
	\begin{equation}
	\RMSOl =
		\sqrt{
		\sum_{(n,l,m) \in \Ibas} \
		\sum_{i \in \IoccA} \frac{1}{\NelecA \ N_{\text{bas}, l}}
			\Big(C^\alpha_{nlm, i} \Big)^2
			+ \sum_{i \in \IoccB} \frac{1}{\NelecB \ N_{\text{bas}, l}}
			\Big( C^\beta_{nlm, i} \Big)^2
		}
		\label{eqn:DefRMSOl}
	\end{equation}
	where $C^\alpha_{\mu i}$ and $C^\beta_{\mu i}$
	are the orbital coefficients of the $\alpha$ and $\beta$ orbitals
	(see \eqref{eqn:HFCoeffMatrix})
	and
	\[
		N_{\text{bas}, l} := \Big| \left\{ (n',l',m') \, \big|\, (n',l',m') \in \Ibas
			\ \text{and} \ l' = l \right\} \Big|
	\]
	is the number of basis functions in the \CS basis which has angular momentum
	quantum number $l$.
\end{defn}

By construction $\RMSOl$ is the \RMS-averaged coefficient for a particular angular
momentum quantum number $l$ in the occupied SCF orbitals.
It therefore provides a measure which angular momentum quantum numbers $l$
are required in the current basis set for describing the ground state properly.
Conversely values of $\RMSOl$ below the convergence threshold $\epsilonconv$
of the \SCF procedure indicates that all \CS basis functions of this angular momentum
value $l$ can be safely removed from the \CS basis set without
influencing the accuracy of the \HF calculation significantly.
In many cases this property of $\RMSOl$ can assist in finding a good
value of $\lmax$ for truncating the orbital angular momentum.

\begin{figure}[p]
	\centering
	\includeimage{10_results/rmso_period2_vs_l}
	\caption
	[Plot $\RMSOl$ vs $l$ for the \HF ground state of the atoms of the second period]
	{
		Plot $\RMSOl$ vs $l$ for the \HF ground state
		of the atoms of the second period
		if a $(6,5,5)$ \CS basis is employed.
		In each case $\kexp$ was taken close to the optimal value.
		For \ce{Be} and \ce{Ne} a \RHF procedure was used,
		for the other cases \UHF.
	}
	\label{fig:RMSOl_period2}
\end{figure}
\begin{figure}[p]
	\centering
	\includeimage{10_results/rmso_period3_vs_l}
	\caption
	[Plot $\RMSOl$ vs $l$ for the \HF ground state of the atoms of the third period]
	{
		Plot $\RMSOl$ vs $l$ for the \HF ground state
		of the atoms of the third period
		if a $(6,5,5)$ \CS basis is employed.
		In each case $\kexp$ was taken close to the optimal value.
		For \ce{Mg} and \ce{Ar} a \RHF procedure was used,
		for the other cases \UHF.
	}
	\label{fig:RMSOl_period3}
\end{figure}

For example let us consider figures \ref{fig:RMSOl_period2}
and \vref{fig:RMSOl_period3},
which show the variation of $\RMSOl$ vs $l$ for the \HF ground state
for the atoms of the second and third period
if a $(6,5,5)$ Coulomb-Sturmian basis is employed.
In the plot roughly two kinds of behaviour can be identified.
The first kind applies to those atoms which are either closed-shell
like \ce{Be}, \ce{Ne}, \ce{Mg} or \ce{Ar}
or which have a half-filled valence sub-shell
like \ce{Li}, \ce{N}, \ce{Na} or \ce{P}.
For these a very pronounced drop in $\RMSOl$ occurs once a particular
angular momentum value $l$ has been reached.
For \ce{Li} and \ce{Be}, where only $s$-functions are occupied in the ground state,
this happens from $l=0$ to $l=1$
and for the other mentioned atoms from $l=1$ to $l=2$,
which in both cases is in perfect agreement with the behaviour expected
from the physical point of view.
For these atoms truncating at $\lmax = 0$ or $\lmax=1$, respectively,
will not introduce a noticeable error as we will see in the next section.
In contrast to this the other atoms
\ce{B}, \ce{C}, \ce{O}, \ce{F}, \ce{Al}, \ce{Si}, \ce{S} and \ce{Cl}
do not quite follow this trend.
Much rather their $\RMSOl$ value decreases only very moderately over the range
of angular momentum quantum numbers.

\begin{figure}
	\centering
	\includeimage{10_results/rms_lf_N}
	\caption[
		Root mean square coefficient value
		per angular momentum for nitrogen
	]
	{
		Root mean square coefficient value per
		basis function angular momentum quantum number $l$
		for selected orbitals of nitrogen.
		The atom is modelled
		in a $(6,5,5)$ \CS basis using \UHF.
	}
	\label{fig:RMSLF_N}
\end{figure}
\begin{figure}
	\centering
	\includeimage{10_results/rms_lf_C}
	\caption[
		Root mean square coefficient value
		per angular momentum for carbon
	]
	{
		Root mean square coefficient value per
		basis function angular momentum quantum number $l$
		for selected orbitals of carbon.
		The atom is modelled
		in a $(6,5,5)$ \CS basis using \UHF.
	}
	\label{fig:RMSLF_C}
\end{figure}
\begin{figure}
	\centering
	\includeimage{10_results/rms_lf_O}
	\caption[
		Root mean square coefficient value
		per angular momentum for oxygen
	]
	{
		Root mean square coefficient value per
		basis function angular momentum quantum number $l$
		for selected orbitals of oxygen.
		The atom is modelled
		in a $(6,5,5)$ \CS basis using \UHF.
	}
	\label{fig:RMSLF_O}
\end{figure}
A good hint to understand this behaviour provide figures
\ref{fig:RMSLF_N}, \vref{fig:RMSLF_C} and \vref{fig:RMSLF_O}.
These show the \RMS-averaged value of those orbital coefficients,
which share the same angular momentum quantum number $l$ in the corresponding
basis function.
For the modelling of the atoms in each case
a $(6,5,5)$ Coulomb-Sturmian basis with a near-optimal value of $\kexp$ is employed.
Whilst for nitrogen~(figure \ref{fig:RMSLF_N})
the $2s$ function mostly has significant coefficient values
associated to basis functions with $l=0$,
both for carbon~(figure \ref{fig:RMSLF_C}) as well as oxygen~(figure \ref{fig:RMSLF_O})
the basis functions with $l=2$ and $l=4$ are important as well.
Similar observations can be made for the $2p$ functions,
which for \ce{N} and \ce{C} have significant coefficients at angular momenta
$l=1,3,5$ and the $3d$ functions, which require $l=2$ and $l=4$,
sometimes even $l=0$, for a proper description.
This explains why $\RMSOl$ plots for carbon
and oxygen do not show the expected drop from $l=1$ to $l=2$,
since the higher angular momenta play a role
for the occupied $s$-type and $p$-type \SCF orbitals as well.
Equivalent plots to figures \ref{fig:RMSLF_C} and \ref{fig:RMSLF_O}
for the other atoms,
which are not closed-shell
or have a half-filled valence shell,
show very similar features,
which overall explains the slow decrease
in the $\RMSOl$ plots for such atoms.

% TODO OPTIONAL
%\begin{figure}[p]
%	\centering
%	\includeimage{10_results/rmso_ions_vs_l}
%	\caption
%	[Plot $\RMSOl$ vs $l$ for the \HF ground state of selected ions]
%	{
%		Plot $\RMSOl$ vs $l$ for the \HF ground state
%		of selected ions
%		if a $(7,6,6)$ \CS basis is employed.
%		In each case $\kexp$ was taken close to the optimal value.
%	}
%	\label{fig:RMSOl_ions}
%\end{figure}
%A first hint to understand this behaviour provides figure \vref{fig:RMSOl_ions}, where
%$\RMSOl$ plots for the related closed-shell or half-filled valence shell ions
%\ce{B+}, \ce{C-}, \ce{O+}, \ce{F-}, \ce{Al+}, \ce{Si-}, \ce{S+} and \ce{Cl-}
%are depicted.
%In these plots the expected drop between $l=0$ and $l=1$ or $l=1$ and $l=2$
%can be found as expected by chemical intuition,
%such that the electronic configuration most play a role in 

\label{sec:IssuesUHF}
The pending question is now why angular momenta higher than the expected
$l=0$ and $l=1$ are needed for modelling the $s$-like and $p$-like orbitals
in some atoms in the first place.
Since very similar $\RMSOl$ and \RMS orbital coefficient plots
are observed if \cGTO discretisations are used, see appendix \vref{apx:GaussianUHF},
this effect cannot be due to the \CS discretisation we employ.
Much rather it is an artefact of our \UHF treatment of the open-shell systems.
For example \citet{Cook1981} described a similar behaviour
for a \UHF modelling of carbon and fluorine
based on \cGTO discretisations.
He noticed that the $s$-type and $p$-type \SCF orbitals
for both these systems were not only
composed of \cGTO basis functions with $l=0$ and $l=1$,
but much rather were linear combinations of basis functions with
angular momentum quantum numbers in steps of $2$ apart.
So for $s$-like \SCF orbitals $s$, $d$, $g$, \ldots basis functions
were combined in his calculations --- exactly what we observe in figure \ref{fig:RMSLF_C}.
Later it was found that the occurrence of higher angular momenta in
the ground state is a general issue of
\UHF~\cite{Fukutome1981,Cook1984,McWeeny1985}.
\citet{Fukutome1981} provides a very detailed analysis of the
underlying mechanisms including a discussion of the effect of
spin symmetry breaking and \HF instabilities in \UHF and \GUHF.
Roughly speaking for cases with an unevenly occupied electronic configuration,
like a single or two $p$ electrons,
the initial guess for the \UHF Fock operator is
no longer spherically but \emph{axially} symmetric~\cite{McWeeny1985}.
This implies that the obtained \SCF orbitals no longer represent
a spherically symmetric density, but an axially symmetric density instead.
Over the turn of the \UHF iteration
this broken spherical symmetry allows for different
angular momenta to be combined in a single orbital,
such that the resulting \SCF orbitals are no longer of pure
$s$, $p$, $d$, \ldots character.
Let us give a few examples for this.
If a spherically symmetric $s$ orbital is amend it with
a fraction of $d_{z^2}$, then this effectively causes a stretching
of the orbital along the $z$ axis,
which makes it axially symmetric.
Similarly the $p_x$, $p_y$ and $p_z$ orbitals
may be amended with $f_{xz^2}$, $f_{yz^2}$ and $f_{z^3}$
to cause the same stretching along the $z$-axis in each of these.
Even if all $p$ orbitals of this $p$-shell are evenly occupied in the final \HF
ground state, the wave function is then axially symmetric.
If such linear combinations with higher angular momenta
lowers the total \SCF energy
the \UHF procedure for unevenly occupied electron configurations
may well explore these due to the broken spherical symmetry.
In contrast for evenly occupied valence shells,
like the half-filled valence shell atoms \ce{N} and \ce{P},
such a symmetry breaking may not occur,
such that the pure angular momentum character of each of the orbitals is kept
even in an \UHF treatment.

% coupling only to l+2 and the same m
% so we could shrink the basis by imposing that restriction!
% The states can be stationary or unstable points rather then minimaA
%
% Tho odd-even pattern can be explained by considering the parity
% of the wave function of that particular orbital. Even l have even parity
% and odd l have odd parity. Linear combination of functions of even
% and odd parity breks M_L symmetry (???)


% Sometimes we do not only have this issue, but the SCF also does not converge to
% a minimum but only a saddle point that way, so we need to force convergence using a DIIS.

With the aforementioned detailed analysis of the \UHF procedure
we are finally in a position to explain all features of the $\RMSOl$ plots
in figures \ref{fig:RMSOl_period2} and \ref{fig:RMSOl_period3}.
Conversely our discussion shows that $\RMSOl$ is a good diagnostic measure
for understanding which angular momentum quantum numbers
are required for an accurate quantum-chemical modelling.
Since its value for a particular quantum number $l$ indicates
the \RMS-averaged coefficient value
it even provides a quantitative measure for the error,
which is introduced if the range of available angular momentum in a \CS basis set
is truncated to angular momenta belew this value.

% TODO OPTIONAL
%This realisation motivates the next definition
%of a quantity to estimate the largest required angular momentum
%for modelling a given chemical system at target accuracy $\epsilonconv$.
%
%\begin{defn}
%	The \newterm{largest important angular momentum quantum number} is the value
%	\[ \limp = \max \{ l \, | \, (n,l,m) \in \Ibas \ \text{and} \ \RMSOl > \epsilonconv \}, \]
%	\ie the largest quantum number $l$
%	for which the corresponding occupied
%	orbital coefficients are still larger than the
%	desired accuracy of the SCF procedure.
%\end{defn}
%
%As we will see in the next section
%this quantity should only be taken as a guideline for choosing sensible basis sets
%and does not represent a mathematically rigorous error bound at all
%as it sometimes underestimates the amount of angular momentum required.

%
% --------------------------------------------------------------------------
%
\subsection{Basis sets with truncated angular momentum}
From the discussion of the previous section
it becomes clear that at least in some cases it makes sense to
limit not only $\nmax$, but on top of that $\lmax$ as well.
For example for beryllium the clear drop in $\RMSOl$
in figure \ref{fig:RMSOl_period2} suggests that limiting the angular
momentum quantum numbers to $\lmax = 0$ is reasonable.
Similar arguments for \ce{N} and \ce{P} suggest taking $\lmax = 1$ for these atoms.
In other words we would expect the discretisation
of the angular part of the \HF wave function to be already well-converged
for \CS discretisations with $\lmax = 0$ or $\lmax = 1$, respectively.
Consequently we will only need to increase $\nmax$ further and further
in order to converge the remaining radial part of the wave function as well.
Since fixing $\lmax$ reduces the scaling of the basis set size
from cubic in $\nmax$~(see \eqref{eqn:NbasFullShellBasis}) to linear%
~(see \eqref{eqn:NbasAmLimitedBasis}),
we would expect to obtain a much faster convergence rate.

\begin{figure}
	\centering
	\includeimage{10_results/ehf_vs_nlm}
	\caption[
		Relative error in $E_\text{HF}$ versus the basis size
		for selected \CS discretisations
	]{
		Relative error in $E_\text{HF}$ versus the number of basis functions
		for selected \CS basis sets of the form $(\nmax, \lmax, \lmax)$.
		The connected points show basis set progressions
		in which the maximum principle quantum number 
		$\nmax$ is varied in steps of one and the maximum $\lmax$ is fixed.
		The first and last value for $\nmax$ are indicated as small numbers
		next to the plot.
		The same line type is used for all progressions of the same $\lmax$
		and the same colour for all progressions of the same atom.
	}
	\label{fig:ErrorHF_vs_nlm}
\end{figure}
To test our hypothesis figure \vref{fig:ErrorHF_vs_nlm} shows some
example calculations
of beryllium, nitrogen, carbon, oxygen and phosphorus
using progressions of \CS basis sets, where $\lmax$ is limited
to either $0$, $1$ or $2$,
but $\nmax$ is ranged between $4$ and $12$
In each case the relative error of the \HF energy
with respect to the reference values in table \ref{tab:HFReference}
are plotted against the size of the \CS basis
and those error values corresponding to the same atom and the same $\lmax$,
but different $\nmax$, are connected by lines.
We will refer to such a sequence of connected error values
by the term progression in the following.
As usual $\kexp$ is fixed to a sensible value for all calculations
of the same atom
and for beryllium we used \RHF, for the other atoms \UHF.

Even though the basis sets
are now additionally truncated in angular momentum quantum numbers,
the \HF energies for beryllium still converge steadily.
This applies both to the cases $\lmax = 0$ as well as $\lmax = 1$.
Compared to figure \vref{fig:ErrorHF_vs_shell} one notices,
however, a massive improvement in convergence rate.
For the $\lmax = 1$ progressions of nitrogen and phosphorus the same holds true.
Choosing $\lmax = 2$ for nitrogen does not improve the obtained
values very much
in accordance with our the $\RMSOl$ plot~(figure \ref{fig:RMSOl_period2}).
Since the basis now grows faster as $\nmax$ increases,
the convergence rate is now slower, however.
For oxygen and carbon the angular momentum values $l > 2$ are important
for a proper modelling of the ground state as well.
As such it is no surprise that the convergence of the \HF energy
for these two cases stagnates visibly for the $\nmax$-progressions
with $\lmax = 1$ and $\lmax = 2$.
Even though the convergence is initially linear as well,
the curves bend off at some point.
The reason for this is that the truncation of the available
set of angular momentum quantum numbers to at most $\lmax$ causes
an error in the discretisation of the angular part of the \HF wave function.
At some point this error completely dominates,
such that improving the radial part by increasing $\nmax$
does not improve the relative error by much any more.

\begin{figure}
	\centering
	\includeimage{10_results/ehf_vs_nlm_O}
	\caption[
		Relative error in $E_\text{HF}$ versus the number of basis functions
		for oxygen.
	]{
		Relative error in $E_\text{HF}$ versus the number of basis functions
		for oxygen.
		The plot shows the oxygen progressions of figure \ref{fig:ErrorHF_vs_nlm}
		amended with further progressions using larger basis sets,
	}
	\label{fig:ErrorHF_vs_nlm_O}
\end{figure}
The results of our investigations on the oxygen atom are summarised in figure
\vref{fig:ErrorHF_vs_nlm_O}, which shows
the $(\nmax, 1,1)$ and $(\nmax, 2,2)$ progressions already depicted above
as well as one using $(\nmax, 3,3)$ \CS basis sets.
The effect of truncating the angular momentum is clearly visible.
The blue curve with $\lmax =1$ is not able to converge to relative errors
below around $7\E{-5}$,
whilst the orange curve with $\lmax = 2$ can take the error
down to $2\E{-5}$.
The green curve with $\lmax = 3$ on the other hand
converges almost linearly over the full range of $\nmax$ considered.
This can be explained if we take another look at the
$\RMSOl$ plot of oxygen in figure \ref{fig:RMSOl_period2}.
Comparing these findings with the $\RMSOl$ plot of oxygen
in figure \ref{fig:RMSOl_period2} indicates some noteworthy agreements.
Whilst $\text{RMSO}_2$ and $\text{RMSO}_3$ are of a similar size,
there is a larger decrease, about 2 orders of magnitude,
going from $\text{RMSO}_3$ to $\text{RMSO}_4$.
In other words selecting $\lmax = 2$ instead of $\lmax = 1$
does not improve the error in the angular part
as much as going to $\lmax = 3$ does.
This is reflected by the fact that for $\lmax = 3$
an almost linear convergence up to $\nmax = 12$ is obtained,
whilst for $\lmax < 3$, the error in the angular part starts to dominate
from around $\nmax = 10$, such that the convergence slows down.

For constructing \CS basis sets, which converge rapidly at \HF level,
a good balance between the error remaining in the discretisation of the angular
part as well as the error in the discretisation of the radial part is required.
As discussed the former aspect is controlled by selecting $\lmax$,
the latter by selecting $\nmax$.
We found in the case of oxygen that the magnitude of the error
in the discretisation of the angular part and the trends in the $\RMSOl$ versus $l$
plots are related.
We believe this finding to be general
in the sense that a significant drop of $\RMSOl$
from $l$ to $l+1$ indicates
that the \CS basis set progression with $\lmax = l$
will allow convergence to a lower error in the \HF energy
than $\lmax = l-1$ is able to.
Once a large enough value for $\lmax$ is chosen to converge the angular part,
the convergence in the radial part,
\ie the convergence in the progression of increasing $\nmax$,
is initially linear.
For large values of $\nmax$ the remaining error in the angular part
will start to dominate and yet a larger $\lmax$
has to be chosen to make further progress.

From the examples considered in this section we would expect to require
around $\nmax = 10$ to reach a target accuracy of 5 digits,
which equals a relative error of below $10^{-5}$.
For \ce{Li} and \ce{Be}, where $\lmax = 0$ is sufficient this equals
a $(10,0,0)$ basis consisting of only 10 \CS basis functions.
For \ce{N}, \ce{Ne}, \ce{Na}, \ce{Mg}, \ce{P} and \ce{Ar}
a $(10,1,1)$ basis would be required, which has 37 basis functions.
For the difficult cases like \ce{O}, \ce{C} and most other atoms
with a single or 2 unpaired electrions
at least $\lmax = 3$ is required.
A $(10,3,3)$ \CS basis has the enormous number of 126 basis functions,
which still only reaches 5 digits of accuracy in the \HF energy.

% TODO OPTIONAL
% TODO orbital energy convergence?
%	Maybe make a similar plot to fig. \vref{fig:ErrorHF_vs_nlm}
%	showing the relative errors of the orbital energies compared
%	to reference results in N or O or so.
%	Show how the orbital energies are approximated from above.
%	This would help in understanding the signifficance of the coefficients
%	wrt. correlation methods as well.
%
% TODO Not sure if this way of plotting is ideal.
%\begin{figure}
%	\centering
%	\includeimage{10_results/orben_vs_nlm_N}
%	\caption{Plot relative difference in the orbital energy vs $n$.
%		The relative difference is $1 - \varepsilon_i^{(n)} / \varepsilon_i^{(n+1)}$
%		for N.
%	}
%	\label{fig:OrbenNlmN}
%\end{figure}

% TODO OPTIONAL
%	I get the feeling the bad convergence for oxygen and the other
%	open-shell systems is because we use UHF,
%	which implicitly causes other determinants to be mixed into the HF
%	solution.
%	Some literature suggests that UHF destroys the rotational symmetry.
%	\begin{itemize}
%		\item Try to code a rudimentary ROHF and check weather things are better there
%	\end{itemize}
%
% Spin squared values:
%       should    is
%  Li   0.75       0.7500174679889692
%  Be   0
%  B    0.75       0.7535472111793353
%  C    2          2.00941178981773
%  N    3.75       3.755922043215328
%  O    2          2.006538441200304
%  F    1.75       0.7530439372523485

% TODO optional
% \subsection{$m$-tuning for oxygen and carbon}
% Do the calculation of 8.7 but truncate m=1, m=2
% Does not really seem to do the trick

\section{Convergence at correlated level}
% TODO OPTIONAL I do not like the "fraction of correlation energy" plots
%      It should be more sensible to instead show relative errors in the *total*
%      energies.

% TODO
% Plot MP2 correlation energy vs. Nbas for the computations we did on manjula
% converge_correlation. Annotate the points with the actual basis set used.
% Use this plot to justify the choice of basis in the nlm sense for the
% different elements.

In the previous section we took a first look at the convergence
properties of Coulomb-Sturmian-based discretisations at Hartree-Fock level.
In practical quantum-chemical calculations Hartree-Fock is typically
not the final answer,
but only a first step,
such that our discussion of convergence should really not focus on Hartree-Fock alone.
In this spirit the aim of this section is to take our preliminary guidelines
for sensible basis sets at \HF level and describe some adaptions,
which could help to construct sensible \CS basis sets for correlated
quantum-chemical methods.
In line with what we discussed before, we will ignore the dependency
of the \CS basis on the exponent $\kexp$ and keep a sensibly chosen,
fixed exponent value for each atom throughout.
So far we have not carried out many calculations
for investigating the dependency of the correlated ground-state energy
with respect to altering the maximal quantum numbers $\nmax$, $\lmax$ and $\mmax$.
Our results in this section are therefore just exemplary
and should not be taken to be general for Coulomb-Sturmian discretisations
at correlated level.
Moreover we have only considered two Post-HF approaches,
namely {\MP}2 and to a much lesser extent full CI,
such that the behaviour might well deviate in other methods.

\begin{figure}
	\centering
	\includeimage{10_results/Be_fci_cc_vs_nbas}
	\caption[
		Fraction of beryllium correlation energy
		recovered with selected \CS bases
	]{
		Fraction of beryllium correlation energy recovered
		relative to a FCI reference calculation with a $(10, 2, 2)$ \CS basis.
		All calculations employ an exponent of $\kexp = 2.1$.
	}
	\label{fig:EccFciBe}
\end{figure}
A first impression regarding the dependency of the correlation energy
on the \CS basis set provides figure \vref{fig:EccFciBe}.
It shows the fraction of the beryllium atom correlation energy, which is recovered
by selected \CS basis sets and at \FCI and {\MP}2 level,
plotted against the size of the basis.
As the reference, \ie $100\%$ correlation energy,
we take the value obtained in a \FCI calculation employing
the rather large $(10,2,2)$ \CS basis.
When it comes to interpreting this figure one has to be a little careful.
First of all the ground-state energy at \HF level is
not necessarily constant for all of the basis sets employed.
In the depicted cases the changes in $E_\text{HF}$ are,
however, only very little and orders of magnitude
smaller than the changes in correlation energy.
The reason for this is that the selected basis sets only differ
in the maximal angular momentum quantum numbers $\lmax$ and $\mmax$,
whilst the beryllium \HF wave function is already
converged very well in the angular part for $\lmax = 0$.
The second issue with this plot
is that the blue curve somewhat compares apples and pears,
namely a variationally obtained reference correlation energy at \FCI level
with a perturbatively obtained correlation energy using {\MP}2.
Ignoring this fact for a moment,
we find that the \FCI and the {\MP}2 correlation energy curves
follow very similar trends.
Most notable are the two strong increases in the amount of correlation
energy recovered going from $(6,0,0)$ to $(6,1,0)$
and from $(6,2,0)$ to $(6,1,1)$.
Interestingly another increase of $\lmax$,
namely the transition $(6,1,1)$ to $(6,2,1)$
does not have such a pronounced effect.
In line with the arguments presented in the context of \HF
it seems that the angular momentum discretisation
of the {\MP}2 or \FCI ground-state wave functions
are largely converged as soon as $\lmax = \mmax = 1$,
such that further increases of angular momentum have much smaller effects.

Since the \FCI calculations on large basis sets such as $(10,2,2)$ become
extremely costly for larger atoms with more electrons,
we did not perform such calculations except for beryllium.
The only other correlation method,
which is currently available from \molsturm is {\MP}2.
Thus somewhat pragmatically we limited
our investigation of the convergence at correlated level
for the other atoms of the second and third period to {\MP}2 only,
arguing that at least for the case of beryllium we got the same trends.
The results are presented in the tables of appendix
\vref{apx:CSCorrelationConvergence} and graphically
in figures \ref{fig:EccVsNlmEasy} and \vref{fig:EccVsNlmHard}.
These show that the fraction of \emph{total} {\MP}2 energy
which is missed by a particular basis set compared to the most
accurate result we obtained in our calculations for a particular atom.
Again this value is plotted against the size of the basis set.
Similar to the case of beryllium sketched above,
we concentrate on capturing the effect of converging the discretisation
of the angular part of the {\MP}2 wave function
by varying $\lmax$ and $\mmax$.
In figure \ref{fig:EccVsNlmEasy} for the half-filled
and filled valence shells,
a convergence is visible.
For \ce{Li} and \ce{Be},
where $\lmax = \mmax = 0$ converges the angular part of the \HF ground state,
$\lmax = \mmax = 1$ does so pretty much for the {\MP}2 ground state.
For the other atoms shown in figure \ref{fig:EccVsNlmEasy} $\lmax = \mmax = 2$
is at least required.
In other words compared to converging the \HF ground state
we roughly speaking need one extra angular momentum.
For the cases of one and two unpaired electrons,
which are shown in figure \ref{fig:EccVsNlmHard},
the picture is not so conclusive.
Since already the \HF ground state requires $\lmax = 3$
for a decent modelling of the angular part,
this is of course at least required for the {\MP}2 wave function as well.
But figure \ref{fig:EccVsNlmHard} seems to suggest
that $\lmax = 4$ is important as well
since the change from $(6,3,3)$ to $(6,4,4)$ is much
steeper compared to the change from $(6,2,2)$ to $(6,3,3)$ in figure \ref{fig:EccVsNlmEasy}.
Whether even larger angular momentum is required cannot be said
with the currently available results.

\begin{figure}
	\centering
	\includeimage{10_results/Etot_vs_bas_easy}
	\caption[
		Missing fraction of {\MP}2 energy versus \CS basis size
	]{
		Plot of the
		missing fraction of total {\MP}2 energy
		compared to a calculation employing a $(6,3,3)$ \CS basis
		versus the basis size.
		Shown are the atoms of the second and third period
		with a full or half-full valence shell.
	}
	\label{fig:EccVsNlmEasy}
\end{figure}
\begin{figure}
	\centering
	\includeimage{10_results/Etot_vs_bas_hard}
	\caption[
		Missing fraction of {\MP}2 energy versus \CS basis size (plot 2)
	]{
		Plot of the
		missing fraction of total {\MP}2 energy
		compared to a calculation employing a $(6,4,4)$ \CS basis
		versus the basis size.
		Shown are the atoms of the second and third period
		with one or two unpaired electrons.
	}
	\label{fig:EccVsNlmHard}
\end{figure}

Following our discussion above it is probably a little far fetched to assume
that one can properly judge how well
a \CS basis set is able to capture correlation effects
just by looking at {\MP}2 correlation energies.
Nevertheless given how well the trends of {\MP}2 and \FCI agree for beryllium,
it seems likely that at least for the well-behaving cases
with closed or half-filled valence shell
the rule to take one extra angular momentum for the correlated calculation
captures the predominant effects.
On top of the few investigations towards converging the angular part
of the correlated wave function,
no further investigation regarding $\nmax$
and the convergence of the radial part was attempted so far.

\section{The effect of the Coulomb-Sturmian exponent}
\label{sec:kexp}

In our discussion about the properties of Coulomb-Sturmian basis sets
we have  neglected the effect of the \CS exponent $\kexp$ so far.
Our main argument was that a \CS basis is complete regardless of the
value of $\kexp$,
such that for large enough \CS basis sets
the result will not depend on $\kexp$ anyway.
In practice the aim is of course to yield a sensible
discretisation of the wave function in the smallest basis possible.
For getting the best out of a particular truncated set,
the value of $\kexp$ can therefore not be chosen completely arbitrary.
This section is therefore devoted to a discussion
of some general trends regarding
the variation of obtained energies with $\kexp$.
Furthermore we will discuss some methodologies
for obtaining the optimal exponent $\kopt$ with respect to minimising the \HF energy.
These results will be presented for the
atoms of the first two periods of the periodic table.

\begin{figure}
	\centering
	\includeimage{10_results/EHF_terms_vs_k}
	\caption[
		Plot of the \HF energy contributions
		versus the Coulomb-Sturmian exponent $\kexp$.
	]{
		Plot of the \HF energy contributions
		of the beryllium
		atom versus the Coulomb-Sturmian exponent $\kexp$.
		All calculations are done in a $(5,1,1)$ \CS basis.
	}
	\label{fig:EnergyTermsVsK}
\end{figure}

In the \CS basis functions
$\kexp$ only occurs in the radial part \eqref{eqn:RadialPartCsRepeated}.
In the form of the exponential term $\exp(-\kexp r)$
it influences how quickly the basis functions decay at infinity
and in the form of the polynomial prefactor it determines
the curvature of the radial functions as they oscillate between the radial nodes.
Keeping this in mind let us consider figure \vref{fig:EnergyTermsVsK},
which shows the changes to individual energy contributions
of the \HF ground state energy as $\kexp$ is altered.
The largest changes are apparent for the nuclear attraction energy,
which decreases --- initially rather steeply --- as $\kexp$ is increased.
This can be easily understood from a physical point of view:
Since larger values of $\kexp$ imply a more rapid decay
of the basis functions,
the electron density on average stays closer to the nucleus,
which in turn leads to a lower (more negative) interaction energy
between electrons and nucleus.
The converse effect happens for smaller values of $\kexp$,
where the electron density is more expanded
and thus on average further away from the nucleus.
On the other hand the kinetic energy is related to the curvature of the wave function,
which --- as described above --- increases for larger $\kexp$.
In other words the trends of nuclear attraction energy and electronic kinetic energy
oppose each other,
with the kinetic energy being somewhat less effected.
On the scale depicted in figure \ref{fig:EnergyTermsVsK}
the variation of the electron-electron interaction,
\ie both classical Coulomb repulsion as well as the exchange interaction combined,
is much less pronounced.
Only a very minor increase with $\kexp$ can be observed.
The physical mechanism is again similar to the nuclear attraction
energy term,
namely that larger $\kexp$ compresses the wave function
and thus leads to the electrons reside more closely to another,
which increases the Coulomb repulsion between them.
The exchange interaction is effected as well,
but the changes are smaller and thus not visible.

\begin{figure}
	\centering
	\includeimage{10_results/Efci_vs_k}
	\caption[
		Plot of the \HF, {\MP}2 and \FCI energies
		versus the Coulomb-Sturmian exponent $\kexp$.
	]{
		Plot of the \HF, {\MP}2 and \FCI ground state energies
		of beryllium
		versus the Coulomb-Sturmian exponent $\kexp$.
		The optimal exponent $\kopt$ for each
		method is marked by a cross.
		All calculations are done in a $(5,1,1)$ \CS basis.
	}
	\label{fig:FCI_vs_k}
\end{figure}
Summing up all energy contributions leads to the blue curve
in figure \vref{fig:FCI_vs_k},
which shows the total Hartree-Fock energy versus
the Coulomb-Sturmian exponent $\kexp$.
From our discussion of the individual terms
it is apparent that at small values for $\kexp$
the increase in nuclear attraction energy dominates,
such that the \HF energy increases rapidly.
At large distances the kinetic energy and electron-electron
interaction terms win, such that a convex curve
for the plot $E_\text{HF}$ versus $\kexp$ results.
Adding correlation effects by a treatment of the atom at {\MP}2 or \FCI level,
does not change this overall behaviour much.
Up to a large extend the curves are just shifted downwards by the
correlation energy term.
The shift is, however, not completely uniform.
This can be seen if we consider the optimal \CS exponent $\kopt$,
which is denoted by a cross in each of the plots of figure \ref{fig:FCI_vs_k}.
This exponent of minimal energy shifts
to slightly larger values going from \HF to {\MP}2 and finally to \FCI
indicating that the amount of correlation energy is somewhat larger
at exponents slightly above $\kopt$ for Hartree-Fock.
Notice that $\kopt$ not only depends on the method used for modelling
a particular state,
but it well depends on the state as well.
For example for modelling the first $T1$ excited state of beryllium
a smaller value for $\kopt$ is obtained than the \FCI $\kopt$
of the depicted $S0$ ground state.

\begin{figure}
	\centering
	\includeimage{C_ene_gs_vs_k}
	\caption[
		Dependency of the \HF and {\MP}2 energies
		on the \CS exponent $\kexp$
		and the basis set
	]{
		Plot of the unrestricted
		\HF and {\MP}2 ground state energies of carbon
		versus the Coulomb-Sturmian exponent $\kexp$
		in the $(4,2,2)$, $(5,2,2)$ and $(6,2,2)$ basis sets.
		The optimal exponent $\kopt$ at \HF level
		for each basis set is marked by a cross.
	}
	\label{fig:EHF_vs_k}
\end{figure}

Since $\kexp$ only occurs in the radial part of the \CS basis functions
the effect of its variation depends on the
maximal principle quantum number $\nmax$ of the basis set.
As larger and larger values of $\nmax$ are used,
the discretisation of the radial part of the wave function
becomes more and more complete,
such that the choice of $\kexp$ in turn becomes less important.
Figure \vref{fig:EHF_vs_k} shows this for the ground state energy of the carbon atom
at unrestricted \HF and {\MP}2 level.
Whilst a $(4,2,2)$ \CS basis reproduces largely the
shape of the plots in figure \ref{fig:EHF_vs_k},
for $(5,2,2)$ and $(6,2,2)$ the energy versus exponent curves
become visibly flatter close to the optimal exponent (around $\kexp = 2.8$).
The influence of increasing $\nmax$ is not the same
for all values of $\kexp$.
Instead the curves seem to bend down in the range $\kexp > 3$.
In other words choosing a \CS exponent larger than $\kopt$
will generally speaking lead to a smaller error
than choosing a too large exponent%
\footnote{We already noted this aspect in the context of discussing
	the local energy plots in section \vref{sec:BasisCS}.}.
Another conclusion we can draw from figure \ref{EHF_vs_k}
is that the optimal value for the exponent $\kopt$
depends on $\nmax$ as well
as larger basis sets give rise to smaller values for $\kopt$.
We can rationalise by taking the plots of the energy terms
in figure \vref{fig:EnergyTermsVsK} into account.
We already noticed above that the nuclear attraction energy
is influenced by $\kexp$ most strongly.
Additionally it is (by magnitude) the largest contribution to the \HF energy.
In order to yield the minimal ground state energy
in a small basis the dominating effect is therefore to
minimise the nuclear attraction energy as much as possible.
As a result the optimal exponent $\kopt$ takes comparatively large values.
As the basis becomes larger a balanced description
of the complete physics becomes possible,
such that the electron repulsion and kinetic energy terms
are described better as well
and thus smaller values for $\kopt$ result.

Due to the structure of the energy versus exponent curves,
like the ones shown in figure \vref{fig:EHF_vs_k},
one hardly ever needs to know $\kopt$ very accurately.
As long as one uses a reasonable guess,
which is constructed to overestimate $\kopt$
rather than underestimate it, one is usually safe.
If a highly accurate treatment of a particular system is required,
then increasing $\nmax$ has both a much larger effect
and is computationally cheaper than finding the optimal exponent in the smaller basis.
See the next section for details.
% overestimating plays in line with correlated treatment,
% where $\kopt$ becomes larger due to the correlation.

% TODO OPTIONAL
%\begin{figure}
%	\centering
%	\includeimage{C_hfterms_vs_k}
%	\caption{HF energy terms vs $\kexp$ for different basis sets}
%	\label{fig:Eterms_vs_k}
%\end{figure}
%
%In figure \vref{fig:Eterms_vs_k} we observe the same trends
%as in figure \vref{fig:EnergyTermsVsK} for beryllium
%just with the difference that
%the result is shown for different basis sets.
%We observe that the trends get smaller if larger basis sets are chosen.


% TODO OPTIONAL
%\begin{figure}
%	\centering
%	\includeimage{C_orben_622_vs_k}
%	\caption{E orbital vs k, only alpha orbitals are plotted of 622}
%	\label{fig:orben_vs_k}
%\end{figure}
%
%The interpretation of the orbital energy vs $\kopt$ curve
%should be done with care.
%For this note that the minimal sum of the orbital energies
%occurs at a much smaller value for $\kopt$,
%since the electron-electron interactions are counted twice
%hence shifting the optimal value for the sum of the orbital
%energies more towards smaller $\kexp$.
%
%Nevertheless it allows us to qualitatively assess the realisation
%that for the description of the virtual or valence states
%smaller $\kopt$ values are preferable whilst for the core states
%larger $\kopt$ are typically preferred.
%Again since the core-like orbitals have a bigger contribution
%to the energy typically small basis sets tend to try to represent
%these in a good fashion first, hence tending to larger values of $\kopt$.
%Whilst the $\kopt$ decreases as the basis gets larger up to the point
%where its precise value becomes less and less important.


%
% ----------------------------------------------------------
%
\subsection{Determining the optimal exponent $\kopt$}
For variational quantum-chemical methods finding the best Coulomb-Sturmian
exponent $\kopt$ for the ground state is equivalent to minimising the ground state
energy with respect to $\kexp$.
Since such energy curves are
convex~(compare figures \ref{fig:FCI_vs_k} and \ref{fig:EHF_vs_k})
and only scalar functions of a single parameter,
this minimisation can be performed quite effectively
by gradient-free optimisation algorithm such as Brent's method~\cite{Brent1972}.
Starting from a reasonable guess for $\kopt$ convergence to the minimum
is usually achieved in around $10$ iterations.
For achieving this Brent's method will require a similar number of
energy computations using the chosen quantum-chemical method
and the chosen \CS basis.

With respect to the basis, which is used for such a procedure,
there are two things to note.
Firstly we already mentioned in our previous discussion
that $\kexp$ is a parameter,
which only affects the radial part.
In other words for obtaining a situation in which the individual
calculations of the energies
are not dominated by the error in the angular discretisation,
but the current value of $\kexp$,
large enough values for  $\lmax$ and $\mmax$ should be chosen.
Too large values of $\lmax$ will, however, lead to large basis sets,
thus long run times for the energy calculations.
In practice a compromise between accuracy and runtime needs to be found.
Our investigations~(see tables \ref{tab:Kopt2} and \vref{tab:Kopt3})
seem to suggest that one can find reasonable values for $\kopt$
already for basis sets where $\lmax$ is chosen smaller than
the value suggested by the $\RMSOl$ plots.
Secondly one should keep in mind that too large values of $\nmax$
will cause the energy-vs-$\kexp$ curves to become flat around $\kopt$,
which slows down convergence of the optimisation procedure.
Keeping in mind that typically getting \emph{roughly} the
right value for $\kopt$ is good enough,
it is sometimes more sensible to find $\kopt$ in a smaller
basis set,
where the energy-vs-$\kexp$ is more steep and calculations are faster,
and use this value for larger basis sets as well.
For the reasons we discussed in the previous section
such a $\kopt$ from a smaller basis
will always be an \emph{overestimation} of the actual $\kopt$,
which is favourable.

Our investigations have so far only considered
obtaining optimal exponents $\kopt$ at \HF level.
The most challenging aspect for doing so is in fact the stability of the
\SCF procedure itself.
Especially at the beginning of the iteration,
when the Coulomb-Sturmian exponent $\kexp$ is still relatively
far off the optimal value, the core Hamiltonian guess%
\footnote{So far only random guesses, guesses from previous \SCF cycles
	and core Hamiltonian guesses are implemented in \molsturm.}
we employ by default is not very good and frequently
fails to lead to the true \SCF minimum in our \SCF scheme.
Much rather another stationary point on the \SCF Stiefel manifold is found.
If we now continue to use the resulting wrongfully converged \SCF coefficients
as the guess for the next iteration of Brent's method,
we will typically manage to find a $\kopt$,
but this might not be the $\kopt$ of the true \SCF minimum,
\ie the true \HF ground state.
On the other hand if we start from the core Hamiltonian guess each time,
it can happen that the \SCF iterations for different values of $\kexp$
lead to different stationary points on the Stiefel manifold.
This violates a fundamental assumption of Brent's method,
namely the continuity of the objective function.
In other words the optimisation procedure is likely to find
a wrong value for $\kopt$ in this case.

Our remedy is to first make very sure we obtain a reliable
guess for starting the \SCF procedures called during the optimisation
before starting the optimisation procedure energy versus $\kexp$ at all.
In order to do so
we first perform $5$ {\SCF}s starting from totally random guesses
for the input value of $\kexp$ supplied by the user.
From the lowest-energy result of these we then
take the orbital energies $\varepsilon_i$
and use them to estimate a second value for $\kexp$, namely
\begin{equation}
	\kexp \simeq \sqrt{\frac{-2}{\Nelec} \sum_{i\in\Iocc} \varepsilon_i}.
	\label{eqn:KoptHfEstimate}
\end{equation}
The rationale for this heuristic formula is
the energy-dependent decay of the exact wave function~\cite{Kato1951},
which --- assuming \HF to be exact --- would manifest as well in an
energy-dependent decay of the orbitals by themselves.
Applying the formula
\[ \varepsilon_i = - \frac12 k_i \qquad \Leftrightarrow \qquad k_i
	= \sqrt{-2 \varepsilon_i}
\]
to yield the best exponent $k_i$ for describing orbital $i$ and taking the average
over all $k_i$ results in \eqref{eqn:KoptHfEstimate}.
The result from applying \eqref{eqn:KoptHfEstimate} is typically not extremely good,
but in the cases we considered it is at least in the same order of magnitude
as the final $\kopt$,
such that this estimate is easy to compute and corrects for the cases,
where the user's guess was very far off.
For this second $\kexp$ we perform another $5$ \SCF iterations
starting completely from random guesses.
From all $10$ obtained \SCF ground states,
both the $5$ with the $\kexp$ supplied by the user and the $5$ with the $\kexp$
from \eqref{eqn:KoptHfEstimate},
we only keep the solution, which has the lowest \HF energy.
For \emph{all} {\SCF}s which are started during the
subsequent energy versus $\kexp$ optimisation
this solution is used as the initial guess.
In this way all inner {\SCF}s approach the \SCF procedure
from the same reliable guess,
which largely avoids discontinuities in the \HF energies
and thus directs Brent's method to a sensible value for $\kopt$.

This algorithm for finding $\kopt$ is not cheap,
since around 20 to 30 complete {\SCF}s are required for convergence.
It is, however, reliable and allowed us to obtain optimal exponents
for a range of basis sets for all atoms of the second and third period.
These results are shown in tables \vref{tab:Kopt2} and \vref{tab:Kopt3}.
For convenience this procedure is implemented in \molsturm
and can be called from \python using the function
\texttt{find\_kopt} from the module \texttt{molsturm.sturmian.cs}.
\molsturm also offers the function \texttt{empirical\_kopt}
as a cheaper empirical estimate for $\kopt$.
It is based on interpolations using the values from
tables \ref{tab:Kopt2} and \ref{tab:Kopt3},
can thus only be used for atoms of the second and third period.

%
% ----------------------------------------------------------
%
\subsection{Relation of optimal exponent and effective nuclear charge}
\label{sec:ValuesKopt}

In his \citeyear{Slater1930} paper \citet{Slater1930} proposed simple guidelines
for approximating the orbitals of atoms.
For this he introduced for each orbital a shielding parameter $\sigma$,
which was supposed to indicate how much of the nuclear charge is screened
away by the electrons closer to the core.
He then proceeded to describe the functional
form of the atomic orbitals by the simple analytic expression
\begin{equation}
	\chi_{n^\ast, \sigma} = r^{n^\ast - 1} \exp\left( - \frac{(Z - \sigma)r}{n^\ast} \right)
	\equiv r^{n^\ast - 1} \exp\left( -\zeta r\right),
	\label{eqn:SlaterShielding}
\end{equation}
along with empirical rules to find $n^\ast$ and $Z - \sigma$,
the \newterm{effective nuclear charge}.
We already met functions like \eqref{eqn:SlaterShielding}
as basis functions for solving the Hartree-Fock problem
when we discussed Slater-type orbitals in section \vref{sec:STO}.
In the same chapter we mentioned the close relationship between
the Coulomb-Sturmians and the Slater-type orbitals
in the sense that the \CS exponent $\kexp$
plays the role of the Slater exponent $\zeta$
with the subtle difference that for \CS basis sets all functions
need to carry the same exponent.

\newcommand{\zCl}{\ensuremath \zeta_\text{Clementi}}
The rough results obtained by Slater's rules
were later refined by \citet{Clementi1963},
who determined optimal values for $\zeta$ by performing \HF calculations.
In turn they used these values to define a new set of shielding parameters
and thus a new set of effective nuclear charges.
Their optimisation procedure was very similar to the procedure
we followed to find $\kopt$,
namely they optimised the energy variationally with respect to the
Slater exponents $\zeta$.
Both the similarity of the form of both types
of functions as well as the similarity of the procedures followed
indicates that our $\kopt$ and the optimal exponents $\zCl$ from
\citeauthor{Clementi1963} should bear some resemblance.

\begin{figure}
	\centering
	\includeimage{10_results/kopt_vs_atnum}
	\caption[
		Plot atomic number versus the optimal Coulomb-Sturmian exponent
	]{
		Plot of the atomic number versus the optimal Coulomb-Sturmian exponent
		$\kopt$ for the neutral atoms of the second and the third period.
		For comparison the occupation-averaged value of the \citet{Clementi1963} optimal
		Slater exponent $\zeta_\text{Clementi}$ are shown as well.
	}
	\label{fig:KoptVsAtnum}
\end{figure}
As a first attempt to characterise this similarity we propose
to compare $\kexp$ to the average value of $\zCl$ taken
in all occupied orbitals of a particular atom.
A plot of these values
across the second and third period of the periodic
table is shown in figure \vref{fig:KoptVsAtnum}.
Over the full depicted range the magnitude of $\kopt$ and $\zCl$ stays similar.
Furthermore except the sharp drop going from atom number 10 to 11
the roughly linear increase of $\zCl$ is reproduced by $\kopt$.
One reason why the diverging feature between atom number 10 and 11
is observed is that we chose to use a different, larger \CS basis set
for determining $\kopt$ in the third period.
In our discussion related to figure \ref{fig:EHF_vs_k}
we already mentioned that larger basis sets tend to
yield a lower value of $\kopt$.
The observed drop in figure \ref{fig:KoptVsAtnum} is, however,
much larger than any lowering induced by increasing the basis
we observed in our calculations~(see tables \ref{tab:Kopt2} and \ref{tab:Kopt3}).
One possible additional explanation could be the reduction
of information, which is implied by taking the average of all $\zCl$.
For example when changes in the physics of the electronic structure of the atom
cause relative adjustments of the exponents $\zCl$,
this is not captured by the average $\zCl$.
Especially when going to a new shell,
\ie when adding a new, more expanded orbital
with only a single electron in it,
the structure of the electron density does indeed change more
compared to the previous atom as in other cases.
Whilst the Slater-type orbital basis has more degrees of freedom
in form of the multiple exponents to adapt to this,
the \CS basis needs to balance the errors,
which could lead to the observed deviation from the trend in the previous period.

Overall figure \ref{fig:KoptVsAtnum} suggests
that there is some connection between $\kopt$ and the average $\zCl$.
Considering the relationship between $\zCl$ and the effective nuclear charge in turn,
we could think of $\kopt$ as a measure for the average effective nuclear charge,
which is felt by the individual orbitals.

% TODO OPTIONAL
%An alternative way of viewing this is
%that as a new shell becomes occupied it requires
%the representation of an extra (more expanded) orbital.
%So initially $\kexp$ gets reduced.
%As we move across the period the nuclear charge increases
%and since the shielding abilities of electrons in the same shell
%is not that strong the orbitals become more contracted and occupy less space
%thus $\kexp$ increases again.
%As we move to larger periods this effect of change is averaged out
%again by the more constant core-like orbitals.

% TODO OPTIONAL
% I am not sure this is correct
%Over the period $\kopt$ follows a roughly linear trend,
%where the increase is less steep in the third period compared to the second.
%We can understand that by considering once again
%that $\kopt$ gives an average over all orbitals.
%Since especially the low-energy orbitals close to the nucleus
%are shielded very little,
%their value of $\kopt$ increases constantly along the periodic table.
%The $\kopt$ for the valence orbitals, however, changes much more.
%Especially at the beginning of the period, where atomic radii are large,
%the $\kopt$ for the valence orbitals are small,
%whilst this changes as the atoms get smaller across the period
%and hence $\zeta_\text{Clementi}$ and $\kopt$ get larger.
%This effect is diminished, however,
%by the constant $\kopt$ increase of the
%much larger number of core-like orbitals
%and thus the rate of change of $\kopt$ across a period gets smaller,
%the slope thus smaller as well.
%Furthermore this implies that the rapid $\kopt$ change
%going from one period to the next (e.g. Ne to Na)
%will get smaller and smaller as well
%as the effect of the valence-like orbitals gets averaged out.

% TODO OPTIONAL I am not sure this makes sense either
%Along this line one ansatz for the physical interpretation of $\kopt$
%is to relate it to the average shielding in an atom as follows.
%By Slater's original definition \eqref{eqn:SlaterShielding},
%we can identify the relationship
%\[ Z_\text{eff} = Z - \sigma = n^\ast \zeta_\text{Clementi} \]
%between the effective nuclear charge $Z_\text{eff}$ felt
%by a particular atomic orbital
%and the optimal exponent $\zeta_\text{Clementi}$~\cite{Slater1930},
%where $n^\ast$ is a function of the quantum number $n$ of the orbital
%under consideration.
%By means of a similar rationale to what has been sketched above,
%we can therefore find a chemical interpretation for $\kopt$,
%namely that it gives an understanding of the average
%charge which is felt by the orbitals of a particular atom
%or in other words the average shielding which takes place in the atom.

\include{table_kopt}

% TODO OPTIONAL
% \section{Suggestions for Coulomb-Sturmian basis sets}
% TODO Table of optimal values for k per basis set.
%      together with their energy and a comparison to literature and UHF values
%      create the UHF values via a CBL expansion from cc-pvNz calculations with Orca
%      N = 5, 6, 7 maybe. Do this for RHF and closed-shell as well to see the difference
%      This probably should be at the end of the chapter leading over to the comparison
%      with the Gaussians

\include{table_kopt}

See table \vref{tab:Kopt2} and table \vref{tab:Kopt3} % and table \vref{tab:Kopt4}


% Note that molsturm uses the fitted kopt values as a builtin ad hoc estimate
	% Explain procedure for finding the optimal k
	% explain justification for the "out of my hat formula"
	% comapare with various definitions of Z_eff
	% compare with solution to 1-electron hydrogenic equation energy scaling with Z



% Show and discuss the ideal exponent
% Maybe hint at or suggest good values for nlm

% Since the CS are complete in the radial part,
% we can essentially select sensible values for l and m and converge them
% with respect to these quantities thereby determining suggested values for l and m
% for other types of basis functions as well.
% => AM choice should apply to gaussians (and fe and \ldots), too!


Procedure:
\begin{enumerate}
	\item Find a not too small and not too large basis
	\item Optimise k
	\item Compute at optimal k a full-shell basis set
	\item find $\RMSOl$
	\item limit $\lmax$ and $\mmax$ properly
	\item Go to large-enough value for $\nmax$ to reach desired convergence
\end{enumerate}

\subsection{A word about the physics and the chemistry}
Let us briefly recall that Coulomb-Sturmians are the result of
solving equation \eqref{eqn:CS},
which is very much related to the Schrödinger equation for hydrogenic atoms.
Most importantly the angular part of both the Coulomb-Sturmian
as well as the solution to the hydrogenic problem (see section \vref{sec:HydrogenAtom})
is the same, namely a spherical harmonic $Y_l^m$.
Along this lines we would expect that the \CS basis functions
which are most important to describe the electronic structure
of an atom are related to the quantum numbers expected
for the occupied atomic orbitals in this atom by the Aufbau principle
or the Madelung rule.



A very remarkable features of the CS basis is how well it captures all the expectations
from the chemical and phyisical behaviour, even though this has not been taken
into account in the above consideration, all chemical trends of the
periodic table which are known are nicely reproduced and hence
we would guess, that they are very good at capturing chemical effects as well.

\todo[inline,caption={}]{
	\begin{itemize}
		\item Compute Sturmian k-dependence for more than one example
		\item Contrast the results for the periodic table?
	\end{itemize}

	% Explain shape of E-vs-k graph:
	% large k favour valance orbitals and disfavour core-like orbitals
	% kmall k vice versa
	% valance orbitals have smaller contribution to the HF energy
	% Hence steeper slope at small k side
	%
	% Hence more complete basis leads to smaller kopt
	%    - Since more basis functions the large-k contributions are represented
	%      in the extra basis functions with larger n, hence we can choose a smaller k
	%      to also represent the valence orbitals well (This is not true!)
	%    - Explains also why the valance orbitals only become bonding if basis is large enough
	%
	% Compute E-vs-k graph for a few examples.
	% Compute Orben vs k grapk

	% Larger lmax helps nothing fer the first 3 periods

	% Fit approximate formula for atomic number vs k_opt

	% Show convergence graph for eigenvalues (e.g. 311, 411, 511, 611, 711, ... for one example
	%       Go into largeness of the basis for proper description of the virtual orbitals
	% Show convergence graph for eigenvalues for k

	% m tuning not looked into very closely

	% TODO
	% FCI calculation of something to compare
	%       interpret FCI
	% Can one really make the argument that a smaller orbital energy value is a better approximation?

	% Smaller optimal exponent for the energy (physical argument):
	%   - Double counting of e-e interaction leads to electrons avoiding another
	%   - Hence smaller exponent (and higher effective energy, i.e. smaller magnitude)

	% Compare HF energy and HF orbital energies to CBS limit
}


% \section{Comparison Sturmians and cGTOs}
\to do[inline,caption={}]{
	% Take the values already presented above
	% and contrast them
	% Maybe try to plot a FCI density and compare with the gaussian and sturmian HF density

	% Look at relative and not absolute errors

	\begin{itemize}
		\item Compute some crude calculations for the periodic table
		\item Use both atoms and a few ions
		\item Show some numbers for the comparison Sturmians - cGTOs
	\end{itemize}
	% Improve HF energy values by adding only s functions
	%         (leads to smaller basis sets and should still work)
}

\begin{table}
	\centering
	\begin{tabular}{lc|ccc|ccc}
		       &            & \multicolumn{3}{c}{\CS}     & \multicolumn{3}{c}{Gaussian} \\
		system & literature & basis & $\Nbas$ & relative error & basis & $\Nbas$ & relative error \\
	\end{tabular}
	\caption{Comparison of \CS and Gaussian results for a few example systems}
	\label{tab:CSvsGaussian}
	% TODO OPTIONAL Just build this table at first summarising the results obtained.
	%      Do not yet do the Gaussian calculation
\end{table}

% TODO OPTIONAL closed or half-shell ions are ready.


% Compare S1 and T1 of Beryllium with gaussian calculation of an equivalent basis at FCI
% Are we also better in capturing the correlation energy at FCI level?

% Super accurate Be FCI calculation for comparison
% Be FCI 10,2,2 k = 1.985  E_FCI = -14.663

\section{Coulomb-Sturmian-based excited states calculations}
\label{sec:SturmianADC}
This section provides an outlook towards excited states calculations
employing Coulomb-Sturmians as the underlying basis functions.
As mentioned in section \vref{sec:MolsturmState} the \python interface
of \molsturm allowed us to link it to multiple third-party packages.
One of these is \adcman~\cite{Wormit2014},
which in this manner can be employed to perform excited states calculations
based on the algebraic diagrammatic construction scheme at
{\ADC}(1), {\ADC}(2), {\ADC}(2)-x~\cite{Schirmer1982}
and {\ADC}(3)~\cite{Trofimov1999} level
based on any basis function type supported by \molsturm.

\begin{figure}
	\centering
	\includeimage{10_results/be_adc2_vs_n}
	\caption[
		Convergence of a \CS-based {\ADC}(2) calculation of beryllium
	]{
		Convergence of a \CS-based {\ADC}(2)~\cite{Schirmer1982}
		calculation of beryllium.
		Plotted are the singlet excitation energies
		going from the ground state $2s2s$ to the denoted excited state.
		We show the results from a progression
		of \CS calculations with exponent $\kexp = 2.0$
		as well as bases sets of the form $(\nmax, 1, 1)$.
		For comparison the last two data points
		show the results from a \cGTO-based calculation
		using cc-pVTZ~\cite{Prascher2011}
		as well as the experimental values from \citet{Moore1949}.
	}
	\label{fig:SturmianAdcResults}
\end{figure}
This section reports the first successful {\ADC}(2)
calculation using Coulomb-Sturmians for the discretisation.
In figure \ref{fig:SturmianAdcResults} we show the singlet excitation energies
of the beryllium atom
as a progression with increasing \CS basis set size from $(4,1,1)$ to $(10,1,1)$.
For comparison the figure further indicates an equivalent calculation
using cc-pVTZ~\cite{Prascher2011} as well as the experimental values~\cite{Moore1949}.
Within the \CS basis set progression the results converge from above as expected.
Judging from the plots
a maximum principle quantum number around $\nmax = 10$
seems to be at least required to converge the radial part.
This agrees with our findings for the ground state,
see for example figures \vref{fig:ErrorHF_vs_nlm} and \vref{fig:ErrorHF_vs_nlm_O},

Comparing the computed excitation energies to the experimental values
the $(10,1,1)$ basis set
performs worse than cc-pVTZ at the first excited state $2s2p$,
but better for the $2s3s$ and the $2s3p$ states.
This result is, however, a little misleading for two reasons.
First the cc-pVTZ basis set and the $(10,1,1)$ \CS basis are not exactly comparable,
since they have a deviating structure.
Whilst cc-pVTZ contains 10 contracted Gaussian
functions with angular momentum up to $l = 4$,
$(10,1,1)$ contains 37 uncontracted Coulomb-Sturmians
with angular momentum at most $l = 1$.
Second the \CS basis has not really been optimised at all
with respect to {\ADC}(2) as a method or with respect to the excited states of beryllium.
For example the employed \CS exponent of $2.0$ is a good value for describing
the ground state of beryllium, but it is certainly not an optimal value
for describing the excited states.
Further there is some indications from example calculations
that at least angular momentum $l = 2$
is required for a proper description of the $2s2p$ excited state.
In figure \ref{fig:SturmianAdcResults} this amounts to explain,
why the observed convergence to a \emph{higher} excitation energy
than the \cGTO result or experiment is observed.

Keeping both these aspects in mind it is therefore not yet possible
to directly compare the \CS and the \cGTO results.
But given than no attempts to optimise the \CS basis towards the {\ADC}(2)
excited states setting have been made,
it is still remarkable to find the observed convergence.
A further, more systematic investigation could easily lead
to a clarification of the picture and allow to contrast
the different properties of both discretisations with respect to computing atomic spectra.


% TODO OPTIONAL
% For potassium l_max = m_max = 2 seems to be good for HF
% Do some more calculations to check the correlation effects
%
% We want the potassium 30s atomic orbital
%\todo[inline,caption={}]{
%	\begin{itemize}
%		% TODO Can we actually do this?
%		% Try to do the orbital rotation thingy
%		\item hint at it or show some examples
%		\item Maybe show comparison of convergence rates (rel. error vs. basis size)
%			in the sensible Sturmian $nlm$ basis sets vs similar Gaussian ones
%		\item Look at how the Gaussian basis sets have been constructed
%			(especially the correlation consistent and pc-n ones)
%			and compare how the Sturmians behave if similar
%			Constructions are used, also motivated from the previous section.
%	\end{itemize}
%}

