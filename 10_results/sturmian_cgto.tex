\section{Comparison Sturmians and cGTOs}
\to do[inline,caption={}]{
	% Take the values already presented above
	% and contrast them
	% Maybe try to plot a FCI density and compare with the gaussian and sturmian HF density

	% Look at relative and not absolute errors

	\begin{itemize}
		\item Compute some crude calculations for the periodic table
		\item Use both atoms and a few ions
		\item Show some numbers for the comparison Sturmians - cGTOs
	\end{itemize}
	% Improve HF energy values by adding only s functions
	%         (leads to smaller basis sets and should still work)
}

\begin{table}
	\centering
	\begin{tabular}{lc|ccc|ccc}
		       &            & \multicolumn{3}{c}{\CS}     & \multicolumn{3}{c}{Gaussian} \\
		system & literature & basis & $\Nbas$ & relative error & basis & $\Nbas$ & relative error \\
	\end{tabular}
	\caption{Comparison of \CS and Gaussian results for a few example systems}
	\label{tab:CSvsGaussian}
	% TODO OPTIONAL Just build this table at first summarising the results obtained.
	%      Do not yet do the Gaussian calculation
\end{table}

% TODO OPTIONAL closed or half-shell ions are ready.


% Compare S1 and T1 of Beryllium with gaussian calculation of an equivalent basis at FCI
% Are we also better in capturing the correlation energy at FCI level?

% Super accurate Be FCI calculation for comparison
% Be FCI 10,2,2 k = 1.985  E_FCI = -14.663
