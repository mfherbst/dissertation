\section{Coulomb-Sturmian-based ADC calculations}
This section provides a first look into excited states calculations
employing Coulomb-Sturmians as the underlying basis functions.
As mentioned in section \vref{sec:MolsturmState} the modular design
of the \molsturm package allowed us to link it to a few third-party
packages employing our \python interface.
One of them is \adcman~\cite{Wormit2014},
which can be employed to perform excited states calculations
based on the algebraic diagrammatic construction scheme at
{\ADC}(1), {\ADC}(2), {\ADC}(2)-x~\cite{Schirmer1982}
and {\ADC}(3)~\cite{Trofimov1999} level.

\begin{figure}
	\centering
	\includeimage{10_results/be_adc2_vs_n}
	\caption[
		Convergence of a \CS-based {\ADC}(2) calculation of beryllium
	]{
		Convergence of a \CS-based {\ADC}(2)~\cite{Schirmer1982}
		calculation of beryllium.
		Plotted are the singlet excitation energies
		going from the ground state $2s2s$ to the denoted excited state.
		We show the results from a progression
		of \CS calculations with exponent $\kexp = 2.0$
		as well as bases sets of the form $(\nmax, 1, 1)$.
		For comparison the last two data points
		show the results from a \cGTO-based calculation
		using cc-pVTZ~\cite{Prascher2011}
		as well as the experimental values from \citet{Moore1949}.
	}
	\label{fig:KoptVsAtnum}
\end{figure}
This section now reports the first successful {\ADC}(2)
calculation employing Coulomb-Sturmians for the discretisation
on the beryllium atom.
In figure \ref{fig:KoptVsAtnum} we show the singlet excitation energies
as a progression with increasing \CS basis set size from $(4,1,1)$ to $(10,1,1)$.
For comparison the figure also indicates an equivalent calculation
using cc-pVTZ~\cite{Prascher2011} as well as the experimental values~\cite{Moore1949}.
Within the \CS basis set progression the results converge nicely from above.
Compared to cc-pVTZ the $(10,1,1)$ basis set
performs worse at the first excited state $2s2p$~($\unit[1.0]{eV}$ versus $\unit[0.6]{eV}$ error) but significantly better for the $2s3s$ and the $2s3p$ states,
where the cc-pVTZ error is four times as large
for $2s3s$~($\unit[2.2]{eV}$ versus $\unit[0.5]{eV}$)
and twice as large for $2s3p$~($\unit[2.0]{eV}$ versus $\unit[1.2]{eV}$).

When it comes to the comparison with experiment there are more
errors to consider than just the discretisation error.
In our treatment we neglect relativistic effects
and {\ADC}(2) as an approximate method for solving excited states
of the electronic Schrödinger equation naturally contains
an intrinsic method error as well.
Nevertheless the initial results we have presented
seem to suggest that Coulomb-Sturmians
are able to give a more balanced description of more than one state
compared to \cGTO basis functions.
For this reason they could be a promising alternative
for modelling excited states phenomena.


% TODO OPTIONAL
% For potassium l_max = m_max = 2 seems to be good for HF
% Do some more calculations to check the correlation effects
%
% We want the potassium 30s atomic orbital
%\todo[inline,caption={}]{
%	\begin{itemize}
%		% TODO Can we actually do this?
%		% Try to do the orbital rotation thingy
%		\item hint at it or show some examples
%		\item Maybe show comparison of convergence rates (rel. error vs. basis size)
%			in the sensible Sturmian $nlm$ basis sets vs similar Gaussian ones
%		\item Look at how the Gaussian basis sets have been constructed
%			(especially the correlation consistent and pc-n ones)
%			and compare how the Sturmians behave if similar
%			Constructions are used, also motivated from the previous section.
%	\end{itemize}
%}
