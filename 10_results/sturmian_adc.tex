\section{Coulomb-Sturmian-based ADC calculations}
\label{sec:SturmianADC}
This section provides an outlook towards excited states calculations
employing Coulomb-Sturmians as the underlying basis functions.
As mentioned in section \vref{sec:MolsturmState} the \python interface
of \molsturm allowed us to link it to multiple third-party packages.
One of these is \adcman~\cite{Wormit2014},
which in this manner can be employed to perform excited states calculations
based on the algebraic diagrammatic construction scheme at
{\ADC}(1), {\ADC}(2), {\ADC}(2)-x~\cite{Schirmer1982}
and {\ADC}(3)~\cite{Trofimov1999} level
based on any basis function type supported by \molsturm.

\begin{figure}
	\centering
	\includeimage{10_results/be_adc2_vs_n}
	\caption[
		Convergence of a \CS-based {\ADC}(2) calculation of beryllium
	]{
		Convergence of a \CS-based {\ADC}(2)~\cite{Schirmer1982}
		calculation of beryllium.
		Plotted are the singlet excitation energies
		going from the ground state $2s2s$ to the denoted excited state.
		We show the results from a progression
		of \CS calculations with exponent $\kexp = 2.0$
		as well as bases sets of the form $(\nmax, 1, 1)$.
		For comparison the last two data points
		show the results from a \cGTO-based calculation
		using cc-pVTZ~\cite{Prascher2011}
		as well as the experimental values from \citet{Moore1949}.
	}
	\label{fig:SturmianAdcResults}
\end{figure}
This section reports the first successful {\ADC}(2)
calculation using Coulomb-Sturmians for the discretisation.
In figure \ref{fig:SturmianAdcResults} we show the singlet excitation energies
of the beryllium atom
as a progression with increasing \CS basis set size from $(4,1,1)$ to $(10,1,1)$.
For comparison the figure further indicates an equivalent calculation
using cc-pVTZ~\cite{Prascher2011} as well as the experimental values~\cite{Moore1949}.
Within the \CS basis set progression the results converge from above as expected.
Judging from the plots
a maximum principle quantum number around $\nmax = 10$
seems to be at least required to converge the radial part.
This agrees with our findings for the ground state,
see for example figures \vref{fig:ErrorHF_vs_nlm} and \vref{fig:ErrorHF_vs_nlm_O},

Comparing the computed excitation energies to the experimental values
the $(10,1,1)$ basis set
performs worse than cc-pVTZ at the first excited state $2s2p$,
but better for the $2s3s$ and the $2s3p$ states.
This result is, however, a little misleading for two reasons.
First the cc-pVTZ basis set and the $(10,1,1)$ \CS basis are not exactly comparable,
since they have a deviating structure.
Whilst cc-pVTZ contains 10 contracted Gaussian
functions with angular momentum up to $l = 4$,
$(10,1,1)$ contains 37 uncontracted Coulomb-Sturmians
with angular momentum at most $l = 1$.
Second the \CS basis has not really been optimised at all
with respect to {\ADC}(2) as a method or with respect to the excited states of beryllium.
For example the employed \CS exponent of $2.0$ is a good value for describing
the ground state of beryllium, but it is certainly not an optimal value
for describing the excited states.
Further there is some indications from example calculations
that at least angular momentum $l = 2$
is required for a proper description of the $2s2p$ excited state.
In figure \ref{fig:SturmianAdcResults} this amounts to explain,
why the observed convergence to a \emph{higher} excitation energy
than the \cGTO result or experiment is observed.

Keeping both these aspects in mind it is therefore not yet possible
to directly compare the \CS and the \cGTO results.
But given than no attempts to optimise the \CS basis towards the {\ADC}(2)
excited states setting have been made,
it is still remarkable to find the observed convergence.
A further, more systematic investigation could easily lead
to a clarification of the picture and allow to contrast
the different properties of both discretisations with respect to computing atomic spectra.


% TODO OPTIONAL
% For potassium l_max = m_max = 2 seems to be good for HF
% Do some more calculations to check the correlation effects
%
% We want the potassium 30s atomic orbital
%\todo[inline,caption={}]{
%	\begin{itemize}
%		% TODO Can we actually do this?
%		% Try to do the orbital rotation thingy
%		\item hint at it or show some examples
%		\item Maybe show comparison of convergence rates (rel. error vs. basis size)
%			in the sensible Sturmian $nlm$ basis sets vs similar Gaussian ones
%		\item Look at how the Gaussian basis sets have been constructed
%			(especially the correlation consistent and pc-n ones)
%			and compare how the Sturmians behave if similar
%			Constructions are used, also motivated from the previous section.
%	\end{itemize}
%}
