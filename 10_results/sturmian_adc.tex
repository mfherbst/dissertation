\section{Coulomb-Sturmian-based ADC calculations}
\label{sec:SturmianADC}
This section provides a first look into excited states calculations
employing Coulomb-Sturmians as the underlying basis functions.
As mentioned in section \vref{sec:MolsturmState} the modular design
of the \molsturm package allowed us to link it to a few third-party
packages employing our \python interface.
One of them is \adcman~\cite{Wormit2014},
which can be employed to perform excited states calculations
based on the algebraic diagrammatic construction scheme at
{\ADC}(1), {\ADC}(2), {\ADC}(2)-x~\cite{Schirmer1982}
and {\ADC}(3)~\cite{Trofimov1999} level.

\begin{figure}
	\centering
	\includeimage{10_results/be_adc2_vs_n}
	\caption[
		Convergence of a \CS-based {\ADC}(2) calculation of beryllium
	]{
		Convergence of a \CS-based {\ADC}(2)~\cite{Schirmer1982}
		calculation of beryllium.
		Plotted are the singlet excitation energies
		going from the ground state $2s2s$ to the denoted excited state.
		We show the results from a progression
		of \CS calculations with exponent $\kexp = 2.0$
		as well as bases sets of the form $(\nmax, 1, 1)$.
		For comparison the last two data points
		show the results from a \cGTO-based calculation
		using cc-pVTZ~\cite{Prascher2011}
		as well as the experimental values from \citet{Moore1949}.
	}
	\label{fig:SturmianAdcResults}
\end{figure}
This section now reports the first successful {\ADC}(2)
calculation employing Coulomb-Sturmians for the discretisation
on the beryllium atom.
In figure \ref{fig:SturmianAdcResults} we show the singlet excitation energies
as a progression with increasing \CS basis set size from $(4,1,1)$ to $(10,1,1)$.
For comparison the figure also indicates an equivalent calculation
using cc-pVTZ~\cite{Prascher2011} as well as the experimental values~\cite{Moore1949}.
Within the \CS basis set progression the results converge nicely from above,
where for the first few excited states we require
a maximum principle quantum number around $\nmax = 10$
to reach a decent convergence in the radial part.
This agrees with the convergence plots for the ground state,
see for example figures \vref{fig:ErrorHF_vs_nlm} and \vref{fig:ErrorHF_vs_nlm_O}.

Comparing the computed excitation energies with experiment
the $(10,1,1)$ basis set
performs worse than cc-pVTZ at the first excited state $2s2p$,
but better for the $2s3s$ and the $2s3p$ states.
This result is, however, a little misleading for three reasons.
First of all the \CS exponent of $2.0$ is a good value for describing
the ground state of beryllium, but it is certainly not an optimal value
for describing the excited states,
such that an optimisation with respect to this parameter could lead
to improvements of the \CS results.
Secondly there is some indication, that functions of angular momentum $l = 2$
are required for a proper description of the $2s2p$ excited state.
This explains the observed convergence to a \emph{higher} excitation energy
in figure \ref{fig:SturmianAdcResults},
where we used a $(\nmax, 1,1)$ progression for the calculation.
Thirdly the cc-pVTZ basis set --- even in decontracted form ---
is much smaller compared to a $(10,1,1)$ \CS basis,
such that it is fact more comparable to a $(7,1,1)$ or an $(8,1,1)$ \CS
basis in the number of basis functions.
Compared to the $(7,1,1)$ basis the cc-pVTZ in fact does better
for all three states shown,
but the $(7,1,1)$ results are not far off~(between $\unit[0.5]{eV}$
and $\unit[2]{eV}$).
Keeping in mind that we have so far not optimised
the \CS basis set for the description of the excited states
of beryllium or {\ADC}(2) as a method,
this result still looks promising.


% TODO OPTIONAL
% For potassium l_max = m_max = 2 seems to be good for HF
% Do some more calculations to check the correlation effects
%
% We want the potassium 30s atomic orbital
%\todo[inline,caption={}]{
%	\begin{itemize}
%		% TODO Can we actually do this?
%		% Try to do the orbital rotation thingy
%		\item hint at it or show some examples
%		\item Maybe show comparison of convergence rates (rel. error vs. basis size)
%			in the sensible Sturmian $nlm$ basis sets vs similar Gaussian ones
%		\item Look at how the Gaussian basis sets have been constructed
%			(especially the correlation consistent and pc-n ones)
%			and compare how the Sturmians behave if similar
%			Constructions are used, also motivated from the previous section.
%	\end{itemize}
%}
