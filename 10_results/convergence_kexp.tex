\section{The effect of the Coulomb-Sturmian exponent}
\label{sec:kexp}

In our discussion about the properties of Coulomb-Sturmian basis sets
we have  neglected the effect of the \CS exponent $\kexp$ so far.
Our main argument was that a \CS basis is complete regardless of the
value of $\kexp$,
such that for large enough \CS basis sets
the result will not depend on $\kexp$ anyway.
In practice the aim is of course to yield a sensible
discretisation of the wave function in the smallest basis possible.
For getting the best out of a particular truncated set,
the value of $\kexp$ can therefore not be chosen completely arbitrary.
This section is therefore devoted to a discussion
of some general trends regarding
the variation of obtained energies with $\kexp$.
Furthermore we will discuss some methodologies
for obtaining the optimal exponent $\kopt$
with respect to minimising the \HF energy.
These results will be presented for the
atoms of the first two periods of the periodic table
and discussed from a physical point of view.

\begin{figure}
	\centering
	\includeimage{10_results/EHF_terms_vs_k}
	\caption[
		Plot of the \HF energy contributions
		versus the Coulomb-Sturmian exponent $\kexp$.
	]{
		Plot of the \HF energy contributions
		of the beryllium
		atom versus the Coulomb-Sturmian exponent $\kexp$.
		All calculations are done in a $(5,1,1)$ \CS basis.
	}
	\label{fig:EnergyTermsVsK}
\end{figure}

In the \CS basis functions
$\kexp$ only occurs in the radial part \eqref{eqn:RadialPartCsRepeated}.
In the form of the exponential term $\exp(-\kexp r)$
it influences how quickly the basis functions decay at infinity
and in the form of the polynomial prefactor it determines
the curvature of the radial functions as they oscillate between the radial nodes.
Keeping this in mind let us consider figure \vref{fig:EnergyTermsVsK},
which shows the changes to individual energy contributions
of the \HF ground state energy as $\kexp$ is altered.
The largest changes are apparent for the nuclear attraction energy,
which decreases --- initially rather steeply --- as $\kexp$ is increased.
This can be easily understood from a physical point of view:
Since larger values of $\kexp$ imply a more rapid decay
of the basis functions,
the electron density on average stays closer to the nucleus,
which in turn leads to a lower (more negative) interaction energy
between electrons and nucleus.
The converse effect happens for smaller values of $\kexp$,
where the electron density is more expanded
and thus on average further away from the nucleus.
On the other hand the kinetic energy is related to the curvature of the wave function,
which --- as described above --- increases for larger $\kexp$.
In other words the trends of nuclear attraction energy and electronic kinetic energy
oppose each other,
with the kinetic energy being somewhat less effected.
On the scale depicted in figure \ref{fig:EnergyTermsVsK}
the variation of the electron-electron interaction,
\ie both classical Coulomb repulsion as well as the exchange interaction combined,
is much less pronounced.
Only a very minor increase with $\kexp$ can be observed.
The physical mechanism is again similar to the nuclear attraction
energy term,
namely that larger $\kexp$ compresses the wave function
and thus leads to the electrons reside more closely to another,
which increases the Coulomb repulsion between them.
The exchange interaction is effected as well,
but the changes are smaller and thus not visible.

\begin{figure}
	\centering
	\includeimage{10_results/Efci_vs_k}
	\caption[
		Plot of the \HF, {\MP}2 and \FCI energies
		versus the Coulomb-Sturmian exponent $\kexp$.
	]{
		Plot of the \HF, {\MP}2 and \FCI ground state energies
		of beryllium
		versus the Coulomb-Sturmian exponent $\kexp$.
		The optimal exponent $\kopt$ for each
		method is marked by a cross.
		All calculations are done in a $(5,1,1)$ \CS basis.
	}
	\label{fig:FCI_vs_k}
\end{figure}
Summing up all energy contributions leads to the blue curve
in figure \vref{fig:FCI_vs_k},
which shows the total Hartree-Fock energy versus
the Coulomb-Sturmian exponent $\kexp$.
From our discussion of the individual terms
it is apparent that at small values for $\kexp$
the increase in nuclear attraction energy dominates,
such that the \HF energy increases rapidly.
At large distances the kinetic energy and electron-electron
interaction terms win, such that a convex curve
for the plot $E_\text{HF}$ versus $\kexp$ results.
Adding correlation effects by a treatment of the atom at {\MP}2 or \FCI level,
does not change this overall behaviour much.
Up to a large extend the curves are just shifted downwards by the
correlation energy term.
The shift is, however, not completely uniform.
This can be seen if we consider the optimal \CS exponent $\kopt$,
which is denoted by a cross in each of the plots of figure \ref{fig:FCI_vs_k}.
This exponent of minimal energy shifts
to slightly larger values going from \HF to {\MP}2 and finally to \FCI
indicating that the amount of correlation energy is somewhat larger
at exponents slightly above $\kopt$ for Hartree-Fock.
Notice that $\kopt$ not only depends on the method used for modelling
a particular state,
but it well depends on the state as well.
For example for modelling the first $T1$ excited state of beryllium
a smaller value for $\kopt$ is obtained than the \FCI $\kopt$
of the depicted $S0$ ground state.

\begin{figure}
	\centering
	\includeimage{C_ene_gs_vs_k}
	\caption[
		Dependency of the \HF and {\MP}2 energies
		on the \CS exponent $\kexp$
		and the basis set
	]{
		Plot of the unrestricted
		\HF and {\MP}2 ground state energies of carbon
		versus the Coulomb-Sturmian exponent $\kexp$
		in the $(4,2,2)$, $(5,2,2)$ and $(6,2,2)$ basis sets.
		The optimal exponent $\kopt$ at \HF level
		for each basis set is marked by a cross.
	}
	\label{fig:EHF_vs_k}
\end{figure}

Since $\kexp$ only occurs in the radial part of the \CS basis functions
the effect of its variation depends on the
maximal principle quantum number $\nmax$ of the basis set.
As larger and larger values of $\nmax$ are used,
the discretisation of the radial part of the wave function
becomes more and more complete,
such that the choice of $\kexp$ in turn becomes less important.
Figure \vref{fig:EHF_vs_k} shows this for the ground state energy of the carbon atom
at unrestricted \HF and {\MP}2 level.
Whilst a $(4,2,2)$ \CS basis reproduces largely the
shape of the plots in figure \ref{fig:EHF_vs_k},
for $(5,2,2)$ and $(6,2,2)$ the energy versus exponent curves
become visibly flatter close to the optimal exponent (around $\kexp = 2.8$).
The influence of increasing $\nmax$ is not the same
for all values of $\kexp$.
Instead the curves seem to bend down in the range $\kexp > 3$.
In other words choosing a \CS exponent larger than $\kopt$
will generally speaking lead to a smaller error
than choosing a too large exponent%
\footnote{We already noted this aspect in the context of discussing
	the local energy plots in section \vref{sec:BasisCS}.}.
Another conclusion we can draw from figure \ref{EHF_vs_k}
is that the optimal value for the exponent $\kopt$
depends on $\nmax$ as well
as larger basis sets give rise to smaller values for $\kopt$.
We can rationalise by taking the plots of the energy terms
in figure \vref{fig:EnergyTermsVsK} into account.
We already noticed above that the nuclear attraction energy
is influenced by $\kexp$ most strongly.
Additionally it is (by magnitude) the largest contribution to the \HF energy.
In order to yield the minimal ground state energy
in a small basis the dominating effect is therefore to
minimise the nuclear attraction energy as much as possible.
As a result the optimal exponent $\kopt$ takes comparatively large values.
As the basis becomes larger a balanced description
of the complete physics becomes possible,
such that the electron repulsion and kinetic energy terms
are described better as well
and thus smaller values for $\kopt$ result.

Due to the structure of the energy versus exponent curves,
like the ones shown in figure \vref{fig:EHF_vs_k},
one hardly ever needs to know $\kopt$ very accurately.
As long as one uses a reasonable guess,
which is constructed to overestimate $\kopt$
rather than underestimate it, one is usually safe.
If a highly accurate treatment of a particular system is required,
then increasing $\nmax$ has both a much larger effect
and is computationally cheaper than finding the optimal exponent in the smaller basis.
See the next section for details.
% overestimating plays in line with correlated treatment,
% where $\kopt$ becomes larger due to the correlation.

% TODO OPTIONAL
%\begin{figure}
%	\centering
%	\includeimage{C_hfterms_vs_k}
%	\caption{HF energy terms vs $\kexp$ for different basis sets}
%	\label{fig:Eterms_vs_k}
%\end{figure}
%
%In figure \vref{fig:Eterms_vs_k} we observe the same trends
%as in figure \vref{fig:EnergyTermsVsK} for beryllium
%just with the difference that
%the result is shown for different basis sets.
%We observe that the trends get smaller if larger basis sets are chosen.


% TODO OPTIONAL
%\begin{figure}
%	\centering
%	\includeimage{C_orben_622_vs_k}
%	\caption{E orbital vs k, only alpha orbitals are plotted of 622}
%	\label{fig:orben_vs_k}
%\end{figure}
%
%The interpretation of the orbital energy vs $\kopt$ curve
%should be done with care.
%For this note that the minimal sum of the orbital energies
%occurs at a much smaller value for $\kopt$,
%since the electron-electron interactions are counted twice
%hence shifting the optimal value for the sum of the orbital
%energies more towards smaller $\kexp$.
%
%Nevertheless it allows us to qualitatively assess the realisation
%that for the description of the virtual or valence states
%smaller $\kopt$ values are preferable whilst for the core states
%larger $\kopt$ are typically preferred.
%Again since the core-like orbitals have a bigger contribution
%to the energy typically small basis sets tend to try to represent
%these in a good fashion first, hence tending to larger values of $\kopt$.
%Whilst the $\kopt$ decreases as the basis gets larger up to the point
%where its precise value becomes less and less important.


%
% ----------------------------------------------------------
%
\subsection{Determining the optimal exponent $\kopt$}
\todoil{TODO}

% need reasonable basis set such that errors in the angular part
% are not too large, other than that angular part is not important.

Due to the convexity of the curve energy-vs-k (see \vref{fig:FCI_vs_k})
and the fact that we are dealing with a scalar function,
the optimisation of the energy wrt. $k$ can be performed by
Brent's method~\cite{Brent1972},
which does not require energy derivatives.

For \CS basis sets the core hamiltonian guess is in many cases not very good
in the sense that it can give rise to the wrong SCF minimum.
A good example to reproduce this is beryllium with a $(3, 2, 2)$ \CS basis
set and $\kexp = 1.9$.
Note that this exponent is only $0.1$ from the optimal one,
but nevertheless hcore fails.
This behaviour is retained for $(4,2,2)$ and $(5,2,2)$ basis sets,
which are already quite reasonable for beryllium.

In a similar manor especially for the atoms of the third
period the SCF stationary point obtained with \molsturm
is very much dependent on the values of $\kexp$ and $(n,l,m)$.
If we get humps on the surface because we fall from one SCF
minimum to another,
the optimisation for $\kexp$ typically produces erroneous results.

In other words the key starting point for finding $\kopt$
is to obtain a reliable guess for the atom.
Our approach is to start 5 SCF calculations with the
selected \CS basis set at two values of $\kexp$:
One supplied by the user and one guessed from the \HF orbital
energies of the first calculation.
Out of these randomly initialised SCF calculations,
the lowest-energy version is taken as the guess
for all SCF calculations performed during the search
of Brent's method.
Typically this converges within 10 iteration steps
to $\kopt$.

The heuristic for finding a reasonable value for
$\kopt$ from \HF energies follows the idea that the
energy of an orbital is related to its decay at infinite distance
according to \citet{Kato1951}.
In other words, since this is the case for sturmians, too,
we get that the energy of an orbital is related to its optimal
exponent by
\[
	\varepsilon_i = -\frac{k_i^2}{2}
\]
thus
\[
	k_i = \sqrt{-2 \varepsilon_i}
\]
and since we have many of these $\varepsilon$:
\[
	\kopt \simeq \sqrt{\frac{-2}{\Nelec} \sum_i \varepsilon_i}
\]

This guess method is implemented in \molsturm in the
module \texttt{molsturm.scf\_guess}
as the function \texttt{best\_of\_n} and the method
for finding the optimal value of $\kexp$ is implemented
as \texttt{molsturm.sturmian.cs.find\_kopt}

% Mention builtin estimation function based on linear fits.


Optimal values for second and third period
see table \vref{tab:Kopt2} and table \vref{tab:Kopt3}
\include{table_kopt}


%
% ----------------------------------------------------------
%
\subsection{Interpretation of the optimal exponent}
\todoil{TODO}
\label{sec:ValuesKopt}


% Introduce relationship to shielding and slaters / clementis work
% establish: effective nuclear charge increases across the period
% due to averaging trends become less apparent as we move to larger period numbers

Figure \vref{fig:KoptVsAtnum} shows the optimal Coulomb-Sturmian exponents
for the neutral atoms across the second and the third period of the periodic table.
As can be seen the values for $\kopt$ follow a rough linear increase
as we go from the alkaline metals to the noble gases,
where the slope of the increase is steeper along the second period.
Note, however, that a deviation from this linear trend
is very much apparent in the third period.

\citet{Slater1930} introduces the concept of a nuclear shielding parameter $\sigma$
in order to construct simple analytic expressions of the form
\begin{equation}
	\chi_{n^\ast, \sigma} = r^{n^\ast - 1} \exp\left( - \frac{(Z - \sigma)r}{n^\ast} \right)
	\equiv r^{n^\ast - 1} \exp\left( -\zeta r\right)
	\label{eqn:SlaterShielding}
\end{equation}
for the atomic orbitals.
\citet{Clementi1963} later determined optimal values for the Slater exponents $\zeta$
by performing \HF calculations,
variationally optimising to yield the lowest energy.
We will refer to these exponents as $\zeta_\text{Clementi}$ in this section.

Comparing \eqref{eqn:SlaterShielding} and \eqref{eqn:RadialPartCsRepeated}
we easily note that $\kexp$ more or less plays the role of $\zeta$
with the subtle difference that for \CS basis sets all functions
need to carry the same exponent.
Nevertheless the procedure we follow to find $\kopt$ and the procedure
\citeauthor{Clementi1963} followed to find $\zeta_\text{Clementi}$
bears a lot of resemblance.

As a result it is not surprising that the behaviour of $\kopt$
along the periodic table is related to $\zeta_\text{Clementi}$ as well.
In fact it kind of forms a middle ground between the $\zeta_\text{Clementi}$
value of the HO, the highest occupied orbital,
and the occupation-averaged value for $\zeta_\text{Clementi}$
when it comes to the trends observed.
Naturally its actual value is closer to the occupation-average
of $\zeta_\text{Clementi}$,
since we are forced to a single $\kexp$ value.
Therefore the optimal exponent $\kopt$ bears resemblance of both
the occupation-averaged value of $Z_\text{eff} / n^\ast$
as well as the trends of $Z_\text{eff} / n^\ast$ of highest occupied
atomic orbitals.

Note that by Slater's original definition,
we can identify the relationship
\[ Z_\text{eff} = n^\ast \zeta_\text{Clementi} \]
between the effective nuclear charge $Z_\text{eff}$ and
the optimal exponents $\zeta_\text{Clementi}$~\cite{Slater1930},
where $n^\ast$ is a function of the quantum number $n$ of the orbital
under consideration.
By means of a similar rationale to what has been sketched above,
we can therefore find a chemical interpretation for $\kopt$,
namely that it gives an understanding of the average
charge which is felt by the orbitals of a particular atom
or in other words the average shielding which takes place in the atom.
\todoil{Can one state it like that?}

\begin{figure}
	\centering
	\includeimage{10_results/kopt_vs_atnum}
	\caption[
		Plot atomic number versus the optimal Coulomb-Sturmian exponent
	]{
		Plot of the atomic number versus the optimal Coulomb-Sturmian exponent
		$\kopt$ for the neutral atoms of the second and the third period.
		For comparison the occupation-averaged value of the \citet{Clementi1963} optimal
		Slater exponent $\zeta_\text{Clementi}$ as well as the
		value $\zeta_\text{Clementi}$ for the highest occupied orbital~(HO)
		are shown.
	}
	\label{fig:KoptVsAtnum}
\end{figure}

Note that the slope of the graph $\kopt$ vs atomic number in the third period
is much more flat than the equivalent graph in the second period.
We can understand that by considering once again
that $\kopt$ gives an average over all orbitals.
Since especially the low-energy orbitals close to the nucleus
are shielded very little,
their value of $\kopt$ increases constantly along the periodic table.
The $\kopt$ for the valence orbitals, however, changes much more.
Especially at the beginning of the period, where atomic radii are large,
the $\kopt$ for the valence orbitals are small,
whilst this changes as the atoms get smaller across the period
and hence $\zeta_\text{Clementi}$ and $\kopt$ get larger.
This effect is diminished, however,
by the constant $\kopt$ increase of the
much larger number of core-like orbitals
and thus the rate of change of $\kopt$ across a period gets smaller,
the slope thus smaller as well.
Furthermore this implies that the rapid $\kopt$ change
going from one period to the next (e.g. Ne to Na)
will get smaller and smaller as well
as the effect of the valence-like orbitals gets averaged out.

An alternative way of viewing this is
that as a new shell becomes occupied it requires
the representation of an extra (more expanded) orbital.
So initially $\kexp$ gets reduced.
As we move across the period the nuclear charge increases
and since the shielding abilities of electrons in the same shell
is not that strong the orbitals become more contracted and occupy less space
thus $\kexp$ increases again.
As we move to larger periods this effect of change is averaged out
again by the more constant core-like orbitals.

Yet a third line of argument is the importance of nuclear attraction
vs. kinetic energy.
As the effective charge increases across the period
nuclear attraction becomes more and more important leading to larger values of $k$
(which support a larger (negative) magnitude of nuclear attraction energy)
and kinetic energy (which would favour smaller $k$) becomes less important.


