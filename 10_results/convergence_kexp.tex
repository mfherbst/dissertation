\section{The effect of the Coulomb-Sturmian exponent}
\todoil{TODO}
\label{sec:kexp}
discuss $\kexp$
For the remaining discussion in this sections
we have limited ourselves to rather small basis sets,
that is we typically set $\lmax$ and $\mmax$ to values
of 2 or less and $\nmax$ not larger than 6
to save computational time.
This should be a reasonable compromise,
but further investigation has to show this.

As can be seen in figure \vref{fig:EnergyTermsVsK}
Smaller values of $\kexp$ lead to a slower decay of the orbitals,
hence the electrons are forced to reside at a more
distant place to the nucleus on average,
which in turn leads to a rise in nuclear attraction energy.
On the other hand larger $\kexp$ lead to more compressed orbitals,
which drives up the repulsive coulomb interaction as well as the kinetic energy.

\begin{figure}
	\centering
	\includeimage{10_results/EHF_terms_vs_k}
	\caption{Terms of the \HF calculation vs $\kexp$ for Be with a $(5,1,1)$ \CS basis}
	\label{fig:EnergyTermsVsK}
\end{figure}
% TODO Electron-electron interaction mostly agnostic to k, a little lower with smaller k (i.e. larger orbitals)

The net result for \HF as well as MP2 and FCI is shown in figure \vref{fig:FCI_vs_k}.
One easily notices that the dependency of the energy wrt. $\kexp$ is a
convex curve.
Notice, that the optimal exponent $\kexp$ for the FCI S1 ground state
is larger than the optimal exponent $\kexp$ for the description
of the HF ground state.
MP2 is in between but much closer to HF.

\begin{figure}
	\centering
	\includeimage{10_results/Efci_vs_k}
	\caption{Energies vs $\kexp$ for Be with a $(5,1,1)$ \CS basis}
	\label{fig:FCI_vs_k}
\end{figure}

The optimal value for $\kexp$ is very much dependent on the
precise state.
For the T1 state for example, the optimal $\kexp$
is around the value for HF.

As can be seen in figure \vref{fig:EHF_vs_k}
for carbon,
$\kopt$ depends on the choice of the basis as well.
Generally speaking larger basis sets will favour
smaller values of $\kopt$.
This can be explained by the fact that \CS basis sets
all share the same exponent.
Since the largest contribution of the \HF energy
originates from the nuclear attraction energy,
small basis sets will overemphasize
the nuclear attraction interaction,
i.e. they will tend to larger values for $\kopt$
in order to minimise the nuclear attraction energy.
As the basis becomes larger,
the extra basis functions allow for a balanced
description of both the nuclear attraction
as well as the electron repulsion
and kinetic energy terms,
which therefore allows to choose a smaller $\kopt$.

\begin{figure}
	\centering
	\includeimage{C_ene_gs_vs_k}
	\caption{\HF energy vs $\kexp$ for a couple of basis sets and carbon.}
	% Maybe do not have the minimum annotated in the graph but only have in
	% the text instead
	\label{fig:EHF_vs_k}
\end{figure}
Figure \vref{fig:EHF_vs_k} shows the change in the curvature
of the energy vs. $\kexp$ graphs as the basis set size is increased.
Generally speaking the curves become flatter and both the values of the
\HF as well as the MP2 energies become more agnostic to the precise value of $\kexp$.
Furthermore the optimal value for $\kexp$ shifts towards smaller values.
This can be explained by the fact that with larger basis sets
the basis has more freedom to represent both the core and the valence-like orbitals.
Since the core-like orbitals, which are more contracted (larger $\kexp$, contribute
larger to this energy, the small basis will choose a larger $\kexp$.
Equivalently one can consider the dependence of the nuclear attraction
energy wrt. $\kexp$.

Another fact worth mentioning is that the MP2 correlation energy typically
is much less dependant on too large values of $\kexp$ than too small values
of $\kexp$.
So for a correlated treatment of the carbon atom a too large value for $\kexp$
is better than a too small value.
In agreement with this recommendation is the realisation that
MP2 calculations tend to have a larger $\kopt$ value, hence tending a bit more
towards the FCI value for $\kopt$.
Therefore if the \HF value for $\kexp$ is choosen larger than the optimum,
the correlated treatment typically even benefits from this a bit.


\begin{figure}
	\centering
	\includeimage{C_hfterms_vs_k}
	\caption{HF energy terms vs $\kexp$ for different basis sets}
	\label{fig:Eterms_vs_k}
\end{figure}

In figure \vref{fig:Eterms_vs_k} we observe the same trends
as in figure \vref{fig:EnergyTermsVsK} for beryllium
just with the difference that
the result is shown for different basis sets.
We observe that the trends get smaller if larger basis sets are chosen.


% TODO OPTIONAL
%\begin{figure}
%	\centering
%	\includeimage{C_orben_622_vs_k}
%	\caption{E orbital vs k, only alpha orbitals are plotted of 622}
%	\label{fig:orben_vs_k}
%\end{figure}

The interpretation of the orbital energy vs $\kopt$ curve
should be done with care.
For this note that the minimal sum of the orbital energies
occurs at a much smaller value for $\kopt$,
since the electron-electron interactions are counted twice
hence shifting the optimal value for the sum of the orbital
energies more towards smaller $\kexp$.

Nevertheless it allows us to qualitatively assess the realisation
that for the description of the virtual or valence states
smaller $\kopt$ values are preferable wlist for the core states
larger $\kopt$ are typically preferred.
Again since the core-like orbitals have a bigger contribution
to the energy typically small basis sets tend to try to represent
these in a good fashion first, hence tending to larger values of $\kopt$.
Whilst the $\kopt$ decreases as the basis gets larger up to the point
where its precise value becomes less and less important.


\todo[inline,caption={}]{
	\begin{itemize}
		\item Optimal k for HF and MP2 is similar
		\item For FCI a bit larger
		\item For MP2 grows
		\item With larger basis set optimal k gets smaller
		\item optimal k for orbitals smaller than for HF energy in general
		\item Core-like states favour larger k
	\end{itemize}
}


\subsection{Determining the optimal exponent $\kopt$}
Due to the convexity of the curve energy-vs-k (see \vref{fig:FCI_vs_k})
and the fact that we are dealing with a scalar function,
the optimisation of the energy wrt. $k$ can be performed by
Brent's method~\cite{Brent1972},
which does not require energy derivatives.

For \CS basis sets the core hamiltonian guess is in many cases not very good
in the sense that it can give rise to the wrong SCF minimum.
A good example to reproduce this is beryllium with a $(3, 2, 2)$ \CS basis
set and $\kexp = 1.9$.
Note that this exponent is only $0.1$ from the optimal one,
but nevertheless hcore fails.
This behaviour is retained for $(4,2,2)$ and $(5,2,2)$ basis sets,
which are already quite reasonable for beryllium.

In a similar manor especially for the atoms of the third
period the SCF stationary point obtained with \molsturm
is very much dependent on the values of $\kexp$ and $(n,l,m)$.
If we get humps on the surface because we fall from one SCF
minimum to another,
the optimisation for $\kexp$ typically produces erroneous results.

In other words the key starting point for finding $\kopt$
is to obtain a reliable guess for the atom.
Our approach is to start 5 SCF calculations with the
selected \CS basis set at two values of $\kexp$:
One supplied by the user and one guessed from the \HF orbital
energies of the first calculation.
Out of these randomly initialised SCF calculations,
the lowest-energy version is taken as the guess
for all SCF calculations performed during the search
of Brent's method.
Typically this converges within 10 iteration steps
to $\kopt$.

The heuristic for finding a reasonable value for
$\kopt$ from \HF energies follows the idea that the
energy of an orbital is related to its decay at infinite distance
according to Kato
\todo{cite}.
In other words, since this is the case for sturmians, too,
we get that the energy of an orbital is related to its optimal
exponent by
\[
	\varepsilon_i = -\frac{k_i^2}{2}
\]
thus
\[
	k_i = \sqrt{-2 \varepsilon_i}
\]
and since we have many of these $\varepsilon$:
\[
	\kopt \simeq \sqrt{\frac{-2}{\Nelec} \sum_i \varepsilon_i}
\]

This guess method is implemented in \molsturm in the
module \texttt{molsturm.scf\_guess}
as the function \texttt{best\_of\_n} and the method
for finding the optimal value of $\kexp$ is implemented
as \texttt{molsturm.sturmian.cs.find\_kopt}

\subsection{Discussion of the values for $\kopt$}
\label{sec:ValuesKopt}
% Introduce relationship to shielding and slaters / clementis work
% establish: effective nuclear charge increases across the period
% due to averaging trends become less apparent as we move to larger period numbers

Figure \vref{fig:KoptVsAtnum} shows the optimal Coulomb-Sturmian exponents
for the neutral atoms across the second and the third period of the periodic table.
As can be seen the values for $\kopt$ follow a rough linear increase
as we go from the alkaline metals to the noble gases,
where the slope of the increase is steeper along the second period.
Note, however, that a deviation from this linear trend
is very much apparent in the third period.

\citet{Slater1930} introduces the concept of a nuclear shielding parameter $\sigma$
in order to construct simple analytic expressions of the form
\begin{equation}
	\chi_{n^\ast, \sigma} = r^{n^\ast - 1} \exp\left( - \frac{(Z - \sigma)r}{n^\ast} \right)
	\equiv r^{n^\ast - 1} \exp\left( -\zeta r\right)
	\label{eqn:SlaterShielding}
\end{equation}
for the atomic orbitals.
\citet{Clementi1963} later determined optimal values for the Slater exponents $\zeta$
by performing \HF calculations,
variationally optimising to yield the lowest energy.
We will refer to these exponents as $\zeta_\text{Clementi}$ in this section.

Comparing \eqref{eqn:SlaterShielding} and \eqref{eqn:FormCSBasisFunction}
we easily note that $\kexp$ more or less plays the role of $\zeta$
with the subtle difference that for \CS basis sets all functions
need to carry the same exponent.
Nevertheless the procedure we follow to find $\kopt$ and the procedure
\citeauthor{Clementi1963} followed to find $\zeta_\text{Clementi}$
bears a lot of resemblance.

As a result it is not surprising that the behaviour of $\kopt$
along the periodic table is related to $\zeta_\text{Clementi}$ as well.
In fact it kind of forms a middle ground between the $\zeta_\text{Clementi}$
value of the HOAO, the highest occupied atomic orbital,
and the occupation-averaged value for $\zeta_\text{Clementi}$
when it comes to the trends observed.
Naturally its actual value is closer to the occupation-average
of $\zeta_\text{Clementi}$,
since we are forced to a single $\kexp$ value.
Therefore the optimal exponent $\kopt$ bears resemblance of both
the occupation-averaged value of $Z_\text{eff} / n^\ast$
as well as the trends of $Z_\text{eff} / n^\ast$ of highest occupied
atomic orbitals.

Note that by Slater's original definition,
we can identify the relationship
\[ Z_\text{eff} = n^\ast \zeta_\text{Clementi} \]
between the effective nuclear charge $Z_\text{eff}$ and
the optimal exponents $\zeta_\text{Clementi}$~\cite{Slater1930},
where $n^\ast$ is a function of the quantum number $n$ of the orbital
under consideration.
By means of a similar rationale to what has been sketched above,
we can therefore find a chemical interpretation for $\kopt$,
namely that it gives an understanding of the average
charge which is felt by the orbitals of a particular atom
or in other words the average shielding which takes place in the atom.
\todoil{Can one state it like that?}

\begin{figure}
	\centering
	\includeimage{10_results/kopt_vs_atnum}
	\caption{Plot of the atomic number versus the optimal Coulomb-Sturmian exponent
		$\kopt$ for the neutral atoms of the second and the third period.
		For comparison the occupation-averaged value of the \citet{Clementi1963} optimal
		Slater exponent $\zeta_\text{Clementi}$ as well as the
		value $\zeta_\text{Clementi}$ for the highest occupied atomic orbital~(HOAO)
		are shown.
	}
	\label{fig:KoptVsAtnum}
\end{figure}

Note that the slope of the graph $\kopt$ vs atomic number in the third period
is much more flat than the equivalent graph in the second period.
We can understand that by considering once again
that $\kopt$ gives an average over all orbitals.
Since especially the low-energy orbitals close to the nucleus
are shielded very little,
their value of $\kopt$ increases constantly along the periodic table.
The $\kopt$ for the valence orbitals, however, changes much more.
Especially at the beginning of the period, where atomic radii are large,
the $\kopt$ for the valence orbitals are small,
whilst this changes as the atoms get smaller across the period
and hence $\zeta_\text{Clementi}$ and $\kopt$ get larger.
This effect is diminished, however,
by the constant $\kopt$ increase of the
much larger number of core-like orbitals
and thus the rate of change of $\kopt$ across a period gets smaller,
the slope thus smaller as well.
Furthermore this implies that the rapid $\kopt$ change
going from one period to the next (e.g. Ne to Na)
will get smaller and smaller as well
as the effect of the valence-like orbitals gets averaged out.

An alternative way of viewing this is
that as a new shell becomes occupied it requires
the representation of an extra (more expanded) orbital.
So initially $\kexp$ gets reduced.
As we move across the period the nuclear charge increases
and since the shielding abilities of electrons in the same shell
is not that strong the orbitals become more contracted and occupy less space
thus $\kexp$ increases again.
As we move to larger periods this effect of change is averaged out
again by the more constant core-like orbitals.

Yet a third line of argument is the importance of nuclear attraction
vs. kinetic energy.
As the effective charge increases across the period
nuclear attraction becomes more and more important leading to larger values of $k$
(which support a larger (negative) magnitude of nuclear attraction energy)
and kinetic energy (which would favour smaller $k$) becomes less important.




\include{table_kopt}
See table \vref{tab:Kopt2} and table \vref{tab:Kopt3}
