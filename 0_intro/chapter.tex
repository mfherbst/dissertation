\chapter{Introduction}
\chapquote{%
Humanity needs practical men,
who get the most out of their work, and,
without forgetting the general good,
safeguard their own interests.
But humanity also needs dreamers,
for whom the disinterested development of an enterprise is
so captivating that it becomes impossible for them to devote
their care to their own material profit.}
{Marie Skłodowska Curie~(1867--1934)}

\noindent
Compared to experimental chemistry,
which has been performed for thousands of years,
quantum chemistry is a relatively new field.
It emerged only about 100 years ago from the development of modern physics
at the turn of the 20th century.
From a physical point of view the fundamental equations of quantum-chemistry,
namely the mathematical formulation of quantum mechanics,
have been known since the late 1920s
--- as noted famously by \citet{Dirac1929} in \citeyear{Dirac1929}.
Since solving these equations exactly is impossible except for model systems
and one thus has to live with a hierarchy of approximate treatments in practice.
Over the years sufficiently good numerical methods have been developed,
such that theoretical modelling and simulations
based on quantum-chemical arguments
are nowadays well-established both in industry as well as research.

This thesis will focus on a subfield of quantum chemistry,
namely electronic structure theory,
where the aim is to approximately solve the
electronic Schrödinger equation in order to study
the quantum-mechanical behaviour of electrons in matter.
One way to tackle this is to first solve
a very basic approximation to the Schrödinger equation,
namely the Hartree-Fock problem,
and afterwards build improved solutions on top.
The underlying spaces in which this procedure in principle
has to take place is infinite dimensional,
which is an insurmountable obstacle
if this tasks should be done numerically,
\ie in a computer with only finite amounts of memory.
In practice one therefore restricts the problem size
by only performing the aforementioned procedure
in a subspace spanned by a finite number of basis functions.
This restriction is called discretisation
and has the pleasant side effect that the physical equations
reduce to standard problems of linear algebra,
usually diagonalisation or linear systems.

This naturally leads to the question,
which type of basis functions should be used to span the subspace.
Most available program packages for
quantum-chemical calculations nowadays employ methodologies based on
the linear combination of Gaussian-shaped atomic orbital functions.
This predominance can be rationalised
by the pragmatic historic developments
taking place in the founding years of modern electronic structure theory.
\citet{Boys1950} realised in \citeyear{Boys1950}
that evaluating the integrals
required for solving Hartree-Fock
is much more feasible for Gaussian-type orbitals
compared to the physically more adequate Slater-type orbitals~\cite{Slater1930}.
This idea was picked up and refined later~\cite{Hehre1969}
and eventually set off many developments
both in terms of efficient algorithms as well as methodologies
centred around Gaussian basis functions,
making them widely applicable for simulating electronic structures.

Unfortunately Gaussian-type orbitals have a major downside
caused by their unphysical shape:
They are not able to represent
all features of the electron density well.
Most prominently they fail to describe the cusp of the electron density
at the position of a nucleus as well as its exponential decay behaviour~\cite{Kato1957}.
Whilst in most practical use cases one is able to compensate
by employing specialised Gaussian basis sets~\cite{Jensen2013,Hill2013},
there are strong hints that the accurate computation of some physical properties
like NMR shielding tensors~\cite{Guell2008,Hoggan2009}
requires a correct description of both aforementioned features.
Additionally those specialised basis sets can lead to
numerical instabilities in the resulting linear algebra.

Multiple research groups have therefore looked into alternative types of basis functions.
Examples are plane-waves or projector-augmented wave methods%
~\cite{Kresse1996,Kresse1999,Mortensen2005,Enkovaara2010} or
so-called numerical basis functions~\cite{Frediani2015}
like wavelets%
~\cite{Bischoff2011,Bischoff2012,Bischoff2013,Bischoff2014,Bischoff2014a,Bischoff2017}
or finite elements%
~\cite{Tsuchida1995,Soler2002,Lehtovaara2009,Alizadegan2010,Avery2011PhD,Davydov2015,Boffi2016}.
Furthermore there are Sturmian-type functions
like Coulomb-Sturmians~\cite{Shull1959,Rotenberg1962,Rotenberg1970,Gruzdev1990,Hoggan2009,Randazzo2010}
or generalised Sturmians~\cite{Avery2006,Avery2011PhD,Avery2011,Morales2016,Avery2017,Randazzo2015,Granados2016},
which are related to Slater-type orbitals and amount to correctly represent
the physical electron density,
making them a promising alternative.

\todoil{From here}

The implementation of quantum-chemical methods
employing novel types of basis functions often necessitates
unusual numerical techniques as well.


testing and evaluation of novel basis function types
Implementing these into existing quantum chemistry packages
can be rather challenging,
since these are on the one hand highly optimised towards the methods
they already accommodate,
but on the other hand typically not flexible enough
to meet the differing requirements.

\begin{figure}
	\centering
	\includeimage[width=\textwidth]{4_solving_hf/fock_fe_gauss_sturm}  \\[-1.3em]
	{\smaller
	\begin{tabular}{lc@{\hspace{14pt}}ccl}
		\hspace{0.017\textwidth} &
		\hspace{0.25\textwidth} & \hspace{0.27\textwidth} & \hspace{0.25\textwidth} &
		\hspace{0.1\textwidth} \\
		&$Q_2$ finite elements,     & contracted Gaussians, & Coulomb-Sturmians \\
		&adaptively refined 3D grid & pc-2~\cite{Jensen2007} basis set 
		& $(5,1,1)$ basis with $k=2.0$
	\end{tabular}
	} \\[-0.3em]
	\caption[Structure of the Fock matrix varying for different discretisations.]{
		Varying structure of the Fock matrix
		in an early stage of a Hartree-Fock self-consistent field
		calculation for beryllium
		when discretisations using
		finite elements, contracted Gaussians
		or Coulomb-Sturmians are employed.
	}
	\label{fig:IntroFockStructure}
\end{figure}


To illustrate this latter point let us consider
the usual ansatz of solving the Hartree-Fock~(HF) equations
in a self-consistent field~(SCF) procedure.
Fig.~\ref{fig:FockStructure} shows the structure of the Fock matrix
arising if various basis function types are employed for the discretisation.
% write about different structure

One should mention that the finite-element discretisation depicted here
is still far too inaccurate to yield a sensible description
of the Beryllium atom density.
A realistic description would need in the order of $10^5$ to $10^6$ basis
functions, see \eg reference \cite{Davydov2015}.

a good program supports all of these structures
aim of \molsturm
\contraction-based ansatz helps by providing generality
thus implement \lazyten



In many cases \contraction-based approaches
have more advantages than saving memory.
For both finite elements as well as for Coulomb-Sturmians
for example a \contraction-based ansatz
allows to reformulate the matrix-vector product in a way
that the computational complexity is reduced.
An important point to note is that
a \contraction-based diagonalisation algorithm
does not prevent one from building the matrix in memory still.
As such it is not only favourable for the aforementioned
types of functions,
but it furthermore leads to no disadvantages
for diagonalising a Fock matrix originating from
a contracted Gaussian basis set.

Considering the flexibility which results from such approaches,
they are a perfect ingredient for
a quantum-chemical method development framework
supporting the numerical requirements of various types of basis functions.
Such a framework has been missing up to today to the best of our knowledge.

The main aim of this thesis was to develop a quantum-chemistry
package, which supports such research directions
by offering a common platform for developing, testing and analysing
quantum-chemical methods irrespective of the basis function type employed for
the discretisation.
This is the package \molsturm,
which will be presented in chapter \vref{ch:Molsturm}.




cs as test canidate
we only deal with tiny subaspect:
- single-molecule or single-atom setting.
- treatment in vacuum and computation of the ground state and a few excited states
- just small building block, which hopefully moves field forward 


% ---------------------------------------

A fascinating aspect of quantum chemistry is the interdisciplinarity of the field.
Developing a good quantum-chemical model for a particular aspect of chemistry
firstly requires both chemical as well as physical intuition.
But in many cases the arising problems are numerically challenging
and for solving them in an efficient and reliable manner
one cannot avoid taking a deeper look into the mathematical structures
as well as modern concepts from high-performance computing,
hardware and operating systems design.
Embedded in this interdisciplinary setting,
this thesis will develop the mathematical language of quantum mechanics
and apply it to a subfield of quantum chemistry,
namely electronic structure theory.
The overall aim is the development of novel computational methods
for the modelling of electronic properties of matter.

% ---------------------------------------




The reader is assumed to have some background in classical physics
and a solid idea about basic analysis and algebraic structures.
The other concepts will be revised and developed in the first chapter.

The thesis consists roughly of three parts
chapter 1 and chapter 2 mathematical and numerical background
Beauty of maths in science

chapter 3 quantum-chemical methods
chapter 4 numerical approaches for HF, discretisations and basis function types
chapter 5 contraction-based methods
chapter 6 design of the basis-function independent program molsturm
chapter 7 provides convergence results for Coulomb-Sturmians
chapter 8 concludes and gives an outlook
