\chapter{Introduction}
\chapquote{%
Humanity needs practical men,
who get the most out of their work, and,
without forgetting the general good,
safeguard their own interests.
But humanity also needs dreamers,
for whom the disinterested development of an enterprise is
so captivating that it becomes impossible for them to devote
their care to their own material profit.}
{Marie Skłodowska Curie~(1867--1934)}

\noindent
Compared to experimental chemistry, which has been
performed for thousands of years in the form of alchemy
and as a science at least since the 18th century,
quantum chemistry is a relatively new field.
It emerged only about 100 years ago from the development of modern physics.
Still, as \citet{Dirac1929} already noted famously in \citeyear{Dirac1929},
all fundamental equations of quantum chemistry,
in the form of the mathematical formulation of quantum mechanics,
are known.
Whist solving these equations exactly is possible for model systems,
solving them for the general case of arbitrary
chemical systems can only be done using approximate methods.
Which are best to be employed depends very much on
the intended application,
\ie the complexity of the chemical system
or the properties and behaviour to be simulated.
Over the years a hierarchy of approximations has been developed
for this reason,
ranging from fast and crude on the one hand to slow and numerically exact on the other.
Nowadays the modelling of chemical processes
based on quantum-chemical arguments
is well-established both in industry as well as research.

A chemical system is typically modelled at various levels,
each improving the accuracy on top of the previous,
thus building gradually towards the full picture.
In electronic structure theory, for example,
one neglects the motion of the nuclei
in order to yield a simplified equation --- the electronic Schrödinger equation.
In turn a usual approach to this equation is to first
solve the approximate Hartree-Fock problem
as a starting point and amend this solution in a second step.
Often not every level can be reached due to the increasing
difficulty of the equations,
but as mentioned before,
one may not need the highest level of theory in order to gain
insight into a particular research question.
For example electronic structure theory already allows to model
many aspects of chemical reactivity or spectroscopy
even though the nuclei are assumed to be motionless.

Since the underlying spaces of quantum mechanics are infinite dimensional
any of the aforementioned steps
involve infinite-dimensional spaces as well.
For a numerical treatment in a computer with finite amounts of memory,
this is of course an insurmountable obstacle.
The remedy in practice is yet another approximation,
where one restricts the problem spaces
to a finite number of dimensions by only considering
those subspaces
spanned by a finite number of basis functions.
This restriction is called discretisation
and has the pleasant side effect that the partial differential equations
dictated by quantum mechanics
reduce to standard problems of linear algebra.
In the case of Hartree-Fock, for example,
the discretised problem may be solved by repetitively diagonalising
the so-called Fock matrix.

This naturally leads to the question,
which type of basis functions should be used to span the subspace.
Most available program packages for
quantum-chemical calculations nowadays employ methodologies based on
the linear combination of Gaussian-shaped atomic orbital functions.
This predominance can be rationalised
by the pragmatic historic developments
taking place in the founding years of modern electronic structure theory.
\citet{Boys1950} realised in \citeyear{Boys1950}
that evaluating the integrals
required for solving Hartree-Fock
is much more feasible for Gaussian-type orbitals
compared to the physically more adequate Slater-type orbitals~\cite{Slater1930}.
This idea was picked up and refined later~\cite{Hehre1969}
and eventually set off many developments
both in terms of efficient algorithms as well as methodologies
centred around Gaussian basis functions,
spreading their use for simulating electronic structures.

Unfortunately Gaussian-type orbitals have a major downside
caused by their unphysical shape:
They are not able to represent
all features of the electron density well.
Most prominently they fail to describe the cusp of the electron density
at the position of a nucleus as well as its exponential decay behaviour~\cite{Kato1957}.
Whilst in most practical use cases one is able to compensate
by employing specialised Gaussian basis sets~\cite{Jensen2013,Hill2013},
there are strong hints that the accurate computation of some physical properties
like NMR shielding tensors~\cite{Guell2008,Hoggan2009}
requires a correct description of both aforementioned features.
Additionally those specialised basis sets can lead to
numerical instabilities in the resulting linear algebra,
making the results less reliable.

Multiple research groups have therefore looked into alternative types of basis functions.
Examples are plane-waves or projector-augmented wave methods%
~\cite{Kresse1996,Kresse1999,Mortensen2005,Enkovaara2010} or
so-called numerical basis functions~\cite{Frediani2015}
like wavelets%
~\cite{Bischoff2011,Bischoff2012,Bischoff2013,Bischoff2014,Bischoff2014a,Bischoff2017}
or finite elements%
~\cite{Tsuchida1995,Soler2002,Lehtovaara2009,Alizadegan2010,Avery2011PhD,Davydov2015,Boffi2016}.
Rather promising are Sturmian-type functions
like Coulomb-Sturmians~\cite{Shull1959,Rotenberg1962,Rotenberg1970,Gruzdev1990,Hoggan2009,Randazzo2010}
or generalised Sturmians~\cite{Avery2006,Avery2011PhD,Avery2011,Morales2016,Avery2017,Randazzo2015,Granados2016},
since they on the one hand amount to correctly represent
the physical features of the electron density,
but on the other hand lead to more feasible integrals than Slater-type orbitals.

\begin{figure}
	\centering
	\includeimage[width=\textwidth]{4_solving_hf/fock_fe_gauss_sturm}  \\[-1.3em]
	{\smaller
	\begin{tabular}{lc@{\hspace{14pt}}ccl}
		\hspace{0.017\textwidth} &
		\hspace{0.25\textwidth} & \hspace{0.27\textwidth} & \hspace{0.25\textwidth} &
		\hspace{0.1\textwidth} \\
		&Finite elements     & Gaussians & Coulomb-Sturmians \\
	\end{tabular}
	} \\[-0.3em]
	\caption[Selection of Fock matrices for different discretisations]{
		Selection of Fock matrices taken from a Hartree-Fock
		calculation for beryllium when the indicated basis function types
		were used for the discretisation.
		Elements smaller than $10^{-10}$ are coloured in white.
	}
	\label{fig:IntroFockStructure}
\end{figure}
One of the challenging aspects when considering novel basis function types
are the deviating mathematical properties
compared to conventional Gaussian-type orbitals.
As an example figure~\ref{fig:IntroFockStructure} shows the structure of the
Fock matrix when discretisations
using finite elements, contracted Gaussians
or Coulomb-Sturmians are employed.
The most drastic difference can be seen for a finite-element discretisation,
where the matrix is both much larger as well as sparser 
compared to the other two cases.
In fact the discretisation which gave rise to the
Fock matrix depicted in figure~\ref{fig:IntroFockStructure}
is still too small to yield a
sensible description of the Beryllium atom density.
A realistic description would need in the order of $10^5$ to $10^6$ basis
functions~\cite{Davydov2015}.
For this reason it is not possible to keep the finite-element discretised
Fock matrix completely in memory and diagonalise it at once,
which is the standard approach when employing contracted Gaussians.
Instead iterative diagonalisation methods need to be used for finite elements.
Coulomb-Sturmians are some kind of a middle ground,
where iterative methods are not necessarily required,
but open up the possibility for more efficient algorithms
as we will discuss in more detail in this work.
The implementation of alternative basis function types
therefore requires in many cases numerical techniques,
which are unusual compared to discretisations based on Gaussians.

As mentioned before existing packages for quantum-chemical calculations
predominantly rely on Gaussian-type orbitals
and as a consequence are highly optimised towards their properties.
Implicit assumptions
about the structure of the discretised quantities,
like the Fock matrix shown above,
are thus typically scattered throughout these large codes
making the implementation of alternative types of basis functions
rather challenging and time consuming.
Especially for the initial testing of novel methods
towards their range of applicability
in standard problems of electronic structure theory
this is an obstacle.
On the other hand starting from scratch for each new basis function type
is not an ideal option either,
since one is faced with the task to reprogram all the algorithms for which
already hundreds of man-years of development have been spent in existing programs.

In this thesis I present a different approach,
which is followed by the \molsturm program package I co-develop.
Motivated from the mathematical structure of the Hartree-Fock problem
as well as a typical algorithmic approaches to solve it,
\molsturm is designed to be a quantum-chemical method development framework,
which supports experimentation of all levels as much as possible.
This includes trying new types of basis functions as well as new algorithms.
It is explicitly \emph{not} the goal of \molsturm to again recode
every aspect of quantum-chemical modelling,
but instead to facilitate integration with existing software
by simple and easy-to-use interfaces.
In this sense \molsturm is only as small contribution building on top
of the work of uncountably many other people of the field,
where the hope is to be useful and allow for
further developments of quantum-chemical modelling to take place in the future.

Following along these lines the following chapters
\ref{ch:QM} and chapter \ref{ch:numeigen}
provide the background for treating the problems of quantum mechanics
as well as quantum chemistry numerically.
Starting from the similarities of classical and quantum physics
the reader will be introduced to functional analysis
and spectral theory, always motivated from a chemical perspective.
In chapter \ref{ch:numeigen} I introduce standard projection techniques
for transforming from the beautiful picture of exact mathematics
into the field of numerical linear algebra,
which allows to solve quantum-chemical problems on a computer.

Thereafter chapter \ref{ch:qchem}
deals with the mathematical and numerical structure of quantum chemistry
with a strong focus on the Hartree-Fock problem
as a first step of computational electronic structure theory.
Chapter \ref{ch:NumSolveHF}
follows with a detailed review of numerical techniques and algorithms
for solving Hartree-Fock.

Chapter \ref{ch:LazyMatrices}
picks up on the lessons learned about the numerical structure
of Hartree-Fock and discusses
\contraction-based methods and lazy matrices
in order to program algorithms
in a basis-function independent way.
In chapter \ref{ch:Molsturm} the design of the quantum-chemistry
package \molsturm is presented,
emphasising its usefulness for investigating novel methods
for quantum-chemical calculations
and computational electronic structure theory.

In chapter \ref{ch:CSQChem} we employ \molsturm
to perform a systematic assessment of
the convergence properties of Coulomb-Sturmians
for the quantum-chemical modelling of atoms
and make some suggestions regarding
sensible Coulomb-Sturmian basis sets for ground state calculations of atoms.
Finally chapter \ref{ch:Conclusion} concludes the work and gives
an outlook into further directions of research.
