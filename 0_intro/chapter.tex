\chapter{Preface}
\chapquote{%
Humanity needs practical men,
who get the most out of their work, and,
without forgetting the general good,
safeguard their own interests.
But humanity also needs dreamers,
for whom the disinterested development of an enterprise is
so captivating that it becomes impossible for them to devote
their care to their own material profit.}
{Marie Skłodowska Curie~(1867--1934)}

\todo[inline,caption={}]{
	\begin{itemize}
		\item Historic context, vision
		\item Beauty of maths in science
		\item What this thesis is about
		\item Structure of the thesis
		\item Mention the relationship to other scientists
			and this thesis is just a small building block
		\item mention where the topic and idea for the thesis came from
	\end{itemize}
}


% TODO Flowify
\todoil{Check and enhance the story by looking at more references}
Starting from the exponential form of the analytical solutions of the
Schrödinger equation for hydrogen,
\citet{Slater1930} started introducing exponentially decaying basis functions
in order to compute atomic properties,
the nowadays well-known Slater-type orbitals~(STOs).

Applying the STOs to molecules turned out to be much more
challenging due to the demanding structure of the electron repulsion integrals~(ERIs)
for such types of basis functions.
\todoil{I cannot download that paper so im not too sure what is in it}
In \citeyear{Boys1950} \citeauthor{Boys1950} realised that employing
atom-centered Gaussian type basis functions~(GTOs) leads to integrals
which are much easier to evaluate numerically.

Since they do, however, not describe the physical reality as closely as STOs
\todoil{Check whether the pople paper goes into this}
much larger GTO basis sets are needed to get similar accuracies than STOs.

\citet{Hehre1969} later realised that one could combine the best of both
worlds if one used
basis sets of contracted GTOs.
This set off a vivid development of quantum chemical theory and software packages,
where a lot of work has gone into efficient algorithms
improving the computational
scaling as well as the overall runtime of calculations based on GTOs.
scaling as well as the overall runtime of calculations.
\todoil{Examples for program packages ??}
% end flowify

Almost all major quantum chemistry programs,
\todoil{Cite a few}
which are actively developed nowadays focus on Gaussian-type
basis functions or Gaussian-type orbitals~(GTOs) as they are usually called.
Compared to the Slater-type orbitals~(STOs)
they certainly have a couple of well-known advantages.
Most importantly the structure of the electron repulsion integrals~(ERIs)
is much less demanding and hence makes it feasible to model much larger systems.

The downside of GTOs is of course that they do not describe the physical
reality that well and as a result fail to capture all of the properties
of an electronic structure in a truely black-box fashion.
For example the computation of proper electron affinities or the description
of Rydberg-like excited states requires incorporating diffuse functions
with small Gaussian exponents into the GTO basis set.

As such it is not surprising that other types of basis sets have been attempted as well.
Next to plane-waves or projector-augmented wave methods,
which are commonly employed in the density-functional theory~(DFT)
calculations of metal surfaces,
recently a couple of projects have also investigated the use
of so-called numerical basis functions.
Examples would be the use of wavelets and the finite-element medhod.
\todoil{Cite madness and Dage Sundholm}
\todoil{The Jensen review paper has a couple of nice citations there}

% structure is vastly different
% need flexible framework to try them
% how do they compare -> need fair comparison

\begin{figure}
	\centering
	\missingfigure{Instead of having the figures here,
	refer to fig. \vref{fig:StructureFiniteElementFock}, \vref{fig:StructureGaussianFock}
	and \vref{fig:StructureSturmianFock}}
	\caption{The different structure of the Fock matrix when FEs,
		Sturmians and GTOs are used.
		All matrices are shown early in the SCF process with a Pulay error norm
		larger than 0.1.
		The matrix structures generally improve as the SCF proceeds and contain more zeros.
		%
		Sturmian and Gaussian are almost diagonal dominant,
		even almost strictly diagonal dominant,
		FE matrix is far from it (About half the rows do not satisfy the condition).
		%
		Only the alpha-alpha part of the fock matrix is shown in each case
		and we perform an RHF calculation of the Berylium atom in a small basis
		for each set of basis functions.
		%
		The Sturmian basis is given in lmn order.
		% TODO maybe use mln order ?
	}
	\label{fig:FockStructure}
\end{figure}

The structure which is required in a program package in order to perform even a simple
self-consistent field~(SCF) calculation
based on GTOs and on numerical basis functions differs quite a lot.
As can be seen in \fig~\ref{fig:FockStructure} the structure of the Fock matrix
varies rather dramatically depending on the type of basis function,
which is used.
Most importantly the matrix size and matrix sparsity properties
changes a lot between numerical approaches like
Wavelets or Finite Elements and classical atom-centered AO approaches like GTOs.
As a result none of the widely used program packages
focus on exactly one type of basis function and are highly specific for exactly this type.

Since the structure of the numerical problems to be solved differ
implementing further basis function types into existing packages
is often challenging.
Moreover unlike the well-understood Gaussians in novel basis function types
the kind of solver algorthims to use is less clear and one often needs to be able to experiment.
\todoil{
all separate packages 
see wikipedia page
\url{https://en.wikipedia.org/wiki/List_of_quantum_chemistry_and_solid-state_physics_software}
}
Furthermore we only know of very few programs which have more than experimental
support in more than one type of basis function.
\todoil{I do not want to make this statement to harsh to piss anyone off,
	but the point is that those packages, which do exist are either not widespread
	or lack features in some of the basis functions
	or are hard to obtain (i.e. closed-source, even wikipedia knows no website)
	or I have literally not heard of them anywhere}

\todoil{I feel some more details introducing SCF in general is also appropriate here}

In previous work we have investigated the use of so-called generalised Sturmians as basis functions
in electronic structure theory.
Generalised sturmians are the solution to a special $N$-body Schrödinger-like
equation.
The difference is that the electron-nucleus interaction is scaled by the factor
$\beta_\nu$ in \eqref{eqn:GenSturm}
such that the solutions all become isoenergetic.
This makes Sturmians reproduce the correct long-range decay behaviour of the electron density
as well as properly represent the nuclear cusp at the core.
As such the functional form of generalised Sturmians is very much related to STOs.
Unlike STOs the two-electron integrals can, however, be reformulated in a particular way
to make computing them less demanding.
This requires, however, that the mathematical properties of the Sturmians
can be fully exploited during the computation,
which in turn requires the Fock matrix to be arranged in a specific way.
Moreover it is advantageous to not build the coulomb and exchange matrix in memory at all,
but much rather only use these matrices in the form of matrix-vector products,
since this overall saves an order of magnitude in the computational scaling.
%
\todoil{How much shall we go into detail here to make this clear?}
% Not at all, do it in the lazyten section.
%
This rather unusual demand turned out to make it very difficult to implement Sturmians
as a basis function type into existing packages alongside STOs and GTOs.

% TODO do this do this
\todoil{Go a bit more into why we only do SCF (Explain: Post-HF available, we only need MO basis, structure of AOs not important)}
% TODO do this do this
In \molsturm we have achieved to write an Hartree-Fock SCF code,
which allows us to use both GTOs and Sturmians side by side by the means
of so-called lazy matrices (see section \ref{sec:lazymat}).
We achieve true generality in the type of basis function to be used
up to the point where most of our code is entirely
independent of the basis function types.
Enwrapping \molsturm into a \python layer allowed us to easily re-use existing
software modules (\adcman and \pyscf) in order to go beyond Hartree-Fock.
Even though we currently only have support for GTOs and Sturmians,
extending our code to numerical basis functions can be achieved rather easily
as we will demonstrate in this paper.


\todoil{The previous section gets rather lengthy. It should probably be split up somehow}


% greek indices denote multi-indices
% \mathcal{I} typically is an appropriate index set.


The reader is assumed to have some background in classical physics
and a solid idea about basic analysis and algebraic structures.
The other concepts will be revised and developed in the first chapter.
