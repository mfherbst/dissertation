\chapter{Conclusions}
\label{ch:Conclusion}
\chapquote{%
	We must include in any language with which we hope
	to describe complex data-processing situations
	the capability for describing data.
}{Grace Hopper~(1906--1992)}

\noindent
The present thesis devised a self-consistent field~(\SCF)
scheme for solving the Hartree-Fock~(\HF) problem
based on matrix-vector contraction expressions
and utilised it in order to design and implement the
quantum-chemical method development framework \molsturm.
This was shown to be easily extensible and flexible enough
to perform calculations
employing any kind of basis function.
The features of \molsturm was demonstrated by the means
of examples as well as an application of the package for
investigating the convergence properties
of quantum-chemical calculations
based on Coulomb-Sturmians,
a basis function type which got few attention so far.
First results of Coulomb-Sturmian-based excited states
calculation employing the algebraic-diagrammatic construction scheme
of the polarisation propagator were reported.

% ------------------------------------

Chapter \ref{ch:QM} reviewed the mathematical background of quantum mechanics
and sketched important results of functional analysis
and spectral theory.
In chapter \ref{ch:numeigen} the Ritz-Galerkin ansatz
for treating spectral problems numerically
was discussed, followed by the ideas of common algorithms
to solve the arising eigenproblems.
The emphasis in both chapters
was put on discussing the material from a quantum-chemical perspective,
but to indicate the often overlooked peculiarities,
which occur when transforming from the infinite-dimensional regime
of functional analysis to the finite-dimensional regime of linear algebra.

% ------------------------------------

In the light of this section \ref{sec:ElectronicSchrödinger}
discussed the spectral properties
of the electronic Schrödinger equation and
described common quantum-chemical methods
for solving this equation numerically.
Both the mathematical formulation of (1) Full-CI, in section \ref{sec:FCI},
as well as (2) \HF, in section \ref{sec:HFIntro}, were discussed.
In section \ref{sec:DiscreteHF} multiple formulations of \HF were given
and their numerical and physical properties were contrasted.
In remark \ref{rem:PropertiesDiscretised}
the self-consistent field~(\SCF) procedure
as a scheme to solve the \HF problem was introduced.
The physical aspects missing in an \HF treatment of the electronic
structure were mentioned in section \ref{sec:FailureHF}
and common Post-\HF methods to correct for these were
reviewed in sections \ref{sec:TruncatedCI} to \ref{sec:ExcitedStates}.

% ------------------------------------

A detailed discussion with regards to the basis function types,
which can be employed to discretise the \HF problem,
was provided in section \ref{sec:BasisTypes}.
For this section \ref{sec:BasisDesiredProperties}
summarised a list of desirable properties
and section \ref{sec:LocalEnergy}
outlined the approach taken in this work for
assessing the ability of a basis set to represent the physical
features of the wave function.
Four basis functions were thereafter considered in detail,
namely (1) Slater-type orbitals in section \ref{sec:STO},
(2) Gaussian-type orbitals in section \ref{sec:cGTO},
(3) finite elements in section \ref{sec:FE}
and (4) Coulomb-Sturmians in section \ref{sec:BasisCS}.

In contrast to the first two
both finite elements,
which are piecewise polynomials defined on a real-space grid,
as well as Coulomb-Sturmians,
which are the analytic solutions to a Schrödinger-like equation,
are rather uncommon approaches,
which have been found to show rather unusual numerical demands as well.
For finite elements, for example,
the matrix representation of the exchange matrix term of the \HF equations
was shown to be rather expensive to compute.
An alternative contraction-based ansatz for \HF was discussed,
where storing the Fock matrix in memory is avoided
and instead only matrix-vector product applications are performed.
It was shown that this scheme would reduce
the formal computational scaling
for the exchange term in the context of finite elements.
For Coulomb-Sturmians such a contraction-based \SCF
ansatz similarly opened the possibilities for performance improvements.
Even though both Coulomb-Sturmians as well as Slater-type
orbitals are exponential basis functions
and their functional form is very similar,
it was seen that their integral expressions are much simpler
and fit very well into the context of a contraction-based \SCF.
Section \ref{sec:BasisTakeaway} summarised our discussion of the basis
function types and section \ref{sec:SCFAlgorithms}
reviewed common \SCF algorithms
with respect to their ability to support the contraction-based \SCF.
For the case of the optimal damping algorithm~\cite{Cances2000a}
section \ref{sec:tODA} gave an approximate modification
to carry some advantageous
properties of the complete scheme to the contraction-based setting.

% ---------------------------------------------------------------------

In chapter \ref{ch:LazyMatrices} contraction-based methods
were formally introduced and in section \ref{sec:ContractionPotentialCaveat}
their potentials and drawbacks were evaluated.
The trend of an increasing gap between processor and memory performance
was outlined and used to rationalise that recomputing data
can sometimes be advantageous, even over storing it in main memory.
A typical challenge with contraction-based methods,
namely their tendency to lead to more involved and harder-to-read code,
was identified
and lazy matrices were introduced as a data structure
to tackle this problem.
It was discussed how lazy matrices as a generalisation
to the conventional matrices,
allow to encapsulate arbitrary contraction expressions,
but still maintain the high-level interface of matrices.
This was achieved by employing lazy evaluation,
which means that operations on lazy matrices are only
evaluated when needed
and otherwise cached inside an expression tree for later evaluation.
Whilst the primary application for lazy matrices in this thesis was the quantum-chemical
program package \molsturm,
lazy matrices are more general and could be used for other problems
of physics and chemistry as well.
An implementation of lazy matrices inside the
lazy matrix library \lazyten was presented in section \ref{sec:lazymat}
along with an example showing a simple \SCF
scheme coded in the language of lazyten \ref{sec:LazytenExamples}.

% ---------------------------------------------------------------------

Section \ref{sec:MolsturmDesign} discussed
the design of the \molsturm program package.
In particular it was indicated how it was achieved
that \molsturm has become such a flexible quantum-chemistry framework,
where novel methods,
can be readily implemented and tested.
It was stressed that linking to third-party code
both on the level of the integral library as well
as the Post-\HF layer is easy.
As a result \molsturm can be best thought of as a mediator
between integral libraries and Post-\HF methods.
The current features of \molsturm
were summarised in section \ref{sec:MolsturmState}.

The interplay between \SCF algorithm and integral library
was further detailed in section \ref{sec:GscfGint}.
We mentioned how a combination of a contraction-based \SCF ansatz
with the lazy matrices of \lazyten
has made it possible to separate the code describing
the \SCF algorithms from the integral back end.
As such it has become possible to implement
an \SCF algorithm in the high-level language of \lazyten
in a basis-function independent way,
whilst the integral back end still has the full
control over the way integral data is produced and consumed.
A result of this is that implementing
additional integral back ends or basis function types
can be achieved by providing a single,
well-defined interface.
In this way a connection from \molsturm
to \libint~\cite{Libint2_231,Libint2} and \libcint~\cite{Sun2015}
for Gaussian-type integrals
and to \sturmint~\cite{sturmintWeb}
for Coulomb-Sturmian-type integrals has been achieved.

In section \ref{sec:MolsturmPython}
the \python interface of \molsturm was described.
Key aspects of the design
was to allow full control over all aspects
of the execution of \molsturm by a detailed set of parameters
for the \SCF as well as the linear algebra back end.
The returned data structures were described
and their compatibility with standard \python packages outlined.
It was mentioned how this in turn
makes it very easy to implement Post-\HF methods,
automate calculations or analyse results.
This has been demonstrated by examples
presented in section \ref{sec:MolsturmExamples}.
Furthermore
readily usable links to selected methods
from \pyscf~\cite{Sun2017} and \adcman~\cite{Wormit2014}
have already been established via the
\python interface.

The test suite and the testing strategy
of \molsturm was outlined in section \ref{sec:MolsturmTestSuite}.
Together with the modularised design of the program
this makes sure that even if changes to the \SCF scheme
were needed in the future,
code could be amended in steps
and the correctness of \molsturm verified in each of these steps.

% ------------------------------------

As an application of \molsturm and its link to the
Coulomb-Sturmian integral library \sturmint was used
in chapter \ref{ch:CSQChem} to perform an initial investigation
of the convergence properties of Coulomb-Sturmian-based
quantum-chemical calculations.
The main focus was on \HF calculations of atoms
of the second and third period of the periodic table.
In section \ref{sec:CSconvergenceHF}
a detailed analysis based on the root mean square values
of the occupied coefficients per angular momentum~($\RMSOl$)
allowed to suggest that
a maximal angular momentum quantum number of $\lmax = 0$
is sufficient for Li and Be,
whereas $\lmax = 1$ is required for N, Ne, Na, Mg, P and Ar.
It further allowed to understand that a known fundamental issue of
the unrestricted \HF procedure
was responsible for the slow convergence observed for the
atoms with one or two unpaired electrons.
At correlated level some Full-CI and {\MP}2 calculations
were performed,
which suggested that increasing the angular momentum quantum number
by one is sufficient to capture
most correlation effects for Li, Be, N, Ne, Na, Mg, P and Ar.

Furthermore the effect of modifying the Coulomb-Sturmian
exponent on the resulting \HF energies was analysed in section \ref{sec:kexp}.
Both an ansatz for estimating $\kopt$ as well as
an algorithm for finding optimal exponents,
\ie those leading to minimal energies,
was discussed in section \ref{sec:DetermineKopt}.
An analogy between the optimal exponent and
the effective nuclear charge was indicated in section \ref{sec:ValuesKopt}
and the roughly
linear relationship of $\kopt$ with the atomic number presented.

In section \ref{sec:SturmianADC} the connection from
\molsturm to \adcman via \python
was exploited to perform the first
excited states calculation based on the algebraic-diagrammatic construction
scheme for the polarisation propagator.
The initial results were reported,
which looked promising and motivating for further research.
