\chapter{Conclusions}
\label{ch:Conclusion}
\chapquote{%
	We must include in any language with which we hope
	to describe complex data-processing situations
	the capability for describing data.
}{Grace Hopper~(1906--1992)}

\noindent
The present thesis devised a self-consistent field~(\SCF)
scheme for solving the Hartree-Fock~(\HF) problem
based on matrix-vector contraction expressions.
It was subsequently utilised in order to design and implement the
quantum-chemical method development framework \molsturm,
where novel methods can be readily implemented and tested.
Furthermore \molsturm was used to investigate the convergence properties
of quantum-chemical calculations
based on Coulomb-Sturmians,
a basis function type which got little attention so far.
Initial results of Coulomb-Sturmian-based excited states
calculation employing the algebraic-diagrammatic construction scheme
for the polarisation propagator were reported
as an outlook to future developments.

% ------------------------------------

First chapter \ref{ch:QM} reviewed the mathematical background of quantum mechanics
and sketched important results of functional analysis
and spectral theory.
In chapter \ref{ch:numeigen} the Ritz-Galerkin ansatz
for numerically treating spectral problems
was discussed, followed by the ideas of common algorithms
to solve the arising eigenproblems.
The emphasis in both chapters
was put on discussing this established mathematical
material from a quantum-chemical perspective,
while indicating the often overlooked peculiarities,
which occur when transforming from the infinite-dimensional regime
of functional analysis to the finite-dimensional regime of linear algebra.

% ------------------------------------

In the light of this section \ref{sec:ElectronicSchrödinger}
discussed the spectral properties
of the electronic Schrödinger equation and
described common quantum-chemical methods
for solving this equation numerically.
The mathematical formulation of both full CI, in section \ref{sec:FCI},
as well as \HF, in section \ref{sec:HFIntro}, were discussed.
In section \ref{sec:DiscreteHF} multiple formulations of \HF were given
and their numerical properties were compared.
In remark \ref{rem:PropertiesDiscretised}
the usual self-consistent field~(\SCF) procedure
as a scheme to solve the \HF problem was introduced.
The physical aspects missing in an \HF treatment of the electronic
structure were mentioned in section \ref{sec:FailureHF}
and common Post-\HF methods to correct for these were
reviewed in sections \ref{sec:TruncatedCI} to \ref{sec:ExcitedStates}.

% ------------------------------------

A detailed discussion of the basis function types,
which can be employed to discretise the \HF problem
was given in section \ref{sec:BasisTypes}.
Section \ref{sec:BasisDesiredProperties} and section \ref{sec:LocalEnergy}
first provided a summary of the desirable properties of such a discretisation,
namely feasible resulting numerical problems on the one hand
and a good description of the physical features of the wave function
on the other.
With respect to this
four basis function types were evaluated in particular.

First the well-known properties of the Slater-type orbitals
and the Gaussian-type orbitals were reviewed
in sections \ref{sec:STO} and \ref{sec:cGTO}.
It was mentioned that Slater-type orbitals lead to challenging
integrals when discretising \HF in such a basis,
whereas discretisations employing Gaussian-type orbitals
satisfy physical correctness
for feasible integrals.
The well-known conclusion
that suitable Gaussian basis sets need to be used to correctly
describe electronic structures was emphasised.

In contrast to the first two,
both finite elements,
% which are piecewise polynomials defined on a real-space grid,
as well as Coulomb-Sturmians,
% which are the analytic solutions to a Schrödinger-like equation,
were discussed.
It was demonstrated how both of these basis functions have the possibility
to represent the physics of the wave function properly,
such that they are promising alternatives.
In contrast to Gaussian-type and Slater-type orbitals
their discretisations, however,
gave rise to unusual numerical demands.
For finite elements, for example,
the matrix representation of the exchange matrix term of the \HF equations
was shown to be rather expensive to compute.
Building on the idea of matrix-free methods~\cite{Kronbichler2012},
a novel, contraction-based ansatz for \HF was introduced
to compensate for this.
In this approach the difference is that storing the Fock matrix in memory
is avoided and instead only matrix-vector product applications are performed.
An analysis of the computational complexity
for the exchange term in the context of finite elements was presented,
which showed that a contraction-based scheme
reduces the formal computational scaling from $\bigO(\Nbas^2)$
to $\bigO(\Nbas)$ with $\Nbas$ being the number of finite elements.
For Coulomb-Sturmians such a contraction-based \SCF ansatz
allowed to exploit the available selection rules in the integral kernels
to a further extent,
thus similarly improving performance.
Even though both Coulomb-Sturmians as well as Slater-type
orbitals are exponential basis functions
of related functional form,
it was found that the integral expressions of Coulomb-Sturmians are much simpler
and fit very well into the context of a contraction-based \SCF.

Section \ref{sec:BasisTakeaway} summarised our discussion of the basis
function types and section \ref{sec:SCFAlgorithms}
reviewed common \SCF algorithms
with respect to their ability to support the contraction-based \SCF.
For the case of the optimal damping algorithm~\cite{Cances2000a}
section \ref{sec:tODA} gave an approximate modification
to carry some advantageous
properties of the complete scheme to the contraction-based setting.

% ---------------------------------------------------------------------

In chapter \ref{ch:LazyMatrices} contraction-based methods
were formally introduced and in section \ref{sec:ContractionPotentialCaveat}
their potentials and drawbacks were evaluated.
The trend of an increasing gap between processor and memory performance
was outlined and used to emphasise that recomputing data
can sometimes be advantageous, even over storing it in main memory.
A typical challenge with contraction-based methods,
namely their tendency to lead to more involved and harder-to-read code,
was identified
and lazy matrices were introduced in section \ref{sec:lazymat}
as a data structure to tackle this problem.
It was discussed how lazy matrices, as a generalisation
of conventional matrices,
allow to encapsulate arbitrary contraction expressions,
but still maintain the high-level interface of matrices.
This was achieved by employing lazy evaluation,
which means that operations on lazy matrices are only
evaluated when needed
and otherwise cached inside an expression tree for later evaluation.
Whilst the primary application for lazy matrices in this thesis was the quantum-chemical
program package \molsturm,
lazy matrices are more general and could be used for other problems
of physics and chemistry as well.
For this reason an implementation of lazy matrices was carried out
in the \lazyten library
and its applicability demonstrated in section \ref{sec:lazyten}.
An example showing a simple \SCF scheme coded in the language of \lazyten
was given in section \ref{sec:LazytenExamples}.

% ------------------------------------------------------------

In section \ref{sec:MolsturmDesign}
the design of the \molsturm program package was discussed.
In particular the interplay between
the contraction-based \SCF scheme and the integral library
was detailed in section \ref{sec:GscfGint}.
It was emphasised how the lazy matrix language of \lazyten
on the one hand
enables to write \SCF algorithms
without making explicit reference to the basis function type,
whilst on the other hand
still allowing the integral back end full
control over the way integral data is produced and consumed.
Thus the code describing
the \SCF algorithms has become independent from the code
dealing with the discretisation details.
On top of that a suitable integral abstraction layer
has made implementing additional integral back ends or basis function types very easy.
In this way a connection from \molsturm
to \libint~\cite{Libint2_231,Libint2} and \libcint~\cite{Sun2015}
for Gaussian-type integrals
and to \sturmint~\cite{sturmintWeb}
for Coulomb-Sturmian-type integrals has been achieved.

The test suite and the testing strategy
of \molsturm were outlined in section \ref{sec:MolsturmTestSuite}.
Together with the modularised design of the program
this ensures that even if changes to the \SCF scheme
were needed in the future,
code could be amended in steps
and the correctness of \molsturm verified in each of these steps.

The key aspects of the design of the \python interface of \molsturm
were to enable full control over the algorithmic details
via a detailed set of parameters on the one hand and
to return computed \SCF results in a readily usable data structure on the other.
It was discussed how in this way
many aspects of the \SCF as well as the linear algebra back end,
like the employed diagonalisation algorithms,
can be altered directly from the interface and without changing any code.
It was pointed out this is
of significance when developing methods based on novel basis functions,
since the best numerical approach might not be clear in the beginning.
In such cases \molsturm allows for experimenting
directly from \python scripts or even interactively.

By means of three examples,
(1) fitting a \ce{H2} dissociation curve in section \ref{sec:ex:data},
(2) implementing a coupled-cluster doubles on top of
the \SCF of \molsturm in section \ref{sec:ex:ccd} and
(3) a gradient-free optimisation in section \ref{sec:ex:geo},
the usefulness of the \python interface
for automating calculations,
analysing results
and implementing novel methods has been demonstrated.
Furthermore the \python interface has been used to establish
links to selected methods
from \pyscf~\cite{Sun2017} and \adcman~\cite{Wormit2014},
such that may be used in combination with any of the
basis function types and integral back ends implemented in \molsturm.
As a result one can think of \molsturm as a mediator
between integral libraries and Post-\HF methods.
The current features of \molsturm
were summarised in section \ref{sec:MolsturmState}.

% ------------------------------------

In chapter \ref{ch:CSQChem} the link of \molsturm
to the Coulomb-Sturmian integral library \sturmint was used
in order to perform an initial investigation
of the convergence properties of Coulomb-Sturmian-based
quantum-chemical calculations.
The main focus was on \HF calculations of atoms
of the second and third period of the periodic table.
In section \ref{sec:CSconvergenceHF}
a detailed analysis based on the root mean square values
of the occupied coefficients per angular momentum~($\RMSOl$)
allowed to suggest that
a maximal angular momentum quantum number of $\lmax = 0$
is sufficient for Li and Be,
whereas $\lmax = 1$ is required for N, Ne, Na, Mg, P and Ar.
It further allowed to understand that a known fundamental issue of
the unrestricted \HF procedure
was responsible for the slow convergence observed for the
atoms with one or two unpaired electrons.
At correlated level some full CI and {\MP}2 calculations
were performed,
which suggested that increasing the angular momentum quantum number
by one is sufficient to capture
most correlation effects for Li, Be, N, Ne, Na, Mg, P and Ar.

Furthermore the effect of modifying the Coulomb-Sturmian
exponent on the resulting \HF energies was analysed in section \ref{sec:kexp}.
Both an ansatz for estimating the optimal exponent $\kopt$,
\ie the exponent leading to the minimal energy,
as well as
an algorithm for finding the value of $\kopt$ systematically,
were developed in section \ref{sec:DetermineKopt}.
Following the relationship of Coulomb-Sturmians and Slater-type orbitals
an analogy between the optimal exponent and
the effective nuclear charge was indicated in section \ref{sec:ValuesKopt}
and from this context the observed
linear relationship of $\kopt$ with the atomic number explained.

In section \ref{sec:SturmianADC} the connection from
\molsturm to \adcman via \python
was employed to perform the first
excited states calculation based on the algebraic-diagrammatic construction
scheme for the polarisation propagator.
Initial results were reported,
which looked promising and motivating for future research.
