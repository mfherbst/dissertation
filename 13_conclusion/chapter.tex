\chapter{Conclusions}
\label{ch:Conclusion}
\chapquote{%
	We must include in any language with which we hope
	to describe complex data-processing situations
	the capability for describing data.
}{Grace Hopper~(1906--1992)}

% mention molsturm later .. first discuss only the methods
% go through conclusions as in structure of the thesis
% mention challenges

% only passive

% refer to the appropriate sections
\noindent
The present thesis devised a novel ansatz for solving the Hartree-Fock~(\HF) problem,
which allows to implement a self-consistent field~(\SCF) algorithm
in such a way that it is independent from the type
of basis function used for the discretisation.
This scheme was implemented in the quantum-chemical program package \molsturm,
introduced in chapter \ref{ch:Molsturm},
which lead to a design,
where adding a new integral back end or a new discretisation method for \HF
is possible almost in a plug-and-play fashion.
In combination with a readily scriptable \python interface
this made \molsturm a light-weight method development framework,
where the focus is not on providing every possible method
of quantum chemistry,
but to make it easy to link to third-party packages.
In chapter \ref{ch:CSQChem} a connection between the \SCF of \molsturm,
the integral library \sturmint~\cite{sturmintWeb} as well as
Post-HF methods from \pyscf~\cite{Sun2017} and \adcman~\cite{Wormit2014}
was exploited to perform an initial assessment
of the convergence properties of a discretisation based on Coulomb-Sturmians,
an exponential-type basis function,
which has seen little attention in the context quantum-chemical calculations so far.

In chapter \vref{ch:QM} and chapter \vref{ch:NumSolveHF}
the mathematical background of quantum mechanics was reviewed
and the Ritz-Galerkin method for discretising
spectral problems was presented.
The emphasis was to discuss the material from a quantum-chemical perspective,
but indicate the often overlooked peculiarities,
which occur when transforming from the infinite-dimensional regime
of functional analysis to the finite-dimensional regime of linear algebra.
In the light of this chapter \vref{ch:qchem} elaborated on the
spectral properties of the electronic Schrödinger equation
and discussed a range of quantum-chemical methods
for solving this equation approximately.
Particular emphasis was put on the Hartree-Fock~(\HF)
ansatz for which multiple equivalent mathematical formulations
were provided in section~\vref{sec:HFIntro}.
Their respective discretised forms
were contrasted in section~\vref{sec:DiscreteHF}
followed by an introduction of the self-consistent field~(\SCF) approach
for solving \HF.
The failures of Hartree-Fock and some Post-\HF methods
for correcting them were reviewed thereafter in section \vref{sec:Correlation}.

For solving the Hartree-Fock problem numerically chapter \ref{ch:NumSolveHF}
showed by the means of a detailed discussion 

Chapter \ref{ch:NumSolveHF} characterised existing numerical approaches
for solving the Hartree-Fock problem on many levels
and devised a general scheme supporting multiple basis
function types and multiple \SCF algorithms on a common ground.

By the means of examples both with respect to \SCF algorithms
as well as discretisation methods
it was shown that a contraction-based \SCF ansatz 



picked up on the \SCF method
collecting the numerical requirements of multiple
basis function types used of the discr

devising a common scheme,


on the different mathematical
formulations of Hartree-Fock and devised a common scheme,
which is able to support on the one hand the deviating
numerical requirements of discretisations employing different
basis function types
as well as on the other hand multiple \SCF schemes.



was devoted to characterising
existing numerical approaches for solving the Hartree-Fock problem
and devised a scheme,
which supports the numerical requirements of discretisations employing different
underlying basis function types.
For this we found a contraction-based ansatz to most general.



where the main result was a contraction-based coefficient-based \SCF
scheme as a means to support


and to  a common structure,
which is able 


and elaborate on a common ground suitable to support
all basis function types and all \SCF algorithms known

The main result of this chapter was
that a 


For this the focus in this chapter was twofold,
first to find a common 

show the common structure between various \SCF
algorithms and second to contrast discretisations
employing different types of basis functions
regarding their ability to reproduce the physical wave function
as well as the numerical properties of the resulting problems of linear algebra.

% basic point is that only contraction-based methods are able
% to support all basis function types


focusing mainly on the structural similarities
of \SCF algorithms as well as a discussion of
discretisations based on different types of basis functions.



Based on the various mathematical formulations of the \HF problem,
section \vref{sec:OverviewSCF}
distinguished between two different types of \SCF algorithms,
namely coefficient-based and density-matrix-based.
The former category performs a minimisation of the \HF energy
with respect to the orbital coefficients,
whilst the latter minimises it with respect to the density matrix.



It was argued how both of these schemes are equivalent
and thus in theory each algorithm of the first kind could
be transformed into an algorithm of the second kind.
Section \vref{sec:SCFAlgorithms} gave examples for both categories.
For the example of the density-matrix-based
optimal damping algorithm~\cite{Cances2000a}
section \vref{sec:tODA} showed the construction of
an approximate scheme in the coefficient-based setting
keeping some of the advantage of the original.





One main aspect was to elaborate on the common structures
of various \SCF algorithms in
\vref{sec:OverviewSCF} and \vref{sec:SCFAlgorithms}.
Based on the mathematical 




In section \vref{sec:BasisTypes}
various basis function types
including Slater-type orbitals in section \ref{sec:STO},



a detailed discussion
of basis function types
which could be used to discretise the \HF problem




in section \vref{sec:BasisTypes}.
For this first a list of desirable properties
was presented in section \vref{sec:BasisDesiredProperties}
and the concept of a local energy,
a well-established quantity in the quantum Monte Carlo community,
was introduced as a measure for the relative deviation from being an eigenstate
in section \vref{sec:LocalEnergy}.
Both this metric as well as the relative error
was used to contrast the ability of various basis function types
to represent the exact solution to the Schrödinger equation of the Hydrogen atom.
Our discussion included
Slater-type orbitals in section \vref{sec:STO},
which describe the physical features of the wave function very well,
but the evaluation of the required integrals is difficult.
Furthermore we considered contracted Gaussian-type orbitals~({\cGTO}s)
as a measure to yield feasible integrals,
but fail to represent the proper physics.
In contrast to these rather common approaches we presented
finite elements in section \vref{sec:FE},
which are piecewise polynomials defined on a real-space grid.
Whilst they are mathematically well-understood
with guaranteed convergence properties
and the ability to adapt to the problem automatically,
their numerical structure, however, is extremely challenging
and can only be tackled if linear scaling methods are employed.
In this context we showed that only a contraction-based ansatz,
which focuses on formulating the algorithms
in terms of the contraction of the Fock matrix with other
vectors or matrices,
is able to achieve this.
Furthermore only coefficient-based \SCF schemes could be employed
for this type of basis function.
Section \vref{sec:BasisCS} reviewed Coulomb-Sturmians
an alternative exponential-type basis function,
which is able to reproduce both the nuclear cusp as well as the exponential
decay properly
and is furthermore complete,
such that convergence is guaranteed
with increasing basis set size.
The integrals are feasible.
only atoms possible, generalisations exist.
To exploit the mathematical properties of Coulomb-Sturmians
again a contraction-based \SCF is required.








\SCF takeaway (hint of similar structure) \vref{sec:SCFtakeaway}
two-step coefficient-update or density-matrix-update and Fock-update
coefficient-based {\SCF}s support \contraction-based ansatz
common ground for \SCF schemes and basis function types.



%coefficient based vs density-matrix-based
where the overall aim was to find the similarities as well as the common
structure between all approaches.
Both different types of basis functions which could be used
for discretising the \HF problem as well as
a range of standard \SCF algorithms were introduced.
The advantages and disadvantages of different basis function
types were contrasted
and the challenges with the deviating numerical structure
of finite elements as well as Coulomb-Sturmians
were mentioned
as well as the potential prospect to combine the structure
of all \SCF algorithms
inside a contraction-based \SCF,
where the Fock matrix is no longer held in memory
% possible with all basis functions too
% no need for dense methods
% some call for iterative methods


% ---------------------------------------------------------------------


\lazyten lazy matrix library
Chapter \ref{ch:LazyMatrices} was devoted to a detailed discussion
of contraction-based methods.
Contraction-based methods not a new thing \vref{sec:ContractionAlgos}
We noted that their disadvantage that more computations
need to be performed is overcome by trends in the current developments
of hardware
in section \vref{sec:ContractionPotentialCaveat}
In fact recomputing data has in some cases become cheaper than storing
and retrieving it later.
It was mentioned that another issue with contraction-based methods
namely their need to write harder to read code
is overcome by using lazy matrices.
Lazy matrices were introduced as a concept to write contraction-based
methods at a high level domain-specific language.
They are in contrast to stored matrices where all data has to be in dense memory.
produce the same interface as matrices,
but evaluation is lazy.
Whilst our application for lazy matrices is the quantum-chemical
program package molsturm,
lazy matrices are more general and could be used for other problems
of physics and chemistry as well.
lazy matrix library \lazyten \vref{sec:lazymat}
Examples showing simple \SCF coded in the language of lazyten \vref{sec:LazytenExamples}

The program package \molsturm is introduced and its design described
in chapter \vref{ch:Molsturm}.
It heavily employs lazy matrices as a mediator between
the \SCF algorithms as well as the integral back end.
% Plug and play integral back ends and basis function types
% SCF algorithms can be modified and extended in one place
% readily usable interfaces to post-hf methods, which are independent of
% basis function type as well (see \cGTO adc and \CS adc)
% flexible to be able to adapt to future changes
% program algorithms in basis-function independent way
which can mediate between all kinds of integral back ends
%
% Molsturms ability to aid with developments of novel methods demonstrated
% both in examples of molsturm section
easy to link third party libraries and make use of them
to extend molsturm
\vref{sec:MolsturmExamples}
The current state and possibilities of molsturm
were summarised in section \vref{sec:MolsturmState}
plug-and-play implementation

In chapter \vref{ch:CSQChem} we took a first
look at the convergence properties of Coulomb-Sturmian
basis sets in the context of quantum-chemical calculations.
The emphasis was on Hartree-Fock,
but initial results employing correlated methods look promising, too.
First \ADC(2) calculation with \molsturm was underdone
and presented in section \vref{sec:SturmianADC}
which looked promising and motivating for further research.





in the context of \HF as well as some Post-HF methods.



which has so far only seen little attention in the context of quantum-chemical
calculations.

not focused on implementing every available quantum-chemical method,
bu

This has lead to a light-weight method development framework,
where both implementing a novel discretisation method
as well as adding a link to a third-party library has become easy.


\molsturm's functionality 

for performing
calculations at Post-\HF level has become easy.
We demonstrated this by the means of examples in section \vref{sec:MolsturmExamples}
as well as a detailed study in chapter \vref{ch:CSQChem},
which performed calculations
at Hartree-Fock as well as various correlated levels of theory


In this way a connection from \molsturm
to \libint~\cite{Libint2_231,Libint2} and \libcint~\cite{Sun2015}
for Gaussian-type integrals
and to \sturmint~\cite{sturmintWeb}
for Coulomb-Sturmian-type integrals has been achieved.
For performing Post-HF methods
interfaces to \pyscf~\cite{Sun2017}
for performing \FCI as well as \adcman~\cite{Wormit2014}
for performing {\ADC}(1), {\ADC}(2), {\ADC}(2)-x and {\ADC}(3)
have been achieved.

