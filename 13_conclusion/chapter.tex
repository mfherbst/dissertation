\chapter{Conclusions}
\label{ch:Conclusion}
\chapquote{%
	We must include in any language with which we hope
	to describe complex data-processing situations
	the capability for describing data.
}{Grace Hopper~(1906--1992)}
\todoil{Draft version}

% more citations
\noindent
The present thesis devised a flexible contraction-based self-consistent field~(\SCF)
scheme for solving the Hartree-Fock~(\HF) problem
and utilised it in order to design and implement the
quantum-chemical method development framework \molsturm,
which can easily be extended to perform calculations
employing any kind of basis function.
The applicability of \molsturm was demonstrated by the means
of examples as well as an application of the package for
investigating the convergence properties
of quantum-chemical calculations
based on Coulomb-Sturmians,
a basis function type which got few attention so far.
First results of a Coulomb-Sturmian-based excited states
calculation employing the algebraic-diagrammatic construction scheme
of the polarisation propagator were reported.

% ------------------------------------

Chapter \ref{ch:QM} reviewed the mathematical background of quantum mechanics
and sketched important results of functional analysis
and spectral theory.
In chapter \ref{ch:numeigen} the Ritz-Galerkin ansatz
for treating spectral problems numerically
was discussed, including the ideas of common algorithms
to solve the arising eigenproblems.
The emphasis in both chapters
was put on discussing the material from a quantum-chemical perspective,
but to indicate the often overlooked peculiarities,
which occur when transforming from the infinite-dimensional regime
of functional analysis to the finite-dimensional regime of linear algebra.

% ------------------------------------

In the light of this chapter \ref{ch:qchem}
discussed the spectral properties
of the electronic Schrödinger equation in section \ref{sec:ElectronicSchrödinger} and
turned the attention towards common quantum-chemical methods
for solving this equation numerically.
Both the mathematical formulation of Full-CI in section \ref{sec:FCI}
as well as Hartree-Fock~(\HF) in section \ref{sec:HFIntro} were discussed.
In section \ref{sec:DiscreteHF}
a discretisation ansatz based on spinors of
single-particle basis functions was employed
to derive multiple discretised formulations of \HF.
The arising numerical and physical properties were contrasted
and the self-consistent field~(\SCF) procedure
was introduced in remark \ref{rem:PropertiesDiscretised}
as a scheme to solve the \HF problem.
The physical aspects missing in an \HF treatment of the electronic
structure was discussed in section \ref{sec:FailureHF}
and common Post-\HF methods to correct for these were
discussed in sections \ref{sec:TruncatedCI} to \ref{sec:ExcitedStates}.
%An alternative to \HF in the form of density-functional theory
%was briefly mentioned in \ref{sec:DFT}.

% ------------------------------------

A detailed discussion with regards to possible basis function types,
which could be employed to discretise the \HF problem,
was provided in section \ref{sec:BasisTypes}.
For this section \ref{sec:BasisDesiredProperties}
provided a summary of the desirable properties of basis functions
in the context of quantum-chemical calculations,
followed by section \ref{sec:LocalEnergy},
where the local energy as a measure
for the ability of a basis to represent an eigenstate
of the Schrödinger operator was introduced.

Four basis functions were considered in detail.
First Slater-type orbitals were discussed in section \ref{sec:STO},
followed by contracted Gaussian-type orbitals
in section ~\ref{sec:cGTO}.
In contrast to these rather common approaches 
finite elements were presented in section \ref{sec:FE}.
Finite elements are piecewise polynomials defined on a real-space grid.
Whilst these on the one hand show very interesting properties,
like % todo
, they suffer from % todo
For these the most problematic aspect
was the representation of the exchange term of the Hartree-Fock equations.
It was argued that storing the discretised form of this term as a matrix in memory
is not feasible

is not feasible due to the very demanding numerical structure.
As a remedy a contraction-based ansatz was discussed,
where one avoids storing matrices in memory
and instead constructs an algorithm based
on matrix-vector product expressions alone.
As the final basis type section \ref{sec:BasisCS} reviewed Coulomb-Sturmians.
These are exponential-type basis functions,
which are related to Slater-type orbitals.
In the presented analysis they showed to be able to represent both
the nuclear cusp as well the exponential decay behaviour.
Their integral expressions are much simpler to evaluate
compared to Slater-type orbitals making them a promising alternative to these.
It was discussed how a contraction-based ansatz
towards the \SCF can lead to an improved computational scaling for
Coulomb-Sturmians as well.

In section \ref{sec:BasisTakeaway}
the common features of all discussed basis functions
were summarised and the basis types contrasted.
It was indicated that a contraction-based \SCF scheme
would be able to support discretisations based on all basis functions.
Motivated by this thought
common \SCF algorithms were reviewed in section \ref{sec:SCFAlgorithms}.
For the case of the optimal damping algorithm~\cite{Cances2000a}
section \ref{sec:tODA} gave a modification
to approximately carry some advantageous
properties to the contraction-based setting.

% ---------------------------------------------------------------------

A detailed discussion of contraction-based methods was
presented in chapter \ref{ch:LazyMatrices}.
In section \ref{sec:ContractionPotentialCaveat}
we evaluated their potential and drawbacks
with reference to current trends in the development
of computer hardware.
The increasing gap between processor and memory performance
was outlined and used to outline the
advantages of recomputing data instead of storing them in main memory.
It was mentioned that another issue with contraction-based methods
namely their need to write harder to read code
is overcome by using lazy matrices.
Lazy matrices were introduced as a concept to write contraction-based
methods at a high level domain-specific language.
They are in contrast to stored matrices where all data has to be in dense memory.
produce the same interface as matrices,
but evaluation is lazy,
\ie only performed when needed.
Whilst the primary application for lazy matrices in this thesis was the quantum-chemical
program package \molsturm,
lazy matrices are more general and could be used for other problems
of physics and chemistry as well.
An implementation of lazy matrices inside the
lazy matrix library \lazyten was presented in section \ref{sec:lazymat}
along with an example showing a simple \SCF
scheme coded in the language of lazyten \ref{sec:LazytenExamples}.

% ---------------------------------------------------------------------

Section \ref{sec:MolsturmDesign} discussed
the design of the \molsturm program package.
In particular it was indicated how
\molsturm has become a flexible quantum-chemistry framework,
where novel methods,
including novel basis function types,
can be easily implemented and tested
and where linking to third-party code is easy.
Because of this \molsturm can be thought of as a mediator
to produce \SCF results in a very general fashion
on top of which one may stick any Post-\HF method.

It was indicated in section \ref{sec:GscfGint}
how a combination of a contraction-based \SCF ansatz
with the lazy matrices of \lazyten
has made it possible to separate the code describing
the \SCF algorithms from the integral back end.
As such it has become possible to implement
an \SCF algorithm in the high-level language of \lazyten
in a basis-function independent way,
whilst the integral back end still has the full flexibility
to decide the numerical details
of the contraction expression in the way required by the basis function type.
Furthermore this has made implementing
additional integral back ends or basis function types very easy,
since only changes in one particular well-defined integral
interface are required to make
a new basis function type available to the rest of \molsturm,
including all third-party packages on top.
In this way a connection from \molsturm
to \libint~\cite{Libint2_231,Libint2} and \libcint~\cite{Sun2015}
for Gaussian-type integrals
and to \sturmint~\cite{sturmintWeb}
for Coulomb-Sturmian-type integrals has been achieved.

In section \ref{sec:MolsturmPython} it was described
how the \python interface of \molsturm
has been designed to allow full control over all aspects
of \molsturm's execution by a detailed set of parameters
and how the obtained results are returned
in readily usable data structures.
It was mentioned how this in turn
makes it very easy to combine \molsturm
with third-party packages
or automate calculations and further analysis.
So far the \python interface of \molsturm
has enabled to establish
links to selected methods
from \pyscf~\cite{Sun2017} and \adcman~\cite{Wormit2014}.

A brief overview of the test suite
ensuring that future restructuring of code does not break things is
given in section \ref{sec:MolsturmTestSuite}.
Together with the flexible design
indicated in \ref{sec:MolsturmDesign}
this makes sure that even if changes to the \SCF scheme
were needed in the future,
these could be made and validity of the state of molsturm would not
be influenced.

By the means of examples presented in \ref{sec:MolsturmExamples}
this was shown in practice.
The current state and possibilities of molsturm
were summarised in section \ref{sec:MolsturmState}.


% ------------------------------------

As an application of \molsturm its link to the
Coulomb-Sturmian integral library \sturmint~\cite{sturmintWeb} was used
in chapter \ref{ch:CSQChem} to perform an initial investigation
of the convergence properties of Coulomb-Sturmian-based
quantum-chemical calculations.
The main focus was on understanding the convergence
properties of \HF calculations of atoms
of the second and third period of the periodic table.
In section \ref{sec:CSconvergenceHF}
a detailed analysis based on the root mean square values
of the occupied coefficients per angular momentum~($\RMSOl$)
allowed to suggest that
a maximal angular momentum of $\lmax = 0$
is sufficient for Li and Be,
whereas $\lmax = 1$ is required for N, Ne, Na, Mg, P and Ar.
It further allowed to understand that the
underlying cause of the slow convergence of the other
atoms B, C, O, F, Al, Si, S and Cl is an issue due to \UHF.
At correlated level some Full-CI and {\MP}2 calculations
were performed,
which suggested that increasing the angular momentum
by one is sufficient to capture
most correlation effects for Li, Be, N, Ne, Na, Mg, P and Ar.
Furthermore the effect of modifying the Coulomb-Sturmian
exponent on the resulting \HF energies was analysed in section \ref{sec:kexp}
and a routine for estimating $\kopt$ as well as finding
optimal exponents, \ie those leading minimal energies,
was discussed in section \ref{sec:DetermineKopt}.
An analogy between the optimal exponent and
the effective nuclear charge was indicated in section \ref{sec:ValuesKopt}
and the roughly
linear relationship of $\kopt$ with the atomic number presented.
In section \ref{sec:SturmianADC} the connection from
\molsturm to \adcman~\cite{Wormit2014} via \python
was exploited to perform the first
excited states calculation based on the algebraic-diagrammatic construction
scheme for the polarisation propagator.
The inital results were reported,
which looked promising and motivating for further research.
