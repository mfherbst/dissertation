\chapter{Conclusions}
\label{ch:Conclusion}
\chapquote{%
	We must include in any language with which we hope
	to describe complex data-processing situations
	the capability for describing data.
}{Grace Hopper~(1906--1992)}
\todoil{Draft version}

% refer to the appropriate sections
\noindent
The present thesis devised a contraction-based self-consistent field~(\SCF)
scheme for solving the Hartree-Fock~(\HF) problem
and leveraged it in order to design and implement
the quantum-chemical program package \molsturm.
It was shown in section \ref{sec:MolsturmDesign}
that \molsturm's flexible design readily supports the addition
of further integral back ends or discretisation methods.
By the means of examples it was demonstrated in section \ref{sec:MolsturmExamples}
how \molsturm's \python interface allows to combine the package
with existing third-party libraries
for rapidly developing new methods,
or to aid systematic comparisons
% automatisation.

First the mathematical background of quantum mechanics
was reviewed in chapter \ref{ch:QM}
and the Ritz-Galerkin method for discretising
spectral problems was presented in chapter \ref{ch:NumSolveHF}.
The emphasis in both chapters
was to discuss the material from a quantum-chemical perspective,
but indicate the often overlooked peculiarities,
which occur when transforming from the infinite-dimensional regime
of functional analysis to the finite-dimensional regime of linear algebra.

% TOdo maybe shorten this part
In chapter \vref{ch:qchem} common quantum chemical methods
were discussed from a mathematical perspective.
This includes the spectral properties of the Schrödinger equation
in section \ref{sec:ElectronicSchrödinger},
as well as the mathematical formulation of Full-CI
in section \ref{sec:FCI}.
Originating from this Hartree-Fock~(\HF)
was introduced as a rank-1 approximation to Full-CI
and its mathematical properties discussed.
Employing a spin-adapted formulation based on spinors
different discretised formulations of Hartree-Fock~(\HF)
as a minimisation problem
were presented in section \ref{sec:DiscreteHF}.
% mention coefficient matrix and density matrix
An analogous classification of the self-consistent field~(\SCF) schemes
into coefficient-based and density-matrix-based was 
Some failures of \HF and common methodologies
to correct for these were briefly
discussed in section \ref{sec:Correlation} for completeness.



then numerical hf chapter
% contrast basis functions


% ------------------------------------

Chapter \ref{ch:NumSolveHF} was devoted to characterising
different approaches for numerically solving the Hartree-Fock problem
and to identify similarities and common structures
in the numerical approaches typically required.

% basis function aspect
Following a review of desirable properties
for basis function types employed for discretisations in \HF
in section \vref{sec:BasisDesiredProperties}
as well as a discussion
of some measures for judging the errors when it comes to properly
representing the physical wave function
in section \vref{sec:LocalEnergy},
including the concept of a local energy,
we discussed four basis function types in general
regarding their numerical properties,
their Fock matrix structure
as well as their ability to represent the physical wave function


Slater-type orbitals were introduced in section \vref{sec:STO}.
As exponential-type these are a very good description of the physics,
but the evaluation of the required integrals is comparatively challening.
Constracted Gaussian-type orbitals
as the most common approach in electronic structure theory
were discussed next in section ~\ref{sec:cGTO}.
Furthermore we considered contracted Gaussian-type orbitals~({\cGTO}s)
as a measure to yield feasible integrals,
but fail to represent all aspects of the proper physics
or specialised measures need to be employed.
In contrast to these rather common approaches we presented
finite elements in section \vref{sec:FE},
which are piecewise polynomials defined on a real-space grid.
Whilst they are mathematically well-understood
with guaranteed convergence properties
and the ability to adapt to the problem automatically,
their numerical structure, however, is extremely challenging
and can only be tackled if linear scaling methods are employed.
Section \vref{sec:BasisCS} reviewed Coulomb-Sturmians
an alternative exponential-type basis function,
which is able to reproduce both the nuclear cusp as well as the exponential
decay properly
and is furthermore complete,
such that convergence is guaranteed
with increasing basis set size.
Compared to Slater-type orbitals their integrals stay feasible
since all basis functions share the same common exponent.
They are only able to represent atoms,
but generalisations exist, which show similar properties
and they are just the first step.
Originating for the arising deviating numerical structures
an algorithmic based on matrix-vector contraction expressions
was outlined,
which focuses on formulating the algorithms
in terms of the contraction of the Fock matrix with other
vectors or matrices.
It was shown that this ansatz
can lead to a formulation
of the \SCF algorithms in a basis-function independent way
and in the case of finite-elements and Coulomb-Sturmians
could even lead to scaling improvements.
Furthermore only coefficient-based \SCF schemes could be employed
for this type of basis function.

% scf aspect
Based on the various mathematical formulations of the \HF problem,
section \vref{sec:OverviewSCF}
distinguished between two different types of \SCF algorithms,
namely coefficient-based and density-matrix-based.
The former category performs a minimisation of the \HF energy
with respect to the orbital coefficients,
whilst the latter minimises it with respect to the density matrix.
It was argued how both of these schemes are equivalent
and thus in theory each algorithm of the first kind could
be transformed into an algorithm of the second kind.
Section \vref{sec:SCFAlgorithms} gave examples for both categories.
For the example of the density-matrix-based
optimal damping algorithm~\cite{Cances2000a}
section \vref{sec:tODA} showed the construction of
an approximate scheme in the coefficient-based setting
keeping some of the advantage of the original.

In a review of standard \SCF algorithms
we found them to be expressable as a two step
two-step coefficient-update or density-matrix-update and Fock-update
process
and that coefficient-algorithms are generally required
to support a contraction-based approach to the \SCF problem.

% ---------------------------------------------------------------------


\lazyten lazy matrix library
Chapter \ref{ch:LazyMatrices} was devoted to a detailed discussion
of contraction-based methods.
Contraction-based methods not a new thing \vref{sec:ContractionAlgos}
We noted that their disadvantage that more computations
need to be performed is overcome by trends in the current developments
of hardware
in section \vref{sec:ContractionPotentialCaveat}
In fact recomputing data has in some cases become cheaper than storing
and retrieving it later.
It was mentioned that another issue with contraction-based methods
namely their need to write harder to read code
is overcome by using lazy matrices.
Lazy matrices were introduced as a concept to write contraction-based
methods at a high level domain-specific language.
They are in contrast to stored matrices where all data has to be in dense memory.
produce the same interface as matrices,
but evaluation is lazy.
Whilst our application for lazy matrices is the quantum-chemical
program package molsturm,
lazy matrices are more general and could be used for other problems
of physics and chemistry as well.
lazy matrix library \lazyten \vref{sec:lazymat}
Examples showing simple \SCF coded in the language of lazyten \vref{sec:LazytenExamples}

% ---------------------------------------------------------------------

Following a review of existing quantum-chemical software
with flexibility and generality in mind in section \vref{sec:MolsturmRelated}
the design of the \molsturm framework was discussed in section \vref{sec:MolsturmDesign}.







The program package \molsturm is introduced and its design described
in chapter \vref{ch:Molsturm}.
It heavily employs lazy matrices as a mediator between
the \SCF algorithms as well as the integral back end.
% Plug and play integral back ends and basis function types
% SCF algorithms can be modified and extended in one place
% readily usable interfaces to post-hf methods, which are independent of
% basis function type as well (see \cGTO adc and \CS adc)
% flexible to be able to adapt to future changes
% program algorithms in basis-function independent way
which can mediate between all kinds of integral back ends
%
% Molsturms ability to aid with developments of novel methods demonstrated
% both in examples of molsturm section
easy to link third party libraries and make use of them
to extend molsturm
\vref{sec:MolsturmExamples}
The current state and possibilities of molsturm
were summarised in section \vref{sec:MolsturmState}
plug-and-play implementation







not focused on implementing every available quantum-chemical method,
bu

This has lead to a light-weight method development framework,
where both implementing a novel discretisation method
as well as adding a link to a third-party library has become easy.


\molsturm's functionality 

for performing
calculations at Post-\HF level has become easy.
We demonstrated this by the means of examples in section \vref{sec:MolsturmExamples}
as well as a detailed study in chapter \vref{ch:CSQChem},
which performed calculations
at Hartree-Fock as well as various correlated levels of theory

% molsturms functionality: look at molsturm's state section


In this way a connection from \molsturm
to \libint~\cite{Libint2_231,Libint2} and \libcint~\cite{Sun2015}
for Gaussian-type integrals
and to \sturmint~\cite{sturmintWeb}
for Coulomb-Sturmian-type integrals has been achieved.
For performing Post-HF methods
interfaces to \pyscf~\cite{Sun2017}
for performing \FCI as well as \adcman~\cite{Wormit2014}
for performing {\ADC}(1), {\ADC}(2), {\ADC}(2)-x and {\ADC}(3)
have been achieved.


a novel approach for solving the Hartree-Fock~(\HF) problem
based on a contraction-based self-consistent field~(\SCF) 

which allowed to implement a self-consistent field~(\SCF) algorithm
in such a way that it is independent from the
type of basis function used for the discretisation.
Inside the quantum-chemical program package \molsturm
this allowed to achieve a flexibility
where adding a new integral back end or a new discretisation method for \HF
is possible almost in a plug-and-play fashion.



Using a design focused on flexibility such
an implementation was achieved inside the
quantum-chemical program package \molsturm,
where adding a new integral back end or a new discretisation method for \HF

this made \molsturm a light-weight method development framework,
where the focus is not on providing every possible method
of quantum chemistry,
but to make it easy to link to third-party packages.
In chapter \ref{ch:CSQChem} a connection between the \SCF of \molsturm,
the integral library \sturmint~\cite{sturmintWeb} as well as
Post-HF methods from \pyscf~\cite{Sun2017} and \adcman~\cite{Wormit2014}
was exploited to perform an initial assessment
of the convergence properties of a discretisation based on Coulomb-Sturmians,
an exponential-type basis function,
which has seen little attention in the context quantum-chemical calculations so far.


% ------------------------------------

As an application of \molsturm and its link to the
Coulomb-Sturmian integral library \sturmint,
chapter \ref{ch:CSQChem} performed an initial investigation
of the convergence properties of Coulomb-Sturmian-based
quantum-chemical calculations.
The main focus was on understanding the convergence
properties of \HF calculations on the atoms
of the second and third period of the periodic table.
A detailed analysis based on the root mean square values
of the occupied coefficients per angular momentum
allowed to suggest $\lmax = 0$ for Li and Be,
$\lmax = 1$ for N, Ne, Na, Mg, P and Ar.
It further allowed to understand that the
underlying cause of the slow convergence of the other
atoms is an issue due to \UHF.
At correlated level some Full-CI and {\MP}2 calculations
were performed,
which suggested that to capture most correlation effects
an extra angular momentum is sufficient
for Li, Be, N, Ne, Na, Mg, P and Ar.
The effect of modifying the Sturmian exponent on the result
was analysed in detail
% with larger basis set size precise value not important
and a routine for estimating $\kopt$ as well as finding
optimal values was discussed.
% issues with bad convergence in the beginning.
Need clever scheme%
Resulting values were presented.
%
An analogy between the optimal exponent and
the effective nuclear charge is indicated and the roughly
linear relationship of $\kopt$ with the atomic number presented.
%
In section \ref{sec:SturmianADC} the connection from
\molsturm to \adcman~\cite{Wormit2014} via \python
was exploited to perform the first
excited states calculation based on the algebraic-diagrammatic construction
scheme for the polarisation propagator.
The inital results were reported,
which looked promising and motivating for further research.
