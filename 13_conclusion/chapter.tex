\chapter{Conclusions}
\label{ch:Conclusion}
\chapquote{%
	We must include in any language with which we hope
	to describe complex data-processing situations
	the capability for describing data.
}{Grace Hopper~(1906--1992)}
\todoil{Draft version}

% more citations
\noindent
The present thesis devised a flexible contraction-based self-consistent field~(\SCF)
scheme for solving the Hartree-Fock~(\HF) problem
and utilised it in order to design and implement the
quantum-chemical method development framework \molsturm,
which can easily be extended to perform calculations
employing any kind of basis function.
The applicability of \molsturm was demonstrated by the means
of examples as well as an application of the package for
investigating the convergence properties
of quantum-chemical calculations
based on Coulomb-Sturmians,
a basis function type which got few attention so far.
First results of Coulomb-Sturmian-based excited states
calculation employing the algebraic-diagrammatic construction scheme
of the polarisation propagator were reported.

% ------------------------------------

Chapter \ref{ch:QM} reviewed the mathematical background of quantum mechanics
and sketched important results of functional analysis
and spectral theory.
In chapter \ref{ch:numeigen} the Ritz-Galerkin ansatz
for treating spectral problems numerically
was discussed, including the ideas of common algorithms
to solve the arising eigenproblems.
The emphasis in both chapters
was put on discussing the material from a quantum-chemical perspective,
but to indicate the often overlooked peculiarities,
which occur when transforming from the infinite-dimensional regime
of functional analysis to the finite-dimensional regime of linear algebra.

% ------------------------------------

In the light of this section \ref{sec:ElectronicSchrödinger}
discussed the spectral properties
of the electronic Schrödinger equation and
described common quantum-chemical methods
for solving this equation numerically.
Both the mathematical formulation of Full-CI in section \ref{sec:FCI}
as well as Hartree-Fock~(\HF) in section \ref{sec:HFIntro} were discussed.
In section \ref{sec:DiscreteHF} multiple formulations of \HF were given
and their numerical and physical properties were contrasted.
In remark \ref{rem:PropertiesDiscretised}
the self-consistent field~(\SCF) procedure
as a scheme to solve the \HF problem was introduced.
The physical aspects missing in an \HF treatment of the electronic
structure was discussed in section \ref{sec:FailureHF}
and common Post-\HF methods to correct for these were
reviewed in sections \ref{sec:TruncatedCI} to \ref{sec:ExcitedStates}.

% ------------------------------------

A detailed discussion with regards to the basis function types,
which can be employed to discretise the \HF problem,
was provided in section \ref{sec:BasisTypes}.
For this section \ref{sec:BasisDesiredProperties}
summarised a list of desirable properties
and section \ref{sec:LocalEnergy}
outlined the approach taken for
assessing the ability of a basis set to represent the physical
features of the wave function in this work.
Four basis functions were thereafter considered in detail,
namely (1) Slater-type orbitals in section \ref{sec:STO},
(2) Gaussian-type orbitals in section \ref{sec:cGTO},
(3) finite elements in section \ref{sec:FE}
and (4) Coulomb-Sturmians in section \ref{sec:BasisCS}.

In contrast to the first two
both finite elements,
which are piecewise polynomials defined on a real-space grid,
as well as Coulomb-Sturmians,
which are the analytic solutions to a Schrödinger-like equation,
are rather uncommon approaches,
which have been found to show deviating numerical demands.
For finite elements, for example,
the matrix representation of the exchange matrix term of the Hartree-Fock equations
was shown to be rather expensive to compute.
An alternative contraction-based ansatz for \HF was discussed,
where storing the Fock matrix in memory is avoided
and instead only matrix-vector product applications are performed.
It would be shown that this scheme would reduce
the formal computational scaling
for the exchange term in the context of finite elements.
For Coulomb-Sturmians such a contraction-based \SCF
ansatz similarly opened the possibilities for performance improvements.
Even though both Coulomb-Sturmians as well as Slater-type
orbitals are exponential basis functions
and their functional form is very similar,
it was seen that their integral expressions are much simpler
and fit very well into the context of a contraction-based \SCF.
Section \ref{sec:BasisTakeaway} summarised our discussion of the basis
function types and section \ref{sec:SCFAlgorithms}
reviewed common \SCF algorithms
with respect to their ability to support the contraction-based \SCF.
For the case of the optimal damping algorithm~\cite{Cances2000a}
section \ref{sec:tODA} gave an approximate modification
to carry some advantageous
properties to the contraction-based setting.

% ---------------------------------------------------------------------
\todoil{Draft from here}

In chapter \ref{ch:LazyMatrices} contraction-based methods
were formally introduced and in section \ref{sec:ContractionPotentialCaveat}
their potentials and drawbacks were evaluated
with reference to current trends in the development of computer hardware.
The increasing gap between processor and memory performance
was outlined and used to rationalise that recomputing data
may be advantageous even over storing it in main memory.
Another issue with contraction-based methods,
namely their need to write more involved and harder-to-read code,
was discussed.
As a remedy lazy matrices were introduced as a data structure
to write contraction-based
methods at a high level, in the sense of a domain-specific language.

% put into:   it was discussed that thing

In contrast to the usual, stored matrices,
where all data has to be in dense memory,
lazy matrices are a generalisation,
able to encapsulate arbitrary contraction expressions.

They are designed to produce the same interface as stored matrices,
\ie support all kinds of matrix-matrix and matrix-vector operations.
The important difference is that
evaluation is lazy,
\ie only performed when needed,
and performs via building and working on an expression tree.



Whilst the primary application for lazy matrices in this thesis was the quantum-chemical
program package \molsturm,
lazy matrices are more general and could be used for other problems
of physics and chemistry as well.
An implementation of lazy matrices inside the
lazy matrix library \lazyten was presented in section \ref{sec:lazymat}
along with an example showing a simple \SCF
scheme coded in the language of lazyten \ref{sec:LazytenExamples}.

% ---------------------------------------------------------------------

Section \ref{sec:MolsturmDesign} discussed
the design of the \molsturm program package.
In particular it was indicated how
\molsturm has become a flexible quantum-chemistry framework,
where novel methods,
including novel basis function types,
can be easily implemented and tested
and where linking to third-party code is easy.
Because of this \molsturm can be thought of as a mediator
to produce \SCF results in a very general fashion
on top of which one may stick any Post-\HF method.

It was indicated in section \ref{sec:GscfGint}
how a combination of a contraction-based \SCF ansatz
with the lazy matrices of \lazyten
has made it possible to separate the code describing
the \SCF algorithms from the integral back end.
As such it has become possible to implement
an \SCF algorithm in the high-level language of \lazyten
in a basis-function independent way,
whilst the integral back end still has the full flexibility
to decide the numerical details
of the contraction expression in the way required by the basis function type.
Furthermore this has made implementing
additional integral back ends or basis function types very easy,
since only changes in one particular well-defined integral
interface are required to make
a new basis function type available to the rest of \molsturm,
including all third-party packages on top.
In this way a connection from \molsturm
to \libint~\cite{Libint2_231,Libint2} and \libcint~\cite{Sun2015}
for Gaussian-type integrals
and to \sturmint~\cite{sturmintWeb}
for Coulomb-Sturmian-type integrals has been achieved.

In section \ref{sec:MolsturmPython} it was described
how the \python interface of \molsturm
has been designed to allow full control over all aspects
of \molsturm's execution by a detailed set of parameters
and how the obtained results are returned
in readily usable data structures.
It was mentioned how this in turn
makes it very easy to combine \molsturm
with third-party packages
or automate calculations and further analysis.
So far the \python interface of \molsturm
has enabled to establish
links to selected methods
from \pyscf~\cite{Sun2017} and \adcman~\cite{Wormit2014}.

A brief overview of the test suite
ensuring that future restructuring of code does not break things is
given in section \ref{sec:MolsturmTestSuite}.
Together with the flexible design
indicated in \ref{sec:MolsturmDesign}
this makes sure that even if changes to the \SCF scheme
were needed in the future,
these could be made and validity of the state of molsturm would not
be influenced.

By the means of examples presented in \ref{sec:MolsturmExamples}
this was shown in practice.
The current state and possibilities of molsturm
were summarised in section \ref{sec:MolsturmState}.


% ------------------------------------

As an application of \molsturm its link to the
Coulomb-Sturmian integral library \sturmint~\cite{sturmintWeb} was used
in chapter \ref{ch:CSQChem} to perform an initial investigation
of the convergence properties of Coulomb-Sturmian-based
quantum-chemical calculations.
The main focus was on understanding the convergence
properties of \HF calculations of atoms
of the second and third period of the periodic table.
In section \ref{sec:CSconvergenceHF}
a detailed analysis based on the root mean square values
of the occupied coefficients per angular momentum~($\RMSOl$)
allowed to suggest that
a maximal angular momentum of $\lmax = 0$
is sufficient for Li and Be,
whereas $\lmax = 1$ is required for N, Ne, Na, Mg, P and Ar.
It further allowed to understand that the
underlying cause of the slow convergence of the other
atoms B, C, O, F, Al, Si, S and Cl is an issue due to \UHF.
At correlated level some Full-CI and {\MP}2 calculations
were performed,
which suggested that increasing the angular momentum
by one is sufficient to capture
most correlation effects for Li, Be, N, Ne, Na, Mg, P and Ar.
Furthermore the effect of modifying the Coulomb-Sturmian
exponent on the resulting \HF energies was analysed in section \ref{sec:kexp}
and a routine for estimating $\kopt$ as well as finding
optimal exponents, \ie those leading minimal energies,
was discussed in section \ref{sec:DetermineKopt}.
An analogy between the optimal exponent and
the effective nuclear charge was indicated in section \ref{sec:ValuesKopt}
and the roughly
linear relationship of $\kopt$ with the atomic number presented.
In section \ref{sec:SturmianADC} the connection from
\molsturm to \adcman~\cite{Wormit2014} via \python
was exploited to perform the first
excited states calculation based on the algebraic-diagrammatic construction
scheme for the polarisation propagator.
The inital results were reported,
which looked promising and motivating for further research.
