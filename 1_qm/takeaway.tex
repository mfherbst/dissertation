\section{Take-away}
Many observations of the one-particle Hydrogenic Schrödinger operator $\Op{H}$
of section \vref{sec:HydrogenAtom}
generalise to the more complicated many-body atomic and molecular Hamiltonians
we will introduce in chapter \vref{ch:qchem}.

Most importantly all these Hamiltonians are unbounded operators,
which become self-adjoint by making the domain
a subspace of the Sobolev spaces $H^2(\R^{3\Nelec}, \C)$.
Their essential spectrum is non-trival,
but luckily one can show~\todo{cite HVZ and Zhislin 61} that
\[ \forall \lambda \in \sigma_{\text{disc}}(\Op{H}),
	\mu \in \sigma_{\text{ess}}(\Op{H}) \quad \lambda < \mu, \]
\ie that the discrete spectrum always sits below the essential spectrum.

In remark \vref{rem:NumericalEssentialSpectrum} we discussed
that the essential spectrum cannot be approximated reliably
by a finite-dimensional Hilbert space with only compact operators.
Our best shot is therefore to follow a numerical approach,
which aims at the description of the low end of the spectrum.
This thus avoids $\sigma_{\text{ess}}(\Op{H})$
and allows to obtain good approximations to at least a few
eigenpairs corresponding to the discrete spectrum $\sigma_{\text{disc}}(\Op{H})$,
\ie bound states.

In practice this is hardly a restriction.
Originating from the laws of thermodynamics
a sensible quantum-mechanical description of a system
typically only requires the lowest energy state,
i.e. the \newterm{ground state}, and the next few \newterm{excited state}s
following most closely in energy.
Care only needs to be taken to choose a sensible approximation
method and a large enough approximation space.
Otherwise one cannot be sure whether the obtained eigenstates
are approximations to true discrete states
or spurious states originating from discretising scattering states
of the continuum.
