\section{Takeaway}
\label{sec:SpectralTakeAway}
% TODO OPTIONAL \to doil{Cold give a short forecast to the upcoming work}
Many observations of the one-particle hydrogen-like Schrödinger operator $\Op{H}$
of section \vref{sec:HydrogenAtom}
generalise to the more complicated many-body atomic and molecular Hamiltonians
we will introduce in chapter \vref{ch:qchem}.

Most importantly all these Hamiltonians are unbounded operators,
which become self-adjoint by making the domain
a subspace of the Sobolev spaces $H^2(\R^{3\Nelec}, \C)$.
Their essential spectrum is non-trival,
but luckily one can show~\cite{Zhislin1959,Zhislin1960,Reed1978,Teschl2014} that
\[ \forall \lambda \in \sigma_{\text{disc}}(\Op{H}),
	\mu \in \sigma_{\text{ess}}(\Op{H}) \quad \lambda < \mu, \]
\ie that the discrete spectrum always is located below the essential spectrum.

In remark \vref{rem:NumericalEssentialSpectrum} we discussed
that the essential spectrum cannot be approximated reliably
by a finite-dimensional Hilbert space with only compact operators.
Our best ansatz is therefore to follow a numerical approach,
which aims at the description of the low end of the spectrum.
This thus avoids $\sigma_{\text{ess}}(\Op{H})$
and allows to obtain reliable approximations to at least a few
eigenpairs corresponding to the discrete spectrum $\sigma_{\text{disc}}(\Op{H})$,
\ie~bound states.

Even though this is a restriction, one gets by for many cases.
The rationale for this are the laws of thermodynamics,
which imply that a sensible quantum-mechanical description of a system
typically only requires the lowest energy state,
\ie~the \newterm{ground state}, and the next few \newterm{excited state}s
following most closely in energy.
Assuming this is the case,
care only needs to be taken to choose a sensible approximation
method and a large enough approximation space.
Otherwise one cannot be sure whether the obtained eigenstates
are approximations to true discrete states
or spurious states originating from discretising scattering states
of the continuum.

This assumption does, however, break down for a couple of cases,
such as plasma states,
strong field physics or similar.
But even without extreme energies, the description of certain
processes such as resonance decayss or Rydberg-like states
requires the description of high-energy bound states,
which can be embedded inside the continuum,
\ie where the corresponding eigenvalues are part of the essential spectrum.
This makes a numerical modelling challenging.
We will mostly ignore this aspect in this work.
