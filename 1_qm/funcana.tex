\section{Elements of functional analysis}

In section \vref{sec:IntroQM} we discussed that the state
of a \QM system is governed by a vector in a complex-valued Hilbert space,
whilst information about the system itself can be obtained by studying
appropriate self-adjoint operators defined on such Hilbert spaces.
Precisely this is the mathematical field of functional analysis,
which we will briefly review in this section.
We assume basic familiarity with the concept of a vector space as well
as Lebesgue integrability for our treatment.
A more basic material developing the concept of Lebesgue integrability
and Hilbert spaces from standard Euclidean geometry can be found
in a related work by the author~\cite{DiracNotation}.

\subsection{Hilbert spaces}
\label{sec:Hilbert}
\nomenclature{$\F$}{Either the field of real numbers $\R$ or the field of complex numbers $\C$.}

\todoil{Rephrase}
The key structures,
which a Hilbert space possesses on top of a usual vector space
is the so-called completeness under the norm induced
by the inner product of the Hilbert space.
For our treatment the field $\F$ can be identified
either with the field of all real numbers $\R$
or the field of all complex numbers $\C$.

\begin{defn}
	An \newterm{inner product space} over a field $\F$
	is a vector space $V$ (over the same field)
	that is further equipped with an inner product, i.e. a map
	\[ \braket{~\cdot~}{~\cdot~}_V : V \times V \to \F \]
	that satisfies (for all vectors $x,y,z \in V$ and all $\alpha \in \set{F}$)
	\begin{align}
		\label{eqn:innProdConjSym}
			&\braket{x}{y}_V^\ast = \braket{y}{x}_V &&
			\text{\textit{(conjugate symmetry)}} \\
		\label{eqn:innProdLinLeft}
			&\braket{x}{\alpha y + z}_V = \alpha \braket{x}{y}_V + \braket{x}{y}_V &&
			\text{\textit{(linearity in the first argument)}} \\
		\label{eqn:innProdPosDef}
			&\braket{x}{x}_V \ge 0 \quad \text{and}
			\quad \braket{x}{x}_V = 0 ~\Rightarrow~ x = 0  &&
			\text{\textit{(positive-definiteness)}},
	\end{align}
	where the asterisk ``$^\ast$'' denotes complex conjugation.
	We typically drop the ``$V$'' subscript from the notation of the inner
	product if the underlying vector space is clear from context.
\end{defn}

\begin{defn}
	Given a vector space $V$ over the field $\F$, a \newterm{norm} is a map
	$\norm{~\cdot~}_V : V \to \R$
	such that the following axioms hold for all vectors $x,y \in V$ and all $\alpha \in \set{F}$:
	\begin{align}
		\label{eqn:normScalability}
			&\norm{\alpha x}_V = \abs{\alpha} \, \norm{x}_V &&
			\text{\textit{(absolute scalability)}} \\
		\label{eqn:normTriaIneq}
			&\norm{x + y}_V \le \norm{x}_V + \norm{y}_V &&
			\text{\textit{(triangle inequality)}} \\
		\label{eqn:normPointSep}
			&\text{If}~\norm{x}_V = 0 \quad \Rightarrow
			\quad \text{$x$ is the zero vector} &&
			\text{\textit{(norm separates points)}}
	\end{align}
	Again we may drop the subscript if the norm employed is clear from context.
\end{defn}

\begin{prop}
	For every inner product space exists the so-called \newterm{induced norm}
	\begin{equation}
		\norm{x} = \sqrt{ \braket{x}{x} } \qquad \forall x \in V.
		\label{eqn:normInduced}
	\end{equation}
	\begin{proof}
		See~\cite{DiracNotation}.
	\end{proof}
\end{prop}

\begin{defn}
	A vector space%
	\footnote{Completeness is a more general property,
		which also applies to sets with less structure
		called metric spaces}
	$V$ is called \newterm{complete} if every Cauchy sequence
	of vectors in $V$ has a limit which is in $V$ as well,
	i.e. if for every converging sequence in $V$ the limit
	is in $V$ as well.
\end{defn}

\begin{defn}
	A \newterm{Hilbert space} $\mathcal{H}$ is an inner product space,
	which is complete with respect to the induced norm.
\end{defn}

\nomenclature{$L^2(\R^N, \C)$}{The Hilbert space of square-integrable complex-valued functions, see \vref{prop:L2HilbertSpace}}
\begin{prop}
	\label{prop:L2HilbertSpace}
	Given two functions $f, g : \R^N \to \C$
	let us define the inner product
	\begin{equation}
		\braket{f}{g}_{L^2} := \int_{\R^N} \cc{f}(\vec{x}) g(\vec{x}) \D \vec{x}
		\label{eqn:defL2InnProd}
	\end{equation}
	and the corresponding induced norm
	\begin{equation}
		\norm{f}_{L^2} := \sqrt{\braket{f}{f}_{L^2}} = \left( \int_{\R^N} \abs{f(\vec{x})}^2 \D \vec{x} \right)^{1/2}.
		\label{eqn:defL2Norm}
	\end{equation}
	In both cases the integral is to be understood in the Lebesgue sense
	and we identify
	\[ \vec{x} \equiv (x_1, \ldots, x_N) \]
	for ease of notation.

	The set of \newterm{square-integrable functions}
	\[ L^2(\R^N, \C) := \big\{ f : \R^N \to \C ~\big|~ \norm{f}_{L^2} < \infty \big\}, \]
	where the norm $\norm{~\cdot~}_{L^2}$ stays finite
	is a Hilbert space over the field $\C$.
	\begin{proof}
		See~\cite{Adams2003}
	\end{proof}
\end{prop}

\todoil{Separable space}

\subsection{Hilbert spaces in Quantum Mechanics}
\todoil{Say that this is general and other domains are possible as well.}



\todoil{Discuss which Hilbert space to take for the \QM description
and give a brief rationalisation why}


		
		in \QM.




		in \QM we employ functions $\Psi : \R^N \to \C^m$ from a
		complex separable Hilbert space $\mathcal{H}$
		to describe the state of the system.
		The number of spin components $m$ is determined by the precise
		\QM treatment we employ.
		For the remainder of this chapter we will restrict ourselves
		to a non-relativistic spin-free \QM, which implies $m=1$.




		The so-called \newterm{Born interpretation} of \QM
		associates the meaning of a probability density
		with the norm $\norm{\Psi(x_1, \ldots x_N)}$
		of a state at each point in space $(x_1, \ldots x_N)$.
		
		
		with a probability density at a particular
		point in space $(x_1, \ldots x_N)$.
		

		Such a statistical 
		This requires the $\Psi$

		Since the Born interpretation associates with the
		norm $\norm{\Psi(x_1, \ldots x_N)}$ of the state at a particular
		point in space $(x_1, \ldots x_N)$ in 

		For reasons related to the Born interpretation,
		which associates with the modulus $\norm{\Psi(x_1, \ldots x_N)}$

		Typical choices for the Hilbert space are the Lebesgue spaces
		$L^2(\R^N, \C^m)$,
		where $m$ is $1, 2$ or $4$ depending on the number of spin components.


\subsection{Sobolev spaces}
Many operators of quantum mechanics involve derivatives as well.
Whilst Lebesgue spaces are natural when it comes to doing statistics,
they are too weak to make sure 

\todo[inline,caption={}]{
	If needed: \\
\begin{defn}
	A set $\Omega$ is bounded iff 
\end{defn}
\begin{defn}
	A set $\Omega$ is compact iff it is bounded and complete.
\end{defn}

\nomenclature{$C^\infty_0(\R^N, \C)$}{The Banach space of infinitely differentiable
complex-valued functions with compact support}
\begin{defn}
	Define $C^\infty_0$
\end{defn}
}

\begin{defn}
	A function $f \in L^2(\R^N, \C)$ has a \newterm{weak partial derivative}
	$g \in L^2(\R^N, \C)$ with respect to $x_i$ if
	\[ \forall \varphi \in C^\infty_0(\R^N): \braket{g}{\varphi}_{L^2} = - \braket{f}{\frac{\partial}{\partial x_i}\varphi}_{L^2}, \]
	where $C^\infty_0(\R^N)$ is the of infinitely differential
	functions $\R^N \to \C$ with compact support.
	To denote the weak derivative
	one may write $g = \partial_{x_i} f$ like in the usual case.
	It can further be shown that if $f$ has a classical, strong derivative
	then it also has a weak derivative, which coincides with the strong derivative.
	
	For ease of notation we also write 
	\[ D^\alpha f \]
\end{defn}

\nomenclature{$H^1(\R^N, \C)$}{The Hilbert space of complex-valued functions with square-integrable first derivative}
\nomenclature{$H^2(\R^N, \C)$}{The Hilbert space of complex-valued functions with square-integrable second derivative}
\begin{defn}
	Definition of \newterm{Sobolev space}
	\begin{equation}
		H^k(\R^N, \C) := \left\{ f \in L^2(\R^N, \C) \middle| D^\alpha f \in L^2(\R^N, \C) \text{ for $\abs{\alpha} \le k$} \right\}
		\label{eqn:defSobolev}
	\end{equation}
	with inner product
	\[ \braket{f}{g}_{H^k} := \sum_{\abs{\alpha} \le k} \braket{D^\alpha f}{D^\alpha g}_{L^2}. \]
	All spaces of this sequence are Hilbert spaces
\end{defn}

\begin{defn}
	The completion of $C^\infty_0(\R^N, \C)$ with respect to the norm $\norm{~\cdot~}_{H^k}$ is called $H^k_0(\R^N, \C)$. It is a proper subspace of $H^k$ and also a Hilbert space.
\end{defn}

\begin{defn}
	A subset $S$ is dense in a complete metric space $M$ if it is a proper subset
	and each sequence in $S$ which is Cauchy converges to a point in $M$.
\end{defn}

Overview of dense embeddings to do with Sobolev spaces
and how completion under respective norms yields different sobolev spaces.



$C^\infty_0$ is dense in $L^2$ (Hachbusch 1986)
and dense in $H^1$ and dense in $H^1_0$

also define $H_0^n(\Omega, \C)$
Give inclusions including $C^\infty_0$
Mention what is dense

\begin{thm}
	Sobolev embedding theorem
\end{thm}
