\section{Elements of functional analysis}

In section \vref{sec:IntroQM} we discussed that the state
of a \QM system is governed by a vector in a complex-valued Hilbert space,
whilst information about the system itself can be obtained by studying
appropriate self-adjoint operators defined on such Hilbert spaces.
Precisely this is the mathematical field of functional analysis,
which we will briefly review in this section.
We assume basic familiarity with the concepts of a field and a vector space
in this section.
\todoil{Cite standard literature}
See the standard mathematical literature or the handout for an earlier talk
by the author~\cite{DiracNotation} for more information about these concepts.


\subsection{Hilbert spaces}
\label{sec:Hilbert}
\nomenclature{$\F$}{Either the field of real numbers $\R$ or the field of complex numbers $\C$.}







\begin{defn}
	A vector space over a field $\F$ is a non-empty set $V$
	with the binary operations
	\begin{itemize}
		\item Vector addition: $V \times V \to V : (x,y) \mapsto x+y$
		\item Scalar multiplication $\F \times V \to V: (\alpha,x) \mapsto \alpha x$
	\end{itemize}
	that satisfy some axioms listed below. The elements of $V$ are referred to as vectors
	and the elements of the field $\F$ are called scalars.

	The axioms to be satisfied are:
	\begin{itemize}
		\item $(V,+)$ is an abelian group, i.e. it holds
		\begin{align}
			\label{eqn:vectorAss}        \forall x,y,z \in V: &\quad x + (y + z) = (x + y) + z		&&\text{"associativity of addition"} \\
			\label{eqn:vectorZero}       \exists 0 \in V: &\quad \forall x \in V : x + 0 = x		&&\text{"existence of zero vector"} \\
			\label{eqn:vectorInverse}    \forall x \in V: &\quad \exists -x \in V : x + (-x) = 0	&&\text{"existence of additive inverses"} \\
			\label{eqn:vectorComm}       \forall x,y \in V: &\quad x + y = y + x			&&\text{"commutativity of addition"}
		\end{align}
		\item The so-called compatibility of scalar multiplication and
			field multiplication:
			\begin{equation}
				\forall \alpha,\beta \in \F, x \in V: \quad (\alpha \beta) x = \alpha (\beta x)
				\label{eqn:vectorCompField}
			\end{equation}
		\item The scalar multiplication with the multiplicative identity of
			$\F$ (i.e. the scalar $1$) should satisfy
			\begin{equation}
				\forall x \in V: \quad 1\, x = x
				\label{eqn:vectorMultIdent}
			\end{equation}
		\item The following distributivity relations:
			\begin{align}
				\label{eqn:vectorDistrVecAdd}\forall \alpha \in \set{F}, x,y \in V: & \quad \alpha (x+y) = \alpha x + \alpha y\\
				\label{eqn:vectorDistrFldAdd}\forall \alpha,\beta \in \set{F}, x \in V: & \quad (\alpha + \beta) x = \alpha x + \beta x
			\end{align}
	\end{itemize}
\end{defn}

\begin{nte}
	As stated above, this is all rather technical, so let us sum briefly up the rationale and the key ideas:
	\begin{itemize}
		\item Recall the picture of \eqref{eqn:vectorEuclidAdd} where we pasted arrows together in order to add them. 
			It is obvious that such an operation should satisfy \eqref{eqn:vectorAss} and \eqref{eqn:vectorComm}, just because the precise order in which we apply the vectors does not matter.
			Surely if one adds the zero vector to any arrow, that arrow is unchanged, rationalising \eqref{eqn:vectorZero}.
			\eqref{eqn:vectorInverse} is also easy to understand if one sees the inverse as the original arrow with head and tail swapped.
		\item Condition \eqref{eqn:vectorCompField} establishes a formal consistency when scaling vectors multiple times.
		\item \eqref{eqn:vectorMultIdent} gives us a uniquely defined scalar multiplication, since it defines the outcome of the multiplication operator for one specific scalar.
			The result for other scalars can than be deduced by applying distributivity and \eqref{eqn:vectorCompField}. 
			It is remotely comparable to choosing the potential offset when calculating the energy.
		\item The distributivity relations \eqref{eqn:vectorDistrVecAdd} and \eqref{eqn:vectorDistrFldAdd} just give us the intuitive result that the order of vector/scalar addition and scaling of vectors does not matter.
	\end{itemize}
\end{nte}

\noindent
The next step is to add a scalar product, also known as an inner product.
\begin{defn}
	An \emph{inner product space} over $\set{F}$ is a vector space $V$ (over the same field) that is further equipped with an inner product, i.e. a map
	\[ (\cdot,\cdot) : V \times V \to \set{F} \]
	that satisfies (for all vectors $x,y,z \in V$ and all $\alpha \in \set{F}$)
	\begin{align}
		\label{eqn:innProdConjSym}  &(x,y)^\ast = (y,x) &&                                                 \text{"Conjugate symmetry"} \\
		\label{eqn:innProdLinLeft}  &(\alpha x + y,z) = \alpha (x,z) + (y,z) &&                            \text{"Linearity in the first argument"} \\
		\label{eqn:innProdPosDef}   &(x,x) \ge 0 \quad \text{and} \quad (x,x) = 0 ~\Rightarrow~ x = 0  &&  \text{"Positive-definiteness"}
	\end{align}
	here the asterisk $^\ast$ denotes complex conjugation.
\end{defn}

\begin{rem}
	%TODO: have that?
	Some other well-known properties of the inner product are in fact not axioms but can be easily deduced by combining two of the above:
	\begin{itemize}
		\item Conjugate symmetry \eqref{eqn:innProdConjSym} implies \emph{antilinearity} (or \emph{conjugate linearity}) in the second argument
			\begin{equation}
				(z, \alpha x + y) = \alpha^\ast (z,x) + (z,y)
				\label{eqn:innProdAntiLinRight}
			\end{equation}
			In other words the inner product is what is called a \emph{sesquilinear form}.

		\item If $\set{F} = \R$, then Conjugate symmetry \eqref{eqn:innProdConjSym} simplifies to proper symmetry, ie
			\[ (x,y) = (y,x). \]
			The antilinearity in the second argument then also reduces to proper linearity
			\begin{equation}
				(z, \alpha x + y) = \alpha (z,x) + (z,y),
				\label{eqn:innProdLinRight}
			\end{equation}
			such that this time we get a so-called \emph{bilinear form}.
	\end{itemize}
\end{rem}

\begin{nte}
	From your previous encounters with Dirac notation $\langle \cdot | \cdot \rangle$ you might think that it satisfies all these properties and should be regarded as an inner product as well.
	In a loose sense this is not actually a bad description and it probably suffices as working knowledge.
	From a mathematically precise perspective this is, however, not correct. 
	Nevertheless both objects, the inner product $(\cdot,\cdot)$ and the Dirac bra-c-ket $\langle \cdot | \cdot \rangle$, are highly related. This will become clear from the formal definition of the latter, which we will present later.
\end{nte}



\noindent
Next we proceed to discuss the norm:
\begin{defn}
	Given a vector space over the field $\set{F}$, a \emph{norm} is a map
	\[ \norm{\cdot} : V \to \R \]
	such that the following axioms hold for all vectors $x,y \in V$ and all $\alpha \in \set{F}$:
	\begin{align} 
		\label{eqn:normScalability}  &\norm{\alpha x} = \abs{\alpha} \norm{x} &&                           \text{"Absolute scalability"} \\
		\label{eqn:normTriaIneq}     &\norm{x + y} \le \norm{x} + \norm{y}    &&                           \text{"Triangle inequality"} \\
		\label{eqn:normPointSep}     &\text{If}~\norm{x} = 0 \quad \Rightarrow \quad \text{$x$ is the zero vector} &&  \text{"Norm separates points"}
	\end{align}
\end{defn}

\begin{rem}
	Whilst absolute scalability \eqref{eqn:normScalability} seems like a reasonable thing to ask for when defining the norm (again think about scaling arrows), the other two might not be as obvious from the start.
	\begin{itemize}
		\item The abstract formulation of the triangle inequality \eqref{eqn:normTriaIneq} is --- as the name suggests --- a direct generalisation of the geometric constraints for a basic triangle:
			One side always has to be shorter than the sum of the two others. 
			From the triangular structure that vector addition gives (see fig. \vref{fig:TriangleInequality}) the axiom can be easily illustrated in the case of Euclidean vectors.
		\item The norm separates points condition again is one of those axioms to give a uniquely defined structure, like \eqref{eqn:vectorMultIdent} for the scaling of vectors.
			The choice is the most natural as it leads to a positive norm which can be seen as follows.
			From absolute scalability we have $\norm{-x} = \norm{x}$ for each vector $x \in V$ and furthermore for the zero vector $\norm{0} = 0$.
			By the triangle inequality $0 = \norm{0} = \norm{x-x} \le \norm{-x} + \norm{x} = 2 \norm{x}$. 
			Dividing both sides by 2 gives the result $\norm{x} \ge 0$.
	\end{itemize}
\end{rem}


\begin{prop}
	For every inner product space there exists a norm given by 
	\begin{equation}
		\norm{x} = \sqrt{ (x,x) } \qquad \forall x \in V,
		\label{eqn:normInduced}
	\end{equation}
	the so-called \emph{induced norm}.
\end{prop}
\begin{proof}
	Consider arbitrary $x,y \in V$ and $\alpha \in \set{F}$
	\begin{itemize}
		\item The point separating property \eqref{eqn:normPointSep} is a trivial consequence of \eqref{eqn:innProdPosDef}.
		\item We get absolute scalability from linearity \eqref{eqn:innProdLinLeft} and antilinearity \eqref{eqn:innProdAntiLinRight} of the scalar product
			\[ \norm{\alpha x} = \sqrt{ (\alpha x,\alpha x) } = \sqrt{ \alpha \alpha^\ast (x,x) } = \abs{\alpha} \norm{x} \]
		\item Using \eqref{eqn:normPointSep} and again linearity we find
			\begin{align}
				\nonumber
				0 &\le \norm{x \pm y}^2 \\
				\nonumber
				  &= (x \pm y, x \pm y) = (x,x) \pm (x,y) \pm (y,x) + (y,y) \\
				\nonumber
				  &= (x \pm y, x \pm y) = (x,x) \pm (x,y) \pm (x,y)^\ast + (y,y) \\
				  &= \norm{x}^2 + \norm{y}^2 \pm 2\Re (x,y)
				\label{eqn:normInducedProof1}
			\end{align}
			Now define normalised vectors $\tilde{x} = \frac{x}{\norm{x}}$ and $\tilde{y} = \frac{y}{\norm{y}}$.
			In an analogous way to the negative branch in \eqref{eqn:normInducedProof1} above, we have
			\[ 0 \le \norm{\tilde{x}}^2 + \norm{\tilde{y}}^2 - 2\Re (\tilde{x},\tilde{y}) = 2 - 2 \Re (\tilde{x},\tilde{y}) ,\]
			such that $\Re  (\tilde{x},\tilde{y}) \le 1$.
			Now consider
			\begin{align}
				\nonumber
				\left(\norm{x} + \norm{y}\right)^2 - \norm{x + y}^2 &\stackrel{\eqref{eqn:normInducedProof1}}{=} 2 \norm{x} \norm{y} - 2 \Re (x,y) \\
				\nonumber
				&= 2 \norm{x} \norm{y} - 2 \Re ( \norm{x} \tilde{x}, \norm{y} \tilde{y} )  \\
				\nonumber
				&= 2 \norm{x} \norm{y} \left[ 1 - \Re (\tilde{x},\tilde{y}) \right] \\
				&\ge 0
				\label{eqn:normInducedProof2}
			\end{align}
			where we used linearity in the penultimate step, to get the (real) factors $\norm{x}$ and $\norm{y}$ out front. 
			Taking the square root proves the triangle inequality \eqref{eqn:normTriaIneq}.
	\end{itemize}
\end{proof}

\begin{cor}[Cauchy-Schwarz Inequality]
	\label{eqn:CS}
	Let $V$ be an inner product space with inner product $(\cdot,\cdot)$ and induced norm $\norm{\cdot}$. Then for all $x,y \in V$
	\[ \norm{x} \norm{y} \ge \Re (x,y) \]
	and furthermore
	\[ \sqrt{2} \norm{x} \norm{y} \ge \abs{(x,y)}. \]
\end{cor}
\begin{proof}
	The first part has been implicitly achieved in \eqref{eqn:normInducedProof2}. For the second part, note
	\[ \abs{(x,y)}^2 = \Big( \Re (x,y) \Big)^2 + \Big( \Im (x,y) \Big)^2  = \Big( \Re (x,y) \Big)^2 + \Big( \Re (y,x) \Big)^2 \le 2 \norm{x}^2 \norm{y}^2.\]
	Taking the square-root shows the claim.
\end{proof}

\noindent
Finally, we arrive at the definition of a \emph{Hilbert space}:
\begin{defn}
	A Hilbert space $V$ is an inner product space over the field $\set{F}$, that is complete with respect to the induced norm.
\end{defn}

\begin{nte}
	A formal discussion of completeness is omitted here.
	The picture we developed in section \vref{ideaCompleteness} will be sufficient for our purposes and shall serve as an intuitive guideline whenever the term \emph{complete} arises. \\

	As we said in the last chapter, Euclidean space is indeed complete under its induced norm and can hence be regarded as a Hilbert space.
	We therefore succeeded in finding an abstract formalism that represents large parts of the internal mathematical structure of Euclidean space.
\end{nte}

\section{Lebesgue space}
In this subsection we will try to get a little insight into the space of square-integrable functions $L^2$.
As it will turn out later, this space is the mathematical object on which Dirac Notation lives in most practical scenarios.
In a similar fashion to completeness, the precise formulation of \emph{integrability} is well out of scope here\footnote{In fact the whole topic of measure theory deals with just this definition.}. 
For us this section is merely meant to scratch the surface and introduce the formalism we will need later on. \\

\noindent
The first problem one runs into very quickly when considering square-integrable functions is that our familiar Riemann integral definition is too tight. 
This means that there are too few functions that can be properly Riemann-integrated in order to prove the theorems in this subsection.
The way to make progress is to broaden the definition of the integral and integrability. 
Hence we consider

\begin{rem}[The Lebesgue integral] 
	\label{intLebesgue}
	Instead of a formal definition of the Lebesgue integral, we will just state a few of its properties in order to get a feeling for how it works.
	\begin{itemize}
		\item Whenever the Riemann integral exists, the values of the Riemann and the Lebesgue integrals agree. 
			The only difference between the two integral definitions is that there exist some ``problematic cases'' where the Lebesgue integral still allows integration, but the Riemann integral does not.
			So if the Riemann integral exists (i.e. if we can do the usual integration and the result is a finite value), we can evaluate it and we are done for the calculation of the integral in both definitions.
		\item For a Lebesgue integral over a domain $\Omega \subset \R^1$, changing the value of the integrand at up to countably infinite number of points (in an arbitrary way) does not change the value of the integral. 
			Note, that this means for example 
			\begin{align}
				\nonumber
				&\int_\R \left\{
				\begin{array}{ll}
					1 & \text{for $x \in \set{Q}$} \\
					0 & \text{for $x \in \R \backslash \set{Q}$}
				\end{array}
				\right\} \D[x] &&\text{only Lebesgue-integrable} \\
				%
				\nonumber
				= &\int_\R 0 \D[x] && \text{Lebesgue- and Riemann-integrable}\\
				= &\int_\R \left\{
				\begin{array}{ll}
					1 & \text{for $x \in \{\pi, e, 3, 15, 42\}$} \\
					0 & \text{else}
				\end{array}
				\right\} \D[x] &&\text{Lebesgue- and Riemann-integrable}, 			
				\label{exampleLebesqueIntegral}
			\end{align}
			which trivially evaluates to zero (Consider the Riemann-integrable second term).
			For higher dimensions analogous statements are true:
			e.g. for integrals over subsets of $\R^2$, we can change the value of the integrand at up to countably infinite number of lines or points; 
			similarly for 3D integrals at planes, lines or points, \ldots. 
			Often this allows us to alter an integrand in a way that integration in a Riemann sense is possible. 
			For example in the first integral of \eqref{exampleLebesqueIntegral}, we changed the points where the function is 1 to 0 in order to get a Riemann-integrable version. 
			This is possible since $\set{Q}$ is a set with only up to countably infinite number of elements\footnote{It is not necessarily intuitive to understand, what this means. Just imagine there are two kinds of infinities: For one kind one can abstractly speaking find a way to enumerate each element (the ``countable infinity''), for the other one this is not possible.}.
		\item A direct consequence of the above point is that for each $\Omega \subset \R^n$ there exist an infinitely large number of functions $f : \Omega \to \C$, such that $\int_\Omega f \D[x] = 0$. This will be important very soon.
		\item From now on all integrals in this script are to be understood as Lebesgue integrals.
	\end{itemize}
\end{rem}

\noindent
That being said, let us consider complex integrability. 
\begin{defn}
We call a \emph{complex-valued} function $g : \Omega \to \C$ \emph{integrable} on the domain $\Omega \subset \R^n$ if
	\begin{equation}
		\abs{ \int_\Omega g(x) \D x } < \infty,
		\label{defIntegrable}
	\end{equation}
i.e. if the modulus of the integral over the domain remains finite. 
\end{defn}
\begin{defn}
	A complex-valued function $f : \Omega \to \C$ is called \emph{square-integrable} on the domain $\Omega \subset \R^n$ if
	the function $\abs{f}^2 : \Omega \to \R : x \mapsto \abs{f(x)}^2$ is integrable on that very same domain. 
	This means nothing else but 
	\[ \int_\Omega \abs{f}^2 \D x < \infty. \]
\end{defn}

\nomenclature{$L^2(\Omega, \C)$}{The Hilbert space of square-integrable complex-valued functions, see \vref{lem:L2HilbertSpace}}
\begin{nte}
	Usually one denotes the set of square-integrable functions as
	\begin{equation} 
		L^2(\Omega, \C) := \big\{ f : \Omega \to \C~\big|~\text{$f$ is square-integrable} \big\}.
		\label{defL2}
	\end{equation}
\end{nte}

\pagebreak[2]
\noindent
Before we will eventually proceed to show that $L^2(\Omega)$ is actually a Hilbert space we need a few more intermediate results.
\begin{lem}[Minkowski’s inequality in $L^2$]
	\label{lemMinkowski}
	Let $f,g \in L^2(\Omega)$, i.e. square-integrable, then
	\begin{itemize}
		\item $f + g$ is square-integrable on $\Omega$ as well
		\item $\sqrt{\int_\Omega \abs{f + g}^2 \D[x]} \le \sqrt{\int_\Omega \abs{f}^2 \D[x]} + \sqrt{\int_\Omega \abs{g}^2 \D[x]}$
	\end{itemize}
\end{lem}
\begin{proof}
	See for example \cite{Adams2003}.
\end{proof}

\begin{nte}
	When we say the resulting function $f \circ g$ of an operation $\circ$ is to be \emph{defined pointwise},
	we mean that $f\circ g$ can be defined by considering the action of $\circ$ in each point $x \in \Omega$ on the function values $f(x)$ and $g(x)$.
	For example the function $f+g$ from lemma \ref{lemMinkowski} could be defined as the function for which
	\[ \Big(f+g\Big)(x) = f(x) + g(x) \qquad \forall x \in \Omega.\]
	Similarly $\Re f$ is the function with
	\[ \Big( \Re f \Big)(x) = \Re \Big( f(x) \Big) \qquad \forall x \in \Omega.\]
\end{nte}

\begin{lem}
	\label{aLittleHoelder}
	For all complex-valued functions $f,g$, which are defined and integrable on $\Omega \subset \R^n$
	\[ \int_\Omega \abs{ f g } \D x \le \sqrt {\int_\Omega \abs{f}^2 \D x} \cdot \sqrt{\int_\Omega \abs{g}^2 \D x}. \]
\end{lem}
\begin{proof}
	This is a special case of the so-called Hölder's inequality, a proof of which can be found in \cite{Adams2003}.
\end{proof}
\noindent
From this we deduce
\begin{lem}
	\label{WellDefL2InnProd}
	For all complex-valued functions $f,g$ integrable on $\Omega \subset \R^n$, it holds
	\begin{align*}
		&\abs{ \int_\Omega f g^\ast \D x } \le \sqrt{ \int_\Omega \abs{f}^2 \D x} \cdot \sqrt{\int_\Omega \abs{g}^2 \D x} \\
		\intertext{and}
		&f,g \in L^2(\Omega) \quad \Rightarrow \quad \text{$f g^\ast$ is integrable},
	\end{align*}
	where the function $f g^\ast$ is to be constructed from $f$ and $g$ in a pointwise sense.
\end{lem}
\begin{proof}
	Consider first an integrable function $h$ which is defined on $\Omega \subset \R^n$, we have
	\[ \int_\Omega \abs{h} \D x = \int_\Omega \sqrt{ \left( \Re h \right)^2 + \left( \Im h \right)^2 } \D x \ge \int_\Omega \abs{ \Re h } \D x.\]
	Noting that $\abs{ \Re h } \ge \pm \Re h$ and thus 
	\[ \int_\Omega \abs{ \Re h } \D x \ge \pm \int_\Omega \Re h \D x\]
	we get
	\begin{equation}
		\int_\Omega \abs{h} \D x \ge \abs{\int_\Omega \Re h \D x}.
		\label{ProofWellDefL2InnProd1}
	\end{equation}
	Now note that for each complex number $z$ we may find a $\phi \in \R$ such that $z\exp(\I \phi)$ is a real number. 
	So specifically we can choose $\phi$ such that
	\begin{align*}
		\exp(\I \phi) \int_\Omega h \D x &= \Re \left( \exp(\I \phi) \int_\Omega h \D x \right) \\
		&=\Re \left(  \int_\Omega \exp(\I \phi) h \D x \right) \\
		&=\int_\Omega \Re \Big( \exp(\I \phi) h \Big) \D x \\
	\end{align*}
	and therefore
	\[ \abs{\int_\Omega h \D x} = \abs{\exp(\I \phi) \int_\Omega h \D x} = \abs{\int_\Omega \Re \Big( \exp(\I \phi) h \Big) \D x} \stackrel{\eqref{ProofWellDefL2InnProd1}}{\le} \int_\Omega \abs{\exp(\I \phi)  h} \D x = \int_\Omega \abs{h} \D x \]
	Now set $h = f g^\ast$ and use lemma \ref{aLittleHoelder} to obtain the first result
	\begin{equation}
		\abs{ \int_\Omega f g^\ast \D x } \le \int_\Omega \abs{f g^\ast}\D x = \int_\Omega \abs{f g}\D x \le  \sqrt{\int_\Omega \abs{f}^2 \D x} \cdot \sqrt{\int_\Omega \abs{g}^2 \D x}.
		\label{ProofWellDefL2InnProd2}
	\end{equation}
	If furthermore $f,g \in L^2(\Omega)$ then the integrals in the RHS of \eqref{ProofWellDefL2InnProd2} are finite (since the functions are square-integrable) and thus
	\[ \abs{ \int_\Omega f g^\ast \D x } \le \int_\Omega \abs{f g}\D x \le  \sqrt{\int_\Omega \abs{f}^2 \D x} \cdot \sqrt{\int_\Omega \abs{g}^2 \D x} < \infty, \]
	which implies the second result, that $f g^\ast$ is integrable.
\end{proof}

%TODO
\begin{rem} For this lemma it is easy to see that an equivalent version cannot be constructed if the Riemann integral definition is to be used. 
	To see this first define the two complex-valued functions $f,g : \R \to \C$ with
	\begin{align*}
		f(x) &= \left\{
		\begin{array}{ll}
			-1 & \text{for $x \in \set{Q}$} \\
			1 & \text{for $x \in \R \backslash \set{Q}$}
		\end{array}
		\right.
		&
		g(x) &= \left\{
		\begin{array}{ll}
			\I & \text{for $x \in \set{Q}$} \\
			1 & \text{for $x \in \R \backslash \set{Q}$}
		\end{array}
		\right. .
	\end{align*}
	Clearly 
	\[ \forall x \in \R:\qquad \abs{f(x)} = \abs{g(x)} = 1,\]
	such that in both the Riemann and the Lebesgue sense
	\[ \int_{[0,1]} \abs{f}^2 \D x = \int_{[0,1]} 1 \D x = 1 = \int_{[0,1]} 1 \D x = \int_{[0,1]} \abs{g}^2 \D x.\]
	This makes $f$ and $g$ square-integrable functions on the domain $\Omega = [0,1]$. 
	On the other hand 
	\[ \big(f g^\ast\big)(x) = \left\{
		\begin{array}{ll}
			\I & \text{for $x \in \set{Q}$} \\
			1 & \text{for $x \in \R \backslash \set{Q}$}
		\end{array}
		\right., \]
	which makes $(f g^\ast)$ Lebesgue, but not Riemann integrable. 
	For lemma \ref{WellDefL2InnProd} this provides a counterexample if the Riemann integral definition is used.
\end{rem}

\begin{lem}
	\label{lem:L2HilbertSpace}
	For $\Omega \subset \R^n$, $L^2(\Omega)$ is an inner product space over $\C$ with the inner product given by the form
	 \begin{equation}
		(u,v)_\Omega = \int_\Omega u v^\ast \D x.
		\label{L2innerProduct}
	\end{equation}
	and a pointwise definition of vector addition and scalar multiplication, i.e. for functions $f$ and $g$
	\[ \Big(f+g\Big)(x) = f(x) + g(x) \qquad \forall x \in \Omega\]
	and a scalar multiple of $f$ with $\alpha \in \C$
	\[ \Big(\alpha f\Big)(x) = \alpha f(x)\qquad \forall x \in \Omega.\]

\end{lem}
\begin{proof}
	First of all we need to verify that the pointwise vector operations are well-defined. 
	This is the case if for all $f,g \in L^2(\Omega)$ and all $\alpha \in \C$ uniquely defined functions $\alpha f$ and $f+g$ exist that are also a member of $L^2(\Omega)$.
	Uniqueness follows from the properties of $\C$. 
	For the sum $f+g$ the first part of Minkowski’s inequality \eqref{lemMinkowski} guaranties that $f+g \in L^2(\Omega)$ if $f,g \in L^2(\Omega)$.
	For the scalar product this is a consequence of the linearity of the integral, i.e.
	\begin{align*}
				&\text{$f$ is square-integrable on $\Omega$} \\
		\Leftrightarrow \quad &\int_\Omega \abs{f(x)}^2 \D x < \infty \\
		\Leftrightarrow \quad& \int_\Omega \abs{\alpha f(x)}^2 \D x = \abs{\alpha}^2~\underbrace{\int_\Omega \abs{f(x)}^2 \D x}_{<\infty} < \infty \qquad \forall \alpha \in \C \\
		\Leftrightarrow \quad& \text{$\alpha f$ is square-integrable on $\Omega$} \qquad \forall \alpha \in \C
	\end{align*}
	Showing now that $L^2(\Omega)$ is a vector space is pretty simple.
	The zero vector is just the zero function and the additive inverse to $f$ is $-f$, i.e. the function with the signs of all function values reversed. 
	To illustrate the argument we will show that the distributivity relation \eqref{vectorDistrVecAdd} holds. 
	The remaining steps are left as an exercise for the reader.
	
	\noindent
	So consider arbitrary $\alpha \in \C$ and arbitrary $f,g \in L^2(\Omega)$. We have 
	\[ \forall x \in \Omega: \qquad \Big( \alpha (f+g) \Big)(x) = \alpha (f(x) + g(x)) = \alpha f(x) + \alpha g(x) = \Big( \alpha f + \alpha g \Big)(x). \]
	Therefore the functions given by $\alpha (f+g)$ and $\alpha f + \alpha g$ have to be identical (since they are identical at all points $x$), which indeed proves \eqref{vectorDistrVecAdd}. \\

	\noindent
	Now we try to show that $L^2(\Omega)$ is an inner product space. 
	The integral in the definition of the inner product \eqref{L2innerProduct} exists due to the second part of lemma \ref{WellDefL2InnProd}, which gives that $u v^\ast$ is integrable.
	Conjugate symmetry \eqref{innProdConjSym} and linearity in the first argument \eqref{innProdLinLeft} are trivially satisfied by \eqref{L2innerProduct}, due to the properties of the integral itself.
	Furthermore for all $f \in L^2(\Omega)$
	\[(f,f)_\Omega = \int_\Omega f f^\ast \D x = \int_\Omega \abs{f}^2 \ge 0.\]
	The remaining point to show in order to verify that \eqref{L2innerProduct} gives indeed a positive-definite inner product is
	\begin{equation}
		(f,f)_\Omega = 0 \Rightarrow f = 0.
		\label{eqnL2PosDef}
	\end{equation}
	Recall the properties of the Lebesgue integral from remark \ref{intLebesgue}. 
	If the integral in $(f,f)_\Omega$ is zero for one specific function $f$ then we can construct arbitrarily many functions for which this is the case as well.
	In a strict sense statement \eqref{eqnL2PosDef} can hence never be satisfied. 
	The mathematical trick we employ here to overcome this limitation is to redefine the meaning of the equality $=$ when comparing functions from $L^2(\Omega)$.
	One calls this new meaning of $=$ ``equality almost everywhere'' and it describes something along the lines\footnote{The mathematical precise formalism uses the notion of an equivalence relation and factor groups for the proper definition of $L^2(\Omega)$.} of
	\[ f = g \quad \text{almost everywhere} \qquad \Leftrightarrow \qquad \int_\Omega f \D x = \int_\Omega g \D x .\]
	In this weaker sense of equality between the $L^2(\Omega)$ functions $f$ and $0$, the relation \eqref{eqnL2PosDef} can indeed be proved, which makes $L^2(\Omega)$ a proper inner product space.
\end{proof}

\noindent
The last step --- showing that $L^2(\Omega)$ is a Hilbert space --- is pretty involved and we will just state the result here:
\begin{thm}
	\label{l2hilbert}
	For (most) $\Omega \subset \R^n$, $L^2(\Omega)$ is a Hilbert space under the induced norm
	\begin{equation}
		\norm{f}_{L^2(\Omega)} = \sqrt{(f,f)_\Omega} = \sqrt{\int_\Omega \abs{f}^2 \D x} .
		\label{L2norm}
	\end{equation}
\end{thm}
\begin{proof}
	See \cite{Adams2003}
\end{proof}

\begin{rem}
	Now what have these 4 pages of (more or less) rigorous mathematical labour given us?
	The important point is that we were able to show how the infinite dimensional function space $L^2(\Omega)$ behaves abstractly very similar to the finite dimensional Euclidean space.
	In other words one could well do analysis or linear algebra with functions instead of numbers or vectors. 
	This might seem a little odd in the first place, but the fact that $L^2(\Omega)$ is a Hilbert space lets us deal with this in a fairly intuitive way.
	For Quantum Mechanics, where the wave functions are pretty much all members of (subspaces of) $L^2(\Omega)$, getting an idea of the structure of $L^2(\Omega)$ and the operations we can perform in this space can become quite important.
\end{rem}


\nomenclature{$H^1(\Omega, \C)$}{The Hilbert space of complex-valued functions with square-integrable first derivative}
\nomenclature{$H^2(\Omega, \C)$}{The Hilbert space of complex-valued functions with square-integrable second derivative}



Sobolev spaces lift 



\begin{defn}
	% bounded
\end{defn}

\begin{defn}
	A metric space $M$ is complete if every sequence in $M$
	which is Cauchy converges to a limit, which is in $M$ as well.
\end{defn}

\begin{defn}
	A set is compact if it is bounded and complete.
\end{defn}

\begin{defn}
	A subset $S$ is dense in a complete metric space $M$ if it is a proper subset
	and each sequence in $S$ which is Cauchy converges to a point in $M$.
\end{defn}

$C^\infty_0$ is dense in $L^2$ (Hachbusch 1986)
and dense in $H^1$

also define $H_0^n(\Omega, \C)$
Give inclusions including $C^\infty_0$
Mention what is dense

Sobolev spaces
