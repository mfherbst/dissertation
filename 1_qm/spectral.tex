\section{Spectral theory}
\label{sec:spectral}

In this chapter we will broaden our discussion focusing
on linear operators between the state functions of a Hilbert space.
We will discuss certain common classes of operators
including self-adjoint and compact operators as well as their spectral properties.
We will see that most operators, including the ones required for
atomic physics and quantum chemistry,
do not show all the nice properties we would like to rely on.
For example one might not be able to find eigenfunctions
for all eigenvalues and the ones one is able to determine
might not amount to span the Hilbert space completely.
For this reason we will hint at techniques
relevant to the Hilbert space $L^2(\R^d, \C)$
and a few of the relevant operators of \QM,
which will allow us to recover at least part of the eigenspectrum
with the numerical methods discussed in chapter \vref{ch:numeigen}.

\subsection{Bounded and self-adjoint operators}

Mathematically a linear operator is defined as such:
\begin{defn}
	A \newterm{linear operator} on a Hilbert space $\hilbert$
	is the linear map $\Op{A} : D(\Op{A}) \to \hilbert$,
	where $D(\Op{A}) \subset \hilbert$ is a subspace
	called the \newterm{domain} of $\Op{A}$.
\end{defn}
Typically we employ just the term \newterm{operator} to refer to linear operators.
Recall that a mapping is called \newterm{linear} if
for all $u, v \in \hilbert$ and all $\alpha \in \C$
\begin{align}
	\label{eqn:OperatorLinear}
	\Op{A}\left(u + v\right) &= \Op{A}u + \Op{A}v &
	\Op{A}\left(\alpha u\right) &= \alpha \Op{A}u
\end{align}
hold.
Even though not strictly necessary, we will assume for our treatment
that the Hilbert space is separable and that the domain of an operator
is always dense in it.

\newcommand{\opnorm}[1]{\norm{#1}_{\mathcal{L}(\hilbert)}}
\begin{prop}
	The inner product of $\hilbert$ induces the so-called operator norm
	\[ \opnorm{\Op{A}} := \sup_{\substack{u \in D(\Op{A}),\\ u\neq0}}
		\frac{\norm{\Op{A} u}_\hilbert}{\norm{u}_{\hilbert}}. \]
	\begin{proof}
		See \cite[Satz II.1.4]{Werner2011}
	\end{proof}
\end{prop}

The first important classification we will discuss here
is the notion of bounded and unbounded operators.
\begin{defn}
	\label{defn:OpBounded}
	An operator $\Op{A}$ on $\hilbert$ is \newterm{bounded} iff
	\[ \opnorm{\Op{A}} < \infty, \]
	\ie if it has finite operator norm.
	A bounded operator is referred to as \newterm{continuous}%
	\footnote{In fact this is a consequence from the fact that a bounded linear operator
	between normed vector spaces is always continuous.} as well.
	Operators, which are not bounded are \newterm{unbounded operators}.
\end{defn}
In our example of $\hilbert = L^2(\R^d, \C)$ an operator is hence bounded
if its action on a square-integrable function yields another function,
which stays square-integrable.
In the introductory paragraph of section \vref{sec:Sobolev}
we already noted that the radial derivative of
the square-integrable function $1/r$, namely $-1/r^2$,
is not square-integrable.
It should therefore not come as a surprise that
operators containing derivatives
--- like the kinetic energy operator of \QM ---
are in general not bounded.

An alternative way to think about unbounded operators
is accessible via the concept of an \newterm{operator extension}.
\begin{defn}
	Let $\Op{A}$ and $\Op{B}$ be operators on $\hilbert$.
	$\Op{B}$ is an extension of $\Op{A}$ if
	\mbox{$D(\Op{A}) \subset D(\Op{B})$}
	and if $\forall u \in D(\Op{A}): \ \Op{A}u = \Op{B}u$.
\end{defn}

\begin{prop}
	Let $\Op{A}$ be an operator on $\hilbert$.
	The following statements are equivalent:
	\begin{itemize}
		\item $\Op{A}$ is unbounded.
		\item $\Op{A}$ does not possess a bounded extension.
	\end{itemize}
	\begin{proof} ~
		\begin{itemize}
			\item[$\Rightarrow$] Since $\Op{A}$ is unbounded,
				there exists a $x \in D(\Op{A})$ with
				\[ \nexists \, \alpha \in \R: \quad \norm{\Op{A} x}_\hilbert = \alpha \norm{x}_{\hilbert}. \]
				For any extension $\Op{B}$ by construction $x \in D(\Op{B})$ and
				$\Op{B} x = \Op{A} x$, such that for this $x$
				\[ \nexists \, \beta\in \R: \quad \norm{\Op{B} x}_\hilbert = \beta \norm{x}_{\hilbert}. \]
				Thus any such $\Op{B}$ is unbounded.
			\item[$\Leftarrow$]
				For  $D(\Op{A}) = \hilbert$ the statement is trivial
				and w.l.o.g we assume $D(\Op{A}) \neq \hilbert$.
				Further we proceed by
				proving the equivalent statement,
				that each bounded operator
				$\Op{A}$ possesses at least one bounded extension.

				Given $\Op{A}$ with $D(\Op{A}) \subsetneq \hilbert$ we can choose a $v \in \hilbert$
				with $v \perp D(\Op{A})$ and $\norm{v} = 1$.
				An extension $\Op{B}$ of $\Op{A}$ can be defined pointwise
				\[ \Op{B} u \equiv \Op{A} \big(u - v \braket{v}{u}\big) \]
				with $u \in D(\Op{B}) = D(\Op{A}) \oplus \{v\}$.
				For any $u \in D(\Op{B}$ by construction
				$u - v \braket{v}{u} \in D(\Op{A})$
				such that we can find a $\alpha \in \R$ satisfying
				\begin{align*}
					\norm{\Op{B} u}
					&= \norm{\Op{A} (u - v \braket{v}{u})}
					= \alpha \norm{u - v \braket{v}{u}} \\
					&\leq \alpha \left(\norm{u} + \norm{v}^2 \norm{u} \right)
					= 2 \alpha \norm{u}.
				\end{align*}
				In other words $\Op{B}$ is a bounded extension of $\Op{A}$.
		\end{itemize}
	\end{proof}
\end{prop}

There also exists a middle ground,
namely so-called \newterm{semi-bounded operator}s, defined as such:
\begin{defn}
	\label{defn:SemiBounded}
	An operator $\Op{A}$ on $\hilbert$ with domain $D(\Op{A})$
	is called semi-bounded from below if there exists a constant $C$
	such that for all $u \in D(\Op{A})$:
	\[ \braket{u}{\Op{A} u} = \braket{\Op{A} u}{u} \quad \text{and}
	\quad \braket{u}{\Op{A} u} \geq -C \braket{u}{u}. \]
\end{defn}

Starting from definition \ref{defn:OpBounded}
it is easy to show that a bounded operator $\Op{A}$ on a Hilbert
space $\hilbert$ maps Cauchy sequences to Cauchy sequences,
\ie if $(x_n) \in \hilbert$ is Cauchy, so is $(\Op{A} x_n)$.
In this sense a somewhat stronger version of boundedness
is compactness, defined as:
\begin{defn}
	An operator $\Op{A} : D(\Op{A}) \to \hilbert$ on a Hilbert space $\hilbert$
	is \newterm{compact}
	if for any sequence $(x_n)$ that converges weakly in $D(\Op{A})$,
	$\Op{A} x_n$ converges strongly in $\hilbert$.

	Recall that a sequence $(x_n)$ is called \newterm{weakly convergent}
	if for all $\phi \in \hilbert$ the sequence $(y_n)$ with
	$y_n = \braket{x_n}{\phi}_\hilbert$
	is \newterm{strongly convergent}, \ie Cauchy.
\end{defn}
Compactness is of importance for us in the context of spectral theory,
since compact operators have particularly nice spectral properties.
As expected one may easily show, that~\cite{Werner2011}
\begin{prop}
	A compact operator $\Op{A}$ defined on a Hilbert space is bounded as well.
\end{prop}

\begin{rem}
\label{rem:OpRietz}
Each operator $\Op{A}$ on a Hilbert space $\hilbert$ can be
uniquely identified with a sesquilinear form
$a : \hilbert \times \hilbert \to \C$, defined by
\begin{equation}
	\hilbert \times \hilbert \ni (u, v) \mapsto a(u, v) := \braket{u}{\Op{A} v}_\hilbert \in \C.
	\label{eqn:formA}
\end{equation}
This is a consequence of the Riesz representation theorem \cite{DiracNotation}.
%% TODO OPTIONAL \vref{thm:Riesz}.

In many applications, including the numerical treatment discussed in chapter
\vref{ch:numeigen},
the sesquilinear form $a$ is more intuitive to employ than the operator $\Op{A}$ itself.
\end{rem}

Using the identification of the previous remark,
we may define the terms symmetric and self-adjoint.
\begin{defn}
	\label{def:SymmetricOperator}
	An operator $\Op{A}$ on $\hilbert$ is called \newterm{symmetric}
	if
	\[ \forall (u, v) \in D(\Op{A}) \times D(\Op{A}): \qquad \braket{\Op{A} u}{v} = \braket{u}{\Op{A} v} \]
\end{defn}
In Physics textbooks a symmetric operator is usually called Hermitian.

\begin{defn}
	Let $\Op{A}$ be a linear operator on $\hilbert$ with (dense) domain $D(\Op{A})$
	and let $D(\tp{\Op{A}})$ be the space
	\[ D(\tp{\Op{A}}) := \left\{ v \in \hilbert \, \middle| \,
		\exists f_v \in \hilbert \ \text{such that} \
		\forall u \in D(\Op{A}): \braket{\Op{A} u}{v} = \braket{u}{f_v} \right\},
	\]
	where for each $v$ the $f_v$ is unique due to the denseness
	of $D(\Op{A})$ in $\hilbert$ and the Riesz representation theorem.

	Then the \newterm{adjoint} of $\Op{A}$
	is the linear operator $\tp{\Op{A}}$ with domain $D(\tp{\Op{A}})$
	defined by
	\[ \forall v \in D(\tp{\Op{A}}) \quad \braket{\Op{A} u}{v} = \braket{u}{\tp{\Op{A}} v} \]
\end{defn}

\begin{defn}
	A self-adjoint operator is an operator $\Op{A}$ for which $\tp{\Op{A}} = \Op{A}$,
	or equivalently an operator which is symmetric \emph{and} where
	$D(\Op{A}) = D(\tp{\Op{A}})$.
\end{defn}

\begin{rem}
	For a \emph{bounded} linear operator $\Op{A}$ with $D(\Op{A}) = \hilbert$
	one can find%
	\footnote{Since $D(\Op{A})$ in our treatment is dense in $\hilbert$,
	one can always find a unique, bounded extension of $\Op{A}$ with complete
	domain $\hilbert$ if $\Op{A}$ if bounded}
	a definition for the adjoint,
	which is more usual in the literature of quantum physics.

	\noindent
	Namely by means of the identification
	\[ \forall (u,v) \in \hilbert \times \hilbert \quad \braket{\Op{A} u}{v} = \braket{u}{\tp{\Op{A}} v} \]
	one can find a unique adjoint $\tp{\Op{A}}$ for each bounded operator $\Op{A}$.
	This operator will be bounded, too%
	\footnote{Note, that this makes the set of bounded operators on $\hilbert$
	a so-called $C^\ast$ algebra.}.

	Comparing with the definition of a symmetric operator we find that for
	bounded operators the property of symmetric and self-adjoint are equivalent.
\end{rem}

\begin{rem}
	Even though symmetric and self-adjoint are related concepts,
	symmetric operators are not very useful in practice.
	Only self-adjoint operators have the nice mathematical properties,
	we require for quantum mechanics, namely a real spectrum
	and a spectral decomposition into bound and continuous states.
	See the next section for details.

	Most operators in \QM are not self-adjoint albeit being symmetric
	if defined in a na\"{i}ve way.
	In many cases this issue can be circumvented
	by choosing an appropriate operator extension.
	We will discuss this 
	is section \vref{sec:SpectrumLaplace} and \vref{sec:HydrogenAtom}
	considering the spectrum of the Laplace operator
	\eqref{eqn:LaplaceOperatorHydrogen}
	and the hydrogen-like operator \eqref{eqn:OpHydrogen}.
\end{rem}

\begin{rem}
	As a summary of the terms introduced in this section,
	we note the following implications for an operator $\Op{A}$
	on a Hilbert space $\hilbert$.
	\begin{itemize}
		\item $\Op{A}$ compact
			$\Rightarrow$ $\Op{A}$ bounded $\Rightarrow$ $\Op{A}$ semi-bounded
		\item $\Op{A}$ self-adjoint $\Rightarrow$ $\Op{A}$ symmetric
		\item If $\Op{A}$ bounded: $\Op{A}$ self-adjoint $\Leftrightarrow$
			$\Op{A}$ symmetric.
	\end{itemize}
\end{rem}

\subsection{Spectra of self-adjoint operators}
In this section we will clarify the notion of a spectrum
for a self-adjoint operator in infinite dimensions
and the connections to the probably more familiar concepts
of eigenvalues and eigenvectors in finite dimensions.

\newcommand{\shiftop}{(\Op{A} - \lambda \id_{\hilbert})}
\nomenclature{$\id_{\hilbert}$}{Identity operator on the Hilbert space $\hilbert$.}
\begin{defn}
	\label{defn:Spectrum}
	Let $\Op{A}$ be a self-adjoint%
	\footnote{Strictly speaking the operator only needs to be closed for this definition.
	An operator is closed if its graph $\{ (u, \Op{A}u) \, | \, u \in D(\Op{A}) \}$
	is a closed subspace of $\hilbert \times \hilbert$.
	This is true for all self-adjoint operators.}
	linear operator on $\Op{H}$.
	\begin{itemize}
		\item We call the open set%
			\footnote{$\id_{\hilbert}$ is the identity operator on the Hilbert space $\hilbert$.}
			\[ \rho(\Op{A}) = \left\{ \lambda \in \C \ \middle| \
				\shiftop \
				\text{is bijective on $D(\Op{A})$} \right\} \]
			the \newterm{resolvent set} of $\Op{A}$.
		\item The closed set $\sigma(\Op{A}) = \C \, \backslash \, \rho(\Op{A})$
			is then the \newterm{spectrum} of $\Op{A}$.
	\end{itemize}
\end{defn}

\noindent
We can further show:~\cite[p.~102]{Helffer2013}
\begin{prop}
	If $\Op{A}$ is self-adjoint, then $\sigma(\Op{A}) \subset \R$.
\end{prop}

Another way of phrasing definition \vref{defn:Spectrum} is that
the spectrum is the set of all points where $\shiftop$ is not bijective.
This implies that both points where $\shiftop$ is not injective
as well as points where $\shiftop$ is not surjective are part of the spectrum.

Recall that an eigenpair $(\lambda, v) \in \C \times \hilbert$ satisfies
\[ \Op{A} v - \lambda v = 0 \quad \Leftrightarrow \quad v \in \ker \shiftop \quad \Rightarrow \quad \ker \shiftop \neq \{0\}, \]
since $v \neq 0$.
Since only non-injective operators can have a non-trivial kernel,
this implies that $\shiftop$ is necessarily non-injective
for $(\lambda, v)$ to be an eigenpair.
Unlike in finite dimensions%
\footnote{In finite dimensions one can always find an operator extension for any
injective operator to be surjective as well.}
it may well happen in infinite dimensions, that $\shiftop$ is injective,
but not surjective.
Therefore one may find points in the spectrum, which are not eigenvalues.
This is expressed more formally in the next definition.

% TODO OPTIONAL \to doil{In this and the next decomposition, how does this relate to resonance states?}
\begin{defn}
	If $\Op{A}$ is self-adjoint, we can decompose
	$\sigma(\Op{A}) = \sigma_P(\Op{A}) \, \dot{\cup} \, \sigma_C(\Op{A})$
	with
	\begin{itemize}
		\item the \newterm{point spectrum}
			\[ \sigma_P(\Op{A}) = \left\{ \lambda \in \R \, \middle|
				\, \shiftop\ \text{is non-injective} \right\}
				= \{ \text{eigenvalues of $\Op{A}$} \}.\]
		\item the \newterm{continuous spectrum}%
			\footnote{$\overline{A}$ denotes the closure of the set $A$.}
			\[ \sigma_C(\Op{A}) = \overline{\left\{ \lambda \in \R \, \middle|\, \shiftop\ \text{injective, but not surjective} \right\}}.\]
	\end{itemize}
\end{defn}
This definition can be understood physically by the so-called
RAGE\footnote{Named after Ruelle, Amrein, Georgescu and Enss, who all worked on it.}
theorem~\cite{Last1996}.
It draws
a connection between the point spectrum $\sigma_P(\Op{A})$
and the so-called \newterm{bound state}\textbf{s} of an operator
and between the continuous spectrum $\sigma_C(\Op{A})$
and the so-called \newterm{scattering state}\textbf{s}.
Bound states are characterised by the property,
that they have --- at all times --- a non-vanishing function value
only in a finite region of space.
Scattering states on the other hand are \emph{not} eigenstates
and they will vanish from any arbitrarily large, bounded part of space
if enough time has passed.

The decomposition of the spectrum into point and continuous spectrum
is not the only possibility.
Especially from the point of view of numerical modelling
the following, an alternative approach is more helpful
as we shall see later.

\begin{defn}
	\label{defn:EssentialSpectrum}
	For any self-adjoint%
	\footnote{Again only closed is strictly required.}
	operator,
	we can decompose \linebreak
	$\sigma(\Op{A}) = \sigma_\text{disc}(\Op{A}) \, \dot{\cup} \, \sigma_\text{ess}(\Op{A})$
	with
	\begin{itemize}
		\item the \newterm{discrete spectrum}%
			\footnote{For a mathematically more precise description, see~\cite[p.~103 and p.~132]{Helffer2013}.}
			\[ \sigma_\text{disc}(\Op{A}) \simeq \left\{ \lambda \in \R \, \middle|
			\, \text{$\lambda$ is an isolated eigenvalue of $\Op{A}$
			with finite multiplicity} \right\}, \]
		\item the \newterm{essential spectrum}
			\[\sigma_\text{ess}(\Op{A}) 
			= \sigma(\Op{A}) \, \backslash \, \sigma_\text{disc}(\Op{A}). \]
	\end{itemize}
\end{defn}

\noindent
By construction the essential spectrum consists of
\begin{itemize}
	\item the continuous spectrum,
	\item eigenvalues of infinite multiplicity,
	\item eigenvalues embedded inside the continuous spectrum.
\end{itemize}
% TODO OPTIONAL \to doil{Figure if time is left.}
It will become clear in a moment,
why approximate numerical methods can only be easily used
on the discrete spectrum.
For this we need to discuss the special case of
compact, self-adjoint operators in more detail.
If an operator is compact,
its spectrum has a particularly simple form.
\begin{prop}
	\label{prop:CompactSpectrum}
	If $\Op{A}$ is a compact operator on the Hilbert space $\hilbert$:
	\begin{itemize}
		\item $0 \in \sigma(\Op{A})$
		\item $\sigma(\Op{A}) \, \backslash \, \{0\}
			= \sigma_P(\Op{A}) \, \backslash \, \{0\}$
		\item Only one of these cases is true:
			\begin{itemize}
				\item[\textopenbullet] $\sigma(\Op{A}) = \{0\}$
				\item[\textopenbullet] $\sigma(\Op{A}) \, \backslash \, \{0\}$ is finite.
				\item[\textopenbullet]
					$\sigma(\Op{A}) \, \backslash \, \{0\}$ can be described
					as a sequence of points tending to $0$.
			\end{itemize}
	\end{itemize}
	\begin{proof}
		See~\cite[p.~56]{Helffer2013}.
	\end{proof}
\end{prop}
In other words the continuous spectrum of a compact operator
may at most contain the value $0$ --- even in infinite dimensions.
Furthermore there is a nice result for the eigenfunctions of a compact, self-adjoint
operator:
\begin{prop}
	\label{prop:CompactBasis}
	Let $\hilbert$ be a separable Hilbert space and $\Op{A}$ a compact,
	self-adjoint operator on $\hilbert$.
	The eigenfunctions of $\Op{A}$,
	\ie the set of all functions $\{ u_k \}_{k \in \mathcal{I}} \subset \hilbert$
	with
	\[
		\Op{A} u_k - \lambda_k  u_k = 0
	\]
	for a $\lambda_k \in \sigma(\Op{A}) \backslash \{0\}$
	are a Hilbertian basis for $\hilbert$.
	In other words they satisfy
	\begin{align*}
	\braket{u_k}{u_l} &= \delta_{kl} \quad \forall \, k,l \in \mathcal{I}
	&&\text{and}&
	\spacespan \left( \{ u_k \}_{k \in \mathcal{I}} \right) &= \hilbert.
	\end{align*}
	\begin{proof}
		See~\cite[p.~60]{Helffer2013}.
	\end{proof}
\end{prop}

With propositions \ref{prop:CompactSpectrum} and \ref{prop:CompactBasis}
at hand, compact operators start to look a lot like the familiar case
of complex square matrices.
In fact one can show that~\cite[p.~43]{Helffer2013}
\begin{prop}
	\label{prop:FiniteDimCompact}
	Any linear operator on a finite-dimensional Hilbert space is compact.
\end{prop}
This is essentially a consequence of the fact that in finite dimensions
weak and strong convergence are equivalent.

\begin{rem}
With proposition \ref{prop:FiniteDimCompact}
we can reduce the setting of self-adjoint operators on a finite-dimensional
Hilbert space to the following:
\begin{itemize}
	\item In remark \vref{rem:HilbertCd} we said that the vectors
		of a $d$-dimensional Hilbert space
		can be represented as column vectors from $\C^d$.
		In a similar sense $\Op{A}$ can be identified by a finite matrix from
		$\C^{d \times d}$.
	\item The eigenfunctions of $\Op{A}$ are a complete orthonormal
		basis for the underlying Hilbert space.
		$\Op{A}$ has only real eigenvalues.
	\item Apart from zero $\Op{A}$ has only a point spectrum.
		The essential spectrum and the continuous spectrum at most consist of $0$.
\end{itemize}
\end{rem}

\begin{rem}
	\label{rem:NumericalEssentialSpectrum}
As we will see in the next sections \vref{sec:SpectrumLaplace}
and \vref{sec:HydrogenAtom} both the Laplace operator $\Delta$ as well as the
Hamiltonian $\Op{H}$ corresponding to hydrogen-like systems
are not compact on the Hilbert space $L^2(\R^3, \C)$,
since both of these operators are not even bounded.
Furthermore both of these operators do possess a non-trivial essential spectrum.

If a numerical approach for computing the spectra for these operators
should be used,
one naturally needs to restrict oneself to a finite-dimensional subspace
for solving the problem.
See section \vref{sec:RitzGalerkin} in the next chapter for details.
Because of prop. \vref{prop:FiniteDimCompact} our \emph{approximations}
to $\Delta$ and $\Op{H}$ will be compact.
As we just discussed these will therefore
at most have the value zero in their continuous spectrum.

Ignoring this $0$ for a moment,
we can state that
both the point spectrum as well as the continuous spectrum
of the infinite-dimensional operator will be mapped to the discrete
spectrum of the approximation.
For approximations to the discrete spectrum this is not a big problem.
As we go to infinite accuracy in our approximation,
we will recover more and more digits of the discrete eigenvalues
provided that our approximation is sensible.
For those eigenvalues which are part of the essential spectrum, however,
things are not so simple, because they might be surrounded by
discrete approximations to the continuous spectrum.
In general distinguishing between true eigenvalues
and so-called \newterm{spurious eigenvalues} inside the approximation
to the essential spectrum is difficult.
See the discussion in remark \vref{rem:PracticalNum} for some
further details.
\end{rem}

It is therefore very important to know
the spectral properties of the exact operator
in order to understand which part of the spectrum one may obtain.
Let us discuss in the following a few examples,
which are important for our treatment of \QM.

\subsection{The Laplace operator}
\label{sec:SpectrumLaplace}
Let us first consider the $d$-dimensional analogue of the Laplace operator
introduced in \eqref{eqn:LaplaceOperatorHydrogen}.
In Cartesian coordinates it reads
\begin{equation}
	\Delta = \sum_{i=1}^d \frac{\partial^2}{\partial x_i^2}.
	\label{eqn:LaplaceOperatorD}
\end{equation}
Since this operator is essentially a scaled form of the kinetic energy operator $\Op{T}$
(see \eqref{eqn:OpHydrogen}),
we expect it to be self-adjoint and have real eigenvalues.

%TODO OPTIONAL some reasoning about this would be great,
%     explaining it in more detail
As it turns out, however,
the naive choice of taking the domain of the operator to be the full
quantum-mechanical Hilbert space $D(\Delta) = L^2(\R^d, \C)$
is not helpful as this operator cannot be made self-adjoint.
Only upon using the Sobolev space domain $D(\Delta) = H^2(\R^d, \C)$,
we get a self-adjoint operator $\Delta$.
Its spectrum is $\sigma(\Delta) = \sigma_C(\Delta) = [0, \infty)$%
~\cite[example 3.2.2]{Davies2007}.
In other words it is a semi-bounded operator with no eigenvalues and no
discrete spectrum at all.

\subsection{The Laplace-Beltrami operator on the unit sphere}
\label{sec:SpectrumLaplaceBeltrami}
\nomenclature{$Y_l^m(\theta, \varphi)$}{Spherical harmonic with angular
momentum quantum number $l$ and azimuthal quantum number $m$.}
\nomenclature{$P^{m}_l$}{Associated Legendre polynomial with orders $l$ and $m$.}

In contrast to the previous section,
let us now consider the Laplace operator on the surface of the unit sphere
\[
	\set{S}^2 := \left\{ \vec{r} \in \R^3 \, \middle| \, x^2 + y^2 + z^2 = 1 \right\}.
\]
For this it is most convenient to consider
the spherical coordinate system,
\ie instead of parametrising the vector $\vec{r}$ as a
Cartesian column vector $(x, y, z)^T$,
we specify it as $(r, \theta, \varphi)$ with
\begin{align*}
	r &= \norm{\vec{r}} = \sqrt{x^2+y^2+z^2} & \theta &= \arccos\frac{z}{r} & \varphi &= \arctan\frac{y}{x}.
\end{align*}
The condition for the unit sphere than reduces to $r\stackrel{!}{=}1$.

Since the sphere has no longer a Euclidean geometry
but a curved manifold
the operator equivalent to \eqref{eqn:LaplaceOperatorD} takes
the deviating functional form
\newcommand{\laplaceSphere}{\Delta_{\set{S}^2}}
\begin{equation}
	\laplaceSphere  u = \frac{1}{\sin \theta} \frac{\partial}{\partial \theta}
\left( \sin \theta \frac{\partial u}{\partial \theta}  \right)
+ \frac{1}{(\sin \theta)^2} \frac{\partial^2}{\partial \varphi^2} u
	\label{eqn:LaplaceBeltramiS}
\end{equation}
in spherical polar coordinates.
\eqref{eqn:LaplaceBeltramiS} is sometimes called the
\newterm{Laplace-Beltrami operator} as well.

By taking the domain $D(\laplaceSphere) = H^2(\set{S}^2)$
the operator $\laplaceSphere$ is self-adjoint~\cite[p.~120]{Helffer2013}.
Furthermore one can show that the spectrum is (apart from $0$) fully discrete%
\footnote{This follows since the inverse $\laplaceSphere^{-1}$ is compact%
~\cite[p.~44]{Helffer2013}.}.
Therefore this can be explicitly calculated by solving
the ansatz $\laplaceSphere \, Y = \lambda Y$
for the eigenpairs $(\lambda, Y)$.
This results in the \newterm{spherical harmonics}
\begin{equation}
	Y_l^m(\theta, \varphi) = \sqrt{\frac{2l + 1}{4 \pi} \frac{(l - m)!}{(l+m)!}}%
	\, P_l^m(\cos \theta) \, e^{im\varphi}
	\label{eqn:SphericaHarmonics}
\end{equation}
where $P_l^m$ is the associated Legendre polynomial with orders $l$ and $m$.
The eigenvalue corresponding to $Y_l^m(\theta, \varphi)$ is $-l (l+1)$.
Due to the restriction
\[ -l \le m \le l \]
this eigenvalue is $(2l+1)$-fold degenerate.
Our spherical harmonics obviously satisfy
\begin{equation}
	-\laplaceSphere \, Y_l^m(\theta, \varphi) = l (l+1) Y_l^m(\theta, \varphi).
	\label{eqn:Laplace}
\end{equation}
For the next section let us briefly note,
that the Laplace-Beltrami operator on the unit sphere
and the Laplace operator in $3$ dimensions,
expressed in spherical polar coordinates, are related by
\newcommand{\laplaceRadial}{\frac{\partial}{\partial r} \left( r^2 \frac{\partial}{\partial r} \right)}
\begin{equation}
r^2 \Delta = \laplaceRadial + \laplaceSphere.
	\label{eqn:LaplaceCorrespondance}
\end{equation}
This allows to show
\begin{equation}
	- r^2 \Delta \, Y_l^m(\theta, \varphi) =
	-\laplaceSphere \, Y_l^m(\theta, \varphi) = l (l+1) \, Y_l^m(\theta, \varphi).
	\label{eqn:LaplaceSphericalHarmonic}
\end{equation}

An important consequence of the discreteness of the spectrum
of the Laplace-Beltrami operator is that the spherical harmonics form
a complete basis for $H^2(\set{S}^2)$.

\subsection{The Schrödinger operator for a hydrogen-like atom}
\label{sec:HydrogenAtom}
\nomenclature{$_1F_1\!\left(a|b|z\right)$}
{Confluent hypergeometric function, see equation \eqref{eqn:ConfluentHypergeometric}.}
One might wonder if a pure Laplace operator as in section \vref{sec:SpectrumLaplace}
only possesses an essential spectrum,
how this develops if a potential is added,
like the $Z / r$ in the case of
the hydrogen-like Schrödinger operator \eqref{eqn:OpHydrogen}
\[
	\Op{H} = -\frac12 \Delta - \frac{Z}{r}.
\]
As the Hilbert space for this operator we take the \QM space $L^2(\R^3, \C)$
and an appropriate domain to make it self-adjoint is $H^2(\R^3, \C)$%
~\cite[p.~38]{Helffer2013}.
One can show~\cite{Davies2007,Teschl2014} that $\sigma_C(\Op{H}) = [0, \infty)$
and all discrete eigenvalues from $\sigma_P(\Op{H})$ are less than zero.
Thus $\sigma_\text{disc} = \sigma_P$ and $\sigma_\text{ess} = \sigma_C$.
The point spectrum of $\Op{H}$ can be conveniently
determined by solving the Schrödinger equation \eqref{eqn:TISE}
\begin{equation}
	( \Op{H} - E_{\mu} ) \Psi_\mu = 0,
	\label{eqn:HydrogenComplete}
\end{equation}
where $\Psi \in H^2(\R^3, \C)$ and $E \in \R^-$.
Without jumping ahead too far let us assume that the state
$\Psi_\mu$
may be uniquely identified by three quantum numbers $\mu \equiv (n, l, m)$.

\noindent
Using \eqref{eqn:LaplaceCorrespondance} we may write the Hamiltonian as
\begin{equation}
	\label{eqn:HydrogenLaplaceBeltrami}
	\Op{H} = -\frac{1}{2r^2} \laplaceRadial -\frac{1}{2r^2} \laplaceSphere - \frac{Z}{r}.
\end{equation}
A careful inspection of \eqref{eqn:HydrogenLaplaceBeltrami}
in contrast with \eqref{eqn:Laplace}
suggests a product ansatz
\[ \Psi_{nlm}(\vec{r}) = R_{nl}(r) Y_l^m(\theta, \phi). \]
%
With \eqref{eqn:LaplaceSphericalHarmonic} this yields the radial equation
\begin{equation}
	\left( - \frac{1}{2r^2} \laplaceRadial + \frac{l (l+1)}{2 r^2} - \frac{Z}{r} - E_\mu \right) R_{nl}(r) = 0
	\label{eqn:HydrogenRadial}
\end{equation}
which has the solutions~\cite{Mueller2000}
\begin{equation}
	 R_{nl}(r) = N_{nl} \left(\frac{2Zr}{n}\right)^l \exp\left(-\frac{Zr}{n} \right)
\;_1F_1\!\left(l+1-n \middle| 2l+2 \middle|\frac{2Zr}{n}\right)
	\label{eqn:HydrogenRadialSolution}
\end{equation}
where $_1F_1\!\left(a|b|\zeta\right)$ is a \newterm{confluent hypergeometric function},
namely~\cite{Avery2006}
\begin{equation}
	_1F_1\left(a \middle| b \middle| \zeta\right) =
	\sum_{k=0}^\infty \frac{a^{\bar{k}}}{k! \, b^{\bar{k}}} \zeta^k =
	1 + \frac{a}{b} \zeta + \frac{a(a+1)}{2b(b+1)} \zeta^2 + \cdots
	\label{eqn:ConfluentHypergeometric}
\end{equation}
with $a^{\bar{k}}$ being the rising factorial of $a$.
The normalisation constant is
\[ N_{nl} = \frac{2 \left( \frac{Z}{n} \right)^{3/2}}{(2l+1)!} \sqrt{ \frac{(l+n)!}{n (n-l-1)!}} \]
and the corresponding energy eigenvalues are
\begin{equation}
	E_{\mu} = - \frac{Z^2}{2n^2}.
	\label{eqn:HydrogenEnergyLevels}
\end{equation}

If one follows through the derivation properly,
one notices that the quantum numbers $n$, $l$ and $m$ are integer and need to satisfy
the following conditions:
\begin{align}
	\label{eqn:HydrogenIndexCondition}
	n &> 0 & 0 \leq l &< n & -l \leq &m \leq l
\end{align}
Furthermore since all involved equations are of Sturm-Liouville form,
the set of all solutions
\[
	\{\Psi_{nlm} \}_{n,l,m \text{ satisfy \eqref{eqn:HydrogenIndexCondition}}}
\]
forms the orthonormal basis for a Hilbert space $\mathcal{H}_\text{H}$.

We saw in examples \vref{exmp:H1sH2} and \vref{exmp:H1sH1}
that $\exp(-r) \in H^2(\R^3, \C)$,
which implies
\begin{equation}
	\Psi_{1s}(r, \theta, \phi) = \Psi_{100}(r, \theta, \phi)
	= \sqrt{\frac{Z^3}{\pi}} \exp(-Z r) \in H^2(\R^3, \C).
	\label{eqn:FunctionGShydrogen}
\end{equation}
From the functional form of $R_{nl}$ and $Y_l^m$ it is clear,
that all eigenstates $\Psi_{nlm}$ are infinitely differentiable everywhere
except at $r = 0$.
See \cite{Kato1957} and references therein for details.
The polynomial in $r$ in front of the exponential factor of the radial part $R_{nl}$
has exponents in $r$ in the range $[l, n-1]$
such that the eigenstate with $l = n-1 = 0$, i.e.~$\Psi_{1s}$ is the least smooth.
This implies $\Psi_{nlm} \in H^2(\R^3, \C)$ and thus
$\mathcal{H}_\text{H}$ is a (true) subspace%
\footnote{
	This is a true subspace, \ie non-identical to $H^2(\R^3, \C)$,
	since the scattering states are not part of it.
}
of $H^2(\R^3, \C)$.
