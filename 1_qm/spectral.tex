\section{Spectral theory}
\label{sec:spectral}

In this chapter we will broaden our discussion focusing
on linear operators between the state functions of a Hilbert space.
We will discuss certain common classes of operators
including self-adjoint and compact operators as well as their spectral properties.
We will see that most operators, including the ones required for
atomic physics and quantum chemistry,
do not show all the nice properties we would like to rely on.
For example one might not be able to find eigenfunctions
for all eigenvalues and the ones one is able to determine
might not amount to span the Hilbert space completely.
For this reason we will hint at techniques for the Hilbert space
$L^2(\R^d, \C)$ and a few of the relevant operators of \QM,
which will allow us to recover at least part of the eigenspectrum
using the numerical methods discussed in chapter \vref{ch:numeigen}.

\subsection{Bounded and self-adjoint operators}

Mathematically a linear operator is defined as such:
\begin{defn}
	A \newterm{linear operator} on a Hilbert space $\hilbert$
	is the linear map $\Op{A} : D(\Op{A}) \to \hilbert$,
	where $D(\Op{A}) \subset \hilbert$ is a subspace
	called the \newterm{domain} of $\Op{A}$.
\end{defn}
Typically we employ just the term \newterm{operator} to refer to linear operators.
Recall that a mapping is called \newterm{linear} if 
for all $u, v \in \hilbert$ and all $\alpha \in \C$
\begin{align}
	\label{eqn:OperatorLinear}
	\Op{A}\left(u + v\right) &= \Op{A}u + \Op{A}v &
	\Op{A}\left(\alpha u\right) &= \alpha \Op{A}u
\end{align}
hold.
Even though not strictly necessary, we will assume for our treatment
that the Hilbert space is separable and that the domain of an operator
is always dense in it.

\newcommand{\opnorm}[1]{\norm{#1}_{\mathcal{L}(\hilbert)}}
\begin{prop}
	The inner product of $\hilbert$ induces the so-called operator norm
	\[ \opnorm{\Op{A}} := \sup_{\substack{u \in D(\Op{A}),\\ u\neq0}}
		\frac{\norm{A u}_\hilbert}{\norm{u}_{\hilbert}}. \]
	\begin{proof}
		~ \\
		\todoil{Find reference}
	\end{proof}
\end{prop}

The first important classification we will discuss here
is the notion of bounded and unbounded operators.
\begin{defn}
	\label{defn:OpBounded}
	A operator $\Op{A}$ on $\hilbert$ is \newterm{bounded} iff
	\[ \opnorm{\Op{A}} < \infty, \]
	\ie if it has finite operator norm.
	A bounded operator is also sometimes referred to as \newterm{continuous}%
	\footnote{In fact this is a consequence from the fact that a bounded linear operator
	between normed vector spaces is always continuous.}.
\end{defn}
In our example of $\hilbert = L^2(\R^d, \C)$ an operator is hence bounded
if its action on a square-integrable function yields another function,
which stays square-integrable.
We already noted in the introductory paragraph of section \vref{sec:Sobolev},
based on the radial derivative of $1/r$,
that differential operators applied to square-integrable functions
might yield derivatives, which are no longer square-integrable.
In other words all operators containing differential operations
--- like the kinetic energy operator of \QM ---
are not be bound in general.
Instead they are typically unbounded operators,
which we will define in the following.

\begin{defn}
	Let $\Op{A}$ and $\Op{B}$ be operators on $\hilbert$.
	$\Op{B}$ is an \newterm{extension} of $\Op{A}$ if
	\mbox{$D(\Op{A}) \subset D(\Op{B})$}
	and if $\forall u \in D(\Op{A}): \ \Op{A}u = \Op{B}u$.
\end{defn}

\begin{defn}
	A operator $\Op{A}$ on $\hilbert$ which does not possess
	a bounded extension is called an \newterm{unbounded operator} on $\hilbert$.
\end{defn}
\todoil{Think how this can be possible, e.g. that the operator by itself is not bounded,
	but the extension is}

There also exists a middle ground,
namely so-called \newterm{semi-bounded operator}s, defined as such:
\begin{defn}
	An operator is semi-bounded from above/below if
	\todoil{TODO}
% TODO define semi-bounded operator and bounded below and above
% Page 29 in Helffer2013
\end{defn}

Starting from definition \ref{defn:OpBounded}
it is easy to show that a bounded operator $\Op{A}$ on a Hilbert
space $\hilbert$ maps Cauchy sequences to Cauchy sequences,
\ie if $(x_n) \in \hilbert$ is Cauchy, so is $(\Op{A} x_n)$.
In this sense a somewhat stronger version of boundedness
is compactness, defined as:
\begin{defn}
	A operator $\Op{A} : D(\Op{A}) \to \hilbert$ on a Hilbert space $\hilbert$
	is \newterm{compact}
	if for any sequence $(x_n)$ that converges weakly in $\hilbert$,
	$\Op{A} x_n$ converges strongly in $\hilbert$.

	Recall that a sequence $(x_n)$ is called \newterm{weakly convergent}
	if for all $\phi \in C^\infty_0$ the sequence $(y_n)$ with
	$y_n = \braket{x_n}{\phi}_\hilbert$
	is \newterm{strongly convergent}, \ie Cauchy.
	\todoil{Check that I use the correct test function space and
		norms and inner products
		$\hilbert$ versus $D(\Op{A})$}
	% use norm of D(A) in weak convergence and norm of H in strong
\end{defn}
Compactness is of importance for us in the context of spectral theory,
since compact operators have particularly nice spectral properties.
As expected one may easily show, that~\todo{Find citatation}
\begin{prop}
	A compact operator $\Op{A}$ on a Hilbert spaces $\hilbert$ is bounded as well.
\end{prop}

\begin{rem}
\label{rem:OpRietz}
Each operator $\Op{A}$ on a Hilbert space $\hilbert$ can be
uniquely identified with a sesquilinear form
$a : \hilbert \times \hilbert \to \C$, defined by
\begin{equation}
	\hilbert \times \hilbert \ni (u, v) \mapsto a(u, v) := \braket{u}{\Op{A} v}_\hilbert \in \C.
	\label{eqn:formA}
\end{equation}
This is a consequence of the Riesz representation theorem \vref{thm:Riesz}.

In many applications, including the numerical treatment discussed in chapter
\vref{ch:numeigen},
the sesquilinear form $a$ is more intuitive to employ than the operator $\Op{A}$ itself.
One sometimes refers to $\Op{A}$ as the \newterm{strong formulation}
and to the form $a$ as the \newterm{weak formulation}.
\end{rem}

Using the identification of the previous remark,
we may define the terms symmetric and self-adjoint.
\begin{defn}
	A operator $\Op{A}$ on $\hilbert$ is called \newterm{symmetric} if
	\[ \forall (u, v) \in D(\Op{A}) \times D(\Op{A}): \qquad \braket{\Op{A} u}{v} = \braket{u}{\Op{A} v} \]
\end{defn}

\begin{defn}
	Let $\Op{A}$ be a linear operator on $\hilbert$ with (dense) domain $D(\Op{A})$
	and let $D(\tp{\Op{A}})$ be the space
	\[ D(\tp{\Op{A}}) := \left\{ v \in \hilbert \, \middle| \,
		\exists f_v \in \hilbert \ \text{such that} \
		\forall u \in D(\Op{A}): \braket{\Op{A} u}{v} = \braket{u}{f_v} \right\},
	\]
	where for each $v$ the $f_v$ is unique due to the denseness
	of $D(\Op{A})$ in $\hilbert$ and the Riesz representation theorem.

	Then the \newterm{adjoint} of $\Op{A}$
	is the linear operator $\tp{\Op{A}}$ with domain $D(\tp{\Op{A}})$
	defined by
	\[ \forall v \in D(\tp{\Op{A}}) \quad \braket{\Op{A} u}{v} = \braket{u}{\tp{\Op{A}} v} \]
\end{defn}

\begin{defn}
	A self-adjoint operator is an operator $\Op{A}$ for which $\tp{\Op{A}} = \Op{A}$,
	or equivalently an operator which is symmetric \emph{and} where
	$D(\Op{A}) = D(\tp{\Op{A}})$.
\end{defn}

\begin{rem}
	For a \emph{bounded} linear operator $\Op{A}$ with $D(\Op{A}) = \hilbert$%
	\footnote{Since $D(\Op{A})$ in our treatment is dense in $\hilbert$,
	one can always find a unique, bounded extension of $\Op{A}$ with complete
	domain $\hilbert$ if $\Op{A}$ if bounded}
	one can find a definition for the adjoint,
	which is more usual in the literature of quantum physics.

	\noindent
	Namely by means of the identification
	\[ \forall (u,v) \in \hilbert \times \hilbert \quad \braket{\Op{A} u}{v} = \braket{u}{\tp{\Op{A}} v} \]
	one can find a unique adjoint $\tp{\Op{A}}$ for each bounded operator $\Op{A}$.
	This operator will be bounded, too%
	\footnote{Note, that this makes the set of bounded operators on $\hilbert$
	a so-called $C^\ast$ algebra.}.

	Comparing with the definition of a symmetric operator we find that for
	bounded operators the property of symmetric and self-adjoint are equivalent.
\end{rem}

\begin{rem}
	Even though symmetric and self-adjoint are related concepts,
	symmetric operators are not very useful in practice.
	Only self-adjoint operators have the nice mathematical properties,
	we require for quantum mechanics, namely a real spectrum
	and a spectral decomposition into bound and continuous states.
	See the next section for details.

	Most operators in \QM are not self-adjoint albeit being symmetric
	if defined in a naive way.
	In many cases this issue can be circumvented
	by choosing an appropriate operator extension.
	We will discuss this 
	is section \vref{sec:SpectrumLaplace} and \vref{sec:HydrogenAtom}
	considering the spectrum of the Laplace \eqref{eqn:LaplaceOperatorHydrogen}
	and the Hydrogenic operator \eqref{eqn:OpHydrogen}.
\end{rem}

\begin{rem}
	As a summary of the terms introduced in this section,
	we note the following implications for an operator $\Op{A}$
	on a Hilbert space $\hilbert$.
	\begin{itemize}
		\item $\Op{A}$ compact
			$\Rightarrow$ $\Op{A}$ bounded $\Rightarrow$ $\Op{A}$ semi-bounded
		\item $\Op{A}$ self-adjoint $\Rightarrow$ $\Op{A}$ symmetric
		\item If $\Op{A}$ bounded: $\Op{A}$ self-adjoint $\Leftrightarrow$
			$\Op{A}$ symmetric.
	\end{itemize}
\end{rem}

\subsection{Spectra of self-adjoint operators}
In this section we will clearify the notion of a spectrum
for a self-adjoint operator in infinite dimensions
and the connections to the probably more familiar concepts
of eigenvalues and eigenvectors in finite dimensions.

\newcommand{\shiftop}{(\Op{A} - \lambda \id_{\hilbert})}
\begin{defn}
	\label{defn:Spectrum}
	Let $\Op{A}$ be a self-adjoint%
	\footnote{Strictly speaking the operator only needs to be closed for this definition.
	An operator is closed if its graph $\{ (u, \Op{A}u) \, | \, u \in D(\Op{A}) \}$
	is a closed subspace of $\hilbert \times \hilbert$.
	This is true for all self-adjoint operators.}
	linear operator on $\Op{H}$.
	\begin{itemize}
		\item We call the open set
			\[ \rho(\Op{A}) = \left\{ \lambda \in \C \ \middle| \
				\shiftop \
				\text{is bijective on $D(\Op{A})$} \right\} \]
			the \newterm{resolvent set} of $\Op{A}$.
		\item The closed set $\sigma(\Op{A}) = \C \, \backslash \, \rho(\Op{A})$
			is then the \newterm{spectrum} of $\Op{A}$.
	\end{itemize}
\end{defn}

\noindent
We can further show:~\cite[p. 102]{Helffer2013}
\begin{prop}
	If $\Op{A}$ is self-adjoint, then $\sigma(\Op{A}) \subset \R$.
\end{prop}

Another way of phrasing definition \vref{defn:Spectrum} is that
the spectrum is the set of all points where $\shiftop$ is not bijective.
This implies that both points where $\shiftop$ is not injective
as well as points where $\shiftop$ is not surjective are part of the spectrum.

Recall that an eigenpair $(\lambda, v) \in \C \times \hilbert$ satisfies
\[ \Op{A} v - \lambda v = 0 \quad \Leftrightarrow \quad v \in \ker \shiftop \quad \Rightarrow \quad \ker \shiftop \neq \{0\}, \]
since $v \neq 0$.
Since only non-injective operators can have a non-trivial kernel,
this implies that $\shiftop$ is necessarily non-injective
for $(\lambda, v)$ to be an eigenpair.
Unlike in finite dimensions%
\footnote{In finite dimensions one can always find a operator extension for any
injective operator to be surjective as well.}
it may well happen in infinite dimensions, that $\shiftop$ is injective,
but not surjective.
Therefore one may find points in the spectrum, which are not eigenvalues.
This is expressed more formally in the next definition.

\begin{defn}
	If $\Op{A}$ is self-adjoint, we can decompose
	$\sigma(\Op{A}) = \sigma_P(\Op{A}) \, \dot{\cup} \, \sigma_C(\Op{A})$
	with
	\begin{itemize}
		\item the \newterm{point spectrum}
			\[ \sigma_P(\Op{A}) = \left\{ \lambda \in \R \, \middle|
				\, \shiftop\ \text{is non-injective} \right\}
				= \{ \text{eigenvalues of $\Op{A}$} \}.\]
		\item the \newterm{continuous spectrum}%
			\footnote{$\overline{A}$ denotes the closure of the set $A$.}
			\[ \sigma_C(\Op{A}) = \overline{\left\{ \lambda \in \R \, \middle|\, \shiftop\ \text{injective, but not surjective} \right\}}.\]
	\end{itemize}
\end{defn}
This definition can be understood physically by the so-called
RAGE\footnote{After the initials of the discoverers 
Ruelle, Amrein, Georgescu and Volker Enss}
theorem
\todo{Lookup references}.
It draws
a connection between the point spectrum $\sigma_P(\Op{A})$
and the so-called \newterm{bound state}s of an operator
and between the continuous spectrum $\sigma_C(\Op{A})$
and the so-called \newterm{scattering state}s.
Bound states are characterised by the property,
that they have --- at all times --- a non-vanishing function value
only in a finite region of space.
Scattering states on the other hand are \emph{not} eigenstates
and they will vanish from any arbitrarily large, bounded part of space
if enough time has passed.

The decomposition of the spectrum into point and continuous spectrum
is not the only possibility.
Especially from the point of view of numerical modelling
the following, alternative approach is more helpful
as we shall see later.

\begin{defn}
	For any self-adjoint%
	\footnote{Again only closed is strictly required.}
	operator,
	we can decompose \linebreak
	$\sigma(\Op{A}) = \sigma_\text{disc}(\Op{A}) \, \dot{\cup} \, \sigma_\text{ess}(\Op{A})$
	with
	\begin{itemize}
		\item the \newterm{discrete spectrum}%
			\footnote{For a mathematically more precise description, see~\cite[p.103 and p.132]{Helffer2013}.}
			\[ \sigma_\text{disc}(\Op{A}) \simeq \left\{ \lambda \in \R \, \middle|
			\, \text{$\lambda$ is an isolated eigenvalues of $\Op{A}$
			with finite multiplicity} \right\}, \]
		\item the \newterm{essential spectrum}
			\[\sigma_\text{ess}(\Op{A}) 
			= \sigma(\Op{A}) \, \backslash \, \sigma_\text{disc}(\Op{A}). \]
	\end{itemize}
\end{defn}

\noindent
By construction the essential spectrum consists of
\begin{itemize}
	\item the continuous spectrum,
	\item eigenvalues of infinite multiplicity,
	\item eigenvalues embedded inside the continuous spectrum.
\end{itemize}
\todoil{Figure?}
It will become clear in a moment
why only modelling the discrete spectrum can be done easily by
approximative, numerical methods.
For this we need to discuss the special case of
compact, self-adjoint operators in a bit more details.
Already if an operator is compact,
its spectrum has a particularly simple form.
\begin{prop}
	\label{prop:CompactSpectrum}
	If $\Op{A}$ is a compact operator on the Hilbert space $\hilbert$:
	\begin{itemize}
		\item $0 \in \sigma(\Op{A})$
		\item $\sigma(\Op{A}) \, \backslash \, \{0\}
			= \sigma_P(\Op{A}) \, \backslash \, \{0\}$
		\item Only one of these cases is true:
			\begin{itemize}
				\item[\textopenbullet] $\sigma(\Op{A}) = \{0\}$
				\item[\textopenbullet] $\sigma(\Op{A}) \, \backslash \, \{0\}$ is finite.
				\item[\textopenbullet]
					$\sigma(\Op{A}) \, \backslash \, \{0\}$ can be described
					as a sequence of points tending to $0$.
			\end{itemize}
	\end{itemize}
	\begin{proof}
		See~\cite[p. 56]{Helffer2013}.
	\end{proof}
\end{prop}
In other words the continuous spectrum of a compact operator
may at most contain the value $0$ --- even in infinite dimensions.
Furthermore there is a nice result for the eigenfunctions of a compact, self-adjoint
operator:
\begin{prop}
	\label{prop:CompactBasis}
	Let $\hilbert$ be a separable Hilbert space and $\Op{A}$ a compact,
	self-adjoint operator on $\hilbert$,
	then $\hilbert$ admits for a Hilbertian basis,
	\ie a basis set $\{ u_k \}_{k \in \mathcal{I}}$ with
	\begin{align*}
	\braket{u_k}{u_l} &= \delta_{kl} \quad \forall \, k,l \in \mathcal{I}
	&&\text{and}&
	\spacespan \left( \{ u_k \}_{k \in \mathcal{I}} \right) &= \hilbert.
	\end{align*}
	\begin{proof}
		See~\cite[p. 60]{Helffer2013}.
	\end{proof}
\end{prop}

With propositions \ref{prop:CompactSpectrum} and \ref{prop:CompactBasis}
at hand, compact operators start to look a lot like the familiar case
of complex square matrices.
In fact one can show that
\begin{prop}
	\label{prop:FiniteDimCompact}
	Any linear operator on a finite-dimensional Hilbert space is compact.
	\todoil{Find reference}
\end{prop}

\begin{rem}
With proposition \ref{prop:FiniteDimCompact}
we can reduce the setting of self-adjoint operators on a finite-dimensional
Hilbert space to the following:
\begin{itemize}
	\item In remark \vref{rem:HilbertCd} we said that the vectors
		of a $d$-dimensional Hilbert space
		can be represented as column vectors from $\C^d$.
		In a similar sense $\Op{A}$ can be identified by a finite matrix from
		$\C^{d \times d}$.
	\item The eigenfunctions of $\Op{A}$ are a complete orthonormal
		basis for the underlying Hilbert space.
		$\Op{A}$ has only real eigenvalues.
	\item Apart from zero $\Op{A}$ has only a point spectrum.
		The essential spectrum and the continuous spectrum at most consist of $0$.
\end{itemize}
\end{rem}

\begin{rem}
As we will see in the next sections \vref{sec:SpectrumLaplace}
and \vref{sec:HydrogenAtom} both the Laplace operator $\Delta$ as well as the
Hamiltonian $\Op{H}$ corresponding to hydrogenic systems
are not compact on the Hilbert space $L^2(\R^3, \C)$,
since both of these operators are not even bounded.
Furthermore both of these operators do possess a non-trivial essential spectrum.

If a numerical approach for computing the spectra for these operators
should be used,
one naturally needs to restrict oneself to a finite-dimensional subspace
for solving the problem.
See section \vref{sec:RitzGalerkin} in the next chapter for details.
Because of prop. \vref{prop:FiniteDimCompact} our \emph{approximations}
to $\Delta$ and $\Op{H}$ will be compact.
As we just discussed these will therefore
at most have the value zero in their continuous spectrum.

Ignoring this $0$ for a moment,
we can state that
both the point spectrum as well as the continuous spectrum
of the infinite-dimensional operator will be mapped to the discrete
spectrum of the approximation.
For approximations to the discrete spectrum this is not a big problem.
As we go to infinite accuracy in our approximation,
we will recover more and more digits of the discrete eigenvalues
provided that our approximation is sensible.
For those eigenvalues which are part of the essential spectrum, however,
things are not so simple, because they might be surrounded by
discrete approximations to the continuous spectrum.
In general distinguishing between true eigenvalues
and so-called \newterm{spurious eigenvalues} inside the approximation
to the essential spectrum is hard up to impossible.
\end{rem}

It is therefore very important to know
the spectral properties of the exact operator
in order to understand which part of the spectrum one may obtain.
Let us discuss in the following a few examples,
which are important for our following treatment.

\subsection{The Laplace operator}
\label{sec:SpectrumLaplace}
% This section should make clear
% The importance of the Sobolev space $H_2$ for quantum mechanics




Most operators of \QM are not bounded and thus not compact.
But at least all we will cover are semi-bounded.
This first of all allows to find a self-adjoint extension as we will see in the examples
in the next section.
They additionally have the nice property that the first few eigenvalues are
solely part of the discrete spectrum and the essential spectrum only starts at a
later point, \ie that
\[ \forall \lambda \in \sigma_{\text{disc}}(\Op{A}),
    \mu \in \sigma_{\text{ess}}(\Op{A}) \quad \lambda < \mu \]
We will see in section \vref{sec:RitzGalerkin} in the next chapter,
how this allows to construct sensible approximations
for these discrete eigenvalues even though the operators are not compact
(and neither are their resolvents in some cases).



or alternatively in spherical polar coordinates
as $(r, \theta, \varphi)$ with
\begin{align*}
	r &= \norm{\vec{r}} = \sqrt{x^2+y^2+z^2} & \theta &= \arccos\frac{z}{r} & \varphi &= \arcsin\frac{y}{x}.
\end{align*}




Let us consider the $N$-dimensional analogue of the Laplace operator
introduced in \eqref{eqn:LaplaceOperatorHydrogen}.
In Cartesian coordinates it reads
\[ \Delta = \sum_{i=1}^d \frac{\partial^2}{\partial x_i^2}. \]
This operator plays a crucial role in quantum mechanics.
We will see later\todo{refer} that this operator is part of the
kinetic operator of the Schrödinger equation.
For a physical equation we expect that the operator has
real, i.e. measurable, eigenvalues.
This typically requires the operator to be self-adjoint.
In other words from a physical standpoint we would expect
the Laplace operator to have real eigenvalues.

As it turns out,
the Laplacian operator $T_0 = \Delta$ with the domain $D(T_0) = L^2(\R^d)$
does not have \emph{any}
eigenfunctions $u \in L^2(\R^d)$,
not even in the weak sense of distributions~\cite{Helffer2013}.
In other words we are unable to find an eigenfunctions $u \in L^2(\R^d)$
and a $\lambda \in \R$,
such that it holds
\[ \forall \phi \in C^\infty_0(\R^d):  \int_{\R^d}\phi(x) (\Delta u)(x) \D x = \lambda \int_{\R^d} \phi(x) u(x) \D x. \]
Furthermore the Laplacian operator $T_0$ is not self-adjoint.

It turns out that the domain of the operator is the culprit here.
Converse to the problem we posed it is actually possible
to find \emph{for each} $\lambda \in \C$ an eigenfunction from the
slightly different so-called space of distributions $D'(\R^d)$.
This suggests that the construction of an appropriate, related
operator $T_1$, for which we can find an eigenspectrum.
In fact one may do so in a way, that $T_1$ is even self-adjoint
by the means of the \newterm{Friedrichs extension}.
In the particular case we discussed, there are actually
multiple ways to achieve this.
One way assumes a Dirichlet boundary, where the potential
eigenfunctions go to zero
one obtains the restricted domain
$D(T_1) = H^2(\R^d) \cap H^1_0(\R^d)$
for the self-adjoint Laplacian extension.
Unlike for other operators such an extension for the Laplacian
is not unique.
Another construction assuming a Neumann boundary is possible, too.
It again yields a subset of $H^2(\R^d)$ just with an appropriate
so-called trace condition enforcing the derivative to go to zero
at the boundary.
Most notably this implies that only under specification
of the boundary condition the operator has an eigenspectrum.
This is an observation which is true for many operators in quantum mechanics.

If $u \in H^2(\R^d)$ then $\Delta u \in L^2(\R^d)$

\subsection{The Laplace-Beltrami operator on the unit sphere}
\label{sec:SpectrumLaplaceBeltrami}
\nomenclature{$Y_l^m(\theta, \varphi)$}{Spherical harmonic function with angular
momentum quantum number $l$ and azimuthal quantum number $m$}
\nomenclature{$P^{m}_l$}{Associated Legendre polynomial with orders $l$ and $m$.}

In the previous section, we saw that the Laplace operator $\Delta$
is self-adjoint and has hence real eigenvalues provided that
we choose for its domain a subdomain of $D(\Delta) = H^2(\R^3)$.

In this section we want to discuss the solution of Laplace's equation
on the surface of the unit sphere
\[
	\set{S}^2 := \{ \vec{r} \in \R^3 | x^2 + y^2 + z^2 = 1 \}.
\]
For this it is most convenient to consider
the spherical polar coordinate system,
\ie instead of specifying the vector $\vec{r} = (x, y, z)^T$,
we specify the parameters $r, \theta, \phi$ or
\[ \vec{r} = \mm{ r \, \sin \theta \, \cos \phi \\ r \, \sin \theta \sin \phi \\
r \, \cos \theta }. \]
The condition for the unit sphere than reduces to $r=1$

Since the sphere is no longer a Euclidean geometry
but much rather a curved manifold some of the treatment presented
in the previous chapter no longer holds.
Nevertheless an equivalent to the Friedrichs extension can be found
in order to make the Laplace-Beltrami operator
\[ \Delta_{\set{S}^2} u = \frac{1}{\sin \theta} \frac{\partial}{\partial \varphi}
\left( \sin \varphi \frac{\partial u}{\partial \varphi}  \right)
+ \frac{1}{(\sin \varphi)^2} \frac{\partial^2}{\partial \theta^2} u\]
self-adjoint.
\todoil{Insert details}

The above polar expression for the Laplace-Beltrami operator
allows to explicitly solve the equation
\newcommand{\laplaceSphere}{\Delta_{\set{S}^2}}
\[ \laplaceSphere u = \lambda u \]
for eigenfunctions $u$ and eigenvalues $\lambda$,
which results in the spherical harmonics
\[ Y_l^m(\theta, \varphi) = N_{lm} P_l^m(\cos \theta) e^{im\varphi} \]
where $N_{lm}$ is a normalisation constant
and eigenvalues $-l (l+1)$:
\begin{equation}
	-\laplaceSphere Y_l^m(\theta, \varphi) = l (l+1) Y_l^m(\theta, \varphi)
	\label{eqn:Laplace}
\end{equation}
Importantly the quantum number $l$ is positive and $m$ is restricted by
\[ -l \le m \le l, \]
such that there are exactly $(2l+1)$ spherical harmonics with the same eigenvalue
$-l (l+1)$.

For our following arguments it will be important to note
that the 3d Laplacian in spherical polar coordinates
and the Laplace-Beltrami operator for the unit sphere
are related by
\newcommand{\laplaceRadial}{\frac{\partial}{\partial r} \left( r^2 \frac{\partial}{\partial r} \right)}
\begin{equation}
r^2 \Delta = \laplaceRadial + \laplaceSphere
	\label{eqn:LaplaceCorrespondance}
\end{equation}
such that
\begin{equation}
	- r^2 \Delta Y_l^m(\theta, \varphi) = -\laplaceSphere Y_l^m(\theta, \varphi) = l (l+1) Y_l^m(\theta, \varphi)
	\label{eqn:LaplaceSphericalHarmonic}
\end{equation}


\subsection{The Schrödinger operator for a hydrogenic atom}
\label{sec:HydrogenAtom}
We will discuss in the next chapter the physical origin.
For now we will just state that the Schrödinger operator for the Hydrogenic atom
(in so-called atomic units)
is given by the expression
\begin{equation}
	\Op{H} = - \frac12 \Delta - \frac{Z}{r} = -\frac{1}{2r^2} \laplaceRadial -\frac{1}{2r^2} \laplaceSphere - \frac{Z}{r}
	\label{eqn:HydrogenOperator}
\end{equation}
in other words the scaled Laplace operator and a term for a radial-symmetric Coulombic
potential.

The Hilbert space for this operator is $L^2(\R^3, \C)$
and the domain is $C^\infty_0(\R^3, \C)$
By a similar argument to section \vref{sec:SpectrumLaplace} based on the
Friedrichs extension one notices that this operator is self-adjoint
if the domain is chosen to be $H^2(\R^3, \C$.
Let us now find the analytic expressions of the eigenfunctions of this operator,
i.e. solve the time-independent Schrödinger equation
\begin{equation}
	( \Op{H} - E ) \Psi = 0
	\label{eqn:HydrogenEigenproblem}
\end{equation}
where $\Psi \in L^2(\R^3, \C)$ and $E \in \R$.
\todoil{More details on this later}

Without jumping ahead too far let us assume that both the energy as well
as the state $\Psi$ are characterised by three quantum numbers $n$, $l$ and $m$.
A further careful inspection of \eqref{eqn:HydrogenEigenproblem}
in contrast with \eqref{eqn:Laplace} suggests a product ansatz
\[ \Psi_{nlm}(\vec{r}) = R_{nl}(r) Y_l^m(\theta, \phi) \]
This yields to the radial equation
\begin{equation}
	- \left( \frac{1}{2r^2} \laplaceRadial + \frac{l (l+1)}{2 r^2} - \frac{Z}{r} - E \right) R_{nl}(r) = 0
	\label{eqn:HydrogenRadial}
\end{equation}
which has the solutions~\cite{Mueller2000}
\begin{equation}
	 R_{nl}(r) = N_{nl} \left(\frac{2Zr}{n}\right)^l \exp\left(-\frac{Zr}{n} \right)
\;_1F_1\left(l+1-n | 2l+2|\frac{2Zr}{n}\right)
	\label{eqn:HydrogenRadialSolution}
\end{equation}
where $_1F_1\left(a|b|\zeta\right)$ is a confluent hypergeometric function,
namely~\cite{Avery2006}
\[ _1F_1\left(a \middle| b \middle| \zeta\right) =
\sum_{k=0}^\infty \frac{a^{\bar{k}}}{k! \, b^{\bar{k}}} \zeta^k =
1 + \frac{a}{b} \zeta + \frac{a(a+1)}{2b(b+1)} \zeta^2 + \cdots \]
where $a^{\bar{k}}$ is the rising factorial of $a$.
The normalisaton constant is
\[ N_{nl} = \frac{2 \left( \frac{Z}{n} \right)^{3/2}}{(2l+1)!} \sqrt{ \frac{(l+n)!}{n (n-l-1)!}} \]
The energy eigenvalues are
\[ E_{nlm} = - \frac{Z^2}{2n^2}. \]
\todoil{One should mention that these are all to be found,
	since the radial functions are complete in the radial sense
and the spherical harmonics are complete in the angular sense.}

If one walks through the derivation properly,
one notices that the principle quantum number $n$ is always larger than zero
and that $l < n$ applies as well.
\todoil{If I remember correctly this is a result of Sturm-Liouville theory}
Together with the restrictions on $l$ and $m$ itself this results in the
following restrictions on the quantum numbers:
\begin{align*}
	n &> 0 & 0 \leq l &< n & -l \leq &m \leq l
\end{align*}
Taking a look at the confluent hypergeometric function we use, it is easy to see
that the power of the polynomial in $r$ is between zero and
\[
	\deg \;_1F_1\left(l + 1 -m \middle| 2l+2 \middle| \frac{2Zr}{n} \right) = n - l -1 \]
such that overall the polynomial part of the radial function
has a power between $l$ and
\[ \deg \frac{R_{nl}(r)}{\exp(2Zr/n)} = n - 1. \]

We will now argue that the solutions $\Psi \in H^1(\R^3, \C)$.
It is easy to see that the derivative of $\Psi$ is continuous and bound
everywhere but the origin.
See \cite{Kato1957} and references therein for details.
Since the smoothness at the origin is better the larger
the larger the exponents of the polynomial in $r$,
let us consider the worst case, i.e. the $s$-functions with $l = 0$.
Taking a look at the definition of $_1F_1(a,b,\zeta)$
one notices that the smallest exponent, irrespective of the value of $n$ is $0$.
Thus we will discuss the smoothness properties of
\[
	\Psi_{1s}(r, \theta, \phi) = \Psi_{100}(r, \theta, \phi)
	= \sqrt{\frac{Z^3}{\pi}} \exp(-Z r)
\]
at the origin,
which in fact is the least smooth of all hydrogen solutions.

Since this function is radially symmetric,
we only need to consider the derivative along one direction.
Without loss of generality we choose the $x$ direction.
Taking the first derivative we find
\[
	\frac{\partial \Psi_{1s}}{\partial x} = -\sqrt{\frac{Z^5}{\pi}} \frac{x}{r} \exp(-Z r)
\]
As expected at the origin we encounter a discontinuity in the derivative.
Inserting $x = r \sin\theta \cos \phi$ we do notice
\[
	\lim_{r \to 0} \frac{\partial \Psi_{1s}}{\partial x} = -\sqrt{\frac{Z^5}{\pi}} \sin\theta \cos \phi,
\]
which stays finite.

The second derivative is H1 as well (see Hardy inequality)


The derivative is thus bounded at zero as well
and as a result square-integrable.
On the other hand the second derivative
\[
	\frac{\partial^2 \Psi_{1s}}{\partial x^2}
	= \left( -\frac{Z x^2}{r^2} -\frac{x^2}{r^3} + \frac{1}{r} \right)
		\sqrt{\frac{Z^5}{\pi}} \exp(-Z r)
\]
is unbound at the origin.
Careful inspection shows that it is not square-integrable.
\todoil{Do that}
Therefore $\Psi_{1s} \in H^1(\R^3,\C)$, but $\Psi_{1s} \not\in H^2(\R^3, \C)$.







Show that Solutions are in $H^2$ by considering
the least smooth (i.e. the 1s orbital) and showing that this
is in $H^2$.
That way we know that $H^1$ is a good space for approximating the solutions
$\Rightarrow$ FE formalism can be applied.

\todoil{Try to check whether a productansatz for regularising the nuclear potential
	in combination with CS for the nuclear cusp factor and FE for the rest
could actually work}


% Derivative is smooth and bounded everywhere but at origin
% Argue that the worst function is for l = 0.
% Take derivative wrt. x
% Then show that it's absolute value at the origin just contains a hole
% hence derivative is L^2 integrable in the weak sense


