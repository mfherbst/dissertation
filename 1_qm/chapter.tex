\chapter{Mathematical structure of quantum mechanics}
\chapquote{
	Whenever we proceed from the known into the unknown we
	may hope to understand,
	but we may have to learn at the same time a new
	meaning of the word ``understanding''.
}{Werner Heisenberg~(1901--1976)}

\nomenclature{$\norm{~\cdot~}_2$}{$l_2$-norm or Euclidean norm}
\nomenclature{$\norm{~\cdot~}_\infty$}{$l_\infty$-norm or maximum norm}
\nomenclature{$\norm{~\cdot~}_1$}{Sum of absolute values of a vector}

At the turn of the 19th century it was discovered
that classical mechanics in the formulations provided
by Lagrange, Hamilton and Poission
are not able to capture all effects observed in experiments.
Especially the phenomenon of the blackbody radiation spectrum,
but also the photoelectric effect could not be explained by the available theory.
Finally in 1900 Max Planck somewhat reluctantly introduced the idea of
discrete, i.e. quantised, energy levels
in order to explain the blackbody radiation spectrum,
building on earlier ideas by Ludwig Boltzmann.
This started the development of a quantised theory of mechanics
with major contributions by nowadays famous
names such as Niels Bohr, Max Born, Louis de Broglie, Paul Dirac, Albert Einstein, Vladimir Fock, Pascual Jordan, John von Neumann, Erwin Schrödinger and many more.

In this chapter we will discuss the mathematical structure
of quantum mechanics in light of the problems of atomic physics
and quantum chemistry.
Our discussion will focus on explaining both the mathematical
peculiarities of quantum mechanics as well as hinting
at the tricks required to make progress in a physical setting.
We follow mostly the excellent material by \citet{Shankar1994},
\citet{Mueller2000} and \citet{Helffer2013} is our discussion.

\section{Correspondence of classical and quantum mechanics}
\label{sec:QMCorrespondence}
% TODO OPTIONAL
%\to doil{One could add a few more references for the general QM introduction part,
%	\eg to the original Schrödinger paper
\newcommand{\clphase}{(q_1, \ldots, q_d, p_1, \ldots p_d)}
%
%TODO OPTIONAL \to doil{Hint at other formulations of classical mechanics}
According to the Hamiltonian formulation of classical mechanics
a physical system with $d$ degrees of freedom
is described by a set of generalised coordinates
$q_1, \ldots, q_d$ along with their canonical momenta $p_1, \ldots, p_d$.
It is assumed that any physically measurable quantity $F$
only depends on these system parameters.
In other words one may define a so-called \newterm{observable}
$F\clphase$ as a function $\R^{2d} \to \R$, \ie
from a vector in \newterm{phase space} to the measured value.
Clearly the coordinates $q_i$ and momenta $p_i$ are observables as well.
The most important observable is the total energy function or \newterm{Hamiltonian}
\begin{equation}
	H\clphase
	\equiv \underbrace{\frac{1}{2} \sum_{k=1}^d \frac{p_k^2}{m_k}}_{= T(p_1, \ldots, p_d)}
	+ V(q_1, \ldots, q_d),
	\label{eqn:HamiltonianClassical}
\end{equation}
where $m_k$ is the mass of the particle associated with
the degree of freedom $k$,
$T$ is the kinetic energy observable and $V$ the total potential energy observable.
In the formalism of Hamiltonian mechanics, $H$ governs the time evolution
of the system, namely
\begin{align}
	\frac{\D p_k}{\D t} &= - \frac{\partial H}{\partial q_k}, &
	\frac{\D q_k}{\D t} &= \frac{\partial H}{\partial p_k}
	&&\forall k \in \{1, \ldots, d\}.
	\label{eqn:HamiltonianEqnMotion}
\end{align}

These expressions allow to generalise the description of the time evolution
to any other arbitrary observable as well.
To make the connection to quantum mechanics more apparent,
let us introduce for this purpose the so-called \newterm{Poisson bracket}.
It is the skew-symmetric form
\begin{equation}
	\left\{ F, G \right\}_P \equiv \sum_{j=1}^d \left(
	\frac{\partial  F}{\partial q_j} \frac{\partial G}{\partial p_j}
	- \frac{\partial  F}{\partial p_j} \frac{\partial G}{\partial q_j}
	\right).
	\label{eqn:PoissonBracket}
\end{equation}
According to the Liouville equation
\begin{equation}
	\frac{\D F}{\D t} = \frac{\partial F}{\partial t} + \left\{ F, H \right\}_P.
	\label{eqn:PoissonTimeEvolution}
\end{equation}
it relates the Hamiltonian $H$ to the time evolution of any arbitrary observable $F$.
One may also show the relationships
\nomenclature{$\delta_{ij}$}{Kronecker delta. Equal to $0$ if $i \neq j$, else $1$.}
\begin{align}
	\{ q_k, q_l \}_P = \{p_k, p_l\}_P &= 0 & \{ p_k, q_l\}_P
	= - \delta_{kl} &&\forall k,l \in \{1, \ldots, d\}
	\label{eqn:PoissonProperties}
\end{align}
between the principle system observables.

\subsection{Moving to quantum mechanics}
\label{sec:IntroQM}
\defineabbr{QM}{QM\xspace}{Quantum mechanics}
We will now introduce (non-relativistic)
quantum mechanics~(QM) in a rather pragmatic manner,
namely by stating a summary what changes in the \QM formulation
compared to the classical one.
The full mathematical details are not yet stated at this point
and not all terms defined.
The reader should take this overview as a motivation
for a more detailed treatment of the mathematical concepts
further down this chapter.

\nomenclature{$\hilbert$}{Arbitrary Hilbert space,
	typically the Hilbert space of quantum mechanics, which is $L^2(\R^d, \C^s)$
	for $d$ spatial and $s$ spin dimensions.
}
\begin{enumerate}[label=(\alph*)]
	\item Instead of phase space vectors $\clphase \in \R^{2d}$,
		in \QM a particular state of the system is represented
		by functions $\Psi : \R^d \to \C$
		mapping from space coordinates to a complex number.
		These originate from a complex separable Hilbert space $\hilbert$.
	\item A classical observable $F$ is represented by a
		self-adjoint operator $\Op{F}$ on the Hilbert space $\hilbert$ in \QM.
	\item For each classical observable one may construct an equivalent
		corresponding \linebreak
		quantum-mechanical operator.
		For example the observable $q_k$ corresponds%
		\footnote{In fact alternative constructions for the position and momentum
				operator are possible as well. In this work the so-called
				\textit{position representation} is presented.}
		to~\cite{Shankar1994, Mueller2000}
		\[ q_k \longrightarrow \op{x}_k = x_k, \]
		\ie just the multiplication with $x_k$,
		the $k$-th coordinate of the system.
		On the other hand $p_k$ corresponds to~\cite{Shankar1994, Mueller2000}
		\[ p_k \longrightarrow \op{p}_k
			= \frac{\hbar}{\I} \frac{\partial}{\partial x_k}, \]
		an appropriately scaled derivative with respect to $x_k$.
	\item Relationships originating from classical mechanics
		can (usually) be transformed into their \QM analogue.
		For this replace all occurrences of the Poisson bracket
		and the contained classical observables with the commutator
		\begin{equation}
			\comm{\Op{F}}{\Op{G}} = \Op{F} \Op{G} - \Op{G} \Op{F}
			\label{eqn:Operator}
		\end{equation}
		and corresponding operators, \ie~\cite{Shankar1994}
		\[ \{F, G\}_P \longrightarrow \frac{\I}{\hbar} \comm{\Op{F}}{\Op{G}}. \]
	\item The measured values of an observable $F$
		are the eigenvalues $\lambda_k$ of $\Op{F}$ \emph{only}.
	\item Assume that we can find a complete and countable
		set of eigenpairs $\{ (\lambda_\mu, \Psi_\mu) \}_{\mu \in \mathcal{I}}$
		for the operator%
		\footnote{We will see in section \vref{sec:spectral}
		that this is \emph{not} always possible.
		A more general treatment of spectral theory involving
		\textit{spectral projectors} allows to reformulate point (f) for cases
		where such eigenpairs cannot be found.
		See \cite{Helffer2013} for details.}
		$\Op{F}$, where $\mathcal{I}$ is an appropriate index set.
		One may then compute the expectation value of a measurement
		on a normalised%
		\footnote{unit normalised, \ie $\braket{\Phi}{\Phi}_\hilbert = 1$}
		state $\Phi$ as
		\begin{equation}
			\langle \Op{F} \rangle
				= \braket{\Phi}{\Op{F} \, \Phi}_\hilbert
				= \sum_{\mu\in\mathcal{I}} \lambda_\mu \,
							\abs{\braket{\Psi_\mu}{\Phi}_\hilbert}^2,
			\label{eqn:QMExpectation}
		\end{equation}
		where $\braket{\slot}{\slot}_\hilbert$ is the inner product
		of the Hilbert space $\hilbert$.
\end{enumerate}
Now that we discussed the \textit{ad hoc} modification of classical mechanics
in order to yield a theory based on the postulates of \QM,
let us see how one is able to deduce useful results of \QM
from the analogous expressions in classical mechanics.
For example from \eqref{eqn:PoissonProperties} we can immediately
deduce important commutator relations between the position and momentum operators:
\begin{align}
	\comm{\op{x}_k}{\op{x}_l} &= \comm{\op{p}_k}{\op{p}_l} = 0 & \comm{\op{x}_k}{\op{p}_l} = \frac{\hbar}{\I} \delta_{kl}.
	\label{eqn:compatibilityOperator}
\end{align}
Similarly from \eqref{eqn:PoissonTimeEvolution} we obtain
\begin{equation}
	\frac{\D \Op{F}}{\D t} = \frac{\partial \Op{F}}{\partial t} + \frac{\I}{\hbar} \comm{\Op{F}}{\Op{H}},
	\label{eqn:HeisenbergEqnMotion}
\end{equation}
the equation of motion, which governs the time-evolution of the operator $\Op{F}$
in the so-called \newterm{Heisenberg picture} of \QM.
Taking the statistical average over \eqref{eqn:HeisenbergEqnMotion}
results in the Ehrenfest theorem
\begin{equation}
	\frac{\D \langle \Op{F} \rangle}{\D t} = \frac{\partial \langle \Op{F} \rangle}{\partial t} + \frac{1}{\I \hbar} \left\langle \comm{\Op{F}}{\Op{H}} \right\rangle.
	\label{eqn:EhrenfestTheorem}
\end{equation}
This result allows to rationalise the \newterm{correspondence principle}
of classical mechanics and \QM,
which we have developed so far.
Comparing \eqref{eqn:EhrenfestTheorem} and \eqref{eqn:PoissonTimeEvolution}
and keeping in mind that the expectation value $\langle \Op{F} \rangle$
as well as the classical observable $F$ both tell us about the result
of a measurement,
we can deduce that --- on average --- classical mechanics still holds.
Thus we may expect the classical expressions to carry some meaning
in the \QM sense as well.

\subsection{The Schrödinger equation}
Even though the Heisenberg picture developed
in the previous section is descriptive for deducing the analogy between
classical mechanics and \QM,
it is not very suitable for the kinds of problems we will be looking
at in the remainder of this thesis.
More suitable for our needs is the \newterm{Schrödinger picture} of \QM,
which differs in the way it treats time evolution.
In the Heisenberg picture the state function $\Psi$ is time-independent
and the operators evolve.
In the Schrödinger picture it is the other way round, i.e. $\Psi$ may change over time
and the operators are static.
Both pictures are related by a unitary transformation
in the Hilbert space $\hilbert$
governed by the Stone-von Neumann theorem\cite{Neumann1931,Neumann1932,Stone1932}.

The equivalent expression to \eqref{eqn:HeisenbergEqnMotion}
in the Schrödinger picture is the \newterm{time-dependent Schrödinger equation}%
~\cite{Shankar1994,Mueller2000}
\begin{equation}
	\Op{H} \Psi = \I \hbar \frac{\partial}{\partial t} \Psi.
	\label{eqn:TDSE}
\end{equation}
Similar to \eqref{eqn:HeisenbergEqnMotion} the key operator governing the
time-evolution of the system is $\Op{H}$.
By analogy to its classical counterpart $\Op{H}$ is referred to as the
\newterm{\QM Hamiltonian} or just Hamiltonian as well.
In fact many properties of a system may already be deduced by considering the
eigendecomposition of its Hamiltonian $\Op{H}$ alone.
In light of this the ansatz
\begin{equation}
	\Op{H} \Psi_\mu = E_\mu \Psi_\mu
	\label{eqn:TISE}
\end{equation}
for finding the Hamiltonian's eigenpairs
\[ (E_\mu, \Psi_\mu) \in \R \times \hilbert \]
is given the name \newterm{time-independent Schrödinger equation}~(TISE) as well.
\defineabbr{TISE}{TISE\xspace}{Time-independent Schrödinger equation}

\pagebreak
\subsection{The Hamiltonian of the hydrogen-like atom}
\nomenclature{$\Delta$}{Laplace operator.}
\nomenclature{$\Op{T}$}{Kinetic energy operator.}
\nomenclature{$\Op{V}$}{Potential energy operator.}
Employing the correspondence principle it is often very convenient
to construct the \QM Hamiltonian of a system starting
from the classical energy expression.
Let us consider a hydrogen-like system,
where a particle of positive charge $Ze$
is clamped at the origin and surrounded by a single electron.
In Cartesian coordinates the position of the electron can be described
by the vector $\vec{r} \equiv \rtp{(x_1, x_2, x_3)}$ and its momentum by
the vector $\vec{p}$.
The classical kinetic energy and potential energy of such a system
are given by
\begin{align}
	T &= \frac{\vec{p} \cdot \vec{p}}{2 m_e} &
	&\text{and}&
	V &= - \frac{Ze^2}{\norm{\vec{r}}_2} = - \frac{Ze^2}{r}. \\
\intertext{respectively. The appropriate \QM analogues are}
	\Op{T} &= -\frac{\hbar^2}{2 m_e} \Delta &
	&\text{and}&
	\Op{V} &= - \frac{Ze^2}{r}
	\label{eqn:OpHydrogenConventional}
\end{align}
where the \newterm{Laplace operator} in $3$ dimensions
\begin{align}
	\Delta = \sum_{i=1}^3 \frac{\partial^2}{\partial x_i^2}
	\label{eqn:LaplaceOperatorHydrogen}
\end{align}
was introduced. The full Hamiltonian therefore reads
\begin{equation}
	\Op{H} = \Op{T} + \Op{V} = -\frac{\hbar^2}{2 m_e} \Delta - \frac{Ze^2}{r}.
	\label{eqn:OpHydrogenFullUnits}
\end{equation}

The eigenvalues of the hydrogen-like Hamiltonian \eqref{eqn:OpHydrogenFullUnits}
can be determined analytically,
see section \vref{sec:HydrogenAtom} for details.
The lowest eigenvalue is an energy of around $\unit[-2.18\E{-18}]{J}$,
hardly a convenient number.
In fact most energy values in quantum chemistry and physics
are of this order of magnitude.
Similarly many other relevant quantities like the charge of the involved
particles or typical lengths are only very small numbers.
For this reason so-called \newterm{atomic units} are typically employed.
These are generated by a unitary transformation of the Hilbert space,
which effectively yields
\[ \hbar \equiv e \equiv m_e \equiv a_0 \equiv E_h \equiv 1. \]
See table \vref{tab:AtomicUnits} for the values of these quantities
in terms of the usual SI units.
%
Employing this transformation on \eqref{eqn:OpHydrogenFullUnits} gives rise to
\begin{equation}
	\Op{H} = -\frac12 \Delta - \frac{Z}{r}
	\label{eqn:OpHydrogen}
\end{equation}
for the hydrogen-like Hamiltonian in atomic units.
Its lowest energy eigenvalue is $-1/2E_h$,
certainly a more pleasant number.
Additionally the sketched transformation has simplified
the expression of the operator from \eqref{eqn:OpHydrogenFullUnits}
to \eqref{eqn:OpHydrogen},
which is in fact a general observation for the relevant
equations of quantum physics and chemistry,
providing another justification for their use.
From now on we will work exclusively in these units.
A more detailed discussion of atomic units,
approaching the subject from a slightly different angle can be found
in \cite{Quack2011}.

\begin{table}
	\centering
	\begin{tabular}{lllll}
		\toprule
		symbol & name & atomic unit of & value in SI units \\
		\midrule
		$\hbar$ & Planck constant & action & $\unit[1.055\E{-34}]{Js}$ \\
		$e$ & elementary charge & charge & $\unit[1.602\E{-19}]{C}$ \\
		$m_e$ & electron mass & mass & $\unit[9.109\E{-31}]{kg}$ \\
		$a_0$ & Bohr radius & length & $\unit[5.292\E{-11}]{m}$ \\
		$E_h$ & Hartree energy & energy & $\unit[4.360\E{-18}]{J}$ \\
		\bottomrule
	\end{tabular}
	\caption[Atomic units and their relationship to SI units]
		{Atomic units and their relationship to SI units.
			Values taken from \cite{CODATA2014}.}
	\label{tab:AtomicUnits}
\end{table}

\section{Elements of functional analysis}

In section \vref{sec:IntroQM} we discussed that the state
of a \QM system is governed by a vector in a complex-valued Hilbert space,
whilst information about the system itself can be obtained by studying
appropriate self-adjoint operators defined on such Hilbert spaces.
Precisely this is the mathematical field of functional analysis,
which we will briefly review in this section.
We assume basic familiarity with the concept of a vector space as well
as Lebesgue integrability for our treatment.
A more basic material developing the concept of Lebesgue integrability
and Hilbert spaces from standard Euclidean geometry can be found
in a related work by the author~\cite{DiracNotation}.

\subsection{Hilbert spaces}
\label{sec:Hilbert}
\nomenclature{$\F$}{Either the field of real numbers $\R$ or the field of complex numbers $\C$.}

\todoil{Rephrase}
The key structures,
which a Hilbert space possesses on top of a usual vector space
is the so-called completeness under the norm induced
by the inner product of the Hilbert space.
For our treatment the field $\F$ can be identified
either with the field of all real numbers $\R$
or the field of all complex numbers $\C$.

\begin{defn}
	An \newterm{inner product space} over a field $\F$
	is a vector space $V$ (over the same field)
	that is further equipped with an inner product, i.e. a map
	\[ \braket{\slot}{\slot}_V : V \times V \to \F \]
	that satisfies (for all vectors $x,y,z \in V$ and all $\alpha \in \set{F}$)
	\begin{align}
		\label{eqn:innProdConjSym}
			&\braket{x}{y}_V^\ast = \braket{y}{x}_V &&
			\text{\textit{(conjugate symmetry)}} \\
		\label{eqn:innProdLinLeft}
			&\braket{x}{\alpha y + z}_V = \alpha \braket{x}{y}_V + \braket{x}{y}_V &&
			\text{\textit{(linearity in the first argument)}} \\
		\label{eqn:innProdPosDef}
			&\braket{x}{x}_V \ge 0 \quad \text{and}
			\quad \braket{x}{x}_V = 0 ~\Rightarrow~ x = 0  &&
			\text{\textit{(positive-definiteness)}},
	\end{align}
	where the asterisk ``$^\ast$'' denotes complex conjugation.
	We typically drop the ``$V$'' subscript from the notation of the inner
	product if the underlying vector space is clear from context.
\end{defn}

\begin{defn}
	Given a vector space $V$ over the field $\F$, a \newterm{norm} is a map
	$\norm{\slot}_V : V \to \R$
	such that the following axioms hold for all vectors $x,y \in V$ and all $\alpha \in \set{F}$:
	\begin{align}
		\label{eqn:normScalability}
			&\norm{\alpha x} = \abs{\alpha} \, \norm{x} &&
			\text{\textit{(absolute scalability)}} \\
		\label{eqn:normTriaIneq}
			&\norm{x + y} \le \norm{x} + \norm{y} &&
			\text{\textit{(triangle inequality)}} \\
		\label{eqn:normPointSep}
			&\text{If}~\norm{x} = 0 \quad \Rightarrow
			\quad \text{$x$ is the zero vector} &&
			\text{\textit{(norm separates points)}}
	\end{align}
	If such a norm can be found for a particular vector space $V$,
	one typically refers to $V$ as a \newterm{normed vector space} as well.
\end{defn}

\begin{prop}
	For every inner product space exists the so-called \newterm{induced norm}
	\begin{equation}
		\norm{x}_V = \sqrt{ \braket{x}{x}_V } \qquad \forall x \in V.
		\label{eqn:normInduced}
	\end{equation}
	One may drop the subscript on the norm
	if it is clear from context.
	\begin{proof}
		See~\cite{DiracNotation}.
	\end{proof}
\end{prop}

\begin{defn}
	\label{def:Completeness}
	A vector space $V$ is called \newterm{complete} if every Cauchy sequence
	of vectors in  $V$ has a limit which is in $V$ as well.
\end{defn}
Remember that a sequence $(x_n) \in V$ is called Cauchy if
\[ \forall \varepsilon > 0 \quad
	\exists M \in \N \quad \text{such that} \quad
	\norm{x_n - x_m}_V < \varepsilon \quad \forall n,m > M,
\]
which is equivalent to the sequence converging to a limit.
An equivalent way of phrasing \abbr{def}{n} \vref{def:Completeness}
is therefore,
that a space $V$ is complete if every converging sequence
converges inside $V$.
Note that the completeness property of a space is
tightly bound to the norm which is used.
A space might well be complete with respect to one norm,
but not with respect to another.

\begin{exmp}
	To make this a little more clear let us consider a counterexample.
	It is well known that the sequence
	\[ x_n = \sum_{k=0}^n \frac{1}{k!} \in \set{Q} \]
	converges to Euler's number, \ie
	\[ \lim_{n\to\infty} x_n = e \not\in \set{Q}. \]
	In other words $\set{Q}$ (seen as a vector space over itself)
	is not complete.

	One may, however, build the \newterm{completion} of $\set{Q}$
	with respect to a norm by just including all limiting points of
	all sequences in $\set{Q}$. In fact this is one way
	of defining the set of real numbers $\R$.
\end{exmp}

\begin{defn}
	A subspace $S$ of a vector space $V$ is called \newterm{dense in $V$}
	if each vector $x \in V$ either is a member of $S$ or one may find
	a Cauchy sequence in $S$ for which $x$ is the limit point.
\end{defn}
In other words $S$ is dense in $V$ if it amounts to approximate
all vectors of $V$ up to an arbitrarily desired accuracy.

\begin{exmp}
	$\set{Q}$ is dense in $\R$. 
	Essentially this is the reason why we may approximate any real number
	up to arbitrary accuracy by an appropriate sum of countably many fractions.
\end{exmp}

\begin{nte}
	The concepts of complete spaces and dense subspaces
	are applicable to more mathematical structures than just vector spaces.
	In the broadest sense these properties can be applied to so-called
	metric spaces, which are related to normed vector spaces,
	but in fact have less structure.
\end{nte}

With the concept of completeness in mind,
we have all we need to properly define a Hilbert space.
\begin{defn}
	A \newterm{Hilbert space} $\hilbert$ is an inner product space,
	which is complete with respect to the induced norm.
\end{defn}
In other words a Hilbert space is a space,
where the inner product naturally defines a way to measure distances
and take limits and where we can be sure that these limits are in the space as well.
If you think back to standard calculus in $\R^d$,
this in some sense makes sure that we can perform any kinds of limiting processes
like differentiation inside Hilbert spaces without ever having to worry
about leaving them.
For our goal of defining differential operators on them,
this is exactly what we need.

\nomenclature{$L^2(\R^d, \C)$}{The Hilbert space of square-integrable complex-valued functions, see \vref{prop:L2HilbertSpace}}
\begin{prop}
	\label{prop:L2HilbertSpace}
	Given two functions $f, g : \R^d \to \C$,
	let us define the inner product
	\begin{equation}
		\braket{f}{g}_{L^2} := \int_{\R^d} \cc{f}(\vec{x}) g(\vec{x}) \D \vec{x}
		\label{eqn:defL2InnProd}
	\end{equation}
	and the corresponding induced norm
	\begin{equation}
		\norm{f}_{L^2} := \sqrt{\braket{f}{f}_{L^2}} = \left( \int_{\R^d} \abs{f(\vec{x})}^2 \D \vec{x} \right)^{1/2}.
		\label{eqn:defL2Norm}
	\end{equation}
	In both cases the integral is to be understood in the Lebesgue sense
	and we identify
	\[ \vec{x} \equiv (x_1, \ldots, x_d) \]
	for ease of notation. The set
	\[ L^2(\R^d, \C) := \big\{ f : \R^d \to \C ~\big|~ \norm{f}_{L^2} < \infty \big\}, \]
	where the norm $\norm{\slot}_{L^2}$ stays finite
	is the set of \newterm{square-integrable functions}.
	It forms a is a Hilbert space over the field $\C$.
	\begin{proof}
		See~\cite{Adams2003}
	\end{proof}
\end{prop}

Finally let us note some important properties of $L^2(\R^d, \C)$.

\begin{defn}
	A Hilbert space is \newterm{separable}%
	\footnote{In the broader context of metric spaces,
	a separable space has a countable, dense subset}
	iff it admits a countable orthonormal basis.
\end{defn}
One can show that all separable Hilbert spaces are
isometrically isomorphic towards another,
since they are all isometrically isomorphic to the space of
square-summable sequences $l^2$ in $\C$
by the means of a suitable basis chosen in each space.

\begin{prop}
	$L^2(\R^d, \C)$ is separable.
	\begin{proof}
		\todoil{TODO Find reference}
	\end{proof}
\end{prop}

In other words the space $L^2(\R^d, \C)$ is rather nice,
because all functions inside it can be approximated
by only a countable set of basis functions
and it further admits for limits to be taken.

\subsection{Hilbert spaces in Quantum Mechanics}
\todo[caption={},inline]{
	Make sure to bring across:
	\begin{itemize}
		\item In \QM we need $L^2$ because of statistics
		\item All separable Hilbert spaces are equivalent
		\item Separability is nice, because this allows for a countable set to be good enough.
	\end{itemize}
}

Having discussed Hilbert spaces, let us return to the question,
which Hilbert space is most applicable in \QM.

In our introduction \vref{sec:IntroQM} we said that
we employ functions $\Psi : \R^d \to \C$ from a
complex separable Hilbert space $\hilbert$
Now the so-called \newterm{Born interpretation} of \QM
associates the meaning of a probability density
with the norm $\norm{\Psi(x_1, \ldots x_N)}$
of a state at each point in space $(x_1, \ldots x_N)$.
From a more detailed analysis with respect to the implications
for the statistical properties of $\Psi$
this requires us to take a $\Psi \in L^2(\R^d, \C)$.

In fact the choice of states from $L^2(\R^d, \C)$
is only correct for \emph{non-relativistic, spin-free} \QM.
If more than one spin component are required,
like $4$ in relativistic \QM or $2$ in spin-adapted \QM,
we need to take states from the spaces $L^2(\R^d, \C^4)$ or $L^2(\R^d, \C^2)$,
respectively.
Since this thesis only deals with non-relativistic \QM
we only care about $L^2(\R^d, \C^2)$ or $L^2(\R^d, \C)$.
For simplicity we will only analyse the mathematical structures
based on $L^2(\R^d, \C)$ in this and the next chapter
and introduce spin \textit{ad-hoc} whenever needed
keeping in mind that all results from this and the next chapter
can be trivially generalised to a \QM based on $L^2(\R^d, \C^2)$ as well.

From a computational point the choice of $L^2(\R^d, \C)$ is rather convenient,
because this allows us to approximate a \QM state with only a countable
set of parameters
encoded in the countable dense subset of $L^2(\R^d, \C)$ we employ for the
approximation.
And by choosing more and more parameters and better suitable parameters
the denseness ensures that we converge to a sensible limiting state
in $L^2(\R^d, \C)$.

\subsection{Sobolev spaces}
Many operators of quantum mechanics involve derivatives as well.
Whilst Lebesgue spaces are natural when it comes to doing statistics,
they are too weak to make sure
that derivatives of these functions are in $L^2(\R^d, \C)$ as well.
As many \QM operators (like the momentum operator) involve
derivatives this space is often not suitable.

\begin{defn}
	A function $f \in L^2(\R^d, \C)$ has a \newterm{weak partial derivative}
	$g \in L^2(\R^d, \C)$ with respect to $x_i$ if
	\[ \forall \varphi \in C^\infty_0(\R^d): \braket{g}{\varphi}_{L^2} = - \braket{f}{\frac{\partial}{\partial x_i}\varphi}_{L^2}, \]
	where $C^\infty_0(\R^d, \C)$ is the space of all
	infinitely differentiable complex-valued functions with compact support.
	To denote the weak derivative
	one may write $g = \frac{\partial}{\partial x_i} f$ like in the usual case.
	It can further be shown that if $f$ has a classical, strong derivative
	then it also has a weak derivative, which coincides with the strong derivative.
	For ease of notation we also write
	\[ D^\alpha f = \frac{\partial^{\abs{\alpha}}}{ \prod_{i=1}^d \partial x_i^{\alpha_i} }. \]
\end{defn}

\nomenclature{$H^1(\R^d, \C)$}{The Hilbert space of complex-valued functions with square-integrable first derivative}
\nomenclature{$H^2(\R^d, \C)$}{The Hilbert space of complex-valued functions with square-integrable second derivative}
\begin{defn}
	The \newterm{Sobolev space} defined by
	\begin{equation}
		H^k(\R^d, \C) := \left\{ f \in L^2(\R^d, \C) \middle| D^\alpha f \in L^2(\R^d, \C) \text{ for $\abs{\alpha} \le k$} \right\}
		\label{eqn:defSobolev}
	\end{equation}
	with inner product
	\begin{equation}
	\braket{f}{g}_{H^k} := \sum_{\abs{\alpha} \le k} \braket{D^\alpha f}{D^\alpha g}_{L^2}.
		\label{eqn:defSobolevInnProd}
	\end{equation}
	and induced norm
	\begin{equation}
		\norm{f}_{H^k} = \sum_{\abs{\alpha} \le k} \norm{D^\alpha f}_{L^2}
		\label{eqn:defSobolevNorm}
	\end{equation}
	is a Hilbert space.
\end{defn}

\begin{defn}
	The completion of $C^\infty_0(\R^d, \C)$
	with respect to the norm $\norm{\slot}_{H^k}$
	is the Sobolev space $H^k_0(\R^d, \C)$.
	It is a proper subspace of $H^k$ and also a Hilbert space.
\end{defn}
\todoil{Explanation what this means}

\begin{figure}
	\centering
	\includeimage{1_qm/sobolev}
	\caption{Overview of the spaces discussed in this chapter.
		Apart from $C^\infty_0(\R^d, \C)$ all mentioned spaces are Hilbert spaces.
		In each case $A \subset B$ denotes that $A$ is a proper, dense
		subspace of $B$.}
	\label{fig:sobolevRelations}
\end{figure}
The various relationships between the spaces we discussed
in this chapter have been summarised in \fig \vref{fig:sobolevRelations}.
Note that by definition
\[ L^2(\R^d, \C) = H^0(\R^d, \C) = H^0_0(\R^d, \C). \]

\begin{exmp}
	\label{exmp:H1sH1}
	The function
	\[ \Psi_{1s}(\vec{r}) = \exp\left(- \sqrt{x^2 + y^2 + z^2} \right) = \exp(-r), \]
	which arises as an eigenfunction
	of the hydrogen-like Hamiltonian \eqref{eqn:OpHydrogen}%
	~(see section \vref{sec:HydrogenAtom})
	belongs to $H^1(\R^3, \C)$
	\begin{proof}
	Since the function is bounded, we clearly have $\Psi_{1s} \in L^2(\R^3, \C)$.
	Furthermore for any $\alpha \in \{x, y, z\}$:
	\begin{equation}
		\norm{ \frac{\partial \Psi_{1s}}{\partial \alpha} }_{L^2}
		= \norm{ - \frac{\alpha}{r} \exp(-r) }_{L^2}
		= \int_{\R^3} \frac{\alpha^2}{r^2} \exp(-2r) \D\vec{r}
		\leq \int_{\R^3} \frac{r^2}{r^2} \exp(-2r) \D\vec{r}
		\label{eqn:ProofH1sH1}
	\end{equation}
	Due to the properties of the Lebesgue integral,
	we may ignore the removable discontinuity at $\vec{r} = \vec{0}$
	and instead write
	\[ \norm{ \frac{\partial \Psi_{1s}}{\partial \alpha} }_{L^2}
		\leq \int_{\R^3} \exp(-2r) \D\vec{r} = \norm{\Psi_{1s}}_{L^2} < \infty. \]
	This shows that $\exp(-r) \in H^1(\R^3, \C)$,
	since each term of \eqref{eqn:defSobolevNorm} is bound.
	\end{proof}
\end{exmp}

\begin{prop}[Hardy's inequality]
	\label{prop:Hardy}
	For all $u \in H^1(\R^3, \C)$, we have
	\[ \int_{\R^3} \norm{\nabla u}_2^2 \D\vec{r}
		\geq \frac{1}{4} \int_{\R^3} \frac{\abs{u}^2}{r^2} \D\vec{r} \]
	\begin{proof}
		For a proof of the special case $u \in C_0^\infty(\R^3, \C)$
		see \cite[p. 30]{Helffer2013}.
		The more general case we claim here,
		follows from the denseness of $C_0^\infty(\R^3, \C)$ in $H^1(\R^3, \C)$
		and continuity of the integrands on both sides
		with respect to the $H^1$ norm.
		\todoil{I do not think I understand this}
	\end{proof}
\end{prop}

\begin{cor}
	\label{cor:Hardy}
	If $u \in H^1(\R^3, \C)$, then $\frac{u}{r} \in L^2(\R^3, \C)$.
	\begin{proof}
		One easily rewrites Hardy's inequality to
		\[
			\norm{u}_{H^1} \geq  \sum_{\alpha\in\{x,y,z\}} \int_{\R^3} \abs{\frac{\partial u}{\partial \alpha}} \D\vec{r}
			\geq \int_{\R^3} \norm{\nabla u}_2^2 \D\vec{r}
			\geq \frac{1}{4} \int_{\R^3} \frac{\abs{u}^2}{r^2} \D\vec{r}
			= \frac{1}{4} \norm{\frac{u}{r}}_{L^2}
		\]
		which proves the claim.
	\end{proof}
\end{cor}

\begin{exmp}
	\label{exmp:H1sH2}
	We now want to use corollary \vref{cor:Hardy} to prove that
	$\Psi_{1s} \in H^2(\R^3,\C)$.
	\begin{proof}
		Considering our result from \eqref{eqn:ProofH1sH1}
		we find that for all $\alpha, \beta \in \{x,y,z\}$:
		\begin{align*}
			\norm{\frac{\partial^2 \Psi_{1s}}{\partial \alpha \partial \beta}}_{L^2}
			&\leq \norm{ \frac{\delta_{\alpha\beta}}{r} \exp(-r)}_{L^2}
			+ \norm{\frac{\alpha\beta}{r^3}\exp(-r)}_{L^2}
			+ \norm{\frac{\alpha\beta}{r^2}\exp(-r)}_{L^2}
		\end{align*}
		Now let us employ $\abs{\alpha\beta} \le r^2$
		and the fact that a removable singularity
		can be ignored in the Lebesgue integral to arrive at
		\begin{align*}
			\norm{\frac{\partial^2 \Psi_{1s}}{\partial \alpha \partial \beta}}_{L^2}
			&\leq \norm{\frac{1}{r} \exp(-r)}_{L^2}
			+ \norm{\frac{r^2}{r^3}\exp(-r)}_{L^2}
			+ \norm{\frac{r^2}{r^2}\exp(-r)}_{L^2} \\
			&= 2\norm{\frac{1}{r} \exp(-r)}_{L^2} + \norm{\exp(-r)}_{L^2} \\
			&< \infty,
		\end{align*}
		where in the last line we used that
		$\exp(-r) \in·H^1(\R^3, \C)$ and corrollary \vref{cor:Hardy}.
		This implies the claimed statement.
\end{proof}
\end{exmp}

\begin{exmp}
	In an analogous manner to examples \vref{exmp:H1sH1} and \vref{exmp:H1sH2}
	one could attempt to probe whether the one-dimensional function
	$f(x) = \exp(-\abs{x})$ lives in $H^1(\R,\C)$ or $H^2(\R,\C)$.
	Whilst the former can be easily verified,
	one finds $f \not\in H^2(\R,\C)$.
	\todoil{Show this}
\end{exmp}

This rather surprising result is a consequence of the second part of
the Sobolev embedding theorem
of which we only present a slightly specialised form  here.
\begin{thm}[Sobolev embedding]
	Given $r, k, d \in \N$ with
	\[ k > \frac{d}{2} > 0 \quad \text{and} \quad k -\frac{d}{2} > r \]
	one may find an embedding
	\[ H^k(\R^d) \subset C^r(\R^d) \]
	between the Sobolev space $H^k(\R^d)$ and the space of the $r$ times
	continuously differentiable functions, $C^r(\R^d)$.
\end{thm}
This embedding theorem allows to give an idea what is to be expected about
the smoothness of a function in $H^k$.
Interestingly the smaller the dimensionality the more smooth such a function
has to be.

\subsection{Representation theorems}
\todoil{If there is time talk about the Ritz and the Lax-Milgram}

\begin{thm}[Rietz's representation theorem]
	\label{thm:Riesz}
	bla
	\begin{proof}
		\todoil{Find reference}
	\end{proof}
\end{thm}

\section{Spectral theory}
\todo[inline,caption={}]{
	\begin{itemize}
		\item Elaborate on the spectral theory of bounded operators
		\item Sturm-Liouville theory
		\item See Erics notes
		\item Keep it short, just what is needed for HF
	\end{itemize}
}

\url{https://en.wikipedia.org/wiki/Self-adjoint_operator}
\url{https://en.wikipedia.org/wiki/Spectral_theorem}
\url{https://en.wikipedia.org/wiki/Extensions_of_symmetric_operators}

% Discuss the spectral properties of the Laplace and Schrödinger operators


% Important:
%https://en.wikipedia.org/wiki/Self-adjoint_operator
%If A is symmetric and D o m ( A ) = H {\displaystyle \mathrm {Dom} (A)=H} {\displaystyle \mathrm {Dom} (A)=H}, then A is necessarily bounded.
%Hence if the operator is not bounded (as most operators in QM) in cannot be self-adjoint.


self-adjoint operator
(show that it has real eigenvalues)

symmetric operator


% This section should make clear
% The importance of the Sobolev space $H_2$ for quantum mechanics

\subsection{The Laplace operator}

or alternatively in spherical polar coordinates
as $(r, \theta, \varphi)$ with
\begin{align*}
	r &= \norm{\vec{r}} = \sqrt{x^2+y^2+z^2} & \theta &= \arccos\frac{z}{r} & \varphi &= \arcsin\frac{y}{x}.
\end{align*}




\label{sec:SpectrumLaplace}
Let us consider the $N$-dimensional analogue of the Laplace operator
introduced in \eqref{eqn:LaplaceOperatorHydrogen}.
In Cartesian coordinates it reads
\[ \Delta = \sum_{i=1}^d \frac{\partial^2}{\partial x_i^2}. \]
This operator plays a crucial role in quantum mechanics.
We will see later\todo{refer} that this operator is part of the
kinetic operator of the Schrödinger equation.
For a physical equation we expect that the operator has
real, i.e. measurable, eigenvalues.
This typically requires the operator to be self-adjoint.
In other words from a physical standpoint we would expect
the Laplace operator to have real eigenvalues.

As it turns out,
the Laplacian operator $T_0 = \Delta$ with the domain $D(T_0) = L^2(\R^d)$
does not have \emph{any}
eigenfunctions $u \in L^2(\R^d)$,
not even in the weak sense of distributions~\cite{Helffer2013}.
In other words we are unable to find an eigenfunctions $u \in L^2(\R^d)$
and a $\lambda \in \R$,
such that it holds
\[ \forall \phi \in C^\infty_0(\R^d):  \int_{\R^d}\phi(x) (\Delta u)(x) \D x = \lambda \int_{\R^d} \phi(x) u(x) \D x. \]
Furthermore the Laplacian operator $T_0$ is not self-adjoint.

It turns out that the domain of the operator is the culprit here.
Converse to the problem we posed it is actually possible
to find \emph{for each} $\lambda \in \C$ an eigenfunction from the
slightly different so-called space of distributions $D'(\R^d)$.
This suggests that the construction of an appropriate, related
operator $T_1$, for which we can find an eigenspectrum.
In fact one may do so in a way, that $T_1$ is even self-adjoint
by the means of the \newterm{Friedrichs extension}.
In the particular case we discussed, there are actually
multiple ways to achieve this.
One way assumes a Dirichlet boundary, where the potential
eigenfunctions go to zero
one obtains the restricted domain
$D(T_1) = H^2(\R^d) \cap H^1_0(\R^d)$
for the self-adjoint Laplacian extension.
Unlike for other operators such an extension for the Laplacian
is not unique.
Another construction assuming a Neumann boundary is possible, too.
It again yields a subset of $H^2(\R^d)$ just with an appropriate
so-called trace condition enforcing the derivative to go to zero
at the boundary.
Most notably this implies that only under specification
of the boundary condition the operator has an eigenspectrum.
This is an observation which is true for many operators in quantum mechanics.

If $u \in H^2(\R^d)$ then $\Delta u \in L^2(\R^d)$

\subsection{The Laplace-Beltrami operator on the unit sphere}
\label{sec:SpectrumLaplaceBeltrami}
\nomenclature{$Y_l^m(\theta, \varphi)$}{Spherical harmonic function with angular
momentum quantum number $l$ and azimuthal quantum number $m$}
\nomenclature{$P^{m}_l$}{Associated Legendre polynomial with orders $l$ and $m$.}

In the previous section, we saw that the Laplace operator $\Delta$
is self-adjoint and has hence real eigenvalues provided that
we choose for its domain a subdomain of $D(\Delta) = H^2(\R^3)$.

In this section we want to discuss the solution of Laplace's equation
on the surface of the unit sphere
\[
	\set{S}^2 := \{ \vec{r} \in \R^3 | x^2 + y^2 + z^2 = 1 \}.
\]
For this it is most convenient to consider
the spherical polar coordinate system,
\ie instead of specifying the vector $\vec{r} = (x, y, z)^T$,
we specify the parameters $r, \theta, \phi$ or
\[ \vec{r} = \mm{ r \, \sin \theta \, \cos \phi \\ r \, \sin \theta \sin \phi \\
r \, \cos \theta }. \]
The condition for the unit sphere than reduces to $r=1$

Since the sphere is no longer a Euclidean geometry
but much rather a curved manifold some of the treatment presented
in the previous chapter no longer holds.
Nevertheless an equivalent to the Friedrichs extension can be found
in order to make the Laplace-Beltrami operator
\[ \Delta_{\set{S}^2} u = \frac{1}{\sin \theta} \frac{\partial}{\partial \varphi}
\left( \sin \varphi \frac{\partial u}{\partial \varphi}  \right)
+ \frac{1}{(\sin \varphi)^2} \frac{\partial^2}{\partial \theta^2} u\]
self-adjoint.
\todoil{Insert details}

The above polar expression for the Laplace-Beltrami operator
allows to explicitly solve the equation
\newcommand{\laplaceSphere}{\Delta_{\set{S}^2}}
\[ \laplaceSphere u = \lambda u \]
for eigenfunctions $u$ and eigenvalues $\lambda$,
which results in the spherical harmonics
\[ Y_l^m(\theta, \varphi) = N_{lm} P_l^m(\cos \theta) e^{im\varphi} \]
where $N_{lm}$ is a normalisation constant
and eigenvalues $-l (l+1)$:
\begin{equation}
	-\laplaceSphere Y_l^m(\theta, \varphi) = l (l+1) Y_l^m(\theta, \varphi)
	\label{eqn:Laplace}
\end{equation}
Importantly the quantum number $l$ is positive and $m$ is restricted by
\[ -l \le m \le l, \]
such that there are exactly $(2l+1)$ spherical harmonics with the same eigenvalue
$-l (l+1)$.

For our following arguments it will be important to note
that the 3d Laplacian in spherical polar coordinates
and the Laplace-Beltrami operator for the unit sphere
are related by
\newcommand{\laplaceRadial}{\frac{\partial}{\partial r} \left( r^2 \frac{\partial}{\partial r} \right)}
\begin{equation}
r^2 \Delta = \laplaceRadial + \laplaceSphere
	\label{eqn:LaplaceCorrespondance}
\end{equation}
such that
\begin{equation}
	- r^2 \Delta Y_l^m(\theta, \varphi) = -\laplaceSphere Y_l^m(\theta, \varphi) = l (l+1) Y_l^m(\theta, \varphi)
	\label{eqn:LaplaceSphericalHarmonic}
\end{equation}


\subsection{The Schrödinger operator for a Hydrogenic atom}
We will discuss in the next chapter the physical origin.
For now we will just state that the Schrödinger operator for the Hydrogenic atom
(in so-called atomic units)
is given by the expression
\begin{equation}
	\Op{H} = - \frac12 \Delta - \frac{Z}{r} = -\frac{1}{2r^2} \laplaceRadial -\frac{1}{2r^2} \laplaceSphere - \frac{Z}{r}
	\label{eqn:HydrogenOperator}
\end{equation}
in other words the scaled Laplace operator and a term for a radial-symmetric Coulombic
potential.

The Hilbert space for this operator is $L^2(\R^3, \C)$
and the domain is $C^\infty_0(\R^3, \C)$
By a similar argument to section \vref{sec:SpectrumLaplace} based on the
Friedrichs extension one notices that this operator is self-adjoint
if the domain is chosen to be $H^2(\R^3, \C$.
Let us now find the analytic expressions of the eigenfunctions of this operator,
i.e. solve the time-independent Schrödinger equation
\begin{equation}
	( \Op{H} - E ) \Psi = 0
	\label{eqn:HydrogenEigenproblem}
\end{equation}
where $\Psi \in L^2(\R^3, \C)$ and $E \in \R$.
\todoil{More details on this later}

Without jumping ahead too far let us assume that both the energy as well
as the state $\Psi$ are characterised by three quantum numbers $n$, $l$ and $m$.
A further careful inspection of \eqref{eqn:HydrogenEigenproblem}
in contrast with \eqref{eqn:Laplace} suggests a product ansatz
\[ \Psi_{nlm}(\vec{r}) = R_{nl}(r) Y_l^m(\theta, \phi) \]
This yields to the radial equation
\begin{equation}
	- \left( \frac{1}{2r^2} \laplaceRadial + \frac{l (l+1)}{2 r^2} - \frac{Z}{r} - E \right) R_{nl}(r) = 0
	\label{eqn:HydrogenRadial}
\end{equation}
which has the solutions~\cite{Mueller2000}
\begin{equation}
	 R_{nl}(r) = N_{nl} \left(\frac{2Zr}{n}\right)^l \exp\left(-\frac{Zr}{n} \right)
\;_1F_1\left(l+1-n | 2l+2|\frac{2Zr}{n}\right)
	\label{eqn:HydrogenRadialSolution}
\end{equation}
where $_1F_1\left(a|b|\zeta\right)$ is a confluent hypergeometric function,
namely~\cite{Avery2006}
\[ _1F_1\left(a \middle| b \middle| \zeta\right) =
\sum_{k=0}^\infty \frac{a^{\bar{k}}}{k! \, b^{\bar{k}}} \zeta^k =
1 + \frac{a}{b} \zeta + \frac{a(a+1)}{2b(b+1)} \zeta^2 + \cdots \]
where $a^{\bar{k}}$ is the rising factorial of $a$.
The normalisaton constant is
\[ N_{nl} = \frac{2 \left( \frac{Z}{n} \right)^{3/2}}{(2l+1)!} \sqrt{ \frac{(l+n)!}{n (n-l-1)!}} \]
The energy eigenvalues are
\[ E_{nlm} = - \frac{Z^2}{2n^2}. \]
\todoil{One should mention that these are all to be found,
	since the radial functions are complete in the radial sense
and the spherical harmonics are complete in the angular sense.}

If one walks through the derivation properly,
one notices that the principle quantum number $n$ is always larger than zero
and that $l < n$ applies as well.
\todoil{If I remember correctly this is a result of Sturm-Liouville theory}
Together with the restrictions on $l$ and $m$ itself this results in the
following restrictions on the quantum numbers:
\begin{align*}
	n &> 0 & 0 \leq l &< n & -l \leq &m \leq l
\end{align*}
Taking a look at the confluent hypergeometric function we use, it is easy to see
that the power of the polynomial in $r$ is between zero and
\[
	\deg \;_1F_1\left(l + 1 -m \middle| 2l+2 \middle| \frac{2Zr}{n} \right) = n - l -1 \]
such that overall the polynomial part of the radial function
has a power between $l$ and
\[ \deg \frac{R_{nl}(r)}{\exp(2Zr/n)} = n - 1. \]

We will now argue that the solutions $\Psi \in H^1(\R^3, \C)$.
It is easy to see that the derivative of $\Psi$ is continuous and bound
everywhere but the origin.
See \cite{Kato1957} and references therein for details.
Since the smoothness at the origin is better the larger
the larger the exponents of the polynomial in $r$,
let us consider the worst case, i.e. the $s$-functions with $l = 0$.
Taking a look at the definition of $_1F_1(a,b,\zeta)$
one notices that the smallest exponent, irrespective of the value of $n$ is $0$.
Thus we will discuss the smoothness properties of
\[
	\Psi_{1s}(r, \theta, \phi) = \Psi_{100}(r, \theta, \phi)
	= \sqrt{\frac{Z^3}{\pi}} \exp(-Z r)
\]
at the origin,
which in fact is the least smooth of all hydrogen solutions.

Since this function is radially symmetric,
we only need to consider the derivative along one direction.
Without loss of generality we choose the $x$ direction.
Taking the first derivative we find
\[
	\frac{\partial \Psi_{1s}}{\partial x} = -\sqrt{\frac{Z^5}{\pi}} \frac{x}{r} \exp(-Z r)
\]
As expected at the origin we encounter a discontinuity in the derivative.
Inserting $x = r \sin\theta \cos \phi$ we do notice
\[
	\lim_{r \to 0} \frac{\partial \Psi_{1s}}{\partial x} = -\sqrt{\frac{Z^5}{\pi}} \sin\theta \cos \phi,
\]
which stays finite.

The second derivative is H1 as well (see Hardy inequality)


The derivative is thus bounded at zero as well
and as a result square-integrable.
On the other hand the second derivative
\[
	\frac{\partial^2 \Psi_{1s}}{\partial x^2}
	= \left( -\frac{Z x^2}{r^2} -\frac{x^2}{r^3} + \frac{1}{r} \right)
		\sqrt{\frac{Z^5}{\pi}} \exp(-Z r)
\]
is unbound at the origin.
Careful inspection shows that it is not square-integrable.
\todoil{Do that}
Therefore $\Psi_{1s} \in H^1(\R^3,\C)$, but $\Psi_{1s} \not\in H^2(\R^3, \C)$.







Show that Solutions are in $H^2$ by considering
the least smooth (i.e. the 1s orbital) and showing that this
is in $H^2$.
That way we know that $H^1$ is a good space for approximating the solutions
$\Rightarrow$ FE formalism can be applied.

\todoil{Try to check whether a productansatz for regularising the nuclear potential
	in combination with CS for the nuclear cusp factor and FE for the rest
could actually work}


% Derivative is smooth and bounded everywhere but at origin
% Argue that the worst function is for l = 0.
% Take derivative wrt. x
% Then show that it's absolute value at the origin just contains a hole
% hence derivative is L^2 integrable in the weak sense



\section{Take-away}
\label{sec:SpectralTakeAway}
Many observations of the one-particle Hydrogenic Schrödinger operator $\Op{H}$
of section \vref{sec:HydrogenAtom}
generalise to the more complicated many-body atomic and molecular Hamiltonians
we will introduce in chapter \vref{ch:qchem}.

Most importantly all these Hamiltonians are unbounded operators,
which become self-adjoint by making the domain
a subspace of the Sobolev spaces $H^2(\R^{3\Nelec}, \C)$.
Their essential spectrum is non-trival,
but luckily one can show~\todo{cite HVZ and Zhislin 61} that
\[ \forall \lambda \in \sigma_{\text{disc}}(\Op{H}),
	\mu \in \sigma_{\text{ess}}(\Op{H}) \quad \lambda < \mu, \]
\ie that the discrete spectrum always sits below the essential spectrum.

In remark \vref{rem:NumericalEssentialSpectrum} we discussed
that the essential spectrum cannot be approximated reliably
by a finite-dimensional Hilbert space with only compact operators.
Our best shot is therefore to follow a numerical approach,
which aims at the description of the low end of the spectrum.
This thus avoids $\sigma_{\text{ess}}(\Op{H})$
and allows to obtain good approximations to at least a few
eigenpairs corresponding to the discrete spectrum $\sigma_{\text{disc}}(\Op{H})$,
\ie bound states.

In practice this is hardly a restriction.
Originating from the laws of thermodynamics
a sensible quantum-mechanical description of a system
typically only requires the lowest energy state,
i.e. the \newterm{ground state}, and the next few \newterm{excited state}s
following most closely in energy.
Care only needs to be taken to choose a sensible approximation
method and a large enough approximation space.
Otherwise one cannot be sure whether the obtained eigenstates
are approximations to true discrete states
or spurious states originating from discretising scattering states
of the continuum.

