\chapter{Mathematical foundation of quantum mechanics}
\label{ch:QM}
\chapquote{
	Whenever we proceed from the known into the unknown we
	may hope to understand,
	but we may have to learn at the same time a new
	meaning of the word ``understanding''.
}{Werner Heisenberg~(1901--1976)}

\nomenclature{iff}{if and only if}
\nomenclature{$\norm{\slot}_2$}{$l_2$-norm or Euclidean norm.}
% TODO Unused
%\nomenclature{$\norm{\slot}_\infty$}{$l_\infty$-norm or maximum norm; largest absolute vector element.}
%\nomenclature{$\norm{\slot}_1$}{$l_1$-norm; sum of absolute vector elements.}

\noindent
At the turn of the 19th century it was discovered
that classical mechanics in the formulations provided
by Joseph-Louis Lagrange and William Hamilton
is not able to capture all effects observed in experiments.
Especially the phenomenon of the black-body radiation spectrum,
but also the photoelectric effect could not be explained.
Finally in 1900 Max Planck somewhat reluctantly introduced the idea of discrete,
i.e. quantised, energy levels
in order to explain the black-body radiation spectrum,
building on earlier ideas by Ludwig Boltzmann.
This started the development of a quantised theory of mechanics
with major contributions by nowadays famous names such as
Niels Bohr, Max Born, Louis de Broglie, Paul Dirac, Albert Einstein,
Vladimir Fock, Pascual Jordan, John von Neumann, Erwin Schrödinger
and others.

In this chapter we will discuss the mathematical structure
of quantum mechanics in light of the problems of atomic physics
and quantum chemistry.
The discussion will focus on the mathematical fields of functional
analysis and spectral theory as these are key in order
to understand the peculiarities with computing discrete energy levels.
The connection to atomic physics is made clear wherever possible.
The parts of the chapter, where we build up the required
mathematical language might seem rather technical, still.
In the chapter we follow the excellent material by \citet{Shankar1994},
\citet{Mueller2000} and \citet{Helffer2013}.

\section{The correspondence of classical and quantum mechanics}
According to the Hamiltonian formulation of classical mechanics
a physical system with $d$ degrees of freedom
is described by a set of generalised coordinates
$q_1, \ldots, q_d$ along with their canonical momenta $p_1, \ldots, p_d$
Any physically measurable quantity $F$
can now be obtained as a function of these system parameters.
In other words one may define a so-called \newterm{observable}
\newcommand{\clphase}{(q_1, \ldots, q_d, p_1, \ldots p_d)}
$F\clphase$ as a function $\R^{2d} \to \R$
from a vector in \newterm{phase space} to the measured value.
Naturally the coordinates $q_i$ and $p_i$ are observables as well.
The most important observable is the total energy function or \newterm{Hamiltonian}
\begin{equation}
	H\clphase
	\equiv \underbrace{\frac{1}{2} \sum_{k=1}^d m_k p_k^2}_{= T(p_1, \ldots, p_d)}
	+ V(q_1, \ldots, q_d),
	\label{eqn:HamiltonianClassical}
\end{equation}
where $m_k$ is the mass of the particle associated with
the degree of freedom $k$,
$T$ is the kinetic energy observable and $V$ the total potential energy observable.
In the formalism of Hamiltonian mechanics $H$ governs the time evolution
of the system, namely
\begin{align}
	\frac{\D p_i}{\D t} &= - \frac{\partial H}{\partial q_i} &
	\frac{\D q_i}{\D t} &= \frac{\partial H}{\partial p_i}.
	\label{eqn:HamiltonianEqnMotion}
\end{align}

Using these expressions one may of course describe the time evolution
any arbitrary observable as well.
To make the connection to quantum mechanics more apparent,
let us introduce for this purpose the so-called \newterm{Poisson bracket}.
It is the skew-symmetric form
\begin{equation}
	\left\{ F, G \right\}_P \equiv \sum_{j=1}^d \left(
	\frac{\partial  F}{\partial q_j} \frac{\partial G}{\partial p_j}
	- \frac{\partial  F}{\partial p_j} \frac{\partial G}{\partial q_j}
	\right)
	\label{eqn:PoissonBracket}
\end{equation}
and according to
\begin{equation}
	\frac{\D F}{\D t} = \frac{\partial F}{\partial t} + \left\{ F, H \right\}_P.
	\label{eqn:PoissonTimeEvolution}
\end{equation}
it relates the Hamiltonian to the time evolution of any arbitrary observable.
One may also show the relationships
\begin{align}
	\{ q_k, q_l \}_P = \{p_k, p_l\}_P &= 0 & \{ p_k, q_l\}_P = - \delta_{kl} &&\forall k,l \in \{1, \ldots, d\}
	\label{eqn:PoissonProperties}
\end{align}
between the principle system observables.

\subsection{The move to quantum mechanics}
\label{sec:IntroQM}
\defineabbr{QM}{QM\xspace}{quantum mechanics}
We will now introduce (non-relativistic)
quantum mechanics~(QM) in a rather pragmatic manor,
namely by stating a summary what changes in \QM formulation
compared to the classical one.
We will abstain from stating all details at the moment
as the following overview should mainly motivate a more detailed
treatment of the involved mathematical concepts further down this chapter.

\begin{enumerate}
	\item Instead of phase space vectors $\clphase \in \R^{2d}$,
		in \QM one employs functions $\Psi : \R^d \to \C$
		living in a complex separable Hilbert space $\mathcal{H}$
		in order to describe a particular state of the system.
	\item A classical observable $F$ is represented by a
		self-adjoint operator $\Op{F}$ on the Hilbert space $\mathcal{H}$ in \QM.
	\item For each classical observable one may construct an equivalent
		corresponding quantum-mechanical operator.
		For example the observable $q_k$ corresponds%
		\footnote{In fact alternative constructions for the position and momentum
				operator are possible as well. We present the so-called
				\textit{position representation} here.}
		to
		\[ q_k \longrightarrow \op{x}_k = x_k, \]
		\ie just the multiplication with $x_k$,
		the $k$-th coordinate of the system.
		On the other hand $p_k$ corresponds to
		\[ p_k \longrightarrow \op{p}_k
			= \frac{\hbar}{\I} \frac{\partial}{\partial x_k}, \]
		an appropriately scaled derivative with respect to $x_k$.
	\item Relationships originating from classical mechanics
		can (usually) be transformed into their \QM analogue.
		For this replace all occurrences of the Poisson bracket
		and the contained classical observables with the commutator
		\begin{equation}
			\comm{\Op{F}}{\Op{G}} = \Op{F} \Op{G} - \Op{G} \Op{F}
			\label{eqn:Operator}
		\end{equation}
		and corresponding operators, \ie
		\[ \{F, G\}_P \longrightarrow \frac{\I}{\hbar} \comm{\Op{F}}{\Op{G}}. \]
	\item The measured values of an observable $F$
		are the eigenvalues $\lambda_k$ of $\Op{F}$ \emph{only}.
	\item Assume that we can find a complete and countable
		set of eigenpairs $\{ (\lambda_\mu, \psi_\mu) \}_{\mu \in \mathcal{I}}$
		for the operator $\Op{F}$
		indexed by elements of the set $\mathcal{I}$%
		\footnote{See \cite{Helffer2013} for a more general treatment.}.
		One may then compute the expectation value of a measurement
		for a pure state $\Psi$ normalised to unity as
		\begin{equation}
			\langle \Op{F} \rangle
				= \braket{\Psi}{\Op{F} \, \Psi}_\mathcal{H}
				= \sum_{\mu\in\mathcal{I}} \lambda_\mu \,
							\abs{\braket{\psi_\mu}{\Psi}_\mathcal{H}}^2,
			\label{eqn:QMExpectation}
		\end{equation}
		where $\braket{~\cdot~}{~\cdot~}_\mathcal{H}$ is the inner product
		of the Hilbert space $\mathcal{H}$.
\end{enumerate}
The statements we present above are in fact very much related
to the postulates of \QM and probably similarly appear as
an \textit{ad hoc} modification of classical mechanics.
One may, however, deduce useful results of \QM from the correspondence
we sketched.
For example from \eqref{eqn:PoissonProperties} one can immediately
deduce important commutator relations between the position and momentum operators:
\begin{align}
	\comm{\op{x}_k}{\op{x}_l} &= \comm{\op{p}_k}{\op{p}_l} = 0 & \comm{\op{x}_k}{\op{p}_l} = \frac{\hbar}{\I} \delta_{kl}.
	\label{eqn:compatibilityOperator}
\end{align}
Similarly from \eqref{eqn:PoissonTimeEvolution} we obtain
\begin{equation}
	\frac{\D \Op{F}}{\D t} = \frac{\partial \Op{F}}{\partial t} + \frac{\I}{\hbar} \comm{\Op{F}}{\Op{H}},
	\label{eqn:HeisenbergEqnMotion}
\end{equation}
the equation of motion, which governs the time-evolution of the operator $\Op{F}$
in the so-called \newterm{Heisenberg picture} of \QM.
Taking the statistical average over \eqref{eqn:HeisenbergEqnMotion}
results in the Ehrenfest theorem
\begin{equation}
	\frac{\D \langle \Op{F} \rangle}{\D t} = \frac{\partial \langle \Op{F} \rangle}{\partial t} + \frac{1}{\I \hbar} \left\langle \comm{\Op{F}}{\Op{H}} \right\rangle.
	\label{eqn:EhrenfestTheorem}
\end{equation}
This results allows to rationalise the \newterm{correspondence principle}
of classical mechanics and \QM,
which we have developed so far.
Comparing \eqref{eqn:EhrenfestTheorem} and \eqref{eqn:PoissonTimeEvolution}
and keeping in mind that the expectation value $\langle \Op{F} \rangle$
as well as the classical observable $F$ both tell us about the result
of a measurement,
we can deduce that --- on average --- classical mechanics still holds.
Thus we may expect the classical expressions to carry some meaning
in the \QM sense as well.

\subsection{The Schrödinger equation}
Even though the Heisenberg picture we developed
in the previous section is great for deducing the analogy between
classical mechanics and \QM,
it is not very suitable for the kinds of problems we will be looking
at in the remainder of this thesis.
Our problems are best described in the \newterm{Schrödinger picture} of \QM,
which differs to the Heisenberg picture by the way it treats time evolution.
In the Heisenberg picture the state function $\Psi$ is time-independent
and the operators evolve.
In the Schrödinger picture it is the other way round, i.e. $\Psi$ may change over time
and the operators are static.
Both pictures are related by a unitary transformation
in the Hilbert space $\mathcal{H}$,
see the Stone-von Neumann theorem for details.

The equivalent expression to \eqref{eqn:HeisenbergEqnMotion}
in the Schrödinger picture is the \newterm{time-dependent Schrödinger equation}
\begin{equation}
	\Op{H} \Psi(\vec{x}) = \I \hbar \frac{\partial}{\partial t} \Psi(\vec{x}).
	\label{eqn:TDSE}
\end{equation}
Similar to \eqref{eqn:HeisenbergEqnMotion} the key operator governing the
time-evolution of the system is $\Op{H}$, the \QM Hamiltonian,
named exactly like the corresponding classical observable $H$.
In fact many useful spectroscopic properties of a system described by $\Op{H}$
may already be deduced by considering the eigendecomposition of this operator.
In light of this one typically refers to the ansatz
\begin{equation}
	\Op{H} \Psi_\mu(\vec{x}) = E_\mu \Psi_\mu(\vec{x})
	\label{eqn:TISE}
\end{equation}
for obtaining the eigenpairs
$\{(E_\mu, \Psi_\mu) \}_{\mu \in \mathcal{I}} \subset \R \times \mathcal{H}$
by the name \newterm{time-independent Schrödinger equation} as well.

\subsection{Atomic units}
Employing the correspondence principle it is often very convenient
to construct the \QM Hamiltonian of a system from the classical energy expression.
Let us consider the hydrogen-like system,
where a particle of positive charge $Z \cdot e$
is clamped at the origin and is surrounded by a single electron.
In Cartesian coordinates the position of the electron can be described by the Cartesian
vector $\vec{r} \rtp{\equiv (x, y, z)}$.
Such a system has the classical kinetic energy and potential energy
\begin{align*}
	T &= \frac{\vec{p} \cdot \vec{p}}{2 m_e} & V &= - \frac{Ze^2}{\norm{\vec{r}}} = - \frac{Ze^2}{r}.
\end{align*}
respectively. The appropriate \QM analogues are
\begin{align}
	\Op{T} &= -\frac{\hbar^2}{2 m_e} \Delta & \Op{V} = - \frac{Ze^2}{r}
	\label{eqn:OpHydrogenConventional}
\end{align}
where the \newterm{Laplace operator}
\begin{align}
	\Delta = \sum_{\alpha \in \{x,y,z\}} \frac{\partial^2}{\partial \alpha^2}
	\label{eqn:LaplaceOperatorHydrogen}
\end{align}
was introduced. The full Hamiltonian therefore reads
\[ \Op{H} = \Op{T} + \Op{V} = -\frac{\hbar^2}{2 m_e} \Delta - \frac{Ze^2}{r}. \]

For considering more complicated systems,
it is often convenient to employ so-called \newterm{atomic units}.
These are generated by a unitary transformation of the Hilbert space,
which effectively yields
\[ \hbar \equiv e \equiv m_e \equiv a_0 \equiv E_h \equiv 1. \]
See table \vref{tab:AtomicUnits} for the values of these quantities
in terms of the usual SI units.
The hydrogen-like Hamiltonian \vref{eqn:OpHydrogenConventional}
in atomic units takes the simple form
\begin{equation}
	\Op{H} = -\frac12 \Delta - \frac{Z}{r}
	\label{eqn:OpHydrogen}
\end{equation}
We will from now work exclusively in these units.
\begin{table}
	\centering
	\begin{tabular}{lllll}
		\toprule
		symbol & name & atomic unit of & value in SI units \\
		\midrule
		$\hbar$ & Planck constant & momentum & $\unit[1.055\E{-34}]{Js}$ \\
		$e$ & elementary charge & charge & $\unit[1.602\E{-19}]{C}$ \\
		$m_e$ & electron mass & mass & $\unit[9.110\E{-31}]{kg}$ \\
		$a_0$ & Bohr radius & length & $\unit[5.292\E{-11}]{m}$ \\
		$E_h$ & Hartree energy & energy & $\unit[4.360\E{-18}]{J}$ \\
		\bottomrule
	\end{tabular}
	\caption{Atomic units and their relationship to SI units}
	\label{tab:AtomicUnits}
\end{table}

\section{Elements of functional analysis}
\label{sec:FunAna}

The mathematical field of functional analysis
is concerned with the study of Banach and Hilbert spaces
as well as the properties of mappings between such structures.
In this work we will neglect Banach spaces
and focus on Hilbert spaces only
due to their exceptional importance in the mathematical structure
of quantum mechanics, see the previous section \ref{sec:IntroQM}.
After some general reminders, we will take a closer look
at the Lebesgue space $L^2(\R^d, \C)$
as well as Sobolev spaces in the context of \QM.

In this section we assume familiarity with the concept of a vector space
as well as some intuitive understanding of the Lebesgue integral.
For a more detailed discussion developing such concepts
by generalising standard Euclidean geometry,
see a related work by the author~\cite{DiracNotation}.

\subsection{Hilbert spaces}
\label{sec:Hilbert}
\nomenclature{$\F$}{Either the field of real numbers $\R$ or the field of complex numbers $\C$.}

Hilbert spaces are generalising some concepts of
two- or three-dimensional Euclidean space to
larger vector spaces of possibly infinite dimensions.
Most notably taking limits or computing length and angles
is possible in the same natural way as for Euclidean geometry,
thus allowing to perform vector calculus or
to numerically approximate in a sound way.
For our treatment the field $\F$ can be typically identified
with the field of all complex numbers $\C$ or the real numbers $\R$.

The first ingredients to a Hilbert space are ways to measure
angles and distances, \ie an inner product and a norm.
\begin{defn}
	An \newterm{inner product space} over a field $\F$
	is a vector space $V$ (over the same field)
	that is further equipped with an \newterm{inner product}, i.e. a map
	\[ \braket{\slot}{\slot}_V : V \times V \to \F \]
	that satisfies%
	(for all vectors $x,y,z \in V$ and all $\alpha \in \set{F}$)
	\begin{align}
		\label{eqn:innProdConjSym}
			&\braket{x}{y}_V^\ast = \braket{y}{x}_V &&
			\text{\textit{(conjugate symmetry)}} \\
		\label{eqn:innProdLinRight}
			&\braket{x}{\alpha y + z}_V = \alpha \braket{x}{y}_V + \braket{x}{y}_V &&
			\text{\textit{(linearity in the last argument)}} \\
		\label{eqn:innProdPosDef}
			&\braket{x}{x}_V \ge 0 \quad \text{and}
			\quad \braket{x}{x}_V = 0 ~\Rightarrow~ x = 0  &&
			\text{\textit{(positive-definiteness)}},
	\end{align}
	where the asterisk ``$^\ast$'' denotes complex conjugation.
	We typically drop the ``$V$'' subscript from the notation of the inner
	product if the underlying vector space is clear from context.
\end{defn}

\begin{rem}
	Some literature uses a deviating definition for the inner product,
	where not the second,
	but the first argument in \eqref{eqn:innProdLinRight} is linear,
	\ie where \eqref{eqn:innProdLinRight} would be replaced by
	\[ \braket{\alpha y + z}{x}_V = \alpha \braket{y}{x}_V + \braket{z}{x}_V. \]
	Our definition is in better agreement with the usual convention
	of quantum physics and quantum chemistry
	due to the resemblance of Dirac notation~\cite{DiracNotation}.
\end{rem}

\begin{defn}
	Given a vector space $V$ over the field $\F$, a \newterm{norm} is a map
	$\norm{\slot} : V \to \R$
	such that the following axioms hold for all vectors $x,y \in V$ and all $\alpha \in \set{F}$:
	\begin{align}
		\label{eqn:normScalability}
			&\norm{\alpha x} = \abs{\alpha} \, \norm{x} &&
			\text{\textit{(absolute scalability)}} \\
		\label{eqn:normTriaIneq}
			&\norm{x + y} \le \norm{x} + \norm{y} &&
			\text{\textit{(triangle inequality)}} \\
		\label{eqn:normPointSep}
			&\text{If}~\norm{x} = 0 \quad \Rightarrow
			\quad \text{$x$ is the zero vector} &&
			\text{\textit{(norm separates points)}}
	\end{align}
	If such a norm can be found for a particular vector space $V$,
	one typically refers to $V$ as a \newterm{normed vector space} as well.
\end{defn}

\begin{prop}
	For every inner product space exists the so-called \newterm{induced norm}
	\begin{equation}
		\norm{x}_V = \sqrt{ \braket{x}{x}_V } \qquad \forall x \in V.
		\label{eqn:normInduced}
	\end{equation}
	One may drop the subscript on the norm
	if it is clear from context.
	\begin{proof}
		See~\cite{DiracNotation}.
	\end{proof}
\end{prop}

The second ingredient for a Hilbert space is a property
called \newterm{completeness}. Formally it is defined as such:
\begin{defn}
	\label{def:Completeness}
	A vector space $V$ is called \newterm{complete} if every
	\newterm{Cauchy sequence} of vectors in $V$ has a limit $V$.
\end{defn}
Let us first recall, that a sequence $(x_n) \in V$ is called Cauchy if
\[ \forall \varepsilon > 0 \quad
	\exists M \in \N \quad \text{such that} \quad
	\norm{x_n - x_m}_V < \varepsilon \quad \forall n,m > M.
\]
One can show that every converging sequence is Cauchy.
A roughly equivalent way of phrasing definition \vref{def:Completeness}
is therefore,
that a space $V$ is complete iff every converging sequence
of elements from $V$ has a limit which is from $V$ \emph{as well}.

\begin{exmp}
	\label{exmp:QdenseR}
	To make the concept of completeness more clear,
	let us consider a counterexample.
	For this let us leave the setting of vector spaces for a second
	and more broadly think about sequences defined on sets of numbers%
	\footnote{This is fine, since completeness is in fact a property on
		so-called metric spaces, which are related to normed vector
		spaces, but have much less structure.},
	where the concept of completeness applies as well.

	\noindent
	It is well known that the sequence
	\[ x_n = \sum_{k=0}^n \frac{1}{k!} \in \set{Q} \]
	converges to Euler's number, \ie
	\[ \lim_{n\to\infty} x_n = e \not\in \set{Q}. \]
	In other words $\set{Q}$ is not complete.

	One may, however, build the \newterm{completion} of $\set{Q}$
	by just including all limiting points of all sequences in $\set{Q}$.
	In fact this is one way of defining the set of real numbers $\R$.
\end{exmp}

\begin{rem}
	A subtle point about completeness is that it depends
	on the norm which is used to determine whether a sequence
	is Cauchy or not.
	In other words a vector space may be complete with respect to one norm,
	but not with respect to another.
	Similarly the completion of a space with respect to two different
	norms may yield different spaces.

	In practice the choice of the norm is only of importance for
	infinite-dimensional vector spaces,
	since for finite-dimensional real or complex vector spaces
	all norms are equivalent\footnote{That is they induce the same topology.}
	anyway.
\end{rem}

\noindent
Finally we can state
\begin{defn}
	A \newterm{Hilbert space} $\hilbert$ is an inner product space,
	which is complete with respect to the induced norm.
\end{defn}
In other words a Hilbert space is a space,
where the inner product naturally defines a way to measure distances
and take limits, that is perform calculus.
Thinking ahead towards the integral and differential operators
we will define on such Hilbert spaces,
this is exactly what we will need.

Before we look into some Hilbert spaces relevant for \QM,
let us first clarify the concept of \newterm{denseness} and \newterm{separability}.
\begin{defn}
	A subspace $S$ of a vector space $V$ is called \newterm{dense in $V$}
	if each vector $x \in V$ either is a member of $S$ or one may find
	a Cauchy sequence in $S$ for which $x$ is the limit point.
\end{defn}
In other words $S$ is dense in $V$ if we can --- for each element of $V$
--- construct a sequence of approximations inside the smaller space $S$,
representing the desired element up to arbitrary accuracy.
Denseness is therefore one of the fundamental properties required for approximation.

\begin{exmp}
	Returning to example \vref{exmp:QdenseR} we note,
	that $\set{Q}$ is dense in $\R$.
	This guarantees that we may approximate any real number
	up to arbitrary accuracy by an appropriate sum of fractions,
	which is one of the assumptions behind any floating point operation
	performed on the computer.
\end{exmp}

\begin{defn}
	A Hilbert space is \newterm{separable}%
	\footnote{In the broader context of metric spaces,
	a separable space has a countable, dense subset.}
	iff it admits a countable orthonormal basis.
\end{defn}

\begin{rem}
	\label{rem:HilbertCd}
	If a Hilbert space is separable we can find a basis set
	$\{ \varphi_\mu \}_{0 \leq \mu \leq \Ibas}$
	of at most countably infinite cardinality,
	\ie where $\Ibas \subseteq \N$.
	With this we can write for each $\Psi \in \hilbert$:
	\begin{equation}
		\Psi = \sum_{\mu \in \Ibas} c_\mu \varphi_\mu.
		\label{eqn:HilbertBasisExpansion}
	\end{equation}
	This in turn uniquely identifies each $\Psi$ with a sequence
	$(c_\mu)_{\mu\in\N}$ of complex numbers.
	By this means each complex, separable Hilbert space is
	isomorphic to the space of complex-valued, square-summable sequences $l^2(\N, \C)$.
	One can easily show that this isomorphism is even an isometry, \ie
	\[ \norm{\Psi}_\hilbert = \norm{(c_\mu)}_{l^2}
		= \sqrt{ \sum_{\mu=0}^\infty \abs{c_\mu}^2 }. \]
	By transitivity all separable Hilbert spaces are isometrically isomorphic.

	In our remaining discussion we will only encounter complex,
	separable Hilbert spaces. This implies:
	\begin{itemize}
		\item If $\Psi \in \hilbert$ is a vector in a Hilbert space,
			we can always identify it with a
			(possibly infinite) column vector of complex coefficients.
		\item Finite-dimensional Hilbert spaces are isomorphic to $\C^d$,
			where $d$ is the dimensionality.
			Their vectors are thus identified by a column of complex numbers
			of finite size.
	\end{itemize}
\end{rem}

\begin{rem}
	\label{rem:HilbertApproximation}
A consequence of remark \vref{rem:HilbertCd} is that we can
numerically approximate all separable Hilbert spaces rather naturally.
For example by restricting the sum in \eqref{eqn:HilbertBasisExpansion}
to only a finite number of $d$ basis functions,
we can make sure that the resulting $\Psi$ is located
in only a $d$-dimensional subspace $\hilbert^{(d)} \subset \hilbert$.
Moreover this subspace is dense, since in the limit of taking
all basis functions, we get exactly $\hilbert$.
In turn since $\hilbert^{(d)}$ is finite-dimensional,
we can identify each approximation to $\Psi$ with a vector in $\C^d$,
which can be represented numerically on the computer,
regardless of the structure of $\hilbert$.
\end{rem}

\subsection{The Hilbert spaces of Quantum Mechanics}
\nomenclature{$L^2(\R^d, \C)$}{The Hilbert space of square-integrable complex-valued functions, see \vref{prop:L2HilbertSpace}}

Now that we have the required jargon at hand,
let us discuss the question which Hilbert space to take for quantum mechanics.
In section \ref{sec:IntroQM} we said that the state functions
$\Psi : \R^d \to \C$ are taken from a complex, separable Hilbert space.
In our treatment we adhere to the
\newterm{Copenhagen interpretation} or \newterm{Born interpretation}
of the quantum-mechanical state $\Psi$,
which associates the meaning of a probability density
with the magnitude $\abs{\Psi(x_1, x_2, \ldots x_d)}$
of the state function at each point in space $(x_1, \ldots x_N)$.
A more detailed analysis in light of this probabilistic meaning of $\Psi$
suggests to take these functions from the Hilbert space of square-integrable functions
$L^2(\R^d, \C)$, defined as such:

\begin{prop}
	\label{prop:L2HilbertSpace}
	Given two functions $f, g : \R^d \to \C$,
	we can define an inner product
	\begin{equation}
		\braket{f}{g}_{L^2} := \int_{\R^d} \cc{f}(\vec{x}) g(\vec{x}) \D \vec{x}
		\label{eqn:defL2InnProd}
	\end{equation}
	and the corresponding induced norm
	\begin{equation}
		\norm{f}_{L^2} := \sqrt{\braket{f}{f}_{L^2}} = \left( \int_{\R^d} \abs{f(\vec{x})}^2 \D \vec{x} \right)^{1/2},
		\label{eqn:defL2Norm}
	\end{equation}
	where the integral --- in both cases --- is to be understood in the Lebesgue sense
	and we identified
	\[ \vec{x} \equiv (x_1, \ldots, x_d) \]
	for ease of notation. The set%
	\footnote{More precisely $L^2$ is constructed as a quotient group
		over the equivalence relation of functions being identical
		\textit{almost everywhere}. In this sense functions are considered
		to be the same iff they differ at most on a set of Lebesgue measure zero.
		See \cite{DiracNotation} for details.}
	\[ L^2(\R^d, \C) := \big\{ f : \R^d \to \C ~\big|~ \norm{f}_{L^2} < \infty \big\}, \]
	where the norm $\norm{\slot}_{L^2}$ stays finite
	is the set of \newterm{square-integrable functions}.
	It forms a Hilbert space over the field $\C$.
	\begin{proof}
		See~\cite{Adams2003}
	\end{proof}
\end{prop}

\noindent
Luckily one can show~\todo{Find reference}
\begin{prop}
	$L^2(\R^d, \C)$ is separable.
\end{prop}
\noindent
such that $L^2(\R^d, \C)$ truely satisfies all the requirements
for being a suitable Hilbert space of \QM
as introduced in section \vref{sec:IntroQM}.

In section \ref{sec:IntroQM} we did not introduce the most general theory of \QM.
More precisely taking states from $L^2(\R^d, \C)$
is only correct for \emph{non-relativistic, spin-free} \QM.
If more than one spin component are required,
like $4$ in relativistic \QM or $2$ in spin-adapted \QM,
we need to take states from the spaces $L^2(\R^d, \C^4)$ or $L^2(\R^d, \C^2)$,
respectively.
This work does not treat relativistic \QM at all,
such that only $L^2(\R^d, \C)$ and $L^2(\R^d, \C^2)$ are relevant for us.
For simplicity we will assume $\hilbert = L^2(\R^d, \C)$
in the mathematical analysis presented in this and the next chapter
and only introduce spin \textit{ad hoc} in the form of $L^2(\R^d, \C^2)$
whenever this is needed in the remaining chapters.
This is sound because of remark \vref{rem:HilbertCd},
where we pointed out that all infinite-dimensional
Hilbert spaces are isometrically isomorphic.
This implies that all of the properties we have showed or will show based
on the Hilbert space $L^2(\R^d, \C)$ can be generalised
to $L^2(\R^d, \C^s)$ with $s \geq 1$ rather trivially.

\subsection{Sobolev spaces}
\label{sec:Sobolev}
Many operators of quantum mechanics including all the operators,
which we will discuss in detail, involve taking derivatives of states.
Whilst Lebesgue spaces are suitable for doing statistics,
their mathematical structure does not make sure
that derivatives of functions from $L^2(\R^d, \C)$ stay in $L^2(\R^d, \C)$.
For example $1/r$ is square-integrable on $\R^3$,
whilst its radial derivative $-1/r^2$ is not.
One way to tackle this, is to take an appropriate subspace of $L^2(\R^d, \C)$,
which allows taking a certain number of derivatives.
As it turns out the appropriate treatment for the numerical solution
of partial differential equations,
does not require the usual or \newterm{strong derivatives},
much rather weak derivatives are sufficient.
These are defined as such:

\begin{defn}
	A function $f \in L^2(\R^d, \C)$ has a \newterm{weak partial derivative}
	$g \in L^2(\R^d, \C)$ with respect to $x_i$ if
	\[ \forall \varphi \in C^\infty_0(\R^d): \quad \braket{g}{\varphi}_{L^2} = - \braket{f}{\frac{\partial}{\partial x_i}\varphi}_{L^2}, \]
	where $C^\infty_0(\R^d, \C)$ is the space of all
	infinitely differentiable complex-valued functions with compact support.
	To denote the weak derivative
	one may write $g = \frac{\partial}{\partial x_i} f$ like in the strong case.
	It can further be shown that if $f$ has a classical, strong derivative
	then it also has a weak derivative, which coincides with the strong derivative.
	For ease of notation we also write
	\[ D^{\vec{\alpha}} f = \frac{\partial^{\norm{\vec{\alpha}}_1}}{ \prod_{i=1}^d \partial x_i^{\alpha_i} }, \]
	where $\vec{\alpha} \in \N^d$ and as usual
	\[ \norm{\vec{\alpha}}_1 = \sum_{i=1}^d \abs{\alpha_i} \]
\end{defn}

\noindent
With the weak derivative at hand we can define the so-called Sobolev spaces,
which allow to make certain guarantees about the number of (weak)
derivatives, which can be taken.
A full family of such spaces exist. We will only present two kinds here.

\nomenclature{$H^1(\R^d, \C)$}{The Hilbert space of complex-valued functions with square-integrable first derivative}
\nomenclature{$H^2(\R^d, \C)$}{The Hilbert space of complex-valued functions with square-integrable second derivative}
\begin{defn}
	The \newterm{Sobolev space} defined by
	\begin{equation}
		H^k(\R^d, \C) := \left\{ f \in L^2(\R^d, \C) ~\middle|~ D^{\vec{\alpha}} f \in L^2(\R^d, \C) \text{ for $\norm{\vec{\alpha}}_1 \le k$} \right\}
		\label{eqn:defSobolev}
	\end{equation}
	with inner product
	\begin{equation}
		\braket{f}{g}_{H^k} := \sum_{\norm{\vec{\alpha}}_1 \le k} \braket{D^{\vec{\alpha}} f\,}{\,D^{\vec{\alpha}} g}_{L^2}
		\label{eqn:defSobolevInnProd}
	\end{equation}
	and induced norm
	\begin{equation}
		\norm{f}_{H^k} = \sum_{\norm{\vec{\alpha}}_1 \le k} \norm{D^{\vec{\alpha}} f}_{L^2}
		\label{eqn:defSobolevNorm}
	\end{equation}
	is a Hilbert space~\cite{Adams2003}.
\end{defn}

\begin{defn}
	The completion of $C^\infty_0(\R^d, \C)$
	with respect to the norm $\norm{\slot}_{H^k}$
	is the Sobolev space $H^k_0(\R^d, \C)$.
	It is a proper subspace of $H^k$ and a Hilbert space as well~\cite{Adams2003}.
\end{defn}
Colloquially speaking if a function is a member of $H^k(\R^d, \C)$ or $H^k_0(\R^d, \C)$,
we can assume that the $k$-th derivative of this function remains square-integrable.
These spaces will become rather important in the next section \vref{sec:spectral},
where we will need them to define self-adjoint operators upon them.
As a summary the relationships between the spaces we discussed
in this section have been summarised in \fig \vref{fig:sobolevRelations}.
Note, that by definition
\[ L^2(\R^d, \C) = H^0(\R^d, \C) = H^0_0(\R^d, \C). \]

\begin{figure}
	\centering
	\includeimage{1_qm/sobolev}
	\caption{Overview of the spaces discussed in this section.
		Apart from $C^\infty_0(\R^d, \C)$ all mentioned spaces are Hilbert spaces.
		In each case $A \subset B$ denotes that $A$ is a proper, dense
		subspace of $B$.}
	\label{fig:sobolevRelations}
\end{figure}

To finish our discussion of Sobolev spaces let us determine
in which Sobolev space the function
\begin{equation}
	\label{eqn:H1sFunctionalForm}
	\Psi_{1s}(\vec{r}) = \exp\left(- \sqrt{x^2 + y^2 + z^2} \right) = \exp(-r)
\end{equation}
is located.
This function and trivial generalisations thereof
will be of relevance for our future treatment,
since it arises naturally as an eigenfunction of the hydrogen-like
Hamiltonian \eqref{eqn:OpHydrogen}~(see section \vref{sec:HydrogenAtom})
and is furthermore an important building block of the Coulomb-Sturmians%
~(see section \vref{sec:BasisCS}).

\begin{exmp}
	\label{exmp:H1sH1}
	The function $\Psi_{1s}$ of \eqref{eqn:H1sFunctionalForm} belongs to $H^1(\R^3, \C)$.
	\begin{proof}
	Since the function is bounded~\footnote{Does not go to infinity anywhere},
	we clearly have $\Psi_{1s} \in L^2(\R^3, \C)$.
	Furthermore for any $\alpha \in \{x, y, z\}$:
	\begin{equation}
		\norm{ \frac{\partial \Psi_{1s}}{\partial \alpha} }_{L^2}
		= \norm{ - \frac{\alpha}{r} \exp(-r) }_{L^2}
		= \int_{\R^3} \frac{\alpha^2}{r^2} \exp(-2r) \D\vec{r}
		\leq \int_{\R^3} \frac{r^2}{r^2} \exp(-2r) \D\vec{r}
		\label{eqn:ProofH1sH1}
	\end{equation}
	Due to the properties of the Lebesgue integral,
	we may ignore the removable discontinuity at $\vec{r} = \vec{0}$
	and instead write
	\[ \norm{ \frac{\partial \Psi_{1s}}{\partial \alpha} }_{L^2}
		\leq \int_{\R^3} \exp(-2r) \D\vec{r} = \norm{\Psi_{1s}}_{L^2} < \infty. \]
	This shows that $\exp(-r) \in H^1(\R^3, \C)$,
	since each term of \eqref{eqn:defSobolevNorm} is bound.
	\end{proof}
\end{exmp}

For the next step, showing $\Psi_{1s} \in H^2(\R^3, \C)$,
we need two results relating $H^1(\R^3, \C)$ and $L^2(\R^3, \C)$.
\begin{prop}[Hardy's inequality]
	\label{prop:Hardy}
	For all $u \in H^1(\R^3, \C)$, we have
	\[ \int_{\R^3} \norm{\nabla u}_2^2 \D\vec{r}
		\geq \frac{1}{4} \int_{\R^3} \frac{\abs{u}^2}{r^2} \D\vec{r} \]
	\begin{proof}
		For a proof of the special case $u \in C_0^\infty(\R^3, \C)$
		see \cite[p. 30]{Helffer2013}.
		The more general case we claim here,
		follows from the denseness of $C_0^\infty(\R^3, \C)$ in $H^1(\R^3, \C)$
		and continuity of the integrands on both sides
		with respect to the $H^1$ norm.
	\end{proof}
\end{prop}

\begin{cor}
	\label{cor:Hardy}
	If $u \in H^1(\R^3, \C)$, then $\frac{u}{r} \in L^2(\R^3, \C)$.
	\begin{proof}
		One easily rewrites Hardy's inequality to
		\[
			\norm{u}_{H^1} \geq  \sum_{\alpha\in\{x,y,z\}} \int_{\R^3} \abs{\frac{\partial u}{\partial \alpha}} \D\vec{r}
			\stackrel{\text{(triangle)}}{\geq}
			\int_{\R^3} \norm{\nabla u}_2^2 \D\vec{r}
			\stackrel{\text{(Hardy)}}{\geq}
			\frac{1}{4} \int_{\R^3} \frac{\abs{u}^2}{r^2} \D\vec{r}
			= \frac{1}{4} \norm{\frac{u}{r}}_{L^2}
		\]
		which proves the claim.
	\end{proof}
\end{cor}

\begin{exmp}
	\label{exmp:H1sH2}
	We now want to use corollary \vref{cor:Hardy} to prove that
	$\Psi_{1s} \in H^2(\R^3,\C)$.
	\begin{proof}
		Considering our result from \eqref{eqn:ProofH1sH1}
		we find that for all $\alpha, \beta \in \{x,y,z\}$:
		\begin{align*}
			\norm{\frac{\partial^2 \Psi_{1s}}{\partial \alpha \partial \beta}}_{L^2}
			&\leq \norm{ \frac{\delta_{\alpha\beta}}{r} \exp(-r)}_{L^2}
			+ \norm{\frac{\alpha\beta}{r^3}\exp(-r)}_{L^2}
			+ \norm{\frac{\alpha\beta}{r^2}\exp(-r)}_{L^2}
		\end{align*}
		Noting $\abs{\alpha\beta} \le r^2$
		and ignoring the removable singularities in the Lebesgue integral,
		we arrive at
		\begin{align*}
			\norm{\frac{\partial^2 \Psi_{1s}}{\partial \alpha \partial \beta}}_{L^2}
			&\leq \norm{\frac{1}{r} \exp(-r)}_{L^2}
			+ \norm{\frac{r^2}{r^3}\exp(-r)}_{L^2}
			+ \norm{\frac{r^2}{r^2}\exp(-r)}_{L^2} \\
			&= 2\norm{\frac{1}{r} \exp(-r)}_{L^2} + \norm{\exp(-r)}_{L^2} \\
			&< \infty,
		\end{align*}
		where in the last line we used
		$\exp(-r) \in H^1(\R^3, \C)$ and corrollary \vref{cor:Hardy}.
\end{proof}
\end{exmp}

\begin{rem}
	Analogously to what we sketched in examples \vref{exmp:H1sH1} and \vref{exmp:H1sH2},
	one could attempt to probe whether the one-dimensional function
	$f(x) = \exp(-\abs{x})$ lives in $H^1(\R,\C)$ or $H^2(\R,\C)$.
	Whilst the former can be easily verified,
	one finds $f \not\in H^2(\R,\C)$.
\end{rem}

This rather surprising result is a consequence of the second part of
the Sobolev embedding theorem
of which we only present a slightly specialised form here.
\begin{thm}[Sobolev embedding]
	Given $r, k, d \in \N$ with
	\[ k > \frac{d}{2} > 0 \quad \text{and} \quad k -\frac{d}{2} > r \]
	one may find an embedding
	\[ H^k(\R^d) \subset C^r(\R^d) \]
	between the Sobolev space $H^k(\R^d)$ and the space of the $r$ times
	continuously differentiable functions, $C^r(\R^d)$.
\end{thm}
This embedding theorem allows to get an idea what is to be expected about
the smoothness of a function in $H^k$.
Interestingly the smaller the dimensionality the more smooth such a function
has to be.

\subsection{Representation theorems}
\todoil{If there is time talk about the Ritz and the Lax-Milgram}

\begin{thm}[Riesz's representation theorem]
	\label{thm:Riesz}
	bla
	\begin{proof}
		\todoil{Find reference}
	\end{proof}
\end{thm}

\begin{lem}[Lax-Milgram]
	\label{thm:LaxMilgram}
	bla
\end{lem}

\section{Spectral theory}
\label{sec:spectral}

In this chapter we will broaden our discussion focusing
on linear operators between the state functions of a Hilbert space.
We will discuss certain common classes of operators
including self-adjoint and compact operators as well as their spectral properties.
We will see that most operators, including the ones required for
atomic physics and quantum chemistry,
do not show all the nice properties we would like to rely on.
For example one might not be able to find eigenfunctions
for all eigenvalues and the ones one is able to determine
might not amount to span the Hilbert space completely.
For this reason we will hint at techniques
relevant to the Hilbert space $L^2(\R^d, \C)$
and a few of the relevant operators of \QM,
which will allow us to recover at least part of the eigenspectrum
with the numerical methods discussed in chapter \vref{ch:numeigen}.

\subsection{Bounded and self-adjoint operators}

Mathematically a linear operator is defined as such:
\begin{defn}
	A \newterm{linear operator} on a Hilbert space $\hilbert$
	is the linear map $\Op{A} : D(\Op{A}) \to \hilbert$,
	where $D(\Op{A}) \subset \hilbert$ is a subspace
	called the \newterm{domain} of $\Op{A}$.
\end{defn}
Typically we employ just the term \newterm{operator} to refer to linear operators.
Recall that a mapping is called \newterm{linear} if
for all $u, v \in \hilbert$ and all $\alpha \in \C$
\begin{align}
	\label{eqn:OperatorLinear}
	\Op{A}\left(u + v\right) &= \Op{A}u + \Op{A}v &
	\Op{A}\left(\alpha u\right) &= \alpha \Op{A}u
\end{align}
hold.
Even though not strictly necessary, we will assume for our treatment
that the Hilbert space is separable and that the domain of an operator
is always dense in it.

\newcommand{\opnorm}[1]{\norm{#1}_{\mathcal{L}(\hilbert)}}
\begin{prop}
	The inner product of $\hilbert$ induces the so-called operator norm
	\[ \opnorm{\Op{A}} := \sup_{\substack{u \in D(\Op{A}),\\ u\neq0}}
		\frac{\norm{A u}_\hilbert}{\norm{u}_{\hilbert}}. \]
	\begin{proof}
		See \cite[Satz II.1.4]{Werner2011}
	\end{proof}
\end{prop}

The first important classification we will discuss here
is the notion of bounded and unbounded operators.
\begin{defn}
	\label{defn:OpBounded}
	An operator $\Op{A}$ on $\hilbert$ is \newterm{bounded} iff
	\[ \opnorm{\Op{A}} < \infty, \]
	\ie if it has finite operator norm.
	A bounded operator is referred to as \newterm{continuous}%
	\footnote{In fact this is a consequence from the fact that a bounded linear operator
	between normed vector spaces is always continuous.} as well.
\end{defn}
In our example of $\hilbert = L^2(\R^d, \C)$ an operator is hence bounded
if its action on a square-integrable function yields another function,
which stays square-integrable.
In the introductory paragraph of section \vref{sec:Sobolev}
we already noted that the radial derivative of
the square-integrable function $1/r$, namely $-1/r^2$,
is not square-integrable.
It should therefore not come as a surprise that
operators containing derivatives
--- like the kinetic energy operator of \QM ---
are not bounded in general.
Instead they are typically unbounded operators,
which we will define in the following.

\begin{defn}
	Let $\Op{A}$ and $\Op{B}$ be operators on $\hilbert$.
	$\Op{B}$ is an \newterm{extension} of $\Op{A}$ if
	\mbox{$D(\Op{A}) \subset D(\Op{B})$}
	and if $\forall u \in D(\Op{A}): \ \Op{A}u = \Op{B}u$.
\end{defn}

\begin{defn}
	An operator $\Op{A}$ on $\hilbert$ which does not possess
	a bounded extension is called an \newterm{unbounded operator} on $\hilbert$.
\end{defn}
% TODO OPTIONAL
%\to doil{Think how this can be possible, e.g. that the operator by itself is not bounded,
%	but the extension is}

There also exists a middle ground,
namely so-called \newterm{semi-bounded operator}s, defined as such:
\begin{defn}
	\label{defn:SemiBounded}
	An operator $\Op{A}$ on $\hilbert$ with domain $D(\Op{A})$
	is called semi-bounded from below if there exists a constant $C$
	such that for all $u \in D(\Op{A})$:
	\[ \braket{u}{\Op{A} u} = \braket{\Op{A} u}{u} \quad \text{and}
	\quad \braket{u}{\Op{A} u} \geq -C \braket{u}{u}. \]
\end{defn}

Starting from definition \ref{defn:OpBounded}
it is easy to show that a bounded operator $\Op{A}$ on a Hilbert
space $\hilbert$ maps Cauchy sequences to Cauchy sequences,
\ie if $(x_n) \in \hilbert$ is Cauchy, so is $(\Op{A} x_n)$.
In this sense a somewhat stronger version of boundedness
is compactness, defined as:
\begin{defn}
	An operator $\Op{A} : D(\Op{A}) \to \hilbert$ on a Hilbert space $\hilbert$
	is \newterm{compact}
	if for any sequence $(x_n)$ that converges weakly in $D(\Op{A})$,
	$\Op{A} x_n$ converges strongly in $\hilbert$.

	Recall that a sequence $(x_n)$ is called \newterm{weakly convergent}
	if for all $\phi \in \hilbert$ the sequence $(y_n)$ with
	$y_n = \braket{x_n}{\phi}_\hilbert$
	is \newterm{strongly convergent}, \ie Cauchy.
\end{defn}
Compactness is of importance for us in the context of spectral theory,
since compact operators have particularly nice spectral properties.
As expected one may easily show, that~\cite{Werner2011}
\begin{prop}
	A compact operator $\Op{A}$ defined on a Hilbert space is bounded as well.
\end{prop}

\begin{rem}
\label{rem:OpRietz}
Each operator $\Op{A}$ on a Hilbert space $\hilbert$ can be
uniquely identified with a sesquilinear form
$a : \hilbert \times \hilbert \to \C$, defined by
\begin{equation}
	\hilbert \times \hilbert \ni (u, v) \mapsto a(u, v) := \braket{u}{\Op{A} v}_\hilbert \in \C.
	\label{eqn:formA}
\end{equation}
This is a consequence of the Riesz representation theorem \cite{DiracNotation}.
%% TODO OPTIONAL \vref{thm:Riesz}.

In many applications, including the numerical treatment discussed in chapter
\vref{ch:numeigen},
the sesquilinear form $a$ is more intuitive to employ than the operator $\Op{A}$ itself.
\end{rem}

Using the identification of the previous remark,
we may define the terms symmetric and self-adjoint.
\begin{defn}
	\label{def:SymmetricOperator}
	An operator $\Op{A}$ on $\hilbert$ is called \newterm{symmetric}
	if
	\[ \forall (u, v) \in D(\Op{A}) \times D(\Op{A}): \qquad \braket{\Op{A} u}{v} = \braket{u}{\Op{A} v} \]
\end{defn}
In Physics textbooks a symmetric operator is usually called Hermitian.

\begin{defn}
	Let $\Op{A}$ be a linear operator on $\hilbert$ with (dense) domain $D(\Op{A})$
	and let $D(\tp{\Op{A}})$ be the space
	\[ D(\tp{\Op{A}}) := \left\{ v \in \hilbert \, \middle| \,
		\exists f_v \in \hilbert \ \text{such that} \
		\forall u \in D(\Op{A}): \braket{\Op{A} u}{v} = \braket{u}{f_v} \right\},
	\]
	where for each $v$ the $f_v$ is unique due to the denseness
	of $D(\Op{A})$ in $\hilbert$ and the Riesz representation theorem.

	Then the \newterm{adjoint} of $\Op{A}$
	is the linear operator $\tp{\Op{A}}$ with domain $D(\tp{\Op{A}})$
	defined by
	\[ \forall v \in D(\tp{\Op{A}}) \quad \braket{\Op{A} u}{v} = \braket{u}{\tp{\Op{A}} v} \]
\end{defn}

\begin{defn}
	A self-adjoint operator is an operator $\Op{A}$ for which $\tp{\Op{A}} = \Op{A}$,
	or equivalently an operator which is symmetric \emph{and} where
	$D(\Op{A}) = D(\tp{\Op{A}})$.
\end{defn}

\begin{rem}
	For a \emph{bounded} linear operator $\Op{A}$ with $D(\Op{A}) = \hilbert$
	one can find%
	\footnote{Since $D(\Op{A})$ in our treatment is dense in $\hilbert$,
	one can always find a unique, bounded extension of $\Op{A}$ with complete
	domain $\hilbert$ if $\Op{A}$ if bounded}
	a definition for the adjoint,
	which is more usual in the literature of quantum physics.

	\noindent
	Namely by means of the identification
	\[ \forall (u,v) \in \hilbert \times \hilbert \quad \braket{\Op{A} u}{v} = \braket{u}{\tp{\Op{A}} v} \]
	one can find a unique adjoint $\tp{\Op{A}}$ for each bounded operator $\Op{A}$.
	This operator will be bounded, too%
	\footnote{Note, that this makes the set of bounded operators on $\hilbert$
	a so-called $C^\ast$ algebra.}.

	Comparing with the definition of a symmetric operator we find that for
	bounded operators the property of symmetric and self-adjoint are equivalent.
\end{rem}

\begin{rem}
	Even though symmetric and self-adjoint are related concepts,
	symmetric operators are not very useful in practice.
	Only self-adjoint operators have the nice mathematical properties,
	we require for quantum mechanics, namely a real spectrum
	and a spectral decomposition into bound and continuous states.
	See the next section for details.

	Most operators in \QM are not self-adjoint albeit being symmetric
	if defined in a na\"{i}ve way.
	In many cases this issue can be circumvented
	by choosing an appropriate operator extension.
	We will discuss this 
	is section \vref{sec:SpectrumLaplace} and \vref{sec:HydrogenAtom}
	considering the spectrum of the Laplace operator
	\eqref{eqn:LaplaceOperatorHydrogen}
	and the hydrogen-like operator \eqref{eqn:OpHydrogen}.
\end{rem}

\begin{rem}
	As a summary of the terms introduced in this section,
	we note the following implications for an operator $\Op{A}$
	on a Hilbert space $\hilbert$.
	\begin{itemize}
		\item $\Op{A}$ compact
			$\Rightarrow$ $\Op{A}$ bounded $\Rightarrow$ $\Op{A}$ semi-bounded
		\item $\Op{A}$ self-adjoint $\Rightarrow$ $\Op{A}$ symmetric
		\item If $\Op{A}$ bounded: $\Op{A}$ self-adjoint $\Leftrightarrow$
			$\Op{A}$ symmetric.
	\end{itemize}
\end{rem}

\subsection{Spectra of self-adjoint operators}
In this section we will clearify the notion of a spectrum
for a self-adjoint operator in infinite dimensions
and the connections to the probably more familiar concepts
of eigenvalues and eigenvectors in finite dimensions.

\newcommand{\shiftop}{(\Op{A} - \lambda \id_{\hilbert})}
\nomenclature{$\id_{\hilbert}$}{Identity operator on the Hilbert space $\hilbert$.}
\begin{defn}
	\label{defn:Spectrum}
	Let $\Op{A}$ be a self-adjoint%
	\footnote{Strictly speaking the operator only needs to be closed for this definition.
	An operator is closed if its graph $\{ (u, \Op{A}u) \, | \, u \in D(\Op{A}) \}$
	is a closed subspace of $\hilbert \times \hilbert$.
	This is true for all self-adjoint operators.}
	linear operator on $\Op{H}$.
	\begin{itemize}
		\item We call the open set%
			\footnote{$\id_{\hilbert}$ is the identity operator on the Hilbert space $\hilbert$.}
			\[ \rho(\Op{A}) = \left\{ \lambda \in \C \ \middle| \
				\shiftop \
				\text{is bijective on $D(\Op{A})$} \right\} \]
			the \newterm{resolvent set} of $\Op{A}$.
		\item The closed set $\sigma(\Op{A}) = \C \, \backslash \, \rho(\Op{A})$
			is then the \newterm{spectrum} of $\Op{A}$.
	\end{itemize}
\end{defn}

\noindent
We can further show:~\cite[p.~102]{Helffer2013}
\begin{prop}
	If $\Op{A}$ is self-adjoint, then $\sigma(\Op{A}) \subset \R$.
\end{prop}

Another way of phrasing definition \vref{defn:Spectrum} is that
the spectrum is the set of all points where $\shiftop$ is not bijective.
This implies that both points where $\shiftop$ is not injective
as well as points where $\shiftop$ is not surjective are part of the spectrum.

Recall that an eigenpair $(\lambda, v) \in \C \times \hilbert$ satisfies
\[ \Op{A} v - \lambda v = 0 \quad \Leftrightarrow \quad v \in \ker \shiftop \quad \Rightarrow \quad \ker \shiftop \neq \{0\}, \]
since $v \neq 0$.
Since only non-injective operators can have a non-trivial kernel,
this implies that $\shiftop$ is necessarily non-injective
for $(\lambda, v)$ to be an eigenpair.
Unlike in finite dimensions%
\footnote{In finite dimensions one can always find an operator extension for any
injective operator to be surjective as well.}
it may well happen in infinite dimensions, that $\shiftop$ is injective,
but not surjective.
Therefore one may find points in the spectrum, which are not eigenvalues.
This is expressed more formally in the next definition.

\todoil{In this and the next decomposition, how does this relate to resonance states?}
\begin{defn}
	If $\Op{A}$ is self-adjoint, we can decompose
	$\sigma(\Op{A}) = \sigma_P(\Op{A}) \, \dot{\cup} \, \sigma_C(\Op{A})$
	with
	\begin{itemize}
		\item the \newterm{point spectrum}
			\[ \sigma_P(\Op{A}) = \left\{ \lambda \in \R \, \middle|
				\, \shiftop\ \text{is non-injective} \right\}
				= \{ \text{eigenvalues of $\Op{A}$} \}.\]
		\item the \newterm{continuous spectrum}%
			\footnote{$\overline{A}$ denotes the closure of the set $A$.}
			\[ \sigma_C(\Op{A}) = \overline{\left\{ \lambda \in \R \, \middle|\, \shiftop\ \text{injective, but not surjective} \right\}}.\]
	\end{itemize}
\end{defn}
This definition can be understood physically by the so-called
RAGE\footnote{Named after Ruelle, Amrein, Georgescu and Enss, who all worked on it.}
theorem~\cite{Last1996}.
It draws
a connection between the point spectrum $\sigma_P(\Op{A})$
and the so-called \newterm{bound state}s of an operator
and between the continuous spectrum $\sigma_C(\Op{A})$
and the so-called \newterm{scattering state}s.
Bound states are characterised by the property,
that they have --- at all times --- a non-vanishing function value
only in a finite region of space.
Scattering states on the other hand are \emph{not} eigenstates
and they will vanish from any arbitrarily large, bounded part of space
if enough time has passed.

The decomposition of the spectrum into point and continuous spectrum
is not the only possibility.
Especially from the point of view of numerical modelling
the following, an alternative approach is more helpful
as we shall see later.

\begin{defn}
	For any self-adjoint%
	\footnote{Again only closed is strictly required.}
	operator,
	we can decompose \linebreak
	$\sigma(\Op{A}) = \sigma_\text{disc}(\Op{A}) \, \dot{\cup} \, \sigma_\text{ess}(\Op{A})$
	with
	\begin{itemize}
		\item the \newterm{discrete spectrum}%
			\footnote{For a mathematically more precise description, see~\cite[p.~103 and p.~132]{Helffer2013}.}
			\[ \sigma_\text{disc}(\Op{A}) \simeq \left\{ \lambda \in \R \, \middle|
			\, \text{$\lambda$ is an isolated eigenvalue of $\Op{A}$
			with finite multiplicity} \right\}, \]
		\item the \newterm{essential spectrum}
			\[\sigma_\text{ess}(\Op{A}) 
			= \sigma(\Op{A}) \, \backslash \, \sigma_\text{disc}(\Op{A}). \]
	\end{itemize}
\end{defn}

\noindent
By construction the essential spectrum consists of
\begin{itemize}
	\item the continuous spectrum,
	\item eigenvalues of infinite multiplicity,
	\item eigenvalues embedded inside the continuous spectrum.
\end{itemize}
% TODO OPTIONAL \to doil{Figure if time is left.}
It will become clear in a moment
why only modelling the discrete spectrum can be done easily by
approximative, numerical methods.
For this we need to discuss the special case of
compact, self-adjoint operators in more detail.
If an operator is compact,
its spectrum has a particularly simple form.
\begin{prop}
	\label{prop:CompactSpectrum}
	If $\Op{A}$ is a compact operator on the Hilbert space $\hilbert$:
	\begin{itemize}
		\item $0 \in \sigma(\Op{A})$
		\item $\sigma(\Op{A}) \, \backslash \, \{0\}
			= \sigma_P(\Op{A}) \, \backslash \, \{0\}$
		\item Only one of these cases is true:
			\begin{itemize}
				\item[\textopenbullet] $\sigma(\Op{A}) = \{0\}$
				\item[\textopenbullet] $\sigma(\Op{A}) \, \backslash \, \{0\}$ is finite.
				\item[\textopenbullet]
					$\sigma(\Op{A}) \, \backslash \, \{0\}$ can be described
					as a sequence of points tending to $0$.
			\end{itemize}
	\end{itemize}
	\begin{proof}
		See~\cite[p.~56]{Helffer2013}.
	\end{proof}
\end{prop}
In other words the continuous spectrum of a compact operator
may at most contain the value $0$ --- even in infinite dimensions.
Furthermore there is a nice result for the eigenfunctions of a compact, self-adjoint
operator:
\todoil{There is something missing here \ldots look it up in Helffer,
most especially no reference to the eigenfuctions!}
\begin{prop}
	\label{prop:CompactBasis}
	Let $\hilbert$ be a separable Hilbert space and $\Op{A}$ a compact,
	self-adjoint operator on $\hilbert$,
	then $\hilbert$ admits for a Hilbertian basis,
	\ie~a basis set $\{ u_k \}_{k \in \mathcal{I}}$ with
	\begin{align*}
	\braket{u_k}{u_l} &= \delta_{kl} \quad \forall \, k,l \in \mathcal{I}
	&&\text{and}&
	\spacespan \left( \{ u_k \}_{k \in \mathcal{I}} \right) &= \hilbert.
	\end{align*}
	\begin{proof}
		See~\cite[p.~60]{Helffer2013}.
	\end{proof}
\end{prop}

With propositions \ref{prop:CompactSpectrum} and \ref{prop:CompactBasis}
at hand, compact operators start to look a lot like the familiar case
of complex square matrices.
In fact one can show that~\cite[p.~43]{Helffer2013}
\begin{prop}
	\label{prop:FiniteDimCompact}
	Any linear operator on a finite-dimensional Hilbert space is compact.
\end{prop}
This is essentially a consequence of the fact that in finite dimensions
weak and strong convergence are equivalent.

\begin{rem}
With proposition \ref{prop:FiniteDimCompact}
we can reduce the setting of self-adjoint operators on a finite-dimensional
Hilbert space to the following:
\begin{itemize}
	\item In remark \vref{rem:HilbertCd} we said that the vectors
		of a $d$-dimensional Hilbert space
		can be represented as column vectors from $\C^d$.
		In a similar sense $\Op{A}$ can be identified by a finite matrix from
		$\C^{d \times d}$.
	\item The eigenfunctions of $\Op{A}$ are a complete orthonormal
		basis for the underlying Hilbert space.
		$\Op{A}$ has only real eigenvalues.
	\item Apart from zero $\Op{A}$ has only a point spectrum.
		The essential spectrum and the continuous spectrum at most consist of $0$.
\end{itemize}
\end{rem}

\begin{rem}
	\label{rem:NumericalEssentialSpectrum}
As we will see in the next sections \vref{sec:SpectrumLaplace}
and \vref{sec:HydrogenAtom} both the Laplace operator $\Delta$ as well as the
Hamiltonian $\Op{H}$ corresponding to hydrogen-like systems
are not compact on the Hilbert space $L^2(\R^3, \C)$,
since both of these operators are not even bounded.
Furthermore both of these operators do possess a non-trivial essential spectrum.

If a numerical approach for computing the spectra for these operators
should be used,
one naturally needs to restrict oneself to a finite-dimensional subspace
for solving the problem.
See section \vref{sec:RitzGalerkin} in the next chapter for details.
Because of prop. \vref{prop:FiniteDimCompact} our \emph{approximations}
to $\Delta$ and $\Op{H}$ will be compact.
As we just discussed these will therefore
at most have the value zero in their continuous spectrum.

Ignoring this $0$ for a moment,
we can state that
both the point spectrum as well as the continuous spectrum
of the infinite-dimensional operator will be mapped to the discrete
spectrum of the approximation.
For approximations to the discrete spectrum this is not a big problem.
As we go to infinite accuracy in our approximation,
we will recover more and more digits of the discrete eigenvalues
provided that our approximation is sensible.
For those eigenvalues which are part of the essential spectrum, however,
things are not so simple, because they might be surrounded by
discrete approximations to the continuous spectrum.
In general distinguishing between true eigenvalues
and so-called \newterm{spurious eigenvalues} inside the approximation
to the essential spectrum is hard up to impossible.
\end{rem}

It is therefore very important to know
the spectral properties of the exact operator
in order to understand which part of the spectrum one may obtain.
Let us discuss in the following a few examples,
which are important for our treatment of \QM.

\subsection{The Laplace operator}
\label{sec:SpectrumLaplace}
Let us first consider the $d$-dimensional analogue of the Laplace operator
introduced in \eqref{eqn:LaplaceOperatorHydrogen}.
In Cartesian coordinates it reads
\begin{equation}
	\Delta = \sum_{i=1}^d \frac{\partial^2}{\partial x_i^2}.
	\label{eqn:LaplaceOperatorD}
\end{equation}
Since this operator is essentially a scaled form of the kinetic energy operator $\Op{T}$
(see \eqref{eqn:OpHydrogen}),
we expect it to be self-adjoint and have real eigenvalues.

%TODO OPTIONAL some reasoning about this would be great,
%     explaining it in more detail
As it turns out, however,
the naive choice of taking the domain of the operator to be the full
quantum-mechanical Hilbert space $D(\Delta) = L^2(\R^d, \C)$
is not helpful as this operator cannot be made self-adjoint.
Only upon using the Sobolev space domain $D(\Delta) = H^2(\R^d, \C)$,
we get a self-adjoint operator $\Delta$.
Its spectrum is $\sigma(\Delta) = \sigma_C(\Delta) = [0, \infty)$%
~\cite[example 3.2.2]{Davies2007}.
In other words it is a semi-bounded operator with no eigenvalues and no
discrete spectrum at all.

\subsection{The Laplace-Beltrami operator on the unit sphere}
\label{sec:SpectrumLaplaceBeltrami}
\nomenclature{$Y_l^m(\theta, \varphi)$}{Spherical harmonic function with angular
momentum quantum number $l$ and azimuthal quantum number $m$.}
\nomenclature{$P^{m}_l$}{Associated Legendre polynomial with orders $l$ and $m$.}

In contrast to the previous section,
let us now consider the Laplace operator on the surface of the unit sphere
\[
	\set{S}^2 := \left\{ \vec{r} \in \R^3 \, \middle| \, x^2 + y^2 + z^2 = 1 \right\}.
\]
For this it is most convenient to consider
the spherical coordinate system,
\ie instead of parametrising the vector $\vec{r}$ as a
Cartesian column vector $(x, y, z)^T$,
we specify it as $(r, \theta, \varphi)$ with
\begin{align*}
	r &= \norm{\vec{r}} = \sqrt{x^2+y^2+z^2} & \theta &= \arccos\frac{z}{r} & \varphi &= \arctan\frac{y}{x}.
\end{align*}
The condition for the unit sphere than reduces to $r\stackrel{!}{=}1$.

Since the sphere has no longer a Euclidean geometry
but a curved manifold
the operator equivalent to \eqref{eqn:LaplaceOperatorD} takes
the deviating functional form
\newcommand{\laplaceSphere}{\Delta_{\set{S}^2}}
\begin{equation}
	\laplaceSphere  u = \frac{1}{\sin \theta} \frac{\partial}{\partial \theta}
\left( \sin \theta \frac{\partial u}{\partial \theta}  \right)
+ \frac{1}{(\sin \theta)^2} \frac{\partial^2}{\partial \varphi^2} u
	\label{eqn:LaplaceBeltramiS}
\end{equation}
in spherical polar coordinates.
\eqref{eqn:LaplaceBeltramiS} is sometimes called the
\newterm{Laplace-Beltrami operator} as well.

By taking the domain $D(\laplaceSphere) = H^2(\set{S}^2)$
the operator $\laplaceSphere$ is self-adjoint~\cite[p.~120]{Helffer2013}.
Furthermore one can show that the spectrum is (apart from $0$) fully discrete%
\footnote{This follows since the inverse $\laplaceSphere^{-1}$ is compact%
~\cite[p.~44]{Helffer2013}.}.
Therefore this can be explicitly calculated by solving
the ansatz $\laplaceSphere \, Y = \lambda Y$
for the eigenpairs $(\lambda, Y)$.
This results in the \newterm{spherical harmonics}
\begin{equation}
	Y_l^m(\theta, \varphi) = \sqrt{\frac{2l + 1}{4 \pi} \frac{(l - m)!}{(l+m)!}}%
	\, P_l^m(\cos \theta) \, e^{im\varphi}
	\label{eqn:SphericaHarmonics}
\end{equation}
where $P_l^m$ is the associated Legendre polynomial with orders $l$ and $m$.
The eigenvalue corresponding to $Y_l^m(\theta, \varphi)$ is $-l (l+1)$.
Due to the restriction
\[ -l \le m \le l \]
this eigenvalue is $(2l+1)$-fold degenerate.
Our spherical harmonics obviously satisfy
\begin{equation}
	-\laplaceSphere \, Y_l^m(\theta, \varphi) = l (l+1) Y_l^m(\theta, \varphi).
	\label{eqn:Laplace}
\end{equation}
For the next section let us briefly note,
that the Laplace-Beltrami operator on the unit sphere
and the Laplace operator in $3$ dimensions,
expressed in spherical polar coordinates, are related by
\newcommand{\laplaceRadial}{\frac{\partial}{\partial r} \left( r^2 \frac{\partial}{\partial r} \right)}
\begin{equation}
r^2 \Delta = \laplaceRadial + \laplaceSphere.
	\label{eqn:LaplaceCorrespondance}
\end{equation}
This allows to show
\begin{equation}
	- r^2 \Delta \, Y_l^m(\theta, \varphi) =
	-\laplaceSphere \, Y_l^m(\theta, \varphi) = l (l+1) \, Y_l^m(\theta, \varphi).
	\label{eqn:LaplaceSphericalHarmonic}
\end{equation}

An important consequence of the discreteness of the spectrum
of the Laplace-Beltrami operator is that the spherical harmonics form
a complete basis for $H^2(\set{S}^2)$.

\subsection{The Schrödinger operator for a hydrogen-like atom}
\label{sec:HydrogenAtom}
One might wonder if a pure Laplace operator as in section \vref{sec:SpectrumLaplace}
only possesses an essential spectrum,
how this develops if a potential is added,
like the $Z / r$ in the case of
the hydrogen-like Schrödinger operator \eqref{eqn:OpHydrogen}
\[
	\Op{H} = -\frac12 \Delta - \frac{Z}{r}.
\]
As the Hilbert space for this operator we take the \QM space $L^2(\R^3, \C)$
and an appropriate domain to make it self-adjoint is $H^2(\R^3, \C)$%
~\cite[p.~38]{Helffer2013}.
One can show~\cite{Davies2007,Teschl2014} that $\sigma_C(\Op{H}) = [0, \infty)$
and all discrete eigenvalues from $\sigma_P(\Op{H})$ are less than zero.
Thus $\sigma_\text{disc} = \sigma_P$ and $\sigma_\text{ess} = \sigma_C$.
The point spectrum of $\Op{H}$ can be conveniently
determined by solving the Schrödinger equation \eqref{eqn:TISE}
\begin{equation}
	( \Op{H} - E_{\mu} ) \Psi_\mu = 0,
	\label{eqn:HydrogenComplete}
\end{equation}
where $\Psi \in H^2(\R^3, \C)$ and $E \in \R^-$.
Without jumping ahead too far let us assume that the state
$\Psi_\mu$
may be uniquely identified by three quantum numbers $\mu \equiv (n, l, m)$.

\noindent
Using \eqref{eqn:LaplaceCorrespondance} we may write the Hamiltonian as
\begin{equation}
	\label{eqn:HydrogenLaplaceBeltrami}
	\Op{H} = -\frac{1}{2r^2} \laplaceRadial -\frac{1}{2r^2} \laplaceSphere - \frac{Z}{r}.
\end{equation}
A careful inspection of \eqref{eqn:HydrogenLaplaceBeltrami}
in contrast with \eqref{eqn:Laplace}
suggests a product ansatz
\[ \Psi_{nlm}(\vec{r}) = R_{nl}(r) Y_l^m(\theta, \phi). \]
%
With \eqref{eqn:LaplaceSphericalHarmonic} this yields the radial equation
\begin{equation}
	\left( - \frac{1}{2r^2} \laplaceRadial + \frac{l (l+1)}{2 r^2} - \frac{Z}{r} - E_\mu \right) R_{nl}(r) = 0
	\label{eqn:HydrogenRadial}
\end{equation}
which has the solutions~\cite{Mueller2000}
\begin{equation}
	 R_{nl}(r) = N_{nl} \left(\frac{2Zr}{n}\right)^l \exp\left(-\frac{Zr}{n} \right)
\;_1F_1\!\left(l+1-n \middle| 2l+2 \middle|\frac{2Zr}{n}\right)
	\label{eqn:HydrogenRadialSolution}
\end{equation}
where $_1F_1\!\left(a|b|\zeta\right)$ is a \newterm{confluent hypergeometric function},
namely~\cite{Avery2006}
\begin{equation}
	_1F_1\left(a \middle| b \middle| \zeta\right) =
	\sum_{k=0}^\infty \frac{a^{\bar{k}}}{k! \, b^{\bar{k}}} \zeta^k =
	1 + \frac{a}{b} \zeta + \frac{a(a+1)}{2b(b+1)} \zeta^2 + \cdots
	\label{eqn:ConfluentHypergeometric}
\end{equation}
with $a^{\bar{k}}$ being the rising factorial of $a$.
The normalisation constant is
\[ N_{nl} = \frac{2 \left( \frac{Z}{n} \right)^{3/2}}{(2l+1)!} \sqrt{ \frac{(l+n)!}{n (n-l-1)!}} \]
and the corresponding energy eigenvalues are
\begin{equation}
	E_{\mu} = - \frac{Z^2}{2n^2}.
	\label{eqn:HydrogenEnergyLevels}
\end{equation}

If one follows through the derivation properly,
one notices that the quantum numbers $n$, $l$ and $m$ are integer and need to satisfy
the following conditions:
\begin{align}
	\label{eqn:HydrogenIndexCondition}
	n &> 0 & 0 \leq l &< n & -l \leq &m \leq l
\end{align}
Furthermore since all involved equations are of Sturm-Liouville form,
the set of all solutions
\[
	\{\Psi_{nlm} \}_{n,l,m \text{ satisfy \eqref{eqn:HydrogenIndexCondition}}}
\]
forms the orthonormal basis for a Hilbert space $\mathcal{H}_\text{H}$.

We saw in examples \vref{exmp:H1sH2} and \vref{exmp:H1sH1}
that $\exp(-r) \in H^2(\R^3, \C)$,
which implies
\begin{equation}
	\Psi_{1s}(r, \theta, \phi) = \Psi_{100}(r, \theta, \phi)
	= \sqrt{\frac{Z^3}{\pi}} \exp(-Z r) \in H^2(\R^3, \C).
	\label{eqn:FunctionGShydrogen}
\end{equation}
From the functional form of $R_{nl}$ and $Y_l^m$ it is clear,
that all eigenstates $\Psi_{nlm}$ are infinitely differentiable everywhere
except at $r = 0$.
See \cite{Kato1957} and references therein for details.
The polynomial in $r$ in front of the exponential factor of the radial part $R_{nl}$
has exponents in $r$ in the range $[l, n-1]$
such that the eigenstate with $l = n-1 = 0$, i.e.~$\Psi_{1s}$ is the least smooth.
This implies $\Psi_{nlm} \in H^2(\R^3, \C)$ and thus
$\mathcal{H}_\text{H}$ is a (true) subspace%
\footnote{
	This is a true subspace, \ie non-identical to $H^2(\R^3, \C)$,
	since the scattering states are not part of it.
}
of $H^2(\R^3, \C)$.

\section{Take-away}

