\section{The correspondence of classical and quantum mechanics}
According to the Hamiltonian formulation of classical mechanics
a physical system with $N$ degrees of freedom
is described by a set of generalised coordinates
$q_1, \ldots, q_N$ along with their canonical momenta $p_1, \ldots, p_N$.
Any physically measurable quantity $F$
can now be obtained as a function of these system parameters.
In other words one may define a so-called \newterm{observable}
\newcommand{\clphase}{(q_1, \ldots, q_N, p_1, \ldots p_N)}
$F\clphase$ as a function $\R^{2N} \to \R$
from the phase space to the measured value.
Naturally the coordinates $q_i$ and $p_i$ are observables as well.
The most important observable is the total energy function or \newterm{Hamlitonian}
\begin{equation}
	H\clphase
	\equiv \underbrace{\frac{1}{2} \sum_{k=1}^N m_k p_k^2}_{= T(p_1, \ldots, p_N)}
	+ V(q_1, \ldots, q_N),
	\label{eqn:HamiltonianClassical}
\end{equation}
where $m_k$ is the mass of the particle associated with
the degree of freedom $k$,
$T$ is the kinetic energy observable and $V$ the total potential energy observable.
In the formalism of Hamiltonian mechanics $H$ governs the time evolution
of the system, namely
\begin{align}
	\frac{\D p_i}{\D t} &= - \frac{\partial H}{\partial q_i} &
	\frac{\D q_i}{\D t} &= \frac{\partial H}{\partial p_i}.
	\label{eqn:HamiltonianEqnMotion}
\end{align}

Using these expressions one may of course describe the time evolution
any arbitrary observable as well.
To make the connection to quantum mechanics more apparent,
let us introduce for this purpose the so-called \newterm{Poisson bracket}.
It is the skew-symmetric form
\begin{equation}
	\left\{ F, G \right\}_P \equiv \sum_{j=1}^n \left(
	\frac{\partial  F}{\partial q_j} \frac{\partial G}{\partial p_j}
	- \frac{\partial  F}{\partial p_j} \frac{\partial G}{\partial q_j}
	\right)
	\label{eqn:PoissonBracket}
\end{equation}
and according to
\begin{equation}
	\frac{\D F}{\D t} = \frac{\partial F}{\partial t} + \left\{ F, H \right\}_P.
	\label{eqn:PoissonTimeEvolution}
\end{equation}
it relates the Hamiltonian to the time evolution of any arbitrary observable.
One may also show the relationships
\begin{align}
	\{ q_k, q_l \}_P = \{p_k, p_l\}_P &= 0 & \{ p_k, q_l\}_P = - \delta_{kl} &&\forall k,l \in \{1, \ldots, N\}
	\label{eqn:PoissonProperties}
\end{align}
between the principle system observables.

\subsection{The move to quantum mechanics}
\defineabbr{QM}{QM\xspace}{quantum mechanics}
We will now introduce quantum mechanics~(QM) in a rather pragmatic manor,
namely as a recipe for deducing the \QM formulation
from the classical one
employing the so-called \newterm{correspondence principle}.
Some of the mathematical terms used in this section
will only be mentioned here and discussed in more detail
later on in this chapter.

The most important changes do be underdone for
moving from classical mechanics to \QM%
\footnote{More precisely: We consider here the position representation of \QM.}
are:
\begin{enumerate}
	\item Instead of phase space vectors $\clphase \in \R^{2N}$,
		employ functions $\Psi : \R^N \to \C$ from a
		complex separable Hilbert space $\mathcal{H}$
		in order to describe the state of the system.
	\item A classical observable $F$ is represented by a
		self-adjoint operator $\Op{F}$ on the Hilbert space $\mathcal{H}$.
	\item For each classical observable one may construct an equivalent
		corresponding quantum-mechanical operator.
		For example the observable $q_k$ corresponds to
		\[ q_k \longrightarrow \op{x}_k = x_k, \]
		\ie just the multiplication with $x_k$,
		the $k$-th coordinate of the system.
		On the other hand $p_k$ corresponds to
		\[ p_k \longrightarrow \op{p}_k
			= \frac{\hbar}{\I} \frac{\partial}{\partial x_k}, \]
		the scaled derivative with respect to $x_k$.
	\item All occurrences of the Poisson bracket with observables
		is to be replaced by the commutator
		\begin{equation}
			\comm{\Op{F}}{\Op{G}} = \Op{F} \Op{G} - \Op{G} \Op{F}
			\label{eqn:Operator}
		\end{equation}
		and corresponding operators, \ie
		\[ \{F, G\}_P \longrightarrow \frac{\comm{\Op{F}}{\Op{G}}}{\I \hbar}. \]
	\item The measured values of an observable $F$
		are the eigenvalues $\lambda_k$ of $\Op{F}$ \emph{only}.
		This implies, that for a pure state $\Psi$ with $\norm{\Psi} = 1$
		one may compute the expectation value of a measurement as
		\[ \langle \Op{F} \rangle = \int_{\R^N} \cc{\Psi}(\vec{x}) \left(\Op{F} \Psi\right)(\vec{x}) \D \vec{x} = \sum_{k\in\mathcal{I}} \int_{\R^N} \lambda_k \cc{\psi}_k(\vec{x}) \Psi(\vec{x}) \D x \]
		where we assume that we can find a complete and countable set
		of eigenpairs $\{ (\lambda_k, \psi_k) \}_{k \in \mathcal{I}}$
		of the operator $\Op{F}$, i.e. that
		\[ \Op{F} \psi_k = \lambda_k \psi_k \]
		Note, that this is not always possible.
		See \cite{Helffer2013} for a more general treatment.
\end{enumerate}
Even though these guidelines probably seem rather \textit{ad hoc},
they are closely related to the postulates of quantum mechanics.
For example from \eqref{eqn:PoissonProperties}
we may immediately deduce the compatibility relations between the space
and position operators
\begin{align}
	\comm{\op{x}_k}{\op{x}_l} &= \comm{\op{p}_k}{\op{p}_l} = 0 & \comm{\op{x}_k}{\op{p}_l} = \frac{\hbar}{\I} \delta_{kl}
	\label{eqn:compatibilityOperator}
\end{align}

The correspondence principle is backed up on a more rigorous
theoretical bases (into which we will not go into).
Consider for example the Ehrenfest theorem
\[
	\frac{\D \langle \Op{F} \rangle}{\D t} = \frac{\partial \langle \Op{F} \rangle}{\partial t} + \frac{1}{\I \hbar} \left\langle \comm{\Op{F}}{\Op{H}} \right\rangle
\]
and its similarity to the classical \eqref{eqn:PoissonTimeEvolution},
suggesting that classical Mechanics still --- on average --- still holds.

For our treatment the most important result of quantum mechanics
is the so-called Schrödinger equation,
which governs the evolution of a system state $\Psi(\vec{x})$ over time:
\[
alb
\]
where $\Op{H}$ is the analogon to the classical Hamiltonian,
namely the Hamiltonian operator.
Its eigenvalues are the energy levels of the system
\[ \Op{H} \Psi = E \Psi \]
The first is the time-dependent
the second the time-independent Schrödinger equation.

Let us build the expression of the \QM Hamiltonian for the
Hydrogen atom according to the correspondence principle

\subsection{Atomic units}
A very natural system of units for the exploration of \QM systems


We will use atomic units for all our considerations from here on.


