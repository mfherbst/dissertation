\section{Correspondence of classical and quantum mechanics}
\todoil{One could add a few more references for the general QM introduction part,
	\eg to the original Schrödinger paper, the Stone-von Neumann theorem
	or so}
\newcommand{\clphase}{(q_1, \ldots, q_d, p_1, \ldots p_d)}
%
According to the Hamiltonian formulation of classical mechanics
a physical system with $d$ degrees of freedom
is described by a set of generalised coordinates
$q_1, \ldots, q_d$ along with their canonical momenta $p_1, \ldots, p_d$.
It is assumed that any physically measurable quantity $F$
only depends on these system parameters.
In other words one may define a so-called \newterm{observable}
$F\clphase$ as a function $\R^{2d} \to \R$, \ie
from a vector in \newterm{phase space} to the measured value.
Clearly the coordinates $q_i$ and momenta $p_i$ are observables as well.
The most important observable is the total energy function or \newterm{Hamiltonian}
\begin{equation}
	H\clphase
	\equiv \underbrace{\frac{1}{2} \sum_{k=1}^d m_k p_k^2}_{= T(p_1, \ldots, p_d)}
	+ V(q_1, \ldots, q_d),
	\label{eqn:HamiltonianClassical}
\end{equation}
where $m_k$ is the mass of the particle associated with
the degree of freedom $k$,
$T$ is the kinetic energy observable and $V$ the total potential energy observable.
In the formalism of Hamiltonian mechanics $H$ governs the time evolution
of the system, namely
\begin{align}
	\frac{\D p_k}{\D t} &= - \frac{\partial H}{\partial q_k}, &
	\frac{\D q_k}{\D t} &= \frac{\partial H}{\partial p_k}
	&&\forall k \in \{1, \ldots, d\}.
	\label{eqn:HamiltonianEqnMotion}
\end{align}

These expressions allow to generalise the description of the time evolution
to any other arbitrary observable as well.
To make the connection to quantum mechanics more apparent,
let us introduce for this purpose the so-called \newterm{Poisson bracket}.
It is the skew-symmetric form
\begin{equation}
	\left\{ F, G \right\}_P \equiv \sum_{j=1}^d \left(
	\frac{\partial  F}{\partial q_j} \frac{\partial G}{\partial p_j}
	- \frac{\partial  F}{\partial p_j} \frac{\partial G}{\partial q_j}
	\right).
	\label{eqn:PoissonBracket}
\end{equation}
According to the Liouville equation
\begin{equation}
	\frac{\D F}{\D t} = \frac{\partial F}{\partial t} + \left\{ F, H \right\}_P.
	\label{eqn:PoissonTimeEvolution}
\end{equation}
it relates the Hamiltonian $H$ to the time evolution of any arbitrary observable $F$.
One may also show the relationships
\begin{align}
	\{ q_k, q_l \}_P = \{p_k, p_l\}_P &= 0 & \{ p_k, q_l\}_P = - \delta_{kl} &&\forall k,l \in \{1, \ldots, d\}
	\label{eqn:PoissonProperties}
\end{align}
between the principle system observables.

\subsection{The move to quantum mechanics}
\label{sec:IntroQM}
\defineabbr{QM}{QM\xspace}{quantum mechanics}
We will now introduce (non-relativistic)
quantum mechanics~(QM) in a rather pragmatic manner,
namely by stating a summary what changes in the \QM formulation
compared to the classical one.
We will abstain from stating all mathematical details at this point.
The reader should take this overview as a motivation
for a more detailed treatment of the mathematical concepts
further down this chapter.

\newcommand{\hilbert}{\mathcal{H}}
\begin{enumerate}
	\item Instead of phase space vectors $\clphase \in \R^{2d}$,
		in \QM a particular state of the system is represented
		by functions $\Psi : \R^d \to \C$
		living in a complex separable Hilbert space $\hilbert$.
	\item A classical observable $F$ is represented by a
		self-adjoint operator $\Op{F}$ on the Hilbert space $\hilbert$ in \QM.
	\item For each classical observable one may construct an equivalent
		corresponding quantum-mechanical operator.
		For example the observable $q_k$ corresponds%
		\footnote{In fact alternative constructions for the position and momentum
				operator are possible as well. We present the so-called
				\textit{position representation} here.}
		to~\cite{Shankar1994, Mueller2000}
		\[ q_k \longrightarrow \op{x}_k = x_k, \]
		\ie just the multiplication with $x_k$,
		the $k$-th coordinate of the system.
		On the other hand $p_k$ corresponds to~\cite{Shankar1994, Mueller2000}
		\[ p_k \longrightarrow \op{p}_k
			= \frac{\hbar}{\I} \frac{\partial}{\partial x_k}, \]
		an appropriately scaled derivative with respect to $x_k$.
	\item Relationships originating from classical mechanics
		can (usually) be transformed into their \QM analogue.
		For this replace all occurrences of the Poisson bracket
		and the contained classical observables with the commutator
		\begin{equation}
			\comm{\Op{F}}{\Op{G}} = \Op{F} \Op{G} - \Op{G} \Op{F}
			\label{eqn:Operator}
		\end{equation}
		and corresponding operators, \ie~\cite{Shankar1994}
		\[ \{F, G\}_P \longrightarrow \frac{\I}{\hbar} \comm{\Op{F}}{\Op{G}}. \]
	\item The measured values of an observable $F$
		are the eigenvalues $\lambda_k$ of $\Op{F}$ \emph{only}.
	\item Assume that we can find a complete and countable
		set of eigenpairs $\{ (\lambda_\mu, \psi_\mu) \}_{\mu \in \mathcal{I}}$
		for the operator $\Op{F}$%
		\footnote{Note that this is \emph{not} possible in general
		as we will see in section \vref{sec:spectral}.
		A more general treatment of spectral theory involving
		\textit{spectral projectors} allows to reformulate 6. for cases
		where such eigenpairs cannot be found.
		See \cite{Helffer2013} for details.},
		where $\mathcal{I}$ is an appropriate index set.
		One may then compute the expectation value of a measurement
		on a normalised%
		\footnote{unit normalised, \ie $\braket{\Psi}{\Psi}_\hilbert = 1$}
		state as
		\begin{equation}
			\langle \Op{F} \rangle
				= \braket{\Psi}{\Op{F} \, \Psi}_\hilbert
				= \sum_{\mu\in\mathcal{I}} \lambda_\mu \,
							\abs{\braket{\psi_\mu}{\Psi}_\hilbert}^2,
			\label{eqn:QMExpectation}
		\end{equation}
		where $\braket{\slot}{\slot}_\hilbert$ is the inner product
		of the Hilbert space $\hilbert$.
\end{enumerate}
Now that we discussed the \textit{ad hoc} modification of classical mechanics
in order to yield a theory based on the postulates of \QM,
let us see how one is able to deduce useful results of \QM
from the analogous expressions in classical mechanics.
For example from \eqref{eqn:PoissonProperties} one can immediately
deduce important commutator relations between the position and momentum operators:
\begin{align}
	\comm{\op{x}_k}{\op{x}_l} &= \comm{\op{p}_k}{\op{p}_l} = 0 & \comm{\op{x}_k}{\op{p}_l} = \frac{\hbar}{\I} \delta_{kl}.
	\label{eqn:compatibilityOperator}
\end{align}
Similarly from \eqref{eqn:PoissonTimeEvolution} we obtain
\begin{equation}
	\frac{\D \Op{F}}{\D t} = \frac{\partial \Op{F}}{\partial t} + \frac{\I}{\hbar} \comm{\Op{F}}{\Op{H}},
	\label{eqn:HeisenbergEqnMotion}
\end{equation}
the equation of motion, which governs the time-evolution of the operator $\Op{F}$
in the so-called \newterm{Heisenberg picture} of \QM.
Taking the statistical average over \eqref{eqn:HeisenbergEqnMotion}
results in the Ehrenfest theorem
\begin{equation}
	\frac{\D \langle \Op{F} \rangle}{\D t} = \frac{\partial \langle \Op{F} \rangle}{\partial t} + \frac{1}{\I \hbar} \left\langle \comm{\Op{F}}{\Op{H}} \right\rangle.
	\label{eqn:EhrenfestTheorem}
\end{equation}
This results allows to rationalise the \newterm{correspondence principle}
of classical mechanics and \QM,
which we have developed so far.
Comparing \eqref{eqn:EhrenfestTheorem} and \eqref{eqn:PoissonTimeEvolution}
and keeping in mind that the expectation value $\langle \Op{F} \rangle$
as well as the classical observable $F$ both tell us about the result
of a measurement,
we can deduce that --- on average --- classical mechanics still holds.
Thus we may expect the classical expressions to carry some meaning
in the \QM sense as well.

\subsection{The Schrödinger equation}
Even though the Heisenberg picture we developed
in the previous section is great for deducing the analogy between
classical mechanics and \QM,
it is not very suitable for the kinds of problems we will be looking
at in the remainder of this thesis.
Our problems are best described in the \newterm{Schrödinger picture} of \QM,
which differs to the Heisenberg picture by the way it treats time evolution.
In the Heisenberg picture the state function $\Psi$ is time-independent
and the operators evolve.
In the Schrödinger picture it is the other way round, i.e. $\Psi$ may change over time
and the operators are static.
Both pictures are related by a unitary transformation
in the Hilbert space $\hilbert$
governed by the Stone-von Neumann theorem.

The equivalent expression to \eqref{eqn:HeisenbergEqnMotion}
in the Schrödinger picture is the \newterm{time-dependent Schrödinger equation}%
~\cite{Shankar1994,Mueller2000}
\begin{equation}
	\Op{H} \Psi = \I \hbar \frac{\partial}{\partial t} \Psi.
	\label{eqn:TDSE}
\end{equation}
Similar to \eqref{eqn:HeisenbergEqnMotion} the key operator governing the
time-evolution of the system is $\Op{H}$.
By analogy to its classical counterpart $\Op{H}$ is referred to as the
\newterm{\QM Hamiltonian} or just Hamiltonian as well.
In fact many properties of a system may already be deduced by considering the
eigendecomposition of its Hamiltonian $\Op{H}$ alone.
In light of this the ansatz
\begin{equation}
	\Op{H} \Psi_\mu = E_\mu \Psi_\mu
	\label{eqn:TISE}
\end{equation}
for finding the Hamiltonian's eigenpairs
\[ \{(E_\mu, \Psi_\mu)\}_{\mu \in \mathcal{I}} \subset \R \times \hilbert \]
is given the name \newterm{time-independent Schrödinger equation} as well.

\subsection{Atomic units}
\nomenclature{$\Delta$}{Laplace operator}
\nomenclature{$\Op{T}$}{Kinetic energy operator}
Employing the correspondence principle it is often very convenient
to construct the \QM Hamiltonian of a system starting
from the classical energy expression.
Let us consider a hydrogen-like system,
where a particle of positive charge $Ze$
is clamped at the origin and surrounded by a single electron.
In Cartesian coordinates the position of the electron can be described
by the vector $\vec{r} \equiv \rtp{(x_1, x_2, x_3)}$ and its momentum by $\vec{p}$.
The classical kinetic energy and potential energy of such a system
is given by
\begin{align}
	T &= \frac{\vec{p} \cdot \vec{p}}{2 m_e} &
	&\text{and}&
	V &= - \frac{Ze^2}{\norm{\vec{r}}_2} = - \frac{Ze^2}{r}. \\
\intertext{respectively. The appropriate \QM analogues are}
	\Op{T} &= -\frac{\hbar^2}{2 m_e} \Delta &
	&\text{and}&
	\Op{V} &= - \frac{Ze^2}{r}
	\label{eqn:OpHydrogenConventional}
\end{align}
where the \newterm{Laplace operator} in $3$ dimensions
\begin{align}
	\Delta = \sum_{i=1}^3 \frac{\partial^2}{\partial x_i^2}
	\label{eqn:LaplaceOperatorHydrogen}
\end{align}
was introduced. The full Hamiltonian therefore reads
\begin{equation}
	\Op{H} = \Op{T} + \Op{V} = -\frac{\hbar^2}{2 m_e} \Delta - \frac{Ze^2}{r}.
	\label{eqn:OpHydrogenFullUnits}
\end{equation}
For considering more complicated systems,
it is often convenient to employ so-called \newterm{atomic units}.
These are generated by a unitary transformation of the Hilbert space,
which effectively yields
\[ \hbar \equiv e \equiv m_e \equiv a_0 \equiv E_h \equiv 1. \]
See table \vref{tab:AtomicUnits} for the values of these quantities
in terms of the usual SI units.
The hydrogen-like Hamiltonian \vref{eqn:OpHydrogenConventional}
in atomic units takes the simple form
\begin{equation}
	\Op{H} = -\frac12 \Delta - \frac{Z}{r}
	\label{eqn:OpHydrogen}
\end{equation}
We will from now work exclusively in these units.
\begin{table}
	\centering
	\begin{tabular}{lllll}
		\toprule
		symbol & name & atomic unit of & value in SI units \\
		\midrule
		$\hbar$ & Planck constant & momentum & $\unit[1.055\E{-34}]{Js}$ \\
		$e$ & elementary charge & charge & $\unit[1.602\E{-19}]{C}$ \\
		$m_e$ & electron mass & mass & $\unit[9.110\E{-31}]{kg}$ \\
		$a_0$ & Bohr radius & length & $\unit[5.292\E{-11}]{m}$ \\
		$E_h$ & Hartree energy & energy & $\unit[4.360\E{-18}]{J}$ \\
		\bottomrule
	\end{tabular}
	\caption{Atomic units and their relationship to SI units}
	\label{tab:AtomicUnits}
\end{table}
