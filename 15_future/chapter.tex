\chapter{Prospects and future work}
\label{ch:Prospects}
\chapquote{%
Humanity needs practical men,
who get the most out of their work, and,
without forgetting the general good,
safeguard their own interests.
But humanity also needs dreamers,
for whom the disinterested development of an enterprise is
so captivating that it becomes impossible for them to devote
their care to their own material profit.}
{Marie Skłodowska Curie~(1867--1934)}

\noindent
With the availability of the \molsturm program package in its
current state, a flexible research tool
for the development and the investigation of novel quantum-chemical methods
has become available.
Especially the possibility to easily extend the present functionality%
~(see section \vref{sec:MolsturmState})
by linking to existing third-party packages,
opens the door to rapidly try novel combinations of basis function types
and existing quantum-chemical methods.
In this way one could seek to find alternatives
for cases where the usual Gaussian-type basis functions have problems,
see section \ref{sec:SturmianExcited} for examples,
or evaluate the properties of novel basis functions
with respect to the methods available from \molsturm,
see sections \ref{sec:newSturmian} and \ref{sec:newFE}.

On the other hand, a framework where multiple basis function types
are available via the some common interface allows to
implement a novel method only once and then subsequently
test it employing a range of basis function types.
This was already discussed in section \vref{sec:ex:ccd}
where a coupled-cluster doubles~(CCD) code building on top of \molsturm was shown.
Already at the present state, such a user code can
directly utilise all basis function types available in \molsturm,
\ie Coulomb-Sturmians as well as Gaussian-type orbitals,
but the performance is far from optimal.
The main reason for this is a limitations of the \lazyten lazy matrix library,
namely that it right now can only deal with vectors and matrices,
but not more general tensors,
which are typically required in Post-\HF methods such as CCD.
For this reason all quantities in the \python interface
of \molsturm are full blocks of data,
destroying all sparsity inherent to the basis function type.
This limitation should be lifted in the future,
see section \ref{sec:GeneraliseLazyten}.
Further aspects where \molsturm can be improved are discussed
in section \ref{sec:newMolsturm}.

\section{Generalisation of \lazyten}
\label{sec:GeneraliseLazyten}
The lazy matrices of \lazyten are a key component of \molsturm,
being both the language in which the self-consistent field~(\SCF)
algorithms are implemented
as well as the data structure
mediating the call to the integral libraries.
Many performance improvements of \lazyten will therefore
directly affect \molsturm as well and are thus worth pursuing.
Additionally \lazyten is a package on its own,
such that all aspect mentioned here could make it more
applicable to a broader context than just
quantum-chemical method development.

Currently the \lazyten library is limited to lazy matrices,
which implies that it only supports the implementation of algorithms
involving matrix-matrix and matrix-vector operations.
As soon as higher-order tensor contractions are required,
these need to be hidden inside the lazy matrix data structures.
Effectively, this implies
that they are either to be implemented manually
or by appropriate calls to some third-party tensor library.
Already at the level of the \SCF
this limitation is apparent,
namely when the Coulomb matrix $\mat{J}$
and the exchange matrix $\mat{K}$ are formed
from the electron repulsion integral~(\ERI) tensor
$\eriMu{\mu\nu}{\kappa\lambda}$, see section \vref{sec:DiscreteHF}.
Right now this is solved by implementing two
hard-coded contraction schemes of the \ERI tensor with
the coefficient matrices resulting in $\mat{J}$ and $\mat{K}$.

In the future, such issues could be tackled
by extending \lazyten towards lazy tensors as well.
Similarly to lazy matrices,
such lazy tensor objects would no longer be
directly mapped to flat data,
but allow the encapsulation of arbitrary expressions
for computing contractions with itself or other tensors.
For the evaluation of operations between lazy tensors
again a lazy evaluation scheme would be employed,
such that the contraction between tensors would first be
stored as a tensor expression tree and only evaluated when needed.
The prospect of the proposed ansatz is
that an object like the \ERI tensor
could be represented as a single entity in \molsturm,
irrespective of the precise structure,
which is sensible for a particular basis function type.
In this way a unified treatment
of a dense, stored \ERI tensor,
like it is the case for Gaussian-type basis functions
or an object consisting of a more elaborate contraction,
like for Coulomb-Sturmians \eqref{eqn:CSeriContraction},
is possible.
In line with the lazy matrices the advantage of such a
generalisation would be that the computational details
are well-separated from the user code,
allowing for a dynamic change of algorithms
reacting to the requirements of basis function types
or the hardware the program runs on.

Undoubtedly, already the design of such a library is challenging
and would need to proceed in steps,
gradually generalising the lazy matrix concept towards
such lazy tensors
by means of real-world examples.
A good testbed for this are the Post-\HF methods of quantum chemistry.

Another question is how to find a suitable evaluation
scheme for dealing with the arising expression tree.
This is a topic of ongoing research
where, both in the context of quantum-chemical calculations
as well as a more general setting,
enormous progress has been made~\cite{Baumgartner2005,Solomonik2014,%
Peise2015,Calvin2015,Calvin2015arxiv,Xerus,Kristensen2016array,%
Kristensen2016streaming,Libtensor}.
The hope for \lazyten is that this existing effort
could be incorporate or reused.

\section{\molsturm program package}
\label{sec:newMolsturm}.
After two years of development,
\molsturm is in a state,
where calculations based on contracted Gaussian basis sets
and Coulomb-Sturmian basis sets can be performed.
Furthermore, as mentioned in section \vref{sec:MolsturmState}
and demonstrated in chapter \vref{ch:CSQChem},
not only \HF, but Full-CI (\FCI) and
methods based on the algebraic-diagrammatic construction scheme (\ADC)
are available
via interfaces to \pyscf~\cite{Sun2017} and \adcman~\cite{Wormit2014}.
For employing these Post-\HF methods
with more than a few tens of basis functions,
a drawback at the moment is performance.

To some extent, the improvements
of \lazyten directly lead to angles for tackling these problems as well.
For example a generalisation of \lazyten to lazy tensors
would allow to use lazy tensors as data structures on the level of \molsturm's interface.
Right now this is prevented by the \ERI tensor,
which --- as indicated above --- cannot be represented as a lazy matrix,
such that a costly step of copying all data from the internal
representation to the data structures of the interface results.
On the level of the \SCF the automatic parallelisation
of the evaluation of the lazy matrix expression trees
would help to improve the efficiency.

% TODO OPTIONAL
%\subsection{Reliable contraction-based self-consistent field algorithm}
%So far none of the \SCF algorithms available in \molsturm
%are perfectly reliable and frequently the implemented \SCF scheme
%amounts to find stationary states of the \HF problem,
%which are not the true \SCF minimum.
%Remedies could be to implement more sophisticated algorithms like
%the energy \DIIS~\cite{Kudin2002}
%or a second-order \SCF scheme~\cite{Salek2007,Hoest2008}
%inside the \contraction-based setting of \molsturm.

\subsection{Interfaces to further quantum-chemistry packages}
\label{sec:ExtensionInterfaces}
Given the modular structure of \molsturm
interfacing to further quantum-chemistry libraries or program packages
can be easily achieved to extend the scope of the package.
A good example is the \texttt{libxc}~\cite{Lehtola2018} library,
which offers a range of exchange-correlation functionals.
Implementing an interface to this library inside appropriate
lazy matrix objects would allow to construct the Kohn-Sham matrix
inside \molsturm's \SCF without any further changes,
such that density-functional theory calculations would become available.
Furthermore a better link to \pyscf~\cite{Sun2017}
would allow to perform configuration-interaction
and coupled-cluster calculations as well as calculations
employing the density matrix renormalisation group approach
directly from \molsturm.
The modular structure of \psifour makes it another good candidate
for extending the number of available methods
on top of \molsturm's \SCF.

On top of that data exchange with existing quantum-chemistry-related software
could be simplified even further by supporting
some wide-spread file formats in \molsturm
for example \texttt{Molpro}'s FCIDUMP~\cite{Knowles1989},
as well as routines for dumping plotting data
in \texttt{Gaussian}~\cite{Frisch2016} cube or VTK~\cite{Avila2010} files.
The \texttt{molden}~\cite{Schaftenaar2000} format
could be supported by means of a fitting function
representing results,
which are not based on Gaussian-type orbitals in the form of a minimal
Gaussian-type basis,
similar to the way suggested in \cite{Stewart1970}
for Slater-type orbitals.

% TODO OPTIONAL
% \subsection{Partial rewrite in \python}

\section{Investigation of Sturmian-type discretisations}
\label{sec:newSturmian}

The original purpose of the \molsturm package has been to devise a program
which could be used for quantum-chemical calculations
employing Sturmian-type basis functions,
which are a category of analytic solutions to Schrödinger-like equations.
The package has outgrown this purpose in the current design,
but investigating the properties of Sturmian-type basis functions is still of interest
as they are on the one hand able to properly represent
the physical features of the wave function,
see section \ref{sec:BasisCS},
and lead to feasible integrals on the other hand.
There is further some indication~\cite{Guell2008,Hoggan2009}
that exponential basis functions such as Sturmian-type functions
might be advantageous
for computing properties like the nuclear magnetic resonance~(NMR) shielding tensor.
Additionally their completeness property
makes them interesting for applications where both bound states
as well as the continuum need to be represented
like Fano-Stieltjes~\cite{Feshbach1958,Feshbach1962,Santra2002}
or other methods to model resonance processes.

\subsection{Convergence properties of Coulomb Sturmian basis sets}
The simplest example for Sturmian-type basis functions are Coulomb-Sturmians.
An initial investigation of such basis functions presented
in chapter \ref{ch:CSQChem} looks overall promising,
but the obtained results are not yet sufficient to provide definitive
construction schemes for Coulomb-Sturmian~(\CS) basis sets
or general estimates for the overall accuracy.
There are three ways this could be improved.

Firstly so far only main group elements of the second and third
period of the periodic table were considered.
Originating from the involvement of the $d$-orbitals
the properties of the electronic structures of the transition metals does, however,
differ compared to the main group elements.
An analysis with respect to the fourth period and beyond
is therefore required to reach more general conclusions.

Secondly, most of the presented investigation has concentrated on the \HF
level with some minor modifications suggested mostly based on {\MP}2 results.
Whilst these two methods are both used in electronic structure theory,
more methods should be added to reach a representative set.
Most notably {\MP}2 as a perturbative approach
for modelling electron correlation effects is very different from
configuration-interaction-based
or coupled-cluster-based approaches
--- both with respect to the way the physics is described
as well as the numerics.
Further convergence studies for example employing the latter kind of methods
would be required.
% TODO OPTIONAL
%\to doil{
%Additionally one should note
%that \MP approaches building on top of a unrestricted \HF ground state
%are known to have convergence issues~\cite{Knowles1985,Handy1985},
%such that perhaps a restricted open-shell \HF approach should be employed
%to describe the reference state for {\MP}2.
%}

Thirdly, the full flexibility towards constructing \CS basis sets has not yet
been exploited in our discussion in section \vref{sec:CSconvergenceHF}.
There is no reason why one should define a basis set
by limiting the angular quantum numbers $l$ and $m$ to the same maximum for all
principle quantum numbers $n$.
As mentioned in section \vref{sec:DenotingCSbasis} a \CS basis set can be any
combination of the quantum number triples $(n, l, m)$,
such that, for example, keeping the full range of $l$ and $m$
up to $n=5$ and then only up a maximum of $l=2$ is possible
and should definitely be tried.
With respect to this it would also be interesting to compare the
observed convergence properties with the typical construction schemes
employed for contracted Gaussian~(\cGTO) basis sets~\cite{Jensen2013,Hill2013}.
Potentially the schemes employed in the \cGTO setting are applicable
to \CS basis sets and vice versa.

In general a detailed comparison of the results obtained
from employing Coulomb-Sturmians as well as \cGTO discretisations seems appropriate.
An interesting question is, for example, the required basis sizes for both
discretisation types to reach a certain accuracy
in the description of the ground or excited-state energies
at various levels of theory.

\subsection{Coulomb-Sturmian-based excited states calculations}
\label{sec:SturmianExcited}
In section \vref{sec:SturmianADC} we already presented
first results for computing excited state energies of atoms based on \ADC.
A more detailed analysis of the convergence properties
could help to proceed with the application of \CS-based \ADC
calculations in order to compute the spectra of atoms.
Due to the completeness of the Coulomb-Sturmians
and their possibility to describe both core region as well as
the exponential decay,
a range of interesting applications for the modelling
of excitation processes come to mind.
Three of them are 
(1) methods where the continuum needs to be modelled,
like Fano-Stieltjes~\cite{Feshbach1958,Feshbach1962,Santra2002},
(2) cases where modelling both the core and the valence shell is required,
like core-valence excitations~\cite{Wenzel2014,Wenzel2016}
as well as
(3) the modelling of expanded bound states,
like the determination of Rydberg-like states~\cite{Kaufmann1989,Riss1993}.

\subsection{Avoiding the Coulomb-Sturmian exponent as a parameter}
An unfavourable aspect of the \CS basis sets
employed in chapter \ref{ch:CSQChem} of this thesis is
the Coulomb-Sturmian exponent $\kexp$.
As was discussed in section \ref{sec:kexp}
this parameter has indeed an influence on the results obtained
and thus needs to be set to a sensible value to obtain sensible results.
If not, than unphysical behaviour can result.
In my calculations I have, for example, found cases
where changing $\kexp$ could change the obtained \SCF minimum
for \HF.
Furthermore, there is some indication
that the ordering of excited states in \ADC calculations
depends on $\kexp$ in some cases.
Using the algorithm described in section \ref{sec:DetermineKopt},
a route for finding an optimal value at \HF level has been sketched.
With respect to excited states methods like \ADC,
it is not immediately obvious how to determine the most optimal exponent,
since the value for describing the ground state with the least error
will differ from the value for representing a particular excited state best.
In turn each excited state will have a different $\kexp$
to give the best description in a particular \CS basis.
Which value or which combination of the values should be taken
is not directly clear.

\noindent
An equivalent problem at \FCI level can be avoided.
The reason is that the relationship
\[ E = - \frac{\kexp^2}{2}, \]
between the \CS exponent $\kexp$ and the energy $E$ of a particular state,
can be employed to re-formulate \FCI in terms of the
Coulomb-Sturmian exponents~\cite{Avery2006}.
In other words instead of solving for the energy of a state,
one solves for the $\kexp$ for each state
and uses this value both to find the corresponding energy
as well as the exponent of the basis functions,
when properties for such a state are to be computed.
A similar reformulation should be possible for \HF
and potentially even for some other Post-\HF methods as well,
even though this is uncertain at the moment.
If this could be achieved both the determination of an optimal
$\kexp$ would become obsolete at \HF level
and for excited states $\kexp$ would adapt automatically to the required state.

\subsection{Molecular Sturmians}
\label{sec:MolecularSturmian}
The implementation of a \CS-based \SCF scheme was always intended
to be only the first step with respect to the exploration
of Sturmian-based quantum-chemical calculations.
More general and at the same time more challenging types of Sturmian basis
functions exist,
which are able to describe molecular systems, for example.
Building on recent advances in the calculation of the \ERI integrals
\eqref{eqn:ERI} for such generalised Sturmian-type orbitals%
~\cite{Avery2006,Avery2011PhD,Avery2011,Morales2016,Avery2017,Randazzo2015,Granados2016}
a Sturmian-based \HF suitable for molecular calculations is within reach
and could be implemented within \sturmint~\cite{sturmintWeb}
building on already existing infrastructure required for the Coulomb-Sturmians.
This would make molecular Sturmian integrals available to \molsturm.

\section{Finite-element-based quantum chemistry}
\label{sec:newFE}
At the beginning of my doctoral studies I already attempted
unsuccessfully to implement a \HF scheme based on finite-elements,
a piecewise polynomial basis based on a real-space grid,
see section \vref{sec:FE}.
Recent advances of the field~\cite{Frediani2015,Toivanen2015,Davydov2015,Boffi2016}
seem promising and
implementing a grid-based basis into \molsturm
would constitute an interesting test case with respect to the aspired generality
of \molsturm's \SCF scheme.
A link to the finite-element library \texttt{deal.ii}~\cite{Arndt2017}
could provide the relevant back end to perform the computations.
Since the scheme we proposed in section \ref{sec:FE}
for evaluating the exchange matrix contraction
expression \eqref{eqn:ExchangeFromPotentials}
has, however, never been implemented in practice,
there are many open questions regarding the numerical details.
Before one can proceed to perform a proper \SCF based on finite elements inside \molsturm,
one therefore needs to investigate questions such as
which quadrature scheme to employ,
how many quadrature points are required,
which finite-element order to use for the inner Poisson solves,
what are sensible convergence thresholds and so on.
All together the task remains challenging.

% TODO OPTIONAL
%Last but not least the ability to treat different kinds of basis functions
%in the same framework simplifies the construction of hybrid basis sets,
%where possibly numerical basis functions and GTOs or STOs are combined.
%Similarly one could employ the strengths of multiple backends for the same type of basis function
%in the sense that one mixes and matches different backends such that one can exploit
%the advantages of both implementations best.

\section{Fuzzing of integral back ends}
\label{sec:Fuzzing}
Using the common interface which \molsturm provides
for accessing the implemented integral libraries
allows to test the correctness
of the algorithms these libraries employ
by comparing the results of
random or semi-random input against each other.
Due to the \python interface of \molsturm, such a process could
even be completely automatised.
Such fuzzing approaches have already been applied with huge success
in the context of hardening security-critical software~\cite{Fuzzing}.
Similarly previous a work by \citet{Knizia2011} to test
the hardness of quantum-chemical software
with respect to numerical instabilities by using random noise
similarly lead to the discovery of unexpected bugs
in the integral evaluation scheme of \texttt{Molpro}.
