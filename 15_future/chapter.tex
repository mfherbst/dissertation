\chapter{Prospects and further work}
\label{ch:Prospects}
\chapquote{%
Humanity needs practical men,
who get the most out of their work, and,
without forgetting the general good,
safeguard their own interests.
But humanity also needs dreamers,
for whom the disinterested development of an enterprise is
so captivating that it becomes impossible for them to devote
their care to their own material profit.}
{Marie Skłodowska Curie~(1867--1934)}
\todoil{Draft version}

\noindent
With the availability of the \molsturm program package in its
current state a flexible research tool
for the development and the investigation of novel quantum-chemical methods
has become available.
Especially the aspect to quickly incorporate third-party packages
for performing Post-\HF methods or to link to new integral back ends
opens the door for many further research directions.
But also on the level of \molsturm's infrastructure
by itself improvements can be achieved.
Many algorithms, including the lazy matrices in \lazyten,
are still only in draft status,
performing a lot worse than what could be achieved with further effort.
Some angles for future work shall be presented here
along with some ideas to achieve these ideas.

\section{Improvements of \lazyten}
The lazy matrices of \lazyten are a key component of \molsturm,
being both the language in which the \SCF algorithms are implemented
as well as being the data structure
which handles the calls to the integral back end libraries
in the integral interface.
Furthermore all calls to the underlying linear algebra packages,
which are required during the execution of \molsturm
are handled by \lazyten.
For this reason any performance
improvement in \lazyten will directly reflect in improvements
in \molsturm as well.
Additionally \lazyten is a package by its own,
which is applicable to a broader context than just
quantum-chemical method development.

\subsection{Generalisation to lazy tensors}
Currently the \lazyten library is limited to lazy matrices,
which implies that it supports the implementation of algorithms
involving matrix-matrix and matrix-vector operations only.
If higher-order tensor contractions are required in the contraction expression,
these either need to be coded manually or one needs to perform
a call to the underlying linear algebra package.
In either case one needs to make sure that at most rank-2 tensors,
\ie matrices, are exposed via the lazy matrix interface.
Already at the level of the \SCF of \molsturm we run into this limitation,
namely when we form the Coulomb matrix $\mat{J}$
and the exchange matrix $\mat{K}$ from the electron repulsion integral tensor
$\eriMu{\mu\nu}{\kappa\lambda}$, see equations \eqref{eqn:Jbas} and \eqref{eqn:Kbas}.
For two types of contraction schemes the resulting code duplication is manageable,
but if a different contraction of the coefficient matrix
with the \ERI tensor was required for whatever reason
a new lazy matrix object needed to be implemented.

The alternative is to extend \lazyten towards lazy tensors as well,
\ie tensor objects, which no longer are flat data but arbitrary encapsulated
expressions for computing contractions with other tensors.
For the evaluation of operations between lazy tensors
again a lazy evaluation scheme would be employed.
In other words contractions between two lazy tensors are again
a lazy tensor and so fourth.
Naturally care has to be taken to deal with the arising complexity
of the implied lazy tensor evaluation trees in a sensible manner.

For certain types of basis functions like the Coulomb-Sturmians,
where the \ERI tensor is a sparse tensor expression by itself,
see equation \eqref{eqn:CSeriContraction},
this has high potential.
Only once the result of a lazy tensor contraction is needed
an evaluation involving the \CS \ERI tensor would be performed
and as a result the maximum number of selection rules could be taken into account.
Recall our discussion in section \vref{sec:BasisCS},
why lazy matrices are helpful for Coulomb-Sturmian-based discretisations
in this context.

Other prospects include the coding of Post-\HF methods
like for example the CCD code we presented in figure \ref{fig:codeCCD}
of section \ref{sec:ex:ccd}.
Here the potential benefit is that a lazy tensor object
could well be a data structure incorporating certain symmetry properties
of the physical problem,
for example the block-sparsity due to spin.
The result is that the tensor contractions required for methods like CCD
can still be coded in a high-level fashion,
but symmetry properties or the selection rules of the underlying
tensor expressions could be taken into account during evaluation.
Similar to the lazy matrices the aim would be to reach a high-level language
for writing tensor contraction expressions,
where the evaluation, which is preformed under the hood
adapts to the properties of the underlying basis function type.

\subsection{Automatic optimisation and parallelisation of evaluation}
Related to the previous idea
is the notion to work on schemes for the automatic
optimisation and parallelisation of the evaluation
of the expression trees occurring in \lazyten.
The idea to automatically optimise matrix or tensor contractions
before executing them is hardly new~\cite{Baumgartner2005,Peise2015,Xerus,Kristensen2016array,Kristensen2016streaming}
and the hope is to incorporate or reuse some of these aspects
in the future.
A good first step in either case would be an even closer integration
whit the Bohrium~\cite{Kristensen2016array,Kristensen2016streaming} package,
such that their efforts for optimising the performance
of array operations could be used as much as possible.


\section{\molsturm program package}
\todoil{TODO}

Major issues are reliablility of the algorithms
and efficiency

A disadvantage of the current \molsturm interface is,
that all data is stored as plain and dense memory.
Both symmetry as well as sparsity properties of the involved tensors
are entirely ignored at the moment.
This yields an unacceptably large requirement in memory
as well as a very unfavourable computational scaling
for the demanding tensor contractions
typically required in Post-HF methods like \CCD (see figure \ref{fig:codeCCD}).
In the future we want to tackle this by extending
\lazyten towards lazy tensors as well.
In practice this would imply that the evaluation of
tensor contractions like in figure \ref{fig:codeCCD} could be implicitly delayed
for as long as possible.
Only once the results are needed in the code,
the calculation would be underdone
for all delayed expressions at once.
This could allow to automatically deduce an ideal evaluation strategy
taking all sparsity and symmetry information
of the involved tensors into account.
Ideally this would happen transparently in the background
without changing the \python interface at all.


\subsection{Reliable contraction-based \SCF scheme}
So far none of the implemented \SCF algorithms fits the \contraction-based
setting of \molsturm perfectly
in the sense that it leads to an efficient
and reliably converging \SCF algorithm.
Part of the problem is that our current \SCF implementation
is done in \cpp,
which makes it somewhat difficult to experiment.
A recently added \python interface to obtain raw integral data
from any of \molsturm's supported integral back ends
opens up the possibility to experiment
with some of the more sophisticated algorithms like
the energy \DIIS~\cite{Kudin2002}
or a second-order \SCF scheme~\cite{Salek2007,Hoest2008}
and try to bring these to the contraction-based setting.


\subsection{Available interfaces}
Another angle for future work would be to integrate with other
quantum-chemistry libraries or program packages.
A good example is the \texttt{libxc}~\cite{Lehtola2018} library,
which offers a range of exchange-correlation functionals.
Implementing an interface to this library inside appropriate
lazy matrix objects would allow to construct the Kohn-Sham matrix
inside \molsturm's \contraction-based \SCF as well,
such that density-functional theory~(DFT) calculations
could be performed.
Along a similar line a better link to \pyscf~\cite{Sun2017}
could allow for the implemented wide range of configuration-interaction,
coupled-cluster and DMRG methods to be used from \molsturm.
The \python interface of \psifour seems like another good candidate
for quickly extending the number of available methods
on top of \molsturm's \SCF.

Similarly we would like to simplify
data exchange with other, more traditional quantum chemistry packages
by supporting some wide-spread file formats of quantum chemistry,
\eg \texttt{MolPro}'s FCIDUMP~\cite{Knowles1989}.
For facilitating plotting of \SCF orbitals or densities
we plan to implement the \texttt{molden}~\cite{Schaftenaar2000} format
as well as routines for dumping the data
in \texttt{Gaussian}~\cite{Frisch2016} cube or VTK~\cite{Avila2010} files.

% TODO OPTIONAL
% \subsection{Partial rewrite in \python}

\section{Investigation of basis function types}

The main purpose of \molsturm was after all to facilitate
the investigation of novel basis function types.
In chapter \vref{ch:CSQChem} we have already applied
\molsturm in this context. 

\subsection{Sturmian-based quantum chemistry}
\todoil{TODO}

Only second and third period
more systematic results


%
%
Improve stabilities of \SCF scheme $\Rightarrow$ Leads to better procedure for finding $\kopt$
%
improvements on basis set sizes my limiting $\mmax$ or not keeping the complete $\lmax$ everywhere
Investigate usefulness of Sturmians for ADC
Comparison to cGTO basis sets
MP2 not a great method for convergence studies (e.g. not variational with respect to FCI solution), so better to rely on CISD or CC methods
play around with more complex CS basis sets
replace rmsol by maximum coefficient per angular momentum to find limp


% Regards CS more work needs to be done
% convergence analysis at correlated level using cheap CI or CC method
% look at formulation of qchem problems in terms of the exponent and not the energy
% often a good exponent for ground state and HF implies that order of
% orbitals / states change e.g. 4d and 4p before 4s

% Other types of CS basis sets,
% rohf rmsol analysis

application of cs to spectra
rydberg states and the liking

molecular sturmian


\subsection{Finite-element-based quantum chemistry}
With the lessons learned during the design phase of the present
contraction-based \SCF scheme,
another take on finite-element-based quantum chemistry
seems a sensible prospect.
This field as also moved forward quite a bit since
I last looked at it in 2014 and 2015, see for example \cite{Davydov2015,Boffi2016}.
A link to a finite-element library like \texttt{deal.ii}~\cite{Arndt2017}
would be required.
Test applicablity of \molsturm to grid-based approaches to quantum chemistry.

% TODO OPTIONAL
%Last but not least the ability to treat different kinds of basis functions
%in the same framework simplifies the construction of hybrid basis sets,
%where possibly numerical basis functions and GTOs or STOs are combined.
%Similarly one could employ the strengths of multiple backends for the same type of basis function
%in the sense that one mixes and matches different backends such that one can exploit
%the advantages of both implementations best.

\subsection{Fuzzing of integral back ends}
The common-ground \SCF of \molsturm where existing integral libraries
can be easily incorporated,
opens for the possibility to automatically test the reliability
of existing integral libraries by comparing the results of
random or semi-random \SCF calculations against another.
Due to our \python interface such a process could be completely automatised,
thus potentially leading to the automatic discovery of bugs in the compared libraries.
Such fuzzing approaches have already been applied with huge success
in the context of hardening security-critical software~\cite{Fuzzing}.
