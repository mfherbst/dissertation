\chapter{Prospects and future work}
\label{ch:Prospects}
\chapquote{%
Humanity needs practical men,
who get the most out of their work, and,
without forgetting the general good,
safeguard their own interests.
But humanity also needs dreamers,
for whom the disinterested development of an enterprise is
so captivating that it becomes impossible for them to devote
their care to their own material profit.}
{Marie Skłodowska Curie~(1867--1934)}

\noindent
With the availability of the \molsturm program package in its
current state, a flexible research tool
for the development and the investigation of novel quantum-chemical methods
has become available.
The possibility to easily extend the present functionality
by linking to existing third-party packages,
opens the door to rapidly test novel combinations of basis function types
and existing quantum-chemical methods.
In this way one could seek to find alternatives for cases,
which are challenging to describe using Gaussian-type basis functions.
Examples are the description of extended states,
potentially embedded in the continuum,
like autoionising or resonance states~\cite{Feshbach1958,Feshbach1962,Riss1993,Santra2002}
or the computation of properties involving a description
of the wave function close to the nuclei,
like nuclear magnetic resonance properties~\cite{Guell2008,Hoggan2009}.

From this respect interesting candidates are Sturmian-type basis functions,
which are exponential-type orbitals,
obtained as analytic solutions to Schrödinger-like equations.
Such functions are able to properly represent
the physical features of the wave function,
see section \ref{sec:BasisCS},
and they lead to feasible integrals
in the Hartree-Fock~(\HF) self-consistent field~(\SCF) procedure.
Furthermore they are complete
and thus able to model continuum-like states as well.
See section \ref{sec:newSturmian} for a more
detailed discussion with respect to
quantum-chemical calculations employing Sturmian-type orbitals.

Additionally, \molsturm is a framework with an interface
in which a novel method only needs to be implemented
once and can subsequently be
used with different types of basis functions.
This was already discussed in section \ref{sec:ex:ccd}
where a coupled-cluster doubles~(CCD) code building on top of \molsturm was shown.
Already at the present state, such a user code can
directly utilise all basis function types available in \molsturm,
\ie Coulomb-Sturmians as well as Gaussian-type orbitals,
but the performance is far from optimal.
Section \ref{sec:newMolsturm} provides some direction
how performance could be improved.

Another aspect of such a framework is
that links to different libraries and programs
providing the \emph{same} quantum-chemical methods can be achieved.
In this way implementations with deviating algorithmic details
can be compared directly.
One application of this would be to verify the correctness
of integral back ends, see section \ref{sec:Fuzzing} for details.

%
% -------------------------------------------------------------------------
%
\section{\molsturm program package}
\label{sec:newMolsturm}
After two years of development,
\molsturm is in a state,
where calculations based on contracted Gaussian basis sets
and Coulomb-Sturmian basis sets can be performed.
Furthermore, as mentioned in section \ref{sec:MolsturmState}
and demonstrated in chapter \ref{ch:CSQChem},
not only \HF, but full configuration interaction (\FCI) and
methods based on the algebraic-diagrammatic construction scheme (\ADC)
are available via interfaces to \pyscf~\cite{Sun2017} and \adcman~\cite{Wormit2014}.
For employing these Post-\HF methods
with more than around a hundred basis functions,
a drawback at the moment is performance
inside the \SCF procedure of \molsturm.

One reason for this is that the \SCF scheme,
which is currently used in \molsturm,
only consists of a few rather simple algorithms,
namely the Roothaan repeated diagonalisation~\cite{Roothaan1951},
the direct inversion in the iterative subspace~(\DIIS) algorithm~\cite{Pulay1982}
and the truncated optimal damping algorithm~(see section \ref{sec:tODA}).
More sophisticated schemes
like the energy \DIIS~\cite{Kudin2002}
or a second-order \SCF scheme~\cite{Salek2007,Hoest2008}
would be more efficient as well as more reliable.
For this the published schemes need to be adapted,
such that they fit into the contraction-based setting of \molsturm,
where the Fock matrix is not stored in memory,
but only employed in the form of a matrix-vector product,
see section \ref{sec:SCFAlgorithms}.
As discussed in section \ref{sec:OverviewSCF}
such modifications are always possible in theory,
but in practice one needs to be careful
that the introduced changes
keep the advantageous mathematical properties of such algorithms
with respect to convergence and stability.

Another aspect for improvement
is the lazy matrix library \lazyten,
which is a key component inside the contraction-based
self-consistent field~(\SCF) of \molsturm,
see section \ref{sec:MolsturmDesign}.
Whenever the Fock matrix is applied to a trial vector
inside the \SCF or the diagonalisation algorithm
employed by the \SCF,
a contraction expression is evaluated.
This proceeds by working on the expression tree,
which represents the Fock matrix,
see section \ref{sec:lazymat}.
In \lazyten this is currently neither parallelised,
nor is the expression tree optimised before the computation begins.
Both these aspects,
\ie automatic parallelisation of linear algebra expressions
as well as finding optimal evaluation schemes,
is ongoing research,
where, both in the context of quantum-chemical calculations
as well as a more general setting,
enormous progress has been made in recent years~\cite{Baumgartner2005,Solomonik2014,%
Peise2015,Calvin2015,Calvin2015arxiv,Xerus,Kristensen2016array,%
Kristensen2016streaming,Libtensor}.
By linking to such libraries
these advances could be incorporated or reused
inside \lazyten leading to performance improvements in the \SCF of \molsturm.


%For example a generalisation of \lazyten to lazy tensors
%would allow to use lazy tensors as data structures on the level of \molsturm's interface.
%Right now this is prevented by the \ERI tensor,
%which --- as indicated above --- cannot be represented as a lazy matrix,
%such that a costly step of copying all data from the internal
%representation to the data structures of the interface results.
%On the level of the \SCF the automatic parallelisation
%of the evaluation of the lazy matrix expression trees
%would help to improve the efficiency.

% TODO OPTIONAL
%\subsection{Reliable contraction-based self-consistent field algorithm}
%So far none of the \SCF algorithms available in \molsturm
%are perfectly reliable and frequently the implemented \SCF scheme
%amounts to find stationary states of the \HF problem,
%which are not the true \SCF minimum.
%Remedies could be to implement more sophisticated algorithms like
%the energy \DIIS~\cite{Kudin2002}
%or a second-order \SCF scheme~\cite{Salek2007,Hoest2008}
%inside the \contraction-based setting of \molsturm.

Additionally the modular structure of \molsturm
makes it fast to interface to
further quantum-chemistry libraries or program packages.
An effort worth pursuing is the \texttt{libxc}~\cite{Lehtola2018} library,
which offers a range of exchange-correlation functionals
for density-functional theory.
Implementing an interface to this library inside appropriate
lazy matrix objects would allow to construct the Kohn-Sham matrix
inside \molsturm's \SCF without any further changes,
such that density-functional theory calculations would become available.
Furthermore a better link to \pyscf~\cite{Sun2017}
would allow to perform configuration-interaction
and coupled-cluster calculations as well as calculations
employing the density matrix renormalisation group approach
directly from \molsturm.
%The modular structure of \psifour makes it another good candidate
%for extending the number of available methods
%on top of \molsturm's \SCF.
The prospect is to try other basis functions
in the context of such methods
and investigate their applicability with respect to quantum-chemical
calculations.

%On top of that data exchange with existing quantum-chemistry-related software
%could be simplified even further by supporting
%some wide-spread file formats in \molsturm
%for example \texttt{Molpro}'s FCIDUMP~\cite{Knowles1989},
%as well as routines for dumping plotting data
%in \texttt{Gaussian}~\cite{Frisch2016} cube or VTK~\cite{Avila2010} files.
%The \texttt{molden}~\cite{Schaftenaar2000} format
%could be supported by means of a fitting function
%representing results,
%which are not based on Gaussian-type orbitals in the form of a minimal
%Gaussian-type basis,
%similar to the way suggested in \cite{Stewart1970}
%for Slater-type orbitals.

%\section{Finite-element-based quantum chemistry}
%\label{sec:newFE}
%At the beginning of my doctoral studies I already attempted
%unsuccessfully to implement a \HF scheme based on finite-elements,
%a piecewise polynomial basis based on a real-space grid,
%see section \ref{sec:FE}.
%Recent advances of the field~\cite{Frediani2015,Toivanen2015,Davydov2015,Boffi2016}
%seem promising and
%implementing a grid-based basis into \molsturm
%would constitute an interesting test case with respect to the aspired generality
%of \molsturm's \SCF scheme.
%A link to the finite-element library \texttt{deal.ii}~\cite{Arndt2017}
%could provide the relevant back end to perform the computations.
%The scheme for evaluating the exchange matrix contraction
%expression \eqref{eqn:ExchangeFromPotentials},
%which was proposed in section \ref{sec:FE}
%has, however, never been implemented in practice,
%there are many open questions regarding the numerical details.
%Before one can proceed to perform a proper \SCF based on finite elements inside \molsturm,
%one therefore needs to investigate questions such as
%which quadrature scheme to employ,
%how many quadrature points are required,
%which finite-element order to use for the inner Poisson solves,
%what are sensible convergence thresholds and so on.
%All together the task remains challenging.

% TODO OPTIONAL
%Last but not least the ability to treat different kinds of basis functions
%in the same framework simplifies the construction of hybrid basis sets,
%where possibly numerical basis functions and GTOs or STOs are combined.
%Similarly one could employ the strengths of multiple backends for the same type of basis function
%in the sense that one mixes and matches different backends such that one can exploit
%the advantages of both implementations best.

%
% -------------------------------------------------------------------------
%
\section{Investigation of Sturmian-type discretisations}
\label{sec:newSturmian}
The original purpose of the \molsturm package has been to devise a program
which could be used for quantum-chemical calculations
employing Sturmian-type basis functions.
The package has outgrown this purpose in the current design,
but investigating the properties
of Sturmian-type basis functions is still of interest
as discussed above.

\subsection{Convergence properties of Coulomb-Sturmian basis sets}
The simplest example for Sturmian-type basis functions are Coulomb-Sturmians.
An initial investigation of such basis functions
in chapter \ref{ch:CSQChem} looked overall promising,
but the obtained results were not yet sufficient to provide definitive
construction schemes for Coulomb-Sturmian~(\CS) basis sets
or general estimates for the overall accuracy.
There are three ways this could be improved.

Firstly so far only main group elements of the second and third
period of the periodic table were considered.
Originating from the involvement of the $d$-orbitals
the properties of the electronic structures of the transition metals do, however,
differ compared to the main group elements.
An analysis with respect to the fourth period and beyond
is therefore required to reach more general conclusions
with respect to sensible \CS basis sets.

Secondly, most of the presented investigation has concentrated on the \HF
level with some minor modifications suggested mostly based on
second-order Møller-Plesset perturbation theory~({\MP}2).
Whilst these two methods are both used in electronic structure theory,
more methods should be added to reach a representative set.
Most notably {\MP}2 as a perturbative approach
for modelling electron correlation effects is very different from
configuration-interaction-based
or coupled-cluster-based approaches
--- both with respect to the way the physics is described
as well as the numerics.
Further convergence studies employing, for example, the latter kind of methods
would be required.
% TODO OPTIONAL
%\to doil{
%Additionally one should note
%that \MP approaches building on top of a unrestricted \HF ground state
%are known to have convergence issues~\cite{Knowles1985,Handy1985},
%such that perhaps a restricted open-shell \HF approach should be employed
%to describe the reference state for {\MP}2.
%}

Thirdly, the full flexibility towards constructing \CS basis sets has not yet
been exploited in the discussion in section \ref{sec:CSconvergenceHF}.
There is no reason why one should define a basis set
by limiting the angular quantum numbers $l$ and $m$ to the same maximum for all
principle quantum numbers $n$.
As mentioned in section \ref{sec:DenotingCSbasis} a \CS basis set can be any
combination of the quantum number triples $(n, l, m)$.
Since alternative construction schemes might allow to reduce
the required basis set size,
they are worth considering.
With respect to this it would also be interesting to compare the
observed convergence properties with the typical construction schemes
employed for contracted Gaussian~(\cGTO) basis sets~\cite{Jensen2013,Hill2013}.
Potentially the schemes employed in the \cGTO setting are applicable
to \CS basis sets and vice versa.

In general a detailed comparison of the results obtained
from employing Coulomb-Sturmians as well as \cGTO discretisations seems appropriate.
An interesting question is, for example, the required basis sizes for both
discretisation types to reach a certain accuracy
in the description of the ground or excited-state energies
at various levels of theory.

\subsection{Coulomb-Sturmian-based excited states calculations}
\label{sec:SturmianExcited}
In section \ref{sec:SturmianADC} first results
for computing excited state energies of atoms based on \ADC
were already presented.
A more detailed analysis of the convergence properties
could help to proceed with the application of \CS-based \ADC
calculations in order to compute the spectra of atoms.
Due to the completeness of the Coulomb-Sturmians
and their possibility to describe both core region as well as
the exponential decay,
a range of interesting applications for the modelling
of excitation processes come to mind.
Three of them are
(1) methods where continuum-like states needs to be modelled,
like Fano-Stieltjes~\cite{Feshbach1958,Feshbach1962,Santra2002},
(2) cases where modelling both the core and the valence shell is required,
like core-valence excitations~\cite{Wenzel2014,Wenzel2016}
as well as
(3) the modelling of expanded bound states,
like the determination of Rydberg-like states~\cite{Kaufmann1989,Riss1993}.

\subsection{Avoiding the Coulomb-Sturmian exponent as a parameter}
An unfavourable aspect of the \CS basis sets
employed in chapter \ref{ch:CSQChem} is
the Coulomb-Sturmian exponent $\kexp$.
As was discussed in section \ref{sec:kexp},
this parameter has indeed an influence on the results obtained.
For example the obtained \SCF minimum could be unphysical,
\ie with occupied orbitals of positive energies,
if $\kexp$ is not chosen in the vicinity of the optimal exponent,
which is the one yielding the lowest possible \HF ground state energy.
Furthermore, there is some indication
that the ordering of excited states in \ADC calculations depends on $\kexp$.
Using the algorithm described in section \ref{sec:DetermineKopt},
a route for finding an optimal value at \HF level has been sketched.
With respect to excited states methods like \ADC,
it is not immediately obvious how to determine the most optimal exponent,
since the value for describing the ground state with the least error
will differ from the value for representing a particular excited state best.
In turn each excited state will have a different $\kexp$
to give the best description in a particular \CS basis.
Which value or which combination of the values should be taken
is not directly clear.

\noindent
An equivalent problem at \FCI level can be avoided.
The reason is that the relationship
\[ E = - \frac{\Nelec \kexp^2}{2}, \]
between the \CS exponent $\kexp$, the number of electrons $\Nelec$
and the energy $E$ of a particular state,
can be employed to re-formulate \FCI in terms of the
Coulomb-Sturmian exponents~\cite{Avery2006}.
In other words instead of solving for the energy of a state,
one solves for the $\kexp$ for each state
and uses this value both to find the corresponding energy
as well as the exponent of the basis functions,
when properties for such a state are to be computed.
A similar reformulation should be possible for \HF
and potentially even for some other Post-\HF methods as well,
even though this is uncertain at the moment.
If this could be achieved both the determination of an optimal
$\kexp$ would become obsolete at \HF level
and for excited states $\kexp$ would adapt automatically to the required state.

\subsection{Molecular Sturmians}
\label{sec:MolecularSturmian}
The implementation of a \CS-based \SCF scheme was always intended
to be only the first step with respect to the exploration
of Sturmian-based quantum-chemical calculations.
More general and at the same time more challenging types of Sturmian basis
functions exist,
which are able to describe molecular systems, for example.
Building on recent advances in the calculation of the \ERI integrals
\eqref{eqn:ERI} for such generalised Sturmian-type orbitals%
~\cite{Avery2006,Avery2011PhD,Avery2011,Morales2016,Avery2017,Randazzo2015,Granados2016}
a Sturmian-based \HF suitable for molecular calculations is within reach
and could be implemented within \sturmint~\cite{sturmintWeb}
building on already existing infrastructure required for the Coulomb-Sturmians.
This would make allow for performing calculations
based on Sturmian-type basis functions for molecules as well.


%
% -------------------------------------------------------------------------
%
\section{Fuzzing of integral back ends}
\label{sec:Fuzzing}
The common interface, which \molsturm provides
for accessing the implemented integral libraries,
allows to test the correctness
of the algorithms employed inside these libraries
by comparing the results of
random or semi-random input against each other.
Due to the \python interface of \molsturm this process could
even be completely automated.
Such fuzzing approaches have already been applied with huge success
in the context of hardening security-critical software~\cite{Fuzzing}.
With respect to quantum-chemical software
a similar work by \citet{Knizia2011},
which tested the hardness of quantum-chemical software
with respect to numerical instabilities using random noise,
lead to the discovery of unexpected bugs
in the integral evaluation scheme of \texttt{Molpro}~\cite{Molpro},
justifying a closer look at this subject.
