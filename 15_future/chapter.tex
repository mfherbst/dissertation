\chapter{Conclusion and future work}
\label{ch:Conclusion}
\chapquote{%
	We must include in any language with which we hope
	to describe complex data-processing situations
	the capability for describing data.
}{Grace Hopper~(1906--1992)}

% TODO Some conclusions?


\todo[inline,caption={}]{
	\begin{itemize}
		\item Conclusion, outlook and future work
		\item Lazyten generalisation
			\begin{itemize}
				\item The update function already indicates that in reality something more general is needed.
			\end{itemize}
		\item Automatic tensor evaluation order
		\item Sturmian potential
	\end{itemize}
}


% Other types of CS basis sets,
% motivate that from Gaussians a little


% molsturm
A good improvement for \molsturm on this front would be a cheap,
but nevertheless effective \contraction-based \SCF scheme.
%
%
A disadvantage of the current \molsturm interface is,
that all data is stored as plain and dense memory.
Both symmetry as well as sparsity properties of the involved tensors
are entirely ignored at the moment.
This yields an unacceptably large requirement in memory
as well as a very unfavourable computational scaling
for the demanding tensor contractions
typically required in Post-HF methods like \CCD (see figure \ref{fig:codeCCD}).
In the future we want to tackle this by extending
\lazyten towards lazy tensors as well.
In practice this would imply that the evaluation of
tensor contractions like in figure \ref{fig:codeCCD} could be implicitly delayed
for as long as possible.
Only once the results are needed in the code,
the calculation would be underdone
for all delayed expressions at once.
This could allow to automatically deduce an ideal evaluation strategy
taking all sparsity and symmetry information
of the involved tensors into account.
Ideally this would happen transparently in the background
without changing the \python interface at all.

Another angle for future work would be to integrate more closely
with existing quantum-chemistry libraries or program packages.
A good example is the \texttt{libxc}~\cite{Lehtola2018} library,
which offers a range of exchange-correlation functionals.
Implementing an interface to this library inside appropriate
lazy matrix objects would allow to construct the Kohn-Sham matrix
inside \molsturm's \contraction-based \SCF as well,
such that density-functional theory~(DFT) calculations
could be performed.
Similarly we would like to simplify
data exchange with other quantum chemistry packages like \pyscf or \psifour
by supporting some wide-spread file formats in quantum chemistry,
\eg \texttt{MolPro}'s FCIDUMP~\cite{Knowles1989}.
For facilitating plotting of \SCF orbitals or densities
we plan to implement the \texttt{molden}~\cite{Schaftenaar2000} format
as well as routines for dumping the data
in \texttt{Gaussian}~\cite{Frisch2016} cube or VTK~\cite{Avila2010} files.

%
%
Improve stabilities of \SCF scheme $\Rightarrow$ Leads to better procedure for finding $\kopt$
%
improvements on basis set sizes my limiting $\mmax$ or not keeping the complete $\lmax$ everywhere
Investigate usefulness of Sturmians for ADC
Comparison to cGTO basis sets
MP2 not a great method for convergence studies (e.g. not variational with respect to FCI solution), so better to rely on CISD or CC methods
play around with more complex CS basis sets
replace rmsol by maximum coefficient per angular momentum to find limp

%
% Hybrid stuff
%
Last but not least the ability to treat different kinds of basis functions
in the same framework simplifies the construction of hybrid basis sets,
where possibly numerical basis functions and GTOs or STOs are combined.
Similarly one could employ the strengths of multiple backends for the same type of basis function
in the sense that one mixes and matches different backends such that one can exploit
the advantages of both implementations best.
