\chapter*{Acknowledgements}
\addcontentsline{toc}{chapter}{Acknowledgements}
\todoil{Draft version}

First and foremost I would like to thank those people,
which provided me with a constant source of advice and support
throughout the recent years,
namely my supervisor Prof. Andreas Dreuw,
my co-supervisor Prof. Guido Kanschat
as well as my advisor Dr. James Avery.
Andreas, thanks for giving me the choice to work on such
a highly interdisciplinary topic
and having presented me with the freedom and trust
to take the original topic and extended to what the thesis became now.
Guido, thanks for the many enlightening discussions
about finite elements,
numerical linear algebra and numerics in general,
for the occasional outspoken comment and criticism
leading to investing second thought.
James, thanks for introducing me to Coulomb-Sturmians
and giving me the chance to work with you on this
exciting topic
as well as the many invitations to Copenhagen
and for showing me Heisenberg's bath tub.

Next I wish to express my gratitude to Dr. Michael Wormit,
with whom I had the pleasure to work with
albeit for an unfortunately too short period of time.
It brings me a consoling thought that some of his
ideas managed to yield fruit in this thesis after all.

Group of Prof. Brian Vinter for hosting me twice
during my visits in Copenhagen
During my time there I quickly felt part of the group
and enjoyed the discussions in the Monday morning breakfast meetings
as well as the friday afternoon beers.
I greatefully acknowledge the computational time at the octuplet cluster at
the eScience group of Køpenhavns Universitet, which was used to carry out
most of the calculations in chapter 8

A fascinating aspect of quantum chemistry is the interdisciplinarity of the field.
Developing a good quantum-chemical model for a particular aspect of chemistry
firstly requires both chemical as well as physical intuition.
But in many cases the arising problems are numerically challenging
and for solving them in an efficient and reliable manner
one cannot avoid taking a deeper look into the mathematical structures
as well as modern concepts from high-performance computing,
hardware and operating systems design.
During the course of my PhD I was therefore very
happy for fruitful impulses from different people and researchers
not limited to a particular field.
I am grateful for the enlightening discussions with
Prof. Eric Cances
Prof. Reinhold Schneider
Henrik Larsson
Adrian Dempwolff
Dr. Tim Stauch
Dr. Dirk Rehn
Manuel Hodecker
Maximilian Scheurer
Dr. Mads Kristensen
Dr. Jenny Wagner
Lucas Fabian Hackl
Jan Janßen
Furthermore I would like to acknowledge the open and curious
environment of the hacker community with endless nights
of discusisons about science, politics,
society, programming and just general nerdiness.
In alphabetical order I would like to give a shout at 
cherti
hauro
janx2
kungi
rami
supaake
and all the other guys from the NoName e.V.

I thank the many people how looked at earlier versions
of this thesis and provided helpful comments for correcting them
Fabian Faulstich
Dr. Jenny Wagner
Adrian Dempwolff

I thank the 
Heidelberg University
University of Kopenhagen
KTH Stockholm
Mathematisches Forschungsinstitut Oberwolfach
for hosting me
I acknowledge the
financial support from the Heidelberg Graduate School
for the Mathematical and Computational Methods for the Sciences
and for the opportunity to take action myself
in the form of organising
the doctoral conference or giving lectures.

People who wrote the software I use very day
This thesis could not have been made without the people
developing \TeX, \LaTeX, \python, \numpy, \texttt{gcc}, \texttt{clang}, \texttt{vim}
as well as the best GNU/LinuX distribution in the world,
namely Debian.

Past and present members of the Dreuw group,
most notably bla

as well as Ellen Vogel and Manfred Trunk for dealing with the
administrational stuff as well as the computer.

Jan und Lucas

I would like to thank Henrik Larsson
for an ongoing friendship leading to afternoons of scientific
debate
and the occasional week of teaching younger pupils
quantum chemistry together.

Thanks to all the great people I met
at the Workshop in heidelberg,
in gelsenkirchen
at the DSA and at all 

Thanks to Jan-Christoph Peters, Peter Schwalb,
Jan Janßen, Lucas Fabian Hackl,
and all the other friends, which have accompanied me for many years.


Än große Dank geht aach an mei ganz Bagage vun dehäm.
Auch wenn ich  mach ämol weider fort bin,
hängt mei Herz doch immer an eich.
Mei Vetter Sven unn Oliver,
Michael Junker, Markus Junker mit em Jun.
Mit eich ebbes zu unnernemme oder äfach blos zu babble
des war oft ä Quell von Kraft fer die mehr anstrende Daage.
Vielmols merci sag ich dem Sonja und dem Jim.
Vun alle denne die wichtichst
sinn sicherlich des Ingrid und de Ortwin Herbst,
mei liewe Eltre dene ich so viel verdanke,
dass Wordde des net im entferndeschde fasse känne.

Finally and above all I thank my beloved fianc\'ee Carine
for every minute we have been, are and will be together.
