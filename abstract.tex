\defineabbr{HF}{HF\xspace}{Hartree-Fock}
\defineabbr{SCF}{SCF\xspace}{Self-consistent field approach}

%
% English
%
\chapter*{Abstract}
\phantomsection
\markboth{\leftheadstyle{Abstract}}{}
\addcontentsline{toc}{chapter}{Abstract}

For the calculation of electronic structures of molecules
state-of-the-art methods predominantly use Gaussian basis functions.
The algorithms employed inside existing code packages
are consequently often highly optimised keeping
only their numerical requirements in mind.
For the investigation of novel approaches,
utilising other basis functions, this is an obstacle,
since requirements might differ.
In contrast, this thesis develops the light-weight program package \molsturm,
which is designed in order to support
Hartree-Fock~(\HF) self-consistent field~(\SCF) calculations
based on many basis function types.
Additionally it is demonstrated how \molsturm
can be easily linked to third-party code,
both on the level of the integral libraries as well as the Post-\HF stage,
thus allowing to leverage existing functionality as much as possible.

In order to arrive at this point,
the mathematical background of quantum mechanics
as well as some numerical techniques are reviewed.
Care is taken to emphasise the often overlooked subtleties when
discretising an infinite-dimensional spectral problem
in order to obtain a finite-dimensional eigenproblem.
Common quantum-chemical methods such as full configuration interaction
and \HF are discussed providing insight
into their mathematical properties.
Different formulations of \HF are contrasted
and appropriate \SCF solution schemes formulated.

Next discretisation approaches based on four different types of basis functions
are compared
both with respect to the computational challenges as well as
their ability to describe the physical features of the wave function.
Next to (1) Slater-type orbitals and (2) Gaussian-type orbitals,
the discussion considers (3) finite elements,
which are piecewise polynomials on a real-space grid,
as well as (4) Coulomb-Sturmians,
which are the analytical solutions to a Schrödinger-like equation.
A novel algorithmic approach based on matrix-vector contraction expressions is developed,
which is able to incorporate the numerical requirements of all cases considered.
It is shown that this ansatz not only allows to formulate
\SCF algorithms in a basis-function independent way,
but furthermore improves the theoretically achievable
computational scaling for finite-element-based discretisations
as well as performance improvements for Coulomb-Sturmian-based discretisations.
The adequacy of standard \SCF algorithms with respect to a contraction-based
setting is reviewed
and for the example of the optimal damping algorithm
an approximate modification to achieve such a setting is presented.

With respect to recent trends in the development of modern computer hardware
the potentials and drawbacks of contraction-based approaches are evaluated.
One drawback, namely the typically more involved and harder-to-read code,
is identified and a data structure named lazy matrix is introduced
to overcome this.
Lazy matrices are a generalisation of the usual matrix concept,
suitable for encapsulating contraction expressions.
Such objects still look like matrices from the user perspective,
including the possibility to perform operations like matrix sums and products.
As a result programming contraction-based algorithms
becomes similarly convenient to working with normal matrices.
An implementation of lazy matrices in the \lazyten linear algebra library
is presented followed by examples
demonstrating the applicability in the context of the \HF problem.

Building on top of the aforementioned concepts the design of \molsturm is outlined.
It is shown how a combination of lazy matrices and a contraction-based \SCF scheme
separates the code describing the \SCF procedure
from the code dealing with the basis function type.
It is discussed how this allows to add a new basis function type
to \molsturm by only making code changes in a single integral interface library.
On top of that we demonstrate by the means of examples
how the readily scriptable interface of \molsturm
can be employed to implement and assess novel quantum-chemical methods
or to combine the features of \molsturm with existing third-party packages.

Finally the thesis discusses an application of \molsturm
towards the investigation of the convergence properties
of Coulomb-Sturmian-based quantum-chemical calculations.
First results for the convergence
of the ground-state energies at \HF level
are reported for atoms of the second and the third period
of the periodic table.
Particular emphasis is put on a discussion about the required
maximal angular momentum quantum numbers
in order to achieve convergence
of the discretisation of the angular part of the wave function.
Some modifications required for a treatment at correlated level are suggested,
followed by a discussion of the effect of the Coulomb-Sturmian exponent.
An algorithm for obtaining an optimal exponent is presented
and some optimal exponents for the atoms of the
second and the third period of the periodic table at \HF level are given.
Furthermore the first results of a Coulomb-Sturmian-based excited
states calculation based on the algebraic-diagrammatic construction
scheme for the polarisation propagator are presented.


%
% German
%
\chapter*{Zusammenfassung}
\phantomsection
\addcontentsline{toc}{chapter}{Zusammenfassung}
\markboth{\leftheadstyle{Zusammenfassung}}{}
\todoil{Draft, still needs to be formulated better}

Für die Berechnung elektronischer Zustände in Molekülen
verwenden aktuelle Methoden vor allem Gaußfunktionen als Basisfunktionen.
Die Algorithmen, welche in den bestehenden Quantenchemiepaketen
verwendet werden sind dementsprechend oft sehr stark
auf jene zugeschnitten,
wobei sie lediglich die numerischen Anforderungen jeder im Blick behalten.
Dies ist ein Hindernis für die Betrachtung neuartiger Methoden,
welche andere Basisfunktionen verwenden wollen,
da die Anforderungen unterschiedlich sein können.
Im Gegensatz  dazu entwickelt diese Doktorarbeit
das schlanke Programmpacket \molsturm,
wessen Struktur explizit so entwickelt wurde,
dass es Berechnungen
basierend auf der Hartree-Fock~(\HF) Methode der selbstkonsistenten Felder~(\SCF)
basierend auf mehreren Basisfunktionstypen durchführen kann.
Weiter wird gezeigt wie \molsturm
auf einfache Weise mit bestehenden anderen Programmen
verbunden werden kann,
sowohl auf der ebene der Integralbibliotheken als auch auf der ebene der Post-\HF
Methoden.
Auf diesem Weg kann bestehende Funktionalität anderer Programme
soweit als möglich weiter eingesetzt werden.

Zunächst wird der mathematische Hintergrund der Quantenmechanik
und einige numerische Techniken rezensiert.
Dabei wird insbesondere auf die oft unterschlagenen Feinheiten
eingegangen,
welche beim diskretisieren eines unendlich-dimensionalen Spektralproblemes
zu einem endlich-dimensionalen Eigenproblem entstehen.
Häufig verwendete quantenchemische Methoden
wie die vollständige Interaktion von Konfigurationen~(full configuration interaction)
oder \HF werden diskutiert
und dabei wird besonders auf deren mathematische Eigenschaften
eingegangen.
Unterschiedliche Formulierungen von \HF werden miteinander verglichen
und passende \SCF Lösungsagorithmen angegeben.

Im weiteren werden Diskretisierungsansätze basierend auf vier verschiedenen
Basisfunktionstypen miteinander verglichen,
wobei sowohl auf Herausforderungen in Bezug auf die Berechnung
der entstehenden Ausdrücke
als auch auf die Fähigkeit der Basen
die physikalischen Eigenschaften der Wellenfunktion zu beschreiben.
Neben (1) Orbitalen vom Slatertyp und (2) Gaußorbitalen
behandelt die vorgestelle Diskussion
(3) finite Elemente,
abschnittsweise Polynome, welche auf einem Gitter definiert sind,
sowie (4) Coulomb-Sturmfunktionen~(Coulomb-Sturmians),
welche die analytischen Lösungen einer der Schrödingergleichung
ähnlichen Differenzialgleichung sind.
Ein neuartiger Algorithmus basierend auf Matrix-Vektor-Kontraktionsausdrücken
wird entwickelt,
welcher die numerischen Anforderungen aller betrachteten Fälle abdecken kann.
Es wird gezeigt, dass mittels dieses Ansatzes
nicht nur \SCF Algorithmen in einer basisfunktionsunabhängigen Weise
formuliert werden können,
sondern ebenfalls,
dass die theoretisch mögliche algorithmische Komplexität
für Diskretisierungen basierend auf finiten Elementen
dadurch reduziert werden kann.
Weiterhin wird gezeigt,
dass damit
Effizienzverbesserungen für Diskretisierungen basierend auf Coulomb-Sturmfunktionen
möglich sind.
Die Eignung üblicherweise verwendeter \SCF Algorithmen
auf eine derartige, kontraktionsbasierte~(contraction-based)
Formulierung wird geprüft
und für das Beispiel des Algorithmus der optimalen Dämpfung~(optimal damping algorithm)
wird eine näherungsweise Änderung vorgeschlagen,
die diesen auf die kontraktionsbasierte Formulierung bringt.

In Bezug auf die jüngsten Entwicklungen bei moderner
Computerhardware
wird auf die Möglichkeiten und Nachteile
von kontraktionsbasierten Methoden eingegangen.
Ein Hindernis bei der Entwicklung kontraktionsbasierter Methoden
ist oft, dass die daraus entstehenden Programmcodes
oft schwerer lesbar sind.
Um dieses Problem zu umgehen wird die Datenstruktur
einer bequeme Matrix~(lazy matrix) eingeführt.
Bequeme Matrizen sind eine Verallgemeinerung des üblichen Matrixkonzeptes,
welche als eine Art Behältnis für Kontraktionsausdrücke aufgefasst
werden können.
Aus Sicht eines Benutzers dieser Datenstruktur
sehen diese allerdings immer noch wie gewöhnliche Matrizen aus,
sodass die üblichen Summenoperationen oder Produktoperationen
in gleicher Weise wie bei gewöhnlichen Matrizen verwendet werden können.
Das Ergebnis ist,
dass man kontraktionsbasierte Algorithmen auf die gleiche Art und Weise
implementieren kann,
wie man es mit gewöhnlichem Matrizen auch würde.
Eine Implementierung des Konzepts der bequemen Matrizen in der
Bibliothek \lazyten wird vorgestellt.
Ebenso werden Beispiele gezeigt,
welche die Eignung von bequemen Matrizen für das \HF Problem demonstrieren.

Basierend auf den oben erwähnten Konzepten wird die Programmstruktur
von \molsturm diskutiert.
Es wird dargelegt wie durch eine Kombination der bequemen Matrizen
und einem kontraktionsbasierten \SCF-Verfahren
der Programmcode,
welcher das \SCF-Verfahren selbst beschreibt,
von jedem Programmcode, welcher die Diskretisierung und die Basisfunktionen
betrifft,
getrennt wurde.
Im weiteren wird darauf eingegangen,
wie ein neuer Basisfunktionstyp
in \molsturm implementiert werden kann,
sodass nur Änderungen an einer einzigen Stelle notwendig sind.
Darüber hinaus wird mittels einiger Beispiele besprochen,
wie über eine einfache, skriptbare~(scriptable) Schnittstelle
neuartige Quantenchemiemethoden
einfach implementiert und getestet werden können
oder wie \molsturm mit der Funktionalität in bestehenden Programmen Dritter
verbunden werden kann.

Zuletzt wird eine Anwendung von \molsturm
auf eine Untersuchung der Konvergenzeigenschaften
von Quantenchemierechnungen basierend auf Coulomb-Sturmfunktionen
vorgestellt.
Erste Ergebnisse für die Konvergenz der Grundzustandsenergie auf \HF-Niveau
werden für die Atome der zweiten und dritten Periode
des Periodensystems vorgestellt.
Im besonderen wird auf die maximal notwendige
Drehimpulsquantenzahl eingegangen
um eine Konvergenz des Winkelanteils der Wellenfunktion
zu erreichen.
Einige Anpassungen
für die Beschreibung mittels Korrelationsmethoden werden vorgestellt,
gefolgt von einer Diskussion des Effektes des Exponenten
der Coulomb-Sturmfunktionen.
Ein Algorithmus um den optimalen Exponenten zu finden
wird vorgestellt und einige optimale Exponenten für die Beschreibung der Atome
der zweiten und dritten Periode des Periodensystems
auf \HF-Niveau werden vorgestellt.
Des weiteren werden erste Ergebnisse einer Berechnung angeregter
Zustände mittels des
algebraisch-diagrammatischen Konstruktionsschemas
für den Polarisationspropagator
basierend auf Coulomb-Sturmfunktionen vorgestellt.
