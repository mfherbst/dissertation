\defineabbr{HF}{HF\xspace}{Hartree-Fock}
\defineabbr{SCF}{SCF\xspace}{Self-consistent field approach}

%
% English
%
\chapter*{Abstract}
\markboth{\leftheadstyle{Abstract}}{}
\addcontentsline{toc}{chapter}{Abstract}

For the calculation of electronic structures of molecules
state-of-the-art methods predominantly employ Gaussian basis functions.
The algorithms employed inside existing code packages
are consequently often highly optimised keeping
only their numerical requirements in mind.
For the investigation of novel approaches
utilising other basis functions, this is an obstacle,
since requirements might differ.
In contrast, this thesis develops the light-weight program package \molsturm,
which is designed in order to support
Hartree-Fock~(\HF) self-consistent field~(\SCF) calculations
based on many basis function types.
Additionally it is demonstrated how \molsturm
can be easily linked to third-party code,
both on the level of the integral libraries as well as the Post-\HF stage,
thus allowing to leverage existing functionality as much as possible.

In order to arrive at this point,
the mathematical background of quantum mechanics
as well as some numerical techniques are reviewed.
Care is taken to emphasise the often overlooked subtleties when
discretising an infinite-dimensional spectral problem
in order to obtain a finite-dimensional eigenproblem.
Common quantum-chemical methods such as full configuration interaction
and \HF are discussed providing insight
into their mathematical properties.
Different formulations of \HF are contrasted
and appropriate \SCF solution schemes formulated.

Next discretisation approaches based on four different types of basis functions
are compared
both with respect to the computational challenges as well as
their ability to describe the physical features of the wave function.
Next to (1) Slater-type orbitals and (2) Gaussian-type orbitals,
the discussion considers (3) finite elements,
which are piecewise linear polynomials on a real-space grid,
as well as (4) Coulomb-Sturmians,
which are the analytical solutions to a Schrödinger-like equation.
A novel algorithmic approach based on matrix-vector contraction expressions is suggested,
which is able to incorporate the numerical requirements of all cases considered.
It is shown that this ansatz not only allows to formulate
\SCF algorithms in a basis-function independent way,
but furthermore improves the theoretically achievable
computational scaling for finite-element-based discretisations
as well as performance improvements for Coulomb-Sturmian-based discretisations.
The adequacy of standard \SCF algorithms with respect to a contraction-based
setting is reviewed
and for the example of the optimal damping algorithm
an approximate modification to achieve such a setting is presented.

With respect to recent trends in the development of modern computer hardware
the potentials and drawbacks of contraction-based approaches are evaluated.
One drawback, namely the typically more involved and harder-to-read code,
is identified and a data structure named lazy matrix is introduced
to overcome this.
Lazy matrices are a generalisation of the usual matrix concept,
suitable for encapsulating contraction expressions.
Such objects still look like matrices from the user perspective,
including the possibility to perform operations like matrix sums and products.
As a result programming contraction-based algorithms
becomes similarly convenient to working with normal matrices.
An implementation of lazy matrices in the \lazyten linear algebra library
is presented followed by examples
demonstrating the applicability in the context of the \HF problem.

Building on top of the aforementioned concepts the design of \molsturm is outlined.
It is shown how a combination of lazy matrices and a contraction-based \SCF scheme
separates the code describing the \SCF procedure
from the code dealing with the basis function type.
It is discussed how this allows to add a new basis function type
to \molsturm by only making code changes in a single integral interface library.
On top of that we demonstrate by the means of examples
how the readily scriptable interface of \molsturm
can be employed to implement and assess novel quantum-chemical methods
or to combine the features of \molsturm with existing third-party packages.

Finally the thesis discusses an application of \molsturm
towards the investigation of the convergence properties
of Coulomb-Sturmian-based quantum-chemical calculations.
First results for the convergence
of the Hartree-Fock ground-state energies at \HF level
are reported for atoms of the second and the third period
of the periodic table.
Particular emphasis is put on a discussion about the required
maximal angular momentum in order to achieve convergence
of the discretisation of the angular part of the wave function.
Some modifications required for a treatment at correlated level are suggested,
followed by a discussion of the effect of the Coulomb-Sturmian exponent.
An algorithm for obtaining an optimal exponent is presented
and some optimal exponents for the atoms of the
second and the third period of the periodic table at \HF level are given.
Furthermore the first results of a Coulomb-Sturmian-based excited
states calculation based on the algebraic-diagrammatic construction
scheme for the polarisation propagator are presented.


%
% German
%
\chapter*{Zusammenfassung}
\addcontentsline{toc}{chapter}{Zusammenfassung}
\markboth{\leftheadstyle{Zusammenfassung}}{}
\todo[inline]{TODO}
