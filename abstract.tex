\defineabbr{HF}{HF\xspace}{Hartree-Fock}
\defineabbr{SCF}{SCF\xspace}{Self-consistent field approach}

%
% English
%
\chapter*{Abstract}
\markboth{\leftheadstyle{Abstract}}{}
\addcontentsline{toc}{chapter}{Abstract}
\todoil{Draft version}

State-of-the art methods for calculating the electronic structures
of molecules predominantly employ Gaussian-type basis functions.
Existing code packages therefore tend to be highly optimised
towards such a basis, assuming its numerical requirements throughout.
Different approaches, making use of other types of basis functions,
may show deviating mathematical properties
and implementing them inside the same program package
may thus become challenging.
On the other hand starting a new package from scratch
is time-consuming and potentially wasteful,
especially if the novel method does not turn out to be fruitful in practice.

In contrast, this thesis develops
a design for a light-weight
quantum-chemical program package,
which focuses on obtaining a general Hartree-Fock~(\HF)
self-consistent field~(\SCF) scheme on the one hand
as well as a readily usable interface on the other.
In this way both the flexibility to support calculations
based on many basis function types in one package is gained
as well as the possibility to link to third-party
libraries for performing Post-\HF methods on top of the results.

In order to arrive at this point,
first the mathematical background of quantum mechanics
as well as the Ritz-Galerkin projection for a variational treatment
of spectral problems is discussed.
Care is taken to emphasise the often overlooked peculiarities when
discretising an infinite-dimensional spectral problem
in order to obtain a finite-dimensional eigenproblem.
Common quantum-chemical methods such as Full-CI
and Hartree-Fock are reviewed providing insight
into their mathematical properties.
By employing the Ritz-Galerkin projection different
formulations of the Hartree-Fock approximation are discretised
leading to an expression for the Fock matrix,
which involves integrals over products of the basis functions
employed in the discretisation.
% Additionally the failures of Hartree-Fock as well as
% common measures to correct these are discussed.

In a following case study the properties of four types of basis functions
with respect to the feasibility of the integrals,
the structure of the Fock matrix
and the ability to represent the physical features of the wave function are contrasted.
Next to Slater-type orbitals and contracted Gaussian-type orbitals,
the discussion considers finite elements,
which are piecewise linear polynomials on a real-space grid,
as well as Coulomb-Sturmians,
which are the analytical solution to a Schrödinger-like equation.
Special demands with respect to the arising numerical problems
are emphasised and similarities elaborated.
Originating from recent advances in the context of finite elements
a contraction-based ansatz is suggested as an algorithmic approach
to deal with each of them on the same level.
The underlying idea of this ansatz is to avoid the storage of the Fock matrix,
but instead construct algorithms,
which only require the computation of the product of the Fock matrix with
arbitrary vectors by the means of a carefully constructed contraction expression.
It is shown that such an approach does not only allow to formulate
an \SCF algorithm which can deal with each of the discussed types
of basis functions,
but furthermore even improve the theoretically achievable
computational scaling for finite-element-based discretisations
as well as Coulomb-Sturmian-based discretisations.
Thereafter the ability of standard \SCF algorithms to support
a contraction-based setting is discussed and
for the example of the optimal damping algorithm
an approximate scheme is constructed,
which allows to carry some features of this algorithm to the
contraction-based setting.

A detailed discussion of contraction-based methods
along with their potential and caveats is presented thereafter.
With reference to trends in the development of modern computer hardware
it is indicated that the implied re-computation of matrix
data in each application of the contraction expression
is not only no problem but furthermore enables to properly adapt to such trends.
To provide an intuitive interface to deal with contraction-based methods,
concept of lazy matrices is introduced,
which generalises the usual matrix concept towards data structures
suitable for encapsulating contraction expressions.
This allows to program contraction-based
methods in a convenient way.
An implementation of this concept in the \lazyten linear algebra
library is presented followed by examples
demonstrating the applicability in the context of the Hartree-Fock problem.

% TODO Clearify the link to the second paragraph
Afterwards we discuss the design of the quantum-chemical method
development framework \molsturm,
which combines both the concept of lazy matrices
as well as a contraction-based \SCF procedure
in order to yield a clear separation between the code describing
the \SCF algorithm as well as the code dealing with the basis function type.
It is discussed how this allows to add a new basis function type
by only making code changes in a single integral interface library.
On top of that we demonstrate the means of examples
how the readily scriptable interface of \molsturm
can be employed to implement new quantum-chemical methods
or combine the features of \molsturm with existing third-party packages.

The final part of the thesis discusses an application of \molsturm
for an initial investigation of the convergence properties
of Coulomb-Sturmian-based quantum-chemical calculations.
The convergence at Hartree-Fock level is considered
in detail for the atoms of the second and the third period of the periodic table
with particular emphasis on the required
maximal angular momentum in order to achieve convergence
of the discretisation of the angular part of the wave function.
Some modifications required for a treatment at correlated level are suggested,
followed by a discussion of the effect of the Coulomb-Sturmian exponent.
A routine for obtaining an optimal exponent is presented
and optimal exponents for the atoms of the
second and the third period of the periodic table at \HF level in various
Coulomb-Sturmian basis sets are given.
Furthermore the first result of a Coulomb-Sturmian-based excited
states calculation based on the algebraic-diagrammatic construction
scheme for the polarisation propagator is presented.


%
% German
%
\chapter*{Zusammenfassung}
\addcontentsline{toc}{chapter}{Zusammenfassung}
\markboth{\leftheadstyle{Zusammenfassung}}{}
\todo[inline]{TODO}
