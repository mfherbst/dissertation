\defineabbr{HF}{HF\xspace}{Hartree-Fock}
\defineabbr{SCF}{SCF\xspace}{Self-consistent field approach}

%
% English
%
\chapter*{Abstract}
\phantomsection
\markboth{\leftheadstyle{Abstract}}{}
\addcontentsline{toc}{chapter}{Abstract}

State-of-the-art methods
for the calculation of electronic structures of molecules
predominantly use Gaussian basis functions.
The algorithms employed inside existing code packages
are consequently often highly optimised keeping
only their numerical requirements in mind.
For the investigation of novel approaches,
utilising other basis functions, this is an obstacle,
since requirements might differ.
In contrast, this thesis develops
the highly flexible program package \molsturm,
which is designed in order to facilitate rapid design,
implementation and assessment
of methods employing different basis function types.
A key component of \molsturm is
a Hartree-Fock~(\HF) self-consistent field~(\SCF) scheme,
which is suitable to be combined
with any basis function type.

First of all
the mathematical background of quantum mechanics
as well as some numerical techniques are reviewed.
Care is taken to emphasise the often overlooked subtleties when
discretising an infinite-dimensional spectral problem
in order to obtain a finite-dimensional eigenproblem.
Common quantum-chemical methods such as full configuration interaction
and \HF are discussed providing insight
into their mathematical properties.
Different formulations of \HF are contrasted
and appropriate \SCF solution schemes formulated.

Next discretisation approaches based on four different types of basis functions
are compared
both with respect to the computational challenges as well as
their ability to describe the physical features of the wave function.
Besides (1) Slater-type orbitals and (2) Gaussian-type orbitals,
the discussion considers (3) finite elements,
which are piecewise polynomials on a grid,
as well as (4) Coulomb-Sturmians,
which are the analytical solutions to a Schrödinger-like equation.
A novel algorithmic approach based on matrix-vector contraction expressions is developed,
which is able to adapt to the numerical requirements of all cases considered.
It is shown that this ansatz not only allows to formulate
\SCF algorithms in a basis-function independent way,
but furthermore improves the theoretically achievable
computational scaling for finite-element-based discretisations
as well as performance improvements for Coulomb-Sturmian-based discretisations.
The adequacy of standard \SCF algorithms with respect to a contraction-based
setting is investigated
and for the example of the optimal damping algorithm
an approximate modification to achieve such a setting is presented.

With respect to recent trends in the development of modern computer hardware
the potentials and drawbacks of contraction-based approaches are evaluated.
One drawback, namely the typically more involved and harder-to-read code,
is identified and a data structure named lazy matrix is introduced
to overcome this.
Lazy matrices are a generalisation of the usual matrix concept,
suitable for encapsulating contraction expressions.
Such objects still look like matrices from the user perspective,
including the possibility to perform operations like matrix sums and products.
As a result programming contraction-based algorithms
becomes similarly convenient as working with normal matrices.
An implementation of lazy matrices in the \lazyten linear algebra library
is developed in the course of the thesis followed by an example
demonstrating the applicability in the context of the \HF problem.

Building on top of the aforementioned concepts the design of \molsturm is outlined.
It is shown how a combination of lazy matrices and a contraction-based \SCF scheme
separates the code describing the \SCF procedure
from the code dealing with the basis function type.
It is discussed how this allows to add a new basis function type
to \molsturm by only making code changes in a single integral interface library.
On top of that we demonstrate by the means of examples
how the readily scriptable interface of \molsturm
can be employed to implement and assess novel quantum-chemical methods
or to combine the features of \molsturm with existing third-party packages.

Finally, the thesis discusses an application of \molsturm
towards the investigation of the convergence properties
of Coulomb-Sturmian-based quantum-chemical calculations.
Results for the convergence
of the ground-state energies at \HF level
are reported for atoms of the second and the third period
of the periodic table.
Particular emphasis is put on a discussion about the required
maximal angular momentum quantum numbers
in order to achieve convergence
of the discretisation of the angular part of the wave function.
Some modifications required for a treatment at correlated level are suggested,
followed by a discussion of the effect of the Coulomb-Sturmian exponent.
An algorithm for obtaining an optimal exponent is devised
and some optimal exponents for the atoms of the
second and the third period of the periodic table at \HF level are given.
Furthermore the first results of a Coulomb-Sturmian-based excited
states calculation based on the algebraic-diagrammatic construction
scheme for the polarisation propagator are presented.


%
% German
%
\chapter*{Zusammenfassung}
\phantomsection
\addcontentsline{toc}{chapter}{Zusammenfassung}
\markboth{\leftheadstyle{Zusammenfassung}}{}

\begin{otherlanguage}{ngerman}
Für die Berechnung elektronischer Zustände in Molekülen
verwenden aktuelle Methoden vor allem Gaußfunktionen.
Die in den bestehenden Quantenchemiepaketen verwenden Algorithmen
sind dementsprechend oft sehr stark
auf diesen Funktionstyp und dessen numerische Anforderungen angepasst.
Dies ist ein Hindernis für die Verwendung
anderer Basisfunktionen bei der Betrachtung dieser Methoden,
da die Anforderungen durchaus unterschiedlich sein können.
Im Gegensatz dazu wird hier
das Programmpacket \molsturm entwickelt,
welches explizit so gestaltet wurde,
dass neue Basisfunktionen in der Elektronenstrukturtheorie
auf einfache Weise eingebunden und getestet werden können.
Ein wichtiger Bestandteil von \molsturm dafür ist
ein Lösungsverfahren für das Hartree-Fock-Problem~(\HF),
welches mit beliebigen Basisfunktionstypen verwendet werden kann.

Zunächst werden der mathematische Hintergrund der Quantenmechanik
und einige numerische Techniken vorgestellt.
Dabei wird insbesondere auf die oft unterschlagenen Feinheiten eingegangen,
welche beim Diskretisieren eines unendlich-dimensionalen Spektralproblemes
hin zu einem endlich-dimensionalen Eigenwertproblem entstehen.
Häufig verwendete quantenchemische Methoden
wie die vollständige Konfigurationswechselwirkung~(full configuration interaction)
oder \HF werden diskutiert.
Unterschiedliche Formulierungen von \HF werden miteinander verglichen
und
Lösungsalgorithmen auf Basis des Verfahrens des
selbstkonsistenten Feldes~(self-consistent field, \SCF) angegeben.

Im Weiteren werden Diskretisierungsansätze basierend auf vier verschiedenen
Basisfunktionstypen miteinander verglichen,
wobei sowohl auf Herausforderungen in Bezug auf die numerische Berechenbarkeit
der entstehenden Ausdrücke
eingegangen wird,
als auch auf die Fähigkeit der Basen,
die physikalischen Eigenschaften der Wellenfunktion zu beschreiben.
Neben (1) Orbitalen vom Slatertyp und (2) Gaußorbitalen
behandelt die vorgestellte Diskussion
(3) finite Elemente,
abschnittsweise Polynome, welche auf einem Gitter definiert sind,
sowie (4) Coulomb-Sturmfunktionen~(Coulomb-Sturmians),
welche die analytischen Lösungen einer der Schrödingergleichung
ähnlichen Differenzialgleichung sind.
Ein neuartiger Algorithmus basierend auf Matrix-Vektor-Kontraktionsausdrücken
wird entwickelt,
welcher die numerischen Anforderungen aller betrachteten Fälle abdeckt.
Es wird gezeigt, dass mittels dieses Ansatzes
nicht nur \SCF-Algorithmen in einer basisfunktionsunabhängigen Weise
formuliert werden können, sondern auch,
dass dadurch die algorithmische Komplexität
für Diskretisierungen basierend auf finiten Elementen reduziert wird
und dass damit
Effizienzverbesserungen
für Diskretisierungen basierend auf Coulomb-Sturmfunktionen
möglich sind.
Die Eignung üblicherweise verwendeter \SCF-Algorithmen
auf eine derartige kontraktionsbasierte~(contraction-based)
Formulierung wird geprüft
und für das Beispiel des Algorithmus der optimalen Dämpfung~(optimal damping algorithm)
wird eine zusätzliche Näherung vorgeschlagen,
die diesen mit der kontraktionsbasierten Formulierung vereinbart.

Vor dem Hintergrund aktuell verfügbarer Hardware werden das Potential
und die Nachteile kontraktionsbasierter Methoden diskutiert.
Ein Hindernis bei der Entwicklung kontraktionsbasierter Methoden
ist oft, dass die daraus entstehenden Quelltexte schwerer lesbar sind.
Um dieses Problem zu umgehen wird die Datenstruktur
einer ``bequemen Matrix''~(lazy matrix) eingeführt.
Bequeme Matrizen sind eine Verallgemeinerung des üblichen Matrixkonzeptes,
welche als eine Art Behältnis~(container) für Kontraktionsausdrücke aufgefasst
werden können.
Von außen, aus Sicht eines Benutzers bequemer Matrizen,
sehen diese weiterhin wie gewöhnliche Matrizen aus,
beispielsweise können sie in üblicher Weise
miteinander addiert oder multipliziert werden.
Die Folge ist,
dass man kontraktionsbasierte Algorithmen auf die gleiche Art und Weise
wie bei der Verwendung von gewöhnlichen Matrizen implementieren kann.
Eine Implementierung des Konzepts der bequemen Matrizen
wird in Form von der Bibliothek \lazyten vorgestellt.
Ebenso wird ein Beispiel gegeben,
welches die Eignung von bequemen Matrizen im Kontext der
Lösung des \HF-Problems demonstriert.

Basierend auf den oben erwähnten Konzepten wird die Programmstruktur
von \molsturm diskutiert.
Es wird dargelegt, wie durch die Anwendung der bequemen Matrizen
innerhalb eines kontraktionsbasierten \SCF-Verfahrens
eine Trennung des Programmcodes,
welcher das \SCF-Verfahren selbst beschreibt
von jenem Programmcode, welcher die Diskretisierung und die Basisfunktionen betrifft,
erreicht werden konnte.
Des Weiteren wird gezeigt,
wie ein neuer Basisfunktionstyp in \molsturm implementiert werden kann,
indem nur an einer einzigen, klar definierten Stelle Änderungen durchgeführt werden.
Darüber hinaus wird mittels einiger Beispiele besprochen,
wie \molsturm über skriptbare~(scriptable) Schnittstellen
mit der Funktionalität bestehender Programme kombiniert werden kann,
um so auf einfache Weise neue Quantenchemiemethoden zu implementieren und zu testen.

Zuletzt wird eine Anwendung von \molsturm für
die Untersuchung der Konvergenzeigenschaften
von Quantenchemierechnungen basierend auf Coulomb-Sturmfunktionen vorgestellt.
Erste Ergebnisse für die Konvergenz der Grundzustandsenergie auf \HF-Niveau
werden für die Atome der zweiten und dritten Periode des Periodensystems vorgestellt.
Im Besonderen wird auf die maximale Drehimpulsquantenzahl eingegangen,
welche benötigt wird,
um eine Konvergenz des Winkelanteils der Wellenfunktion
auf \HF-Niveau zu erreichen.
Mögliche Änderungen dieses Ergebnisses im Hinblick auf die Beschreibung
der Atome mittels Korrelationsmethoden werden kurz angedeutet
und der Effekt des Exponenten der Coulomb-Sturmfunktionen
auf die Grundzustandsenergie diskutiert.
Ein Algorithmus zur Bestimmung des optimalen Exponenten wird konstruiert
und einige optimale Exponenten für die Beschreibung der Atome
der zweiten und dritten Periode des Periodensystems
auf \HF-Niveau werden angegeben.
Des Weiteren werden erste Ergebnisse einer Berechnung elektronisch
angeregter Zustände
mittels des algebraisch-diagrammatischen Konstruktionsschemas
für den Polarisationspropagator basierend auf Coulomb-Sturmfunktionen vorgestellt.
\end{otherlanguage}
