\chapter{Lazy matrices and contraction-based algorithms}
\label{ch:LazyMatrices}
\chapquote{%
There is a race between the increasing complexity of the
systems we build and our ability to develop intellectual
tools for understanding their complexity.
If the race is won by our tools, then systems will eventually
become easier to use and more reliable.
If not, they will continue to become harder to use and less
reliable for all but a relatively small set of common tasks.
Given how hard thinking is, if those intellectual tools are
to succeed, they will have to substitute calculation
for thought.}{Leslie Lamport~(1941--present)}

\section{Contraction-based algorithms}
\label{sec:ContractionAlgos}
\todo[inline,caption={}]{
	\begin{itemize}
		\item Clock speed vs. memory increase plot (where to get the data from?)
		\item Solves some of the issues mentioned in the previous chapter
		\item See presentations
		\item Contraction-based diagonalisation
	\end{itemize}
}

\begin{itemize}
	\item In some cases 
	\item SCF (and many other methods) use iterative solvers
	\item They only need application or contraction of matrices
	\item Can furthermore reduce scaling as we have seen
	\item Typically matrix expression composed of many terms and dependent (with variations) on the problem we want to solve.
	\item But expressions may become complicated:
		\begin{itemize}
			\item Requirements of individual matrix terms differ in storage scheme, contraction scheme
			\item What to do in terms of approximations, matrix properties and the cost per application differs
		\end{itemize}
	\item[$\Rightarrow$] Contraction expression complicated to code
	\item[$\Rightarrow$] We would like to stay flexible
\end{itemize}

\subsection{Advantages and disadvantages of \contraction schemes}
\begin{itemize}
	\item Scaling typically reduced in examples to O(N)
	\item Parallelisation is easier (less data management)
	\item Hardware trends in favour: figure \vref{fig:MemCpuSpeedup}
	%
	\item Matrices more intuitive than dealing with contraction functions
	\item More computations (FLOPS) are needed
	\item Schemes need to be efficient and highly parallelisable
	\item Algorithms typically become more complicated (bookkeeping)
\end{itemize}


\begin{figure}
	\centering
	\includeimage{6_contraction_lazy_matrices/mem_cpu_years}
	\caption{Scale-up of memory band width and CPU clock speed over the years.
	Data taken from \url{https://dave.cheney.net/2014/06/07/five-things-that-make-go-fast}
	}
	\label{fig:MemCpuSpeedup}
\end{figure}


\section{Lazy matrices}
\todo[inline,caption={}]{
	\begin{itemize}
		\item Define
		\item how it works
		\item what it solves
		\item In the context of HF
	\end{itemize}
}

\begin{itemize}
	\item Lazy matrices defined in contrast to stored matrices
	\item Stored matrices: Everything resides in memory
	\item Lazy matrices: Generalisation which lifts this restriction
	\item Only a shell which looks like a matrix, but internally only contains expressions
	\item Subject to lazy evaluation
	\item Everything is delayed until \contraction with vector
	\item For convenience: All matrix operations (including element retrieval) allowed.
\end{itemize}

Let $\mat{U}$ and $\mat{A}$ be lazy matrices and $\vec{x}$ be a normal vector.
We wish to compute the expression
\[ (\mat{I} - \mat{U} \tp{\mat{U}}) \mat{A} (\mat{I} - \mat{U} \tp{\mat{U}}) \vec{x} \]
which occurs in Jacobi-Davidson procedure.
A typical code is
\begin{lstlisting}
DiagonalMatrix<matrix_type> I(std::vector<scalar_type>(100, 1));
auto projector = I - U * transpose(U);
auto mat = projector * A * projector;
auto res = mat * x
\end{lstlisting}


% See for example slide 15 of iwr school talk
% Code example
% Resulting expression tree
% Evaluation order


\label{sec:lazymat}
\todoil{Maybe illustrate the points mentioned here using the coulomb and exchange matrices as actual examples}

% TODO a good paragraph for this part:
Whenever operations are done on a matrix \lazyten does not automatically
evaluate them in all cases.
Much rather it usually just builds up a datastructure,
which keeps track of the operations which \textit{should} be done
at some point in the future.
Whenever an actual call to the \contraction-function happens,
the expression history is considered and evaluated with respect
to the other arguments supplied by the \contraction call.
% end TODO

% For the design chapter it is important to mention here that \contraction-based methods
% are a generalisation of non-contraction based methods.

\todoil{Maybe this section is too long given that we want to publish on this again}

As mentioned above for some basis function types it is advantageous
to not build the coulomb and exchange matrices in memory,
but instead only use them in the form of matrix-vector applications.
Since iterative solvers like Krylov-subspace based methods or Davidson's algorithm
only need the matrix-vector product in order to find a few of the eigenvalues
of a particular matrix,
this is not an issue with respect to the eigenproblem which needs to be solved
as part of the SCF.
Note, that this approach is furthermore not new in quantum chemistry at all.
Other examples where one typically avoids to place the matrix to diagonalise
into memory are the configuration interaction~(CI) matrix
or the algebraic-diagrammatic construction~(ADC) matrix to name a few.

We shall refer to algorithms which focus on implementing a matrix-vector product
instead of the full matrix as \contraction-based algorithms.

This concept has been invented many times
and has hence been called a number of different things
in the different communities.
Other names include \texttt{matrix}-free
\todoil{papers from Kronbichler}
or apply-based
\todoil{papers from Michi Wormit}


The latter name is supposed to indicate that such a scheme is in principle
not only restricted to the case of multiplying a matrix with a vector,
much rather general tensor contractions could be expressed implicitly
by the means of a function computing the contraction rather than
by explicitly performing the contraction form tensor elements
which reside in memory.
\todoil{Also mention the term \texttt{apply}-based?
Check what term is used in literature.
Michi Wormit's thesis uses matrix-vector product as term for ADC
}
% Yes mention them all.

Note that a disadvantage of \contraction-based algorithms is that the expressions
computing the tensor contraction can become rather complicated.
Furthermore it usually is more intuitive to think of the
numerical modelling in terms of tensors, matrices and vectors
rather than \contraction functions.
For this purpose we generalise the concept of a matrix
to so-called \term{lazy matrix} objects.
Whilst a normal or \term{stored matrix} is dense and has all its elements
residing in memory,
a lazy matrix is more general.
It may follow a particular sparse storage scheme
like compressed-row storage
or even more general all its elements may just be arbitrary expressions.
The values of lazy matrices are hence only computed upon request
or whenever a contraction with another object is required.
As such the lazy matrix may have, \ie a special \update function
may be used to modify the expression of the lazy matrix
at a later point.
Note, however, that special storage schemes for storing
sparse matrices are similarly just lazy matrices 
\todoil{
	Updating lazy matrices is a bit like reactive programming
	\url{https://en.wikipedia.org/wiki/Reactive_programming}.
	In fact one could use them to get reactive programming
	into C++ in a simple way.
	Mention this here (or in the \lazyten paper)
}

This on the one hand makes obtaining the lazy matrix elements
expensive, but on the other hands gives a nice matrix-like
interface to more complicated objects.
All evaluation between lazy matrices
(i.e. addition or matrix-matrix multiplication)
is usually delayed until for example a contraction of the resulting
expression with a vector is performed.
One typically refers to this strategy as \term{lazy evaluation}.

Since lazy matrices are a straight generalisation
of usual matrices,
all algorithms written in terms of lazy matrices
are at the same time applicable to dense matrices,
sparse matrices or special matrices like in the case of Sturmians
where a lazy evaluation scheme is needed.
So high-level code written in terms of lazy matrices
does not need to be changed if the low level matrix implementation
is changed from one basis function type to another.

\todoil{One could mention the  processor-memory performance gap \ldots
	    but I would leave it for the \lazyten paper}

% 1/2 paragraph: Introducing the problem of large tensors in QM - apply based algorithms
% 1/2 paragraph: One of the flexibilities of Molsturm is that we can switch between
%                apply-based, sparse, and dense methods without changing high-level
%                code (which simply mimic the formulæ). 
%
% define stored vs. lazy matrix
% lazy: sparse or expressions
% lazy matrix may have state
% other key features of lazy matrices



\section{\lazyten lazy matrix library}

\begin{figure}
	\centering
	\includeimage{6_contraction_lazy_matrices/lazyten_structure}
	\caption{Overview of \lazyten}
	\label{fig:structureLazyten}
\end{figure}




\begin{itemize}
	\item Builds up DAG instead of evaluating
	\item Evaluation is done delayed upon \contraction
	
\end{itemize}

Fazit
\begin{itemize}
	\item contraction based algorithms may lead to lower memory footprint and
		hence to improved scaling
	\item Code can be more readable if if proper abstraction is used
	\item Good for teaching and for experimentation
	\item Abstraction between integrals and SCF algorithms
	\item Plug and play integrals libraries (see next chapter)
	\item Basis-type independent SCF / quantum chemistry
\end{itemize}

\section{Examples}
An example showing the performance and how we can run things with bohrium as well
