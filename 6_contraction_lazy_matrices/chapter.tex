\chapter{Contraction-based algorithms and lazy matrices}
\chaptermark{Contraction-based algorithms \& lazy matrices}
\label{ch:LazyMatrices}
\chapquote{%
There is a race between the increasing complexity of the
systems we build and our ability to develop intellectual
tools for understanding their complexity.
If the race is won by our tools, then systems will eventually
become easier to use and more reliable.
If not, they will continue to become harder to use and less
reliable for all but a relatively small set of common tasks.
Given how hard thinking is, if those intellectual tools are
to succeed, they will have to substitute calculation
for thought.}{Leslie Lamport~(1941--present)}
%
%

\noindent
Summarised in one sentence the basic idea of \contraction-based algorithms
is to avoid storing large tensors in memory
and instead employ highly optimised
\contraction expressions of this tensor with other tensors for computations.
We already saw in the previous chapter that applying
such a strategy to the Fock matrix resulting from a \FE-based
or a \CS-based discretisation of the \HF problem
can lead to an improved formal computational scaling
making the \contraction-based approach rather promising in those cases.
\contraction-based algorithms are, however,
not limited to \SCF procedures or quantum-chemical calculations.

This chapter will give a general introduction into
the concept of \contraction-based algorithms,
contrasting them with traditional approaches,
where the matrices are stored in memory.
Closely connected to \contraction-based approaches is the concept of lazy matrices,
which is a direct generalisation to the traditional matrix concept
in the form of a domain-specific language for \contraction-based algorithms.
Most importantly this implies that algorithms programmed
with lazy matrices can automatically
be used with dense matrices or in a \contraction-based fashion without change to the code.
A preliminary \cpp implementation of lazy matrices
is available in the \lazyten library,
which focuses predominantly on user-friendliness and flexibility
in the computational back-end.
We will demonstrate by the means of examples.

\section{Contraction-based algorithms}
\label{sec:ContractionAlgos}

The underlying idea of \contraction-based methods,
namely to avoid storing large matrices in favour of using
matrix-vector-product expressions,
is hardly new.
In his paper from \citeyear{Davidson1975}
\citet{Davidson1975} not only describes his now famous
iterative diagonalisation method~(see section \ref{sec:Davidson}),
but furthermore he suggests to use an algorithmic expression
for computing the required matrix-vector products.
The use case Davidson had in mind back then was the
diagonalisation of the CI or Full-CI matrix,
which is --- even today --- too large to keep in memory,
see remark \vref{rem:EvalFCIMatrix}.

Nowadays \contraction-based methods
are rather widespread in quantum chemistry.
Even though the \contraction expressions are sometimes given different
names such as \textbf{working equations},
making the concept less clear.
Examples are recent implementations of the algebraic diagrammatic construction~(\ADC)
scheme~\cite{Wormit2009,Wormit2014,Dreuw2014},
which do not build the complete \ADC matrix to be diagonalised,
and efficient coupled-cluster schemes~\cite{Helgaker2013},
which similarly avoid constructing the matrix
governing the \CC fixpoint problem explicitly.
Instead both methods use appropriate tensor contractions
and compute matrix-vector products on the fly
during the respective iterative solves.
A somewhat related take on this are the recent
\textbf{matrix-free methods}~\cite{Kronbichler2012}
for solving partial differential equations in a finite-element discretisation
without building the system matrix in memory at all.

From the algorithmic point of view one should notice,
that some methods, like the dense eigensolvers and linear solvers
implemented in LAPACK\todo{cite}
do require random access into the matrix,
are thus not available for a \contraction-based ansatz.
In practice this an acceptable restriction.
Firstly because for large matrices dense methods become
unfavourably expensive anyway%
\footnote{Usually exactly because they necessarily keep everything in memory.}.
Secondly because many diagonalisation methods and methods for solving linear systems
do not need the problem matrix in memory.
Instead they can be operated just like the Davidson,
by coding an expression for delivering the required matrix-vector products.
In this category practically all Krylov-subspace approaches can be found,
including widely-adopted algorithms like
\todoil{cite}\noindent
Arnoldi, Lanczos, conjugate gradient or GMRES.
In the context of eigenproblems
one should mention that such iterative methods
have an additional disadvantage.
It is typically very costly to obtain a large number of eigenpairs
of the diagonalised matrix.
Fortunately for large matrices this is hardly needed
and techniques like Chebychev preconditioning~\todo{cite}
or shift and invert~(see section \vref{sec:ShiftInvert})
allow to effectively direct the diagonalisation routines
towards the part of the eigenspectrum one is truely interested in.

On the one hand employing a \contraction-based method thus do not really
restrict the range of numerical problems, which can be tackled.
On the other hand avoiding the storage of the problem matrix
immediately reduces the scaling in memory from quadratic (in system size) to linear.
The rationale for this is that the memory bottleneck
in most subspace algorithms is storing the generated subspace,
\ie a fixed number of vectors, which take linear storage.
This makes \contraction-based methods especially
attractive for problems where memory is a bottleneck.
For this reason this concept has been
introduced in a range of fields of numerics and scientific computing
under different names.
Terms like \textbf{apply-based} method, \textbf{matrix-free} method
or phrases like using \textbf{matrix-vector product expressions}
or using \textbf{matrix-vector products}
overall largely describe the same concept.
I personally like the term \textbf{\contraction-based} best,
because under the hood
evaluating such matrix-vector products in many cases,
that I came across,
involves expressions with contractions over tensors with rank larger than 2.
Consider for example the coupled-cluster doubles working equations~\eqref{eqn:CCDworking}
or the contraction expression for the exchange matrix
in a \CS-based discretisation of Hartree-Fock~\eqref{eqn:ApplicationKcs}.
Additionally calling such algorithms \contraction-based
indicates that the idea to substitute storage by expressions
is more general than the matrix-vector product.
In theory one could think of similar approaches for
higher-order tensor contractions as well.

\subsection{Advantages and disadvantages}
\label{sec:ContraAdvDisadv}
One of the things one needs to be constantly aware of,
when designing \contraction-based methods
is that most of the computational time will most probably be spent
in evaluating the \contraction expression itself.
For example a typical diagonalisation using Arnoldi's or Davidson's method
requires a number in the low to mid hundreds of matrix-vector products to be computed.
Usually this step therefore dominates the overall
computational time and needs to be implemented efficiently.

Even in the naive manner presented
in equations \eqref{eqn:ExchangeApply} and \eqref{eqn:ApplicationKcs}
the \contraction expressions
of the exchange term of the Fock matrix with a trial vector
look all but simple.
In practice the actual expressions to implement
will most probably be more involved,
since issues like the following will need to be addressed
in an algorithm achieving the maximal efficiency:
\todoil{References for the guys in the itemize}
\begin{itemize}
	\item \textbf{Adoption to Hardware} and \textbf{parallelisation}:
		The features provided by modern hardware
		have of course changed a lot over the years.
		This includes aspects like vectorisation
		or the recent trend to employ general-purpose graphics cards
		in scientific calculations.
		A good algorithm takes modern features into account
		and shows a parallelisation scheme,
		which exploits the available hardware as good as possible.
		Notice, that in many cases the requirements
		can be contradictory,
		such that achieving best performance in all circumstances
		is a real challenge if not impossible.
	\item \textbf{Storing intermediates:}
		Often one can identify subexpressions
		of a large \contraction expression,
		where it makes sense to store it between individual
		executions of the \contraction.
	\item \textbf{Order of contractions:}
		For more complicated expressions involving
		multiple tensor contractions at once the order
		in which the contractions are executed can be crucial
		to achieve best computational scaling
		as well as a low memory footprint.
	\item \textbf{Approximations:}
		Especially in iterative procedures one is typically not
		interested in the numerically exact result
		of a \contraction.
		Much rather the iterative procedure will only solve
		the problem up to a certain accuracy threshold,
		such that computing elements,
		which are smaller than this threshold is a waste of computational time.
		Sometimes this can be incorporated into a \contraction expression
		by prescreening the elements to compute or by other approximations.
\end{itemize}
Furthermore problem matrices
are usually composed out of different terms
with potentially different structure.
In the case of the \SCF the Fock matrix, for example,
is as sum of the local one-electron terms $\mat{T}$ and $\mat{V}_0$,
which can just be computed,
the Coulomb term $\mat{J}$,
which requires the solution to a Poisson equation,
and the exchange term $\mat{K}$ which requires to solve
multiple equations on a single apply.
It is therefore not hard to imagine
that the best approach to the issues raised above might
well differ from term to term.
This implies that adding terms like $\mat{T}$, $\mat{V}_0$, $\mat{J}$
and $\mat{K}$ together to form the Fock matrix is conceptionally
more involved compared to the traditional case,
where all these matrices reside in memory.
On the upside the traditional case offers much less flexibility
to address the above issues differently on each individual term.

This rather paradoxical effect
can be observed in other cases as well.
Usually it originates from the fact that the delayed evaluation of the matrix elements
\emph{during} an actual contraction with a vector,
may allow to perform the
tensor contractions in a more favourable order.
Naturally the drawback of this is that computations need to be repeated
many times over.
\begin{figure}
	\centering
	\includeimage{6_contraction_lazy_matrices/mem_cpu_years}
	\caption[Scale-up of memory bus speed and CPU clock speed]
	{Scale-up of memory bus speed and CPU clock speed
		relative to 1980 for selected hardware in each year.
		Data taken from \cite{Gocon2014}.}
	\label{fig:MemCpuSpeedup}
\end{figure}
In the light of the hardware trends this does no longer seem like an issue.


Especially when it comes to hardware trends,
the \contraction-based methods have another advantage.
Figure \vref{fig:MemCpuSpeedup} shows the relative change
in CPU clock speed versus memory bandwith compared to the year 1980.
The gap between the CPU and the memory speedups
is clearly visible%
\footnote{The original source \cite{Gocon2014}
does not provide a clear description how the data points in each year
where computed from the range of processors and chipsets,
which were released in this year.
Nevertheless the trend is so clear, that I consider this aspect
to have little influence on the overall trends.}.
The takeaway from this is that
computing data multiple times is nowadays in many cases
more efficient than storing it and retrieving it from memory.
The \contraction-based methods go exactly in this direction,
replacing storage by repeated computation.
As we saw in the previous chapter
for \FE-based or \CS-based discretisations of \HF,
the formal scaling in memory and time
can become lower by recomputing data as well.

Additionally one should note that parallelisation becomes
easier if less memory / data needs to be managed
and can be recomputed on different nodes or in the caches of different
processors as needed.





% transition to lazy tensors

Additionally for these more complicated cases
it is not exactly clear what the best answer will be,
such that the flexibility to experiment with different aspects is key.
Another aspect, which is often forgotten,
but strongly emphasises the need for flexibility
is that hardware frequently changes.
So the best implementation to the current state of hardware,
will be outdated in a few years time,
calling for frequent changes to the implemented \contraction expression.

Such changes should not influence the outer iterative process
like the iterative diagonalisation or the actual \SCF algorithm,
since not that much effort has to be put into these parts
of the program for optimisation
and thus these are not that much influenced by the hardware changes for example.
This calls for a good layer of abstraction between the
iterative routine and the implementation of the matrix-vector \contraction expressions.

Hard to think about problem in terms of expressions.


\section{Lazy matrices}
\label{sec:lazymat}
The idea of lazy matrices is to encapsulate the coding \contraction-based
in a domain-specific language,
which makes it feel as if one was dealing with actual matrices
instead of \contraction expressions.
Even though not all complications can be hidden,
the resulting syntax allows to write algorithms in a high-level manner
being agnostic to the underlying implementation of the \contraction expression.
This will turn out to be the key aspect
leading to the basis-type of the quantum-chemistry package \molsturm.

For this purpose we generalise the concept of a matrix
to objects we call a \newterm{lazy matrix}.
Whilst a conventional or \newterm{stored matrix} is dense
and has all its elements residing in a continuous chunk of memory,
this restriction does no longer hold for a lazy matrix.
It may for example follow a particular sparse storage scheme
like compressed-row storage,
but does not even need to be associated to any kind of storage at all.
In the most general sense it can be thought of as an
arbitrary \contraction expression for computing the matrix elements,
which is dressed to look like an ordinary matrix from the outside.

In other words one may still obtain individual matrix elements,
add, subtract or multiply such lazy matrix objects together
or apply them to a bunch of vectors or a stored matrix.
Not all of these operations may be equally fast
than there counterparts on stored matrices, however.
Most importantly obtaining individual elements of such a matrix
can become rather costly,
since they might involve a computation as well
and not just a lookup into memory.

On the upside one gains a much more flexible data structure
where a familiar matrix-like interface
can be added to more complicated objects.
Most notably a lazy matrix may well be non-linear or can have state,
which may be changed by a \update function
in order to influence the represented values at a later point.
An example where this is sensible would be the Coulomb and Exchange matrices,
where the values of these matrices depend on the set of occupied coefficients,
which have been obtained in the previous iterations.
Other examples include the \update of an accuracy threshold for a \contraction expression,
which might change between iterations.

\begin{figure}
	\centering
	\includeimage{6_contraction_lazy_matrices/expression_tree} \\[0.8em]
	\includeimage{6_contraction_lazy_matrices/expression_tree2}
	\caption[Examples for lazy matrix expression trees]
	{
		Examples for lazy matrix expression trees.
		The upper represents the instruction
		$\mat{D} = \mat{A} + \mat{B}$
		and the lower the multiplication of the result $\mat{D}$
		with $\mat{C}$.
	}
	\label{fig:LazyMatrixExpressionTree}
\end{figure}
All evaluation between lazy matrices
like addition, subtraction or matrix-matrix multiplication
is usually delayed until a contraction of the resulting
expression with a vector or a stored matrix is performed
and thus the represented values are unavoidably needed.
This evaluation strategy is called \newterm{lazy evaluation}
in programming language theory~\todo{cite https://dl.acm.org/citation.cfm?doid=72551.72554}, explaining the name of these data structures.
To make this more clear consider the lazy matrix instructions
\begin{equation}
	\begin{aligned}
		\label{eqn:LazyMatrixInstructions}
		\mat{D} &= \mat{A} + \mat{B}, \\
		\mat{E} &= \mat{D} \mat{C}, \\
		\vec{y} &= \mat{E} \vec{x},
	\end{aligned}
\end{equation}
where $\mat{A}$, $\mat{B}$ and $\mat{C}$ are lazy matrices
and $\vec{x}$ is a vector stored in memory.
The first two do not give rise to any computation being done.
They only amount to build an expression tree in the returned
lazy matrix $\mat{E}$ as illustrated in figure \vref{fig:LazyMatrixExpressionTree}.
The final line is a matrix-vector product with a stored vector,
where an actual stored result should be returned in the vector $\vec{y}$.
In the lazy matrix sense this triggers the complete expression tree to be
evaluated in appropriate order,
leading effectively to an evaluation of the expression
\begin{equation}
	\vec{y} = \left( \mat{A} + \mat{B} \right) \mat{C} \vec{x}
	\label{eqn:LazyMatrixFinalExpression}
\end{equation}
at once at this very instance.
\eqref{eqn:LazyMatrixFinalExpression} can be evaluated entirely only using
matrix-vector \contraction expressions.
For example one could first form the product $\tilde{\vec{x}} \equiv \mat{C} \vec{x}$
using the  matrix-vector-product expression of the lazy matrix $\vec{C}$.
Afterwards one would form $\mat{A} \tilde{\vec{x}}$ and $\mat{B} \tilde{\vec{x}}$
again by appropriate \contraction expressions
and finally add the result to give $\vec{y}$.
This is just one way to perform the evaluation.
An implementation of the lazy matrix language is free to choose
a different route for evaluating \eqref{eqn:LazyMatrixFinalExpression}
by reordering the expression if it considers this useful.
If $\mat{C}$ for example was made up of a sum $\mat{F} + \mat{G}$,
it could use distributivity to write
\[ \left( \mat{A} + \mat{B} \right) \left( \mat{F} + \mat{G} \right) \vec{x}
	= \mat{A} \left( \mat{F} \vec{x} \right) + \mat{A} \left( \mat{G}\vec{x}\right)
	+ \mat{B} \left( \mat{F} \vec{x} \right) + \mat{B} \left( \mat{G}\vec{x}\right).
\]
Which of these routes is best differs very much on the structure
of the lazy matrices being part of the expression to evaluate.
But other factors like the operating system or hardware on which
the program code is executed are not unimportant either.
Since the evaluation is delayed
until the call to $\mat{E}\vec{x}$ gets executed at the actual program runtime,
all of this can in theory be taken into account for deciding
which route to take.
Naturally the design of an appropriate cost function
will be in practice all but easy
as previous attempts have shown~\todo{cite a few}.

In either case such decision happen in the evaluation back end
and are well-abstracted by the lazy matrix language
from the instructions \eqref{eqn:LazyMatrixInstructions},
which stay intelligible and understandable.
Furthermore if the structure of the matrices changes,
since for example the discretisation scheme changes,
the evaluation route will automatically adapt
given that the cost function is sensibly chosen.

\section{\lazyten lazy matrix library}




Nothings stops us to directly evaluate and keep expressions like $\mat{A} + \mat{B}$
in memory if both matrices are stored once this result is needed for an apply.

Since lazy matrices are a straight generalisation of usual matrices,
all algorithms written in terms of lazy matrices
are at the same time applicable to dense matrices,
sparse matrices or special matrices like in the case of Sturmians
where a lazy evaluation scheme is needed.
In the context of an \SCF this implies that high-level code
written in terms of lazy matrices
does not need to be changed if the low level matrix implementation
is changed from one basis function type to another.






introduce lazyten (statistics: number of lines of code)
basic ideas behind lazyten / structures
examples (davidson, exchange / coulomb <-- update function)
-> show that it is more intuitive to think of \contraction expressions as lazy matrices

advantage for scf and quantum chemistry
hint at lazy tensors?
transition to molsturm



% ----
the \SCF algorithm
and the implementation of the matrix-vector \contraction expressions.
An additional advantage of such a layer would be that
modifications of the system matrix can be transparently expressed
to the iterative schemes.
For example adding extra terms to the Fock matrix
describing an electric field or a polarisable embedding
or other contributions does not really require
any change to the \SCF code at all.
A good layer of abstraction between the actual \SCF algorithm
and the implementation of the matrix-vector contractions
is therefore highly desirable to enable flexibility on both sides.

need flexibility in matrix expressions in real-world application
like \SCF (extra terms of varying complexity from external field,
polarisable embedding, fragment approaches, and so on)
non-linearity of Coulomb and Exchange
would be nice to transparently express this

% ----

Additionally for these more complicated cases
it is not exactly clear what the best answer will be,
such that the flexibility to experiment with different aspects is key.
Another aspect, which is often forgotten,
but strongly emphasises the need for flexibility
is that hardware frequently changes.
So the best implementation to the current state of hardware,
will be outdated in a few years time,
calling for frequent changes to the implemented \contraction expression.

Such changes should not influence the outer iterative process
like the iterative diagonalisation or the actual \SCF algorithm,
since not that much effort has to be put into these parts
of the program for optimisation
and thus these are not that much influenced by the hardware changes for example.
This calls for a good layer of abstraction between the
iterative routine and the implementation of the matrix-vector \contraction expressions.

Hard to think about problem in terms of expressions.

% ---

designed to add lazy matrix support to existing
linear algebra packages.
It effectively provides a layer of abstraction between the code
describing the algorithms


In the previous chapter we already mentioned
how similar ideas can reduce the formal
computational scaling of the \SCF procedure
for certain kinds of discretisations
and additionally could even lead to a basis-function
agnostic \SCF.
% ---


Lazy matrices can deal with the non-linearity of Coulomb and exchange
and it makes it easy to add terms (maybe show example ??)

Let $\mat{U}$ and $\mat{A}$ be lazy matrices and $\vec{x}$ be a normal vector.
We wish to compute the expression
\[ (\mat{I} - \mat{U} \tp{\mat{U}}) \mat{A} (\mat{I} - \mat{U} \tp{\mat{U}}) \vec{x} \]
which occurs in Jacobi-Davidson procedure.
A typical code is
\begin{lstlisting}
DiagonalMatrix<matrix_type> I(std::vector<scalar_type>(100, 1));
auto projector = I - U * transpose(U);
auto mat = projector * A * projector;
auto res = mat * x
\end{lstlisting}


% See for example slide 15 of iwr school talk
% Code example
% Resulting expression tree
% Evaluation order


\todoil{Maybe illustrate the points mentioned here using the coulomb and exchange matrices as actual examples}



\begin{figure}
	\centering
	\includeimage{6_contraction_lazy_matrices/lazyten_structure}
	\caption{Overview of \lazyten}
	\label{fig:structureLazyten}
\end{figure}




Fazit
\begin{itemize}
	\item contraction based algorithms may lead to lower memory footprint and
		hence to improved scaling
	\item Code can be more readable if if proper abstraction is used
	\item Good for teaching and for experimentation
	\item Abstraction between integrals and SCF algorithms
	\item Plug and play integrals libraries (see next chapter)
	\item Basis-type independent SCF / quantum chemistry
\end{itemize}





\lazyten is the linear algebra interface of \molsturm.
It not only implements the lazy matrix datastructures,
which define the common interface of \gint and \gscf,
but further contains code to make standard external
iterative or direct solver implementations available
from a lazy matrices-based setting.

As can be seen in figure \ref{fig:lazyten} \lazyten provides
a common interface for the \contraction-based algorithms in \gscf,
regardless of the linear algebra~(LA) backend or the solver implementation,
which is used to solve a particular problem.
Lazy matrices like the integrals from \gint as well as built-in structures
like for example a lazy matrix representing the inverse of a matrix
can be used transparently and even in combination with stored matrices
by the means of the automatic bookkeeping done by \lazyten.
Whenever a call to for example the \contraction function enforces
evaluation of a lazy matrix expression,
\lazyten is flexible with respect to the LA backend which is used for
this step as well.
Both armadillo as a LAPACK-based backend as well as Bohrium as a array-operation based backend
are currently available and by recompiling \molsturm with the appropriate configure options,
the backend can be switched.

When it comes to the choice of the solver algorithms, \lazyten is even more flexible.
All algorithms for solving linear problems or eigenproblems are available as 
convenient high-level interface from \gscf.
For example a linear problem $A x = b$ with a lazy matrix $A$, known right-hand side $b$
and unknown $x$ can either be solved by a call to a \texttt{solve} function or by
the means of calling \texttt{inverse(A)} on $A$ and then subsequently
multiplying the inverse matrix with $b$, quite literally coding it as $x = A^{-1} b$.
In both cases \lazyten will perform some introspection regarding the properties of
the lazy matrix expression behind $A$ and consider the number of solution
vectors as well as the required accuracy in order to automatically determine the algorithm to use
for solving the linear problem at hand.
By the means of keyword arguments the user can influence or override the automatic
choice made by \lazyten.

Similarly the eigensolvers can be accessed from \gscf via a common interface,
which abstracts the precise algorithm and allows \lazyten to make an automatic
choice by looking at the precise structure of the Fock matrix at hand.
If all eigenpairs are desired or the matrix is stored in memory anyway,
\lazyten will favour the dense eigensolver algorithms from LAPACK,
whereas if only few eigenpairs are desired,
Krylov-subspace based methods from ARPACK are used.
In either case the precise algorithm can also me chosen from the \python interface
of \molsturm by supplying an appropriate keyword string.


\section{Examples}
An example showing the performance and how we can run things with bohrium as well

