\chapter{Contraction-based algorithms and lazy matrices}
\chaptermark{Contraction-based algorithms \& lazy matrices}
\label{ch:LazyMatrices}
\chapquote{%
There is a race between the increasing complexity of the
systems we build and our ability to develop intellectual
tools for understanding their complexity.
If the race is won by our tools, then systems will eventually
become easier to use and more reliable.
If not, they will continue to become harder to use and less
reliable for all but a relatively small set of common tasks.
Given how hard thinking is, if those intellectual tools are
to succeed, they will have to substitute calculation
for thought.}{Leslie Lamport~(1941--present)}
%
%

\noindent
Summarised in one sentence the main idea of \contraction-based algorithms
is to avoid storing large matrices or tensors in memory
and instead employ highly optimised
\contraction expressions between tensors for the necessary computations.
We already saw in the previous chapter that applying
such a strategy to the Fock matrix resulting from a \FE-based
or a \CS-based discretisation
can lead to an improved formal computational scaling
making these methods a promising approach.
\contraction-based algorithms are, however,
not at all limited to \SCF procedures or quantum-chemical calculations.
This chapter will give a general overview
of \contraction-based methods,
giving some examples where these methods are employed
as well as discussing pros and cons.

Closely connected to \contraction-based methods is the concept of lazy matrices,
which is a direct generalisation to the conventional matrices
in the form of a domain-specific language
for coding \contraction-based algorithms.
Main goal of the lazy matrix language is to result code,
which can be used both with matrices stored in memory
and additionally in a \contraction-based fashion without noteworthy changes.
A preliminary \cpp implementation of lazy matrices
with focus on user-friendliness and flexibility
is available in the \lazyten library.

\section{Contraction-based algorithms}
\label{sec:ContractionAlgos}

The underlying idea of \contraction-based methods,
namely to avoid storing large matrices in favour of using
matrix-vector-product expressions,
is hardly new.
In his paper from \citeyear{Davidson1975}
\citet{Davidson1975} not only describes his now famous
iterative diagonalisation method~(see section \ref{sec:Davidson}),
but furthermore he suggests to use an algorithmic expression
for computing the required matrix-vector products.
The use case Davidson had in mind back then was the
diagonalisation of the CI or Full-CI matrix,
which is --- even today --- too large to keep in memory,
see remark \vref{rem:EvalFCIMatrix}.

Nowadays \contraction-based methods
are rather widespread in quantum chemistry.
Even though the \contraction expressions are sometimes given different
names such as \textbf{working equations},
making the concept less clear.
Examples are recent implementations of the algebraic diagrammatic construction~(\ADC)
scheme~\cite{Wormit2009,Wormit2014,Dreuw2014},
which do not build the complete \ADC matrix to be diagonalised,
and efficient coupled-cluster schemes~\cite{Helgaker2013},
which similarly avoid constructing the matrix
governing the \CC fixpoint problem explicitly.
Instead both methods use appropriate tensor contractions
and compute matrix-vector products on the fly
during the respective iterative solves.
A somewhat related take on this are the recent
\textbf{matrix-free methods}~\cite{Kronbichler2012}
for solving partial differential equations in a finite-element discretisation
without building the system matrix in memory at all.

From the algorithmic point of view one should notice,
that some methods, like the dense eigensolvers and linear solvers
implemented in LAPACK\todo{cite}
do require random access into the matrix,
are thus not available for a \contraction-based ansatz.
In practice this an acceptable restriction.
Firstly because for large matrices dense methods become
unfavourably expensive anyway%
\footnote{Usually exactly because they necessarily keep everything in memory.}.
Secondly because many diagonalisation methods and methods for solving linear systems
do not need the problem matrix in memory.
Instead they can be operated just like the Davidson,
by coding an expression for delivering the required matrix-vector products.
In this category practically all Krylov-subspace approaches can be found,
including widely-adopted algorithms like
\todoil{cite}\noindent
Arnoldi, Lanczos, conjugate gradient or GMRES.
In the context of eigenproblems
one should mention that such iterative methods
have an additional disadvantage.
It is typically very costly to obtain a large number of eigenpairs
of the diagonalised matrix.
Fortunately for large matrices this is hardly needed
and techniques like Chebychev preconditioning~\todo{cite}
or shift and invert~(see section \vref{sec:ShiftInvert})
allow to effectively direct the diagonalisation routines
towards the part of the eigenspectrum one is truely interested in.

On the one hand employing a \contraction-based method thus do not really
restrict the range of numerical problems, which can be tackled.
On the other hand avoiding the storage of the problem matrix
immediately reduces the scaling in memory from quadratic (in system size) to linear.
The rationale for this is that the memory bottleneck
in most subspace algorithms is storing the generated subspace,
\ie a fixed number of vectors, which take linear storage.
This makes \contraction-based methods especially
attractive for problems where memory is a bottleneck.
For this reason this concept has been
introduced in a range of fields of numerics and scientific computing
under different names.
Terms like \textbf{apply-based} method, \textbf{matrix-free} method
or phrases like using \textbf{matrix-vector product expressions}
or using \textbf{matrix-vector products}
overall largely describe the same concept.
I personally like the term \textbf{\contraction-based} best,
because under the hood
evaluating such matrix-vector products in many cases,
that I came across,
involves expressions with contractions over tensors with rank larger than 2.
Consider for example the coupled-cluster doubles working equations~\eqref{eqn:CCDworking}
or the contraction expression for the exchange matrix
in a \CS-based discretisation of Hartree-Fock~\eqref{eqn:ApplicationKcs}.
Additionally calling such algorithms \contraction-based
indicates that the idea to substitute storage by expressions
is more general than the matrix-vector product.
In theory one could think of similar approaches for
higher-order tensor contractions as well.

\subsection{Advantages and disadvantages}
\label{sec:ContraAdvDisadv}
One of the things one needs to be constantly aware of,
when designing \contraction-based methods
is that most of the computational time will most probably be spent
in evaluating the \contraction expression itself.
For example a typical diagonalisation using Arnoldi's or Davidson's method
requires a number in the low to mid hundreds of matrix-vector products to be computed.
Usually this step therefore dominates the overall
computational time and needs to be implemented efficiently.

Even in the naive manner presented
in equations \eqref{eqn:ExchangeApply} and \eqref{eqn:ApplicationKcs}
the \contraction expressions
of the exchange term of the Fock matrix with a trial vector
look all but simple.
In practice the actual expressions to implement
will most probably be more involved,
since issues like the following will need to be addressed
in an algorithm achieving the maximal efficiency:
\todoil{References for the guys in the itemize}
\begin{itemize}
	\item \textbf{Adoption to Hardware} and \textbf{parallelisation}:
		The features provided by modern hardware
		have of course changed a lot over the years.
		This includes aspects like vectorisation
		or the recent trend to employ general-purpose graphics cards
		in scientific calculations.
		A good algorithm takes modern features into account
		and shows a parallelisation scheme,
		which exploits the available hardware as good as possible.
		Notice, that in many cases the requirements
		can be contradictory,
		such that achieving best performance in all circumstances
		is a real challenge if not impossible.
	\item \textbf{Storing intermediates:}
		Often one can identify subexpressions
		of a large \contraction expression,
		where it makes sense to store it between individual
		executions of the \contraction.
	\item \textbf{Order of contractions:}
		For more complicated expressions involving
		multiple tensor contractions at once the order
		in which the contractions are executed can be crucial
		to achieve best computational scaling
		as well as a low memory footprint.
	\item \textbf{Approximations:}
		Especially in iterative procedures one is typically not
		interested in the numerically exact result
		of a \contraction.
		Much rather the iterative procedure will only solve
		the problem up to a certain accuracy threshold,
		such that computing elements,
		which are smaller than this threshold is a waste of computational time.
		Sometimes this can be incorporated into a \contraction expression
		by prescreening the elements to compute or by other approximations.
\end{itemize}
Furthermore problem matrices
are usually composed out of different terms
with potentially different structure.
In the case of the \SCF the Fock matrix, for example,
is as sum of the local one-electron terms $\mat{T}$ and $\mat{V}_0$,
which can just be computed,
the Coulomb term $\mat{J}$,
which requires the solution to a Poisson equation,
and the exchange term $\mat{K}$ which requires to solve
multiple equations on a single apply.
It is therefore not hard to imagine
that the best approach to the issues raised above might
well differ from term to term.
This implies that adding terms like $\mat{T}$, $\mat{V}_0$, $\mat{J}$
and $\mat{K}$ together to form the Fock matrix is conceptionally
more involved compared to the traditional case,
where all these matrices reside in memory.
On the upside the traditional case offers much less flexibility
to address the above issues differently on each individual term.

This rather paradoxical effect
can be observed in other cases as well.
Usually it originates from the fact that the delayed evaluation of the matrix elements
\emph{during} an actual contraction with a vector,
may allow to perform the
tensor contractions in a more favourable order.
Naturally the drawback of this is that computations need to be repeated
many times over.
\begin{figure}
	\centering
	\includeimage{6_contraction_lazy_matrices/mem_cpu_years}
	\caption[Scale-up of memory bus speed and CPU clock speed]
	{Scale-up of memory bus speed and CPU clock speed
		relative to 1980 for selected hardware in each year.
		Data taken from \cite{Gocon2014}.}
	\label{fig:MemCpuSpeedup}
\end{figure}
In the light of the hardware trends this does no longer seem like an issue.


Especially when it comes to hardware trends,
the \contraction-based methods have another advantage.
Figure \vref{fig:MemCpuSpeedup} shows the relative change
in CPU clock speed versus memory bandwith compared to the year 1980.
The gap between the CPU and the memory speedups
is clearly visible%
\footnote{The original source \cite{Gocon2014}
does not provide a clear description how the data points in each year
where computed from the range of processors and chipsets,
which were released in this year.
Nevertheless the trend is so clear, that I consider this aspect
to have little influence on the overall trends.}.
The takeaway from this is that
computing data multiple times is nowadays in many cases
more efficient than storing it and retrieving it from memory.
The \contraction-based methods go exactly in this direction,
replacing storage by repeated computation.
As we saw in the previous chapter
for \FE-based or \CS-based discretisations of \HF,
the formal scaling in memory and time
can become lower by recomputing data as well.

Additionally one should note that parallelisation becomes
easier if less memory / data needs to be managed
and can be recomputed on different nodes or in the caches of different
processors as needed.





% transition to lazy tensors

Additionally for these more complicated cases
it is not exactly clear what the best answer will be,
such that the flexibility to experiment with different aspects is key.
Another aspect, which is often forgotten,
but strongly emphasises the need for flexibility
is that hardware frequently changes.
So the best implementation to the current state of hardware,
will be outdated in a few years time,
calling for frequent changes to the implemented \contraction expression.

Such changes should not influence the outer iterative process
like the iterative diagonalisation or the actual \SCF algorithm,
since not that much effort has to be put into these parts
of the program for optimisation
and thus these are not that much influenced by the hardware changes for example.
This calls for a good layer of abstraction between the
iterative routine and the implementation of the matrix-vector \contraction expressions.

Hard to think about problem in terms of expressions.


\section{Lazy matrices}
\label{sec:lazymat}
The idea of lazy matrices is to encapsulate the coding \contraction-based
in a domain-specific language,
which makes it feel as if one was dealing with actual matrices
instead of \contraction expressions.
Even though not all complications can be hidden,
the resulting syntax allows to write algorithms in a high-level manner
being independent from the underlying implementation of the \contraction expression.
This will turn out to be the key aspect
leading to the basis-type of the quantum-chemistry package \molsturm.

For this purpose we generalise the concept of a matrix
to objects we call a \newterm{lazy matrix}.
Whilst a conventional or \newterm{stored matrix} is dense
and has all its elements residing in a continuous chunk of memory,
this restriction does no longer hold for a lazy matrix.
It may for example follow a particular sparse storage scheme
like compressed-row storage,
but does not even need to be associated to any kind of storage at all.
In the most general sense it can be thought of as an
arbitrary \contraction expression for computing the matrix elements,
which is dressed to look like an ordinary matrix from the outside.

In other words one may still obtain individual matrix elements,
add, subtract or multiply such lazy matrix objects together
or apply them to a bunch of vectors or a stored matrix.
Not all of these operations may be equally fast
than there counterparts on stored matrices, however.
Most importantly obtaining individual elements of such a matrix
can become rather costly,
since they might involve a computation as well
and not just a lookup into memory.

On the upside one gains a much more flexible data structure
where a familiar matrix-like interface
can be added to more complicated objects.
Most notably a lazy matrix may well be non-linear or can have a state,
which may be changed by a \update function
in order to influence the represented values at a later point.
An example where this is sensible would be the Coulomb and Exchange matrices,
where the values of these matrices depend on the set of occupied coefficients,
which have been obtained in the previous iterations.
Other examples include the \update of an accuracy threshold for a \contraction expression,
which might change between iterations.

\begin{figure}
	\centering
	\includeimage{6_contraction_lazy_matrices/expression_tree} \\[0.8em]
	\includeimage{6_contraction_lazy_matrices/expression_tree2}
	\caption[Examples for lazy matrix expression trees]
	{
		Examples for lazy matrix expression trees.
		The upper represents the instruction
		$\mat{D} = \mat{A} + \mat{B}$
		and the lower the multiplication of the result $\mat{D}$
		with $\mat{C}$.
	}
	\label{fig:LazyMatrixExpressionTree}
\end{figure}
All evaluations between lazy matrices
like addition, subtraction or matrix-matrix multiplication
is usually delayed until a contraction of the resulting
expression with a vector or a stored matrix is performed
and thus the represented values are unavoidably needed.
This evaluation strategy is called \newterm{lazy evaluation}
in programming language theory~\cite{Hudak1989},
explaining the name of these data structures.
To make this more clear consider the lazy matrix instructions
\begin{equation}
	\begin{aligned}
		\label{eqn:LazyMatrixInstructions}
		\mat{D} &= \mat{A} + \mat{B}, \\
		\mat{E} &= \mat{D} \mat{C}, \\
		\vec{y} &= \mat{E} \vec{x},
	\end{aligned}
\end{equation}
where $\mat{A}$, $\mat{B}$ and $\mat{C}$ are lazy matrices
and $\vec{x}$ is a vector stored in memory.
The first two do not give rise to any computation being done.
They only amount to build an expression tree in the returned
lazy matrix $\mat{E}$ as illustrated in figure \vref{fig:LazyMatrixExpressionTree}.
The final line is a matrix-vector product with a stored vector,
where an actual stored result should be returned in the vector $\vec{y}$.
In the lazy matrix sense this triggers the complete expression tree to be
evaluated in appropriate order,
leading effectively to an evaluation of the expression
\begin{equation}
	\vec{y} = \left( \mat{A} + \mat{B} \right) \mat{C} \vec{x}
	\label{eqn:LazyMatrixFinalExpression}
\end{equation}
at once at this very instance.
\eqref{eqn:LazyMatrixFinalExpression} can be evaluated entirely only using
matrix-vector \contraction expressions.
For example one could first form the product $\tilde{\vec{x}} \equiv \mat{C} \vec{x}$
using the matrix-vector-product expression of the lazy matrix $\mat{C}$.
Afterwards one would form $\mat{A} \tilde{\vec{x}}$ and $\mat{B} \tilde{\vec{x}}$
again by appropriate \contraction expressions
and finally add the result to give $\vec{y}$.
This is just one way to perform the evaluation.
An implementation of the lazy matrix language is free to choose
a different route for evaluating \eqref{eqn:LazyMatrixFinalExpression}
by reordering the expression if it considers this useful.
If $\mat{C}$ for example was made up of a sum $\mat{F} + \mat{G}$,
it could use distributivity to write
\[ \left( \mat{A} + \mat{B} \right) \left( \mat{F} + \mat{G} \right) \vec{x}
	= \mat{A} \left( \mat{F} \vec{x} \right) + \mat{A} \left( \mat{G}\vec{x}\right)
	+ \mat{B} \left( \mat{F} \vec{x} \right) + \mat{B} \left( \mat{G}\vec{x}\right).
\]
Which of these routes is best differs very much on the structure
of the lazy matrices being part of the expression to evaluate.
But other factors like the operating system or hardware on which
the program code is executed are not unimportant either.
Since the evaluation is delayed
until the call to $\mat{E}\vec{x}$ gets executed at the actual program runtime,
all of this can in theory be taken into account for deciding
which route to take.
Naturally the design of an appropriate cost function
is not easy as previous works have shown%
~\cite{Baumgartner2005,Solomonik2014,Peise2015,Calvin2015,Calvin2015arxiv,%
Kristensen2016array,Kristensen2016streaming}

In either case such decision happen in the evaluation back end
and are well-abstracted by the lazy matrix language
from the instructions \eqref{eqn:LazyMatrixInstructions},
which stay intelligible and understandable.
Furthermore if the structure of the matrices changes,
since for example the discretisation scheme changes,
the evaluation route will automatically adapt
given that the cost function is sensibly chosen.

%
% --------
%
\defineabbr{LA}{LA\xspace}{Linear algebra}
\pagebreak[2]
\section{Lazy matrix library \lazyten}
\label{sec:lazyten}
\begin{figure}
	\centering
	\includeimage{6_contraction_lazy_matrices/lazyten_structure}
	\caption[Structure of the \lazyten lazy matrix library]{%
		Structure of the \lazyten lazy matrix library~\cite{lazytenWeb}
		and its interfaces to the 3rd party codes
		armadillo~\cite{Armadillo}, Bohrium~\cite{Kristensen2016array,Kristensen2016streaming},
		LAPACK~\cite{LAPACK} and ARPACK~\cite{ARPACK}.
		Support for Eigen~\cite{Eigen} and Anasazi~\cite{Anasazi} is planned.
	}
	\label{fig:structureLazyten}
\end{figure}
An initial implementation of the lazy matrix language has been
achieved in the \cpp library \lazyten~\cite{lazytenWeb}.
Not all aspects of lazy matrix concept are yet covered, however.
For example many opportunities to achieve performance improvements
by reordering the lazy matrix expression tree are currently missing.
On the other hand \lazyten goes a bit beyond the lazy matrix specification
in the sense that it has become a full abstraction layer for linear algebra.
As depicted in figure \vref{fig:structureLazyten}
the goal of \lazyten is to provide a common interface
for \contraction-based methods
with access to different linear algebra back ends or solver implementations.
Not everything has been achieved as planned,
but nevertheless \lazyten is already used in production
by the \molsturm quantum-chemistry framework discussed in the next section.

\lazyten is open-source software licensed under the
GNU General Public License.
Its source code can be obtained from \url{https://lazyten.org} free of charge.
As of December 2017 \lazyten amounts around 22500 source lines of code
excluding comments and blanks,
but including the helper library \krims~\cite{krimsWeb}
as well as examples and tests.

Inside the framework of \lazyten combining custom lazy matrices
as well as built-in structures,
like a lazy matrix representing the inverse of a matrix,
can be achieved transparently.
Even a combination with stored matrices in any of these expressions is possible.
In this manner code working on \lazyten matrix objects
will continue to work if the type of one of the involved objects is changed.
In other words replacing a plain stored matrix by an involved lazy matrix,
which exploits the sparsity properties of the represented quantity
much better,
can typically be done without changing any of the code operating on such a matrix.

This is possible, since the interface of \lazyten provides
high-level routines to perform linear solves
and to access eigensolvers,
where the call passes through a branching layer,
which mediates between the available back ends depending on the structure
of the problem matrix.
By providing appropriate parameters to the high-level
function a user of the implemented code may still overwrite
which solver implementation
is chosen and what precise setup parameters are passed to it.
Currently a selection of methods from the LAPACK~\cite{LAPACK}
linear algebra library as well as the ARPACK~\cite{ARPACK} package
for Arnoldi diagonalisation methods is available from \lazyten.
The selection mechanism between the different algorithms
for one particular task is not yet extremely sophisticated.
Generally it will for example favour
direct diagonalisation methods from LAPACK~\cite{LAPACK}
if many eigenpairs are requested
or if the supplied system matrix is already dense.
On the other hand Arnoldi methods are selected for lazy matrices
and if only very few eigenpairs are desired.

Whenever an operation like a product of a lazy matrix
with a stored vector unavoidably requires computation,
\lazyten addresses the employed \LA back end through an abstracted interface,
such that switching behaviour on this layer is possible as well.
At the present stage armadillo~\cite{Armadillo}
as a LAPACK-based back end
as well as Bohrium~\cite{Kristensen2016array,Kristensen2016streaming}
as an array-operation-based back end are currently available.
Rather inconveniently switching the back end
right now requires to recompile \lazyten
with the appropriate configure options.

For evaluating a lazy matrix \contraction expression the \LA back end
is typically not extremely important,
since it is only required for very few operations.
Consider for example the third line of \eqref{eqn:LazyMatrixInstructions} above,
where evaluation of the product $\mat{E}\vec{x}$ is required.
Most work is done by the matrix-vector \contraction expressions
of the lazy matrices $\mat{A}$, $\mat{B}$ and $\mat{C}$.
Only for the final sum of the vectors
$\mat{A} \tilde{\vec{x}}$ and $\mat{B} \tilde{\vec{x}}$
\lazyten passes on to the \LA back end.
The impact of changing the back end is naturally larger for operations
between stored matrices or vectors,
where it is used to evaluate all arising expressions.

\subsection{Examples}
\label{sec:LazytenExamples}
To finish off this section,
we demonstrate the high-level syntax of \lazyten
in two example cases.
First consider a general linear problem $\mat{A} \vec{x} = \vec{b}$
with known right-hand side $\vec{b}$ and unknown $\vec{x}$.
The system matrix $\mat{A}$ shall
be represented by the \lazyten matrix object \texttt{A}
and the right-hand side $\vec{b}$ by the object \texttt{b},
which is taken to be a simple stored vector of type \code{SmallVector<double>}.
In \lazyten there are two absolutely equivalent ways to solve this problem.
First
\begin{lstlisting}[style=c++]
SmallVector<double> x(b.size());
solve(A, x, b);
\end{lstlisting}
or equivalently
\begin{lstlisting}[style=c++]
auto invA = inverse(A);
auto x = invA * b;
\end{lstlisting}
\ie quite literally coding the application of the inverse.
In both cases the last line will cause the problem to be passed
to a linear solver algorithm in order to solve it iteratively
or by direct methods.
The user may provide extra parameters to the calls of \texttt{solve}
or \texttt{inverse} in order to influence
the selected eigensolver algorithm
or provide some means of preconditioning the problem matrix.

The second example is more relevant to the scope of this
work and brings us back to the end of section \vref{sec:SCFtakeaway},
where we discussed the possibility of
an \SCF routine,
which is independent from the type of basis function used.
Figure \vref{fig:LazytenRhfCode} shows a code fragment
from a very simple Roothaan repeated diagonalisation \SCF routine%
~(see section \vref{sec:RoothaanRepeatedDiag})
for closed-shell systems coded in the syntax of \lazyten.

\begin{figure}
	\centering
	\loadnt[firstline=40,lastline=87,style={c++},showstringspaces=false]
		{6_contraction_lazy_matrices/rhf_code.cpp}
	\caption
	[Code fragment of a basis-type independent
		Hartree-Fock procedure]
	{Code fragment of a simple basis-type independent
		Hartree-Fock procedure implemented with \lazyten.
		The procedure follows the Roothaan repeated diagonalisation
		algorithm in the specialisation for closed-shell system%
		~(see section \ref{sec:RoothaanRepeatedDiag}).
	}
	\label{fig:LazytenRhfCode}
\end{figure}
Before the depicted code segment is executed,
the integral library is given information about the chemical system
and the desired discretisation and returns
the objects \texttt{Tbb}, \texttt{Vbb}, \texttt{Jbb}, \texttt{Kbb}
and \texttt{Sbb},
which represent the matrices
$\mat{T}$, $\mat{V}_0$, $\mat{J}$, $\mat{K}$ and $\mat{S}$
as they are defined in \eqref{eqn:Tbas} to \eqref{eqn:Sbas}.
Additionally parameters appearing in the code
include \texttt{n\_alpha}, the number of alpha electrons
and \texttt{n\_orb}, the number of \SCF orbitals to compute in each step.

Alongside the comments the code should largely be self-explanatory.
In lines 1 to 6 a core Hamiltonian guess is obtained by diagonalising
$\mat{T} + \mat{V}_0$~(see \ref{sec:CoreHamiltonian}).
Then the Coulomb and exchange lazy matrices
are updated to the guess coefficients in lines 9 and 10.
Depending on the implementation of these lazy matrices,
this might already involve the computation
of the matrices \eqref{eqn:Jbas} and \eqref{eqn:Kbas},
but for others this might just update an internal reference
to the current set of coefficients and apart from that do nothing.
From what we discussed in the previous chapter
it should be clear that the former is better-suited for a \cGTO
discretisation and the latter from a \FE-based discretisation for example.

Afterwards the main loop starts, where first the Fock matrix
expression is built in line 17 and then diagonalised in line 21.
Then the current energies are computed in lines 27 to 30
following \eqref{eqn:HFEnergyFunctionalCoeff} making
vivid use of the \texttt{outer\_prod\_sum} routine.
Right now this routine is required
in such a case originating from the unfortunate decision
to represent a matrix and a set of vectors as two inherently
different data structures.
Effectively it computes products such as
$\mat{C}^T \left(\mat{J} \mat{C}\right)$
from the matrices $\mat{C}$ and $\mat{J} \mat{C}$
represented as a list of vectors.
The remaining lines 32 to 43 print the current iteration
to the user, check for convergence and update the state
of $\mat{J}$ and $\mat{K}$ for the next iteration.

Despite its simplicity the depicted code is independent of
the type of basis function used to discretise the problem as
\lazyten automatically adapts the
executed eigensolver routines for the calls in lines 6 and 18
to the structure of the Fock matrix.
Indirectly it is thus the structure of the matrices
\texttt{Tbb}, \texttt{Vbb}, \texttt{Jbb}, \texttt{Kbb} and \texttt{Sbb}
and usually%
\footnote{Since the automatic selection methods are not yet extremely advanced,
it is necessary
to overwrite the automatic choice from user code from time to time.}
not the code depicted in figure \ref{fig:LazytenRhfCode}
which decides, which eigensolver algorithms are chosen.
Given that the basic heuristics currently implemented,
the depicted code would for example perform a \contraction-based \SCF
for a \CS-based discretisation and use direct eigensolves for a \cGTO-based
discretisation.
In the light of this \lazyten becomes a very effective
abstraction layer between the details of the lazy matrix implementation,
\ie the integral evaluation,
and the \SCF algorithm.

In the \SCF depicted in figure \ref{fig:LazytenRhfCode}
many expressions like lines 17 and 18 or the energy computation
are designed to resemble the equivalent equations one would derive on paper
up to a very large extent.
Nevertheless the matrices like
\texttt{Tbb}, \texttt{Vbb}, \texttt{Jbb}, \texttt{Kbb} and \texttt{Sbb}
could be stored or lazy for the code to work.
Adding an extra term to the Fock matrix expression in line 17 can be done by a simple
addition of another matrix object
irrespective whether the added object is lazy or stored.
In either case its structure would be taken into account
during the following diagonalisation
without explicit user interaction.
Still the user could influence the behaviour of the called solver
by providing appropriate parameters explicitly.
For this reason we believe \lazyten to be very suitable for teaching
or experimentation with novel methods,
since many details are abstracted and one may at first concentrate
on the algorithm and not on numerics.

Overall \lazyten therefore amounts to yield a very intuitive syntax
for \contraction-based methods in the form of lazy matrices,
where algorithms can be written at a high level.
By means of changing the implementation behind the
employed lazy matrix objects the code can be fixed but still flexible
to changes in the available hardware or
if novel types of basis functions with unusual matrix structures
become available.

