\section{Contraction-based algorithms}
\label{sec:ContractionAlgos}

The underlying idea of \contraction-based methods,
namely to avoid storing large matrices in favour of using
matrix-vector-product expressions,
is hardly new.
In his paper from \citeyear{Davidson1975}
\citet{Davidson1975} not only describes his now famous
iterative diagonalisation method~(see section \ref{sec:Davidson}),
but furthermore he suggests to use an algorithmic expression
for computing the required matrix-vector products.
The use case Davidson had in mind back then was the
diagonalisation of the CI or Full-CI matrix,
which is --- even today --- too large to keep in memory,
see remark \vref{rem:EvalFCIMatrix}.

Nowadays \contraction-based methods
are rather widespread in quantum chemistry.
Even though the \contraction expressions are sometimes given different
names such as \textbf{working equations},
making the concept less clear.
Examples are recent implementations of the algebraic diagrammatic construction~(\ADC)
scheme~\cite{Wormit2009,Wormit2014,Dreuw2014},
which do not build the complete \ADC matrix to be diagonalised,
and efficient coupled-cluster schemes~\cite{Helgaker2013},
which similarly avoid constructing the matrix
governing the \CC root-finding problem explicitly.
Instead both methods use appropriate tensor contractions
and compute matrix-vector products on the fly
during the respective iterative solves.
A somewhat related take on this are the recent
\textbf{matrix-free methods}~\cite{Kronbichler2012}
for solving partial differential equations in a finite-element discretisation
without building the system matrix in memory at all.

From the algorithmic point of view one should notice,
that some methods, like the dense eigensolvers and linear solvers
implemented in LAPACK\todo{cite}
do require random access into the matrix,
are thus not available for a \contraction-based ansatz.
In practice this an acceptable restriction.
Firstly because for large matrices dense methods become
unfavourably expensive anyway%
\footnote{Usually exactly because they necessarily keep everything in memory.}.
Secondly because many diagonalisation methods and methods for solving linear systems
do not need the problem matrix in memory.
Instead they can be operated just like the Davidson,
by coding an expression for delivering the required matrix-vector products.
In this category practically all Krylov-subspace approaches can be found,
including widely-adopted algorithms like
\todoil{cite}\noindent
Arnoldi, Lanczos, conjugate gradient or GMRES.
In the context of eigenproblems
one should mention that such iterative methods
have an additional disadvantage.
It is typically very costly to obtain a large number of eigenpairs
of the diagonalised matrix.
Fortunately for large matrices this is hardly needed
and techniques like Chebychev preconditioning~\todo{cite}
or shift and invert~(see section \vref{sec:ShiftInvert})
allow to effectively direct the diagonalisation routines
towards the part of the eigenspectrum one is truely interested in.

On the one hand employing a \contraction-based method thus do not really
restrict the range of numerical problems, which can be tackled.
On the other hand avoiding the storage of the problem matrix
immediately reduces the scaling in memory from quadratic (in system size) to linear.
The rationale for this is that the memory bottleneck
in most subspace algorithms is storing the generated subspace,
\ie a fixed number of vectors, which take linear storage.
This makes \contraction-based methods especially
attractive for problems where memory is a bottleneck.
For this reason this concept has been
introduced in a range of fields of numerics and scientific computing
under different names.
Terms like \textbf{apply-based} method, \textbf{matrix-free} method
or phrases like using \textbf{matrix-vector product expressions}
or using \textbf{matrix-vector products}
overall largely describe the same concept.
I personally like the term \textbf{\contraction-based} best,
because under the hood
evaluating such matrix-vector products in many cases,
that I came across,
involves expressions with contractions over tensors with rank larger than 2.
Consider for example the coupled-cluster doubles working equations~\eqref{eqn:CCDworking}
or the contraction expression for the exchange matrix
in a \CS-based discretisation of Hartree-Fock~\eqref{eqn:ApplicationKcs}.
Additionally calling such algorithms \contraction-based
indicates that the idea to substitute storage by expressions
is more general than the matrix-vector product.
In theory one could think of similar approaches for
higher-order tensor contractions as well.

\subsection{Potential and caveats}
\label{sec:ContractionPotentialCaveat}

\begin{table}
	\centering
	\begin{tabular}{lrr}
		\toprule
		Storage layer  & Latency $/\unit{ns}$ & FLOPs \\
		\midrule
		L1 cache       & $0.5$                &  $13$  \\
		L2 cache       & $7$                  &  $180$ \\
		Main memory    & $100$                &  $2600$ \\
		SSD read       & $1.5\E{4}$           &  $4\E{5}$ \\
		HDD read       & $1\E{7}$             &  $3\E{8}$ \\
		\bottomrule
	\end{tabular}
	\caption[Duration needed for random access into storage]{
		Ballpark duration needed for random access into selected
		layers of storage.
		The right-hand side column represents the peak amount
		of floating point operations a Sandy Bridge CPU with $\unit[3.2]{GHz}$
		clock frequency could perform in the same time
		assuming perfect pipelining.
		Data taken from \cite{CpuData} and \cite{LatencyWeb}.
		Notice, that the seek time on HDDs averages out
		in sequential HDD reads.
		For example reading $\unit[1]{MB}$ from disk only takes about $\unit[2\E{7}]{ns}$,
		\ie only twice as long as the seek by itself.
		For other types of storage this effect is less pronounced.
	}
	\label{tab:FLOPcost}
\end{table}

Historically the main driving force behind \contraction-based
methods was to circumvent the memory bottleneck.
Since the amounts of memory available in the mainframes of the 70s
was much more limited compared to today,
the only alternative to recomputing the data as needed
would have been to store the system matrices on disk.
Taking a look at table \vref{tab:FLOPcost}
we notice that in the time needed to perform random access to HDDs
in the order of millions of floating point operations can be performed.
Overall it is therefore not hard to understand why
people went for recomputing the data instead.

Nowadays the amounts of available main memory has increased substantially,
such that for larger and larger system sizes
the system matrices can now be stored in memory.
Nevertheless one should not forget that accessing main memory costs time as well,
which could be equally well spent for computations.
Assuming perfect pipelining
on the order of 2500 floating point operations can be performed
while the CPU waits for a random value to be loaded from main memory%
~(see table \ref{tab:FLOPcost}).
This is of course orders of magnitude lower than the corresponding
value for a random read from SSD or HDD,
but one should notice that this does not improve as much for sequential
reads as it does for HDDs.
In other words this 2500 flops of computation are
in some sense lost for each memory access
--- whether it is sequential or completely random.

\begin{figure}
	\centering
	\includeimage{6_contraction_lazy_matrices/mem_cpu_years}
	\caption[Scale-up of memory bus speed and CPU clock speed]
	{Scale-up of memory bus speed and CPU clock speed
		relative to 1980 for selected hardware in each year.
		Data taken from \cite{Gocon2014}.}
	\label{fig:MemCpuSpeedup}
\end{figure}
Another aspect one should take into account are the historic trends.
Figure \ref{fig:MemCpuSpeedup} clearly%
\footnote{The original source \cite{Gocon2014}
does not provide a clear description how the data points in each year
where computed from the range of processors and chipsets,
which were released in this year.
Nevertheless the trend is so clear, that I consider this aspect
to have little influence on the picture.}
shows the so-called \newterm{processor-memory performance gap}.
This is meant to express that the available memory bandwidth
has increased by a lesser amount relative to 1980 as the CPU clock speed.
Since the number of flops per second is directly related to the CPU clock speed,
one can extrapolate that the ratio of computable FLOPs per memory access
will likely increase in the upcoming years.
In other words \contraction-based methods
will become more and more attractive in the future,
just because they amount to exploit the steeper increase of the CPU clock
speed curve in figure \ref{fig:MemCpuSpeedup} much better.

Notice, that the aforementioned advantage of \contraction-based methods
to exploit the emerging hardware trends
is an effect which comes \emph{on top}
of the possible reduction in formal computational scaling,
which we found for \FE-based or \CS-based discretisations of \HF in the previous chapter.
This reduction in scaling is not an effect
which is limited to the case of an \SCF routine,
but can be observed in other cases as well.
The underlying reason is in many cases
that the delayed evaluation of the matrix elements
\emph{alongside} the contraction with an actual trial vector
allows to reorder the required summations more freely.
In other words giving up the storage of values
implies that we do no longer need to comply with one particular storage format,
allowing to more freely choose an optimal evaluation strategy.
Somewhat paradoxically this implies that
\contraction-based method have the potential
to be faster even though more computational work is done.

Let us consider an iterative diagonalisation method,
like the Arnoldi or Davidson scheme
in a \contraction-based ansatz.
All steps but the matrix-vector product expression
are performed in the generated iterative subspace,
which by construction has a lower dimensionality
than the full system matrix.
Therefore the matrix-vector product,
\ie the \contraction expression is the computational bottleneck.
A typical diagonalisation requires
around low to mind hundreds of matrix-vector products.
Implementing this \contraction expression
in a highly efficient manner is
key to make a \contraction-based ansatz fast.

Even in the naive manner presented
in equations \eqref{eqn:ExchangeApply} and \eqref{eqn:ApplicationKcs}
the \contraction expressions
of the exchange term of the Fock matrix with a trial vector
look all but simple.
In a practical implementation achieving maximal performance
the required procedures will most probably be more involved,
since issues like the following will all need to be addressed.
\todoil{References for the guys in the itemize}
\begin{itemize}
	\item \textbf{Adoption to Hardware} and \textbf{parallelisation}:
		The features provided by modern hardware
		have of course changed a lot over the years.
		This includes aspects like vectorisation
		or the recent trend to employ general-purpose graphics cards
		in scientific calculations.
		A good algorithm takes modern features into account
		and shows a parallelisation scheme,
		which exploits the available hardware as good as possible.
		Notice, that in many cases the requirements
		can be contradictory,
		such that achieving best performance in all circumstances
		is a real challenge if not impossible.
	\item \textbf{Storing intermediates:}
		Often one can identify subexpressions
		of a large \contraction expression,
		where it makes sense to store it between individual
		executions of the \contraction.
	\item \textbf{Order of contractions:}
		For more complicated expressions involving
		multiple tensor contractions at once the order
		in which the contractions are executed can be crucial
		to achieve best computational scaling
		as well as a low memory footprint.
	\item \textbf{Approximations:}
		Especially in iterative procedures one is typically not
		interested in the numerically exact result
		of a \contraction.
		Much rather the iterative procedure will only solve
		the problem up to a certain accuracy threshold,
		such that computing elements,
		which are smaller than this threshold is a waste of computational time.
		Sometimes this can be incorporated into a \contraction expression
		by prescreening the elements to compute or by other approximations.
\end{itemize}
On top of that one should keep in mind that problem matrices
are usually composed out of different terms
with potentially different structure.
In the case of the \SCF the Fock matrix, for example,
is as sum of the local one-electron terms $\mat{T}$ and $\mat{V}_0$,
which can just be computed,
the Coulomb term $\mat{J}$,
which requires the solution to a Poisson equation,
and the exchange term $\mat{K}$ which requires to solve
multiple equations on a single apply.
It is therefore not hard to imagine
that the best approach to the issues raised above might
well differ from term to term.
Compared to the traditional case where all these matrices
are kept in memory,
adding the terms $\mat{T}$, $\mat{V}_0$, $\mat{J}$
and $\mat{K}$ to form the Fock matrix $\mat{F}$
is thus much more involved in the \contraction-based ansatz.

In passing let us note,
that the parallelisation of \contraction-based methods
is typically both easier and more efficient than for conventional methods.
The rationale is that less stored data generally implies
that there is less data to manage between different cores or nodes
and thus less data to communicate,
because it is recomputed on each worker as required.
Unlike communicating data,
recomputing data is embarrassingly parallel after all.

Overall we can conclude that \contraction-based methods are more flexible
and amount to comply better with the current hardware trends.
Since \contraction expressions are the computational hot-spot,
the procedures one needs to implement are, however,
more involved and consequently the resulting code can become less intuitive
and hard to modify or adapt.
