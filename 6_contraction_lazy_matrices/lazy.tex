\section{Lazy matrices}
\label{sec:lazymat}
The idea of lazy matrices is to encapsulate the coding \contraction-based
in a domain-specific language,
which makes it feel as if one was dealing with actual matrices
instead of \contraction expressions.
Even though not all complications can be hidden,
the resulting syntax allows to write algorithms in a high-level manner
being independent from the underlying implementation of the \contraction expression.
This will turn out to be the key aspect
leading to the basis-type of the quantum-chemistry package \molsturm.

For this purpose we generalise the concept of a matrix
to objects we call a \newterm{lazy matrix}.
Whilst a conventional or \newterm{stored matrix} is dense
and has all its elements residing in a continuous chunk of memory,
this restriction does no longer hold for a lazy matrix.
It may for example follow a particular sparse storage scheme
like compressed-row storage,
but does not even need to be associated to any kind of storage at all.
In the most general sense it can be thought of as an
arbitrary \contraction expression for computing the matrix elements,
which is dressed to look like an ordinary matrix from the outside.

In other words one may still obtain individual matrix elements,
add, subtract or multiply such lazy matrix objects together
or apply them to a bunch of vectors or a stored matrix.
Not all of these operations may be equally fast
than there counterparts on stored matrices, however.
Most importantly obtaining individual elements of such a matrix
can become rather costly,
since they might involve a computation as well
and not just a lookup into memory.

On the upside one gains a much more flexible data structure
where a familiar matrix-like interface
can be added to more complicated objects.
Most notably a lazy matrix may well be non-linear or can have a state,
which may be changed by a \update function
in order to influence the represented values at a later point.
An example where this is sensible would be the Coulomb and Exchange matrices,
where the values of these matrices depend on the set of occupied coefficients,
which have been obtained in the previous iterations.
Other examples include the \update of an accuracy threshold for a \contraction expression,
which might change between iterations.

\begin{figure}
	\centering
	\includeimage{6_contraction_lazy_matrices/expression_tree} \\[0.8em]
	\includeimage{6_contraction_lazy_matrices/expression_tree2}
	\caption[Examples for lazy matrix expression trees]
	{
		Examples for lazy matrix expression trees.
		The upper represents the instruction
		$\mat{D} = \mat{A} + \mat{B}$
		and the lower the multiplication of the result $\mat{D}$
		with $\mat{C}$.
	}
	\label{fig:LazyMatrixExpressionTree}
\end{figure}
All evaluations between lazy matrices
like addition, subtraction or matrix-matrix multiplication
is usually delayed until a contraction of the resulting
expression with a vector or a stored matrix is performed
and thus the represented values are unavoidably needed.
This evaluation strategy is called \newterm{lazy evaluation}
in programming language theory~\cite{Hudak1989},
explaining the name of these data structures.
To make this more clear consider the lazy matrix instructions
\begin{equation}
	\begin{aligned}
		\label{eqn:LazyMatrixInstructions}
		\mat{D} &= \mat{A} + \mat{B}, \\
		\mat{E} &= \mat{D} \mat{C}, \\
		\vec{y} &= \mat{E} \vec{x},
	\end{aligned}
\end{equation}
where $\mat{A}$, $\mat{B}$ and $\mat{C}$ are lazy matrices
and $\vec{x}$ is a vector stored in memory.
The first two do not give rise to any computation being done.
They only amount to build an expression tree in the returned
lazy matrix $\mat{E}$ as illustrated in figure \vref{fig:LazyMatrixExpressionTree}.
The final line is a matrix-vector product with a stored vector,
where an actual stored result should be returned in the vector $\vec{y}$.
In the lazy matrix sense this triggers the complete expression tree to be
evaluated in appropriate order,
leading effectively to an evaluation of the expression
\begin{equation}
	\vec{y} = \left( \mat{A} + \mat{B} \right) \mat{C} \vec{x}
	\label{eqn:LazyMatrixFinalExpression}
\end{equation}
at once at this very instance.
\eqref{eqn:LazyMatrixFinalExpression} can be evaluated entirely only using
matrix-vector \contraction expressions.
For example one could first form the product $\tilde{\vec{x}} \equiv \mat{C} \vec{x}$
using the matrix-vector-product expression of the lazy matrix $\vec{C}$.
Afterwards one would form $\mat{A} \tilde{\vec{x}}$ and $\mat{B} \tilde{\vec{x}}$
again by appropriate \contraction expressions
and finally add the result to give $\vec{y}$.
This is just one way to perform the evaluation.
An implementation of the lazy matrix language is free to choose
a different route for evaluating \eqref{eqn:LazyMatrixFinalExpression}
by reordering the expression if it considers this useful.
If $\mat{C}$ for example was made up of a sum $\mat{F} + \mat{G}$,
it could use distributivity to write
\[ \left( \mat{A} + \mat{B} \right) \left( \mat{F} + \mat{G} \right) \vec{x}
	= \mat{A} \left( \mat{F} \vec{x} \right) + \mat{A} \left( \mat{G}\vec{x}\right)
	+ \mat{B} \left( \mat{F} \vec{x} \right) + \mat{B} \left( \mat{G}\vec{x}\right).
\]
Which of these routes is best differs very much on the structure
of the lazy matrices being part of the expression to evaluate.
But other factors like the operating system or hardware on which
the program code is executed are not unimportant either.
Since the evaluation is delayed
until the call to $\mat{E}\vec{x}$ gets executed at the actual program runtime,
all of this can in theory be taken into account for deciding
which route to take.
Naturally the design of an appropriate cost function
will be in practice all but easy
as previous attempts have shown~\todo{cite a few}.

In either case such decision happen in the evaluation back end
and are well-abstracted by the lazy matrix language
from the instructions \eqref{eqn:LazyMatrixInstructions},
which stay intelligible and understandable.
Furthermore if the structure of the matrices changes,
since for example the discretisation scheme changes,
the evaluation route will automatically adapt
given that the cost function is sensibly chosen.

%
% --------
%
\defineabbr{LA}{LA\xspace}{Linear algebra}
\section{\lazyten lazy matrix library}
\begin{figure}
	\centering
	\includeimage{6_contraction_lazy_matrices/lazyten_structure}
	\caption[Structure of the \lazyten lazy matrix library]{%
		Structure of the \lazyten lazy matrix library~\cite{lazytenWeb}
		and its interfaces to the 3rd party codes
		armadillo~\cite{Armadillo}, Bohrium~\cite{Kristensen2016array,Kristensen2016streaming},
		LAPACK~\cite{LAPACK} and ARPACK~\cite{ARPACK}.
		Support for Eigen~\cite{Eigen} and Anasazi~\cite{Anasazi} is planned.
	}
	\label{fig:structureLazyten}
\end{figure}
An initial implementation of the lazy matrix language has been
achieved in the \cpp library \lazyten~\cite{lazytenWeb}.
Not all aspects of lazy matrix concept are yet covered, however.
For example many opportunities to achieve performance improvements
by reordering the lazy matrix expression tree are currently missing.
On the other hand \lazyten goes a bit beyond the lazy matrix specification
in the sense that it has become a full abstraction layer for linear algebra.
As depicted in figure \vref{fig:structureLazyten}
the goal of \lazyten is to provide a common interface
for \contraction-based methods
with access to different linear algebra back ends or solver implementations.
Not everything has been achieved as planned,
but nevertheless \lazyten is already used in production
by the \molsturm quantum-chemistry framework discussed in the next section.

\lazyten is open-source software licensed under the
GNU General Public License.
Its source code can be obtained from \url{https://lazyten.org} free of charge.
As of December 2017 \lazyten amounts around 22500 source lines of code
excluding comments and blanks,
but including the helper library \krims~\cite{krimsWeb}
as well as examples and tests.

Inside the framework of \lazyten combining custom lazy matrices
as well as built-in structures,
like a lazy matrix representing the inverse of a matrix,
can be achieved transparently.
Even a combination with stored matrices in any of these expressions is possible.
In this manner code working on \lazyten matrix objects
will continue to work if the type of one of the involved objects is changed.
In other words replacing a plain stored matrix by an involved lazy matrix,
which exploits the sparsity properties of the represented quantity
much better,
can typically be done without changing any of the code operating on such a matrix.

This is possible, since the interface of \lazyten provides
high-level routines to perform linear solves
and to access eigensolvers,
where the call passes through a branching layer,
which mediates between the available back ends depending on the structure
of the problem matrix.
By providing appropriate parameters to the high-level
function a user of the implemented code may still overwrite
which solver implementation
is chosen and what precise setup parameters are passed to it.
Currently a selection of methods from the LAPACK~\cite{LAPACK}
linear algebra library as well as the ARPACK~\cite{ARPACK} package
for Arnoldi diagonalisation methods is available from \lazyten.
The selection mechanism between the different algorithms
for one particular task is not yet extremely sophisticated.
Generally it will for example favour
direct diagonalisation methods from LAPACK~\cite{LAPACK}
if many eigenpairs are requested
or if the supplied system matrix is already dense.
On the other hand Arnoldi methods are selected for lazy matrices
and if only very few eigenpairs are desired.

Whenever an operation like a product of a lazy matrix
with a stored vector unavoidably requires computation,
\lazyten addresses the employed \LA back end through an abstracted interface,
such that switching behaviour on this layer is possible as well.
At the present stage armadillo~\cite{Armadillo}
as a LAPACK-based back end
as well as Bohrium~\cite{Kristensen2016array,Kristensen2016streaming}
as an array-operation-based back end are currently available.
Rather inconveniently switching the back end
right now requires to recompile \lazyten
with the appropriate configure options.

For evaluating a lazy matrix \contraction expression the \LA back end
is typically not extremely important,
since it is only required for very few operations.
Consider for example the third line of \eqref{eqn:LazyMatrixInstructions} above,
where evaluation of the product $\mat{E}\vec{x}$ is required.
Most work is done by the matrix-vector \contraction expressions
of the lazy matrices $\mat{A}$, $\mat{B}$ and $\mat{C}$.
Only for the final sum of the vectors
$\mat{A} \tilde{\vec{x}}$ and $\mat{B} \tilde{\vec{x}}$
\lazyten passes on to the \LA back end.
The impact of changing the back end is naturally larger for operations
between stored matrices or vectors,
where it is used to evaluate all arising expressions.

\subsection{Examples of \lazyten code}
\label{sec:LazytenExamples}
To finish off this section,
we demonstrate the high-level syntax of \lazyten
in two example cases.
First consider a general linear problem $\mat{A} \vec{x} = \vec{b}$
with known right-hand side $\vec{b}$ and unknown $\vec{x}$.
The system matrix $\mat{A}$ shall
be represented by the \lazyten matrix object \texttt{A}
and the right-hand side $\vec{b}$ by the object \texttt{b},
which is taken to be a simple stored vector of type \code{SmallVector<double>}.
In \lazyten there are two absolutely equivalent ways to solve this problem.
First
\begin{lstlisting}[style=c++]
SmallVector<double> x(b.size());
solve(A, x, b);
\end{lstlisting}
or equivalently
\begin{lstlisting}[style=c++]
auto invA = inverse(A);
auto x = invA * b;
\end{lstlisting}
\ie quite literally coding the application of the inverse.
In both cases the last line will cause the problem to be passed
to a linear solver algorithm in order to solve it iteratively
or by direct methods.
The user may provide extra parameters to the calls of \texttt{solve}
or \texttt{inverse} in order to influence
the selected eigensolver algorithm
or provide some means of preconditioning the problem matrix.

\todoil{From here}
The second example is more relevant to the scope of this
work and brings us back to the end of section \vref{sec:SCFtakeaway},
where we discussed the possibility of
an \SCF routine,
which is independent from the type of basis function used.
Figure \vref{fig:LazytenRhfCode} shows a code fragment
from a very simple Roothaan repeated diagonalisation \SCF routine%
~(see section \vref{sec:RoothaanRepeatedDiag})
for closed-shell systems coded in the syntax of \lazyten.

\begin{figure}
	\centering
	\loadnt[firstline=40,lastline=87,style={c++},showstringspaces=false]
		{6_contraction_lazy_matrices/rhf_code.cpp}
	\caption
	[Code fragment of a basis-type independent
		Hartree-Fock procedure]
	{Code fragment of a simple basis-type independent
		Hartree-Fock procedure implemented with \lazyten.
		The procedure follows the Roothaan repeated diagonalisation
		algorithm in the specialisation for closed-shell system%
		~(see section \ref{sec:RoothaanRepeatedDiag}).
	}
	\label{fig:LazytenRhfCode}
\end{figure}
Before the depicted code segment is executed,
the integral library is given information about the chemical system
and the desired discretisation and returns
the objects \texttt{Tbb}, \texttt{Vbb}, \texttt{Jbb}, \texttt{Kbb}
and \texttt{Sbb},
which represent the matrices
$\mat{T}$, $\mat{V}_0$, $\mat{J}$, $\mat{K}$ and $\mat{S}$
as they are defined in \eqref{eqn:Tbas} to \eqref{eqn:Sbas}.
Additionally parameters appearing in the code
include \texttt{n\_alpha}, the number of alpha electrons
and \texttt{n\_orb}, the number of \SCF orbitals to compute in each step.

Alongside the comments the code should largely be self-explanatory.
In lines 1 to 6 a core Hamiltonian guess is obtained by diagonalising
$\mat{T} + \mat{V}_0$~(see \ref{sec:CoreHamiltonian}).
Then the Coulomb and exchange lazy matrices
are updated to the guess coefficients in lines 9 and 10.
Depending on the implementation of these lazy matrices,
this might already involve the computation
of the matrices \eqref{eqn:Jbas} and \eqref{eqn:Kbas},
but for others this might just update an internal reference
to the current set of coefficients and apart from that do nothing.
From what we discussed in the previous chapter
it should be clear that the former is better-suited for a \cGTO
discretisation and the latter from a \FE-based discretisation for example.

Afterwards the main loop starts, where first the Fock matrix
expression is built in line 17 and then diagonalised in line 21.
Then the current energies are computed in lines 27 to 30
following \eqref{eqn:HFEnergyFunctionalCoeff} making
vivid use of the \texttt{outer\_prod\_sum} routine.
Right now this routine is required
in such a case originating from the unfortunate decision
to represent a matrix and a set of vectors as two inherently
different data structures.
Effectively it computes products such as
$\mat{C}^T \left(\mat{J} \mat{C}\right)$
from the matrices $\mat{C}$ and $\mat{J} \mat{C}$
represented as a list of vectors.
The remaining lines 32 to 43 print the current iteration
to the user, check for convergence and update the state
of $\mat{J}$ and $\mat{K}$ for the next iteration.

Despite its simplicity the depicted code is independent of
the type of basis function used to discretise the problem as
\lazyten automatically adapts the
executed eigensolver routines for the calls in lines 6 and 18
to the structure of the Fock matrix.
Indirectly it is thus the structure of the matrices
\texttt{Tbb}, \texttt{Vbb}, \texttt{Jbb}, \texttt{Kbb} and \texttt{Sbb}
and usually%
\footnote{Since the automatic selection methods are not yet extremely advanced,
it is necessary
to overwrite the automatic choice from user code from time to time.}
not the code depicted in figure \vref{fig:LazytenRhfCode}
which decides, which eigensolver algorithms are chosen.
Given that the basic heuristics currently implemented,
the depicted code would for example perform a \contraction-based \SCF
for a \CS-based discretisation and use direct eigensolves for a \cGTO-based
discretisation.
In the light of this \lazyten becomes a very effective
abstraction layer between the details of the lazy matrix implementation,
\ie the integral evaluation,
and the \SCF algorithm.

In the \SCF depicted in figure \ref{fig:LazytenRhfCode}
many expressions like lines 17 and 18 or the energy computation
are designed to resemble the equivalent equations one would derive on paper
up to a very large extent.
Nevertheless the matrices like
\texttt{Tbb}, \texttt{Vbb}, \texttt{Jbb}, \texttt{Kbb} and \texttt{Sbb}
could be stored or lazy for the code to work.
Adding an extra term to the Fock matrix expression in line 17 can be done by a simple
addition of another matrix object
irrespective whether the added object is lazy or stored.
In either case its structure would be taken into account
during the following diagonalisation
without explicit user interaction.
Still the user could influence the behaviour of the called solver
by providing appropriate parameters explicitly.
For this reason we believe \lazyten to be very suitable for teaching
or experimentation with novel methods,
since many details are abstracted and one may at first concentrate
on the algorithm and not on numerics.

Overall \lazyten therefore amounts to yield a very intuitive syntax
for \contraction-based methods in the form of lazy matrices,
where algorithms can be written at a high level.
By means of changing the implementation behind the
employed lazy matrix objects the code can be fixed but still flexible
to changes in the available hardware or
if novel types of basis functions with unusual matrix structures
become available.
